\documentclass[12pt]{report}

\usepackage{amsmath}
\usepackage{amssymb}
\usepackage{amsthm}
\usepackage{amsopn}
\usepackage{kpfonts}
\usepackage{graphicx}
\usepackage{kbordermatrix}
\usepackage{tikz}
\usetikzlibrary{arrows, petri, topaths}%
\usepackage{tkz-berge}
\usepackage{multicol}

\usepackage{framed}
\usepackage{mathtools}
\usepackage{float}
\usepackage{subfig}
% \usepackage{cmbright}

\theoremstyle{plain}
\newtheorem{theorem}{Theorem}[chapter]
\newtheorem{lemma}[theorem]{Lemma}
\newtheorem{corollary}[theorem]{Corollary}
\newtheorem{prop}[theorem]{Proposition}
\newtheorem{exercise}{Exercise}[chapter]

\newtheorem*{example}{Example}
\newtheorem*{proof*}{Proof}

\theoremstyle{definition}
\newtheorem*{defi}{Definition}
\newenvironment{definition}
    {\begin{samepage}\begin{framed}\begin{defi}}
    {\end{defi}\end{framed}\end{samepage}}





\usepackage{hyperref} 
\hypersetup{
    colorlinks = true,
    linkcolor = black,
}

\makeatletter
\renewcommand*\env@matrix[1][*\c@MaxMatrixCols c]{%
  \hskip -\arraycolsep
  \let\@ifnextchar\new@ifnextchar
  \array{#1}}
\makeatother

\renewcommand*\contentsname{\hfill Table Of Contents \hfill}

\newcommand{\optionalsection}[1]{\section[* #1]{(Important) #1}}
\newcommand{\deriv}[3]{\left. \frac{\partial #1}{\partial #2} \right|_{#3}}

\title{Economics}
\author{Jacob Denson}

\begin{document}

\pagenumbering{gobble}
\maketitle
\tableofcontents
\pagenumbering{arabic}

\chapter{Chapter 1: On the Experience of Moral Confusion}

The main point of this chapter is that modern societies are very confused about the definition, role, and behaviour of debt in society, which makes a naive understanding of debt dangerous unsatisfactory. A naive conception of debt repayment is dangerous, since the apparent self evidential nature of debt systems makes terrible things appear bland and utterly remarkable when phrased in the language of debt. The precise quantification of debt turns morality into impersonal arithmetic, thus justifying the outrageous. Violence is turned into mathematics.

One instance of the publics naivety towards debt is that many people believe individuals should be morally obligated to pay their debts (morality is even often phrased in the language of debt, which makes the statement almost self evident). But in such a society banks would not fullfill their economic role of generating money for efficient generators of wealth; with no risk of losing money by giving out a loan, banks would instead be willing to give out loans to anyone (there would be no risk in giving out such a loan).

One of the most prevelent aspects of this is that political power structures are able to justify their actions by phrasing their actions through the language of debt. Classically, when one conquered a group of people through war, slavery was viewed as a reasonable consequence since that other groupn of people ``owed the conquerors their lives'', since they were not killed by the conquerers. When France colonized Madagascar through war, they imposed heavy tax on the country since Madagascar `owed them the cost of the war'. Things are more subtle nowadays, where we see the US debt to allied countries as having little consequence on the country (which is very different from the consequences of debt in the third world, Madagascar, and Haiti). After all, what's the difference between a gangster forcing you to give him 1000 dollars, or forcing you to give him a 1000 dollar loan? Thus we see nowadays that categories of debt vary greatly depending on the \emph{political power} of the debtor and creditor.

For instance, the International Monetary Fund, with the backing of politically powerful western countries, was able to effectively enforce the repayment of debts made to the third world, via forcing the imposition of economic policies on countries requesting refinancing which destroyed social systems in the countries and guaranteed the deaths of millions due to disease and famine. Since they were able to guarantee the return of their money, they had little risk, and could loan out money to any corrupt politicians, who would funnel that money directly into their swiss bank accounts. The IMF would then place the blame of repaying debt on the people of those countries. The repayment feels bland and obviously necessary, but has lead to the deaths of famines by removing social systems that would not have cost much to fund in the first place.

There is a feedback loop here. Since political powers use debt as a way to justify power relations, it is natural for definitions of debt to feel incoherent. For instance, int he west one feels morally obligated to pay back debts, but also feels immoral when giving out debts in the first place. In certain parts of the Himilayas, the different political organization (the debtors caste is also the ruling caste) means that giving out debts does not feel immoral, and many feel it is completely usual to give their daughter out the daughter to be used sexually as collateral for paying for a wedding of that same daughter.

\end{document}