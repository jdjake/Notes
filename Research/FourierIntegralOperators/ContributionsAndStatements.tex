\documentclass[12pt]{article}

\usepackage{amsmath}
\usepackage{amssymb}
\usepackage{amsthm}
\usepackage{esint}
\usepackage{fancyhdr}
\usepackage{comment}
\usepackage{sectsty}

\sectionfont{\large}
\subsectionfont{\normalsize}

\usepackage{mathptmx}
\usepackage[margin=0.75in]{geometry}

\fancypagestyle{plain}{
    \fancyhead[RO]{Jacob Denson}
    \fancyhead[LO]{}
    \fancyfoot{}
}

\usepackage{etoolbox}
\patchcmd{\thebibliography}{\section*{\refname}}{}{}{}

\pagestyle{fancy}
\fancyhead[RO]{Jacob Denson}
\fancyhead[LO]{}
\fancyfoot{}

\DeclareMathOperator{\RR}{\mathbb{R}}
\DeclareMathOperator{\CC}{\mathbb{C}}

\theoremstyle{plain}
\newtheorem{theorem}{Theorem}
\newtheorem{lemma}[theorem]{Lemma}
\newtheorem{corollary}[theorem]{Corollary}
\newtheorem{prop}[theorem]{Proposition}

\theoremstyle{remark}
\newtheorem*{example}{Example}
\newtheorem*{remark}{Remark}

\theoremstyle{definition}
\newtheorem*{defi}{Definition}
\newenvironment{definition}
    {\begin{samepage}\begin{framed}\begin{defi}}
    {\end{defi}\end{framed}\end{samepage}}

\title{Contributions and Statements}
\author{}
\date{\today}

% Remove the author and date fields and the space associated with them
% from the definition of maketitle!
\makeatletter
\renewcommand{\@maketitle}{
\newpage
 \begin{center}%
  {\LARGE \@title \par}%
 \end{center}%
 \par} \makeatother

\begin{document}

\maketitle

\section*{Part I - Contributions to Research and Development}

\subsection*{a. Articles Published or Accepted in Peer-Reviewed Journals}

\begin{thebibliography}{9}

\bibitem{OurPaper}
Denson, J., and Pramanik, M., and Zahl, J. (2021) Large Sets Avoiding Rough Patterns. Springer Optimization and Its Applications. 168: 59-75 (MSc Work).

\end{thebibliography}

%\vspace{-2em}

\subsection*{b. Articles Submitted to Peer-Reviewed Journals}

\begin{thebibliography}{9}

\makeatletter
\addtocounter{\@listctr}{1}
\makeatother

\bibitem{MyPaper}
Denson, J. (Submitted October, 2021) Large Salem Sets Avoiding Nonlinear Configurations. Submitted to Analysis \& PDE: 38 Pages. (PhD Work)

\end{thebibliography}

%\vspace{-2em}

\subsection*{c. Other Peer-Reviewed Contributions}

\begin{thebibliography}{9}

\makeatletter
\addtocounter{\@listctr}{2}
\makeatother

\bibitem{MyThesis}
Denson, J. (2019). Cartesian Products Avoiding Patterns. University of British Columbia. (MSc Work)

\end{thebibliography}

%\vspace{-2em}

\subsection*{d. Non Peer-Reviewed Contributions}

\begin{thebibliography}{9}

\makeatletter
\addtocounter{\@listctr}{3}
\makeatother

\bibitem{ConferenceQuantum}
Denson, J. (2021). Capacity of Rank Decreasing Operators. Brascamp Lieb Inequalities Summer School, Kopp: 80-83. (PhD Work)

%\bibitem{ConferenceQuantumPresentation}
%Denson, J*. (2021). Algorithmic Aspects of Brascamp-Lieb Inequalities (Oral Presentation given Internationally at the 2021 Bonn Harmonic Analysis Summer School). Brascamp Lieb Inequalities Summer School, Kopp. (PhD Work)

\bibitem{ConferenceSalem}
Denson, J*. (2020-2021). Salem Sets Avoiding Patterns. (Oral Presentation given Institutionally at the UW-Madison Analysis Student Seminar in 2020, and Nationally at the Ottawa Math Conference in 2020, and at the Ohio River Analysis Meeting in 2021).. (PhD Work)

\bibitem{ConferencePosters}
Denson, J*. (2018-2019). Fractals Avoiding Fractal Sets. (Oral Presentation given Nationally at the 2018 CMS Winter Meeting, and Internationally at the 2018 Mid-Atlantic Analysis Seminar, and GAHA 2019). (Poster Presented Nationally at the 2018 CMS Winter Meeting, and Internationally at the 2019 February Fourier Talks and 2019 Madison Lectures in Fourier Analysis). (Msc Work)

\bibitem{MyThesis}
Denson, J. (2018). Proofs in Three Bits or Less. CMS Notes From the Margin. 13: 1-3. (MSc Work)

\end{thebibliography}

\section*{Part II - Most Significant Contributions to Research}

%\vspace{-0.5em}

\subsection*{Large Sets Avoiding Rough Patterns [1]}

This paper forms a major part of my MSc Thesis research project, and thus a summary of the results of this paper can be found in the Thesis section of the report (this paper proves Theorem 27 of that thesis). The results of this paper evolved over a year and a half of weekly conversations with my two advisors, Malabika Pramanik and Joshua Zahl. But I was mainly responsible for writing up the paper describing our work, meeting with my advisors weekly to review the current draft and obtain a style we were all happy with. This process forced me to tighten up my mathematical writing style. Before we started writing the paper, we had agree to publish as a chapter in the Springer Volume ``Harmonic Analysis and Applications''. The main reasons were because the volume had a well respected peer review process, we knew the deadline for submission to the book, which gave us a precise schedule to write up the paper, and the deadline implied that if accepted, our article would be published expediently.

\subsection*{Large Salem Sets Avoiding Nonlinear Configurations [2]}

After graduating from UBC, I worked on improving the Fourier dimension constructions in my thesis [3], which left the question of matching Fourier and Hausdorff dimension bounds unresolved. Most results about patterns and Fourier dimension in the literature concern linear patterns, so improving Fourier dimension constructions for nonlinear patterns is quite groundbreaking. Using more sophisticated probabilistic concentration bounds than standard results used in the construction of sets avoiding linear patterns, I was able to reduce the study of the pattern avoidance problem to the study of a family of oscillatory integrals with `nonsmooth' amplitude functions. These oscillatory integrals are natural, but don't seem to have been considered in harmonic analysis or probability, and it does not seem we have the tools to estimate them completely. Thus the techniques of this paper are primed to produce further interesting results in the field. We were able to obtain a nontrivial, but imperfect analysis of these oscillatory integrals in various cases, and we can avoid them entirely for certain patterns exhibiting a very weak translation-invariance property, thus completely removing the barrier between Fourier and Hausdorff dimensions in this case.

This project is solely my own. I started developing the ideas behind the project after I finished my Master's degree, and my current PhD advisor does not study pattern avoidance problems. I have submitted the paper for publication in the journal ``Analysis \& PDE'', since many members of the editorial board are interested in geometric measure theory and harmonic analysis, including Izabella Łaba, who has even published multiple papers on the existence and avoidance of patterns in sets. Moreover, various analogoues of the result of [2] for Hausdorff dimension have been published in this journal (e.g. Pramanik and Fraser, 2018), so the journal seems like a natural fit.

\section*{Part III - Applicant's Statement}

\subsection*{Research Experience}

Being able to translate questions between discrete and continuous settings has proved very useful in my work on pattern avoidance. But it is also incredibly useful to study radial multipliers, where one converts the problem into the study of $L^p$ bounds on a sum, which is highly related to the incidence geometry of spheres. One advantage I have found I have had over several of my peers is a thorough background in various aspects of discrete mathematics useful to harmonic analysis, instigated by my undergraduate degree in theoretical computing science, and cultivated during my Master's degree. In particular, I have taken the time to study current techniques in incidence geometry and the polynomial method, by taking course while at UBC, and organizing reading groups at UW-Madison. As described in my outline, current research indicates these techniques are likely very useful in the study of radial multipliers. 

\subsection*{Relevant Activities}

Being able to translate questions between applied and pure mathematical communities also seems productive to modern research. For instance, I learned the probabilistic techniques that lead to the results of [2] in a topics class organized for data scientists. Conversely, applied mathematicians can be convinced about the interest of pure mathematical results. During my Master's thesis, I attended a conference where a colleague invited me to attend the February Fourier Talks, a talk attended mainly by applied mathematicians. Presenting a poster on pattern avoidance there was quite intimidating. %, since applied mathematicians almost seem to live in a different theoretical world.
In response to my poster, one applied mathematician remarked that ``Fractals have been a dead field for 40 years''. I responded by talking to him about Kakeya sets and how they arise in the study of the instability of natural cutoff filters that occur in signals processing. This and other conversations must have proved quite interesting to applied mathematicians, since I was awarded the 2nd place poster prize at the conference.

\end{document}