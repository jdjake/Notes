\documentclass[12pt]{article}

\usepackage{amsmath}
\usepackage{amssymb}
\usepackage{amsthm}
\usepackage{esint}
\usepackage{fancyhdr}
\usepackage{comment}

\usepackage{mathptmx}
\usepackage[margin=0.75in]{geometry}

\fancypagestyle{plain}{
    \fancyhead[RO]{Jacob Denson}
    \fancyfoot{}
}

\pagestyle{fancy}
\fancyhead[RO]{Jacob Denson}
\fancyfoot{}

\DeclareMathOperator{\RR}{\mathbb{R}}
\DeclareMathOperator{\CC}{\mathbb{C}}

\theoremstyle{plain}
\newtheorem{theorem}{Theorem}
\newtheorem{lemma}[theorem]{Lemma}
\newtheorem{corollary}[theorem]{Corollary}
\newtheorem{prop}[theorem]{Proposition}

\theoremstyle{remark}
\newtheorem*{example}{Example}
\newtheorem*{remark}{Remark}

\theoremstyle{definition}
\newtheorem*{defi}{Definition}
\newenvironment{definition}
    {\begin{samepage}\begin{framed}\begin{defi}}
    {\end{defi}\end{framed}\end{samepage}}

\title{Outline of Proposed Research}
\author{}
\date{\today}

% Remove the author and date fields and the space associated with them
% from the definition of maketitle!
\makeatletter
\renewcommand{\@maketitle}{
\newpage
 \null
 \begin{center}%
  {\LARGE \@title \par}%
 \end{center}%
 \par} \makeatother

\begin{document}

\maketitle

\vspace{1em}

The question of the $L^p$ boundedness of translation-invariant operators on $\RR^n$ has proved central to the development of modern harmonic analysis. Indeed, answers to these questions underpin any subtle understanding of the Fourier transform, since with essentially any such operator $T$, we can associate a function $m: \RR^n \to \CC$, known as the \emph{symbol} of $T$, such that for any function $f$, the Fourier transform of $Tf$ obeys the relation $\widehat{Tf} = m \widehat{f}$; thus translation invariant operators are also called \emph{Fourier multiplier operators}. Initial questions about Fourier multipliers emerged from classical questions concerning the convergence properties of Fourier series, and in the study of the classical equations of physics, like the heat and wave equation. Such operators often have rotational symmetry, so it is natural to restrict our attention to multiplier operators which are also rotation-invariant. These operators are called \emph{radial Fourier multipliers}, since the associated symbol is then a radial function. This research project proposes the study of necessary and sufficient conditions to guarantee $L^p$ boundedness of radial multiplier operators, stimulated by recent developments which indicate lines of attack for three related problems in the field.

The general study of the boundedness of Fourier multipliers was intiated in the 1960s. It was quickly realized that the most fundamental estimates for a translation-invariant operator $T$ were $L^p$ estimates of the form $\| Tf \|_{L^p(\RR^n)} \lesssim \| f \|_{L^p(\RR^n)}$ in the range $1 \leq p \leq 2$. For $p = 1$ and $p = 2$, mathematicians found simple necessary and sufficient conditions to ensure boundedness \cite{Hormander1}. But the problem of finding necessary and sufficient conditions for boundedness in the range $1 < p < 2$ proved impenetrable. Indeed, many interesting problems about the boundedness of \emph{specific} Fourier multiplier operators for values of $p$ in the range, such as the Bochner-Riesz conjecture, remain largely unsolved today.

Thus it came as a surprise when recent results indicated necessary and sufficient conditions for \emph{radial} Fourier multiplier operators to be bounded for values of $p$ in this range. First came the result of \cite{GarrigosandSeeger}, which gave a simple sufficient and necessary condition for bounds of the form $\| Tf \|_{L^p(\RR^n)} \lesssim \| f \|_{L^p(\RR^n)}$ to hold for a particular radial multiplier operator $T$, uniformly over \emph{radial} functions $f$, precisely in the range $1 < p < 2 - 2/(n+1)$. It is natural to conjecture that the same criterion, applied to the same range of $p$, gives the bound $\| Tf \|_{L^p(\RR^n)} \lesssim \| f \|_{L^p(\RR^n)}$ for \emph{general} functions $f$. In this outline we call this statement the \emph{radial multiplier conjecture}. We now know, by the results of \cite{HeoandNazarovandSeeger} and \cite{Cladek}, that the radial multiplier conjecture is true when $n > 4$ and $1 < p < 2 - 4/(n+1)$, and when $n = 4$ and $1 < p < 2 - 3.79/(n+1)$. We also know \cite{Cladek} the criterion in the conjecture is sufficient to obtain a \emph{restricted weak type} bound $\| Tf \|_{L^p(\RR^n)} \lesssim \| f \|_{L^{p,1}(\RR^n)}$ when $n = 3$ and $1 < p < 2 - 3.66/(n+1)$. But the radial multiplier conjecture has not yet been completely resolved in any dimension $n$, we do not have any strong type $L^p$ bounds when $n = 3$, and we don't have any bounds whatsoever when $n = 2$.

If fully proved, the radial multiplier conjecture would imply the Bochner-Riesz conjecture, and thus the Kakeya and restriction conjectures as a result. All three consequences are major unsolved problems in harmonic analysis, so a complete resolution of the conjecture is far beyond the scope of current research techniques. On the other hand, the Bochner Riesz conjecture is completely resolved when $n = 2$, while in contrast, no results related to the radial multiplier conjecture are known in this dimension at all. And in any dimension $n > 2$, the range under which the Bochner-Riesz multiplier is known to hold \cite{GuoandOhandWangandWuandZhang} is strictly larger than the range under which the radial multiplier conjecture is known to hold, even for the restricted weak-type bounds obtained in \cite{Cladek}. Thus it still seems within hope that the techniques recently applied to improve results for Bochner-Riesz problem, such as broad-narrow analysis \cite{BourgainandGuth}, the polynomial Wolff axioms \cite{KatzandRogers}, and methods of incidence geometry and polynomial partitioning \cite{Zahl2} can be applied to give improvements to current results characterizing the boundedness of general radial Fourier multipliers.

Our hopes are further emboldened when we consult the proofs in \cite{HeoandNazarovandSeeger} and \cite{Cladek}, which reduce the radial multiplier conjecture to the study of upper bounds of quantities of the form $\| \sum_{(y,r) \in \mathcal{E}} F_{y,r} \|_{L^p(\RR^n)}$, where $\mathcal{E} \subset \RR^n \times (0,\infty)$ is a finite collection of pairs, and $F_{y,r}$ is an oscillating function supported on a $O(1)$ neighborhood of a sphere of radius $r$ centered at a point $y$. The $L^p$ norm of this sum is closely related to the study of the tangential intersections of these spheres, a problem successfully studied in more combinatorial settings using incidence geometry and polynomial partitioning methods \cite{Zahl}, which provides further estimates that these methods might yield further estimates on the radial multiplier conjecture.

When $n = 3$, \cite{Cladek} is only able to obtain bounds on the $L^p$ sums in the last paragraph when $\mathcal{E}$ is a Cartesian product of two subsets of $(0,\infty)$ and $\RR^n$. This is why only restricted weak-type bounds have been obtained in this dimension. It is therefore an interesting question whether different techniques enable one to extend the $L^p$ bounds of these sums when the set $\mathcal{E}$ is \emph{not} a Cartesian product, which would allow us to upgrade the result of \cite{Cladek} in $n = 3$ to give strong $L^p$ bounds. This question also has independent interest, because it would imply new results for the `endpoint' local smoothing conjecture, which concerns the regularity of solutions to the wave equation in $\RR^n$. Incidence geometry has been recently applied to yield results on the `non-endpoint' local smoothing conjecture \cite{GuthandWangandZhang}, which again suggests these techniques might be applied to yield the estimates needed to upgrade the result of \cite{Cladek} to give strong $L^p$-type bounds.

A third line of questioning about the radial multiplier conjecture is obtained by studying natural analogues of Fourier multiplier operators on Riemannian manifolds. Using functional calculi, for any function $m: [0,\infty) \to \mathbf{C}$, we can associate an operator $m(\sqrt{-\Delta})$ on a compact Riemannian manifold $M$, where $\Delta$ is the Laplace-Beltrami on $M$. Just like multiplier operators on $\RR^n$ are crucial to an understanding of the Fourier transform, the operators $m(\sqrt{-\Delta})$ are crucial to understand the behaviour of eigenfunctions of the Laplace-Beltrami operator on $M$.

The direct analogue of the boundedness problems for multipliers is trivial in this setting, since any compactly supported function $m$ will induce an operator $T = m(\sqrt{-\Delta})$ satisfying estimates of the form $\| Tf \|_{L^p(M)} \lesssim \| f \|_{L^p(M)}$ for all $1 \leq p \leq 2$. The correct formulation of the problem is instead to consider the alternate bound $\sup_{R > 0} \| m(\sqrt{-\Delta}/R) f \|_{L^p(M)} \lesssim \| f \|_{L^p(M)}$. In fact, a transference principle of Mitjagin \cite{Mitjagin} implies that if the latter bound holds for a multiplier $m(\sqrt{-\Delta})$, then a bound of the form $\| Tf \|_{L^p(\RR^n)} \lesssim \| f \|_{L^p(\RR^n)}$ holds where $T$ is the Fourier multiplier operator associated with the symbol $m(|\xi|)$. Thus boundedness in this alternate sense is at least as hard on a compact manifold $M$ as it is in $\RR^n$. The research project here is to try and extend the radial multiplier conjecture to this setting.

For general compact manifolds difficulties arise in generalizing the conjecture, connected to the fact that analogues of the Kakeya / Nikodym conjecture are false in this general setting \cite{Minicozzi}. But these problems do not arise for constant curvature manifolds, like the sphere. The sphere also has over special properties which make it especially amenable to analysis, such as the fact that solutions to the wave equation on spheres are periodic. Best of all, there are already results which achieve the analogue of \cite{GarrigosandSeeger} on the sphere. Thus it seems reasonable that current research techniques can obtain interesting results for radial multipliers on the sphere, at least in the ranges established in \cite{HeoandNazarovandSeeger} or even \cite{Cladek}.

In conclusion, the results of \cite{HeoandNazarovandSeeger} and \cite{Cladek} indicate three lines of questioning about radial Fourier multiplier operators, which current research techniques place us in reach of resolving. The first question is whether we can extend the range of exponents upon which the conjecture of \cite{GarrigosandSeeger} is true, at least in the case $n = 2$ where Bochner-Riesz has been solved. The second is whether we can use more sophisticated arguments to prove the $L^p$ sum bounds obtained in \cite{Cladek} when $n = 3$ when the sums are no longer cartesian products, thus obtaining strong $L^p$ characterizations in this settiong, as well as new results about the endpoint local smoothing conjecture. The third question is whether we can generalize these bounds obtained in these two papers to study radial Fourier multipliers on the sphere. 

\begin{comment}

\newpage


Many fundamental questions remain unsolved about multiplier operators, even if we restrict our study to radial Fourier multipliers. Indeed, the Bochner-Riesz conjecture, a major conjecture in the field which despite decades of research still seems quite far from being resolved, concerns the boundedness properties of a particular compactly supported radial Fourier multiplier. This research project proposes the study of a family of related problems in the study of radial multipliers, which, stimulated by recent developments in the theory of planar radial Fourier multipliers, seem 

outside of the class of operators associated with a compactly supported, smooth symbol. Indeed, 

However, even restricted to the class of radial Fourier multipliers, many fundamental questiosn remain unsolved. Indeed, 

Since many equations of physics are rotationally symmetric, it is natural to restrict our attention to translation-invariant operators which are also \emph{rotation-invariant}, 


But much still remains unknown about such operators today, even for simple, particular examples, such as that given by a compactly supported radially symmetric bump function $m(\xi)$

 where $m$ is given by a simple 



 they are given , has proved to be a central question throughout the development of harmonic analysis. Such questions naturally arise from classical questions, such as the convergence properties of Fourier series, and in the study of the classical equations of physics, such as the heat or wave equation. But much still remains unknown about Fourier multiplier operators, even if one restricts to the natural subclass of \emph{radial} Fourier multipliers, i.e. translation-invariant operators which are also invariant under rotations. For instance, the Bochner-Riesz conjecture asks to determine the boundedness properties of a single radial Fourier multiplier operator, namely, 


 whether for classical questions such as the convergence properties of Fourier series, or more modern questions such as averaging problems over low dimensional curves and surfaces.

For the past half century, determining the boundedness of

A Fourier multiplier operator takes in as input a function formed from the superposition of planar waves travelling in various directions and at various frequencies, and transforms this function by rescaling the amplitudes of each of these waves independently. Simple examples of such operators include the Hilbert transform and any constant coefficient linear differential operator. Radial Fourier multiplier operators form an important subclass, which are multipliers with the property that the associated rescaling factor for each wave depends solely on the frequency of the wave, and not the direction it is travelling in. If instead of considering planar waves, one instead considers Fourier multipliers for `surface', or `ground' waves, travelling on other geometric spaces, such as spheres, torii, or other higher dimensional manifolds, then one obtains the study of 'variable coefficient' Fourier multipliers. This research project studies conditions guaranteeing the stability of `variable-coefficient' radial Fourier multipliers, stimulated by recent developments in the theory of planar radial Fourier multipliers which indicate lines of attack to three related problems.

Fourier multiplier operators are crucial to the understanding of many areas of pure mathematics and the sciences. They occur in complex analysis, via the `Hilbert transform' and it's variants, and Riemannian geometry, where the `Riesz transform' allows one to decompose vector fields into incompressible and conservative vector fields. Outside of mathematics, they are a key tool in signals processing (where Fourier multiplier operators are called 'filters'), forming key components of physical systems such audio encoding software, and medical technology like MRI and CAT scanners. Many solutions to partial differential equations can be expressed in terms of Fourier multiplier operators, whether classical equations like the heat equation, Schr\"{o}dinger equation, and wave equation (ADD CITATION), or more specialized equations, such as the bidomain Allen-Kahn equation, used to model a human heart's electrical activity for various applications, including the analysis of the effects of cardiac pacemakers on the function of a heart (ADD YOICHIRO MORI CITATION). Thus Fourier multiplier operators are crucial to the understanding of physical phenomena. Radial Fourier multipliers naturally occur in these settings, since most physical phenomena obeys some form of rotational symmetry.

% MAYBE INCLUDE THIS FOR AN APPLICATION TO WAVES ON SPHERES: https://reader.elsevier.com/reader/sd/pii/S0167278916306352?token=D6888E33CEC29E51A0DD831AB7647DC6CEA0D43F9E5AE1811A7B7FFC86FDA8E772AC5ED83B68421DBDC803A713432BCA&originRegion=us-east-1&originCreation=20211008055535

In order to apply a Fourier multiplier operator in a physical scenario, it is crucial to understand the stability properties of the operator. A filter cannot be used to process a signal unless there is a guarantee that the filter will behave well under the effects of noisy input values. And instabilities of the operators involved in the study of cardiac pacemakers, via the Allen-Kahn equation, may prove fatal, leading to a lethal cardiac arrythmia. In pure mathematics, understanding the stability of Fourier multiplier operators is also important, giving us insights such as the theory of local smoothing for the wave equation (INCLUDE CITATION), and (INCLUDE ANOTHER EXAMPLE HERE). The study of the stability of Fourier multiplier operators is thus a central question in harmonic analysis, and also has many important consequences outside of mathematics.

In harmonic analysis, we frame the stability problem for a Fourier multiplier operator in terms of analyzing the boundedness of the multiplier, viewed as an operator between two Banach spaces, often from an $L^p$ space to itself for some $1 \leq p \leq 2$. Showing an operator is bounded between two Banach spaces thus guarantees the stability of the operator when the input is perturbed by a quantity with small norm. The classical study of the stability of Multiplier operators was initiated in the 1960s by H\"{o}rmander (INCLUDE 1960s PAPER CITATION), who obtained a simple necessary and sufficient criterion for a Fourier multiplier operator to be bounded from $L^1(\RR^n)$ to $L^1(\RR^n)$ and from $L^2(\RR^n)$ to $L^2(\RR^n)$. Over the next half century, various sufficient criteria were found to guarantee the boundedness of operators for intermediate values of $p$, such as the H\"{o}rmander Mikhlin multiplier theorem, a result still being iterated upon today (Grafakos, Slavikona CITATION 2017). But no simple necessary and \emph{sufficient} conditions have been found in the past 50 years which guarantee a Fourier multiplier is bounded from $L^p(\RR^n)$ to $L^p(\RR^n)$ for any other value of $p \not \in \{ 1, 2, \infty \}$, with many speculating a simple condition may not even exist for other values of $p$.

Thus it was surprising when (CITATION) showed that for \emph{radial} Fourier multipliers, such a simple necessary and sufficient criterion to ensure boundedness exists from $L^p(\RR^n)$ to $L^p(\RR^n)$ \emph{does} exist for $n \geq 4$ and $1 < p < 2(n-1)/(n+1)$. This same criterion was later shown (CITATION) to be necessary and sufficient for boundedness when $n = 3$ and $1 < p < 13/12$ or improved to the range $1 < p < 36/29$ when $n = 4$, under the additional assumption that the Fourier multiplier operator is compactly supported. It remains an incredibly important and difficult open question to determine whether the criterion developed in (CITATION) continues to hold in the larger range $1 < p < 2d/(d+1)$, which is know to be the maximal interval under which the simple criterion developed by (CITATION) could hold. A positive proof of this result would answer several major conjectures in harmonic analysis, including the Bochner-Riesz conjecture, the Kakeya conjecture, and the restriction conjecture, and so further analysis of the techniques developed by results such as (CITATION) and (CITATION) is integral to the development of the field of harmonic analysis.

One way we can better understand the techniques developed in (CITATION) and (CITATION) is to see whether the criterion developed in these papers continues to hold in the variable coefficient setting, where the curvature of the underlying spaces upon which the waves are travelling complicates the analysis such that one cannot simply trivially reduce the analysis to the planar case. On the other hand, there is evidence to suggest that such an analysis is both within research of a research project, and is nontrivial enough to have fruitful consequences. Variable coefficient generalizations of results in harmonic analysis have already been in other settings, such as the result of (SOGGE CITATION), who found a way to generalize the Tomas-Stein restriction theorem to a variable coefficient setting, with the consequence that the result implied much better control on the $L^p$ norms of eigenfunctions for the Laplace-Beltrami operator on general Riemannian manifolds. (WHAT CONSEQUENCES WOULD OUR VARIABLE COEFFICIENT FORMULATION HAVE?)

Given the importance of the study of variable coefficient generalizations of the results of (CITATION) and (CITATION) to both harmonic analysis and it's applications in the general sciences, and the evidence to suggest that such a result is within reach, we feel that the analysis of such results should be pursued as a research project.

\end{comment}

\bibliographystyle{plain}
\bibliography{ResearchOutline}

\end{document}