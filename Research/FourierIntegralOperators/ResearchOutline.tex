\documentclass[12pt]{article}

\usepackage{amsmath}
\usepackage{amssymb}
\usepackage{amsthm}
\usepackage{esint}

\usepackage{mathptmx}
\usepackage[margin=0.75in]{geometry}

\DeclareMathOperator{\RR}{\mathbf{R}}

\theoremstyle{plain}
\newtheorem{theorem}{Theorem}
\newtheorem{lemma}[theorem]{Lemma}
\newtheorem{corollary}[theorem]{Corollary}
\newtheorem{prop}[theorem]{Proposition}

\theoremstyle{remark}
\newtheorem*{example}{Example}
\newtheorem*{remark}{Remark}

\theoremstyle{definition}
\newtheorem*{defi}{Definition}
\newenvironment{definition}
    {\begin{samepage}\begin{framed}\begin{defi}}
    {\end{defi}\end{framed}\end{samepage}}

\title{Outline of Proposed Research}
\author{Jacob Denson}
\date{\today}

\begin{document}

\maketitle

A Fourier multiplier operator takes in as input a function formed from the superposition of waves travelling in various directions and at various frequencies, and transforms this function by rescaling the amplitudes of each of these waves independently. Radial Fourier multiplier operators form an important subclass, which have the property that the associated rescaling factor of each wave depends solely on the frequency of the wave, and not the direction it is travelling in. If instead of considering planar waves, one instead considers Fourier multipliers for waves travelling on other geometric spaces, such as spheres, torii, or other manifolds, then one obtains the study of 'variable coefficient' Fourier multipliers. This research project will focus on the study of conditions guaranteeing the stability of `variable-coefficient' radial multipliers, stimulated by recent developments in the theory of planar Fourier multipliers.

Fourier multiplier operators are crucial to the understanding of many areas of pure mathematics and the sciences. They occur in complex analysis, via the 'Hilbert transform' and it's variants, and Riemannian geometry, where the 'Riesz transform' allows one to decompose vector fields into incompressible and conservative vector fields. Outside of mathematics, they form a key component of signals processing (where Fourier multiplier operators are called 'filters'), forming key components of physical systems such as MRI scanners, CAT scanners, and audio encoding software. Many solutions to PDEs can be expressed in terms of Fourier multiplier operators, whether classical equations like the heat equation, Schr\"{o}dinger equation, and wave equation (ADD CITATION), or more specialized equations, such as the bidomain Allen-Kahn equation, used to model a human heart's electrical activity for various applications, including the analysis of the effects of cardiac pacemakers on the function of a heart (ADD YOICHIRO MORI CITATION). Thus Fourier multiplier operators are crucial to the understanding of much physical phenomena. Radial Fourier multipliers naturally occur in these settings, since most physical phenomena obeys rotational symmetry. In these situations, understanding the stability properties of Fourier multiplier operators is key to their utility. A filter cannot be used in signals proessing if it is unstable with respect to perturbing signals with random noise. And instabilities of the operators involved in the study of cardiac pacemakers in the Allen-Kahn equation may prove more fatal, leading to lethal heart arrythmia.

% MAYBE INCLUDE THIS FOR AN APPLICATION TO WAVES ON SPHERES: https://reader.elsevier.com/reader/sd/pii/S0167278916306352?token=D6888E33CEC29E51A0DD831AB7647DC6CEA0D43F9E5AE1811A7B7FFC86FDA8E772AC5ED83B68421DBDC803A713432BCA&originRegion=us-east-1&originCreation=20211008055535

The stability of Fourier multiplier operators is a central question in harmonic analysis, where the stability problem is framed in terms of determining the boundedness of the operator between two Banach spaces, often the $L^p$ spaces. The boundedness of the operator thus guarantees the stability of the operator when small pertubations are applied with small norm. The study of this stability was initiated in the 1960s by H\"{o}rmander (INCLUDE 1960 CITATION), who obtained a simple necessary and complete criterion for a Fourier multiplier operator to be bounded from $L^1(\RR^n)$ to $L^1(\RR^n)$, from $L^\infty(\RR^n)$ to $L^\infty(\RR^n)$, and from $L^2(\RR^n)$ to $L^2(\RR^n)$. But for half a century, no simple necessary and sufficient criterion ensuring a multiplier operator to be bounded from $L^p(\RR^n)$ to $L^p(\RR^n)$ for any other exponent $p$ was found, and most believed no such criterion existed. Thus it was surpising when (CITATION) showed that for \emph{radial} Fourier multipliers, such a simple criterion exists for $n \geq 4$ and $1 < p < 2(n-1)/(n+1)$ or $2(n-1)/(n-3) < p < \infty$. This same criterion was later shown to be necessary and sufficient for boundedness when $n = 3$ and $1 < p < 13/12$ or $13 < p < \infty$ and when $n = 4$ and $1 < p < 36/29$ or $36/7 < p < \infty$, under the additional assumption that the Fourier multiplier operator is compactly supported. It remains an incredibly difficult open question to show whether the criterion developed in (CITATION) continues to hold in the larger range $1 < p < 2d/(d+1)$ or $2d/(d-1) < p < \infty$, the maximum interval under which the criterion could hold. Such a result would imply several major conjectures in harmonic analysis, including the Bochner-Riesz conjecture, the Kakeya conjecture, and the restriction conjecture, and it's resolution likely resolves techniques that will require years of development.

A much more achievable goal is to try and see whether these characterizations extend to the case of variable-coefficient radial Fourier multipliers. In this setting, the curvature of space makes the problem harder, and requires the development of novel techniques to solve, which will enable one to obtain further control over Fourier multipliers over arbitrary manifolds. On the other hand, there is reason to be optimistic that one can develop such tools. Such results have been achieved in other scenarios, such as (Sogge), who translated the Tomas-Stein restriction theorem to a variable coefficient setting. Thus it is likely we can achieve results in this area. (EXAMPLE OF CASE WHERE SPHERE CASE OF VARIABLE COEFFICIENT PROBLEM HAS BEEN SOLVED?)



 



\end{document}