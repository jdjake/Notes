    \documentclass[12pt, dvipsnames]{report}

\usepackage{amsmath}
\usepackage{algorithm}
%\usepackage{algorithmic}
\usepackage[noend]{algpseudocode}

\usepackage{amsmath}
\usepackage{amssymb}
\usepackage{amsthm}
\usepackage{amsopn}

\usepackage{txfonts}

\usepackage{tensor}

\usepackage{stmaryrd}


\usepackage{graphicx}

% Probably don't need this on notes anymore
%\usepackage{kbordermatrix}

% Standard tool for drawing diagrams.
\usepackage{tikz}
\usepackage{tkz-berge}
\usepackage{tikz-cd}
\usepackage{tkz-graph}
\usetikzlibrary{arrows,chains,matrix,positioning,scopes,calc}

\tikzset{
    right angle quadrant/.code={
        \pgfmathsetmacro\quadranta{{1,1,-1,-1}[#1-1]}     % Arrays for selecting quadrant
        \pgfmathsetmacro\quadrantb{{1,-1,-1,1}[#1-1]}},
    right angle quadrant=1, % Make sure it is set, even if not called explicitly
    right angle length/.code={\def\rightanglelength{#1}},   % Length of symbol
    right angle length=2ex, % Make sure it is set...
    right angle symbol/.style n args={3}{
        insert path={
            let \p0 = ($(#1)!(#3)!(#2)$) in     % Intersection
                let \p1 = ($(\p0)!\quadranta*\rightanglelength!(#3)$), % Point on base line
                \p2 = ($(\p0)!\quadrantb*\rightanglelength!(#2)$) in % Point on perpendicular line
                let \p3 = ($(\p1)+(\p2)-(\p0)$) in  % Corner point of symbol
            (\p1) -- (\p3) -- (\p2)
        }
    }
}

\usepackage{comment}

%
\usepackage{multicol}

%
\usepackage{framed}

%
\usepackage{mathtools}

%
\usepackage{float}

%
\usepackage{subfig}

%
\usepackage{wrapfig}

%
\let\savewideparen\wideparen
\let\wideparen\relax
\usepackage{mathabx}
\let\wideparen\savewideparen

% Used for generating `enlightening quotes'
\usepackage{epigraph}

% Forget what this is used for :P
\usepackage[utf8]{inputenc}

% Used for generating quotes.
\usepackage{csquotes}

% Allows what to generate links inside
% generated pdf files
\usepackage{hyperref}

% Allows one to customize theorem
% environments in mathematical proofs.
\usepackage{thmtools}

% Gives access to a proof
\usepackage{lplfitch}

% I forget what this is for.
\usepackage{accents}

% A package for drawing simple trees,
% as a substitute for unnesacary TIKZ code
\usepackage{qtree}

% Enables sequent calculus proofs
\usepackage{ebproof}

% For braket notation
\usepackage{braket}

% To change line spacing when using mathematical notations which require some height!
\usepackage{setspace}

%\usepackage[dvipsnames]{xcolor}

\usepackage{float}

% For block commenting
\usepackage{comment}

\usepackage{etoolbox}
\let\bbordermatrix\bordermatrix
\patchcmd{\bbordermatrix}{8.75}{4.75}{}{}
\patchcmd{\bbordermatrix}{\left(}{\left[}{}{}
\patchcmd{\bbordermatrix}{\right)}{\right]}{}{}




\setlength\epigraphwidth{8cm}

\usetikzlibrary{arrows, petri, topaths, decorations.markings}

% So you can do calculations in coordinate specifications
\usetikzlibrary{calc}
\usetikzlibrary{angles}

\theoremstyle{plain}
\newtheorem{theorem}{Theorem}[chapter]
\newtheorem{axiom}{Axiom}
\newtheorem{lemma}[theorem]{Lemma}
\newtheorem{corollary}[theorem]{Corollary}
\newtheorem{prop}[theorem]{Proposition}
\newtheorem{exercise}{Exercise}[chapter]
\newtheorem{fact}{Fact}[chapter]
\newtheorem{definition}{Definition}[chapter]

\newtheorem*{example}{Example}
\newtheorem*{proof*}{Proof}

\theoremstyle{remark}
\newtheorem*{exposition}{Exposition}
\newtheorem*{remark}{Remark}
\newtheorem*{remarks}{Remarks}

\theoremstyle{definition}
\newtheorem*{defi}{Definition}

\usepackage{hyperref}
\hypersetup{
    colorlinks = true,
    linkcolor = black,
}

\usepackage{textgreek}

\makeatletter
\renewcommand*\env@matrix[1][*\c@MaxMatrixCols c]{%
  \hskip -\arraycolsep
  \let\@ifnextchar\new@ifnextchar
  \array{#1}}
\makeatother

\renewcommand*\contentsname{\hfill Table Of Contents \hfill}

\newcommand{\optionalsection}[1]{\section[* #1]{(Important) #1}}
\newcommand{\deriv}[3]{\left. \frac{\partial #1}{\partial #2} \right|_{#3}} % partial derivative involving numerator and denominator.
\newcommand{\lcm}{\operatorname{lcm}}
\newcommand{\im}{\operatorname{im}}
\newcommand{\bint}{\mathbf{Z}}
\newcommand{\gen}[1]{\langle #1 \rangle}

\newcommand{\End}{\operatorname{End}}
\newcommand{\Mor}{\operatorname{Mor}}
\newcommand{\Id}{\operatorname{id}}
\newcommand{\visspace}{\text{\textvisiblespace}}
\newcommand{\Gal}{\text{Gal}}

\newcommand{\xor}{\oplus}
\newcommand{\ft}{\wedge}
\newcommand{\ift}{\vee}

\newcommand{\prob}{\mathbb{P}}
\newcommand{\expect}{\mathbb{E}}
\DeclareMathOperator{\Var}{\mathbb{V}}
\newcommand{\Ber}{\text{Ber}}
\newcommand{\Bin}{\text{Bin}}

\DeclareMathOperator{\sech}{sech}
\newcommand{\cadlag}{c\'{a}dl\'{a}g}
\newcommand{\caglad}{c\'{a}dl\'{a}d}

\newcommand{\loc}[1]{#1_{\text{loc}}}

%\newcommand{\widecheck}[1]{{#1}^{\ft}}

\DeclareMathOperator{\diam}{\text{diam}}

\DeclareMathOperator{\QQ}{\mathbb{Q}}
\DeclareMathOperator{\ZZ}{\mathbb{Z}}
\DeclareMathOperator{\RR}{\mathbb{R}}
\DeclareMathOperator{\HH}{\mathbb{H}}
\DeclareMathOperator{\BB}{\mathbb{B}}
\DeclareMathOperator{\CC}{\mathbb{C}}
\DeclareMathOperator{\AB}{\mathbb{A}}
\DeclareMathOperator{\PP}{\mathbb{P}}
\DeclareMathOperator{\MM}{\mathbb{M}}
\DeclareMathOperator{\VV}{\mathbb{V}}
\DeclareMathOperator{\TT}{\mathbb{T}}
\DeclareMathOperator{\LL}{\mathcal{L}}
\DeclareMathOperator{\DD}{\mathcal{D}}
\DeclareMathOperator{\SW}{\mathcal{S}}
\DeclareMathOperator{\EC}{\mathcal{E}}
\DeclareMathOperator{\AC}{\mathcal{A}}

\DeclareMathOperator{\EE}{\mathbb{E}}
\DeclareMathOperator{\NN}{\mathbb{N}}

\DeclareMathOperator{\II}{\mathbb{I}}

\DeclareMathOperator{\DQ}{\mathcal{Q}}

\DeclareMathOperator{\Ind}{\mathbb{I}}


\DeclareMathOperator{\IA}{\mathfrak{a}}
\DeclareMathOperator{\IB}{\mathfrak{b}}
\DeclareMathOperator{\IC}{\mathfrak{c}}
\DeclareMathOperator{\IP}{\mathfrak{p}}
\DeclareMathOperator{\IQ}{\mathfrak{q}}
\DeclareMathOperator{\IM}{\mathfrak{m}}
\DeclareMathOperator{\IN}{\mathfrak{n}}
\DeclareMathOperator{\IK}{\mathfrak{k}}
\DeclareMathOperator{\ord}{\text{ord}}
\DeclareMathOperator{\Ker}{\textsf{Ker}}
\DeclareMathOperator{\Coker}{\textsf{Coker}}
\DeclareMathOperator{\emphcoker}{\emph{coker}}
\DeclareMathOperator{\pp}{\partial}
\DeclareMathOperator{\tr}{\text{tr}}
\DeclareMathOperator{\Ree}{\text{Re}}


\DeclareMathOperator{\BL}{\text{BL}}

\DeclareMathOperator{\dstrike}{//}

\DeclareMathOperator{\supp}{\text{supp}}

\DeclareMathOperator{\codim}{\text{codim}}

\DeclareMathOperator{\minkdim}{\dim_{\mathbb{M}}}
\DeclareMathOperator{\hausdim}{\dim_{\mathbb{H}}}
\DeclareMathOperator{\sobdim}{\dim_{\mathbb{S}}}
\DeclareMathOperator{\lowminkdim}{\underline{\dim}_{\mathbb{M}}}
\DeclareMathOperator{\upminkdim}{\overline{\dim}_{\mathbb{M}}}
\DeclareMathOperator{\lhdim}{\underline{\dim}_{\mathbb{M}}}
\DeclareMathOperator{\lmbdim}{\underline{\dim}_{\mathbb{MB}}}
\DeclareMathOperator{\packdim}{\text{dim}_{\mathbb{P}}}
\DeclareMathOperator{\fordim}{\dim_{\mathbb{F}}}

\DeclareMathOperator{\CT}{ {{\otimes}^\wedge} }

\DeclareMathOperator{\msupp}{\text{$\mu$-supp}}
\DeclareMathOperator{\singsupp}{\text{sing-supp}}
\DeclareMathOperator{\Char}{\text{Char}}

\DeclareMathOperator*{\argmax}{arg\,max}
\DeclareMathOperator*{\argmin}{arg\,min}

\DeclareMathOperator{\ssm}{\smallsetminus}

\DeclarePairedDelimiter{\inner}{\langle}{\rangle}
\newcommand{\pder}[2]{\frac{\partial #1}{\partial #2}}
\newcommand{\tripnorm}[1]{{\left\vert\kern-0.25ex\left\vert\kern-0.25ex\left\vert #1 
    \right\vert\kern-0.25ex\right\vert\kern-0.25ex\right\vert}}

%\DeclareMathOperator{\span}{\text{span}}

\makeatletter
\newcommand*\bigcdot{\mathpalette\bigcdot@{.5}}
\newcommand*\bigcdot@[2]{\mathbin{\vcenter{\hbox{\scalebox{#2}{$\m@th#1\bullet$}}}}}
\makeatother

\title{Radial Multipliers}
\author{Jacob Denson}

\begin{document}

\maketitle

\tableofcontents

\newpage

\chapter{Notation}

We let $\text{Dil}_t$ be the dilation operator on functions, i.e. the operator such that
%
\[ \text{Dil}_t f(x) = f(x/t). \]
%
Abusing notation, we let
%
\[ \text{Dil}_j f(x) = f(x/2^j) \]
%
denote the dyadic dilations. There is no confusion here since the dyadic dilation operator will only be used along with symbols that stand for integers, like $j$ and $k$, whereas the other dilation operator will be used in all other cases.

It will often be helpful to consider a family of bump functions for the purposes of localization.
%
\begin{itemize}
    \item We begin with a non-negative function $\beta \in C_c^\infty(\RR^d)$ such that:
    % How do we construct beta?
    % We take some bump function phi equal to one on |x| <= a and supported on |x| <= b,
    % and then define beta(x) = phi(x) - phi(2x)
    % Then beta is supported on a/2 <= |x| <= b, and equal to one when b/2 <= |x| <= a
    % We must therefore have a >= b/2 to have a region where the function is equal to one
    % the size of this region is then (a - b/2)
    \begin{itemize}
        \item $\text{supp}(\beta) \subset \{ 0.625 \leq |x| \leq 1.5 \}$.

        \item $\beta(x) = 1$ for $0.75 \leq |x| \leq 1.25$.

        \item For any $x \neq 0$, $\sum_{j \in \ZZ} \text{Dil}_j \beta(x) = 1$.
    \end{itemize}
    
    \item Next, we let $\alpha = \sum_{j < 0} \text{Dil}_j \beta$. Then $\alpha \in C_c^\infty(\RR^d)$, and:
    %
    \begin{itemize}
        \item $\text{supp}(\alpha) \subset \{ |x| \leq 1 \}$.

        \item $\alpha(x) = 1$ for $|x| \leq 0.625$.

        \item For any $x \neq 0$, $\alpha(x) + \sum_{j = 0}^\infty \text{Dil}_j \beta(x) = 1$.
    \end{itemize}

    \item We also consider a function $\tilde{\beta} \in C_c^\infty(\RR^d)$ such that:
    %
    \begin{itemize}
        \item $\text{supp}(\tilde{\beta}) \subset \{ 0.5 \leq |x| \leq 2 \}$
        \item $\tilde{\beta}(x) = 1$ for $0.625 \leq |x| \leq 1.5$.
        \item For any $x \neq 0$, $\sum_{j \in \ZZ} \text{Dil}_j \tilde{\beta}(x) = 1$.
    \end{itemize}
    %
    The second property implies that for any function $f$ with $\text{supp}(f) \subset \{ 0.5 \leq |x| \leq 2 \}$, we have $f \tilde{\beta} = f$. In particular, this implies that $\beta \tilde{\beta} = \beta$.
\end{itemize}

\part{Review of Literature}

\chapter{Fourier Multipliers}

The question of the regularity of translation-invariant operators on $\RR^d$ has proved central to the development of modern harmonic analysis. This is because for essentially any translation invariant operator $T$, we can find a tempered distribution $m: \RR^d \to \CC$, the \emph{symbol} of $T$, such that for any Schwartz function $f \in \mathcal{S}(\RR^d)$,
%
\[ Tf(x) = \int_{\RR^d} m(\xi) \widehat{f}(\xi) e^{2 \pi i \xi \cdot x}\; d\xi. \]
%
In other words, $\widehat{Tf} = m \cdot \widehat{f}$, which is why such operators are also called \emph{Fourier multipliers}. Applying the notation of spectral calculus, one might also write this operator as $m(D)$, where $D = (2 \pi i)^{-1} \nabla$ is a self-adjoint normalization of the gradient operator. Thus the study of the boundedness of translation invariant operators is closely connected to the study of the interactions of the operators
%
\[ E_\xi f(x) = \widehat{f}(\xi) e^{2 \pi i \xi \cdot x}, \]
%
which act as projections onto the common eigenspaces of the components of $D$; related to this is the fact that we can write $m(D)$ as a vector-valued integral of the form
%
\[ m(D) = \int_{\RR^d} m(\xi) E_\xi\; d\xi. \]
%
Thus $m(D)$ is represented as a weighted average of the operators $\{ E_\xi \}$.

The study of translation invariant operators emerges from classical questions in analysis, like those emerging from the study of the convergence of Fourier series, or problems in mathematical physics, through the study of the heat, wave, and Schr\"{o}dinger equation, or other similar equations of this type. These operators also naturally have rotational symmetry, so it is natural to restrict our attention to translation-invariant operators which are also rotation-invariant. Such operators are precisely those operators associated with \emph{radial} symbols $m: \RR^d \to \CC$, i.e. symbol such that there exists a function $h: [0,\infty) \to \RR$ such that
%
\[ m(\xi) = h(|\xi|) \]
%
for some function $h: [0,\infty) \to \CC$. This is the class of \emph{radial Fourier multipliers}. The notation of spectral calculus leads us to write $m(D) = h(\sqrt{-\Delta})$, where $\Delta$ is the Laplacian on $\RR^d$. Thus the study of radial multipliers is closely connected to interactions between the spherical restriction operators
%
\[ E_\lambda f(x) = \int_{|\xi| = \lambda} \widehat{f}(\xi) e^{2 \pi i \xi \cdot x}\; d\xi, \]
%
for $0 < \lambda < \infty$, which are the projections onto the eigenspaces of $\sqrt{-\Delta}$. Similar to the study of $m(D)$, we then have
%
\[ h \Big( \sqrt{-\Delta}\; \Big) = \int_0^\infty h(\lambda) E_\lambda\; d\lambda. \]
%
Thus studying the regularity of radial Fourier multipliers allows us to understand the interactions between the operators $\{ E_\lambda \}$.

It is also often useful to study these operators on the spatial side. Given any translation invariant operator $T$, we can associate a tempered distribution $k: \RR^d \to \CC$, the \emph{convolution kernel} of $T$, such that for any Schwartz function $f \in \mathcal{S}(\RR^d)$,
%
\[ Tf(y) = \int_{\RR^d} k(x) f(y-x)\; dx, \]
%
If $T$ is radial, we can write $k(x) = a(|x|)$ for some function $a: [0,\infty) \to \CC$, and then we have a representation
%
\[ T = \int_0^\infty a(r) S_r\; dr, \]
%
where
%
\[ S_rf(x) = \int_{|y| = r} f(x + y)\; dy, \]
%
are the \emph{spherical averaging operators}. Thus problems about radial translation-invariant operators are also connected to problems involving the interactions between spherical averages over different radii. With the notation as above, the function $k$ is the \emph{Fourier transform} of $m$, and the function $a$ is a \emph{Bessel transform} of $h$, i.e. $a = \mathcal{B}_d h$, where
%
\[ \mathcal{B}_d h(r) = 2 \pi r^{1-d/2} \int_0^\infty h(\lambda) J_{d/2 - 1}(2 \pi \lambda r) \lambda^{d/2}\; d\lambda, \]
%
where $J_{d/2-1}$ is the standard Bessel function of order $d/2 - 1$, which we may define as
%
\[ J_\alpha(\lambda) = \frac{(\lambda / 2)^\alpha}{\Gamma(\alpha + 1/2)} \int_{-1}^1 e^{i \lambda s} (1 - s^2)^{\alpha - 1/2}\; ds. \]
%
For later reference, using oscillatory integrals, we can at least understanding this function for large inputs, i.e. so that as $\lambda \to \infty$, then we may find constants $\{ a_{\alpha,k} \}$ and $\{ b_{\alpha,k} \}$
%
\[ J_\alpha(\lambda) \sim \cos(\lambda - \omega) \sum_{k = 0}^\infty (-1)^k \frac{a_k}{\lambda^{2k+1/2}} + \sin(\lambda - \omega) \sum_{k = 0}^\infty (-1)^k \frac{b_k}{\lambda^{2k + 3/2}}. \]
%
Thus, stated more crudely, we have a bound of the form $|J_\alpha(\lambda)| \lesssim \lambda^{-1/2}$.

In terms of the spectral theory of the Laplacian, the theory of radial multipliers can be extended from $\RR^d$ to the much more general setting of Riemannian manifolds. On any such manifold $X$, we can define a Laplace-Beltrami operator $\Delta$, and provided $X$ is geodesically complete, this operator will be an essentially self-adjoint unbounded operator on $L^2(X)$. Thus we can consider a spectral calculus. In particular, one can consider the study of operators of the form $h(\sqrt{-\Delta})$ for functions $h: [0,\infty) \to \CC$, which tell us, in some sense, how different eigenfunctions of  $\Delta$ interact. Some techniques of analyzing radial multipliers on $\RR^d$ extend to the Riemannian case, whereas in other cases new tools are required.

The main focus of this research project is the study of necessary and sufficient conditions to guarantee the $L^p$ boundedness of radial multiplier operators, both in the Euclidean setting, and also in the setting of Riemannian manifolds. % stimulated by recent developments which indicate lines of attack for three related problems in the field.


\section{Multipliers on Euclidean Space}

The general study of the boundedness properties of Fourier multipliers in multiple variables was initiated in the 1950s, as connections of the theory to partial differential equations became more fully realized \cite{Hormander1}. It was quickly realized that the most fundamental estimates were $L^p$ bounds of the form
%
\[ \| Tf \|_{L^q(\RR^d)} \lesssim \| f \|_{L^p(\RR^d)}. \]
%
%for $1 \leq p \leq 2$, and $q \geq p$, which by duality are equivalent to bounds
%
%\[ \| Tf \|_{L^{p^*}(\RR^d)} \lesssim \| f \|_{L^{q^*}(\RR^d)}, \]
%
It is therefore natural to introduce the spaces $M^{p,q}(\RR^d)$, consisting of all symbols $m$ which induce a Fourier multiplier operator $T$ bounded from $L^p(\RR^d)$ to $L^q(\RR^d)$. The space $M^{p,q}(\RR^d)$ is then a Banach space under the operator norm
%
\[ \| m \|_{M^{p,q}(\RR^d)} = \sup \left\{ \frac{\| T f \|_{L^q(\RR^d)}}{\| f \|_{L^p(\RR^d)}} : f \in \mathcal{S}(\RR^d) \right\}. \]
%
For notational convenience, $M^{p,p}(\RR^d)$ is denoted by $M^p(\RR^d)$. Duality implies that $M^{p,q}(\RR^d)$ is isometric to $M^{q^* \negmedspace ,p^*}(\RR^d)$, where $p^*$ and $q^*$ are the conjugates to $p$ and $q$. And the fact these operators are translation invariant, together with Littlewood's Principle, implies that $M^{p,q}(\RR^d) = \{ 0 \}$ unless $q \geq p$. Combining these two results means we can always reduce our analysis to the study of the spaces $M^{p,q}(\RR^d)$ when $1 \leq p \leq 2$ and when $q \geq p$, or equivalently, when $2 \leq p \leq \infty$ and $q \geq p$.

Some of the spaces $M^{p,q}(\RR^d)$ are very easy to characterize. For instance, $M^{1,q}(\RR^d) = M^{q^*,\infty}(\RR^d)$ are characterized by virtue of the fact that the study of the boundedness of operators with domain $L^1(\RR^d)$, or range $L^\infty(\RR^d)$ is often trivial, i.e. as characterized by Schur's Lemma; for any symbol $m$, if $k = \widehat{m}$ is the associated convolution kernel, then
%
\[ \| m \|_{M^{1,q}(\RR^d)} = \| m \|_{M^{q^*,\infty}(\RR^d)} = \begin{cases} \| k \|_{L^q(\RR^d)} &: q > 1, \\ \| k \|_{M(\RR^d)} &: q = 1, \end{cases} \]
%
where $M(\RR^d)$ is the space of finite signed Borel measures equipped with the total variation norm. The unitary nature of the Fourier transform implies that the spaces $M^{p,2}(\RR^d) = M^{2,p^*}(\RR^d)$ could also be characterized, for $1 \leq p \leq 2$, by the identity
%
\[ \| m \|_{M^{p,2}(\RR^d)} = \| m \|_{L^q(\RR^d)}, \]
%
% | m f^ |_{L^2} <= | m |_q | f |_p
% If f^ is an extremizer for m
where $q = 2p/(2-p)$ for $1 \leq p < 2$, and $q = \infty$ for $p = 2$. This gives a complete summary of the results currently known about characterizations of the remaining spaces; no simple characterization of $M^{p,q}(\RR^d)$ for $1 < p \leq 2$ and $q \geq p$ with $q \neq 2$ has been found in the past half century. Nonetheless, several tools for analyzing Fourier multipliers in this range have been developed, and we end this section by briefly summarizing some relevant results.

We begin with the major tool of \emph{Littlewood-Paley} theory, which makes it natural to restrict to the study of Fourier multipliers compactly suported on dyadic annuli.
\begin{comment}
    We can therefore introduce the Littlewood-Paley projection operators $P_j$, which are Fourier multipliers with symbol $\text{Dil}_{2^j} \phi$. Littlewood-Paley theory guarantees that for $1 < r < \infty$,
%
\[ \| f \|_{L^r(\RR^d)} \sim_r \left( \sum_n \| P_n f \|_{L^r(\RR^d)}^2 \right)^{1/2}. \]
%
We also must introduce a slightly thickened function $\tilde{\phi} \in C_c^\infty(\RR^d)$ supported on $\{ \xi \in \RR^d: 1/4 \leq |\xi| \leq 4 \}$, equal to one on the support of $\phi$, and with $1 = \sum_j \text{Dil}_{2^j} \tilde{\phi}$, then we can introuce the Littlewood-Paley projections $\tilde{P}_j$ with symbol $\text{Dil}_{2^j} \tilde{\phi}$. For $1 < q < \infty$, we thus have
%
\[ \| m(D) f \|_{L^q(\RR^d)} \sim_q \left( \sum_j \| P_j m(D) f \|_{L^q(\RR^d)}^2 \right)^{1/2} = \left( \| P_j m(D) \{ \tilde{P}_j f \} \|_{L^q(\RR^d)}^2 \right)^{1/2}. \]
%
Now
%
\[ P_j m(D) \{ \tilde{P}_j f \} = \text{Dil}_{1/2^j} \{ m_{2^j}(D) \circ \text{Dil}_{2^j} \{ \tilde{P}_j f \} \}, \]
%
and so
%
\[ \| P_j m(D) \{ \tilde{P}_j f \} \|_{L^q(\RR^d)} = 2^{-jd/q} \| m_{2^j}(D) \text{Dil}_{2^j} \{ \tilde{P}_j f \} \|_{L^q(\RR^d)}. \]
%
We thus have
%
\begin{align*}
    \| m(D) f \|_{L^q(\RR^d)} &\lesssim_q \left( \sum_j 4^{jd(1/p - 1/q)} \| m_{2^j} \|_{M^{p,q}(\RR^d)}^2 \| \tilde{P}_j f \|_{L^p(\RR^d)}^2 \right)^{1/2}\\
    &\leq \left( \sup_t t^{d(1/p - 1/q)} \| m_t \|_{M^{p,q}(\RR^d)} \right) \left( \sum_j \| \tilde{P}_j f \|_{L^p(\RR^d)}^2 \right)^{1/2}\\
    &\sim_p \left( \sup_t t^{d(1/p - 1/q)} \| m_t \|_{M^{p,q}(\RR^d)} \right) \| f \|_{L^p(\RR^d)}.
\end{align*}
%
Moreover, this inequality is tight.
\end{comment}
Given a symbol $m$, for a value $t$ we define
%
\[ m_t = (\text{Dil}_{1/t} m) \cdot \beta, \]
%
and for an integer $j$ we define
%
\[ m_j = (\text{Dil}_{1/2^j} m) \cdot \beta. \]
%
Then $m_t$ describes the behaviour of the multiplier $m$ restricted to the annulus of frequencies $|\xi| \sim t$, but rescaled so that this behaviour is now lying on the annulus $|\xi| \sim 1$. Similarily, $m_j$ describes the rescaled behaviour of $m$ on the annulus of frequencies $|\xi| \sim 2^j$. Now we have
%
\[ m(D) = \sum\nolimits_j m_j(D / 2^j). \]
%
Littlewood-Paley theory tells us that if we define $P_j$ to be the Fourier multiplier with symbol $\text{Dil}_j \beta$, for any $g: \RR^d \to \CC$, and any $1 < p < \infty$,
%
\[ \| g \|_{L^p(\RR^d)} \sim_p \left\| \left( \sum_{j \in \ZZ} |P_j g|^2 \right)^{1/2} \right\|_{L^p(\RR^d)}. \]
%
A similar inequality holds where the projections $\{ P_j \}$ are replaced with projections $\{ \tilde{P}_j \}$, which are Fourier multipliers with symbol $\text{Dil}_j \tilde{\beta}$. Now
%
\[ [P_j \circ m(D)] \{ f \} = [m_j(D / 2^j)] \{ f \} = [m_j(D/2^j)] \{ \tilde{P}_j f \}, \]
%
If for an input function $f: \RR^d \to \CC$, we set $f_j = \tilde{P}_j f$, then to show $m \in M^{p,q}(\RR^d)$ for $1 < p,q < \infty$, it suffices to show a vector-valued inequality of the form
%
\[ \left\| \left( \sum\nolimits_j |m_j(D/2^j) f_j|^2 \right)^{1/2} \right\|_{L^q(\RR^d)} \lesssim \left\| \left( \sum\nolimits_j |f_j|^2 \right)^{1/2} \right\|_{L^p(\RR^d)}. \]
%
The square root cancellation here often makes it easy to combine bounds on the multipliers $m_j(D/2^j)$ to obtain bounds for the multiplier $m(D)$, and so it is natural to restrict an analysis to a study of the individual multipliers $m_j(D/2^j)$, which have the perk that they are compactly supported in frequency space on dyadic annuli. The rescaling symmetry of $\RR^d$ means that we can actually restrict our study to the multipliers $m_j(D)$, which are supported on the annulus $\{ 1/2 \leq |\xi| \leq 2 \}$. In the sequel, we call multipliers supported on this annulus \emph{unit scale multipliers}. Because of the square root cancellation in the Littlewood-Paley identity, we will find that most conditions guaranteeing the boundedness of Fourier multiplier operators are given by controlling each of the symbols $\{ m_j \}$ individually, without any assumptions on the interactions between these symbols.

Let us state some sufficient conditions to determine whether a multiplier is bounded. Young's convolution inequality, a very crude bound (not taking into account any oscillation in the convolution kernel) implies that if $p < q$ and $1/p - 1/q = 1 - 1/r$, then
%
\[ \| m \|_{M^{p,q}(\RR^d)} \leq \| k \|_{L^r(\RR^d)}, \]
%
and if $p = q$, then
%
\[ \| m \|_{M^{p,q}(\RR^d)} \leq \| k \|_{M^1(\RR^d)}. \]
%
%The H\"{o}rmander-Mikhlin theorem is a more sophisticated result which does take into account some of this oscillation. It states that if $1 < p < \infty$, and $\varepsilon > 0$, if $m$ is a unit scale multiplier lying in $L^\infty(\RR^d)$, and if $k$ is it's convolution kernel, then
% 
%\[ \| m \|_{M^p(\RR^d)} \lesssim_{p,\varepsilon} \int |k(x)| \langle x \rangle^\varepsilon\; dx. \]
%
%In light of the bound
%
%\[ \int k(x) \langle x \rangle^\varepsilon\; dx \leq \| k(x) \langle x \rangle^{d/2 + 2 \varepsilon} \|_{L^2_x} \| \langle x \rangle^{-d/2 - \varepsilon} \|_{L^2_x} \lesssim \| k(x) \langle x \rangle^{d/2 + 2 \varepsilon} \|_{L^2_x}, \]
%
%we obtain a slightly weaker, but easier to use, inequality of the form
%
%\[ \| m \|_{M^p(\RR^d)} \lesssim_\varepsilon \| m \|_{L^2_{d/2 + \varepsilon}}. \]
%
The \emph{H\"{o}rmander-Mikhlin multiplier theorem} states that for $1 < p < \infty$ and $\varepsilon > 0$, if we let $k_t$ be the convolution kernel corresponding to the multiplier $m_t(D)$, then
%
\[ \| m \|_{M^p(\RR^d)} \lesssim_{p,\varepsilon} \sup_t \int |k_t(x)| \langle x \rangle^\varepsilon\; dx. \]
%
Notice that in this situation, our assumption is on a uniform bound for the individual behaviour of the multipliers $\{ m_t \}$. In this situation, we do not need to postulate any further conditions on the interactions between these multipliers.

%\begin{itemize}
%    \item Sufficient conditions have also been obtained on the side of the Fourier multiplier:
    %
%    \begin{itemize}
%        \item An extension of the Hausdorff-Young inequality, due to Paley, tells us that for $1 < p \leq 2 \leq q < \infty$, with $1/p - 1/q = 1/r$, then
        %
%        \[ \| m \|_{M^{p,q}(\RR^d)} \lesssim \| m \|_{L^{r,\infty}(\RR^d)}. \]
%    \end{itemize}

%    \item One common heuristic to the theory is that the regularity of the symbol $m$, or equivalently, the decya of the convolution kernel $k$ away from the origin, implies some boundedness of the multiplier. There are several results of this form:
    %
%    \begin{itemize}
%        \item A very crude bound 

%        \item A slightly more sophisticated bound is that if $p \leq 2 \leq q$ and $1/p - 1/q = 1/r$, then
        %
%        \[ \| m \|_{M^{p,q}(\RR^d)} \lesssim \| m \|_{L^{r,\infty}(\RR^d)}. \]
        %

Conversely, we have some necessary conditions, that show some control over the mass and oscillatory properties of the multiplier $m$ is necessary in order to have $m \in M^{p,q}(\RR^d)$ for some exponents $p$ and $q$. For instance, if $1 \leq p \leq q \leq 2$, then
%
\[ \sup\nolimits_{t > 0} t^{d(1/p - 1/q)} \| m_t \|_{L^{q^*}(\RR^d)} \lesssim \| m \|_{M^{p,q}(\RR^d)}. \]
%
This implies, in particular, that $M^{p,q}(\RR^d) \subset L^{q^*}_{\text{loc}}(\RR^d - \{ 0 \})$, and in particular, any unit scale multiplier in the class is integrable. However, if $p < 2 < q$, there are unit scale multipliers in $M^{p,q}(\RR^d)$ which are distributions of positive order, i.e. that are not measures. On the convolution kernel side, for any $p \leq q$, if $m$ is a unit scale multiplier, if $\phi \in C_c^\infty(\RR^d)$ has Fourier transform equal to one on the annulus $1/4 \leq |\xi| \leq 4$, then
%
\[ \| k \|_{L^q(\RR^d)} = \| k * \phi \|_{L^q(\RR^d)} = \| m(D) \phi \|_{L^q(\RR^d)} \lesssim \| m \|_{M^{p,q}(\RR^d)}. \]
%
Thus some regularity on the spatial and frequency side is necessary in order for a Fourier multiplier to be bounded.

%For a general, non compactly supported multiplier $m$, if $k_t$ is the convolution kernel associated with the multiplier $m_t$, then one obtains the condition
%
%\[ \sup_{t > 0} \left\{ t^{d(1/p - 1/q)} \cdot \| k_t \|_{L^q(\RR^d)} \right\} \lesssim \| m \|_{M^{p,q}(\RR^d)}, \]
%
%One can phrase this bound in terms of the homogeneous Besov spaces $\dot{B}^{p,q}_s(\RR^d)$, the space consisting of all distributions $u$ on $\RR^d$ such that the norm
%
%\[ \| u \|_{\dot{B}^{p,q}_s(\RR^d)} = \left( \sum_{j = -\infty}^\infty \left( 2^{js} \| P_j u \|_{L^p(\RR^d)} \right)^q \right)^{1/q} = \| 2^{js} P_j u \|_{l^q(\ZZ) L^p(\RR^d)}, \]
%
%is finite, where $\{ P_j \}$ are a family of Little-wood Paley projection operators, with $P_j$ projecting onto a dyadic frequency band of radius $2^j$. If $k$ is the convolution kernel of a multiplier $m$, one can rescale the condition above to read that
%
%\[ \sup_{t > 0} \left\{ t^{-d/p^*} \| P_t k \|_{L^q(\RR^d)} \right\} \lesssim \| m \|_{M^{p,q}(\RR^d)}, \]
%
%i.e. that
%
%\[ \| k \|_{\dot{B}^{q,\infty}_{-d/p^*}} \lesssim \| m \|_{M^{p,q}(\RR^d)}. \]
%
%Thus we conclude that $k$ must satisfy some (admittedly weak) regularity assumptions to be the convolution kernel of a bounded Fourier multiplier.

\section{The Radial Multiplier Conjecture}

Despite the continuing lack of a complete characterization of the classes $M^{p,q}(\RR^d)$, it is surprising that we \emph{can} conjecture a characterization of the subspace of $M^{p,q}(\RR^d)$ for \emph{radial symbols} in this class, for an appropriate range of exponents. This conjecture is best phrased in terms of the result of \cite{GarrigosandSeeger}, which concerned radial multipliers $m$ whose associated operator $T$ is bounded from the $L^p$ norm to the $L^q$ norm \emph{restricted to radial functions}, i.e. such that the norm
%
\[ \| m \|_{M^{p,q}_{\text{rad}}(\RR^d)} = \sup \left\{ \frac{\| Tf \|_{L^q(\RR^d)}}{\| f \|_{L^p(\RR^d)}} : f \in \mathcal{S}(\RR^d)\ \text{and $f$ is radial} \right\} \]
%
is finite. The main result of \cite{GarrigosandSeeger} was that if $d > 1$, if $1 < p < 2d/(d+1)$, and if $p \leq q < 2$, then $M^{p,q}_{\text{rad}}(\RR^d)$ is a subset of $L^1_{\text{loc}}(\RR^d)$, and for any unit scale, integrable, radial multiplier $m$,
%
\[ \| m \|_{M^{p,q}_{\text{rad}}(\RR^d)} \sim_{p,q,d} \| k \|_{L^q(\RR^d)}. \]
%
%More generally, for any locally integrable radial symbol $m$,
%
%\[ \| m \|_{M^{p,q}_{\text{rad}}(\RR^d)} \sim_{p,q,d} \sup_{t > 0} t^{d(1/p - 1/q)} \| k_t \|_{L^q(\RR^d)} = \| k \|_{\dot{B}^{q,\infty}_{-d/p^*}}. \]
%
Moreover, this condition give \emph{precisely the range} under which this characterization holds. It is natural to conjecture that the same constraint continues to hold when we remove the constraint that our inputs $f$ are radial, i.e. that for unit scale, integrable, radial symbols $m$, for $d > 1$, $1 < p < 2d/(d+1)$, and for $p \leq q < 2$,
%
\[ \| m \|_{M^{p,q}(\RR^d)} \sim_{p,q,d} \| k \|_{L^q(\RR^d)} \]
%
%and for general locally integrable symbols $m$,
%
%\[ \| m \|_{M^{p,q}} \sim_{p,q,d} \| k \|_{\dot{B}_{-d/p^*}^{q,\infty}} \]
%
In the sequel, we call this the \emph{radial multiplier conjecture} in $\RR^d$.

\section{Range of Exponents in the Conjecture}

Before we move on, let us analyze the reason for the endpoints in the radial multiplier conjecture. To see why assuming $p \leq 2d/(d+1)$ is necessary, we can consider a variation of the \emph{ball multiplier} of Fefferman, and to see why assuming $p < 2d/(d+1)$ is necessary, we will consider a family of \emph{Bochner-Riesz bump functions}.

%
%For $p = 1$, the radial multiplier conjecture is true \emph{for compactly supported multipliers}, and one does not even need to assume that the multiplier is radial in this case. %Nonetheless, Littlewood-Paley does not hold in this setting.
%so an analogous result for radial multipliers which do not have compact support fails in some sense, i.e. we have
%
%\[ \| m \|_{M^{1,q}(\RR^d)} = \| k \|_{L^q(\RR^d)}, \]
%
%which for $q > 1$, is proportional to
%
%\[ \left( \sum \| P_j k \|_{L^q(\RR^d)}^2 \right)^{1/2} = \| k \|_{\dot{B}^{q,2}_0}. \]
%
%On the other hand, for non compactly supported multipliers $m$ the radial multiplier conjecture should say that
%
%\[ \| m \|_{M^{1,q}(\RR^d)} \sim \| k \|_{\dot{B}^{q,\infty}_0}, \]
%
%so the result fails `in the second order exponents', because Littlewood-Paley theory no longer applies.
% x has units M
% xi has units 1/M
% m is unitless
% Then k = m^ has units M^{-d}
% t also has units 1/M
% The projections P_t m^ have units M^{-d} as well.
% The L^p norm of P_t m^ has units M^{-d} * M^{d/p} = M^{-d/p^*}
% t^s | P_t m^ |_{L^p} has units A M^{-s-d/p^*}
% The L^q_t norm of t^s | P_t m^ |_{L^p} has units M^{-s-d/p^*-d/q}
% The Besov norm of m^ has units M^{-s-d/p^*-d/q}
% k has units M^{-d}
% For a dimensionless function f, k * f is dimensionless
% So | k * f |_{L^q} has units M^{d/q}
% And |f|_p has units M^{d/p}
% so the M^{p,q} norm of m, which is the supremum of quantities | k * f |_q / |f|_p, which have units M^{d/q-d/p}, also has units M^{d/q - d/p}.
% r has units M
% The projections P_r m are unitsless
% The L^p norm of P_r m has units M^{-d/p}
% r^s | P_r m^ |_{L^p} has units M^{s - d/p}
% The L^q_r norm of r^s |P_r m^ |_{L^p} has units M^{s - d/p + d/q}
% M^{1,q} norm has units M^{-d/q^*}
% The B^{p,2}_0 norm of m^ has units M^{-d/p^*-d/2}
% Suppose -d/q^* = -d/p^* - d/2
% 1/p^* - 1/q^* + 1/2 = 0
% Not quite matching up.
%

Let's begin with the multiplier $m(\xi) = \beta(|\xi|) \mathbf{I}(|\xi| \leq 1)$, for which it is easy to establish non-boundedness. This is because $m$ differs from the ball multiplier $\mathbf{I}(|\xi| \leq 1)$ by a compactly supported, smooth symbol, and thus $m$ has all the $L^p$ mapping properties that the ball multiplier has. Thus, by a result of Fefferman \cite{Fefferman}, the ball multiplier does not lie in $M^{p,q}(\RR^d)$ when $d > 1$ for any values of $p$ and $q$ except when $p = q = 2$, so the same is true of the multiplier $m$ given above. Now if $k$ is the convolution kernel of $m$, then polar coordinates gives that
%
\[ k(x) = \int_0^1 r^{d-1} \beta(r) \widehat{\sigma}(r x)\; dr, \]
%
where $\sigma$ is the surface measure of the unit sphere. Now standard asymptotics gives that there exists a constant $\omega$, depending only on $d$, such that
%
\[ \widehat{\sigma}(rx) = 2 r^{- \frac{d-1}{2}} |x|^{- \frac{d-1}{2}} \cos(2 \pi r |x| - \omega) + O(r^{- \frac{d+1}{2}} |x|^{- \frac{d+1}{2}}). \]
%
Plugging in these asymptotics and integrating by parts gives that
%
\[ |k(x)| \sim \langle x \rangle^{- \frac{d+1}{2}}. \]
%
This allows us to conclude that $k \in L^p(\RR^d)$ for $p > 2d/(d+1)$. Because of the $L^p$ mapping properties of $m$ given by Fefferman, we conclude that the radial multiplier conjecture cannot hold for $2d/(d+1) < p < 2$.

To show the radial multiplier conjecture cannot hold for $p = 2d/(d+1)$, we must consider multipliers slightly more carefully near the boundary of the unit ball. To do this, we will find a family of smooth, unit scale multipliers $\{ m_\delta \}$ which become more and more singular near the boundary of the unit ball as $\delta \to 0$, and such that for $2d/(d+1) \leq p \leq q \leq 2$,
%
\[ \lim_{\delta \to 0} \frac{\| m_\delta \|_{M^{p,p}(\RR^d)}}{\| k_\delta \|_{L^p(\RR^d)}} = \infty. \]
%
Consider the multiplier 
%
\[ m_\delta(\xi) = h \left( \frac{|\xi| - 1}{\delta} \right), \]
%
where $h \in C_c^\infty(\RR)$ is supported on $|\lambda| \leq 1$. The $m_\delta$ is an example of a Bochner-Riesz bump. If $k_\delta$ is the convolution kernel corresponding to the multiplier $m_\delta$, then we calculate that
%
\begin{align*}
    k_\delta(|x|) &= 2 \pi |x|^{1 - d/2} \int_{1 - \delta}^{1 + \delta} h \left( \frac{\lambda - 1}{\delta} \right) J_{d/2 - 1}(2 \pi \lambda |x|) \lambda^{d/2}\; d\lambda\\
    &= 2 \pi \delta |x|^{1 - d/2} \int_{-1}^1 h(\lambda) J_{d/2 - 1}(2 \pi (\delta \lambda + 1) |x|) (\delta \lambda + 1)^{d/2}\; d\lambda.
\end{align*}
%
As we have seen, for $s \geq 1$, Bessel function asymptotics tell us that, for any $N > 0$, modulo an error of order $O_N(s^{-N-1/2})$, we can write $J_{d/2-1}(s)$ as a linear combination of terms of the form $s^{-k-1/2} e^{\pm is}$, for $0 \leq k < N-1$. Thus we can write $k_\delta(x)$, modulo an error of order
%
\[ O_N \Bigg( |x|^{- \frac{d-1}{2} - N} \delta^{-N-1/2} \Bigg), \]
%
as a linear combination of terms of the form
%
\[ \delta |x|^{- \frac{d-1}{2} - k} \int_{-1}^1 h(\lambda) (\delta \lambda + 1)^{ \frac{d-1}{2} -k} e^{\pm 2 \pi i (\delta \lambda + 1)}\; d\lambda, \]
%
for $0 \leq k < N-1$. Integrating by parts gives that these terms are, for any $L > 0$,
%
\[ O_L( \delta^{1-L} |x|^{-\frac{d-1}{2} - k - L} ). \]
%
Taking $L$ arbitrarily large gives that
%
\[ \int_{|x| \geq 1/\delta} |k_\delta(x)|^p \lesssim \delta^{p \frac{d+1}{2} - d}. \]
%
For $1 \leq |x| \leq 1/\delta$, we take $L = 0$, which yields that
%
\[ \int_{1 \leq |x| \leq 1/\delta} |k_\delta(x)|^p \lesssim \delta^{p \frac{d+1}{2} - d }. \]
%
Conversely, for $|x| \leq 1$, we use the trivial bound $|k_\delta(x)| \leq \| m_\delta \|_{L^1} \lesssim \delta$. Thus we have
%
\[ \int_{|x| \leq 1} |k_\delta(x)|^p \lesssim \delta^p. \]
%
Putting all these bounds together, together, for $p < 2d/(d-1)$, we conclude that
%
\[ \| k_\delta \|_{L^p(\RR^d)} \lesssim \delta^{- \frac{d}{p} + \frac{d+1}{2}}. \]
%
If the radial multiplier conjecture held at this exponent, we would have for any Schwartz function $f \in \mathcal{S}(\RR^d)$,
%
\[ \| k_\delta * f \|_{L^p(\RR^d)} \lesssim \delta^{ - \frac{d}{p} + \frac{d+1}{2}} \| f \|_{L^p(\RR^d)}. \]
%
We now use the existence of Kakeya sets, ala the proof that the ball multiplier problem is unbounded, to justify that such a bound cannot be possible in the range $2d/(d-1) \leq p \leq 2$. It will be more convenient to show a bound of the form
%
\[ \| k_\delta * f \|_{L^q(\RR^d)} \lesssim \delta^{ \frac{d}{q} - \frac{d-1}{2} } \| f \|_{L^q(\RR^d)} \]
%
is not possible in the range $2 \leq q \leq 2d/(d-1)$, which is equivalent to the original bound by duality.

For each $\delta > 0$, we will choose a function $f_\delta$ so that the inequality above is not possible uniformly in $\delta$ as $\delta \to 0$. The function $f_\delta$ will be chosen to have Fourier support on the annulus $1 - 2 \delta \leq |\xi| \leq 1 + 2 \delta$. Cover the unit sphere of $\RR^d$ by a maximal collection $\Theta$ of $\delta$-separated points. Then $\#(\Theta) \sim \delta^{1-d}$. For each $\theta \in \Theta$, consider a cap centered at $(1 - 1.4\delta) \theta$, with length $\delta$ in the $\theta$ direction, and with length $\delta^{1/2}$ in the $d-1$ directions tangential to $\theta$. The choice is made so that $\kappa_\theta$ has essentially $10\%$ of it's mass on the support of $m_\delta$. We then consider a family of bump functions $\{ \phi_\theta \}$ supported on the family $\{ \kappa_\theta \}$, with magnitude roughly $\delta^{-(d+1)/2}$ on $\Theta$. For any points $\{ x_\theta : \theta \in \Theta \}$, consider the family of functions $\{ \chi_\theta \}$, with
%
\[ \widehat{\chi_\theta}(\xi) = e^{- 2 \pi i x_\Theta \cdot \xi} \phi_\theta(\xi). \]
%
For each $\theta$, $\chi_\theta$ then has mass concentrated on the dual rectangle $T_\theta$ of $\kappa_\theta$, which is centered at $x_\theta$, and has magnitude roughly $\delta^{-(d+1)/2}$ on $T_\theta$. The action of the Fourier multiplier $m_\delta(\sqrt{-\Delta})$, roughly speaking, is to cut the Fourier support of $\chi_\theta$ by a tenth. We lose $90 \%$ of the Fourier mass of $\theta$, but our support is also made ten times thinner in the tangential direction. As a result, $k_\delta * \chi_\theta$ will now have mass concentrated on a tube $T_\theta^*$, with the same center, but which is ten times longer than $T_\Theta$ in the tangential direction. Let's let $T_\Theta^+$ be the portion of $T_\Theta^*$ which lies at the opposite end of $T_\Theta^*$ to $T_\Theta$.

Our construction now relies on \emph{Kakeya} like phenomena. For any $\varepsilon > 0$, there exists a large $\delta_0$ such that for $\delta \leq \delta_0$, we can pick $\{ x_\theta \}$ such that tubes $\{ T_\theta \}$ are disjoint from one another, but such that the tubes $\{ T_\theta^+ \}$ have large overlap, in the sense that
%
\[ | \bigcup_\theta T_\theta^+ | \leq \varepsilon | \bigcup_\theta T_\theta|. \]
%
The supports of $k_\delta * \chi_\theta$ thus have lots of overlap on this set. However, it is difficult to control the sum $\sum_\theta k_\delta * \chi_\theta$ since these functions might have different signs where they overlap. To fix this, we define
%
\[ f_\delta = \sum_\theta \varepsilon_\theta \chi_\theta \]
%
where $\{ \varepsilon_\theta \}$ are independent $\{ -1, +1 \}$ valued Bernoulli random variables, because Khintchine's inequality implies that we have the pointwise inequality, for any $1 < q < \infty$, of the form
%
\[ \left( \EE \left| \sum_\theta \varepsilon_\theta \chi_\theta \right|^q \right)^{1/q} \sim \left( \sum_\Theta |\chi_\theta|^2 \right)^{1/2}. \]
%
It follows that, since the tubes $T_\theta$ are disjoint,
%
\begin{align*}
    \EE \| f_\delta \|_{L^q(\RR^d)}^q &\sim \int \left( \sum_\theta |\chi_\theta(x)|^2 \right)^{q/2}\; dx\\
    &\sim \left( \delta^{- \frac{d+1}{2}} \right)^q \left| \bigcup_\theta T_\theta \right|.
\end{align*}
%
Similarily, Khintchine's inequality can again be applied to
%
\begin{align*}
    k_\delta * f_R &= \sum_\theta \varepsilon_\theta (k_\delta * \chi_\theta).
\end{align*}
%
to conclude that
%
\begin{align*}
    \EE \| k_\delta * f_R \|_{L^q(\RR^d)}^q &\sim \int \left( \sum_\theta |k_\delta * \chi_\theta(x)|^2 \right)^{q/2}\; dx\\
    &\gtrsim \left( \delta^{- \frac{d+1}{2}} \right)^q \int \left( \sum_\theta \mathbf{I}_{T_\theta^+}(x) \right)^{q/2}\; dx.
\end{align*}
%
If the radial multiplier conjecture held for the particular value of $q$ we were considering, and $p$ is chosen so that $1/p + 1/q = 1$, then we would conclude that
%
\begin{align*}
    \left( \delta^{- \frac{d+1}{2}} \right)^q \int \left( \sum_\theta \mathbf{I}_{T_\theta^+}(x) \right)^{q/2}\; dx &\lesssim \EE \| k_\delta * f_R \|_{L^q(\RR^d)}^q\\
    &\lesssim \delta^{\left( \frac{d}{q} - \frac{d-1}{2} \right) q} \EE \| f \|_{L^q}^q\\
    &\lesssim \delta^{\left( \frac{d}{q} - \frac{d-1}{2} \right) q} \left( \delta^{- \frac{d+1}{2}} \right)^q \left| \bigcup_\theta T_\theta \right|\\
    &\lesssim \delta^{- d(q - 1)} \left| \bigcup_\theta T_\theta \right|.
\end{align*}
%
We may rearrange this to read that
%
\[ \int \left( \sum_\theta \mathbf{I}_{T_\theta^+}(x) \right)^{q/2}\; dx \lesssim \delta^{- \frac{d-1}{2} q + d} \left| \bigcup_\theta T_\theta \right|. \]
%
Interpolating the trivial bound
%
\[ \int \left( \sum_\theta \mathbf{I}_{T_\theta^+}(x) \right)\; dx \sim \left| \bigcup_\theta T_\theta \right| \]
%
with the fact that the sum is supported on the finite measure set $\bigcup_\theta T_\theta^+$ yields that
%
\[ \int \left( \sum_\theta \mathbf{I}_{T_\theta^+}(x) \right)^{q/2}\; dx \gtrsim \left| \bigcup_\theta T_\theta^+ \right|^{-(q/2 - 1)} \left| \bigcup_\theta T_\theta \right|^{q/2}. \]
%
Thus we conclude that
%
\[ \left| \bigcup_\theta T_\theta \right|^{q/2 - 1} \lesssim \delta^{- \frac{d-1}{2} q + d} \left| \bigcup_\theta T_\theta^+ \right|^{q/2 - 1}. \]
%
Or equivalently, that
%
\[ \left| \bigcup_\theta T_\theta \right| \lesssim \delta^{\frac{2}{q - 2} - (d-1)} \left| \bigcup_\theta T_\theta^+ \right|. \]
%
This is impossible if $2 \leq q \leq 2d/(d-1)$, since we have chosen this tubes so that as $\delta \to 0$,
%
\[ \left| \bigcup_\theta T_\theta^+ \right| = o \left( \left| \bigcup_\theta T_\theta \right| \right), \]
%
which would give a contradiction in the bound above.

That concludes this argument. We end by remarking that the construction we have made can still be performed for $q > 2d/(d-1)$. If the radial multiplier conjecture is true for the choice of $p$ with $1/p + 1/q = 1$, then this would therefore provide a bound of the form
%
\[ \left| \bigcup_\theta T_\theta \right| \lesssim \delta^{\frac{2}{q - 2} - (d-1)} \left| \bigcup_\theta T_\theta^+ \right|, \]
%
which we might rewrite as
%
\[ \left| \bigcup_\theta T_\theta^+ \right| \gtrsim \delta^{ \frac{d+1}{2} - \frac{2}{q-2}}. \]
%
Thus the radial multiplier conjecture provides bounds on the amount of overlap a union of $\delta$ tubes can have in space, which connects the result explicitly to the Kakeya conjecture. Indeed, this is to be expected, since the radial multiplier conjecture, for a particular range of $p$, would imply the same range for the Bochner-Riesz multiplier conjecture, which, by \cite{Tao}, gives results for the restriction conjecture, and thus the Kakeya maximal conjecture.

\section{Summary of Prior Results}

We now know, by the results of \cite{HeoandNazarovandSeeger} that the radial multiplier conjecture is true when $d \geq 4$ and when $1 < p < (2d-2)/(d+1)$. When $d = 4$, this was improved by \cite{Cladek}, who showed that the conjecture is true here when $1 < p < 36/29$, where $36/29 \approx (2d - 1.79)/(d+1)$. When $d = 3$, \cite{Cladek} also established a \emph{restricted weak type} bound
% 
\[ \| Tf \|_{L^p(\RR^n)} \lesssim \| f \|_{L^{p,1}(\RR^n)} \]
% 13/12
when $d = 3$ and $1 < p < 13/12$, where $13/12 \approx (2d - 1.66)/(d+1)$ But the radial multiplier conjecture has not yet been completely resolved in any dimension $n$, we do not have any strong type $L^p$ bounds when $d = 3$, and we don't have any bounds whatsoever when $d = 2$. One goal of this research project is to investigate whether one can use modern research techniques to improve upon these bounds.

The full proof of the radial multiplier is likely far beyond current research techniques. Indeed, it remains a major open problem in harmonic analysis to determine the range of exponents for which \emph{specific} radial Fourier multipliers are bounded in the range where the conjecture would apply, such as the Fourier multiplier on $\RR^d$ with symbol $m_\lambda(\xi) = \left( 1 - |\xi| \right)^\lambda_+$, the family of \emph{Bochner-Riesz multipliers}. The radial multiplier conjecture characterizes the range of the Bochner-Riesz multipliers, and thus the conjecture would also imply the Kakeya and restriction conjectures. All three of these results are major unsolved problems in harmonic analysis. On the other hand, the Bochner Riesz conjecture is completely resolved when $d = 2$, while in contrast, no results related to the radial multiplier conjecture are known in this dimension at all. And in any dimension $d > 2$, the range under which the Bochner-Riesz multiplier is known to hold \cite{GuoandOhandWangandWuandZhang} is strictly larger than the range under which the radial multiplier conjecture is known to hold, even for the restricted weak-type bounds obtained in \cite{Cladek}. Thus it still seems within hope that the techniques recently applied to improve results for Bochner-Riesz problem, such as broad-narrow analysis \cite{BourgainandGuth}, the polynomial Wolff axioms \cite{KatzandRogers}, and methods of incidence geometry and polynomial partitioning \cite{Zahl2} can be applied to give improvements to current results characterizing the boundedness of general radial Fourier multipliers.

Our hopes are further emboldened when we consult the proofs in \cite{HeoandNazarovandSeeger} and \cite{Cladek}, which reduce the radial multiplier conjecture to the study of upper bounds of quantities of the form
%
\[ \left\| \sum_{(y,r) \in \mathcal{E}} F_{y,r} \right\|_{L^p(\RR^n)}, \]
%
where $\mathcal{E} \subset \RR^n \times (0,\infty)$ is a finite collection of pairs, and $F_{y,r}$ is an oscillating function supported on a $O(1)$ neighborhood of a sphere of radius $r$ centered at a point $y$. The $L^p$ norm of this sum is closely related to the study of the tangential intersections of these spheres, a problem successfully studied in more combinatorial settings using incidence geometry and polynomial partitioning methods \cite{Zahl}, which provides further estimates that these methods might yield further estimates on the radial multiplier conjecture.

When $d = 3$, the results of \cite{Cladek} are only able to obtain bounds on the $L^p$ sums in the last paragraph when $\mathcal{E}$ is a Cartesian product of two subsets of $(0,\infty)$ and $\RR^d$. This is why only restricted weak-type bounds have been obtained in this dimension. It is therefore an interesting question whether different techniques enable one to extend the $L^p$ bounds of these sums when the set $\mathcal{E}$ is \emph{not} a Cartesian product, which would allow us to upgrade the result of \cite{Cladek} in $d = 3$ to give strong $L^p$ bounds. This question also has independent interest, because it would imply new results for the `endpoint' local smoothing conjecture, which concerns the regularity of solutions to the wave equation in $\RR^d$. Incidence geometry has been recently applied to yield results on the `non-endpoint' local smoothing conjecture \cite{GuthandWangandZhang}, which again suggests these techniques might be applied to yield the estimates needed to upgrade the result of \cite{Cladek} to give strong $L^p$-type bounds.

\section{Bessel Transforms and Radial Multipliers}

Given a function $h$ on $[0,\infty)$, we define the \emph{$d$-dimensional Fourier-Bessel transform} of $h$ as
%
 \[ \mathcal{B}_d h(r) = r^{- \frac{d-2}{2}} \int_0^\infty \rho^{d/2} h(\rho) J_{d/2-1}(\rho r)\; d\rho, \]
%
where $J_{d/2-1}$ is the Bessel function of order $d/2 - 1$. If $m$ is the function on $\RR^d$ defined by setting $m(\xi) = h(|\xi|)$, then
%
\[ \widehat{m}(x) = \mathcal{B}_d h(|x|). \]
%
We therefore conclude that the condition in the radial multiplier conjecture for unit scale multipliers becomes that
%
\[ \| m \|_{M^{p,q}(\RR^d)} \sim \left( \int_0^\infty r^{d-1} |\mathcal{B}_d h(r)|^q\; dr \right)^{1/q}. \]
%
%We therefore conclude that, if we let $h_t = \phi \cdot \text{Dil}_{1/t} h$, then the condition in the radial multiplier conjecture becomes
%
%\[ \sup_{t > 0} t^{d(1/p - 1/q)} \left( \int_0^\infty r^{d-1} |\{ \mathcal{B}_d h_t \}(r)|^q \right)^{1/q} < \infty. \]
%
One can use Bessel function asymptotics to relate this quantity to a condition on the one-dimensional Fourier transform of $h$ (extended to an even function on $\RR$), if $h$ is a unit scale multiplier. Indeed, in \cite{GarrigosandSeeger}, Garrig\'{o}s and Seeger show that for $1 < q < 2$, for such multipliers, we have
%
\[ \left( \int_0^\infty r^{d-1} |\mathcal{B}_d h(r)|^q\; dr \right)^{1/q} \sim_{d,q} \left( \int_0^\infty \langle \lambda \rangle^{(d-1)(1 - q/2)} |\widehat{h}(\lambda)|^q \right)^{1/q}. \]
%
This characterization of the conclusion of the radial multiplier conjecture will come in handy later on when we discuss the extension of the radial multiplier conjecture to Riemannian manifolds.
%    If $m(\xi) = h(|\xi|)$ is radial, the condition that
    %
%    \[ \sup_{t > 0} t^{d(1/p - 1/q)} \| k_t \|_{L^q(\RR^d)} < \infty \]
    %
%    can be rephrased in terms involving the Fourier transform of $h$. Namely, we have
    %
%    \begin{align*}
%        & \sup_{t > 0} t^{d(1/p - 1/q)} \| k_t \|_{L^q(\RR^d)}\\
%        &\quad\quad \sim \sup_{t > 0} t^{d(1/p - 1/q)} \left( \int_{t/2 \leq |s| \leq 2t} |\widehat{h}(s)|^q (1 + |s|)^{(d-1)(1 - q/2)}\; dt \right)^{1/q}.
%    \end{align*}
    %
%    The weight inside the norm prevents us from easily converting this condition into a homogeneous Besov condition on the function $w$, but roughly speaking, we have $|\widehat{h}(s)| \lesssim s^{d(1/p-1/q)} \langle s \rangle^{-(d-1)(1-q/2)}$ for \emph{most} inputs $x$.

\section{Multipliers on Riemannian Manifolds}

Fix a geodesically complete Riemannian manifold $X$. The operator $\sqrt{-\Delta}$, defined initially on $C_c^\infty(X)$, is then essentially self-adjoint on $L^2(X)$. The spectral calculus of unbounded operators can then be used to define operators of the form $h(\sqrt{-\Delta})$ for each locally bounded, Borel measurable function $h: [0,\infty) \to \CC$. These operators are analogous to the radial multipliers studied in the Euclidean setting, and we will also call these operators \emph{radial multipliers on $X$}. Just like multiplier operators on $\RR^d$ are crucial to an understanding of the interactions between the functions $e_\xi(x) = e^{2 \pi i \xi \cdot x}$ on $\RR^n$, understanding the operators $h(\sqrt{-\Delta})$ is crucial to understanding the interactions of eigenfunctions of the Laplace-Beltrami operator on $X$.

We let $M^{p,q}(X, \sqrt{-\Delta} )$ denote the family of all locally bounded, Borel measurable functions $h$ such that the operator $T_h = h(\sqrt{-\Delta})$ extends to a bounded operator from $L^p(X)$ to $L^q(X)$, with the analogous operator norm, though, when there is no ambiguity, we will overload notation and write this space as $M^{p,q}(X)$.

To avoid technicalities, we will mainly focus on the study of radial multipliers on \emph{compact} Riemannian manifolds. Such manifolds are automatically geodesically complete. Moreover, on such a manifold, the spectrum of $\Delta$ forms a discrete set $\Lambda \subset (0,\infty)$, and for each $\lambda \in \Lambda$, there is a finite dimensional space $H_\lambda \subset C^\infty(X)$, such that we have an orthogonal decomposition
%
\[ L^2(X) = \bigoplus_{\lambda \in \Lambda} H_\lambda, \]
%
and for $f \in H_\lambda$, $\Delta f = - \lambda^2 f$. For any function $h: [0,\infty) \to \CC$, we then have that for $f \in C^\infty(X)$,
%
\[ h(\sqrt{-\Delta}) f = \sum_{\lambda \in \Lambda} h(\lambda) P_\lambda f, \]
%
where $P_\lambda: L^2(X) \to H_\lambda$ is orthogonal projection onto $H_\lambda$. If, for each $H_\lambda$, we select an orthonormal basis $\{ e_{\lambda,n} \}$, then we can write
%
\[ h(\sqrt{-\Delta}) f = \sum_{\lambda \in \Lambda} \sum_n h(\lambda) \langle f, e_{\lambda,n} \rangle e_{\lambda,n}, \]
%
which allows us to write the operator in terms of inner products.

There are several model cases which are natural to consider, each having constant curvature:
%
\begin{itemize}
    \item The torii $\TT^d = \RR^d / \ZZ^d$, for which
    %
    \[ \Lambda = \{ \sqrt{n}: n > 0 \}, \]
    %
    and such that for $\lambda \in \Lambda$, $H_\lambda$ is spanned by the functions of the form
    %
    \[ e_\xi(x) = e^{i \xi \cdot x} \]
    %
    where $\xi \in \ZZ^d$, and $|\xi| = \lambda$. The study of the operators $h(\sqrt{-\Delta})$ is thus naturally connected to the theory of Fourier series, and is very similar to the theory of multipliers on $\RR^d$.

    \item The sphere $S^d$ in $\RR^{d+1}$. Here
    %
    \[ \Lambda = \left\{ \sqrt{n(n+d-1)}: n > 0 \right\} \]
    %
    If $\lambda_n = \sqrt{n(n+d-1)}$, then $H_{\lambda_n}$ is the space of all \emph{spherical harmonics} of degree $n$. Then
    % a^{d-1} - (a - 2)^{d-1}
    % O(a^{d-2})
    \[ \dim(H_{\lambda_n}) = { d + n - 1 \choose n } - { d + n - 3 \choose n - 2 } \lesssim_d n^{d-2}. \]
    %
    The theory of the Laplacian on $S^d$ is closely connected to the representation theory of the non-commutative Lie group $SO(d+1)$.

    \item For a discrete, cocompact subgroup $\Gamma \subset PSL(2,\RR)$, we can consider the Riemannian manifold $\mathbf{H} / \Gamma$, obtained by quotienting hyperbolic space by the family of isometries corresponding to $\Gamma$. The interesting feature of this group is that the spectral theory of $\Delta$ is closely tied to the study of the representation theory of the non-commutative group Lie group $G / \Gamma$.
\end{itemize}
%
The Killing-Hopf theorem says that every manifold of constant curvature has either $\RR^d$, $S^d$, or $\mathbf{H}^d$ as a universal cover, so these provide a good family of simple manifolds, with a fairly clear family of eigenfunctions, with which we can begin our analysis.

The study of multipliers on a compact Riemannian manifold has a certain technical problem, which the Euclidean case did not have. If $h$ has compact support, then the operator
%
\[ h \left( \sqrt{-\Delta} \right) f = \sum h(\lambda_n) \langle f, e_n \rangle \cdot e_n \]
%
has a finite sum on the right hand side. And so by the triangle inequality, each such operator will be trivially bounded from $L^p(X)$ to $L^q(X)$ for any exponents $p$ and $q$. Thus $M^{p,q}(X)$ trivially contains all compactly supported radial multipliers. This trivializes the study of compactly supported radial multipliers in some sense, which is the complete opposite of the Euclidean case, where Littlewood-Paley type technology allowed us to reduce the study of general multipliers to compactly supported radial multipliers. The key step to fixing this problem is to recognize that Euclidean multipliers automatically have rescaling symmetries, whereas this is not present in the case of compact Riemannian manifolds. To get around this we add a rescaling symmetry into our problem, i.e. we study conditions that ensure we have a bound of the form
%
\[ \sup_R R^{d(1/q - 1/p)} \| \text{Dil}_R h \|_{M^{p,q}(X)} < \infty. \]
%
The exponent of $R$ here emerges from the fact that
%
\[ \| \text{Dil}_R h \|_{M^{p,q}(\RR^d)} = R^{d(1/q - 1/p)} \| h \|_{M^{p,q}(\RR^d)}. \]
%
We let $M^{p,q}_{\text{Dil}}(X)$ denote the family of all multipliers for which the inequality above holds, and give it the norm induced by the quantity on the left hand side, and set $M^p_{\text{Dil}}(X) = M^{p,p}_{\text{Dil}}(X)$. A transference principle of Mitjagin \cite{Mitjagin} (see \cite{KenigStantonTomas} for a version of the result more accessible online) shows that if $X$ is a compact Riemannian manifold, and $h: (0,\infty) \to \CC$ is a bounded, Borel measurable function, then
%
\[ \| h \|_{M^{p,q}(\RR^d, \sqrt{-\Delta})} \lesssim_{X,p,q} \| h \|_{M^{p,q}_{\text{Dil}}(X, \sqrt{-\Delta})}. \]
%
Thus, in some sense, the dilation-invariant Fourier multiplier problem on a compact manifold $X$ is at least as hard as it is on $\RR^d$. Another goal of this research project is to try and extend the radial multiplier conjecture to the setting of dilation invariant bounds for multipliers of the Laplacian on Riemannian manifolds.

The study of dilation invariant radial multipliers on $\TT^d$ is essentially exactly the same as on $\RR^d$, as we might guess from the crude observation that the family of eigenfunctions to the Laplacian are similar in both domains. More precisely, we can show that for any locally bounded, Borel measurable function $h: (0,\infty) \to \RR$, such that every point in $(0,\infty)$ is a Lebesgue point of $h$,
%
\[ \| h \|_{M^{p,q}_{\text{Dil}}(\TT^d)} \sim_{p,q,d} \| h \|_{M^{p,q}(\RR^d)}. \]
%
A proof of this result is given by Theorem 3.6.7 of \cite{Grafakos}. Whether an analogous result remains true for more general Riemannian manifolds remains unclear, especially since the family of eigenfunctions to the Laplacian can take on various different forms on these manifolds.

What \emph{is} easy to establish, is that the theory of multipliers in $M^2_{\text{Dil}}(X)$ is relatively the same. Indeed, applying orthogonality, we calculate that for any function $h: (0,\infty) \to \CC$, we have
%
\begin{align*}
    \| h(\sqrt{-\Delta}) f \|_{L^2(X)} &= \left\| \sum_\lambda h(\lambda) E_\lambda f \right\|_{L^2(X)}\\
    &= \left( \sum_\lambda |h(\lambda)|^2 \| E_\lambda f \|_{L^2(X)}^2 \right)^{1/2}\\
    &\leq \left( \sup_{\lambda \in \sigma(\sqrt{-\Delta})} |h(\lambda)| \right) \left( \sum_\lambda \| E_\lambda f \|_{L^2(X)}^2 \right)^{1/2}\\
    &= \left( \sup_{\lambda \in \sigma(\sqrt{-\Delta})} |h(\lambda)| \right) \| f \|_{L^2(X)}.
\end{align*}
%
Taking $f$ to be an eigenfunction with eigenvalue $\lambda$ which maximizes the value of $|h(\lambda)|$ shows this inequality is tight, i.e. we have
%
\[ \| h \|_{M^{2,2}(X)} = \sup_{\lambda \in \sigma(\sqrt{-\Delta})} |h(\lambda)|. \]
%
Now applying an arbitrary dilation to $h$, we conclude that
%
\[ \| h \|_{M^{2,2}_{\text{Dil}} (X)} = \sup_{\lambda > 0} |h(\lambda)|. \]
%
Thus $M^2_{\text{Dil}}(X)$ consists precisely of the bounded functions on $(0,\infty)$.

How about the spaces $M^{p,2}(X)$, for $1 \leq p < 2$. We know by the transference principle that any multiplier $h$ in $M^{p,2}_{\text{Dil}}(X)$ must satisfy a bound of the form
%
\[ \left( \int_0^\infty r^{d-1} |h(r)|^q \right)^{1/q} < \infty, \]
%
where $q = 2p/(2-p)$. Is this sufficient? The analogous characterization on $\RR^d$ is proved using the Hausdorff-Young inequality, plus orthogonality. One way to state the Hausdorff-Young inequality on $\TT^d$ is that for $1 \leq p \leq 2$, and $f \in C^\infty(\TT^d)$,
%
\[ \left( \sum_\lambda \sum_n |\langle f, e_{\lambda,n} \rangle|^{p^*} \right)^{1/p^*} \leq \| f \|_{L^p(\TT^d)}. \]
%
To obtain an analogue result on a general compact manifold $X$, we can interpolate Parseval's inequality
%
\[ \left( \sum_\lambda \sum_n |\langle f, e_{\lambda,n} \rangle|^2 \right)^2 \leq \| f \|_{L^2(\TT^d)}. \]
%
with the results of Sogge (TODO: FIND PAPER) that say that
%
\[ \sup_{\lambda,n} |\langle f, e_{\lambda,n} \rangle| \lambda^{- \frac{d-1}{2}} \lesssim \| f \|_{L^1(X)}, \]
%
which yields that for $1 \leq p < 2$, that
% 1/p^* = a/infty + (1-a)/2
% a = 1 - 2/p^*
\[ \left( \sum_{\lambda,n} |\langle f, e_{\lambda,n} \rangle|^{p^*} \lambda^{- (d-1)(1/p - 1/2)} \right)^{1/p^*} \lesssim \| f \|_{L^p(X)}. \]
%
We therefore conclude that for $f \in C^\infty(X)$,
%
\begin{align*}
    &\| h(\sqrt{-\Delta}) f \|_{L^2(X)}\\
    &\quad= \left( \sum_\lambda \sum_n |h(\lambda)|^2 |\langle f, e_{\lambda,n} \rangle|^2 \right)^{1/2}\\
    &\quad= \Bigg( \sum_\lambda \sum_n \Big[ |h(\lambda)|^2 \lambda^{(d-1)(1/p - 1/2)(2/p^*)} \Big]\\
    &\quad\quad\quad\quad\quad\quad\quad\quad \Big[ |\langle f, e_{\lambda,n} \rangle|^2 \lambda^{-(d-1)(1/p - 1/2)(2/p^*)} \Big] \Bigg)^{1/2}\\
    &\quad\leq \left( \sum_\lambda \sum_n |h(\lambda)|^{\frac{2p}{2-p}} \lambda^{(d-1)(1 - 1/p)} \right)^{\frac{2-p}{2p}}\\
    &\quad\quad\quad\quad \left( \sum |\langle f, e_{\lambda,n} \rangle|^{p^*} \lambda^{-(d-1)(1/p - 1/2)} \right)^{1/p^*}\\
    &\quad\lesssim \left( \sum_\lambda \sum_n |h(\lambda)|^q \lambda^{(d-1)(1 - 1/p)} \right)^{1/q} \| f \|_{L^p(X)}.
\end{align*}
%
In the model case of $S^d$, we have that
%
\begin{align*}
    \| h(\sqrt{-\Delta}) f \|_{L^2(X)} &\lesssim \left( \sum_n |h(\lambda_n)|^q n^{2(d-1)(1 - 1/p) + (d-2)} \right)^{1/q} \| f \|_{L^p(X)}\\
    &\lesssim \left( \sum_n |h(\lambda_n)|^q n^{(d-1)(2(1 - 1/p) + 1) - 1} \right)^{1/q} \| f \|_{L^p(X)}.
\end{align*}
%
In particular, we have that
%
\begin{align*}
    \| h \|_{M^{p,2}_{\text{Dil}}(X)} &\lesssim \sup_R \left( \sum_\lambda \sum_n |h(\lambda_n/R)|^q n^{(d-1)(2(1 - 1/p) + 1) - 1} \right)^{1/q}.
\end{align*}
%
TODO: Determine if we can do something more optimal here. TODO: In light of Sogge's bounds, there's probably a counterexample here.

The spaces $M^{1,q}_{\text{Dil}}(X)$ are a little more tricky, since we do not have a precise theory of the Fourier transform in the setting of general Riemannian manifolds. To take a look at these bounds, we recall that $L^1 \to L^q$ bounds of an operator are characterized by Schur's test. If $\{ e_n \}$ is a $C^\infty(X)$ basis of eigenfunctions on $X$, with $\Delta e_n = - \lambda_n^2 e_n$, then
%
\[ h(\sqrt{-\Delta}) f(x) = \sum h(\lambda_n) \langle f, e_n \rangle e_n(x) = \int \left( \sum_n h(\lambda_n) e_n(x) \overline{e_n(y)} \right) f(y)\; dy. \]
%
Thus the kernel of $h(\sqrt{-\Delta})$ is precisely $K(x,y) = \sum_n h(\lambda_n) e_n(x) \overline{e_n(y)}$, and we conclude by Schur's test that
%
\[ \| h \|_{M^{1,q}(X)} = \left\| \sum_n h(\lambda_n) e_n(x) \overline{e_n(y)} \right\|_{L^\infty_y L^q_x}. \]
%
In the case $X = \RR^n$, the analogous kernel is
%
\[ K(x,y) = \int_{\RR^d} h(|\xi|) e^{2 \pi i \xi \cdot x} \overline{e^{2 \pi i \xi \cdot y}}, \]
%
which can be explicitly reduced to $K(x,y) = \mathcal{B}_d h(|x - y|)$, and the condition of being contained in $M^{1,q}(\RR^d)$ then becomes that
%
\[ \left( \int r^{d-1} |\mathcal{B}_d h(r)|^q\; dr \right)^{1/q} < \infty. \]
%
If $h$ is compactly supported, and $1 < q < 2$, then we have seen that this condition is equivalent to the condition that
%
\[ \left( \int \langle t \rangle^{(d-1)(1 - q/2)} |\widehat{h}(t)|^q\; dt \right)^{1/q} < \infty. \]
%
In the general setting we do not have quite as nice a formula, but we can still \emph{force} the Fourier transform into the equation to see if it can be used to understand these quantities (which will be necessary for studying the radial multiplier conjecture). There are two approaches here, the first approach is to write
%
\begin{align*}
    K(x,y) &= \sum_n h(\lambda_n) e_n(x) \overline{e_n(y)}\\
    &= \sum_n \left( \int \widehat{h}(t) e^{2 \pi i t \lambda_n} e_n(x) \overline{e_n(y)}\; dt \right) \\
    &= \int \widehat{h}(t) \left( \sum_n e^{2 \pi i t \lambda_n} e_n(x) \overline{e_n(y)} \right)\; dt\\
    &= \int \widehat{h}(t) HW_t(x,y)\; dt,
\end{align*}
%
where $HW_t(x,y) = \sum_n e^{2 \pi i t \lambda_n} e_n(x) \overline{e_n(y)}$ is the kernel of the \emph{half-wave propogator} $e^{2 \pi i t \sqrt{-\Delta}}$ on $X$. The connection between radial multipliers on $X$ and the Fourier transform of their symbol is therefore closely related to the study of the solutions to the half-wave equation $\partial_t = \sqrt{-\Delta}$ on $X$. Alternatively, we can employ the cosine transform, since $h$ is assumed to be extended to an even function, and write
%
\begin{align*}
    K(x,y) &= \sum_n h(\lambda_n) e_n(x) \overline{e_n(y)}\\
    &= \sum_n \left( 2 \int_0^\infty \widehat{h}(t) \cos(2 \pi t \lambda_n) e_n(x) \overline{e_n(y)}\; dt \right)\\
    &= 2 \int_0^\infty \widehat{h}(t) \left( \sum_n \cos(2 \pi t \lambda_n) e_n(x) \overline{e_n(y)} \right)\; dt\\
    &= 2 \int_0^\infty \widehat{h}(t) W_t(x,y)\; dt,
\end{align*}
%
where $W_t(x,y)$ is the kernel of the \emph{wave propogator} $\cos(2 \pi t \sqrt{-\Delta})$ on $X$. Thus the connenction of radial multipliers on $X$ and their Fourier transform is related to the study of the solutions to the wave equation $\partial_t^2 = \Delta$. The half-wave equation and the wave equation are certainly connected, but the latter has the advantage of finite speed of propogation.

An analogous characterization of the form
%
\[ \| h \|_{M^{1,q}(X)} \lesssim \left( \int \langle t \rangle^{(d-1)(1 - q/2)} |\widehat{h}(t)|^q\; dt \right)^{1/q} \]
%
would follow if and only if we could prove a general inequality of the form
%
\[ \left\| \int a(t) W_t(x,y)\; dt \right\|_{L^\infty_y L^q_x} \lesssim \left( \int \langle t \rangle^{(d-1)(1 - q/2)} |a(t)|^q\; dt \right)^{1/q} \]
%
to hold. By interpolation, it would suffice to prove that
%
\[ \left\| \int a(t) W_t(x,y)\; dt \right\|_{L^\infty_y L^{1,\infty}_x} \lesssim \int \langle t \rangle^{\frac{d-1}{2}} |a(t)|\; dt \]
%
and
%
\[ \left\| \int a(t) W_t(x,y)\; dt \right\|_{L^\infty_y L^{2,\infty}_x} \lesssim \left( \int |a(t)|^2\; dt \right)^{1/2} \]


By finite speed of propogation, we may replace this over an integral of $x$ over a ball of radius $t$ about $y$. This integral does become singular as we approach $x$ near the boundary of this ball, i.e. it blows up like $(t^2 - d(x,y)^2)^{- \frac{d-1}{2}}$. This is non-integral, unless $d = 2$, in which case the integral scales like $\theta(t)$. Thus the $L^1$ integral becomes

% r = t sin(theta)
% dr = t cos(theta) dtheta
\begin{align*}
    \int_0^t & r^{d-1} (t^2 - r^2)^{- \frac{d-1}{2}}\; dr\\
    &= t \int_0^{\pi/2} \sin(\theta)^{d-1} \cos(\theta)^{2-d}\; d\theta.
\end{align*}

One way to interpret $\int a(t) W_t(x,y)\; dt\; dx$ is as th



TODO: What techniques can we use to obtain this bound? TODO: Can we come up with a proof of this bound in the model case $X = \RR^d$, i.e. an alternate proof of the characterization of $L^1 \to L^q$ boundedness? TODO: Can we apply the theory of fourier integral operators here?
%
%H\"{o}lder is \emph{not} good enough in this situation, since it leads to quantities of the form
%
%\[ \| (1 + |t|)^{-(d-1)(1 - q/2)} W_t \|_{L^\infty_y L^q_x L^{q^*}_t}, \]
%
%and the singularities of $W_t$ on the light cone mean the $L^{q^*}_t$ norm is infinite for all $x$ and $y$. But we should expect to do better than H\"{o}lder, since we only expect H\"{o}lder's inequality to be tight when $\widehat{h}$ is close to a scalar multiple to $t \mapsto W_t(x,y)$, and this cannot be true for all $x$ and $y$. TODO: Think about this more.

Directly translating the assumptions of the radial multiplier conjecture to this setting yields the following statement: If $h: [0,\infty) \to \RR$ is a function supported at the unit scale, and we define
%
\[ C_{p,q}(h) = \left( \int |\widehat{h}_t(s)|^q \langle s \rangle^{(d-1)(1 - q/2)}\; ds \right)^{1/q}, \]
%\[ C_{p,q}(h) = \sup_{t > 0} t^{d(1/p - 1/q)} \left( \int |\widehat{h}_t(s)|^q (1 + |s|)^{(d-1)(1 - q/2)}\; ds \right)^{1/q}, \]
%
then for what values of $p$ and $q$ is is true that the inequality
%
\[ \| h \|_{M^{p,q}_{\text{Dil}}(X)} \lesssim C_{p,q}(h) \]
%
still holds. Mitjagin's result implies that we require $1 < p < 2d/(d+1)$ and $p \leq q < 2$, and we conjecture that, perhaps under appropriate assumptions on $X$, we can achieve similar ranges of exponents as have been obtained for the Euclidean radial multiplier conjecture.

On general compact manifolds, there are difficulties arising from a generalization of the radial multiplier conjecture, connected to the fact that analogues of the Kakeya / Nikodym conjecture are false in this general setting \cite{Minicozzi}. But these problems do not arise for constant curvature manifolds, like the sphere. The sphere also has over special properties which make it especially amenable to analysis, such as the fact that solutions to the wave equation on spheres are periodic. Best of all, there are already results which achieve the analogue of \cite{GarrigosandSeeger} on the sphere. Thus it seems reasonable that current research techniques can obtain interesting results for radial multipliers on the sphere, at least in the ranges established in \cite{HeoandNazarovandSeeger} or even \cite{Cladek}.

\section{Summary}

In conclusion, the results of \cite{HeoandNazarovandSeeger} and \cite{Cladek} indicate three lines of questioning about radial Fourier multiplier operators, which current research techniques place us in reach of resolving. The first question is whether we can extend the range of exponents upon which the conjecture of \cite{GarrigosandSeeger} is true, at least in the case $d = 2$ where Bochner-Riesz has been solved. The second is whether we can use more sophisticated arguments to prove the $L^p$ sum bounds obtained in \cite{Cladek} when $d = 3$ when the sums are no longer Cartesian products, thus obtaining strong $L^p$ characterizations in this settiong, as well as new results about the endpoint local smoothing conjecture. The third question is whether we can generalize these bounds obtained in these two papers to study radial Fourier multipliers on the sphere. 




\chapter{Distributions on Riemannian Manifolds}

How do we work with distributions on $\RR^d$? We first identify a vector space of test functions, say, the space $\DD(\RR^d)$ of smooth, compactly supported functions, the space $\EC(\RR^d)$ of all smooth functions, or the space $\SW(\RR^d)$ of Schwartz functions. The distributions are then formally defined as the dual space of this class of test functions. To actually work with these distributions, we find an explicit way to represent them, via a bilinear pairing; for $\DD(\RR^d)$, the bilinear pairing $\EC(\RR^d) \times \DD(\RR^d) \to \CC$ given by
%
\[ (\phi,\psi) \mapsto \int_{\RR^d} \phi(x) \psi(x)\; dx. \]
%
This pairing allows us to naturally identify certain distributions in $\DD(\RR^d)^*$ as elements of $\EC(\RR^d)$ via this pairing, and actually, \emph{all distributions} are weak limits of such distributions. Thus we can intuitively study elements of $\DD(\RR^d)^*$ as if they behaved like elements of $\EC(\RR^d)$, at least when integrated against elements of $\DD(\RR^d)$. Reversing this pairing allows us to think of elements of $\EC(\RR^d)^*$ as elements of $\DD(\RR^d)$, and the pairing $\SW(\RR^d) \times \SW(\RR^d) \to \CC$ allows us to think of tempered distributions of $\SW(\RR^d)$ as if they themselves were elements of $\SW(\RR^d)$.

The bilinear pairings here often behave well with respect to natural operations on the respective spaces. For instance, we have integration by parts identities
%
\[ ( \partial^\alpha \phi, \psi ) = (-1)^{|\alpha|} (\phi, \partial^\alpha \psi) \]
%
for all the pairings above. And for the pairing of $\SW(\RR^d)$, we have the multiplication formula
%
\[ ( \phi, \psi ) = \Big( \mathcal{F} \phi , \overline{\mathcal{F} \psi} \Big) \]
%
where $\mathcal{F}$ is the Fourier transform. By taking weak limits, this allows us to extend the derivative and Fourier transform operations for distributions. In particular, we can take the derivatives of elements of $\DD(\RR^d)^*$, $\EC(\RR^d)^*$ and $\SW(\RR^d)^*$, and we can take the Fourier transform of elements of $\SW(\RR^d)^*$. By taking vector pairings, i.e. by setting
%
\[ (v, w) = \int (v \cdot w)\; dx \]
%
where $v$ and $w$ are vector fields, we can define vector-valued distributions, and the theory above extends analogously.

We can also defined other more technical operations on distributions $u$ as if they were functions, provided that we have information about their \emph{wavefront sets} $\text{WF}(u)$, which is a subset of $\RR^d_x \times \RR^d_\xi$, conic in the variable $\xi$, which give information about the position and `direction' of the singularities of $u$. As an example, if $f: \RR^n_x \to \RR^m_y$ is smooth, and $u \in \DD(\RR^m)$ is a distribution such that $\text{WF}(u)$ does not contain any point of the form $(y,\eta)$ for which there exists $x \in f^{-1}(y)$ with $Df(x)^t \eta = 0$, then we can define $f^* u$, which can be interpreted as a weak limit of a sequence $\{ f^* u_n \}$, where $u_n \in \EC(\RR^m)$ converges weakly to $u$ in a way respecting the wavefront set of $u$; the advantage of this is that $f^* u_n$ is just $u_n \circ f$. If $\phi \in \EC(\RR^m)$, then the chain rule
%
\[ [\nabla_x (f^* \phi)](x) = Df(x)^T f^*( \nabla_y \phi )(x) \]
%
can be written as
%
\[ ( \nabla_x (f^* \phi), v) = ( f^*( \nabla_y \phi ), Df \cdot v ),  \]
%
which allows us to extend the chain rule to the pullback of distributions.

Now let's do the same thing on a Riemannian manifold $M^d$. Here we also have natural spaces of test functions, i.e. the spaces $\EC(M)$ and $\DD(M)$ of smooth functions, the latter of which specified to have compact support. Here the natural pairing $\EC(M) \times \DD(M) \to \CC$ is given by integration against the \emph{volume measure} on the manifold $M$, i.e. the pairing is given by
%
\[ (\phi,\psi) \mapsto \int_M \phi(x) \psi(x)\; dV(x). \]
%
This pairing allows us to identify $\EC(M)^*$ and $\DD(M)^*$ with weak limits of elements of $\DD(M)$ and $\EC(M)$ respectively. It will also be convenient to consider \emph{vector-valued distributions}, i.e. the dual spaces of the spaces $\EC(\Gamma(TM))$ and $\DD(\Gamma(TM))$ of smooth vector fields, the latter of which limited to have compact support. These spaces have a pairing given by
%
\[ (X,Y) \mapsto \int_M \langle X, Y \rangle_g\; dV. \]
%
Thus we can identify the dual spaces $\EC(\Gamma(TM))^*$ and $\DD(\Gamma(TM))^*$ with weak limits of vector fields.

On a Riemannian manifold, the natural derivative operators to consider are the \emph{gradient operator}, which, for a given function $f \in \EC(M)$, gives a smooth vector field $\nabla_g f \in \Gamma(TM)$, which has the property that for any other smooth vector field $X \in \Gamma(TM)$,
%
\[ X(f) = \langle X, \nabla_g f \rangle_g. \]
%
We also have a \emph{divergence operator}, which associates with any smooth vector field $X \in \Gamma(TM)$ a smooth function $\nabla_g \cdot X$ such that the integration by parts identity
%
\[ (X, \nabla_g f) = - ( \nabla_g \cdot X, f ) \]
%
holds. This formula gives us a way to interpret the gradient $\nabla_g u \in \DD(\Gamma(TM))^*$ of a general distribution $u \in \DD(M)^*$, by testing the gradient against a general vector field. Similarily, we can consider the divergence $\nabla_g \cdot X \in \DD(M)^*$ of a distributional vector fields $X \in \DD(\Gamma(TM))^*$. Combining these operators gives us the \emph{Laplace-Beltrami operator} $\Delta_g f = \nabla_g \cdot \nabla_g f$, which we can now consider as a map from $\DD(M)^*$ to itself, and from $\EC(M)^*$ to itself. We note that in coordinates, we have
%
\[ \nabla_g f = \sum_{i,j} \frac{\partial f}{\partial x^i} g^{ij} \frac{\partial}{\partial x^j}, \]
%
and the divergence is given by
%
\[ \nabla_g \cdot X = |g|^{-1/2} \sum_i \frac{\partial}{\partial x^i} \left\{ |g|^{1/2} X^i \right\}, \]
%
where $|g|$ is the determinant of the matrix with coefficients $\{ g_{ij} \}$, which we can roughly think of as the volume of the unit ball in the metric.


Every distribution $u$ has an associated wavefront set $\text{WF}(u)$, which forms a conic subset of $T^*M$. We can consider the pullback of a distribution along a smooth map $f: (M,g) \to (N,g')$, to find the chain rule for this pullback, we note that for any vector field $Y$ on $N$, we can calculate that for any $p \in M$,
%
\[ \langle (\nabla_g f^* \phi)_p, X_p \rangle_g = \langle (f^*\nabla_{g'} \phi)_p, f_* X_p \rangle_{g'}. \]
%
Thus
%
\[ (\nabla_g f^* \phi, X) = ( f^* (\nabla_{g'} \phi), f_* X ), \]
%
where we view the elements of the pairing on the right hand side as elements of $\Gamma(f^* (TN))$.

The formulas we have found are very advantageous when studying distributions on Riemannian manifolds, especially in the next chapter, where we construct the Hadamard parametrix.







\chapter{The Lax Parametrix}

Here we discuss a method due to Lax, which constructs a \emph{parametrix} for the equation $\partial_t - 2 \pi i T$ over small times, i.e. for $|t| \lesssim 1$, which is expressed as a \emph{Fourier integral operator}. The equation $\partial_t = 2 \pi i T$ is a pseudodifferential variant of a basic hyperbolic partial differential equation, so based on the type of parametrices one can construct in the hyperbolic setting, one might hope to find a parametrix defined by an oscillatory integral of the form
%
\[ Sf(x,t) = S(t)f(x) = \int s(t,x,y,\xi) e^{2 \pi i \Phi(t,x,y,\xi)} f(y)\; dy\; d\xi, \]
%
such that:
%
\begin{itemize}
    \item $\Phi(t,x,y,\xi) = \phi(x,y,\xi) + t p(y,\xi)$, where $\phi$ is smooth away from $\xi = 0$, homogeneous of degree one, and $\phi(x,y,\xi) \approx (x - y) \cdot \xi$ in the sense that on the support of $s$,
    %
    \[ \partial^\beta_\xi \{ \phi(x,y,\xi) - (x - y) \cdot \xi \} \lesssim_\beta |x - y|^2 |\xi|^{1 - \beta}. \]
    %
    In particular, this implies that $\phi(x,y,\xi) = 0$ when $(x - y) \cdot \xi = 0$.

    \item $s$ is a symbol of order zero, supported on $|x - y| \lesssim 1$ and on $|\xi| \geq 1$, in such a way that
    %
    \[ |\nabla_\xi \phi(x,y,\xi)| \gtrsim |x - y| \quad\text{and}\quad |\nabla_x \phi(x,y,\xi)| \gtrsim |\xi| \]
    %
    for $(x,y) \in \text{supp}_x(s) \times \text{supp}_y(s)$.
\end{itemize}
%
If $(\partial_t - 2 \pi i T) \circ S = 0$ is a smoothing operator on $(-\varepsilon,\varepsilon) \times M$ and $S(0)$ differs from the identity operator by a smoothing operator, then $S - e^{2 \pi i t T}$ is smoothing for $|t| \leq \varepsilon$. The construction of the parametrix $S$ will therefore give us much more information about the behaviour of the propogators $e^{2 \pi i t T}$ over small times.

To find a choice of $\phi$ and $s$ which gives us this parametrix, let us start by determining what properties these functions should satisfy. Let us fix a coordinate system $(x,U)$, where $x(U)$ is a precompact subset of $\RR^n$. Let us assume that in these coordinates, $T$ has symbol $a(x,\xi)$. Then the kernel of $(\partial_t - 2 \pi i T) \circ S$ in this coordinate system is
%
\[ \int (\partial_t + 2 \pi i T(x,D)) \left\{ s(t,\cdot,y,\xi) e^{2 \pi i \Phi(t,\cdot,y,\xi)} \right\}\; d\xi. \]
%
If we set
%
\[ s'(t,x,y,\xi) = e^{- 2 \pi i \Phi(t,x,y,\xi)} (\partial_t - 2 \pi i T(x,D)) \left\{ s(t,\cdot,y,\xi) e^{2 \pi i \Phi(t,\cdot,y,\xi)} \right\} \]
%
then the kernel is
%
\[ \int s'(t,x,y,\xi) e^{2 \pi i \Phi(t,x,y,\xi)}\; d\xi. \]
%
Provided that $s'$ is a symbol of order $-\infty$ for $0 < |t| \leq \varepsilon$, integration by parts shows that $(\partial_t + 2 \pi i T) \circ S$ is smoothing, and so we will try to choose $\phi$ and $s$ so as to obtain such a result.

In our discussion of pseudodifferential operators, we have already discussed an asymptotic formula for $s'$, namely, if
% Phi(t,x,y,xi) = phi(x,y,xi) - t p(y,xi)
% phi(x,y,xi) = (x - y) * xi + O(|x-y|^2 |xi|)
% nabla_x phi = xi + nabla_x O( |x - y|^2 |xi| )
%
\begin{align*}
    r_{x,y}(z) &= \nabla_x \phi(x,z,\xi) \cdot (x - z) - \{ \phi(x,y,\xi) - \phi(z,y,\xi) \}.
\end{align*}
%
then for any $N > 0$, if $a \sim \sum_{k = -\infty}^1 a_k$, where $a_k$ is homogeneous of degree $k$, and if $\xi_\phi = \nabla_x \Phi(t,x,y,\xi) = \nabla_x \phi(x,y,\xi)$,
% Symbol of (partial_t + 2 pi i T) is
% 2 pi i (tau + a(x,xi))
\begin{align*}
    s' & (t,x,y,\xi)\\
    &= \underbrace{ \left( p(y,\xi) - a(x, \xi_\phi) \right) \cdot s(t,x,y,\xi) }_{\text{symbols of order 1}}\\
    &\quad + \underbrace{\partial_t s(t,x,y,\xi)}_{\text{symbol of order $0$}}  \\
    &\quad - \sum_{1 \leq |\beta| < N} \underbrace{\frac{2 \pi i}{\beta! \cdot (2 \pi i)^\beta} \cdot \partial_\xi^\beta a(x, \xi_\phi) \partial_z^\beta \{ e^{2 \pi i r_{x,y}(z)} s(t,z,y,\xi) \} |_{z = y}}_{\text{symbols of order $1 - \lceil |\beta| / 2 \rceil$}}\\
    &\quad + R_N(t,x,y,\xi).
\end{align*}
%
where, because $|\nabla_x \Phi(t,x,y,\xi)| \gtrsim |\xi|$ on the support of $s$,
%
\[ \langle \xi \rangle^{t - \lceil N/2 \rceil} R_N \in L^\infty((-\varepsilon,\varepsilon) \times U \times U \times \RR^d). \]
%
It is simple to  establish estimates of the form
%
\[ |\partial_x^\alpha \partial_y^\beta \partial_\xi^\lambda s'(t,x,y,\xi)| \lesssim \langle \xi \rangle^{N_{\alpha \beta \lambda}}. \]
%
Thus if we can justify that $|s'(t,x,y,\xi)| \lesssim_N \langle \xi \rangle^{-N}$ for all $N > 0$, then it will follow that $s'$ is a symbol of order $-\infty$. We now determine the propoerties of the symbol $s$ and the symbol $\phi$ which will give us these estimates.

To begin with, let us specify the function $\phi$. In order to guarantee that $s'$ is a symbol of order zero, the expansion above shows that $(p(y,\xi) - p(x,\xi_\phi)) \cdot s(t,x,y,\xi)$ must be a symbol of order zero. This will be true if we can pick $\phi$ such that, on the support of $s$, and for $|\xi| \gtrsim 1$,
%
\[ p(x, \nabla_x \phi(x,y,\xi)) = p(y,\xi). \]
%
This is an example of an \emph{Eikonal equation}, e.g. an equation of the form
%
\[ q(z,\nabla_z \psi(z)) = 0 \]
%
for some function $q(z,\zeta)$. In our case, $z = (x,y,\xi)$, so $\zeta = (\zeta_x, \zeta_y,\zeta_\xi)$, and so
%
\[ q(z,\zeta) = p(x,\zeta_x) - p(y,\xi). \]
%
Let us make some further remarks we desire about our choice of function $\phi$:
%
\begin{itemize}
    \item We want $\phi$ to be homogeneous and smooth away from the origin. If we solve the equation for all $|\xi| = 1$, and then extend $\phi$ such that for $\lambda > 0$ and $|\xi| = 1$,
    %
    \[ \phi(x,y,\lambda \xi) = \lambda \phi(x,y,\xi) \psi(\lambda), \]
    %
    where $\psi$ is smooth, equal to one for $|\lambda| \geq 3/4$, and vanishing for $|\lambda| \leq 1/2$, then $\phi$ will satisfy the equation for all $|\xi| \gtrsim 1$. This means that
    %
    \[ p(x,\nabla_x \phi(x,y,\xi)) - p(y,\xi) \]
    %
    is smooth and supported on $|\xi| \lesssim 1$, which implies it is a symbol of order $-\infty$, which suffices for our construction. Thus it suffices to solve the equation for $|\xi| = 1$.

    \item Since $\phi$ is smooth away from the origin and homogeneous, the equation
    %
    \[ |\partial_\xi^\beta \left\{ \phi(x,y,\xi) - (x - y) \cdot \xi \right\}| \lesssim_\beta |x - y|^2 \langle \xi \rangle^{1 - \beta} \]
    %
    holds if, for $|\xi| = 1$, we have $\phi(x,y,\xi) = 0$ whenever $(x - y) \cdot \xi = 0$, and $\nabla_x \phi(x,y,\xi) = \xi$ whenever $x = y$. Thus we have some \emph{initial conditions} for our Eikonal equation.
\end{itemize}
%
The second condition constitutes a type of initial condition for $\phi$, since it specifies it's behaviour on a hypersurface, a kind of Cauchy condition, and thus we should expect these are close to the conditions that give unique solutions to the equation. And the following Lemma indeed shows that there is a unique function $\phi$, defined for $|x - y| \lesssim 1$ and $|\xi| = 1$ with these properties.

\begin{lemma}
    Let $Z$ be a smooth manifold, and let $q(z,\zeta)$ be a real-valued, smooth function defined locally around a point $(z_0,\zeta_0) \in T^*Z$. Let $S$ be a smooth hypersurface in $Z$ passing through $z_0$ with conormal vector $\zeta_S$ at $z_0$, such that
    %
    \[ \frac{\partial q}{\partial \zeta_S}(z_0,\zeta_0) = \lim_{t \to 0} \frac{q(z_0,\zeta_0 + t \zeta_S) - q(z_0,\zeta_0)}{t} \]
    %
    is nonzero. Suppose that $\psi$ is any smooth function defined on $S$ locally about $z_0$, such that $d \psi(z_0)$ agrees with the action of $\zeta_0$ on $T_{x_0} S$. Then there exists a unique smooth function $\phi$ defined in a neighborhood of $z_0$, which agrees with $\psi$ on $S$, satisfies the Eikonal equation $q(z,\nabla_z \phi(z)) = 0$, and has $\nabla_z \phi(z_0) = \zeta_0$.
\end{lemma}
\begin{proof}
    TODO: See Sogge, Theorem 4.1.1.
\end{proof}

In our case,
%
\[ Z = \{ (x,y,\xi) : |\xi| = 1 \}. \]
%
We have $z_0 = (x_0,x_0,\xi_0)$, $\zeta_0 = (\xi,\xi,0)$, and
%
\[ S = \{ (x,y,\xi): |\xi| = 1 \quad\text{and}\quad (x - y) \cdot \xi = 0 \}. \]
%
The conormal vector $\xi_S$ of $S$ at $z_0$ is a multiple of $(\xi_0,-\xi_0,0)$, and so by homogeneity,
%
\[ \frac{\partial q}{\partial \xi_S} = \lim_{t \to 0} \frac{p(x_0,(1 + t)\xi_0) - p(x_0,\xi_0)}{t} = p(x_0,\xi_0), \]
%
which is nonvanishing because $T$ is elliptic. If we define $\psi$ equal to zero on $S$, then $d \psi = 0$, which agrees with the action of $\zeta_0$ on $S$. Thus the theorem applies local uniqueness and existence to solutions to the Eikonal equation, and by compactness of $Z$ we can patch such solutions together to find a solution defined for all $|x - y| \lesssim 1$.

We therefore conclude that there exists a unique choice of $\phi$ such that, if $s$ has small enough support, $s'(t,x,y,\xi)$ is a symbol of order zero. Next, let us see what constraints are forced on us in order to ensure that $S(0)$ differs from the identity by a smoothing operator. The kernel of $U$ is precisely
%
\[ \int s(0,x,y,\xi) e^{2 \pi i \phi(x,y,\xi)}\; d\xi. \]
%
We now show that this operator is actually a \emph{pseudodifferential operator} of order zero, and determine it's symbol up to first order.

To do this, we write $\phi_\alpha(x,y,\xi) = (1 - \alpha) \phi(x,y,\xi) + \alpha (x - y) \cdot \xi$. Let $U_\alpha$ be the operator with kernel
%
\[ \int s(0,x,y,\xi) e^{2 \pi i \phi_\alpha(x,y,\xi)}\; d\xi. \]
%
Assume the support of $s$ is close enough to the diagonal such that
%
\[ |\nabla_\xi \phi_\alpha(x,y,\xi)| \gtrsim |x - y| \]
%
on the support of $s$. Then $\partial_\alpha^n U_t$ has kernel
%
\[ \int (2 \pi i)^n ( \phi_1 - \phi_0 )^n s(0,x,y,\xi) e^{2 \pi i \phi_\alpha(x,y,\xi)}\; d\xi. \]
%
This is an oscillatory integral defined by a symbol of order $n$. However, when $t = 1$, the fact that $\phi(x,y,\xi) \approx (x - y) \cdot \xi$, together with the formula for converting pseudodifferential operators with compound symbols into standard Kohn-Nirenberg type symbols shows that $\partial_\alpha^n U_1$ is actually a pseudodifferential operator of order $-n$. Integration by parts, similarily, shows that $\partial_\alpha^n U_t$ is defined by an oscillator integral against a symbol of order $-n$. But this means that if we define a pseudodifferential operator by the asymptotic formula
%
\[ V \sim \sum \frac{(-1)^n}{n!} \partial_\alpha^n U_1, \]
%
then $U - V$ is smoothing. Indeed, for any $n$, by Taylor's formula we have
%
\[ U = \sum_{k = 0}^{n-1} \frac{(-1)^n}{n!} \partial_\alpha^n U_1 + \frac{(-1)^n}{n!} \int_0^1 \alpha^{n-1} \partial_\alpha^n U_\alpha\; d\alpha \]
%
The integral here is an oscillatory integral defined against a symbol of order $-n$, and thus taking $n \to \infty$ verifies the claim.

It is an important remark that reversing this argument shows that \emph{any} pseudodifferential operator can be written in the form above for the particular choice of $\phi$ we have given. This is a special case of the \emph{equivalence of phase functions} theorem. This in particular guarantees that we can choose a symbol $I(x,y,\xi)$ of order zero such that $U - 1$ is smoothing if and only if $s(0,x,y,\xi) - I(x,y,\xi)$ is a symbol of order $-\infty$. The symbol $I$ can be chosen to be vanishing for $|x - y| \gtrsim 1$, since the difference will be a smoothing pseudodifferential operator.

Next, the quantity
%
\begin{align*}
    &\partial_t s(t,x,y,\xi)\\
    &\quad\quad + \sum_{k = 1}^d \partial_\xi^k a(x,\xi_\phi) \partial_x^k s(t,x,y,\xi)\\
    &\quad\quad + \left( a_0(x,\xi_\phi) + \frac{1}{2 \pi} \sum_{|\beta| = 2} \partial_\xi^\beta p(x,\xi_\phi) \partial_x^\beta \phi(x,y,\xi) \right) s(t,x,y,\xi).
\end{align*}
%
must be a symbol of order $-1$. But because the coefficients of this equation are smooth, and all derivatives are bounded, it follows from the general theory of transport equations that there exists a unique smooth, function $s_0$ defined for $|t| \leq \varepsilon$, which is a symbol of order zero, such that $s_0(0,x,y,\xi) = I(x,y,\xi)$, $s_0$ vanishes for $|x - y| \gtrsim 1$, and satisfies the transport equation
%
\begin{align*}
    &\partial_t s_0(t,x,y,\xi)\\
    &\quad\quad + \sum_{k = 1}^d \partial_\xi^k a(x,\xi_\phi) \partial_x^k s_0(t,x,y,\xi)\\
    &\quad\quad + \left( a_0(x,\xi_\phi) + \frac{1}{2 \pi} \sum_{|\beta| = 2} \partial_\xi^\beta p(x,\xi_\phi) \partial_x^\beta \phi(x,y,\xi) \right) s_0(t,x,y,\xi) = 0.
\end{align*}
%
We have thus justified that the quantity
%
\[ R_0(t,x,y,\xi) = e^{-2 \pi i \Phi(t,x,y,\xi)} (\partial_t - 2 \pi i T)(s_0(t,\cdot,y,\xi) e^{2 \pi i \Phi(t,x,y,\xi)}) \]
%
is a symbol of order $-1$. Now we come to a quirk of this parametrix, which does not occur in the study of hyperbolic partial differential equations. Since the operator $P(x,D)$ is only \emph{pseudolocal} rather than completely local, the remainder term $R_0$ is \emph{not} necessarily supported on a neighborhood of the origin. To fix this, we now successively define the terms $\{ s_k \}$ for $k < 0$, which are symbols of order $-k$, such that $s_k(0,x,y,\xi) = 0$, and
%
\[ TODO: SPECIFY REQUIRED EQUATION. \]
%
Again, solutions exist for small time periods. And this implies that $e^{-2 \pi i \Phi(t,x,y,\xi)} (\partial_t + 2 \pi i T)((s_0 + \dots + s_{-k}) e^{2 \pi i \Phi(t,x,y,\xi)})$ is a symbol of order $-k$ (TODO: Is It), and we can continue the calculation to complete the argument.

% PROBLEMS WITH EXTENDING TO ALL TIMES
%       CANNOT ASSUME |x - y| << 1
%           - SO this means that oscillatory integrals become bad.

\begin{comment}
\begin{lemma}
    Consider an operator of the form
    %
    \[ Sf(x) = \int a(x,y,\xi) e^{2 \pi i \phi(x,y,\xi)} f(y)\; dy\; d\xi, \]
    %
    where $a \in S^r$ and vanishes for $|x - y| \gtrsim 1$, $\phi \in S^1$, and is homogeneous of degree one in $\xi$, $|\nabla_\xi \phi(x,y,\xi)| \gtrsim |x - y|$ on the support of $a$, and for all $r > 0$,
    %
    \[ \partial^\beta_\xi \{ \phi(x,y,\xi) - (x - y) \cdot \xi \} \lesssim_\beta |x - y|^2 |\xi|^{1-|\beta|}. \]
    %
    Then $S$ is well defined, and is actually a pseudodifferential operator of order $r$. If $T$ is the pseudodifferential operator with symbol $(x,\xi) \mapsto a(x,x,\xi)$, then $T - S$ is a pseudodifferential operator of order $r-1$.

    Conversely, for \emph{any} $\Psi DO$ $T$ of order $r$, there exists a symbol $a$ of order $r$ such that for the resulting operator $S$ of order $r$, $S - T$ is a smoothing operator.
\end{lemma}
\begin{proof}
    Let us first define the operator $S$. Let $\phi_0(x,y,\xi) = (x - y) \cdot \xi$, $\phi_1(x,y,\xi) = \phi(x,y,\xi)$, $\phi_t = t \phi_1 + (1 - t) \phi_0$, and define $S_t$ with the phase function $\phi_t$ and symbol $a$. Note that $|\nabla_\xi \phi_t| \gtrsim |x - y|$, uniformly in $t$. This enables us to compute the kernel $K_t(x,y)$ for $0 \leq t \leq 1$. For $t = 0$ we have a pseudodifferential operator, and for $t = 1$, we get the kernel $K(x,y)$ we get to compute. It is also simple to see that, since $|\nabla_\xi \phi_t| \gtrsim |x - y|$, that for large $N$,
    %
    \[ K(x,y)| \lesssim_N \frac{1}{|x - y|^N}, \]
    %
    so we already see that $S$ is somewhat pseudolocal.

    We have
    %
    \[ \frac{\partial^N K_t(x,y)}{\partial t^N} = (2 \pi i)^N \int (\phi_1(x,y,\xi) - \phi_0(x,y,\xi))^N a(x,y,\xi) e^{2 \pi i \phi_t(x,y,\xi)}\; d\xi. \]
    %
    Now $(\phi_1 - \phi_0)^N \cdot a$ is a symbol of order $r + N$. But on the other hand, using the fact that $(\phi_1 - \phi_0)^N \lesssim |x - y|^{2N} |\xi|^N$, and thus vanishes to order $2N$ on the diagonal, then combined with the fact that $|\nabla_\xi \phi(x,y,\xi)| \gtrsim |x - y|$, we actually see via an integration by parts $2N$ times in $\xi$ that we can rewrite the integral in terms of a symbol of order $r - N$ and the same phase $\phi_t$. Applying Taylor's theorem, we write
    %
    \[ K(x,y) = K_1(x,y) = \sum_{k = 0}^{N-1} \frac{1}{k!} \left. \frac{\partial^k K_t(x,y)}{\partial t^k} \right|_{t = 1} + \frac{1}{N!} \int_0^1 t^{N-1} \frac{d^NK_t(x,y)}{dt^N}\; dt. \]
    %
    This integral gives an arbitrarily smooth kernel as $N \to \infty$. Thus if we let $T$ be a pseudodifferential operator of order $r$ such that
    %
    \[ T \sim \sum_{k = 0}^\infty \frac{1}{k!} \left. \frac{\partial^k K_t(x,y)}{\partial t^k} \right|_{t = 1}, \]
    %
    then $T - S$ is a smoothing operator. Now if $\tilde{T}$ is the pseudodifferential operator corresponding to the symbol $a(x,x,\xi)$, then $T - \tilde{T}$, and thus $S - \tilde{T}$, is a pseudodifferential operator of order $r-1$. The converse is similar, working in the opposite direction, i.e. from $t = 1$ to $t = 0$, and is left as an exercise.
\end{proof}
\end{comment}

%Since $I$ is a $\Psi DO$ of order zero, we can find a symbol $a$ of order zero such that if $T$ is the operator with kernel
%
%\[ \int a(x,y,\xi) e^{2 \pi i \phi(x,y,\xi)}\; d\xi, \]
%
%then $T - I$ is a smoothing operator. To ensure that $S(0) - I$ is a smoothing operator, it is natural to insist that $s(0,x,y,\xi) = a(x,y,\xi)$. We note in particular that since $I$ is a $\Psi DO$ with $1$ as a symbol, this implies $s(0,x,x,\xi) - 1$ is a symbol of order $-1$.








\chapter{The Hadamard Parametrix}

The Hadamard parametrix gives an alternate expression as a wave to invert the wave equation on a Riemannian manifold, though one requires much tighter control of the geometry of the underlying Riemannian manifold. We begin with summarizing facts about the fundamental solution of the standard wave equation on $\RR^{d+1}$, before moving onto fundamental solutions of the wave equation on $\RR^{d+1}$ with an arbitrary \emph{constant coefficient} Riemannian metric, and then we move to constructing a parametrix on an arbitrary Riemannian manifold.

%Before we do this, however, let's recall how we define the Laplace-Beltrami operator on a given Riemannian manifold $(M,g)$. We want to extend the formula $\Delta f = \nabla \cdot \nabla f$, which requires we interpret the gradient and divergence on an arbitrary Riemannian manifold. The gradient is easy; for any $f \in C^\infty(M)$, we can consider the covector field $df \in C^\infty(M)$, and we let the gradient $\nabla f$ be the vector field obtained from $df$ via the musical isomorphism, i.e. such that for any other vector field $X \in \Gamma(TM)$,
%
%\[ \langle \nabla f, X \rangle_g = df(X). \]
%
%If, in a given coordinate system $(x,U)$ on $M$, we write
%
%\[ g = \sum_{ij} g_{ij} dx^i dx^j \quad \nabla_g f = \sum (\nabla_g f)^i \frac{\partial}{\partial x^i} \quad\text{and}\quad X = \sum X^i \frac{\partial}{\partial x^i}, \]
%
%then we obtain that
%
%\[ \sum_{ij} g_{ij} (\nabla_g f)^i X^j = \sum \frac{\partial f}{\partial x^i} X^i, \]
%
%Since $X$ is arbitrary, we conclude that
%
%\[ \nabla_g f = \sum \frac{\partial f}{\partial x^i} g^{ij} \frac{\partial}{\partial x^j}. \]
%
%Next, to define the divergence, we hope that the integration by parts identity
%
%\[ \int_M \langle X, \nabla_g f \rangle_g\; dV_g = - \int_M (\nabla_g \cdot X) f\; dV_g \]
%
%for compactly support smooth $f$ and $X$, where $dV_g = |g|^{1/2} dx^1 \cdots dx^d$, and where $|g| = \det(g_{ij})$. If $f$ and $X$ have support on $U$, then we can work in coordinates, and we obtain the Euclidean identity
%
%\[ \int \sum_i \frac{\partial f}{\partial x^i} \left\{ |g|^{1/2} X^i \right\} dx^1 \cdots dx^d = \int - |g|^{1/2} (\nabla_g \cdot X) f\; dx^1 \cdots\; dx^d. \]
%
%But Euclidean integration by parts gives that we should set
%
%\[ \nabla_g \cdot X = |g|^{-1/2} \sum_i \frac{\partial}{\partial x^i} \left\{ |g|^{1/2} X^i \right\}. \]
%
%Combining our definition of the divergence with the gradient, we now have a Laplace-Beltrami operator
%
%\[ \Delta_g f = \nabla_g \cdot \nabla_g f = |g|^{-1/2} \sum_{i,j} \frac{\partial}{\partial x^i} \left\{ |g|^{1/2} g^{ij} \frac{\partial f}{\partial x^j} \right\}. \]
%
%We then define the wave operator $\Box_g$ on $M \times \RR$ by setting $\Box_g f = \partial_t^2 f - \Delta_g f$.

\section{Euclidean Case}

Let's begin with the standard wave equation on $\RR^d$, equipped with the standard metric. We are therefore concerned with constructing a fundamental solution for the D'Alembertian operator $\Box = \partial_t^2 - \Delta_x$ on $\RR^{d + 1}$. One choice of fundamental solution for $\Box$ is the \emph{forward fundamental solution}, defined for $d \geq 2$ by the equation
%
\[ E_+(x,t) = c_d \frac{H(t)}{\text{Im}(|x|^2 - (t + i0)^2)^{\frac{d-1}{2}}} \]
%
where $H$ is the heaviside step function, and
%
\[ c_d = \frac{2}{(n+1) A_{n+1}}, \]
%
and where $A_{n+1}$ denotes the surface area of $S^n$. We have
%
\[ \text{Supp}(E_+) = \{ (x,t): t \geq 0\ \text{and}\ |x| \leq t \}, \]
%
Furthermore, we have
%
\[ \text{Sing\ Supp}(E_+) = \{ (x,t): t \geq 0\ \text{and}\ |x| = t \}, \]
%
i.e. $E_+$ has singular support on the \emph{forward light cone}. When $d$ is odd, $E_+$ actually has \emph{support} on the forward light cone; this is \emph{Huygen's principle}. For $d = 2$, we have
%
\[ E_+(t,x) = \frac{H(t) H(t^2 - |x|^2)}{2 \pi (t^2 - |x|^2)^{1/2}}, \]
%
where the right hand side is locally integrable, and thus defines a distribution. For $d = 3$, we have
%
\[ E_+(t,x) = \frac{H(t) \delta(t^2 - |x|^2)}{2 \pi} = \frac{H(t) \delta(t - |x|)}{4 \pi t}. \]
%
When $d \geq 4$, the forward fundamental solution becomes a distribution of higher order, i.e. becoming more singular on the forward light cone. For $d = 2$ the equation above no longer applies, but we have the simpler formula
%
\[ E_+(t,x) = \frac{H(t) H(t^2 - |x|^2)}{2}. \]
%
It is interesting to note that $E_+$ is the \emph{unique} fundamental solution supported on the interior of the forward light cone.

\begin{lemma}
    If $v$ is a fundamental solution of the D'Alembertian, supported on the interior of the forward light cone, then $v = E_+$.
\end{lemma}
\begin{proof}
    If $u = v - E_+$, then $u$ is supported on the interior of the forward light cone and $\Box u = 0$. But this means that
    %
    \[ u = \delta * u = \Box E_+ * u = E_+ * \Box u = E_+ * 0 = 0, \]
    %
    where these convolutions are well defined precisely because of the support of all the quantities involved.
\end{proof}

The reflection of the forward fundamental solution about the origin $t = 0$ is another fundamental solution to the wave equation, which we denote by $E_-$. It is supported on the interior of the backward light cone, and called the \emph{backward fundamental solution}. Taking convex combinations of these fundamental solutions gives a plethora of other fundamental solutions, like the solution
%
\[ E_{\text{FW}} = \frac{E_+ + E_-}{2}, \]
%
the \emph{Feynman-Wheeler} fundamental solution.

Using the Fourier transform, we can write
%
\[ E_+(x,t) = \frac{H(t)}{2 \pi} \int \frac{\sin(2 \pi t |\xi|)}{|\xi|} e^{2 \pi i \xi \cdot x}\; d\xi. \]
%
Modifying this formula gives Fourier expressions for the backward fundamental solution, and the Feynman-Wheeler fundamental solution. We also have a solution of the form
%
\[ E_F(t,x) = \frac{1}{4 \pi i} \int e^{2 \pi i (\xi \cdot x + |t \xi|)}\; \frac{d\xi}{|\xi|}, \]
%
the \emph{Feynman fundamental solution}. This fundamental solution has support on the entirety of $\RR^{d+1}$, which is counterintuitive given the finite propogation speed of the wave equation.

Let us use these fundamental solutions to solve the Cauchy problem for the wave equation.

\begin{theorem}
    Suppose $f,g \in C^\infty(\RR^d)$, and $F \in C^\infty(\RR^d \times [0,\infty))$. Then the Cauchy problem
    %
    \[ \Box u(x,t) = F(x,t) \]
    %
    with $u(0,t) = f(x)$ and $\partial_t u(x,0) = g(x)$ has a unique solution in $C^\infty(\RR^d \times [0,\infty))$, and we can write this solution as
    %
    \[ u(t) = \partial_t E_+(t) * f + E_+(t) * g + \int_0^t E_+(t-s) * F(s)\; ds. \]
    %
    If we assume $f,g \in \mathcal{S}(\RR^d)$, and $F \in C^\infty(\RR_t, \mathcal{S}(\RR^d_x))$, then we can also write
    %
    \[ u(x,t) = \cos(2 \pi t \sqrt{-\Delta}) f + \frac{\sin(2 \pi t \sqrt{-\Delta})}{\sqrt{-\Delta}} g + \int_0^t \frac{\sin((t - s) \sqrt{-\Delta})}{\sqrt{-\Delta}} F(s)\; ds. \]
\end{theorem}
\begin{proof}
    The Fourier multiplier formula follows from the Fourier expression of $E_+$. The expression above is well defined since $\partial_t E_+(t)$ and $E_+(t)$ are smooth functions of $t$ valued in $\mathcal{E}^*(\RR^d)$. Now
    %
    \[ \Box E_+(t) = \Box \partial_t E_+(t) = 0 \]
    %
    for $t > 0$, so it is clear that $v(t) = \partial_t E_+(t) * f + E_+(t) * g$ solves the wave equation for $t > 0$. Since $(E_+)(0+) = 0$, and $(\partial_t E_+)(0+) = \delta_0$, $v(t)$ has the required initial conditions, so would solve the equation provided there were no forcing term. Thus it suffices to show that
    %
    \[ w(t) = \int_0^t E_+(t-s) * F(s)\; ds \]
    %
    solves the equation $\Box w = F$, with vanishing initial conditions. Now since $E_+(0+) = 0$,
    %
    \[ \partial_t w(t) = \int_0^t (\partial_t E_+)(t - s) * F(s)\; ds \]
    %
    and since $\partial_t E_+(0+) = \delta_0$,
    %
    \begin{align*}
        \partial_t^2 w(t) &= F(t) + \int_0^t (\partial_t^2 E_+)(t - s) * F(s)\; ds.\\
        &= F(t) + \int_0^t (\Delta E_+)(t - s) * F(s)\; ds\\
        &= F(t) + \Delta w(t).
    \end{align*}
    %
    Thus $\Box w = F$. It is clear from the above formulas that $w(0+) = \partial_t w(0+) = 0$. Thus we proved the \emph{existence} of solutions to the wave equation. Uniqueness follows from our uniqueness argument that $E_+$ is the unique fundamental solution, since it suffices to show that there is no nonzero $u \in C^\infty(\RR^d \times [0,\infty))$ with $\Box u = 0$ and with vanishing initial conditions.
\end{proof}

In order to construct fundamental solutions to the wave equation on a Riemannian manifold, it is helpful to note that we can find constants $\{ a_\nu : \nu \geq 1 \}$ such that if
%
\[ E_\nu(x,t) = a_\nu H(t) \text{Im}(|x|^2 - (t + i0)^2)^{\nu - \frac{d-1}{2}} \] 
%
then $\Box E_\nu = \nu E_{\nu - 1}$, where $E_0 = E_+$. As $\nu \to \infty$, these distributions become less and less singular on the forward light cone. We also have a Fourier representation formula; for each $\nu$, $E_\nu$ is a finite linear combination of terms of the form
%
\[ t^j H(t) \int \frac{e^{2 \pi i (x \cdot \xi + t |\xi|)}}{|\xi|^{\nu + k + 1}}\; d\xi \]
%
where $j + k = \nu$. In particular, we see from this formula that $E_\nu$ is a Fourier integral of order at most $d/4 - \nu - 1$.

\section{Constant Coefficient Metric}

Now let's move on to the case of the wave equation on $\RR^{d+1}$, where $\RR^d$ is equipped with a different metric $g = \sum g_{ij} dx^i dx^j$, although one with constant coefficients. The Laplace-Beltrami operator then becomes
%
\[ \Delta_g f = \sum_{i,j} g^{ij} \partial_{i,j} f = \nabla_x \cdot \nabla_g f. \]
%
Let us put the coefficients of the metric in a positive definite $d \times d$ matrix $G$, i.e. with $G_{ij} = g_{ij}$.

To find a fundamental solution here, it is easiest to try and find a change of coordinates where the Laplace-Beltrami operator behaves like a usual Laplace operator. So suppose that $f = h \circ T$, for some invertible linear operator $T: \RR^d \to \RR^d$ with coefficients $T_{ij}$, and some smooth function $h$. The chain rule implies that
%
\[ \nabla_g f = (G^{-1} T^t) (\nabla_x h \circ T) \]
%
and thus
%
\[ \Delta_g f = \nabla_x \cdot \nabla_g f = \sum_{i,j} (TG^{-1}T^t)_{i,j} (\partial_{i,j} h \circ T). \]
%
If we choose $T$ such that $TG^{-1}T^t$ is the identity matrix, then we obtain that
% G^{-1} = T^t T
\[ \Delta_g f = \Delta h \circ T, \]
%
This happens precisely when $G = T^t T$. Abusing notation slightly for functions $u$ on $\RR^{d+1}$, with $u = v \circ T$, we have
%
\[ \Box_g u = \Box v \circ T. \]
%
Taking weak limits allows us to find a fundamental solution. In particular, taking weak limits tells us that if $E$ is the forward fundamental solution to $\Box$ on $\RR^d$, then $E_+ = T^* E$ is the forward fundamental solution to the wave equation $\Box_g$. We have
%
\begin{align*}
    E_+(x,t) &= E(Tx, t)\\
    &= c_d \cdot \frac{H(t)}{\text{Im}(|Tx|^2 - (t + i0)^2)^{\frac{d-1}{2}}}\\
    &= c_d \cdot \frac{H(t)}{\text{Im}(|x|_g^2 - (t + i0)^2)^{\frac{d-1}{2}}},
\end{align*}
%
Note that we do not need to introduce an extra integrating factor from changing coordinates because we are working with distributions tested against the \emph{volume measure} $dV_g = |g|^{1/2} dx$ on $\RR^d$, i.e. for $\phi \in \DD(\RR^d)$, the fundamental solution we have constructed satisfies the equation
%   
\[ \int \int E_+(x,t) \Box_g(x,t) \phi\; dV(x)\; dt = \phi(0,0). \]
%
%
%To determine the behaviour of the Laplace-Beltrami operator, let's see how the metric changes under a \emph{linear change of coordinates}. Fix an invertible linear map $T: \RR^d \to \RR^d$. Then we can consider a new metric $T_* g$ given by
%
%\[ \langle v, w \rangle_{T_* g} = \langle T^{-1} v, T^{-1} w \rangle_g. \]
%
%But then
%
%\[ \langle v, w \rangle_{T_* g} = (T^{-1} v)^t G (T^{-1} w) = v^t (T^{-t} G T^{-1}) w. \]
%
%If $G = T^t T$, then we have $\langle v, w \rangle_{T_* g} = v^t w$, i.e. $T_* g$ is the standard Euclidean metric. This is true, for instance, if $T = G^{1/2}$, or more generally, if we consider the \emph{Cholesky decomposition} of $G$, where $T$ is lower triangle with strictly positive entries on the diagonal.
%
%For any differential operator $L$ defined on functions on $\RR^d$, define $(T_* L) f = L(f \circ T) \circ T^{-1}$. The chain rule implies that
% T^{-1}_k = sum T^{-1}_{kn} x_n
%\[ (T_* \Delta_g) f = \sum_{k,l} (T G^{-1} T^t)_{kl} \left( \frac{\partial^2 f}{\partial x^k \partial x^l} \right) = \Delta f, \]
%
%i.e. $T_* \Delta_g = \Delta$. But this means that $\Delta_g(f \circ T) = T^*(\Delta f)$. More generally, for functions $u$ on $\RR^{d + 1}$, if $\Box_g = \partial_t^2 - \Delta_g$, then
%
%\[ \Box_g(u \circ T) = T^* \Box u. \]
%
%Thus in particular, if $E_{+,0}$ is the forward fundamental solution to the wave equation in the standard metric, then
%
%\[ \Box_g(E_{+,0} \circ T) = T^* \delta = |g|^{-1/2} \delta. \]
%
%Thus the \emph{forward fundamental solution} to $\Delta_g$ is
%
%\[ E_+(x,t) = \frac{c_d H(t)}{\text{Im}(|x|_g^2 - (t + i0)^2)^{\frac{d-1}{2}}}, \]
%
%in the sense that for any test function $\phi \in \DD(\RR^{d+1})$,
%
%\[ \int_{\RR^{d+1}} E_+ \cdot \Box_g \phi\; dV_g = \phi(0,0), \]
%
%where we note we are now integrating with respect to the \emph{volume form} $dV_g = |g|^{1/2} dx$ rather than the Lebesgue measure.
Analogous to the behaviour of the forward fundamental solution with respect to the standard metric, the fundamental solution here has support on the interior of the light cone, i.e. the set
%
\[ \{ (x,t) \in \RR^{d+1}: t \geq 0\ \text{and}\ |x|_g \leq t \}. \]
%
We also have a Fourier transform representation, i.e. as
%
\[ E_+(x,t) = \frac{H(t)}{2 \pi} \int \frac{\sin(2 \pi t |\xi|_g)}{|\xi|_g} e^{2 \pi i \xi \cdot x}\; d\xi \]
%
where $|\xi|_g$ is given by the metric on the \emph{cotangent} space, i.e. the metric given with the matrix of coefficients given by $G^{-1}$ rather than $G$ itself.
% xi' = T^{-t} xi
% dxi' = det(T^{-t}) dxi
%i.e. if $y = Mx$ for some invertible $d \times d$ matrix $M$. Then if
%
%\[ \sum a^i \frac{\partial}{\partial x^i} = \sum b^i \frac{\partial}{\partial y^i}, \]
% y^i = sum M^i_j x^j
%then $a = M^{-1}b$. But this implies that if
%
%\[ X = \sum v^i \frac{\partial}{\partial y^i} \quad\text{and}\quad Y = \sum w^i \frac{\partial}{\partial y^i}, \]
%
%then
%
%\[ \langle X, Y \rangle_g = (M^{-1} v)^T G (M^{-1} w) = v^T (M^{-T} G M^{-1}) w. \]
%
%In particular, if we pick $M = G^{1/2}$, then $M^{-T} G M^{-1} = G^{-1/2} G G^{-1/2}$ is the identity matrix, and so the metric can be identified with the standard Euclidean metric in the coordinates $y$.
%This coordinate formula allows us to find fundamental solutions for the wave equation $\Box_g = \partial_t^2 - \Delta_g$. The calculations above imply that the forward fundamental solution on $\RR^d$ obtained by integrating in the coordinates $y$ is precisely the forward fundamental solution for the standard wave equation. The change of variables formula implies that if $y = Mx$ with $M = G^{1/2}$ as above, and if $f(y,t) = g(x,t)$, then
%
%\begin{align*}
%    \int E_+(y,t) f(y,t)\; dy &= |g|^{1/2} \int E_+(Mx,t) g(x,t)\; dx.
%\end{align*}
%
%Thus we find that, integrating against the $x$ coordinates, the fundamental solution is
%
%\[ E_+(x,t) = |g|^{1/2} E_+(G^{1/2} x, t) = \frac{c_d H(t)}{\text{Im}(|x|_g^2 - (t + i0)^2)^{\frac{d-1}{2}}}. \]
%
%Other quantities we calculated above can be converted to these coordinates. In particular, we see that the forward fundamental solution is supported on the interior of the forward light cone with respect to the metric $g$, i.e. the set $\{ (x,t): t > 0\ \text{and}\ |x|_g \leq t \}$.
%
%Again, similar to the standard metric, we can construct distributions $E_\nu$
%Similarily, we can construct distributions $E_\nu$ with $E_0 = E_+$, and with $\Box_g E_\nu = \nu E_{\nu - 1}$, just by pulling back the same distributions in the Euclidean case. The distribution $E_\nu$ will be a finite linear combination of terms of the form
%
%\[ t^j H(t) \int e^{2 \pi i (x \cdot \xi + t |\xi|_g)} |\xi|_g^{-\nu - 1 - k} \]
%
%where $j + k = \nu$.

\section{General Riemannian Metric}

Let us now address the wave equation on a general Riemannian manifold. Because we have a non constant coefficient equation, we cannot expect to obtain a fundamental solution, instead only a \emph{parametrix}, which gives a solution \emph{modulo smoothing terms}.

To begin with, it will be helpful to work in \emph{normal coordinates}. These are a natural set of coordinates that can be chosen centered at any point $p_0$ on a Riemannian manifold $(M^d,g)$. To obtain this coordinate system, we take \emph{geodesics} emerging from $p_0$. More precisely, if $B$ is a suitably small enough open ball centered at the origin in $T_{p_0}^* M$, then the geodesic exponential map $\exp: B \to M$ is a diffeomorphism, with inverse thus giving a coordinate system $x: U \to B$. For $a \in B$, let $G(a)$ be the $d \times d$ positive definite matrix giving the coefficients of the metric $g$ at the point $x^{-1}(a)$. Since we are working in normal coordinates, the matrix satisfies the identity $G(a) a = G(0) a$ for $a \in B$, which roughly means that the metric behaves like the Euclidean metric as long as we are measuring directions radiating outward from the origin.

In general, we can break the Laplace-Beltrami operator into two terms, i.e. writing
%
\[ \Delta_g = \nabla_x \cdot \nabla_g f + a \cdot \nabla_x f = L_g f + R_g f. \]
%
The first term is a second order differential operator in $f$ with `constant-coefficients' (once we switch to taking gradients with respect to the metric rather than the coordinate system), and the second term is a first order differential equation with a non-constant coefficient vector
%
\[ a = |g|^{-1/2} \nabla_g \{ |g|^{1/2} \}. \]
%
This second term proves to be a problem with constructing a fundamental solution due to the non-constant coefficients. If we could make them disappear, then we claim that we could construct a fundamental solution to the wave equation for small times. Indeed, \emph{suppose we are working in normal coordinates}, and we let
%
\[ E_+(x,t) = c_d \frac{H(t)}{\text{Im} \Big(|x|^2_{g(0)} - (t + i0)^2 \Big)^{\frac{d-1}{2}}}, \]
%
where $g(0)$ is the constant-coefficient metric on $B$ agreeing with $g$ at the origin. Then we have seen that $E$ is a fundamental solution to $\Box_{g(0)}$ with respect to the volume form $dV_{g(0)}$, i.e. so that we have
%
\[ \Box_{g(0)} E = \delta. \]
%
We claim that for sufficiently small times $|t| \lesssim 1$, we also have that, with respect to the volume form $dV_{g(0)}$,
%
\[ (\partial_t^2 - L_g) E_+(x,t) = \delta. \]
%
Thus with respect to the volume form $dV_g$, we also have
%
\[ (\partial_t^2 - L_g) E_+(x,t) = \delta \]
%
Disregarding the operator $R_g$, we have essentially constructed a fundamental solution to $\Box_g$ for small times.

To check this claim, let us perform some calculations with a function of the form $f(x) = F(|x|^2_{g(0)})$, where $F \in C^\infty(\RR)$. We claim that
%
\[ \Delta_{g(0)} f = L_g f. \]
%
Taking weak limits of this equation gives the required claim above. This follows from the more general equation that
%
\[ \nabla_{g(0)} f = \nabla_g f, \]
%
Since then
%
\[ \Delta_{g(0)} f = \nabla_x \cdot \nabla_{g(0)} f = \nabla_x \cdot \nabla_g f = L_g f. \]
%
The reason this equation holds is precisely because $F$ only depends on the \emph{radial direction} of the point $x$, which has nice properties in normal coordinates.

\begin{lemma}
    If we are working in a normal coordinate system, then
    %
    \[ \nabla_{g(0)} f = \nabla_g f. \]
\end{lemma}
\begin{proof}
    We calculate using the chain rule that
    % sum g_{ij} x^i x^j
    \begin{align*}
        \nabla_{g(0)} f &= G(0)^{-1} (\nabla_x f)\\
        &= 2 F'(|x|^2_{g(0)}) x
    \end{align*}
    %
    and (now using the normal coordinate equation)
    %
    \begin{align*}
        \nabla_g f(x) &= G(x)^{-1} (\nabla_x f)\\
        &= 2 F'(|x|^2_{g(0)}) [G(x)^{-1} G(0)] x\\
        &= 2 F'(|x|^2_{g(0)}) [G(x)^{-1} G(x)] x\\
        &= 2 F'(|x|^2_{g(0)}) x.
    \end{align*}
    %
    Comparing calculations shows we have equality.
\end{proof}

More generally, our calculations imply that if we define
%
\[ E_\nu(x,t) = c_d H(t) \text{Im} \Big(|x|^2_{g(0)} - (t + i0)^2 \Big)^{\nu - \frac{d-1}{2}}, \]
%
then $(\partial_t^2 - L_g) E_0 = \delta$, and $(\partial_t^2 - L_g) E_\nu = \nu E_{\nu-1}$. We also have that for $\nu \geq 1$,
%
\[ (\partial_t^2 - L_g) E_\nu = ( (-1/2) E_{\nu-1} ) x. \]
%
We have now essentially performed the calculations to construct a parametrix to $\Delta_g$.

%Indeed, we claim that, \emph{if we are working in normal coordinates}, and we let
%
%\[ E_0(x,t) = \frac{c_d H(t)}{\text{Im}(|x|^2_{g(0)} - (t + i0)^2)^{\frac{d-1}{2}}}, \]
%
%then for small times $t$,
%
%\[ (\partial_t^2 - L_g) E(x,t) = |g(0)|^{-1/2} \delta, \]
%
%where $g(0$ is the metric at the origin. The restriction of times is required in particular so that, by finite propogation speed, $\text{supp}_x(E)$ is compactly contained in $B$.

%To verify this identity, fix a function $F \in C^\infty(\RR)$, and we define
%
%\[ f(x) = F( |x|_{g(0)}^2 ), \]
%
%then $(L_{g(0)} f)(x) = (L_g f)(x)$ for $x$ in a suitably small neighborhood of the origin. This follows from showing that if we define the differential operator $A_g = \sum_i g^{ij} \partial_j$, then

%But we calculate that
%
%\begin{align*}
%    L_{g(0)} f &= \sum g(0)^{jk} \frac{\partial^2 f}{\partial x^j \partial x^k}\\
%    &= F'( |x|^2_{g(0)} ) \sum g(0)^{jk} \frac{\partial^2 |x|^2_{g(0)}}{\partial x^j \partial x^k}\\
%    &= F'( |x|^2_{g(0)} ) \sum g(0)^{jk} (2 g(0)_{jk})
%\end{align*}
%TODO: Check this calculation, e.g. in Sogge Chapter 2. More generally, if we define
%
%\[ E_\nu(x,t) = |g(0)|^{1/2} a_\nu H(t) \text{Im}(|x|^2 - (t + i0)^2)^{\nu - \frac{d-1}{2}}, \]
%
%then $(\partial_t^2 - L_g) E_\nu = \nu E_{\nu-1}$. We will use these higher order terms to correct for the $R_g$ terms we are missing.

Suppose $\alpha_0 \in C^\infty(B)$. Suppose $E_0(x,t) = F_0(|x|_g^2,t)$. Then we can expand out $\Box_g$ to yield that
%
\begin{align*}
    \Box_g \{ \alpha_0 E_0 \} &= (\partial_t^2 - L_g) \{ \alpha_0 E_0 \} - R_g \{ \alpha_0 E_0 \}\\
    &= \alpha_0(0) \delta - \nabla_x \cdot \{ (\nabla_g \alpha_0) E_0 \} - \langle \nabla_g \alpha_0, \nabla_g E_0 \rangle_g - R_g \{ \alpha_0 E_0 \}\\
    &= \alpha_0(0) \delta - \Delta_g \alpha_0 \cdot E_0 - 2 \langle \nabla_g \alpha_0, \nabla_g E_0 \rangle_g - R_g \{ \alpha_0 E_0 \}\\
    &= \alpha_0(0) \delta - \langle \alpha_0 a + 2 \nabla_g \alpha_0, \nabla_g E_0 \rangle_g - (\Delta_g \alpha_0 + a \cdot \nabla_x \alpha_0) \cdot E_0.
\end{align*}
%
We write
%
\begin{align*}
    \langle \alpha_0 a + 2 \nabla_g \alpha_0, \nabla_g E_0 \rangle_g &= 2 F_0'(|x|_g^2,t) \langle \alpha_0 a + 2 \nabla_g \alpha_0, x \rangle_g\\
    &= 2 F_0'(|x|_g^2,t) \left( \langle a, x \rangle_g \cdot \alpha_0 + 2 \langle \nabla_g \alpha_0, x \rangle \right)\\
    &= 2 F_0'(|x|_g^2,t) \left( \langle a, x \rangle_{g(0)} \cdot \alpha_0 + 2 x \cdot \nabla_x \alpha_0 \right)\\
    &= 2 F_0'(|x|_g^2,t) \left( \rho \cdot \alpha_0 + 2 \nabla_x \alpha_0 \cdot x \right).
\end{align*}
%
If we choose $\alpha_0$ such that $\alpha_0(0) = 1$, and $\rho \alpha_0 + 2 \nabla_x \alpha_0 \cdot x = 0$, then we therefore conclude that
%
\[ \Box_g \{ \alpha_0 E_0 \} = \delta - (\Delta_g \alpha_0 + a \cdot \nabla_x \alpha_0) \cdot E_0 = \delta - c_0 E_0. \] 
%
Thus we have a fundamental solution with a remainder $c_0 E_0$. We deal with this additional term using the fact that $\Box_g \{ E_1 \} = E_0$. We choose a smooth function $\alpha_1$ so that
%
\[ \Box_g \{ \alpha_0 E_0 + \alpha_1 E_1 \} = \delta - c_1 E_1, \]
%
and we consider this process, noting that, locally in time, the $\{ E_\nu \}$ become more and more regular as $\nu$ increases. We will be able to choose $\{ \alpha_\nu \}$ for all $\nu$ such that
%
\[ \Box_g \{ \alpha_0 E_0 + \alpha_1 E_1 + \dots + \alpha_n E_n \} = \delta - c_{n+1} E_{n+1}, \]
%
and thus we obtain an arbitrarily good approximation to a fundamental solution. One obtains a similar equation for $\Box_g \{ \alpha_\nu E_\nu \}$ to the one we calculated above for all $\nu$ (see Sogge, Chapter 2), and moreover, we have
%
\[ \alpha_0(x) = |g(x)|^{-1/4} |g(0)|^{1/4} \]
%
and for $\nu \geq 1$,
%
\[ \alpha_\nu(x) = \alpha_0(x) \int_0^1 t^{\nu-1} \frac{\Delta_g \alpha_{\nu-1}(tx)}{\alpha_0(tx)}\; dt. \]
%
By induction on $\nu$, all these equations are smooth.

Since all the quantities here depend smoothly on the metric $g$, and the metric smoothly varies as we vary the point $p_0$ we started with, we can perform this construction in a smoothly varying way for all points $p_0$ on the manifold $M$. If we let $x$ and $y$ vary over a compact set $K \subset M$, we thus obtain a function
%
\begin{align*}
    K_n(x,t;y) = \sum_{\nu = 0}^n \alpha_\nu(x,y) E_\nu( d_g(x,y), t ).
\end{align*}
%
which is well defined for $|t| \lesssim_K 1$. We then have
%
\[ \Box_g K_n(x,y,t) = \delta - (\Delta_g \alpha_{n+1}) E_{n+1}(d_g(x,y), t) = \delta - c_{n+1} E_{n+1}(d_g(x,y), t). \]
%
TODO: How do we get a solution operator from this parametrix. These operators should be FIOs of order $-1/4$, and for a fixed time, operators of order $0$. The remainder should be an FIO of order $-1/4-n-1$, and for a fixed time, of order $-n-1$.
%Thus this operator becomes more and more regular as $n \to \infty$. It follows that if we define the operators
%
%\[ S_n f(x,t) = \int K_n(x,y,t) f(y)\; dV(y), \]
%
%then for $0 < |t| \lesssim 1$,
%
%\[ \Box_g (S_n f)(x,t) = - \int c_{n+1} E_{n+1}(d_g(x,y), t) f(y)\; dV_g(y), \]
%
%and
%
%\[ \Box_g (S_n f)(x,0+) = f(x). \]
%
%Now it is important to note that the errors $\{ c_n E_n \}$ do not converge pointwise to zero as $n \to \infty$, and so we cannot take an infinite series here to get a complete solution operator. TODO: How to do we get a


But from the perspective of regularity, a finite expansion is normally enough. We can express $E_n(d_g(x,y),t)$ as a finite linear combination of terms of the form
%
\[ t^j H(t) \int \frac{e^{2 \pi i (d_g(x,y) \cdot \xi + t |\xi|)}}{|\xi|^{\nu + k + 1}}\; d\xi. \]
% d = 2n
% N = n
% Symbol of order s = -nu - 1
% -nu-1 = m
In particular, this implies that for each $n$, $S_n$ is a Fourier integral operator of order $-5/4$, and if we fix the time, we get a Fourier integral operator of order $-1$.
%
\[ \frac{\sin(t |\xi|)}{|\xi|} \]

Assuming we have localized the support of $f$ to a compact set $K$ contained in a particular coordinate system $(y,V)$, and if $F(x,\xi)$ denotes the Fourier transform of $y \mapsto c_n(x,y) f(y) |g(y)|^{1/2}$, then
%
\begin{align*}
    \int c_n(x,y) E_n(d_g(x,y), t) f(y)\; dV_g(y)
\end{align*}
%
is a linear combination of terms of the form
%
\[ t^j H(t) \int \frac{F(x,\xi)}{|\xi|^{\nu + k + 1}} e^{2 \pi i (d_g(x,y) \cdot \xi + t |\xi|)}\; d\xi \]
%
Now we have
%
\[ \| F |\xi|^s \|_{L^2(\RR^d)} \lesssim \| f \|_{\dot{H}^s(\RR^d)} \]

Thus, modulo an operator that becomes more and more regular as $n \to \infty$, locally in $x$ and $t$, we have found operators which approximate solution operators to the wave equation.

TODO: USE HADAMARD PARAMETRIX TO SHOW WAVE EQUATION HAS FINITE PROPOGATION SPEED. SOGGE Chapter 2.







\section{Explicit Hadamard Parametrix For The Sphere}

TODO

Consider geodesic normal coordinates for the sphere in $\RR^{d+1}_x$, centered at the south pole. Without loss of generality, to compute this metric we may assume our sphere is the locus given by the equation
%
\[ (y_0 - 1)^2 + y_1^2 + \dots + y_d^2 = 1 \]
%
and that the south pole is the origin. The metric in these coordinates is precisely the restriction of the metric $\sum dy_i^2$ on $T \RR^{d+1}$ to $S^d$. To work out the geodesic normal coordinates $x = (x_1,\dots,x_d)$ centered at the south pole. Then
%
\[ y_0 = 1 - \cos(r) \quad\text{and}\quad (y_1,\dots,y_d) = \sin(r) \cdot \frac{x}{|x|} \]
% 1 - cos^2(r)
% gamma(t) = ( t, (2t - t^2)^{1/2} )
% gamma'(t) = (1 , (1 - t) / (2t - t^2)^{1/2} )
% |gamma'(t)|^2 = 1 / (2t - t^2)
% int_0^T dt / sqrt(2t - t^2)
% int_{-1}^{T-1} du / sqrt(1 - u^2)
% u = sin(x)
% du = cos(x) dx
% int_{-pi/2}^{sin^{-1}(T-1)} dx
% pi/2 + sin^{-1}(T - 1)
% y_0 = 1 + sin(r - pi/2)
% y_0 = 1 - cos(r)

% Let u^2 = 2t - t^2
% When t = 1, u = 1
% Project from point (1,2)
% u = ax + (2 - a)

% int_0^a [- 2] da
% y^2 = 2x - x^2
% a^2x^2 = 2x - x^2
% (1 + a^2) x = 2
% x = 2/(1 + a^2)
% y = ax
% dx = -4a/(1 + a^2)
It will be simpler to work in polar coordinates $r$ and $\theta$, where $\theta$ is a unit vector in $\RR^d$. Then
%
\[ y_0 = 1 - \cos(r) \quad\text{and}\quad (y_1,\dots,y_d) = \sin(r) \theta. \]
%
Thus
%
\[ dy_0 = \sin(r)\; dr \]
%
and for $1 \leq i \leq d$,
%
\[ dy_i = \cos(r) \theta_i\; dr + \sin(r) d\theta_i. \]
%
%Now since $r^2 = \sum x_i^2$
%
%\[ dr = \sum (x_i/r) dx_i \]
%
%and $\theta_i = x_i / |x|$, so
%
%\[ d\theta_i = \frac{dx_i}{|x|} - \frac{x_i}{|x|^3} (x \cdot dx) \]
%
Thus $g^2 = dr^2 + \sin(r)^2 \sum_i (d\theta_i)^2$. The Laplace-Beltrami operator in these coordinates is thus
% |det(g)| = sin(r)^{d-1}
% 
\[ \Delta_g = \frac{1}{\sin(r)^{d-2}} \frac{\partial}{\partial r} \left\{ \sin(r)^{d-2} \frac{\partial}{\partial r} \right\} + \frac{1}{\sin r)^2} \Delta_\theta. \]
%
And we have
%
\[ |g| = \sin(r)^{2(d-1)} \]
%
Thus, using the terminology of Sogge, Chapter 2,
%
\[ a^r(x) = - \frac{(d-1)}{\sin(r)} \]
%
\[ \rho = - (d-1) \frac{r}{\sin(r)} \]
% TODO FINISH THIS CALCULATION


%
% If d = 1, alpha_0 = 1.
% If d = 2, alpha_0 = 1 / sin(x)^{1/2}
% alpha_0' = -cos(x)/2sin(x)^{3/2} = - cot(x) / 2 sin(x)^{1/2}

%
\[ \alpha_0(r) = \frac{c}{\sin(x)^{\frac{d-1}{2}}} \]
\[ - \frac{(d-1)}{2} \cot(r) \alpha_0 = \partial_r \alpha_0 \]

The highest order term in the Hadamard Parametrix for the Laplace-Beltrami operator is
%
\[ E_0(t,x) = E_+(t,r) \]
%


\[ g_{rr} = 1 \]
\[ g_{\theta_i \theta_i} = \sin(r)^2 \]
\[ g^{rr} = 1 \]
\[ g^{\theta_i \theta_i} = 1/\sin(r)^2 \]


%
%\begin{align*}
%    g^2 &= dr^2 + \sin(r)^2 \sum_i (d\theta_i)^2\\
%    &= \sum_{i,j} \frac{x_i x_j}{r^2}
%\end{align*}
%\[ g^2 = dr^2 + \sin(r)^2 \sum_i (d\theta_i)^2. \]

%
\[ g^2 = \Big( \cos^2(r) ( 4d (\sin(r) - 1)^2 + 1 ) \Big)\; dr^2 + \dots \]

\[ g^2 = \left( 2 \cos(r) \cos \left( \frac{r}{2} + \frac{\pi}{4} \right) \right)^2\; dr^2 + \dots \]

% 4 x^2 - 8x + 5
% 4 (x - 1)^2 + 1
and for $1 \leq i \leq d$,
% sin|x| (2 - sin|x|) x_i / |x|
\[ dy_i = \left( 2 |x| \cos |x| (1 - \sin |x|) - \sin |x| (2 - \sin |x|) \right) \frac{x_i(x \cdot dx)}{|x|^3} 2 \cos |x| (1 - \sin |x|) \frac{x_i (x \cdot dx)}{|x|^2} - \frac{\sin |x| (2 - \sin |x|) x_i (x \cdot dx)}{|x|^3} \]

it will help to work with polar normal coordinates $x = r \cdot \theta$, where $r > 0$ and $\theta \in S^{d-1}$. Then $y_0 = \sin(r)$ and $(y_1,\dots,y_d) = \sin(r) (2 - \sin(r)) \theta$

% sin^2(r) - 2 sin(r) + |y'|^2 = 2 sin(r) - sin^2(r)

% gamma(t) = (t, sqrt(1 - t^2))
%\[ \int_0^a \frac{dt}{(1 - t^2)^{1/2}} \]
%\[ \int_0^{sin^{-1}(a)} d\theta \]
% sin^{-1}(a)

 point at a height $y_0 = a$ is given by the curve $\gamma(t) = (a, )$
%
\[ \int_0^t  \]

Consider the geodesic normal coordinates $y = (y_1, \dots, y_d)$ centered at the south pole.


What is the metric $G = \{ g \}$ in these coordinates. In the normal $(x,y,z)$ coordinates, the metric
















\chapter{Notes on Bochner-Riesz}

The goal of this section is to compare and contrast approaches to understanding the Bochner-Riesz conjecture on Euclidean space and on compact Riemannian manifolds, in order to reflect on the differences in understanding multipliers on $\RR^d$ vs on a compact manifold $X$ before we attack the more general multiplier problem in this setting. We define the Riesz multipliers via symbols $r_\rho^\delta: [0,\infty] \to [0,\infty)$, defined for $\rho > 0$ and a real number $\delta$ by setting, for $\tau > 0$,
%
\[ r_\rho^\delta(\tau) = (1 - \tau / \rho)_+^\delta. \]
%
Here $s_+ = \max(s,0)$. The resulting radial multipliers on $\RR^n$, and on a compact Riemannian manifold $X$, will be denoted by
%
\[ R_\rho^\delta = r_\rho^\delta \left( \sqrt{-\Delta}\ \right). \]
%
The goal of the Bochner-Riesz conjecture is to determine bounds on the operators $\{ R_\rho^\delta \}$ invariant under dilation of the symbol.

\section{Euclidean Case}

Let's review a reduction of Bochner-Riesz to Tomas Stein:
%
\begin{itemize}
    \item First, we can \emph{rescale the problem}. If $r^\delta = r^\delta_1$, then
    %
    \[ r_\rho^\delta(\lambda) = r^\delta(\lambda / \rho). \]
    %
    Thus if $R^\delta = R^\delta_1$, then $R^\delta_\rho = R^\delta \circ \text{Dil}_{1/\rho}$, and so the operators $\{ R^\delta_\rho \}$ are uniformly bounded from $L^p$ to $L^p$ for all $\rho$ if and only if $R^\delta$ is bounded from $L^p$ to $L^p$.

    \item We now perform a \emph{spatial decomposition}. Let $k^\delta$ be the convolution kernel corresponding to the operator $R^\delta$. We break up the effects of the operator spatially into dyadic annuli, i.e. writing
    %
    \[ k^\delta(x) = \sum_{j = 0}^\infty k^\delta_j(2^j x), \]
    %
    where $k^\delta_0$ is supported on $|x| \leq 2$, and all of the other kernels $k^\delta_j$ are supported on the annuli $\{ 1/2 \leq |x| \leq 1 \}$, and can be written as
    %
    \[ k^\delta_j(x) = \phi \cdot \text{Dil}_{1/2^j} k^\delta \]
    %
    for some $\phi \in C_c^\infty$ supported on the annulus $\{ 1/2 \leq |x| \leq 2 \}$ and equal to one on the annulus $\{ 3/4 \leq |x| \leq 3/2 \}$. We analyze each of the convolution kernels separately and then collect up each of the bounds we obtain by applying the triangle inequality. Thus we let $R^\delta_j$ be the operator with convolution kernel $k^\delta_j$. Provided we can obtain a bound of the form
    %
    \[ \| R^\delta_j f \|_{L^p(\RR^d)} \lesssim 2^{- \varepsilon j} \| f \|_{L^p(\RR^d)} \]
    %
    for some $\varepsilon > 0$, and some implicit constant uniform in $j$, we can sum up the bounds using the triangle inequality to bound $R^\delta$.

    \item Spatial localization means that the operators $\{ R^\delta_j \}$ are \emph{local}, i.e. for any function $f$, the support of $R^\delta_j f$ is contained in a $O(1)$ neighborhood of the support of $f$. A decomposition argument, thus implies that it suffices to obtain a bound of the form
    %
    \[ \| R^\delta_j f \|_{L^p(\RR^d)} \lesssim 2^{- \varepsilon j} \| f \|_{L^p(\RR^d)} \]
    %
    for functions $f$ \emph{supported on balls of radius $1$}, since the general bound will follow from this.

    \item We \emph{reduce to $L^2$ bounds}: Now that $f$ is suported on a ball of radius 1, $R^\delta_j$ is supported on a ball of radius $O(1)$, and so for $p \leq 2$ we have
    %
    \[ \| R^\delta_j f \|_{L^p(\RR^d)} \lesssim \| R^\delta_j f \|_{L^2(\RR^d)}. \]
    %
    Thus it suffices to obtain a bound of the form $\| R^\delta_j f \|_{L^2(\RR^d)} \lesssim \| f \|_{L^p(\RR^d)}$. Switching from the $L^p$ norm to the $L^2$ norm is the most inefficient part of the proof, but it enables us to apply more powerful tools which we only have in $L^2(\RR^d)$. Getting around this reduction is key to improving the currently known Bochner-Riesz bounds.

    \item We reduce the problem to Tomas-Stein. Since we are now in $L^2(\RR^d)$, we can apply Plancherel. If $\psi^\delta_j$ is the Fourier transform of $k^\delta_j$, then we obtain that
    %
    \[ \| R^\delta_j f \|_{L^2(\RR^d)} = \| \psi^\delta_j \cdot \widehat{f} \|_{L^2(\RR^d)}. \]
    %
    A stationary phase calculation shows that $\psi^\delta_j$ has the majority of its mass on an annulus of radius $2^j$ and width $O(1)$, and has magnitude $O(2^{-j\delta})$ there, i.e.
    %
    \[ |\psi^\delta_j(\xi)| \lesssim_N 2^{-\delta j} \langle 2^j - |\xi| \rangle^{-N}. \]
    %
    Thus by Tomas-Stein, if $R_S$ denotes the restriction operator to the unit sphere $S$, we find that
    %
    \begin{align*}
        \| \psi^\delta_j \cdot \widehat{f} \|_{L^2(\RR^d)} &\lesssim_N 2^{-\delta j} \left( \int_0^\infty \langle 2^j |1 - r| \rangle^{-2N} \int_{|\xi| = 1} |\widehat{f}(r \xi)|^2\; d\sigma\; r^{d-1} dr \right)^{1/2}\\
        &\lesssim 2^{-\delta j} \left( \int_0^\infty \langle 2^j |1 - r| \rangle^{-2N} \| R_S \circ \text{Dil}_r f \|^2_{L^2(S^{n-1})} \frac{dr}{r} \right)^{1/2}\\
        &\lesssim  2^{-\delta j} \| f \|_{L^p(\RR^d)} \left( \int_0^\infty \langle 2^j |1 - r| \rangle^{-2N}  r^{2d/p - 1} dr \right)^{1/2}\\
        &\lesssim 2^{-\delta j} \| f \|_{L^p(\RR^d)}.
        % f(x/r) -> r^d f^(rx)
        % 1/2^{j+1} <= |1 - r| <= 1/2^j
    \end{align*}
    %
    This bound is summable in $j$, which yields the required result.
\end{itemize}
%
Let us end our discussion of the Euclidean case by expanding on the computation of the inequality
%
\[ |\psi^\delta_j(\xi)| \lesssim_N 2^{-j \delta} \langle 2^j - |\xi| \rangle^{-N}. \]
%
Before this, let's see why the result is \emph{intuitive}. The function $\psi^\delta_j$ is obtained by localizing the frequency multiplier $m^\delta$ on the spatial side and then rescaling. Thus our result is intuitively saying that the phase-portrait of the multiplier is concentrated on a neighborhood of the set
%
\[ \{ (x,\xi) : |\xi| \leq 1\ \text{and}\ ||\xi| - 1| = 1/|x| \}. \]
% (1 - tau)^delta
% k_delta^j = Dil_{1/2^j} { k } phi
% 2^{-jd} [Dil_{2^j} m] * phi^
% m^delta = sum 2^{jd} Dil_{1/2^j} psi^delta_j
%
This makes sense, since the `high frequency' components of $m^\delta$ should be distributed near the boundary of the unit ball, since this is where the symbol becomes singular; that the spatial part should be inversely proportional to the distance to the boundary can be detected by taking derivatives of $m$ in the frequency variable, i.e. noting that if $||\xi| - 1| \sim 1/2^j$, then
%
\[ |\nabla^N m^\delta(\xi)| \lesssim_{N,\delta} (1 - |\xi|)^{\delta - N} \sim 2^{-j \delta} 2^{jN}. \]
%
And we see the derivative grows in $N$ as a power of $2^j$, which is inversely propoertional to $||\xi| - 1|$. Working more precisely, we have
%
\[ \psi^\delta_j = 2^{-jd} \left[ \widehat{\phi} * \text{Dil}_{2^j} m^\delta \right]. \]
%
The function is the average of $\widehat{\phi}$ over a ball of radius $O(2^j)$ so we immediately obtain a bound by using the rapid decay of $\widehat{\phi}$, thus obtaining that
%
\[ |\psi^\delta_j(\xi)| \lesssim_N \langle 2^j - |\xi| \rangle^{-N}. \]
% 
Thus we see that $\psi^\delta_j$ has the majority of it's support on the ball of radius $2^j$. But we can do much better than this using the fact that $\widehat{\phi}$ is \emph{oscillatory}, since $\phi$ is supported away from the origin, and $m^\delta$ is \emph{mostly} smooth. More precisely, $\widehat{\phi}$ oscillates at frequencies $\sim 1$, so we should expect integration by parts to yield useful decay on a quantity $\widehat{\phi} * \text{Dil}_{2^j} f$ if we had a bound $|\nabla^N f| \ll 2^{Nj}$ for large $N > 0$. This is true of $m^\delta$ away from a thickness $O(2^{-j})$ annulus containing the unit ball. Thus we are motivated to define $m^\delta = a^\delta_j + b^\delta_j$, where
%
\[ a^\delta_j(\xi) = m^\delta(\xi) \chi( 2^j (1 - |\xi|)) \quad\text{and}\quad b^\delta_j(\xi) = m^\delta_j(\xi) ( 1 - \chi( 2^j (1 - |\xi|)) ) \]
%
where $\chi(t)$ is supported on $|t| \leq 1$ and equal to one for $|t| \leq 1/2$. The function $b^\delta_j$ is therefore supported on $|\xi| \leq 1 - 1/2^{j+1}$. For $N > 0$, we have
% 
\[ |\nabla^N m^\delta_j(\xi)| \lesssim_{N,\delta} (1 - |\xi|)^{\delta-N}. \]
%
By the product rule, $\nabla^N b^\delta_j$ is a sum of derivatives of $m^\delta_j$ and of derivatives of $1 - \chi(2^j (|\xi| - 1))$. The support of any derivative of the latter is supported on $|\xi| \geq 1 - 1/2^j$. Thus we have
%
\[ |\nabla^N b^\delta_j(\xi)| \lesssim_N (1 - |\xi|)^{\delta - N} \mathbf{I}(|\xi| \leq 1 - 1/2^{j+1}) + 2^{j(N - \delta)} \mathbf{I}(1 - 1/2^j \leq |\xi| \leq 1). \]
%
Since $\phi$ is supported away from the origin, we may antidifferentiate $\widehat{\phi}$ arbitrarily many times without any singular behaviour emerging. But now averaging the $N$th antiderivative of $\widehat{\phi}$, which is rapidly decaying, with the $N$th derivative of $\text{Dil}_{2^j} b^\delta_j$, which is rapidly decaying outside of an annulus of width $1$ and radius $2^j$, we find that
%
\[ |(\widehat{\phi} * \text{Dil}_{2^j} b^\delta_j)(\xi)| \lesssim 2^{j(d - \delta)} \langle 2^j - \xi \rangle^{-N}. \]
%
The multiplier $a^\delta_j$ is not so smooth, but it is supported on a very thin annulus of radius 1 and thickness $O(2^{-j})$, and $m^\delta$ has magnitude at most $2^{- \delta j}$ on this annulus, which gives that
%
\[ |(\widehat{\phi} * \text{Dil}_{2^j} a^\delta_j)(\xi)| \lesssim 2^{-j\delta} \int_{||\eta| - 2^j| \leq 1} |\widehat{\phi}(\xi - \eta)|\; d\eta \lesssim_N 2^{j(d - \delta)} \langle 2^j - \xi \rangle^{-N}. \]
%
Putting these results together gives the required bound.

%$K_\rho^\delta$ is the kernel of $R_\rho^\delta$, then there exists symbols $a_1$ and $a_2$, of order zero, with $|a_1(x)|, |a_2(x)| \gtrsim 1$ for $|x| \gtrsim 1$, such that
%
%\[ K_\rho^\delta(x) = a_1(x) \frac{e^{2 \pi i |x|}}{\langle x \rangle^{\frac{n+1}{2} + \delta}} + a_2(x) \frac{e^{- 2 \pi i |x|}}{\langle x \rangle^{\frac{n+1}{2} + \delta}} + O \left( \frac{1}{\langle x \rangle^{n+1}} \right) \]

%    &= a_1(x) \frac{e^{2 \pi i |x|}}{\langle x \rangle^{\frac{n+1}{2} + \delta}} + a_2(x) \frac{e^{- 2 \pi i |x|}}{\langle x \rangle^{\frac{n+1}{2} + \delta}} + O \left( \frac{1}{\langle x \rangle^{n+1}} \right),
%\end{align*}
%
%where $a_1$ and $a_2$ are symbols of order zero, with $|a_1(x)|, |a_2(x)| \gtrsim 1$ for $|x| \gtrsim 1$. TODO: Necessity of the $\delta(p)$ values.

%
%\begin{align*}
%    K_\rho^\delta(x) &= \int_0^\rho (1 - \lambda / \rho)^\delta \left( \int_{|\xi| = \lambda} e^{2 \pi i \xi \cdot x}\; d\xi \right)\; d\lambda\\
%    &= \rho^{d-1} \int_0^1 \lambda^{d-1} (1 - \lambda)^\delta \left( \int_{|\xi| = 1} e^{2 \pi i \rho \lambda \xi \cdot x}\; d\xi \right)\; d\lambda\\
%    &= (2 \pi) \rho^{d/2} |x|^{1-d/2} \int_0^1 \lambda^{d/2} (1 - \lambda)^\delta J_{d/2-1}(2 \pi \rho \lambda |x|),
%\end{align*}
%
%where $J_{d/2-1}$ is a Bessel function of the first kind, i.e. defined by the integral formula
% See https://dlmf.nist.gov/10.17
%\begin{align*}
%    J_\nu(s) &= \frac{1}{\pi^{1/2}} \frac{(s/2)^\nu}{\Gamma(\nu + 1/2)} \int_{-1}^1 e^{isx} (1 - x^2)^{\nu - 1/2}\; dx.
%\end{align*}
%
%We are interested with the asymptotic theory here: there exists constants $a_1(k)$ and $a_2(k)$ (See the Digital Library of Mathematical Functions, 10.17: Hankel's Expansions for more detail) such that
%
%\begin{align*}
%    K_\rho^\delta(x) \sim \sum_{k = 0}^\infty a_1(k) \frac{\rho^{d/2-2k}}{|x|^{d/2 - 1 + 2k}} \int_0^1 \lambda^{d/2-2k} (1 - \lambda)^\delta e^{2 \pi i \rho \lambda |x|}
%\end{align*}
%The infinite sum here converges rapidly, so we can interchange the summation with the integral and conclude that
%
%\begin{align*}
%    K_\rho^\delta(x) &= 2 \pi \sum_{k = 0}^\infty \rho^{d/2 + 2k} |x|^{1-d/2+2k} \frac{(-1)^k}{k!} \frac{\pi^{2k}}{\Gamma(d/2 + k)} \int_0^1 \lambda^{d/2+2k} (1 - \lambda)^\delta\; d\lambda\\
%    &= 2 \pi \sum_{k = 0}^\infty \frac{(-1)^k}{k!} \frac{\pi^{2k}}{\Gamma(d/2 + k)} \frac{\Gamma(d/2 + 2k + 1) \Gamma(\delta + 1)}{\Gamma(d/2 + 2k + \delta + 2)} \rho^{d/2+2k} |x|^{1-d/2+2k}
%\end{align*}

\section{Manifold Case}

The analogue of the Tomas Stein theorem on a compact Riemannian manifold $X$ is a result due to Sogge, so let's see if we can obtain a result for compact manifolds using similar techniques:
%
\begin{itemize}
    \item The first problem is that on a compact Riemannian manifold we do not have a rescaling symmetry which we can use to reduce the study of the Bochner-Riesz multipliers $R^\delta_\rho$ to the case $\rho = 1$. Thus we must analyze a general multiplier of the form $R^\delta_\rho$ for all $\rho > 0$. The case of small $\rho$ is easily dealt with using the triangle inequality, so we may assume that $\rho \gtrsim 1$ in what follows.

    \item Now we try and reduce to Sogge's spectral cluster bounds, which are analogous to the Tomas-Stein bounds in $\RR^d$. If we are able to justify that $K^\delta_{\rho,j}$ behaves like a spectral band projection operator, as in the Euclidean setting, we'd be able to apply this bound. Plancherel does not quite have an analogy to the $L^2$ setting on a manifold. But we can instead use the wave operator and it's parametrices, i.e. that
    %
    \begin{align*}
        R^\delta_\rho &= \sum_\lambda r^\delta(\lambda / \rho) E_\lambda\\
        &= \rho \int_0^\infty \widehat{r^\delta}(\rho t) e^{2 \pi i t \sqrt{-\Delta}}\; dt\\
        &= c_\delta \cdot \rho^{-\delta} \int_0^\infty e^{2 \pi i \rho t} (t + i0)^{-\delta - 1} e^{2 \pi i t \sqrt{-\Delta}}\; dt.
    \end{align*}
    %
    The singularity in the definition of this integral occurs at $t = 0$, so the operator should, for large $t$, be relatively well behaved.

    \item Since we expect the function is well behaved for large $t$, let's bound these terms so we may reduce to controlling the integral over $t \lesssim 1$. Fix $\alpha \in C_c^\infty(\RR)$ equal to one in a neighborhood of zero, and consider the behaviour of $R^\delta_\rho$ for large $t$, i.e. the operator
    %
    \[ R^\delta_\rho = c_\delta \cdot \rho^{-\delta} \int_0^\infty (1 - \alpha(t)) \cdot e^{- 2 \pi i \rho t} t^{-\delta - 1} e^{2 \pi i t \sqrt{-\Delta}}\; dt. \]
    %
    If $\psi$ is the inverse Fourier transform of $c_\delta t^{-\delta-1} (1 - \alpha(t))$, then $\psi$ is bounded and rapidly decreasing because all of the derivatives of it's Fourier transform are smooth and integrable. We thus can revert back to the multiplier setting and write
    %
    \[ R^\delta_\rho = \rho^{-\delta} \sum_\lambda \psi(\lambda - \rho) E_\lambda. \]
    %
    The rapid decay here means we can be fairly lazy in controlling this operator, for instance, employing the Sobolev embedding bound
    %
    \[ \| E_\lambda f \|_{L^2(X)} \lesssim \langle \lambda \rangle^{d(1/p - 1/2) - 1/2} \| f \|_{L^p(X)} \]
    %
    and the triangle inequality, using the rapid decay to obtain that
    %
    \[ \| R^\delta_\rho f \|_{L^p(X)} \lesssim \langle \rho \rangle^{-[\delta - d(1/p - 1/2) + 1/2]} \| f \|_{L^p(X)}, \]
    %
    which is better than what we need. Thus we now need only bound the operator
    %
    \[ \tilde{R}^\delta_\rho = c_\delta \cdot \rho^{-\delta} \int_0^\infty \alpha(t) e^{2 \pi i \rho t} (t + i0)^{-\delta - 1} e^{2 \pi i t \sqrt{-\Delta}}\; dt. \]
    %
    The advantage of doing this is because we only have understanding of the wave operator through Fourier integral operators (through the Lax parametrix) for times $t \lesssim 1$.

    \item We now `spatially localize' as in the Euclidean case, though things look different here since we are dealing with the wave equation. We choose $\beta$ such that
    %
    \[ 1 = \chi + \sum_{j = 1}^\infty \text{Dil}_{2^j} \beta. \]
    %
    We then write
    %
    \[ R^\delta_{\rho} = \sum_{j = 0}^{O(\log \rho)} R^\delta_{\rho,j} \]
    %
    where for $j > 0$
    %
    \[ R^\delta_{\rho,j} = c_\delta \cdot \rho^{-\delta} \int_0^\infty \alpha(t) (\text{Dil}_{2^j} \beta)(\rho t) e^{- 2 \pi i \rho t} t^{-\delta - 1} e^{2 \pi i \sqrt{-\Delta}}\; dt, \]
    %
    and
    %
    \begin{align*}
        R^\delta_{\rho,0} &= c_\delta \cdot \rho^{-\delta} \int_0^\infty \alpha(t) \beta(\rho t) e^{-2 \pi i \rho t} t^{-\delta - 1} e^{2 \pi i \sqrt{-\Delta}}\; dt\\
        &= c_\delta \cdot \rho^{-\delta} \int_0^\infty \beta(\rho t) e^{-2 \pi i \rho t} t^{-\delta - 1} e^{2 \pi i \sqrt{-\Delta}}\; dt,
    \end{align*}
    %
    where the last identity follows because the result the support of the integral is on $t \lesssim 1/\rho$, and we are assuming $\rho$ is large so that $\alpha$ may be assumed equal to one on the support of the integral. Thus $R^\delta_\rho$ is an integral over $t \sim 2^j / \rho$. This is analogous to the spatial decomposition we performed in the Euclidean setting, except now we have the wave equation involved, and the `pseudolocal' finite speed of propogation for the wave equation now must substitute for the explicit spatial localization we obtained in the Euclidean decomposition.

    \item Despite the singularity that occurs at the origin, the case $j = 0$ is simplest to deal with. If we define
    %
    \[ m(\lambda) = c_\delta (\widehat{\beta} * r_\delta) \]
    %
    then $R^\delta_{\rho,0}$ is a multiplier operator with symbol
    %
    \[ m_\rho(\lambda) = \rho^{-\delta} \text{Dil}_\rho m. \]
    %
    We have estimates of the form
    %
    \[ | \nabla^N m(\lambda) | \lesssim_N \langle \lambda \rangle^{-M}. \]
    %
    Thus
    %
    \[ | \nabla^N m_\rho(\lambda)| \lesssim_N \rho^{-\delta-N} \langle \lambda / \rho \rangle^{-M}. \]
    %
    In particular, taking $M = N$ and $M = 0$ yields that
    %
    \[ | \nabla^N m_\rho(\lambda)| \lesssim_N \rho^{-\delta} \langle \lambda \rangle^{-N}. \]
    %
    Thus  $\{ \rho^\delta m_\rho \}$ are a uniformly bounded family of symbols of order zero. Thus (TODO: Review estimates for multipliers given by a symbol) we can obtain that
    %
    \[ \| m_\rho(\sqrt{-\Delta}) f \|_{L^p(X)} \lesssim \rho^{-\delta} \| f \|_{L^p(X)} \lesssim \| f \|_{L^p(X)}. \]
    %
    TODO: Check there isn't an error here since the $\rho^{-\delta}$ terms helps us out, but shouldn't our bounds be scale invariant?

    \item Now we deal with the $j > 0$ terms, and we must use the pseudolocal finite speed of propogation of the wave equation as a substitute for explicit localization. Since we have localized to times $t \lesssim 1$. We deal with this by using the Lax parametrix for the wave equation, but first we must ensure the remainder terms from employing the parametrix are well behaved. For $t \lesssim 1$, we can write $e^{2 \pi i t \sqrt{-\Delta}} = Q(t) + R(t)$, where $Q(t)$ is a Fourier integral operator supported on a $O(1)$ neighborhood of the diagonal $\Delta = \{ (x,x): x \in X \}$, and with kernel given in coordinates by
    %
    \[ (x,y) \mapsto \int e^{2 \pi i [\phi(x,y,\xi) + t |\xi|]} q(t,x,y,\xi)\; d\xi \]
    %
    where $q$ is a symbol of order zero, and $\phi$ is homogeneous of order one in $\xi$, with $\phi(x,y,\xi) \approx (x - y) \cdot \xi$, in the sense that
    %
    \[ |\nabla_\xi^N [ \phi(x,y,\xi) - (x - y) \cdot \xi ]| \lesssim_N |x - y|^2 |\xi|^{1-N} \]
    %
    for all $N > 0$. The operators $\{ R(t) \}$ are smoothing, i.e. with a joint kernel $A$ uniformly in $C^\infty([-1,1] \times X \times X)$. Thus we write
    %
    \begin{align*}
        R^\delta_{\rho,j} &= c_\delta \cdot \rho^{-\delta} \int_0^\infty \alpha(t) (\text{Dil}_{2^j} \beta)(\rho t) e^{- 2 \pi i \rho t} t^{-\delta - 1} ( Q(t) + R(t) )\; dt\\
        &= R^\delta_{\rho,j,Q} + R^\delta_{\rho,j,R}.
    \end{align*}
    %
    Let's control the $R(t)$ term. Computing the integral of the kernel defining $R^\delta_{\rho,j,R}$ leads to a term of the form
    %
    \[ c_\delta 2^j \rho^{-1-\delta} ( \widehat{\alpha A} * \text{Dil}_{\rho / 2^j} \widehat{\beta} * r^\delta)(\rho). \]
    %
    The function $\alpha A$ is smooth and compactly supported in the $t$ variable, so it's Fourier transform is rapidly decaying. The same is true of $\widehat{\beta}$, except it is rescaled so we can imagine the majority of it's mass occurs on $|\lambda| \lesssim \rho / 2^j$. Finally, $r^\delta$ is concentrated on $|\lambda| \lesssim 1$. Thus the kernel is pointwise bound from above by a constant times
    %
    \[ 2^j \rho^{-1-\delta} \int_{\rho - O(1)}^{\rho + O(1)} ( \widehat{\alpha A} * \text{Dil}_{\rho / 2^j} \widehat{\beta} )(\lambda)\; d\lambda. \]
    %
    Taking advantage of the oscillation of $\widehat{\beta}$, and the smoothness of $\widehat{\alpha A}$, i.e. integrating by parts, one can shows that for $|\lambda - \rho| \lesssim 1$
    %
    \[  |( \widehat{\alpha A} * \text{Dil}_{\rho / 2^j} \widehat{\beta} )(\lambda)| \lesssim_{N,M} (\rho / 2^j)^N \cdot \rho^{-M} \cdot (\rho / 2^j) = \rho^{1+N-M} 2^{-(N+1)j}, \]
    %
    Taking $N = M$ gives that the kernel is bounded above by
    %
    \[ 2^j \rho^{-1-\delta} \left( \rho 2^{-(N+1)j} \right) = 2^{-Nj}. \]
    %
    But now trivial estimates, e.g. using Schur's lemma implies that
    %
    \[ \| R^\delta_{\rho,j,R} f \|_{L^p(X)} \lesssim_N \rho^{-\delta} 2^{-Nj} \| f \|_{L^p(X)}, \]
    %
    a bound that can be summed in $j$ by taking, e.g. $N = 1$. Thus we are now reduced to the study of the oscillatory integral operators $R^\delta_{\rho,j,Q}$.

    \item Now let's localize. First off, the condition that $K_{\rho,j}$ is supported on the diagonal, and the compactness of $X$, means we need only prove the result restricted to a single coordinate chart. Let $K_{\rho,j,t}$ be the kernel of the operator $R^\delta_{\rho,j,Q}$. Intuitively, the wave equation travels at unit speed, so, since $R^\delta_{\rho,j,Q}$ involves the wave equation localized to times $t \sim 2^j / \rho$, we should expect this kernel to be localized to $|x - y| \lesssim 2^j / \rho$. In fact, we will show that the restricted kernel
    %
    \[ K'_{\rho,j,t}(x,y) = K_{\rho,j,t}(x,y) \cdot \mathbf{I}(|x - y| \geq 2^{j(1 + \varepsilon)} / \rho) \]
    %
    has $L^\infty_y L^1_x$ and $L^\infty_x L^1_y$ bounds of the form $O_{\varepsilon,N}(2^{-jN})$, so that Schur's lemma implies that if we write $(R^\delta_{\rho,j,Q})'$ as the operator with kernel $K'_{\rho,j,t}$, then
    %
    \[ \| (R^\delta_{\rho,j,Q})' f \|_{L^p(X)} \lesssim_N 2^{-jN} \| f \|_{L^p(X)}. \]
    %
    This reduces us to proving localized estimates of the following form: for some $\varepsilon > 0$, and for any function $f$ supported on a ball of radius $2^j / \rho$, we have a bound
    %
    \[ \| R^\delta_{\rho,j,Q} f \|_{L^p( O(2^j / \rho) )} \lesssim 2^{-j \varepsilon} \| f \|_{L^p(X)}. \]
    %
    Notice the localization we get here is slightly weaker than in the Euclidean setting (the operators are localized to balls of radius $O(2^{j(1 + \varepsilon)} / \rho)$ for any $\varepsilon > 0$ rather than localized to balls of radius $O(2^j / \rho)$) which means our bounds here need the slightly greater decay in $j$ (the $O(2^{-j \varepsilon})$ bound above) rather than a bound independent of $j$.

    To prove the bounds for the restricted kernel $K'_{\rho,j,t}$ above, we just apply the principle of nonstationary phase to the integral representation, which says that for $|x - y| \gtrsim 2^{j(1+\varepsilon)} / \rho$ we have, taking the Fourier inversion formula in the $t$ variable,
    %
    \begin{align*}
        K'_{\rho,j,t} &= c_\delta \rho^{-\delta} \int_0^\infty \int \alpha(t) (\text{Dil}_{2^j} \beta)(\rho t) (t + i0)^{-\delta - 1} q(t,x,y,\xi) e^{2 \pi i [\phi(x,y,\xi) + t |\xi| - \rho t]}\; d\xi\; dt\\
        &= \int a^\delta_{\rho,j}(x,y,\xi,|\xi| - \rho) e^{2 \pi i \phi(x,y,\xi)}\; d\xi,
    \end{align*}
    %
    where
    %
    \[ a^\delta_{\rho,j}(x,y,\xi,\cdot) = c_\delta 2^j \rho^{-1-\delta} (\widehat{\alpha q(\cdot, x,y,\xi)} * \text{Dil}_{\rho / 2^j} \beta * r^\delta * q_{\cdot} (x,y,\xi)) \]
    %
    and therefore TODO satisfies estimates of the form
    %
    \[ |\nabla^n_t \nabla^m_\xi a^\delta_{\rho,j}| \lesssim_{n,m,N} 2^{-j \delta} (2^j / \rho)^n \langle 2^j \tau / \rho \rangle^{-N} \langle\xi \rangle^{-m} \]
    %
    Nonstationary phase TODO thus gives the required bounds.

    % Suppose the operator K1 = K I(|x - y| >= 2^{e j} ) has operator norm O_{N,e}( 2^{-Nj} ) for all N > 0
    % Given f, write f = sum f_Q, where f_Q is supported on a cube Q with sidelength 1
    % Then |f|_{Lp}^p = sum |f_Q|_{Lp}^p

    % We write Kf_Q = (K1)f_Q + (1 - K1)f_Q
    
    % Then (1 - K1)f_Q is supported on 2^{e j} * Q, which overlaps with at most O(2^{(e d) j}) other cubes. Thus
    % | (1 - K1) f |_{Lp}^p = | sum (1 - K1)f_Q |_{L^p}^p << 2^{ d e j (p/ p^*)} sum | (1 - K1) f_Q |_p^p
    % If we could show that | (1 - K1) f_Q |_p << 2^{-d e j / p^*}
    % then we would conclude that | (1 - K1) f |_{Lp} << sum |f_Q|_p^p = |f|_{Lp}^p
    
    % Often e is arbitrary, so we need only show any bound of the form | (1 - K1) f_Q |_p << 2^{- ej} to get the required result.

    \item It now suffices to show that for some $\varepsilon > 0$, and for any function $f$ supported on a ball $B$ of radius $2^j / \rho$, we have a bound
    %
    \[ \| R^\delta_{\rho,j,Q} f \|_{L^p( O(1) \cdot B )} \lesssim 2^{-j \varepsilon} \| f \|_{L^p(X)}. \]
    %
    Since we are localized, we can now, like in the Euclidean case, reduce to an $L^2$ bound, i.e. writing
    %
    \[ \| R^\delta_{\rho,j,Q} f \|_{L^p( O(1) \cdot B)} \lesssim (2^j / \rho)^{d(1/p - 1/2)} \| R^\delta_{\rho,j,Q} f \|_{L^2(O(1) \cdot B)}. \]
    %
    It now suffices to note TODO that $R^\delta_{\rho,j,Q}$ is a Fourier multiplier operator with symbol which is pointwise bouned by $O_N(2^{- j \delta}  \langle 2^j \tau / \rho \rangle^{-N})$, so we can now TODO apply Sogge's version of Tomas Stein on manifolds summed over geometric intervals to yield the required bounds.
\end{itemize}

%\section{Proof Involving Carleson-Sj\"{o}lin}

%On $\RR^d$, we can take Fourier transforms, applying stationary phase to determine that if $K_{\rho,\delta}$ is the convolution kernel corresponding to $R_\rho^\delta$, then
%
%\begin{align*}
%    K_\delta(x) &= \int_0^1 \lambda^{n-1} (1 - \lambda)^\delta e^{2 \pi i \xi \cdot x}\; d\lambda\\
%    &= a_1(x) \frac{e^{2 \pi i |x|}}{\langle x \rangle^{\frac{n+1}{2} + \delta}} + a_2(x) \frac{e^{- 2 \pi i |x|}}{\langle x \rangle^{\frac{n+1}{2} + \delta}} + O \left( \frac{1}{\langle x \rangle^{n+1}} \right),
%\end{align*}
%
%where $a_1$ and $a_2$ are symbols of order zero, with $|a_1(x)|, |a_2(x)| \gtrsim 1$ for $|x| \gtrsim 1$. TODO: Necessity of the $\delta(p)$ values.

%\begin{lemma}
%    If $a \in C_c^\infty(\RR^n \times \RR^n)$, and $a(x,y) = 0$ unless $1/2 \leq |x - y| \leq 2$, then
    %
%    \[ \left\| \int e^{2 \pi i \rho |x - y|} a(x,y) f(y) \right\|_{L^q(\RR^n)} \lesssim \rho^{n/q} \| f \|_{L^p(\RR^n)} \]
    %
%    if $q = [(n+1)/(n-1)] p^*$ and $1 \leq p \leq 2$.
%\end{lemma}
%\begin{proof}
%    This is a non homogeneous oscillatory integral operator with wavefront set
    %
%    \[ \left\{ \left( x, y ; \frac{x}{|x - y|}, \frac{y}{|x - y|} \right) \right\} \]
    %
%    which, because of the assumption of the support of $a$, satisfies the Carleson Sj\"{o}lin conditions, and thus the result follows.
%\end{proof}

%The required bounds now follows by applying a dyadic spatial decomposition, rescaling, and applying the result above, hich can be applied because of our explicit computation of the kernel $K_\delta$ above. TODO: Go over this argument in more detail to make sure it actually works.















\chapter{Heo, Nazarov, and Seeger: Initial Radial Conjecture Results}

In this chapter we give a description of the techniques of Heo, Nazarov, and Seeger's 2011 paper \emph{Radial Fourier Multipliers in High Dimensions} \cite{HeoNazrovSeeger2011}. One of the main goals of this paper is to verify the radial multiplier conjecture in $\RR^d$ for $d \geq 4$, and $1 < p < p_d$, where $p_d = 2(d-1)/(d+!)$, i.e. that if $m \in L^\infty(\ZZ)$ is a radial function, $d \geq 4$, and $\beta \in \mathcal{S}(\RR^d)$ is nonzero, then
%
\[ \| m \|_{M^p(\RR^d)} \sim \sup_{t > 0} t^{d/p} \| T_m(\text{Dil}_t \beta) \|_{L^p(\RR^d)} \quad \text{for}\ p \in \left(1, \frac{2(d-1)}{d+1} \right), \]
%
where the implicit constant depends on $p$ and $\beta$. We have
%
\[ \sup_{t > 0} t^{d/p} \| T_m(\text{Dil}_t \beta) \|_{L^p(\RR^d)} \sim \sup_{t > 0} \frac{\| T_m(\text{Dil}_t \beta) \|_{L^p(\RR^d)}}{\| \text{Dil}_t \beta \|_{L^p(\RR^d)}}. \]
%
Thus we find that the boundedness of $T_m$ on $\mathcal{S}(\RR^d)$ is equivalent to it's boundedness on the family of inputs $\{ \text{Dil}_t \beta \}$. If we make the assumption that $m$ is compactly supported, then the assumption is equivalent to the fact that the convolution kernel $k$ associated with $m$ is in $L^p(\RR^n)$.

Another consequence of the techniques of this paper is that an `endpoint' result for local smoothing. Namely, the techniques of the paper imply that if $d \geq 4$, and $q > 2 + 4/(d-3)$, then
%
\[ \frac{1}{2L} \int_{-L}^L \| e^{it \sqrt{-\Delta}} f \|_{L^q(\RR^d)}^q\; dt \lesssim_{q,d} \| (I - L^2 \Delta)^{\alpha/2} f \|_{L^q(\RR^d)}^q, \]
%
where $\alpha = d(1/2 - 1/q) - 1/2$. This is an endpoint result because $\alpha$ can be set \emph{equal} to the tight local smoothing exponent, rather than just arbitrarily close as in results that use decoupling type techniques.

\section{Discretized Reduction}

It is obvious that
%
\[ \| m \|_{M^p(\RR^d)} \gtrsim_\beta \sup_{t > 0} t^{d/p} \| T_m(\text{Dil}_t \beta) \|_{L^p(\RR^d)}, \]
%
so it suffices to show that
%
\[ \| m \|_{M^p(\RR^d)} \lesssim_\beta \sup_{t > 0} t^{d/p} \| T_m(\text{Dil}_t \beta) \|_{L^p(\RR^d)}, \]
%
We will show this via a discrete convolution inequality, which can also be used to prove local smoothing results for the wave equation.

Let $\sigma_r$ be the surface measure for the sphere of radius $r$ centered at the origin in $\RR^d$. Also fix a nonzero, radial, compactly supported function $\psi \in \mathcal{S}(\RR^d)$ whose Fourier transform is non-negative, and vanishes to high order at the origin. Given $x \in \RR^d$ and $r \geq 1$, define $\chi_{x r} = \text{Trans}_x (\sigma_r * \psi)$, which we view as a smooth function oscillation on a thickness $\approx 1$ annulus of radius $r$ centered at $x$. Our goal is to prove the following inequality.

\begin{lemma} \label{lemma1}
    For any $a : \RR^d \times [1,\infty) \to \CC$, and $1 \leq p < p_d$,
    %
    \[ \left\| \int_{\RR^d} \int_1^\infty a(x,r) \chi_{x, r}\; dx\; dr \right\|_{L^p(\RR^d)} \lesssim \left( \int_{\RR^d} \int_1^\infty |a(x,r)|^p r^{d-1} dr dx \right)^{1/p}. \]
    %
    The implicit constant here depends on $p$, $d$, and $\psi$.
\end{lemma}

How does Lemma \ref{lemma1} prove the required result? Suppose $m: \RR^d \to \CC$ is a radial multiplier, so we can consider it's convolution kernel $k: \RR^d \to \CC$, which is also radial. Let $k(x) = b(|x|)$ for some function $b: [0,\infty) \to \CC$. If we set $a(x,r) = g(x) b(r)$ for any function $g: \RR^d \to \CC$, then the function
%
\[ F(x) = \int_{\RR^d} \int_1^\infty a(x',r) \chi_{x', r}\; dx'\; dr, \]
%
is equal to $k * \psi * g$. In this setting, Lemma \ref{lemma1} says that
%
\[ \| k * \psi * g \|_{L^p(\RR^d)} \lesssim \| k \|_{L^p(\RR^d)} \| g \|_{L^p(\RR^d)}. \]
%
The left hand side is a Fourier multiplier operator applied to $g$, with symbol equal to $\widehat{\psi} \cdot m$, which is clearly related to the bounds we want to show. In particular, if $m$ is compactly supported away from the origin, let's say, on the annulus $1/2 \leq |\xi| \leq 2$. If we chose $\psi$ so that $\widehat{\psi}$ is nonvanishing on the annulus $1/4 \leq |\xi| \leq 2$, then the multiplier $1/\widehat{\psi}$ is smooth on the support of $m$, and so satisfies $L^p \to L^p$ bounds for all $1 < p < \infty$ restricted to functions with Fourier support on $m$. In particular, we conclude that $m$ is bounded from $L^p$ to $L^p$ if it's Fourier transform lies in $L^p(\RR^d)$. We can then use other tools (Hardy space technology and the like) to study more general multipliers that aren't compactly supported.

To prove Lemma \ref{lemma1}, it suffices to prove the following discretized estimate where we replace integrals with sums.

\begin{theorem} \label{lemma2}
    Fix a finite family of pairs $\mathcal{E} \subset \RR^d \times [1,\infty)$, which is \emph{discretized} in the sense that for any $(x_1,r_1)$ and $(x_2,r_2)$ in $\mathcal{E}$, one either has $x_1 = x_2$, or $|x_1 - x_2| \geq 1$, and one either has $r_1 = r_2$, or $|r_1 - r_2| \geq 1$. Then for any $a: \mathcal{E} \to \CC$ and $1 \leq p < 2(d - 1)/(d+1)$, 
    %
    \[ \left\| \sum_{(x,r) \in \mathcal{E}} a(x,r) \chi_{x, r} \right\|_{L^p(\RR^d)} \lesssim \left( \sum_{(x,r) \in \mathcal{E}} |a(x,r)|^p r^{p-1} \right)^{1/p}, \]
    %
    where the implicit constant depends on $p$, $d$, and $\psi$, but most importantly, is independent of $\mathcal{E}$.
\end{theorem}

\begin{proof}[Proof of Lemma \ref{lemma1} from Lemma \ref{lemma2}]
    For any $a: \RR^d \times [1,\infty) \to \CC$, if we consider the vector-valued function $\mathbf{a}(x,r) = a(x,r) \chi_{x,r}$, then
    %
    \[ \int_{\RR^d} \int_1^\infty \mathbf{a}(x,r)\; dr\; dx = \int_{[0,1)^d} \int_0^1 \sum\nolimits_{n \in \ZZ^d} \sum\nolimits_{m > 0} \text{Trans}_{n,m} \mathbf{a}(x,r)\; dr\; dx \]
    %
    Minkowski's inequality thus implies that
    %
    \begin{align*}
    \left\| \int_{\RR^d} \int_1^\infty \mathbf{a}(x,r)\; dr\; dx \right\|_{L^p(\RR^d)} &\leq \int_{[0,1)^d} \int_0^1 \left\| \sum\nolimits_{n \in \ZZ^d} \sum\nolimits_{m > 0} \text{Trans}_{n,m} \mathbf{a}(x,r) \right\|_{L^p(\RR^d)}\; dr\; dx\\
    &\lesssim \int_{[0,1)^d} \int_0^1 \left( \sum\nolimits_{n \in \ZZ^d} \sum\nolimits_{m > 0} |a(x - n, r + m)|^p r^{d-1} \right)^{1/p}\; dr\; dx\\
    &\leq \left( \int_{[0,1)^d} \int_0^1 \sum\nolimits_{n \in \ZZ^d} \sum\nolimits_{m > 0} |a(x - n, r + m)|^p r^{d-1}\; dr\; dx \right)^{1/p}\\
    &= \left( \int_{\RR^d} \int_1^\infty |a(x,r)|^p r^{d-1} dr dx \right)^{1/p}. \qedhere
    \end{align*}
\end{proof}

Lemma \ref{lemma2} is further reduced by considering it as a bound on the operator $a \mapsto \sum_{(x,r) \in \mathcal{E}} a(x,r) \chi_{x,r}$. In particular, applying real interpolation, it suffices for us to prove a restricted strong type bound. Given any discretized set $\mathcal{E}$, let $\mathcal{E}_k$ be the set of $(x,r) \in \mathcal{E}$ with $2^k \leq r < 2^{k+1}$. Then Lemma \ref{lemma2} is implied by the following Lemma.

\begin{lemma} \label{lemma3}
    For any $1 \leq p < 2(d - 1)/(d+1)$ and $k \geq 1$,
    %
    \[ \left\| \sum_{(x,r) \in \mathcal{E}} \chi_{x,r} \right\|_{L^p(\RR^d)} \lesssim_p \left( \sum_{k \geq 1} 2^{k(d-1)} \#(\mathcal{E}_k) \right)^{1/p}. \]
\end{lemma}

\begin{remark}
    Note that if $2^k \leq r \leq 2^{k+1}$, then because $\| \chi_{x,r} \|_{L^p(\RR^d)} \sim 2^{k(d-1)/p}$, we can write this as
    %
    \[ \left\| \sum_{(x,r) \in \mathcal{E}} \chi_{x,r} \right\|_{L^p(\RR^d)} \lesssim_p \left( \sum_{(x,r) \in \mathcal{E}} \| \chi_{x,r} \|_{L^p(\RR^d)}^p \right)^{1/p}. \]
    %
    Thus we are proving a kind of $l^p L^p$ decoupling for the functions $\chi_{x,r}$. This is strictly weaker than an $l^2 L^p$ decoupling bound. TODO: Could we possibly get an $l^2 L^p$ decoupling bound here?
\end{remark}

\begin{comment}
\begin{proof}[Proof of Lemma \ref{lemma2} from Lemma \ref{lemma3}]
    Let
    %
    \[ F = \sum_{(x,r) \in \mathcal{E}} \chi_{x,r} \]
    %
    and then for $k \geq 1$, let
    %
    \[ F_k = \sum_{(x,r) \in \mathcal{E}_k} \chi_{x,r}. \]
    %
    Then $F = \sum_k F_k$, and. Applying a dyadic interpolation result (Lemma 2.2 of that paper), the bound
    %
    \[ \| F_k \|_{L^r(\RR^d)} \lesssim 2^k (2^{k(d-r-1)} \#(\mathcal{E}_k)^{1/r}) \]
    %
    which holds for $r$ to the left and right of $p$, can be interpolated to yield that
    %
    \[ \| F \|_{L^p(\RR^d)} \lesssim \left( \sum_k 2^{kp} ( 2^{k(d-r-1)} ) \right)^{1/p} \]


    Applying a dyadic interpolation result (Lemma 2.2 of the paper), Lemma \ref{lemma3} implies that
    %
    \[ \left\| \sum_{(x,r) \in \mathcal{E}} \chi_{x,r} \right\| \]

    %
    \[ \left\| \sum_{(x,r) \in \mathcal{E}} \chi_{x,r} \right\|_{L^p(\RR^d)} \lesssim \left( \sum 2^{kp} 2^{k(d-p-1)} \#(\mathcal{E}_k) \right)^{1/p} = \left( \sum 2^{k(d-1)} \#(\mathcal{E}_k) \right)^{1/p} \]
    %
    This is a restricted strong type bound for Lemma \ref{lemma2}, which we can then interpolate.
\end{proof}
\end{comment}

\section{Density Decomposition}

To control these sums, we apply a `density decomposition', somewhat analogous to a Calderon Zygmund decomposition, which will enable us to obtain $L^2$ bounds. We say a 1-separated set $\mathcal{E}$ in $\RR^d \times [R,2R)$ is of \emph{density type} $(u,R)$ if
%
\[ \#(B \cap \mathcal{E}) \leq u \cdot \diam(B) \]
%
for each ball $B$ in $\RR^{d+1}$ with diameter $\leq R$.

%A covering argument then shows that for any ball $B$,
%
%\[ \#(B \cap \mathcal{E}) \lesssim_d u \cdot \left( 1 + \frac{\diam(B)}{R} \right)^d \cdot \diam(B). \]
%
%(NOTE: WE MIGHT BE ABLE TO DO BETTER USING THE FACT THAT $\mathcal{E} \subset \RR^d \times [R,2R)$, USING THE VALUE $R$).

\begin{theorem} \label{DecompositionTheorem}
    For any 1-separated set $\mathcal{E}_k \subset \RR^d \times [2^k,2^{k+1})$, we can consider a disjoint union $\mathcal{E}_k = \bigcup_{m = 1}^\infty \mathcal{E}_k(2^m)$ with the following properties:
    %
    \begin{itemize}
        \item For each $m$, $\mathcal{E}_k(2^m)$ has density type $(2^m,2^k)$.

        \item If $B$ is a ball of radius $r \leq 2^k$ containing at least $2^m \cdot r$ points of $\mathcal{E}_k$, then
        %
        \[ B \cap \mathcal{E}_k \subset \bigcup_{m' \geq m} \mathcal{E}_k(2^{m'}). \]

        \item For each $m$, there are disjoint balls $\{ B_i \}$, with radii $\{ r_i \}$, each at most $2^k$, such that
        %
        \[ \sum_i r_i \leq \frac{\#(\mathcal{E}_k)}{2^m} \]
        %
        such that $\bigcup B_i^*$ covers $\bigcup_{m' \geq m} \mathcal{E}_k(2^{m'})$, where $B_i^*$ denotes the ball with the same center as $B_i$ but 5 times the radius.
    \end{itemize}
\end{theorem}
\begin{proof}
    Define a function $M: \mathcal{E}_k \to [0,\infty)$ by setting
    %
    \[ M(x,r) = \sup \left\{ \frac{\#(\mathcal{E}_k \cap B)}{\text{rad}(B)} : (x,r) \in B\ \text{and}\ \text{rad}(B) \leq 2^k \right\}. \]
    %
    We can establish a kind of weak $L^1$ estimate for $M$ using a Vitali type argument. Let
    %
    \[ \widehat{\mathcal{E}}_k(2^m) = \{ (x,r) \in \mathcal{E}_k : M(x,r) \geq 2^m \}. \]
    %
    We can therefore cover $\widehat{\mathcal{E}}_k(2^m)$ by a family of balls $\{ B \}$ such that $\#(\mathcal{E}_k \cap B) \geq 2^m \text{rad}(B)$. The Vitali covering lemma allows us to find a disjoint subcollection of balls $B_1,\dots,B_N$ such that $B_1^* ,\dots, B_N^*$ covers $\widehat{\mathcal{E}}_k(2^m)$. We find that
    %
    \[ \#(\mathcal{E}_k) \geq \sum_i \#(B_i \cap \mathcal{E}_k) \geq 2^m \sum_i \text{rad}(B_i), \]
    %
    Setting $\mathcal{E}_k = \widehat{\mathcal{E}}_k(2^m) - \bigcup_{k' > k} \widehat{\mathcal{E}}_{k'}(2^m)$ thus gives the required result.
\end{proof}

To prove Lemma \ref{lemma3}, we perform a decomposition of $\mathcal{E}_k$ for each $k$, into the sets $\mathcal{E}_k(2^m)$, and then define $\mathcal{E}^m = \bigcup_{k \geq 1} \mathcal{E}_k^m$. For appropriate exponents, we will prove $L^p$ bounds on the functions
%
\[ F^m = \sum_{(x,r) \in \mathcal{E}^m} \chi_{x,r} \]
%
which are exponentially decaying in $m$, i.e. that
%
\[ \| F^m \|_{L^p(\RR^d)} \lesssim m \cdot 2^{-m(1/p - 1/p_d)} \left( \sum_k 2^{k(d-1)} \#(\mathcal{E}_k) \right)^{1/p}. \]
%
Thus summing in $m$ using the triangle inequality gives a bound on $F = \sum_m F^m$, in the range $1 < p < p_d$, i.e. that
%
\[ \| F \|_{L^p(\RR^d)} \lesssim \left( \sum_k 2^{k(d-1)} \#(\mathcal{E}_k) \right)^{1/p}, \]
%
proving Lemma \ref{lemma3}. To get the bound on $F^m$, we interpolate being an $L^2$ bound for $F^m$, and an $L^0$ bound (i.e. a bound on the measure of the support of $F^m$). First, we calculate the support of $F^m$.

\begin{lemma} \label{lemma5}
    For each $k$,
    %
    \[ |\text{supp}(F^m_k)| \lesssim 2^{-m} 2^{k(d-1)} \# \mathcal{E}_k. \]
    %
    Thus we have
    %
    \[ |\text{supp}(F^m)| \leq \sum_k |\text{supp}(F^m_k)| \lesssim \sum_k 2^{-m} 2^{k(d-1)} \# \mathcal{E}_k. \]
\end{lemma}
\begin{proof}
    We recall that for each $k$ and $m$, we can find disjoint balls $B_1,\dots,B_N$ with radii $r_1,\dots,r_N \leq 2^k$ such that
    %
    \[ \sum_{i = 1}^N r_i \leq 2^{-m} \# \mathcal{E}_k, \]
    %
    where $\mathcal{E}_k(2^m)$ is covered by the expanded balls $B_1^* \cup \dots \cup B_N^*$. If we write
    %
    \[ F^m_{k,i} = \sum_{(x,r) \in \mathcal{E}_k(2^m) \cap B_i^*} \chi_{x,r}, \]
    %
    then $\text{supp}(F^m_k) \subset \bigcup_i \text{supp}(F^m_{k,i})$. For each $(x,r) \in B_i^* \cap \mathcal{E}_k(2^m)$, the support of $\chi_{x,r}$, an annulus of thickness $O(1)$ and radius $r$, is contained in an annulus of thickness $O(r_i)$ and radius $O(2^k)$ with the same centre as $B_i$. Thus we conclude that
    %
    \[ |\text{supp}(F^m_{k,i})| \lesssim r_i 2^{k(d-1)}, \]
    %
    and it follows that
    %
    \[ |\text{supp}(F^m_k)| \leq \sum_i r_i 2^{k(d-1)} \leq 2^{-m} 2^{k(d-1)} \# \mathcal{E}_k. \qedhere \]
\end{proof}

From interpolation, it therefore suffices to prove the following $L^2$ estimate on the function $F^m$.

\begin{lemma} \label{lemma6}
    Suppose $\mathcal{E} = \bigcup_k \mathcal{E}_k$ is a one-separated set, where $\mathcal{E}_k \subset \RR^d \times [2^k,2^{k+1})$ is a set of density type $(2^m, 2^k)$. Then
    %
    \[ \left\| \sum_{(x,r) \in \mathcal{E}} \chi_{x,r} \right\|_{L^2(\RR^d)} \lesssim \sqrt{m} \cdot 2^{ \frac{m}{d-1} } \left( \sum_k 2^{k(d-1)} \#(\mathcal{E}_k) \right)^{1/2}. \]
\end{lemma}

% The L2 norms of the chi_{x,r} are equal to 2^{k(d-1)/2}, so the
% triangle ienquality implies that the LHS is bounded by sum_k 2^{k(d-1)/2} \#(E_k)

Note that this bound gets worse and worse as $m$ grows, whereas the support bound gets better and better, since annuli are concentrating in a small set, which is bad from the perspective of constructive interference, but absolutely fine from the perspective of a support bound. Interpolation gives a bound exponentially decaying in $m$ for $1 < p < p_d$.

\begin{comment}
\begin{proof}[Proof of Lemma \ref{lemma3} from Lemma \ref{lemma6}]
    Write $F = \sum_{(x,r) \in \mathcal{E}_k} \chi_{x,r}$, and then perform a decomposition $\mathcal{E}_k = \bigcup_{m \geq 0} \mathcal{E}_k(2^m)$, and thus define $F = \sum_{m \geq 0} F_m$, where
    %
    \[ F_m = \sum_{(x,r) \in \mathcal{E}(2^m)} \chi_{x,r}. \]
    %
    We have
    %
    \[ \| F_m \|_{L^2(\RR^d)} \lesssim 2^{\frac{m}{d-1} + \frac{k(d-1)}{2}} \log(2 + 2^m)^{1/2} \cdot \#(\mathcal{E}_k)^{1/2}. \]
    %
    If we interpolate this bound with the support bound for $F_m$, a kind of $L^0$ norm estimate, we conclude that for $0 < p \leq 2$,
    % ( int |F_m|^p )^{1/p} <= |S|^{1/pq^*} int |F_m|^{pq} )^{1/pq}
    % pq = 2
    % Then q = 2/p so 1/q^* = 1 - 1/q = 1 - p/2 = (2 - p)/2
    % so q^* = 2/(2-p)
    % 1/pq^* = (2-p)/2p = (1/p - 1/2)
    \begin{align*}
        \| F_m \|_{L^p(\RR^d)} &\leq |\text{Supp}(F_m)|^{1/p - 1/2} \| F_m \|_{L^2(\RR^d)}\\
        &\lesssim ( 2^{k(d-1) - m})^{1/p - 1/2} 2^{\frac{m}{d-1} + \frac{k(d-1)}{2}} \log(2 + 2^m)^{1/2} \cdot \#(\mathcal{E}_k)^{1/p} \\
        &\lesssim 2^{m(1/p_d - 1/p)} \log(2 + 2^m)^{1/2} 2^{\frac{k(d-1)}{p}} \#(\mathcal{E}_k)^{1/p}.
    \end{align*}
    % int |F_m|^p <= |S|^{1/p-1/2} ( int |F_m|^2 )^{1/2}
    %
    where $p_d = 2(d-1)/(d+1)$. This bound is summable in $m$ for $p < p_d$, which enables us to conclude that
    %
    \[ \| F \|_{L^p(\RR^d)} \lesssim 2^{\frac{k(d-1)}{p}} \#(\mathcal{E}_k)^{1/p}. \]
    %
    Thus for $1 \leq p < p_d$, we obtain the bound stated in Lemma \ref{lemma3}.
\end{proof}
\end{comment}

\section{$L^2$ Bounds}

Proving \ref{lemma6} is where the weak-orthogonality bounds from Lemma \ref{lemma4} come into play. Indeed, we can write the inequality as
%
\[ \left\| \sum_{(x,r) \in \mathcal{E}} \chi_{x,r} \right\|_{L^2(\RR^d)} \lesssim \sqrt{m} \cdot 2^{ \frac{m}{d-1} } \left( \sum_{(x,r) \in \mathcal{E}} \| \chi_{x,r} \|_{L^2(\RR^d)}^2 \right)^{1/2}, \]
%
and if we had perfect orthogonality, or even almost orthogonality, then we could replace the $\sqrt{m} \cdot 2^{\frac{m}{d-1}}$ term with a constant.

To prove this $L^2$ bound, we require an analysis of the interference patterns of the functions $\chi_{x,r}$, which are supported on various annuli, but oscillate on these annuli. We will use almost orthogonality principles to understand these interference patterns which work the best now we have reduced our analysis to $L^2$ bounds.

%If $\psi$ is compactly supported, and $r$ is sufficiently large depending on the size of this support, then $\chi_{x,r}$ is supported on an annulus with centre $x$, radius $r$, and thickness $O(1)$. Thus $\| \chi_{x,r} \|_{L^p(\RR^d)} \sim r^{(d-1)/p}$, which implies that
%
%\[ \left\| \sum_{(x,r) \in \mathcal{E}_k} \chi_{x,r} \right\|_{L^p(\RR^d)} \gtrsim 2^{k(d-1)/p} \#(\mathcal{E}_k)^{1/p}. \]
%
%Thus this bound can only be true if $p \geq 1$, and becomes tight when $p = 1$, where we actually have
%
%\[ \left\| \sum_{(x,r) \in \mathcal{E}_k} \chi_{x,r} \right\|_{L^1(\RR^d)} \sim 2^{k(d-1)} \#(\mathcal{E}_k) \]
%

\begin{lemma} \label{lemma4}
    For any $N > 0$, $x_1,x_2 \in \RR^d$ and $r_1,r_2 \geq 1$,
    %with $|x_1 - x_2| \geq 1$ or $x_1 = x_2$, and $r_1,r_2 > 1$,
    %
    \begin{align*}
        |\langle \chi_{x_1,r_1}, \chi_{x_2,r_2} \rangle| &\lesssim_N (r_1r_2)^{(d-1)/2} (1 + |r_1 - r_2| + |x_1 - x_2|)^{-(d-1)/2}\\
        &\quad\quad\quad\sum_{\pm,\pm} (1 + ||x_1 - x_2| \pm r_1 \pm r_2|)^{-N}.
    \end{align*}
    %
    In particular,
    %
    \[ |\langle \chi_{x_1,r_1}, \chi_{x_2,r_2} \rangle| \lesssim \left( \frac{r_1r_2}{|(x_1,r_1) - (x_2,r_2)|} \right)^{(d-1)/2} \]
\end{lemma}

\begin{remark}
    Suppose $r_1 \leq r_2$. Then Lemma \ref{lemma4} implies that $\chi_{x_1,r_1}$ and $\chi_{x_2,r_2}$ are roughly uncorrelated, except when $|x_1 - x_2|$ and $|r_1 - r_2|$ is small, and in addition, one of the following two properties hold:
    %
    \begin{itemize}
        \item $r_1 + r_2 \approx |x_1 - x_2|$, which holds when the two annuli are `approximately' externally tangent to one another.

        \item $r_2 - r_1 \approx |x_1 - x_2|$, which holds when the two annuli are `approximately' internally tangent to one another.
    \end{itemize}
    %
    Heo, Nazarov, and Seeger do not exploit the tangency information, though utilizing the tangencies seems important to improve the results they obtain. Laura Cladek's paper exploits this tangency information, to some extent, to obtain the improved result in her paper.
\end{remark}

\begin{proof}
%    We may assume $|x_1 - x_2| \geq 1$, for otherwise the inequality holds trivially since unless $|r_1 - r_2| \lesssim 1$, $f_{x_1r_1}$ and $f_{x_2r_2}$ have disjoint support, and if $|r_1 - r_2| \lesssim 1$ then Cauchy Schwartz implies that
    %
%    \begin{align*}
%        |\langle f_{x_1r_1}, f_{x_2r_2} \rangle| &\lesssim (r_1 r_2)^{(d-1)/2}\\
%        &\lesssim_{N,d} (r_1r_2)^{(d-1)/2} (1 + |r_1 - r_2| + |x_1 - x_2|)^{-(d-1)/2} \sum_{\pm,\pm} (1 + ||x_1 - x_2| \pm r_1 \pm r_2|)^{-N}
%    \end{align*}
%
    We write
    %
    \begin{align*}
        \langle \chi_{x_1 r_1}, \chi_{x_2 r_2} \rangle &= \left\langle \widehat{\chi}_{x_1 r_1}, \widehat{\chi}_{x_2 r_2} \right\rangle\\
        &= \int_{\RR^d} \widehat{\sigma_{r_1} * \psi}(\xi) \cdot \overline{\widehat{\sigma_{r_2} * \psi}(\xi)} e^{2 \pi i (x_2 - x_1) \cdot \xi}\; d\xi\\
        &= (r_1 r_2)^{d-1} \int_{\RR^d} \widehat{\sigma}(r_1 \xi) \overline{\widehat{\sigma}(r_2 \xi)} |\widehat{\psi}(\xi)|^2 e^{2 \pi i (x_2 - x_1) \cdot \xi}\; d\xi.
    \end{align*}
    %
    Define functions $A$ and $B$ such that $B(|\xi|) = \widehat{\sigma}(\xi)$, and $A(|\xi|) = |\widehat{\psi}(\xi)|^2$. Then
    %
    \[ \langle \chi_{x_1, r_1}, \chi_{x_2, r_2} \rangle = C_d (r_1r_2)^{d-1} \int_0^\infty s^{d-1} A(s) B(r_1 s) B(r_2 s) B(|x_2 - x_1| s)\; ds. \]
    %
    Using well known asymptotics for the Fourier transform for the spherical measure, we have
    %
    \[ B(s) = s^{-(d-1)/2} \sum_{n = 0}^{N-1} (c_{n,+} e^{2 \pi i s} + c_{n,-} e^{-2 \pi i s}) s^{-n} + O_N(s^{-N}). \]
    %
    But now substituting in, assuming $A(s)$ vanishes to order $100N$ at the origin, we conclude that
    %
    \begin{align*}
        \langle \chi_{x_1 r_1}, \chi_{x_2 r_2} \rangle &= C_d \left( \frac{r_1r_2}{|x_1 - x_2|} \right)^{(d-1)/2} \sum_{n,\tau} c_{n,\tau}  r_1^{-n_1} r_2^{-n_2} |x_2 - x_1|^{-n_3}\\
        &\quad\quad\quad \Bigg\{ \int_0^\infty A(s) s^{-(d-1)/2}  s^{-n_1-n_2-n_3} e^{2 \pi i (\tau_1 r_1 + \tau_2 r_2 + \tau_3 |x_2 - x_1|) s}\; ds \Bigg\}\\
        &\lesssim_N \left( \frac{r_1r_2}{|x_1 - x_2|} \right)^{\frac{d-1}{2}} \left(1 + \frac{1}{|x_1 - x_2|^N} \right) \sum_{\tau} \left( 1 + |\tau_1 r_1 + \tau_2 r_2 + \tau_3 |x_2 - x_1|| \right)^{-5N}\\
        &\lesssim_N \left( \frac{r_1r_2}{|x_1 - x_2|} \right)^{\frac{d-1}{2}} \left(1 + \frac{1}{|x_1 - x_2|^N} \right) \sum_\tau \left( 1 + |\tau_1 \tau_3 r_1 + \tau_2 \tau_3 r_2 + |x_2 - x_1|| \right)^{-5N}.
    \end{align*}
    %
    This gives the result provided that $1 + |x_1 - x_2| \geq |r_1 - r_2| / 10$ and $|x_1 - x_2| \geq 1$. If $1 + |x_1 - x_2| \leq |r_1 - r_2| / 10$, then the supports of $\chi_{x_1,r_1}$ and $\chi_{x_2,r_2}$ are disjoint, so the inequality is trivial. On the other hand, if $|x_1 - x_2| \leq 1$, then the bound is trivial by the last sentence unless $|r_1 - r_2| \leq 10$, and in this case the inequality reduces to the simple inequality
    %
    \[ \langle \chi_{x_1,r_1}, \chi_{x_2,r_2} \rangle \lesssim_N (r_1 r_2)^{(d-1)/2}. \] 
    %
    But this follows immediately from the Cauchy-Schwartz inequality.
\end{proof}

The exponent $(d-1)/2$ in Lemma \ref{lemma4} is too weak to apply almost orthogonality directly to obtain $L^2$ bounds on $\sum_{(x,r) \in \mathcal{E}_k} \chi_{xr}$ on it's own, but together with the density decomposition assumption we will be able to obtain Lemma \ref{lemma6}.

\begin{proof}[Proof of Lemma \ref{lemma6}]
    Without loss of generality, we may assume that the $k$ such that $\mathcal{E}_k \neq \emptyset$ is $10$-separated. Write
    %
    \[ F = \sum_{(x,r) \in \mathcal{E}} \chi_{x,r} \]
    %
    and $F_k = \sum_{(x,r) \in \mathcal{E}_k} \chi_{x,r}$. First, we deal with $F_{\lesssim m} = \sum_{k \leq 10 m} F_k$ trivially, i.e. writing
    %
    \begin{align*}
        \| F \|_{L^2(\RR^d)} &\lesssim m^{1/2} \left( \sum_{k \leq 10m} \| F_k \|_{L^2(\RR^d)}^2 + \| \sum_{k > 10m} F_k \|_{L^2(\RR^d)} \right)^{1/2}.
    \end{align*}
    %
    We then decompose
    %
    \[ \| \sum_{k > 10 m} F_k \|_{L^2(\RR^d)}^2 \leq \sum_{k > 10 m} \| F_k \|_{L^2(\RR^d)}^2 + 2 \sum_{k' > k > 10m} |\langle F_k, F_{k'} \rangle|. \]
    %
    Let us analyze $\langle F_k, F_{k'} \rangle$. The term will become a sum of the form $\langle \chi_{x,r}, \chi_{y,s} \rangle$, where $r \sim 2^k$ and $s \sim 2^{k'}$. Because of our assumption of being 10-separated, we have $r \leq s / 2^{10}$. If $\langle \chi_{x,r}, \chi_{y,s} \rangle \neq 0$, then since the support of $\chi_{y,s}$ is an annulus of radius $s$ centered at $y$, with thickness $O(1)$, and $\chi_{x,r}$ has support on an annulus of radius $r$ centered at $x$, with thickness $O(1)$, the fact that $r$ is comparatively smaller than $s$ implies that $(x,r)$ must be contained in the annulus of radius $s$ centered at $y$, with thickness $O(2^k)$. Such an annulus is covered by $O( 2^{(k'-k)(d-1)} )$ balls of radius $2^k$. Each ball can only contain $2^{k + m}$ points $(x,r)$, and so there can be at most
    %
    \[ O(2^{k'(d-1)} 2^{-k(d-1)} 2^{k+m} ) = O( 2^{k'(d-1) - k(d-2) + m} ). \]
    %
    pairs $(x,r) \in \mathcal{E}_k$ for which $\langle \chi_{x,r}, \chi_{y,s} \rangle \neq 0$. For such pairs we have
    %
    \[ |\langle \chi_{x,r}, \chi_{y,s} \rangle| \lesssim \left( \frac{2^k 2^{k'}}{2^{k'}} \right)^{\frac{d-1}{2}} = 2^{\frac{k(d-1)}{2}}. \]
    %
    Thus we conclude that
    %
    \[ |\langle F_k, \chi_{y,s} \rangle| \lesssim 2^{-k ( \frac{d-3}{2} ) + k'(d-1) + m }. \]
    %
    Summing over $10m < k < k'$, we conclude that since $d \geq 4$,
    %
    \[ \sum_{10m < k < k'} |\langle F_k, \chi_{y,s} \rangle| \lesssim 2^{k'(d-1) + m} \sum_{10m < k < k'} 2^{-k \frac{d-3}{2}} \lesssim 2^{k'(d-1) + m} 2^{-5m} \lesssim 2^{k'(d-1)}. \]
    %
    But this means that
    %
    \[ \sum_{10m < k < k'} |\langle F_k, F_{k'} \rangle| \lesssim 2^{k'(d-1)} \cdot \# ( \mathcal{E}_{k'} ). \]
    %
    This means that
    %
    \[ \| \sum_{k > 10m} F_k \|_{L^2(\RR^d)}^2 \lesssim \sum_{k > 10m} \| F_k \|_{L^2(\RR^d)}^2 + \sum_{k'} 2^{k'(d-1)} \# (\mathcal{E}_{k'}), \]
    %
    and it now suffices to deal with estimates the $\| F_k \|_{L^2(\RR^d)}$, i.e. the interactions of functions supported on radii of comparable magnitude. To deal with these, we further decompose the radii, writing $[2^k,2^{k+1})$ as the disjoint union of intervals $I_{k,\mu} = [2^k + (\mu - 1) 2^{am}, 2^k + \mu 2^{am}]$, for some $a$ to be chosen later. These interval induces a decomposition $\mathcal{E}_k = \bigcup_\mu \mathcal{E}_{k,\mu}$. Again, incurring a constant loss at most, we may assume that the $\mu$ such that $\mathcal{E}_{k,\mu} \neq \emptyset$ are $10$ separated. We write $F_k = \sum F_{k,\mu}$, and we have
    %
    \[ \| F_k \|_{L^2(\RR^d)}^2 = \sum_\mu \| F_{k,\mu} \|_{L^2(\RR^d)}^2 + \sum_{\mu < \mu'} |\langle F_{k,\mu}, F_{k,\mu'} \rangle|. \]
    %
    We now consider $\chi_{x,r}$ and $\chi_{y,s}$ with $r \in I_{k,\mu}$ and $s \in I_{k',\mu}$. Then we must have $|x - y| \lesssim 2^k$ and $2^{am} \leq |r - s| \lesssim 2^k$, and so we have
    %
    \begin{align*}
        |\sum_{\mu < \mu'} \langle F_{k,\mu}, \chi_{y,s} \rangle| &\lesssim 2^{k(d-1)} \sum_{\substack{(x,r) \in \mathcal{E}_k\\ 2^{am} \leq |(x,r) - (y,s)| \lesssim 2^k}} |(x,r) - (y,s)|^{- \frac{d-1}{2}}\\
        &\lesssim 2^{k(d-1)} \sum_{am \leq l \leq k} 2^{-l(d-1)/2} \# \{ (x,r) \in \mathcal{E}_k: |(x,r) - (y,s)| \sim 2^l \}.
    \end{align*}
    %
    Using the density assumption,
    %
    \[ \# \{ (x,r) \in \mathcal{E}_k: |(x,r) - (y,s)| \sim 2^l \} \lesssim 2^{l + m} \]
    %
    and so we obtain that, again using the assumption that $d \geq 4$,
    %
    \[ |\sum_{\mu < \mu'} \langle F_{k,\mu}, \chi_{y,s} \rangle| \lesssim 2^{k(d-1)} 2^{m(1-a(d-3)/2)}. \]
    %
    Now summing over all $(y,s)$, we obtain that
    %
    \[ |\sum_{\mu < \mu'} \langle F_{k,\mu}, F_{k,\mu'} \rangle| \lesssim 2^{k(d-1)} 2^{m(1 - a(d-3)/2)} \#(\mathcal{E}_{k,\mu'}). \]
    %
    and now summing over $\mu'$ gives that
    %
    \[ \| F_k \|_{L^2(\RR^d)}^2 \lesssim \sum_\mu \| F_{k,\mu} \|_{L^2(\RR^d)}^2 + 2^{k(d-1)} 2^{m(1 - a(d-3)/2)} \# \mathcal{E}_k, \]
    %
    which is a good enough bound if we pick $a$ to be large enough. Now we are left to analyze $\| F_{k,\mu} \|_{L^2(\RR^d)}$, i.e. analyzing interactions between annuli which have radii differing from one another by at most $O(2^{am})$. Since the family of all possible radii are discrete, the set $\mathcal{R}_{k,\mu}$ of all possible radii has cardinality $O(2^{am})$. We do not really have any orthogonality to play with here, so we just apply Cauchy-Schwartz, writing $F_{k,\mu} = \sum_{r \in \mathcal{R}_{k,\mu}} F_{k,\mu,r}$, to write
    %
    \[ \| F_{k,\mu} \|_{L^2(\RR^d)}^2 \lesssim 2^{am} \sum_r \| F_{k,\mu,r} \|_{L^2(\RR^d)}^2. \]
    %
    Recall that $\chi_{x,r} = \text{Trans}_x(\sigma_r * \psi)$, where $\psi$ is a compactly supported function whose Fourier transform is non-negative and vanishes to high order at the origin. In particular, we now make the additional assumption that $\psi = \psi_{\circ} * \psi_{\circ}$ for some other compactly function $\psi_{\circ}$ whose Fourier transform is non-negative and vanishes to high order at the origin. Then we find that $F_{k,\mu,r}$ is equal to the convolution of the function
    %
    \[ A_r = \sum_{(x,r) \in \mathcal{E}} \text{Trans}_x \psi_{\circ} \]
    %
    with the function $\sigma_r * \psi_{\circ}$. Using the standard asymptotics for the Fourier transform of $\sigma_r$, i.e. that for $|\xi| \geq 1$,
    %
    \[ |\widehat{\sigma_r}(\xi)| \lesssim r^{d-1} (1 + r |\xi|)^{- \frac{d-1}{2}}, \]
    %
    and since $|\widehat{\psi_\circ}(\xi)| \lesssim_N |\xi|^N$, we get that if $r \geq 1$, then for $|\xi| \leq 1/r$,
    %
    \[ |\widehat{\sigma_r}(\xi) \widehat{\psi_\circ}(\xi)| \lesssim_N r^{d-1-N} \]
    %
    and for $|\xi| \geq 1/r$,
    %
    \[ |\widehat{\sigma_r}(\xi) \widehat{\psi_\circ}(\xi)| \lesssim_N r^{\frac{d-1}{2}} |\xi|^{-N}. \]
    %
    Thus in particular,the $L^\infty$ norm of the Fourier transform of $\sigma_r * \psi_\circ$ is $O(r^{(d-1)/2})$. Now the functions $\psi_{\circ}$ are compactly supported, so since the set of $x$ such that $(x,r) \in \mathcal{E}$ is one-separated, we find that
    %
    \[ \| A_r \|_{L^2(\RR^d)} \lesssim \# \{ x : (x,r) \in \mathcal{E} \}^{1/2}. \]
    %
    But this means that
    %
    \[ \| F_{k,\mu,r} \|_{L^2(\RR^d)} = \| A_r * (\sigma_r * \psi_{\circ}) \|_{L^2(\RR^d)} \lesssim r^{\frac{d-1}{2}} \# \{ x : (x,r) \in \mathcal{E} \}^{1/2}. \]
    %
    Thus we have that
    %
    \[ \| F_{k,\mu} \|_{L^2(\RR^d)}^2 = 2^{am} \cdot \# \mathcal{E}_{k,\mu} \cdot 2^{k(d-1)}. \]
    %
    Summing over $\mu$ gives that
    %
    \[ \| F_k \|_{L^2(\RR^d)}^2 = 2^{k(d-1)} \# \mathcal{E}_k (2^{am}  + 2^{m(1 - a(d-3)/2)}). \]
    %
    Picking $a = 2/(d-1)$ optimizes this bound, giving
    %
    \[ \| F_k \|_{L^2(\RR^d)} \lesssim 2^{m/(d-1)} 2^{k(d-1)/2} (\# \mathcal{E}_k)^{1/2}. \]
    %
    Plugging this into the estimates we got for $F$ gives the required bound.
\end{proof}

\chapter{Cladek: Improvements Using Incidence Geometry}

The results of Heo, Nazarov, and Seeger only apply when $d \geq 4$. Cladek found a method to get an initial radial multiplier conjecture result in $\RR^3$, and an improvmeent of the bounds obtained by Heo, Nazarov, and Seeger when $d = 3$. The idea is to exploit the fact that one need only prove a version of \ref{lemma2} for a set $\mathcal{E} = \mathcal{E}_X \times \mathcal{E}_R$, where $\mathcal{E}_X$ is a one-separated family of points, and $\mathcal{E}_R$ are a family of radii. One can then exploit this Cartesian product structure when analyzing functions of the form
%
\[ F = \sum_{(x,r) \in \mathcal{E}} \chi_{x,r}, \]
%
in particular, improving upon the result of \cite{HeoandNazarovandSeeger}.

\section{Result in 3 Dimensions}

As in \cite{HeoandNazarovandSeeger}, Cladek first performs a density decomposition, i.e. writing
%
\[ F = \sum F_k^m \]
%
where
%
\[ F_k^m = \sum_{(x,r) \in \mathcal{E}_k(2^m)} \chi_{x,r}. \]
%
Cladek then interpolates between an $L^0$ bound and an $L^2$ bound on the resulting functions. The $L^0$ bound is exactly the same bound used in \cite{HeoandNazarovandSeeger}.

\begin{theorem}
    For the function $F$, we have
    %
    \[ |\text{supp}(F_k^m)| \lesssim 2^{-m} 4^k \# \mathcal{E}_k \]
    %
    and thus
    %
    \[ |\text{supp}(F^m)| \lesssim \sum_k 2^{-m} 4^k \# \mathcal{E}_k. \]
\end{theorem}

The $L^2$ bound is improved upon, which is what allows us to obtain a new result in three dimensions.

\begin{lemma} \label{cladeksl2}
    Suppose $\mathcal{E} = \bigcup_k \mathcal{E}_k$ is a one-separated set, where $\mathcal{E}_k \subset \RR^d \times [2^k,2^{k+1})$ is a set of density type $(2^m, 2^k)$. Then
    %
    \[ \left\| \sum_{(x,r) \in \mathcal{E}} \chi_{x,r} \right\|_{L^2(\RR^d)} \lesssim_\varepsilon 2^{[(11/13) + \varepsilon] m} \sum_k 4^k \# \mathcal{E}_k. \]
\end{lemma}

Interpolation thus yields that for a set of density type $2^m$ as in this Lemma,
%
\[ \| \sum_{(x,r) \in \mathcal{E}} \chi_{x,r} \|_{L^p(\RR^d)} \lesssim_\varepsilon 2^{-m(1/p - 12/13 - \varepsilon)} ( \sum_k 4^k \# \mathcal{E}_k )^{1/p}. \]
%
If $1 < p < 13/12$, this sum is favorable in $m$, and may be summed without harm to prove the radial multiplier conjecture for unit scale radial multipliers in this range.

\begin{proof} [Proof of Lemma \ref{cladeksl2}]
    Write
    %
    \[ F_k = \sum_{(x,r) \in \mathcal{E}_k} \chi_{x,r}. \]
    %
    As before, we can throw away terms for $k \leq 10 m$, i.e. obtaining that
    %
    \[ \| \sum F_k \|_{L^2(\RR^d)} \lesssim m^{1/2} \left( \sum_k \| F_k \|_{L^2(\RR^d)}^2 + \sum_{10m < k < k'} |\langle F_k, F_{k'} \rangle| \right)^{1/2}. \]
    %
    Our proof thus splits into two cases: where the radii are incomparable, and where the radii are comparable.

    TODO:
\end{proof}

\section{Results in 4 Dimensions}

TODO





\chapter{Mockenhaupt, Seeger, and Sogge: Exploiting Wave-Equation Periodicity}

The main goal of the paper \emph{Local Smoothing of Fourier Integral Operators and {C}arleson-{S}j\"{o}lin Estimates} is to prove local regularity theorems for a class of Fourier integral operators in $I^\mu(Z,Y;\mathcal{C})$, where $Y$ is a manifold of dimension $n \geq 2$, and $Z$ is a manifold of dimension $n+1$, which naturally arise from the study of wave equations. A consequence of this result will be a local smoothing result for solutions to the wave equation, i.e. that if $2 < p < \infty$, then there is $\delta$ depending on $p$ and $n$, such that if $T: Y \to Y \times \RR$ is the solution operator to the wave equation, and $Y$ is a compact manifold whose geodesics are periodic, then $T$ is continuous from from $L^p_c(Y)$ to $L^p_{\alpha,\text{loc}}(Y \times \RR)$ for $\alpha \leq -(n-1)|1/2 - 1/p| + \delta$. Such a result is called local smoothing, since if we define $Tf(t,x) = T_tf(x)$, then the operator $T_t$ is, for each $t$, a Fourier integral operator of order zero, with canonical relation
%
\[ \mathcal{C}_t = \{ (x,y;\xi,\xi) : x = y + t \widehat{\xi} \}, \]
%
where $\widehat{\xi} = \xi / |\xi|$ is the normalization of $\xi$. Standard results about the regularity of hyperbolic partial differential equations show that each of the operators $T_t$ is continuous from $L^p_c(Y)$ To $L^p_{\alpha,\text{loc}}(Y \times \RR)$ for $\alpha \leq -(n-1)|1/2 - 1/p|$, and that this bound is sharp. Thus $T$ is \emph{smoothing} in the $t$ variable, so that for any $f \in L^p$, the functions $T_t f$ `on average' gain a regularity of $\delta$ over the worst case regularity at each time. The local smoothing conjecture states that this result is true for any $\delta < 1/p$.

The class of Fourier integral operators studied are those satisfying the following condition: as is standard, the canonical relation $\mathcal{C}$ is a conic Lagrangian manifold of dimension $2n + 1$. The fact that $\mathcal{C}$ is Lagrangian implies $\mathcal{C}$ is locally parameterized by $(\nabla_\zeta H(\zeta, \eta), \nabla_\eta H(\zeta, \eta),\zeta,\eta)$, where $H$ is a smooth, real homogeneous function of order one. If we assume $\mathcal{C} \to T^* Y$ is a submersion, then $D_\xi [\nabla_\eta H(\zeta,\eta)]$ has full rank, which implies $D_\eta [\nabla \xi H(\zeta, \eta)] = (D_\xi [\nabla_\eta H(\zeta, \eta)])^T$ has full rank, and thus the projection $\mathcal{C} \to T^* Z$ is an immersion. We make the further assumption that the projection $\mathcal{C} \to Z$ is a submersion, from which it follows that for each $z$ in the image of this projection, the projection of points in $\mathcal{C}$ onto $T^*_z Z$ is a conic hypersurface $\Gamma_z$ of dimension $n$. The final assumption we make is that all principal curvatures of $\Gamma_z$ are non-vanishing.

\begin{remark}
    The projection properties of $\mathcal{C}$ imply that, in $T^* (Z \times Y)$, there exists a smooth phase $\phi$ defined on an open subset of $Z \times T^* Y$, homogeneous in $T^* Y$, such that locally we can write $\mathcal{C}$ as $(z, \nabla_z \phi(z,\eta), \nabla_\eta \phi(z,\eta), \eta)$ for $\eta \neq 0$. Then, working locally on conic sets,
    %
    \[ \Gamma_z = \{ (\nabla_z \phi(z,\eta)) \}, \]
    %
    and the curvature condition becomes that the Hessian $H_{\eta \eta} \langle \nabla_z \phi, \nu \rangle$ has constant rank $n-1$, where $\nu$ is the normal vector to $\Gamma_z$. This is a natural homogeneous analogue of the Carleson-Sj\"{o}lin condition for non-homogeneous oscillatory integral operators, i.e. the Carleson-Sj\"{o}lin condition is allowed to assume $H_{\eta \eta} \phi$ has rank $n$, which cannot be possible in our case, since $\phi$ is homogeneous here. An approach using the analytic interpolation method of Stein or the Strichartz / Fractional Integral approach generalizes the Carleson-Sj\"{o}lin theorem to show that for any smooth, non-homogeneous phase function $\Phi: \RR^{n+1} \times \RR^n \to \RR$, and any compactly supported smooth amplitude $a$ on $\RR^{n+1} \times \RR^n$. Consider the operators
    %
    \[ T_\lambda f(z) = \int a(z,y) e^{2 \pi i \lambda \Phi(z,y)} f(y)\; dy. \]
    %
    If the associated canonical relation $\mathcal{C}$, if $\mathcal{C}$ projects submersively onto $T^* \RR^n$, so that for each $z \in \RR^{n+1}$ in the image of the projection map $\mathcal{C}$, the set $S_z \subset \RR^{n+1}$ obtained from the inverse image of the projection of $\mathcal{C} \to Z$ at $z$ is a $n$ dimensional hypersurface with $k$ non-vanishing curvatures. Then for $1 \leq p \leq 2$,
    %
    \[ \| T_\lambda f \|_{L^q(\RR^{n+1})} \lesssim \lambda^{-(n+1)/q} \| f \|_{L^p(\RR^n)}. \]
    %
    where $q = p^*(1 + 2/k)$.
\end{remark}

\begin{remark}
    We can also see these assumptions as analogues in the framework of cinematic curvature, splitting the $z$ coordinates into `time-like' and 'space-like' parts. Working locally, because $\mathcal{C} \to T^* Y$ is a submersion, we can consider coordinates $z = (x,t)$ so that, with the phase $\phi$ introduced above, $D_x (\nabla_\eta \phi)$ has full rank $n$, and that $\partial_t \phi(x,t,\eta) \neq 0$. Then for each $z = (x,t)$, we can locally write $\partial_t \phi(x,t,\eta) = q(x,t,\nabla_x \phi(x,t,\eta))$, homogeneous in $\eta$, and then
    %
    \[ \mathcal{C} = \{ (x,t,y;\xi,\tau,\eta) : (x,\xi) = \chi_t(y,\eta), \tau = q(x,t,\xi) \}, \]
    %
    where $\chi_t$ is a canonical transformation. Our curvature conditions becomes that $H_{\xi \xi} q$ has full rank $n-1$. This is the cinematic curvature condition introduced by Sogge. %TODO: READ SOGGE, PROPOGATION OF SINGULARITIES AND MAXIMAL FUNCTIONS IN THE PLANE, WHICH INTRODUCES CINEMATIC CURVATURE?
\end{remark}

Under these assumptions, the paper proves that any Fourier integral operator $T$ in $I^{\mu - 1/4}(Z,Y;\mathcal{C})$ maps $L^2_c(Y)$ to $L^q_{\text{loc}}(Z)$ if
%
\[ 2 \left( \frac{n+1}{n-1} \right) \leq q < \infty \quad\text{and}\quad \mu \leq - n (1/2 - 1/q) + 1/q. \]
%
and maps $L^p_c(Y)$ to $L^p_{\text{loc}}(Z)$ if
%
\[ p > 2 \quad\text{and}\quad \mu \leq -(n-1)(1/2 - 1/p) + \delta(p,n). \]
%
If we introduce time and space variables locally as in the remark above, any operator in $I^{\mu - 1/4}(Z,Y;\mathcal{C})$ can be written locally as a finite sum of operators of the form
%
\[ Tf(x) = \int_{-\infty}^\infty T_t f(x), \]
%
where
%
\[ T_t f(x) = \int a(t,x,\eta) e^{2 \pi i \phi(x,t,y,\eta)} f(y)\; dy\; d\eta. \]
%
is a Fourier integral operator whose canonical relation is a locally a canonical graph, then the general theory implies that each of the maps $T_t$ maps $L^2_c(Y)$ to $L^q_{\text{loc}}(X)$ if
%
\[ 2 \leq q \leq \infty \quad\text{and}\quad \mu \leq -n(1/2 - 1/q) \]
%
so that here we get local smoothing of order $1/q$, and also maps $L^p_c(Y)$ to $L^p_{\text{loc}}(X)$ if
%
\[ 1 < p < \infty \quad\text{and}\quad \mu \leq -(n-1)|1/p - 1/2| \]
%
so we get $\delta(p,n)$ smoothing. A consequence of the smoothing, via Sobolev embedding, is a maximal theorem result for the operator $T_t$, i.e. that for any finite interval $I$, the operator
%
\[ Mf = \sup_{t \in I} |T_t f| \]
%
maps $L^p_c(Y)$ to $L^p_{\text{loc}}(X)$ if $\mu < -(n-1)(1/2 - 1/p) - (1/p - \delta(p,n))$. If the local smoothing conjecture held, we would conclude that, except at the endpoint $T^*$ has the same $L^p_c(Y)$ to $L^p_{\text{loc}}(X)$ mapping properties as each of the operators $T_t$. We also get square function estimates, such that for any finite interval $I$, if we consider
%
\[ Sf(x) = \left( \int_I |T_t f(x)|^2\; dt \right)^{1/2}, \]
%
then for
%
\[ 2 \frac{n+1}{n-1} \leq q < \infty \quad\text{and}\quad \mu \leq -n(1/2 - 1/q) + 1/2, \]
%
the operator $S$ is bounded from $L^2_c(Y)$ to $L^q_{\text{loc}}(X)$.

Our main reason to focus on this paper is the results of the latter half of the paper applying these techniques to radial multipliers on compact manifolds with periodic geodesics. Thus we consider a compact Riemannian manifold $M$, such that the geodesic flow is periodic with minimal period $2 \pi \cdot \Pi$. We consider $m \in L^\infty(\RR)$, such that $\sup_{s > 0} \| \beta \cdot \text{Dil}_s m \|_{L^2_\alpha(\RR)} = A_\alpha$ is finite for some $\alpha > 1/2$ and some $\beta \in C_c^\infty(\RR)$. We define a `radial multiplier' operator
%
\[ Tf = \sum_\lambda m(\lambda) E_\lambda f \]
%
where $E_\lambda$ is the projection of $f$ onto the space of eigenfunctions for the operator $\sqrt{-\Delta}$ on $M$ with eigenvalue $\lambda$. We can also write this operator as $m(\sqrt{-\Delta})$. Then the wave propogation operator $e^{2 \pi i t \sqrt{-\Delta}}$ is periodic of period $\Pi$. The Weyl formula tells us that the number of eigenvalues of $\sqrt{-\Delta}$ which are smaller than $\lambda$ is equal to $V(M) \cdot \lambda^n + O(\lambda^{n-1})$.

\begin{theorem}
    Let $m \in L^2_\alpha(\RR)$ be supported on $(1,2)$, and assume $\alpha > 1/2$, then for $2 \leq p \leq 4$, $f \in L^p(M)$, and for any integer $k$,
    %
    \[ \left\| \sup_{2^k \leq \tau \leq 2^{k+1}} |\text{Dil}_\tau m(\sqrt{-\Delta}) f| \right\|_{L^p(M)} \lesssim_\alpha \| m \|_{L^2_\alpha(M)} \| f \|_{L^p(M)}. \]
\end{theorem}
\begin{proof}
    To understand the radial multipliers we apply the Fourier transform, writing
    %
    \[ T_\tau f = (\text{Dil}_\tau m)(\sqrt{-\Delta}) f = m(\sqrt{-\Delta} / \tau) f = \int_{-\infty}^\infty \tau \widehat{m}(t \tau) e^{2 \pi i t \sqrt{-\Delta}} f\; dt. \]
    %
    If we define $\beta \in C_c^\infty((1/2,8))$ such that $\beta(s) = 1$ for $1 \leq s \leq 4$, and set $L_k f = \text{Dil}_{2^k} \beta(\sqrt{-\Delta}) f$, then for $2^k \leq \tau \leq 2^{k+1}$
    %
    \[ T_\tau f = (\text{Dil}_\tau m)(\sqrt{-\Delta}) f = (\text{Dil}_\tau m \cdot \text{Dil}_{2^k} \beta)(\sqrt{-\Delta}) = T_\tau L_k f. \]
    %
    so Cauchy-Schwartz implies that
    %
    \begin{align*}
        |T_\tau f(x)| &= \left| \int_{-\infty}^\infty \tau \widehat{m}(\tau) e^{2 \pi i t \sqrt{-\Delta}} L_k f(x)\; dt \right|\\
        &\leq \| m \|_{L^2_\alpha(M)} \left( \int_{-\infty}^\infty \frac{\tau}{(1 + |t \tau|^2)^\alpha} |e^{2 \pi i t \sqrt{-\Delta}} L_k f(x)|^2 \right)^{1/2}\\
        &\leq \| m \|_{L^2_\alpha(M)} \left( \int_{-\infty}^\infty \frac{2^k}{(1 + |2^k t|^2)^\alpha} |e^{2 \pi i t \sqrt{-\Delta}} L_k f(x)|^2 \right)^{1/2}
    \end{align*}
    %
    Because of periodicity, if we set $w_k(t) = 2^k / (1 + |2^k t|^2)^\alpha$, it suffices to prove that for $\alpha > 1/2$,
    %
    \[ \left\| \left( \int_0^\Pi w_k(t) |e^{2 \pi i t \sqrt{-\Delta}} L_k f(x)|^2\; dt \right)^{1/2} \right\|_{L^p(M)} \lesssim_{\alpha,p} \| f \|_{L^p(M)}. \]
    %
    This is a weighted combination of the wave propogators, roughly speaking, assigning weight $2^k$ for $t \lesssim 1/2^k$, and assigning weight $1/t$ to values $t \gtrsim 1/2^k$.

    For a fixed $0 < \delta$, we can split this using a partition of unity into a region where $t \gtrsim \delta$ and a region where $t \lesssim \delta$, where $\delta$ is independent of $k$. For each $t$, the wave propogation $e^{2 \pi i t \sqrt{-\Delta}}$ is a Fourier integral operator of order zero (we have an explicit formula for small $t$, and the composition calculus for Fourier integral operators can then be used to give a representation of the propogation operators for all times $t$, such that the symbols of these operators are locally uniformly bounded in $S^0$). Thus the square function estimate above can be applied in the region where $t \gtrsim \delta$, because the weighted square integral above has weight $O_\delta(1)$ uniformly in $k$.

    Next, we move onto the region $t \lesssim 1/2^k$. The symbol of the operator $e^{2 \pi i t \sqrt{-\Delta}}$

    Finally we move onto the region $1/2^k \lesssim t \lesssim \delta$. On this region we have $w_k(t) \sim 1/t$, which hints we should try using dyadic estimates. In particular, suppose that for $\gamma \leq \delta$, we have a family of dyadic estimates of the form
    %
    \[ \left\| \left( \int_\gamma^{2\gamma} |e^{2 \pi i t \sqrt{-\Delta}} L_k f|^2\; dt \right)^{1/2} \right\|_{L^p(M)} \lesssim \gamma^{1/2} (1 + \gamma 2^k)^\varepsilon \cdot \| f \|_{L^p(M)}. \]
    %
    Summing over the $O(k)$ dyadic numbers between $1/2^k$ and $\delta$ gives
    %
    \[ \left\| \left( \int_{1/2^k \lesssim t \lesssim \delta} |e^{2 \pi i t \sqrt{-\Delta}} L_k f|^2\; \frac{dt}{t} \right)^{1/2} \right\|_{L^p(M)} \lesssim 2^{\varepsilon k} \| f \|_{L^p(M)} \]






    If we were able to obtain this inequality for some $\varepsilon > 0$, then we could bound


     that for all $0 < \gamma < \Pi/2$


    If we localize near $t \lesssim 1/2^k$ by multiplying by $\phi(2^k t)$ for some compactly supported smooth $\phi$ supported on $|t| \lesssim 1$, then for $t$ on the support of $\phi(2^k t)$ we have a weight proportional to $2^k$, and rescaling shows that it suffices to bound the quantities
    %
    \[ \left\| \left( \int \phi(t) |e^{2 \pi i (t/2^k) \sqrt{-\Delta}} L_k f(x)|^2\; dt \right)^{1/2} \right\| \]

     the family of functions
    %
    \[ \left\| \left( \int |\phi(t) e^{2 \pi i (t / 2^k) \sqrt{-\Delta}} L_k f(x)|^2\; Dt \right)^{1/2} \right\|_{L^p_x} \lesssim \sup \| e^{2 \pi i (t / 2^k) \sqrt{-\Delta}} L_k f \|_{L^p_x} \]


    $a_k(t) = 2^{-k/2} \widehat{\phi}(t/2^k) \beta(\tau/2^k)$

    it suffices to uniformly bound quantities of the form
    %
    \[ \left\| \left( \int 2^k \phi(2^k t) |e^{2 \pi i \sqrt{-\Delta}} L_k f(x)|^2\; dt \right)^{1/2} \right\|_{L^p(M)} \lesssim_{\alpha,p} \| f \|_{L^p(M)} \]
    %
    We now apply a dyadic decomposition to deal with the smaller values of $t$. Let us assume for simplicity of notation that $\delta < 1$, and then consider a partition of unity $1 = \sum_{j = 1}^\infty \phi(2^j t)$ for $0 \leq t \leq 1$, and such that $\phi$ is localized near $1/4 \leq t \leq 2$, then our goal is to bound the quantities
    %
    \[ \left\| \left( \int_{-\infty}^\infty \phi(2^j t) \frac{2^k}{(1 + |2^k t|^2)^\alpha} |A_t L_k f(x)|^2\; dt \right)^{1/2} \right\|_{L^p(M)}, \]
    %
    which are each proportional to
    \[ s \]
\end{proof}







\chapter{Lee and Seeger: Decomposition Arguments For Estimating Fourier Integral Operators}

Let's now discuss a paper \cite{LeeSeeger} entitled \emph{Lebesgue Space Estimates For a Class of Fourier Integral Operators Associated With Wave Propogation}. In this paper, Lee and Seeger prove a variable coefficient version of the result of Heo, Nazarov, and Seeger, i.e. generalizing that result as it applies to sharp local smoothing on $\RR^d$ to the local smoothing of Fourier integral operators satisfying the cinematic curvature condition.

We consider a localized Fourier integral operator $T: \mathcal{D}(Y) \to \mathcal{D}^*(Z)$ of order $\mu - 1/4$, where $\dim(Y) = d$ and $\dim(Z) = d + 1$, with a canonical relation $\mathcal{C}$ (which must be a $2d + 1$ dimensional submanifold of $T^* Z \times T^* Y$ by virtue of the fact it is Lagrangian), satisfying the following properties:
%
\begin{itemize}
    \item The projection map $\pi_{T^* Y}: \mathcal{C} \to T^* Y$ is a submersion. It follows that around any point $(z_0,y_0;\zeta_0,\eta_0)$ we can choose coordinate systems $y$ on $Y$ and $z = (x,t)$ on $Z$ centered at $z_0$ and $y_0$ such that $\zeta_0 = dx_1$, $\eta_0 = dy_1$, and the tangent plane to $\mathcal{C}$ at this point is given by
    %
    \[ dx = dy \quad\text{and}\quad d\xi = d\eta \quad\text{and}\quad d\tau = 0. \]
    %
    In particular, it follows that $\pi_Z : \mathcal{C} \to Z$ is a submersion, and we can locally find a function $\phi(z,\eta)$, homogeneous in $\eta$, such that, locally,
    %
    \[ \mathcal{C} = \{ (z, \nabla_\eta \phi(z,\eta) ; \nabla_z \phi(z,\eta), \eta) \}. \]
    %
    By assumption on the tangent space of $\mathcal{C}$,
    %
    \[ \nabla_\eta \phi(0,e_1) = 0 \quad\text{and}\quad \nabla_z \phi(0,e_1) = e_1. \]
    %
    The equivalence of phase theorem implies we can find a symbol $a(x,t,y,\eta)$ of order $\mu$ such that, after appropriately localizing the operator $T$, we have
    %
    \[ Tf(x,t) = \int a(x,t,y,\eta) e^{2 \pi i [\phi(x,t,\eta) - y \cdot \eta]} f(y)\; d \eta\; dy. \]

    \item The last assumption implies that for each $z_0$, $\Sigma_{z_0} \pi_Z^{-1}(z_0)$ is a $d$ dimensional submanifold of $\mathcal{C}$. Moreover, our choice of coordinates makes it easy to see that the natural map $\Sigma_{z_0} \to T^*_{z_0} Z$ is an immersion, whose image is the immersed hypersurface $\Gamma_{z_0}$ of $T^*_{z_0}$. Indeed, the tangent plane to $\Sigma_{z_0}$ at the point above is given in coordinates by
    %
    \[ dx = dy = dt = d\tau = 0 \quad\text{and}\quad d\xi = d\eta. \]
    %
    And this is projected injectively to the plane defined by $d\tau = 0$ in $T^*_{z_0} Z$. Our other assumption we make about $\mathcal{C}$ is an assumption on \emph{cinematic curvature}. We assume that for each $z_0$, the hypersurface $\Sigma_{z_0}$ is a cone with $l$ nonvanishing principal curvatures, for some $1 \leq l \leq d-1$. Since
    %
    \[ \Sigma_{z_0} = \{ (z_0; \nabla_z \phi(z_0,\eta_0) \}. \]
    %
    The projection assumptions imply that the $(d+1) \times d$ matrix $D_\eta \nabla_z \phi$ has full rank, and the curvature assumptions imply that the Hessian matrix $H_\eta \{ \partial \phi / \partial t \}$ has rank at least $l$ in a neighborhood of our initial point.
\end{itemize}
%
Given these assumptions, the following result is obtained.

\begin{theorem}
    If
    %
    \[ l \geq 3 \quad\text{and}\quad \frac{2l}{l-2} < q < \infty \quad\text{and}\quad \mu \leq \frac{d}{q} - \frac{d-1}{2}, \]
    %
    then $T$ maps $L^q(Y)$ into $L^q(Z)$.
\end{theorem}

If we take $l = d-1$, we get the full assumption of `cinematic curvature' and we can use this to get results about local smoothing of the wave equation on compact Riemannian manifolds, which recovers the local smoothing result of Heo, Nazarov, and Seeger obtained in their paper on radial Fourier multipliers.

\begin{theorem}
    Consider a finite interval $I$, as well as
    %
    \[ d \geq 4 \quad\text{and}\quad \frac{2(d-1)}{d - 3} < q < \infty. \]
    %
    If $M$ is a compact Riemannian manifold, and $\alpha = (d-1)/2 - d/q$, then
    %
    \[ \| e^{it \sqrt{-\Delta}} f \|_{L^q_t(I) L^q_x(M)} \lesssim_I \| f \|_{L^q_\alpha(M)}. \]
\end{theorem}
\begin{proof}
    For any compact time interval $I$, the Lax parametrix construction allows one, for any suitably small coordinate system, to find a phase function
    %
    \[ \phi(x,y,\xi) \approx (x - y) \cdot \xi \]
    %
    and a symbol $a$ of order zero such that
    %
    \[ \text{supp}(a) \subset \{ (x,t,y,\eta) : |x - y| \lesssim 1\ \text{and}\ |\eta| \gtrsim 1 \}, \]
    %
    such that if $\Phi(t,x,y,\eta) = \phi(x,y,\eta) + t |\eta|_g$, then, modulo smoothing operators, for $|t| \lesssim 1$ we have
    %
    \[ (e^{it \sqrt{-\Delta}} f)(x) = \int a(x,t,y,\eta) e^{2 \pi i \Phi(t,x,y,\eta)} f(y)\; d\eta\; dy. \]
    %
    Now define
    %
    \[ Tf(x,t) = \int a(x,t,y,\xi) e^{2 \pi i \Phi(t,x,y,\xi)} f(y)\; d\xi\; dy, \]
    %
    and set
    %
    \[ Sf = T \{ (1-\Delta)^{-\alpha/2} f \}. \]
    %
    Then 
    %
    \[ Sf(x) = \int \left[ a(x,t,y',\eta) (1 + |\xi|^2)^{-\alpha / 2} \right] e^{2 \pi i (\Phi(t,x,y',\eta) + \xi \cdot (y' - y))} f(y)\; d\eta\; dy'\; dy. \]
    %
    % Symbol of order - alpha
    % integrating over N = 2d variables
    % Kernel is defined on n = d + (d+1) = 2d + 1 dimensional space.
    % (mu-1/4) - N/2 + n/4 = -alpha
    % (mu-1/4) - d + (2d + 1)/4 = -alpha
    % (mu-1/4) = (2d - 1)/4 - alpha
    % mu = d/2 - alpha
    Then $S$ is a Fourier integral operator of order $\mu - 1/4$, where $\mu = - \alpha$. The canonical relation of $S$ (as with $T$) is
    %
    \[ \mathcal{C} = \{ (x,t,y; \eta, \omega, \eta) : x = \exp_y(t \xi / |\xi|)\ \text{and}\ \omega = |\xi|_g \}. \]
    %
    One immediately sees that the projection condition is satisfied, and if we are working on a coordinate system localized smaller than the injectivity radius of $M$, for each $z_0 = (x_0,t_0)$,
    %
    \[ \Gamma_{z_0} = \{ (\xi, \omega) : \omega = |\xi|_g \} \]
    %
    is a spherical cone, and thus has $d-1$ nonvanishing principal curvatures. Applying the result, we see that for
    %
    \[ d \geq 4 \quad\text{and}\quad \frac{2(d-1)}{d-3} < q < \infty \quad\text{and}\quad \alpha \geq \frac{d-1}{2} - \frac{d}{q}. \]
    %
    the operator $S$ maps $L^q(M)$ into $L^q(M \times \RR)$, which is equivalent to $T$ mapping $L^q_\alpha(M)$ into $L^q(M \times \RR)$.
\end{proof}

\section{Frequency Localization and Discretization}

Let us describe the idea of the proof. Let $K(z,y)$ denote the kernel of $T$, i.e.
%
\[ K(x,t,y) = \int a(x,t,y,\eta) e^{2 \pi i [\phi(x,t,\eta) - y \cdot \eta]}\; d\eta. \]
%
Without loss of generality, we may assume that $a$ is supported on $|\eta| \geq 1$, since integrals over small frequencies give a smoothing operator. We perform a frequency decomposition, writing
%
\[ K(z,y) = \sum_{i = 1}^\infty 2^{i \mu} K_i(z,y) \]
%
where
%
\[ K_i(z,y) = \int \chi_i(z,y,2^{-i} \eta) e^{2 \pi i [\phi(x,t,\eta) - y \cdot \eta]}\; d\eta \]
%
for some family of functions $\{ \chi_i \}$ supported on a common compact subset of $Z \times Y \times \Xi$, and satisfy estimates of the form $|\nabla^N \chi_i| \lesssim_N 1$, for all $N \geq 0$, uniformly in $i$. We can set
%
\[ \chi_i(z,y,\eta) = 2^{-i \mu} a(z,y, 2^i \eta) \beta(\eta), \]
%
since then $\chi_i$ is supported on $|\eta| \sim 1$ and by virtue of the fact $a$ is a symbol, we have the required estimates. By performing another decomposition, we may assume $\Xi$ is an arbitrarily small neighborhood of $e_1$, such that for $z \in Z$ and $\eta \in \Xi$,
%
\[ \nabla_z \phi(z,\eta) \approx e_1 \quad\text{and}\quad D_z \nabla_\eta \phi(z,\eta) \approx \begin{pmatrix} I_d \\ 0 \end{pmatrix} \]
%
and $H_\eta \{ \partial \phi / \partial t \}$ has rank at least $l$. In this section, we analyze each of these operators separately. If we write $T_i$ for the operator with kernel $K_i$, then here we will prove that
%
\[ \| T_i f \|_{L^p} \lesssim 2^{i \left( \frac{d-1}{2} - \frac{d}{q} \right)} \| f \|_{L^p}. \]
%
for $q > 2l/(l-2)$. In Seeger, Sogge, and Stein, it is proved that
%
\[ \| T_i f \|_{L^\infty} \lesssim 2^{i \frac{d-1}{2}} \| f \|_{L^\infty}. \]
%
By interpolation, it thus suffices to prove a restricted weak type inequality of the form
%
\[ \| T_i \chi_E \|_{L^{q_l,\infty}} \lesssim 2^{i(d/l - 1/2)} |E|^{1/q_l} \]
%
where
%
\[ q_l = 2 + 4/(l-2). \]
%
By duality, it suffices to show that for
%
\[ p_l = 2 - 4/(l+2), \]
%
we have
%
\[ \| T_i^* \chi_E \|_{L^{p_l,\infty}} \lesssim 2^{i(2d-l)/2l} |E|^{1/p_l}, \]
%
which is equivalent to prove that for $t > 0$, the measure of the set
%
\[ \{ y \in M : |T_i^* \chi_E(y)| \geq t \} \]
%
is bounded by $O(t^{-p_l} 2^{i(2d-l)/(l+2)} |E| )$. The operator $T_i^*$ has kernel
%
\[ K_i^*(y,z) = \overline{K_i(z,y)} = \int \chi_i(z,y, 2^{-i} \eta) e^{2 \pi i (y \cdot \eta - \phi(z,\eta))}\; dz \]
%
so we still have a Fourier integral operator, but with a reversed canonical relation. We will obtain these bounds by proving an analogous discretized result at a scale $1/2^i$.

We consider $\mathcal{Z}_i = 2^{-i} \ZZ^{d+1} \cap [-\varepsilon^2,\varepsilon^2]^{d+1}$, for some small constant $\varepsilon > 0$. For each $\mathfrak{z} \in \mathcal{Z}_i$ we consider a function $a_{i,\mathfrak{z}}$ supported on frequencies $|\eta| \sim 2^i$ which make an angle $O(\varepsilon^2)$ with $e_1$, so that
%
\[ |\partial_\eta^\alpha a_{i,\mathfrak{z}}(\eta)| \leq 2^{-i |\alpha|} \]
%
for $|\alpha| \lesssim 1$. Set
%
\[ S_{i,\mathfrak{z}}(y) = \int a_{i,\mathfrak{z}}(\eta) e^{2 \pi i (y \cdot \eta - \phi(\mathfrak{z},\eta))}\; d\eta. \]
%
Our job is to understand the sums $\sum S_{i,\mathfrak{z}}$.

\begin{lemma}
    For each $\mathcal{E} \subset \mathcal{Z}_i$, the measure of the set of $y$ such that
    %
    \[ | \sum_{\mathfrak{z} \in \mathfrak{E}} S_{i,\mathfrak{z}}(y) | \geq t \]
    %
    is
    %
    \[ \lesssim 2^{i \frac{dl - 2}{(l+2)}} t^{- p_l} \cdot \#(\mathcal{E}). \]
\end{lemma}

How does one reduce our problem to this setting? First, suppose we assume that
%
\[ \chi_i(z,y,2^{-i} \xi) = \eta_i(z, 2^{-i} \xi) \cdot \chi_\circ(y) \]
%
for some $\chi_0 \in C_c^\infty(\RR^d)$ supported on a small neighborhood of the origin, and where $\eta_i$ is supported on a set of diameter $O(\varepsilon^2)$ near $(z,\xi) = (0,e_1)$ and with uniformly bounded derivatives in $i$ up to a suitably high order. Then one may set
%
\[ a_{i,\mathfrak{z}}(\xi) = 2^{(i+1)d} \int_{Q_{\mathfrak{z}}} \eta_i(z, 2^{-i} \xi) e^{2 \pi i (\phi(\mathfrak{z},\xi) - \phi(z,\xi))}. \]
%
The phase function has derivatives $O(2^{-i})$, which gives the required results. To get a general result, we apply a Fourier series, writing
%
\[ \chi_i(z,y,2^{-i} \xi) = \sum_{\nu \in \ZZ^d} c_{k,\nu} \eta_{i,\nu}(z, 2^{-i} \xi) e^{2 \pi i y \cdot \eta} \]
%
where the constants, and the derivatives of $\eta_{i,\nu}$, are rapidly decaying in $\nu$.

\section{$L^1$ Estimates}

To understand the individual pieces $S_z$, we consider a maximal $2^{-i/2}$ separated set $\Theta$ covering the unit sphere, and perform a further decomposition
%
\[ a_z(\eta) = \sum_{\theta \in \Theta} a_{z,\theta}(\eta), \]
%
where $a_{z,\theta}$ is supported in a cone with aperture $O(2^{-i/2})$ centered at $\theta$. Then $a_{z,\theta}$ is roughly speaking, supported on a set with length $2^{i/2}$ tangent to the radial direction, and with length $2^i$ in the radial direction. Thus differentiating in the radial direction no longer leads to quite as good derivative estimates, namely, if $u_1,\dots,u_{M}$ are unit vectors tangent to $\theta$, and $M + N \lesssim 1$, then
%
\[ (\theta \cdot \nabla_\eta)^N \prod_{i = 1}^M (u_i \cdot \nabla_\eta) \{ a_{z,\theta} \} \lesssim 2^{-k N - k M / 2}. \]
%
The decomposition of $a_z$ of course leads to a decomposition $S_z = \sum S_{z,\theta}$.

Now because each component of $\nabla_\eta \phi$ is homogeneous of degree $0$, Euler's homogeneous function theorem says that
%
\[ H_\eta \phi(x,\eta) \cdot \eta = 0. \]
%
Integration by parts (TODO: How? Also is there a typo?) yields that
%
\[ |S_{z,\theta}(y)| \lesssim 2^{i \frac{d+1}{2}} \left( 1 + 2^i |(\nabla_\xi \phi(z,\theta) - y) \cdot \theta| + 2^{k/2} | \Pi_{\theta^\perp}( \nabla_\xi \phi(z,\theta) - y ) | \right)^{-O(1)}. \]
%\begin{align*}
%    S_{z,\theta} = \frac{1}{2 \pi i} \int \frac{a_{z,\theta}(\xi)}{|y - \nabla_\xi \phi(z,\xi)|^2} \left[ (y - \nabla_\xi \phi(z,\xi)) \cdot \nabla_\xi e^{2 \pi i (y \cdot \xi - \phi(z,\xi))} \right]\; d\xi\\
%    &= 
%\end{align*}
%
Roughly speaking, this inequality says that, roughly speaking, $S_{z,\theta}$ has magnitude $2^{i \frac{d+1}{2}}$, and is concentrated on a tube centered at $\nabla_\xi \phi(z,\theta)$, with thickness $2^{-i}$ in the radial direction, and thickness $2^{-i/2}$ in the tangential direction to $\theta$. In particular, we find that
%
\[ \| S_{z,\theta} \|_{L^1} \lesssim 1. \]
%
The triangle inequality (probably the best we can do in general in the $L^1$ setting) implies that
%
\[ \| S_z \|_{L^1} \lesssim 2^{i(d-1)/2}. \]
%
This is the bound we will use in $L^1$.

To get more interesting bounds in other $L^p$ spaces, we look at the orthogonality of the functions $\{ S_z \}$. On the Fourier side of things, we have
%
\[ \widehat{S_z}(\eta) = a_z(\eta) e^{- 2 \pi i \phi(z,\eta)}. \]
%
Thus by Parsevel, we have
%
\[ \langle S_z, S_w \rangle = \langle \widehat{S_z}, \widehat{S_w} \rangle = \int a_z(\eta) a_w(\eta) e^{2 \pi i [ \phi(z,\eta) - \phi(w,\eta) ]}. \]
%
TODO: Expand on rest of argument.

\section{Adapting the Argument to Fourier Multipliers}

Let $T = m(-\sqrt{\Delta})$ be a radial multiplier on $\RR^n$, i.e. such that
%
\[ T f(x) = \int m(|\xi|) e^{2 \pi i \xi \cdot (x - y)} f(y)\; d\xi\; dy. \]
%
If $m$ is a symbol, then we can interpret $T$ directly as a Pseudodifferential Operator. But Heo, Nazarov, and Seeger's result discuss families of multipliers $m$ that are not even necessarily smooth, but do satisfy certain integrability conditions. To fix this, we assume a priori that we have applyied a decomposition argument, so we may assume $m$ is compactly supported away from the origin. Then (by Paley-Wiener) $\widehat{m}$ is a smooth symbol of some finite order satisfying some integrability properties, which indicates how we might apply the theory of Fourier integral operators, i.e. by taking the Fourier transform of $m$, we get that
%
\[ Tf(x) = \int \widehat{m}(\rho) e^{2 \pi i [\rho |\xi| + \xi \cdot (x-y)]} f(y)\; d\rho\; d\xi\; dy. \]
%
This is `almost' a Fourier integral operator, except the phase is not smooth unless $\widehat{m}$ is supported away from the origin (fixed by a decomposition argument), and the phase is non-homogeneous. To fix the non-homogeneity, we just isolate the operator in $\rho$, writing
%
\[ Tf(x) = \int_{-\infty}^\infty \widehat{m}(\rho) T_\rho f(x)\; d\rho, \]
%
where
%
\[ T_\rho f(x) = e^{2 \pi i \rho \sqrt{-\Delta}} f(x) = \int e^{2 \pi i [\rho |\xi| + \xi \cdot (x - y)]} f(y)\; d\xi\; dy \]
%
is the propogation operator for the half-wave equation $\partial_t u = \sqrt{-\Delta} \cdot u$. It has phase $\phi(x,y,\xi) = \rho |\xi| + \xi \cdot (x - y)$, and thus we have a stationary frequency value when $x = y - \rho \widehat{\xi}$, where $\widehat{\xi} = \xi / |\xi|$ is the normalization of $\xi$. This has canonical relation











\chapter{Beltran, Hickman, and Sogge: Decoupling for Fourier Integral Operators}

The paper we now discuss extends the theory of decoupling, which was originally used to establish local smoothing for the wave equation on Euclidean space, to the setting of more general FIOs. Here we attempt to study $L^p$ to $L^p$ estimates  for Fourier integral operators given by
%
\[ Tf(x,t) = \int_{\RR^d} e^{2 \pi i \phi(x,t;\xi)} b(x,t;\xi) (1 + |\xi|^2)^{\mu/2} \widehat{f}(\xi)\; d\xi \]
%
where $b$ is a compactly supported symbol of order zero, compactly supported in $x$ and $t$, and $\phi$ is a phase function, homogeneous of degree one in the $\xi$ variable. We let
%
\[ K(x,t;y) = \int e^{2 \pi i [\phi(x,t;\xi) - y \cdot \xi]} b(x,t;\xi) (1 + |\xi|^2)^{\mu/2}\; d\xi \]
%
denote the kernel. Since $\nabla_\xi \phi(x,t;\xi)$ is homogeneous of degree zero in $\phi$, the sets
%
\[ \Sigma_{(x,t)} = \{ \nabla_\xi \phi(x,t,\xi) : \xi \in \RR^n \} \subset \RR^n_y \]
%
are usually manifolds of dimension $n-1$. They are related to the singular support of $K$.

%Let us for simplicity assume the phase function is nondegenerate, and also, that the resulting Lagrangian manifold $\Sigma \subset T^*(\RR^n_x \times \RR_t \times \RR^n_y)$, the natural projection maps $\Sigma \to T^* \RR^n_y$ and $\Sigma \to \RR^n_{x,t}$ are submersions. It follows that for an open set of $x$ and $t$ we can find a hypersurface $\Sigma_{x,t}$ in the cotangent space of $(x,t)$ upon which the operator behaves

To study the $L^p$ behaviour of $T$, we break up the behaviour of the operator dyadically in the $\xi$ variable, thus setting
%
\[ T = T_{\leq 1} + \sum_{n = 1}^\infty T_n, \]
%
where, for a given $\lambda > 0$, we let $T^\lambda$ be an operator with kernel $K^\lambda$ given by
%
\[ K^\lambda(x,t;\xi) = \int e^{2 \pi i [\phi(x,t;\xi) - y \cdot \xi]} b(x,t;\xi) (1 + |\xi|^2)^{\mu/2} \beta(\xi / \lambda)\; d\xi. \]
%
It can be verified that $T_{\leq 1}$ is a pseudo-differential operator of order 0, and is therefore bound on $L^p$ for all $1 < p < \infty$. It therefore suffices to show that as $\lambda \to \infty$,
%
\[ \| T^\lambda f \|_{L^p(\RR^d)} \lesssim \lambda^{- \varepsilon} \| f \|_{L^p(\RR^d)} \]
%
so that we may sum in $n$ in the expansion of $T$ via the triangle inequality to obtain an $L^p$ bound for the original operator.

For large $\lambda$, the principle of stationary phase tells us we should expect $K^\lambda$ to be concentrated in the set
%
\[ \{ (x,t;y) : |\nabla_\xi \phi(x,t;\xi) - y| \leq 1/\lambda\ \text{for some $\xi$} \}, \]
%
since the phase oscillates to a significant degree for $|\nabla \phi(x,t;\xi) - y| \gtrsim 1/\lambda$, roughly a $1/\lambda$ neighborhood of the singular support of $K$. Also we have $\| K^\lambda \|_{L^\infty_x L^\infty_y} \lesssim \lambda^{\mu + d}$ trivially by taking in absolute values. This gives the crude estimate that $\| K_n \|_{L^\infty_x L^1_y} \lesssim \lambda^{\mu + d - 1}$. Thus we obtain by Schur's Lemma that
%
\[ \| T^\lambda f \|_{L^1(\RR^{d+1})} \lesssim \lambda^{\mu + d - 1} \| f \|_{L^1(\RR^d)}. \]
%
We will get a much better bound by a more sophisticated decomposition of the kernels $\{ K^\lambda \}$.

For a given $\lambda$, let $\{ \xi_\nu^\lambda \}$ be a maximal, $\lambda^{-1/2}$ separated subset of the unit sphere in $\RR^n$, where $\nu$ ranges over some set $\Theta^\lambda$ with $\#(\Theta^\lambda) \sim \lambda^{(d-1)/2}$. Let
%
\[ \Gamma^\lambda_\nu = \{ \xi \in \RR^d_\xi : |\xi \cdot \xi^\lambda_\nu| \geq (1 - c \lambda^{-1/2}) \cdot |\xi| \} \]
%
for some suitably small constant $c > 0$. Let $\{ \chi^\lambda_\nu \}$ be a smooth partition of unity, homogeneous of degree zero, adapted to the $\Gamma^\lambda_\nu$. We thus have
%
\[ |D^\alpha \chi^\lambda_\nu(\xi)| \lesssim_\alpha \lambda^{|\alpha|/2} |\xi|^{1 - \alpha}. \]
%
We thus consider operators $T^\lambda_\nu$ with kernels $K^\lambda_\nu$ given by
%
\[ K^\lambda_\nu(x,t;y) = \int e^{2 \pi i (\phi(x,t;\xi) - y \cdot \xi)} b^\lambda_\nu(x,t;\xi) ( 1 + |\xi|^2 )^{\mu/2} \]
%
where
%
\[ b^\lambda_\nu(x,t;\xi) = b(x,t;\xi) \beta(\xi / \lambda) \chi^\lambda_\nu(\xi). \]
%
Stationary phase again tell us that $K^\lambda_\nu(x,t;y)$ satisfies the bounds
%
\[ |K^\lambda_\nu(x,t;y)| \lesssim_N \frac{\lambda^{\mu + (d+1)/2}}{\langle \lambda | \pi_{\xi^\lambda_\nu} (y - \nabla_\xi \phi(x,t,\xi^\lambda_\nu)| + \lambda^{1/2} | \pi_{\xi^\lambda_\nu}^\perp (y - \nabla_\xi \phi(x,t,\xi^\lambda_\nu)) | \rangle^N}. \]
%
This bound immediately yields via Schur's Lemma that for all $1 \leq p \leq \infty$,
%
\[ \| K^\lambda_\nu \|_{L^\infty_{x,t} L^1_y} \lesssim \lambda^\mu, \]
%
and thus that
%
\[ \| T^\lambda_\nu f \|_{L^\infty(\RR^{d+1})} \lesssim \lambda^\mu \| f \|_{L^\infty(\RR^d)}, \]
%
a much better bound than was obtained trivially than from the global sum.

We might hope to then combine this still fairly trivial bound with a square function estimate of the form
%
\[ \| T^\lambda_\nu f \|_{L^p(\RR^{d+1})} \lesssim_\varepsilon \lambda^\varepsilon \| S^\lambda f \|_{L^p(\RR^{d+1})} \]
%
where
%
\[ S^\lambda f = \left( \sum_\nu | T^\lambda_\nu f |^2 \right)^{1/2}, \]
%
which in some sense, captures the orthogonality of the operators $\{ T^\lambda_\nu \}$. This then yields that for $p \geq 2$, that
%
\begin{align*}
    \| T^\lambda_\nu f \|_{L^p_{x,t}} &\lesssim_\varepsilon \lambda^\varepsilon \| T^\lambda_\nu f \|_{L^p_{x,t} l^2_\nu}\\
    &\leq \lambda^\varepsilon \| T^\lambda_\nu f \|_{L^p_{x,t} l^p_\nu}\\
    &= \lambda^\varepsilon \| T^\lambda_\nu f \|_{l^p_\nu L^p_{x,t})}\\
    &\lesssim \lambda^\varepsilon \lambda^{\mu + (d-1)/2} \#(\Theta^\lambda)^{1/p}\\
    &= \lambda^{\varepsilon + \mu + (d-1) / p},
\end{align*}
%
thus giving bounds for $\mu > (d-1)/2$, i.e., the non-endpoint local smoothing.

Wolff noticed that the non-endpoint local smoothing results could be obtained with a weaker bound than a square function estimate, namely, an \emph{$l^p$ decoupling inequality} of the form
%
\[ \| T^\lambda f \|_{L^p(\RR^{d+1})} \lesssim \lambda^{\alpha(p) + \varepsilon} \| T^\lambda_\nu f \|_{l^p_\nu L^p_{x,t}}, \]
%
where if $2 \leq p \leq 2(d+1)/(d-1)$, then
%
\[ \alpha(p) = (d-1)|1/p - 1/2|, \] 
%
and for $2(d+1)/(d-1) \leq p < \infty$,
%
\[ \alpha(p) = (d-1)|1/p - 1/2| - 1/p. \]
%
The $L^p$ norm of the localized pieces is much easier to estimate. For instance, we have
%
\[ \| T^\lambda_\nu f \|_{L^\infty_{x,t}} \lesssim \lambda^\mu \| f \|_{L^\infty}, \]
%
and thus
%
\[ \| T^\lambda_\nu f \|_{l^\infty_\nu L^\infty_{x,t}} \lesssim \lambda^\mu \| f \|_{L^\infty}. \]
%
On the other hand, we have an $L^2$ energy conservation estimate of the form
%
\[ \| T^\lambda_\nu f \|_{L^2_{x,t}} \lesssim \| T^\lambda_\nu f \|_{L^\infty_t L^2_x} \lesssim \lambda^\mu \| f^\lambda_\nu \|_{L^2} \]
%
where $f^\lambda_\nu$ is the localization of $f^\lambda_\nu$ on the Fourier side to the support of $\chi^\lambda_\nu$. This immediately yields via Parseval's inequality and orthogonality that
%
\[ \| T^\lambda_\nu f \|_{l^2_\nu L^2_{x,t}} \lesssim \lambda^\mu \| f^\lambda_\nu \|_{l^2_\nu L^2_x} \lesssim \lambda^\mu \| f \|_{L^2_x}. \]
%
Interpolation thus yields that for $2 \leq p \leq \infty$,
%
\[ \| T^\lambda_\nu f \|_{l^p_\nu L^p_{x,t}} \lesssim \lambda^\mu \| f \|_{L^p}, \]
%
and thus that, together with Wolff's decoupling inequality,
%
\[ \| T^\lambda f \|_{L^p(\RR^{d+1})} \lesssim_\varepsilon \lambda^{\alpha(p) + \mu + \varepsilon} \| f \|_{L^p(\RR^d)}, \]
%
and thus we get boundedness of $T$ for $\mu < \alpha(p)$, which gives $1/p$ degrees of local smoothing.













\part{Attempts To Solve Problems}

\chapter{Relations to Local Smoothing}

Let us now try and prove certain special cases of the radial multiplier conjecture on the sphere $S^d$. Thus we fix a unit scale symbol $h$, and study operators of the form
%
\[ T_R = h \left( \sqrt{-\Delta} / R \right) = \sum h(\lambda / R) E_\lambda, \]
%
where $E_\lambda$ is the projection operator onto the eigenspace corresponding to the eigenvalue $\lambda$. In particular, we wish to characterize the boundedness properties of the operators $T_R$, in terms of appropriate control of the Fourier transform of the function $h$. For an exponent $1 \leq p < 2d/(d+1)$, we set $s_p = (d-1)(1/p - 1/2)$, we assume that the quantities
%
\[ C_p(h) = \left( \int_0^\infty \left[ \langle t \rangle^{s_p} |\widehat{h}(t)| \right]^p\; ds \right)^{1/p}, \]
%
and
%
\[ \| h \|_{L^\infty(0,\infty)} = \sup_{0 < \lambda < \infty} |h(\lambda)| \]
%
are finite, which, by Mitjagin's transference principle, is a necessary condition for the operators $\{ T_R \}$ to be uniformly bounded on $L^p(S^d)$.

The projection operators $\{ E_\lambda \}$ are smoothing, and so for any $R \leq 100$, the triangle inequality trivially implies that
%
\begin{align*}
    \| T_R f \|_{L^p(M)} &\leq \sum |h(\lambda/R)| \| E_\lambda f \|_{L^p(M)}\\
    &\leq \| h \|_{L^\infty(0,\infty)} \sum_{\substack{\lambda \in \Lambda \\ 0 < \lambda \leq 200}} \| E_\lambda f \|_{L^p(M)}\\
    &\lesssim \| h \|_{L^\infty(0,\infty)} \| f \|_{L^p(M)}.
\end{align*}
%
Thus we may assume without loss of generality in what follows that $R \geq 100$.

Thus our goal is to show
%
\begin{equation} \label{TheBigBound}
    \sup_{R \geq 100} \sup_{f \in C^\infty(M)} \frac{\| T_R f \|_{L^p(M)}}{\| f \|_{L^p(M)}} < \infty.
\end{equation}
%
Since $T_R$ is a multiplier with symbol supported on $R/2 \leq \lambda \leq 2R$, we can get away with a slightly weaker inequality. Namely, if for each $R$, we define the subspace $V_R$ of $C^\infty(M)$ spanned by eigenfunctions of $\sqrt{-\Delta}$ with eigenvalue in $R/4 \leq \lambda \leq 4R$, then \eqref{TheBigBound} is equivalent to the bound
%
\begin{equation} \label{WeakerBigBound}
    \sup_{R \geq 100} \sup_{f \in V_R} \frac{\| T_R f \|_{L^p(M)}}{\| f \|_{L^p(M)}} < \infty.
\end{equation}
%
To see this, suppose \eqref{WeakerBigBound} held, and the supremum was equal to $C$. Fix a smooth bump function $\beta$ supported on $\{ 1/4 \leq |t| \leq 4 \}$, and equal to one on $\{ 1/2 \leq |t| \leq 2 \}$, and consider the multiplier operators
%
\[ P_R: C^\infty(M) \to V_R, \]
%
equal to $\beta( \sqrt{-\Delta} / R )$. Since $\beta$ is smooth and compactly supported, it is a symbol of order zero, and thus the family of operators $\{ P_R \}$ are uniformly bounded in the $L^p(M)$ norm for any $1 < p < \infty$. Since $h = h \cdot \beta$, it follows that $T_R = T_R \circ P_R$. Since for any $f \in C^\infty(M)$, $P_R f \in V_R$, applying \eqref{WeakerBigBound} gives that
%
\[ \| T_R f \|_{L^p(M)} = \| T_R (P_R f) \|_{L^p(M)} \leq C \| P_R f \|_{L^p(M)} \lesssim C \| f \|_{L^p(M)}. \]
%
Thus \eqref{TheBigBound} now holds.

The main advantage of a reduction to \eqref{WeakerBigBound} is that, since $\sqrt{-\Delta}$ is an elliptic pseudodifferential operator on $S^d$, for any $s \geq 0$, we have derivative estimates of the form
%
\begin{equation} \label{ManifoldBernsteinInequality}
    \| f \|_{L^p_s(M)} \lesssim_s R^s \| f \|_{L^p(M)}
\end{equation}
%
which hold uniformly in $R \geq 100$ and $f \in V_R$, an analogue of Bernstein's inequality on Euclidean space.

To exploit the fact that $C_p(h)$ is finite, we apply the Fourier transform to the sum defining $T_R$, writing
%
\[ T_R = \int_{-\infty}^\infty R \widehat{h}(R t) e^{2 \pi i t \sqrt{-\Delta}}\; dt. \]
%
where $\big\{ e^{2 \pi i t \sqrt{-\Delta}} \big\}$ are the solution operators to the half wave equation
%
\[ \partial_t = 2 \pi i \sqrt{-\Delta}. \]
%
The three open sets
%
\[ \{ |t| < 100/R \}, \quad \{ 50/R < |t| < 2/5 \}, \quad\text{and}\quad \{ 1/3 < |t| < \infty \} \]
%
cover $\RR$, and so we can find a smooth partition of unity $\chi_{I,R}$, $\chi_{II,R}$, and $\chi_{III,R}$ supported on these sets. Without loss of generality, we may assume all three functions are even, that $\chi_{I,R}(t) = \chi_I(Rt)$, for some smooth, compactly supported function $\chi_I$ adapted to the open set $\{ |t| < 1 \}$, and also assume that $\chi_{III,R} = \chi_{III}$ is independent of $R$. Given this partition of unity, we now write
%
\[ T_R = I_R + II_R + III_R, \]
%
where, for $\Pi \in \{ I, II, III \}$,
%
\[ \Pi_R = \int \chi_{\Pi,R}(t) R \widehat{h}(Rt) e^{2 \pi i t \sqrt{-\Delta}}\; dt. \]
%
For each $R$, the \emph{class} of operators of the form $\Pi_R$, in induced from a multiplier $h$ satisfying the hypothesis of the theorem, is closed under taking adjoints. Indeed, if $\Pi_R$ is obtained from $h$, then $\Pi_R^*$ is obtained from the multiplier $\overline{h}$, which satisfies the same assumptions as $h$. Because of this self-adjointness, if we can prove that for any multiplier $h$ satisfying the assumptions of the theorem, the operators $\{ \Pi_R \}$ are bounded in $L^{p^*}(M)$, uniformly in $R$, then it follows that for any such $h$, it is also true that the operators $\{ \Pi_R \}$ are bounded in $L^p(M)$, uniformly in $R$.

We begin by exploiting this duality property to bound $III_R$, uniformly in $R$, by relating this operator to \emph{local smoothing} for the wave equation. Using the periodicity of the wave equation, we can write
%
\[ III_R = \int_{-1/2}^{1/2} H_R(t) e^{2 \pi i t \sqrt{-\Delta}}\; dt, \]
%
where
%
\[ H_R(t) = \sum_l \chi_{III}(t) R \widehat{h}(R(t + l)) = \sum_l H_{R,l}(t). \]
%
Now
%
\begin{align*}
    &\left( \sum_{l \neq 0} \left( |Rl|^{s_p} \| H_{R,l} \|_{L^p[-1/2,1/2]} \right)^p \right)^{1/p}\\
    &\quad\quad \sim R \left( \int_{-1/2}^{1/2} \sum_{l \neq 0} \left( |R(t + l)|^{s_p} |\widehat{h}(R(t + l))| \right)^p\; dt \right)^{1/p}\\
    &\quad\quad = R \left( \int_{|t| \geq 1/2} \left( |Rt|^{s_p} |\widehat{h}(Rt)| \right)^p\; dt \right)^{1/p} \\
    &\quad\quad\lesssim R^{1/p^*} C_p(h).
\end{align*}
%
and
%
\begin{align*}
    \| H_{R,0} \|_{L^p[-1/2,1/2]} &= \left( \int_{-1/2}^{1/2} |\chi_{III}(t) R \widehat{h}(Rt)|^p \right)^{1/p}\\
    &\leq \left( \int_{1/3 \leq |t| \leq 1/2} |R \widehat{h}(Rt)|^p \right)^{1/p}\\
    &= R^{1/p^*} \left( \int_{R/3}^{R/2} |\widehat{h}(t)|^p \right)^{1/p}\\
    &\lesssim R^{1/p^* - s_p} C_p(h).
\end{align*}
%
Since the family of functions $\{ H_{R,l} \}$ could in general be chosen to be quite correlated with one another, H\"{o}lder's inequality is the best we can do here, allowing us to conclude that, provided $p < 2d/(d+1)$, i.e. so that $p^* > 2d/(d-1)$, and thus $s_p p^* = (d-1)(p^*/2 - 1) > 1$,
% (d-1)(p^*/2 - 1) > 1
% p^* > 2d/(d-1)
\begin{align*}
    \| H_R \|_{L^p[-1/2,1/2]} &\leq \sum_l \| H_{R,l} \|_{L^p[-1/2,1/2]}\\
    &= \| H_{R,0} \|_{L^p[-1/2,1/2]} + \sum_{l \neq 0} \left( |Rl|^{s_p} \| H_{R,l} \|_{L^p[-1/2,1/2]} \right) \left( |Rl|^{-s_p} \right)\\
    &\lesssim R^{1/p^* - s_p} C_p(h) + ( R^{1/p^*} C_p(h) ) \left( \sum_{l \neq 0} |Rl|^{-s_p p^*} \right)^{1/p^*}\\
    &= R^{1/p^* - s_p} C_p(h) \left( 1 + \left( \sum_{l \neq 0} |l|^{-s_p p^*} \right)^{1/p^*} \right)\\
    &= R^{1/p^* - s_p} C_p(h) \left( 1 + \left( \sum_{l \neq 0} |l|^{-s_p p^*} \right)^{1/p^*} \right) \\
    &\lesssim_p C_p(h) R^{1/p^* - s_p}.
\end{align*}
%
A further application of H\"{o}lder's inequality shows that
%
\begin{align*}
    | III_R f | &= \left| \int_{-1/2}^{1/2} H_R(t) e^{2 \pi i t \sqrt{-\Delta}} f\; dt \right|\\
    &\lesssim C_p(h) R^{1/p^* - s_p} \left( \int_{-1/2}^{1/2} |e^{2 \pi i \sqrt{-\Delta}} f|^{p^*}\; dt \right)^{1/p^*}.
\end{align*}
%
Applying the endpoint local smoothing inequality, we conclude that
%
\begin{align*}
    \| III_R f \|_{L^{p^*}(M)} &\lesssim C_p(h) R^{1/p^* - s_p} \| e^{2 \pi i \sqrt{-\Delta}} f \|_{L^{p^*}_t L^{p^*}_x}\\
    &\lesssim C_p(h) R^{1/p^* - s_p} \| f \|_{L^{p^*}_{s_p - 1/p^*}(M)},
\end{align*}
%
But since we are assuming that $f$ lies in $V_R$, we have seen we have estimates of the form
%
\[ \| f \|_{L^{p^*}_{s_p - 1/p^*}(M)} \lesssim R^{s_p - 1/p^*}. \]
%
Thus we conclude that
%
\[ \| III_R f \|_{L^{p^*}(M)} \lesssim C_p(h) \| f \|_{L^{p^*}(M)}. \]
%
We have therefore bounded $III_R$ uniformly in $R$.

We can write
%
\[ I_R = \int \chi_I(t) \widehat{h}(t) e^{2 \pi i (t / R) \sqrt{-\Delta}} \]
%
The operator $I_R$ is thus obtained by averaging the half wave equation over an increasingly smaller family of times as $R \to \infty$. Given the pseudolocal finite speed of propogation of the wave equation, we might expect this operator to behave pseudolocally. In fact, we will conclude that for each $R$, $I_R$ is a pseudodifferential operator of order zero given by a symbol $a_R$, and that the symbols $\{ a_R \}$ form a bounded subset of the symbol class $\mathcal{S}^0$. This is sufficient to conclude we have uniform estimates of the form
%
\[ \| I_R f \|_{L^p(M)} \lesssim \| f \|_{L^p(M)}. \]
%
To see this, if $a$ is the inverse Fourier transform of $t \mapsto \chi_I(t) \widehat{h}(t)$, then
%
\[ I_R = a \left( \frac{\sqrt{-\Delta}}{R} \right). \]
%
If $\psi$ denotes the inverse Fourier transform of $\chi_I$, then we can write
%
\[ a(\lambda) = \int h(\alpha) \cdot \psi \left( \lambda - \alpha \right)\; d\alpha. \]
%
The fact that $h$ is compactly supported, and $\psi$ is Schwartz implies that for $\lambda \gtrsim 1$,
%
\[ |D^N a(\lambda)| \lesssim_{N,M} \| h \|_{L^\infty(\RR)} \lambda^{-M}. \]
%
But these bounds are sufficient to conclude that the family of rescaled functions $a_R(\lambda) = a(\lambda/R)$ are uniformly bounded in $\mathcal{S}^0$, and so the operators
%
\[ I_R = a_R \Big( \sqrt{-\Delta} \Big) \]
% a_R(x) = a(x / R)
% For x >> R, |D^N a_R(x)| << R^{M-N} x^{-M}
% D^N a_R(x) = R^{-N} D^N a(x/R)
are pseudodifferential operators of order zero, and uniformly bounded from $L^p(M)$ to itself for all $1 < p < \infty$.

\begin{comment}
%
\[ |D^N a(\lambda)| \lesssim_N \| h \|_{L^\infty(\RR)} \lambda^{-N}. \]
%
For $|\lambda| \leq 10$, we have
%
\[ |D^N a(\lambda)| \lesssim_N \| h \|_{L^\infty(\RR)} \]

Now
%
\[ |D^N a_R(\lambda)| \lesssim \| h \|_{L^\infty(\RR)} R \max D^N \beta_R() \]
The bounds on the derivatives and support of the functions $\{ \chi_{1,R} \}$ imply that
% |D^N beta_R(lambda)| <<_N R^{N+1}
% but also
% |D^N beta_R(lambda)| << lambda^{-M} int |D^M_t (t^N chi_{1,R})|
%   integral from 1/R to 2/R:
%     R^{M-N-1}
%     contributes O(R^{M-N-1})
%   integral from 3/4 to 1
%       contributes O(1)
% so for M > N, |D^N beta_R(lambda)| <<_{N,M} lambda^{-M} R^{M-N-1}
% for M <= N, |D^N beta_R(lambda)| <<_{N,M} lambda^{-M}

% |D^N \beta_R(\lambda)| = \int t^N chi_{1,R}(t) e^{2 \pi i t \lambda}\; dt
% = lambda^{-M} int t^N chi_{1,R}(t) D^M_t e^{2 pi i t lambda} dt
% <= lambda^{-M} int D^M_t ( t^N chi_{1,R} )
% Taking M > N gives that
% <<_{N,M} lambda^{-M} max_{K <= N} R^{K-N-1} D^{M-K} chi_{1,R} R^{K-M}

\[ |D^N \beta_R(\lambda)| \lesssim_M R^{N+1} \lambda^{-M}. \]

Since $h$ is supported on $1/2 \leq \alpha \leq 2$, and $\beta_R$ is Schwartz, uniformly in $R$ because of the derivative estimates we have on $\chi_{1,R}$, we find that for $\lambda \geq 10$,
%
\[ |D^N a_R(\lambda)| \lesssim_{N,M} \| h \|_{L^\infty(\RR)} R^{-N} \langle \lambda / R \rangle^{-M}. \]
%
For $\lambda \geq R$, picking $M = N$ gives that
%
\[ |D^N a_R(\lambda)| \lesssim_N \| h \|_{L^\infty(\RR)} \lambda^{-N}. \]
%
For $10 \leq \lambda \leq R$, picking $M = 0$ gives that
%
\[ |D^N a_R(\lambda)| \lesssim_N \| h \|_{L^\infty(\RR)} R^{-N} \leq \| h \|_{L^\infty(\RR)} \lambda^{-N}. \]
%
For $0 \leq \lambda \leq 10$, we can directly use the $L^1 \to L^\infty$ bound for the Fourier transform to conclude that
%
\begin{align*}
    |D^N a_R(\lambda)| &\lesssim_N \int_{|t| \leq 1/R} \left| t^N a_R(t) e^{2 \pi i t \lambda} \right|\; dt\\
    &\leq R^{-N} \int_{|t| \leq 1/R} R |\widehat{h}(Rt)|\; dt\\
    &\leq R^{-N} \int_{|t| \leq 1} |\widehat{h}(t)|\; dt\\
    &\leq R^{-N} C_p(h)\\
    &\lesssim_N R^{-N} C_p(h) \lambda^{-N}\\
    &\leq C_p(h) \lambda^{-N}.
\end{align*}
%
Putting these three bounds together gives that the family $\{ a_R \}$ are a family of functions uniformly bounded in the class of symbols of order zero on the real line. But this means that the operators $\{ I_R \}$ are uniformly bounded as operators on $L^p(M)$ for $1 < p < \infty$.
\end{comment}

It remains to analyze the operator $II_R$, obtained by integrating against the half wave propogators over times $50/R \leq |t| \leq 2/5$, i.e. the operator
%
\[ II_R = \int b_R(t) e^{2 \pi i t \sqrt{-\Delta}}\; dt, \]
%
where
%
\[ b_R(t) = \chi_{II,R}(t) R \widehat{h}(Rt). \]
%
Essentially we know the following facts about $b_R$:
%
\begin{itemize}
    \item We have
    %
    \[ \left( \int_{-\infty}^\infty |b_R(t)|^p |t|^{(d-1)(1 - p/2)} \right)^{1/p} \leq R^{-s_{p^*}} C_p(h). \]

    \item $\text{supp}(b_R) \subset \{ 50/R \leq |t| \leq 2/3 \}$.

    \item Uncertainty principle heuristics tell us that, since the Fourier transform of $b_R$ is supported in a ball of radius $O(R)$, $b_R$ should be, heuristically speaking, locally constant at a scale $1/R$. Thus we can write
    %
    \[ b_R = \sum_{50 \leq |i| \leq (2/3)R} c_{R,i} \eta_{R,i}, \]
    %
    where $\eta_{R,i}$ is supported on $\{ i/R - (1/4)(1/R) \leq t \leq i/R + (5/4)(1/R) \}$, and satisfies uniform bounds of the form
    %
    \[ |D^N \eta_{R,i}(x)| \lesssim_N R^N. \]
    %
    for all $N > 0$. The $L^p$ bound for $b_R$ immediately implies that
    %
    \[ \left( \sum_i ( |c_i| |i|^{s_p} )^p \right)^{1/p} \lesssim R. \]
\end{itemize}
%
It is therefore natural to try and obtain bounds of the form
%
\[ \| II_R f \|_{L^{p^*}(M)} \leq R^{s_{p^*}} \left( \int_{-\infty}^\infty |b_R(t)|^p |t|^{(d-1)(1 - p/2)}  \right)^{1/p} \| f \|_{L^{p^*}(M)} \]
%
for all functions $f \in V_R$.

Given the presence of the $L^p$ norm of $b_R$ and the $L^{p^*}$ norm of $f$, a natural first choice might be to try and prove this inequality using H\"{o}lder's inequality, writing
%
\begin{align*}
    |II_R f| &\leq \int \left( |b_R(t)| |t|^{(d-1)(1/p - 1/2)} \right) \left(| e^{2 \pi i t \sqrt{-\Delta}} f | |t|^{-(d-1)(1/p - 1/2)} \right)\; dt\\
    &\leq \left( \int |b_R(t)|^p |t|^{(d-1)(1 - p/2)} \right)^{1/p} \left( \int_{1/R \lesssim |t| \leq 2/5} |e^{2 \pi i t \sqrt{-\Delta}} f|^{p^*} |t|^{-(d-1)(p^*/2 - 1)} \right)^{1/p^*}.
\end{align*}
%
It then suffices to prove that for $f \in V_R$,
%
\[ \left( \int_{1/R \leq |t| \leq 1/2} |t|^{-(d-1)(p^*/2 - 1)} \| e^{2 \pi i t \sqrt{-\Delta}} f \|_{L^{p^*}(M)}^{p^*} \right)^{1/p^*} \lesssim R^{s_p^*} \| f \|_{L^{p^*}(M)}. \]
%
However, this is \emph{not} a valid inequality. TODO: I thought I'd found a counterexample before using periodicity, but I made a mistake, which means the counterexample doesn't quite work. Maybe a sum of different annuli will work, at least to get a logarithmic counterexample? We therefore \emph{cannot} naively H\"{o}lder out the function $b_R$, and we must use additional orthogonality between different times of the wave equation to obtain the result required.

\begin{comment}

Using periodicity, if $R$ is divisible by $1/2$, then we can write
%
\begin{align*}
    &\int_{1/R \lesssim |t| \leq 1/2} \| e^{2 \pi i t \sqrt{-\Delta}} f \|^{p^*}_{L^{p^*}(M)}\\
    &\quad \approx \frac{1}{2R} \int_{1/R \lesssim |t| \lesssim R} \| e^{2 \pi i t \sqrt{-\Delta}} f \|_{L^{p^*}(M)}^{p^*}\; dt - (1 - 1/2R) \int_{|t| \lesssim 1/R} \| e^{2 \pi i t \sqrt{-\Delta}} f \|_{L^{p^*}(M)}^{p^*}\; dt.
\end{align*}
% 2R - 1 copies of |t| <= 1/R
%
The first term is just a rescaled local smoothing term. If $f$ is badly behaved, we can therefore expect to have
%
\begin{align*}
    \frac{1}{2R} \int_{1/R \lesssim |t| \lesssim R} \| e^{2 \pi i t \sqrt{-\Delta}} f \|_{L^{p^*}(M)}^{p^*}\; dt &\approx R^{s_{p^*} p^*} \| f \|_{L^{p^*}(M)}^{p^*}\\
    &= R^{(d-1)(p^*/2 - 1) - 1} \| f \|_{L^{p^*}(M)}^{p^*}
\end{align*}
%
whereas we should also expect by pseudolocal finite speed of propogation and the intuitive local constancy of $f$ on scales $1/R$ that
%
\begin{align*}
    \int_{|t| \lesssim 1/R} |e^{2 \pi i t \sqrt{-\Delta}} f|^{p^*}\; dt &\approx \| f \|_{L^{p^*}(M)}^{p^*}.
\end{align*}
%
Thus in the range of $p^*$ we are considering, the first term has much larger $L^p$ norm, and thus cannot be cancelled out by subtracting the second term. We thus conclude that in the worst case, we should expect
%
\[ \left( \int_{1/R \lesssim |t| \leq 1/2} \| e^{2 \pi i t \sqrt{-\Delta}} f \|_{L^{p^*}(M)}^{p^*} \right)^{1/p^*} \gtrsim R^{(d-1)(1/2 - 1/p^*) - 1/p^*} \| f \|_{L^{p^*}(M)}. \]
%


\end{comment}

One thing that will help us obtain more detailed information about the operator we are studying is the introduction of the Lax Parametrix for the wave equation, which is valid because we are analyzing the wave equation over a small set of times. Recall that for $|t| \lesssim 1$, we can consider a finite sum
%
\[ e^{2 \pi i t \sqrt{-\Delta}} = A_t + \sum_{W \in \mathcal{W}} W_t, \]
%
where $\{ A_t \}$ are a smooth family of smoothing operators, i.e. satisfying the following property:
%
\begin{itemize}
    \item There exists $\alpha \in C^\infty(\RR \times M \times M)$ such that
    %
    \[ A_t f(x) = \int_{S^d} \alpha(t,x,y) f(y)\; dy \]
    %
    for any $f \in C^\infty(M)$.
\end{itemize}
%
The operators $\{ W_t \}$ satisfy the following properties:
%
\begin{itemize}
    \item For each $W \in \mathcal{W}$, we can find a compact set $C'$ of $S^d$, contained in an open subset $U'$ of $S^d$, and we can find a diffeomorphism $a: U' \to U$, where $U$ is a precompact open subset of $\RR^d$, such that the operators $\{ W_{i,t} \}$ are all supported on $C'$, in the sense that for any $f \in C^\infty(M)$, $\text{supp}(f) \subset C'$, and if $\text{supp}(f) \cap C' = \emptyset$, $Wf = 0$. Let $C = a(C')$.

    \item The operator can be written as a Schwartz kernel in coordinates, i.e. for each $t$, we can find a distribution $K_t$, compactly supported on $C \times C$ such that for $f \in C^\infty(M)$,
    %
    \[ W_tf(a(x)) = \int_U K_t(a(x),y) f(a^{-1}(y))\; dy. \]

    \item The kernel $K_t$ can be written as
    %
    \[ K_t(x,y) = \int_{\RR^d} a(x,y,\xi,t) e^{2 \pi i \Phi(x,y,\xi,t)}\; d\xi\; dt, \]
    %
    where
    %
    \begin{itemize}
        \item The phase function $\Phi$ is a smooth function away from $\xi = 0$, homogeneous of degree one in the $\xi$ variable, such that
        %
        \[ \Phi(x,y,\xi,t) = \phi(x,y,\xi) + t |\xi|_g \]
        %
        for a smooth function $\phi$ with $\phi(x,y,\xi) \approx (x - y) \cdot \xi$, in the sense that
        %
        \[ |\partial^\beta_\xi \{ \phi(x,y,\xi) - (x - y) \cdot \xi \}| \lesssim_\beta |x - y|^2 |\xi|^{1-\beta}. \]

        \item The amplitude function $a$ is a symbol of order zero with
        %
        \[ \text{supp}(a) \subset \{ (x,y,\xi,t): |x - y| \lesssim 1\ \text{and}\ |\xi| \gtrsim 1 \}, \]
        %
        In particular, we may choose the support of $a$ to be close enough to the $(x,y)$ diagonal that one has
        %
        \[ |\nabla_\xi \phi(x,y,\xi)| \gtrsim |x - y| \quad\text{and}\quad |\nabla_x \phi(x,y,\xi)| \gtrsim |\xi| \]
        %
        on the support of $a$.
    \end{itemize}
\end{itemize}
%
It follows that we can write
%
\[ II_R = II_{A,R} + \sum_{W \in \mathcal{W}} II_{W,R}, \]
%
where for $S \in \{ A, W \}$,
%
\[ II_{S,R} = \int b_R(t) S_t\; dt. \]
%
Since this is a finite sum, it suffices to bound each of these operators separately, uniformly in $R$.

Let's start by bounding $II_{A,R}$ uniformly in $R$. If $\alpha \in C^\infty(\RR \times M \times M)$ is the smooth kernel of $A$, then we can write
%
\begin{align*}
    II_{A,R} f(x) &= \int b_R(t) \alpha(t,x,y) f(y)\; dt\; dy\\
    &= \sum_i c_i \int \eta_{R,i}(t) \alpha(t,x,y) f(y)\; dt\; dy.
\end{align*}
%
Applying H\"{o}lder's inequality and Schur's lemma gives that
%
\begin{align*}
    \| II_{A,R} f \|_{L^p(M)} &\lesssim R \left( \sum \left( |i|^{-s_p} \| S_{R,i} f \|_{L^p(M)} \right)^{p^*} \right)^{1/p^*}\\
    &\lesssim R \left( \sum \left( |i|^{-s_p} R^{-1} \| f \|_{L^p(M)} \right)^{p^*} \right)^{1/p^*}\\
    &\lesssim \left( \sum |i|^{-p^* s_p} \right)^{1/p^*} \| f \|_{L^p(M)}\\
    &\lesssim \| f \|_{L^p(M)}.
\end{align*}
%
Thus $II_{A,R}$ is uniformly bounded in $L^p(M)$.

Next, we analyze $II_{W,R}$, which will occupy us for the remainder of the argument. Without loss of generality, we may work in Euclidean coordinates, i.e. abusing notation, and assuming that $II_{W,R}$ is the Schwartz operator defined on functions on $U$, such that
%
\[ II_{W,R} f(x) = \int \int_U b_R(t) K_t(x,y) f(y)\; dy\; dt. \]
%
This doesn't cause any issues, since this operator will be uniformly bounded in $L^p(U)$ if and only if the original operators are uniformly bounded in $L^p(M)$. TODO: We can assume inputs $f$ satisfy Bernsteins' inequality.

We perform a further decomposition, writing
%
\[ II_{W,R} = \sum_i \sum_{l = 0}^\infty c_i II_{W,R,i,l}, \]
%
where
%
\[ II_{W,R,i,l} f(x) = \int \int_U \eta_{R,i}(t) K_{t,l}(x,y) f(y)\; dy\; dt, \]
%
and where
%
\[ K_{t,l}(x,y) = \int \chi(\xi / 2^l) a(t,x,y, \xi) e^{2 \pi i \Phi(x,y, \xi,t)}\; d\xi\; dt. \]
%
Finally, for each $l$, we let $\Theta_l$ be a maximal $2^{-l/2}$ separated subset of $S^{d-1}$. Then $\#(\Theta_l) \sim 2^{l(n-1)/2}$. For $\nu \in \Theta_l$, let
%
\[ \Gamma^l_\nu = \{ \xi \in \RR^d: | \pi^\perp_\nu \xi | \lesssim 2^{-l/2} \xi \} \]
%
denote the sector with aperture $\sim 2^{-l/2}$ pointing in the direction $\xi$. Let $\{ \varphi_\nu \}$ be a smooth partition of unity adapted to these sectors, and then set
%
\[ II_{W,R,i,l,\nu} = \int \int_U \eta_{R,i}(t) \chi(\xi) \varphi_\nu(\xi) a(t,x,y, 2^l \xi) e^{2 \pi i 2^l \Phi(x,y,\xi,t)}\; d\xi\; dt. \]
%
A standard stationary phase argument shows that the kernel $K_{W,R,i,l,\nu}$ of $II_{W,R,i,l,\nu}$ satisfies bounds of the form
%
\[ |K_{W,R,i,l,\nu}(x,y)| \lesssim_N \int_{(i-1)/R}^{(i+5/4)/R} \frac{2^{\frac{l(d+1)}{2}}}{(1 + 2^l | \pi_\nu( \nabla_\xi \Phi(x,y,\nu,t) ) | + 2^{l/2} | \pi_\nu^\perp ( \nabla_\xi \Phi(x,y,\nu,t) ) | )^N}\; dt. \]
%
We note that
%
\[ \nabla_\xi \Phi(x,y,\nu,t) = \exp(t \nu) + x - y + O(|x - y|^2), \]
%
so that for a fixed $x$, $K_{W,R,i,l,\nu}$ is supported on 

% \nabla_xi Phi(x,t,\nu)
% (x - y) * xi + t |xi|_g + O(|x - y|^2 |xi|)
%   -> x - y + t/2 ( sum g_{ij} xi^i xi^j )^{-1/2} * ( sum_j g_{ij} xi^j + g_{ii} \xi^i )



We will attempt to use the method of Lee and Seeger, combined with the methods of Heo, Nazarov, and Seeger, to analyze this kernel by discretizing at a scale $1/R$, and applying a variant of the Calderon-Zygmund type decomposition that Heo, Nazarov and Seeger use to show that we have an $l^p L^p$ decoupling type bound for these quantities.

In Heo, Nazarov, and Seeger, it helped to introduce an additional oscillatory factor into the estimates, which is possible because the multiplier they are studying is supported at the unit scale (in particular, away from the origin). In our situation, since $m$ was compactly supported on the annulus $1/2 \leq |\xi| \leq 2$, we can do something similar. Let $\psi$ be an even, compactly supported smooth function on the real line, whose Fourier transform is non-negative, vanishes to high order at the origin, but is nonvanishing on the annulus $1/4 \leq |\xi| \leq 4$. It suffices to show the multiplier
%
\[ \tilde{m}(\lambda) = \widehat{\psi}(\lambda) m(\lambda), \]
%
is in $M_{\text{Dil}}^p(M)$. This is because the operator $\lambda \mapsto 1/\widehat{\psi}(\lambda)$ is smooth on the support of $m$, and thus satisfies uniform $L^p$ estimates on it's dilations by $R$ provided the inputs are linear combinations of eigenfunctions with frequency $\sim R$. But this is sufficient to conclude that $m$ is in $M_{\text{Dil}}^p(M)$. In the sequel, set
%
\[ b_R(t) = \chi_{II,R}(t) \cdot R \cdot (\psi * \widehat{h})(Rt). \]
%
The symbol $\tilde{m}$ satifies all the same estimates that $m$ satisfies, and so the previous part of the argument carries through for this multiplier, and it remains to analyze the times $1/R \lesssim |t| \lesssim 1$ more carefully. It suffices to bound the operators $T_R$, with kernels
%
\begin{align*}
    K_R(x,y) &= \int \int b_R(t) s(x,y,\xi,t) e^{2 \pi i \Phi(t,x,y,\xi)}\; d\xi\; dt\\
    &= \int \int \int \chi_{II,R}(t) [R \psi(R(t - u))] [R\widehat{h}(Ru)] s(x,y,\xi,t) e^{2 \pi i \Phi(t,x,y,\xi)}\; d\xi\; dt\; du\\
    &= \int [R \widehat{h}(Ru)] F_{y,u}(x)\; du,
\end{align*}
%
where
%
\begin{align*}
    F_{y,u}(x) = \int \int \chi_{II,R}(t) [R \psi(R(u - t))] s(x,y,\xi,t) e^{2 \pi i \Phi(t,x,y,\xi)} \; d\xi\; dt.
\end{align*}
%
We are `averaging' (by an oscillatory quantity) over a $O(1/R)$ neighborhood of $u$, and thus we should expect from the wavefront set analysis of $F_{y,u}$ that this function has mass concentrated on an annulus of thickness $O(1/R)$ and radius $u$, centered at $y$.

Now our goal is to prove bounds of the form
%
\[ \left\| \int_{1/R \lesssim |u| \lesssim 1} \int_U f(y) [R \widehat{h}(Ru)] F_{y,u}(x)\; dy\; du \right\|_{L^p_x} \lesssim \| f \|_{L^p_y} \]
%
which are uniform in $R$. We know that
%
\begin{align*}
    &\left( \int_{1/R \lesssim |u| \lesssim 1} |R \widehat{h}(Ru)|^p |u|^{(d-1)(1-p/2)} \; du \right)^{1/p} \lesssim R^{-s_{p^*}} C_p(h).
\end{align*}
%
Moreover, the function $R \widehat{h}(Ru)$ is locally constant at a scale $1/R$ by the uncertainty principle.

Thus our argument will follow if we can justify that for functions $a$, which are supported on $1/R \lesssim |u| \lesssim 1$,
%
\[ \left\| \int \int a(y,u) F_{y,u}\; dy\; du \right\|_{L^p_x} \lesssim R^{s_{p^*}} C_p(h) \left( \int \int |a(y,u)|^p |u|^{(d-1)(1 - p/2)}\; dy\; du \right)^{1/p}. \]
%

By applying the triangle inequality, and applying symmetry, we may assume that $a$ is actually supported on $1/R \lesssim u \lesssim 1$ (i.e. we only consider positive values of $u$). It's a little suspicious that the exponent of $|u|$ on the right hand side is different from the exponent that is seen in Heo-Nasarov-Seeger, but I'll ask Andreas about this later and we'll see if it's a problem.

Discretizing as in that paper, it suffices to prove the following discretized version of this result at a scale $1/R$; if $\mathcal{E}$ is any $1/R$ discretized subset of $(y,u)$ pairs,
%
\[ \left\| \sum_{(y,u) \in \mathcal{E}} a(y,u) F_{y,u} \right\|_{L^p_x} \lesssim R^{s_{p^*}} \left( \sum_{(y,u) \in \mathcal{E}} |a(y,u)|^p |u|^{(d-1)(1 - p/2)} \right)^{1/p}. \]
%
Applying real interpolation, it actually suffices to show, if $\mathcal{E}_k$ is the set of all $(y,u)$ pairs with $2^k / R \leq u \leq 2^{k+1} / R$, then
%
\[ \left\| \sum_{(y,u) \in \mathcal{E}_k} F_{y,u} \right\|_{L^p_x} \lesssim R^{s_{p^*}} \left( \sum_k 2^{k(d-1)(1 - p/2)} \#(\mathcal{E}_k) \right)^{1/p}. \]
%
The values of $k$ for which $\mathcal{E}_k$ is nonempty occur for $1 \lesssim k \lesssim \log R$.

We can further perform a frequency decomposition in $\xi$. We begin by writing, for $j \gtrsim 1$,
%
\begin{align*}
    F_{y,u,j}(x) &= \int \int \chi_{II,R}(t) [R \psi(R(u - t))] \beta(\xi / 2^j) s(x,y,\xi,t) e^{2 \pi i \Phi(t,x,y,\xi)}\\
    &= 2^{jd} \int \int \chi_{II,R}(t) [R \psi(R(u - t))] \beta(\xi) s(x,y,2^j \xi,t) e^{2 \pi i 2^j \Phi(t,x,y,\xi)}\; d\xi\; dt.
\end{align*}
%
If $|d_g(x,y) - u| \gtrsim 1/R$, then the principle of nonstationary phase yields that
%
\[ |F_{y,u,j}(x)| \lesssim_N 2^{j(d-N)} (|d_g(x,y) - u| - 1/R)^{-N}. \]
%
Whereas for $|d_g(x,y) - u| \lesssim 1/R$, taking in absolute values simply gives
%
\[ |F_{y,u,j}(x)| \lesssim 2^{jd}. \]
%
We can do better than this by performing a further decomposition. Let $\Theta_j$ be a maximal $2^{-j/2}$ separated subset of unit vectors on the sphere, consider an associated homogeneous partition of unity $\{ a_\theta \}$ and write
%
\[ F_{y,u,j} = \sum_{\theta \in \Theta_j} F_{y,u,\theta}, \]
%
where
%
\[ F_{y,u,\theta}(x) = \int \int a_\theta(\xi) \chi_{II,R}(t) [R \psi(R(u - t))] s(x,y,\xi,t) e^{2 \pi i \Phi(t,x,y,\xi)} \; d\xi\; dt. \]
%


On the other hand, if $|d_g(x,y) - u| \lesssim 1/R$, then we must employ a stationary phase analysis. If $|d_g(x,y) - u| \lesssim 1/R$, the region upon which TODO. IDEA: Maybe the analysis will be more simple if we use the Hadamard parametrix since the phase simplifies.

% int lambda^{-N} psi^(lambda / R) e^{2 pi i lambda (u - t)} d lambda
% R^{-N} int (lambda/R)^{-N} psi^(lambda / R) e^{2 pi i lambda (u-t)} d lambda
% R^{-N} R psi^{(N)}(R(u - t))




Now perform a frequency decomposition, writing
%
\begin{align*}
    K_R(x,y) &= \sum_{k=1}^\infty \int b_R(t) \eta(\xi / 2^k) s(x,y,\xi,t) e^{2 \pi i \Phi(t,x,y,\xi)}\; d\xi\\
    &= \sum_{k = 1}^\infty \int s_{R,k}(x,y,\xi / 2^k,t) e^{2 \pi i \Phi(t,x,y,\xi)}\; d\xi\\
    &= \sum_{k = 1}^\infty K_{R,k}(x,y),
\end{align*}    
%
where
%
\[ s_{R,k}(x,y,\xi,t) = b_R(t) \eta(\xi) s(x,y,2^k \xi, t). \]
%
%satisfies the estimates
%
\[ |D^{n_1}_x D^{n_2}_y D^{n_3}_\xi D^{n_4}_t s_{R,k}(x,y,\xi,t)| \lesssim_n R^{n_4+1}. \]


\newpage



It remains to analyze the operator over small times, i.e. the operator
%
\[ I_R = \int_{-1/2}^{1/2} R \widehat{h}(Rt) e^{2 \pi i t \sqrt{-\Delta}}. \]
%
Let us write
%
\[ I_R = I_{R,1} + I_{R,2}, \]
%
where
%
\[ I_{R,0} = \int_{-1/2}^{1/2} R \widehat{h}(Rt) \eta(Rt) e^{2 \pi i t \sqrt{-\Delta}}\; dt = \int_{-1/2}^{1/2} a_{R,0}(t) e^{2 \pi i t \sqrt{-\Delta}}\; dt. \]
%
and
%
\[ I_{R,1} = \int_{-1/2}^{1/2} R \widehat{h}(Rt) (1 - \eta(Rt)) e^{2 \pi i t \sqrt{-\Delta}}\; dt = \int_{-1/2}^{1/2} a_{R,1}(t) e^{2 \pi i t \sqrt{-\Delta}}\; dt. \]
%
Now
%
\begin{align*}
    \| a_{R,0} \|_{L^p_t} &= R^{1/p^*} \left( \int |\widehat{h}(t)|^p |\eta(t)|^p\; dt \right)^{1/p}\\
    &\lesssim R^{1/p} C_p(h),
\end{align*}
%
and
%
\begin{align*}
    \| a_{R,1} \|_{L^p_t} &= R^{1/p^*} \left( \int |\widehat{h}(t)|^p |(1 - \eta(t)|^p\; dt \right)^{1/p}\\
    &\lesssim R^{1/p^*} \left( \int_{10}^R |\widehat{h}(t)|^p\; dt \right)^{1/p}\\
    &\lesssim R^{1/p^*} C_p(h) \left( \int_{10}^R |t|^{-(d-1)(p^*/2 - 1)} \right)^{1/p^*}\\
    &\lesssim R^{1/p^*} C_p(h).
\end{align*}
%
We analyze each of these cases separately.

There are several approaches to analyzing $I_{R,1}$ using local smoothing, and they all don't seem to work so well:
%
\begin{itemize}
    \item We apply H\"{o}lder to write
    %
    \begin{align*}
        |I_{R,1} f| &\leq \| a_{R,1} \|_{L^p} \left( \int_{1/R \lesssim |t| \leq 1/2} |e^{2 \pi i t \sqrt{-\Delta}} f|^{p^*} \right)^{1/p^*}\\
        &\leq C_p(h) R^{1/p^*} \left( \int_{1/R \lesssim |t| \leq 1/2} |e^{2 \pi i t \sqrt{-\Delta}} f|^{p^*} \right)^{1/p^*}.
    \end{align*}
    %
    Now by periodicity, we should have
    %
    \[ \left( \int_{1/R \lesssim |t| \leq 1/2} |e^{2 \pi i t \sqrt{-\Delta}} f|^{p^*} \right)^{1/p^*} \lesssim \left( R^{-1} \int_{1/R \lesssim |t| \lesssim R} |e^{2 \pi i t \sqrt{-\Delta}} f|^{p^*} \right)^{1/p^*}. \]
    % 
    and local smoothing implies that
    %
    \[ \left( R^{-1} \int_{1/R \lesssim |t| \lesssim R} \| e^{2 \pi i t \sqrt{-\Delta}} f \|_{L^{p^*}}^{p^*} \right)^{1/p^*} \lesssim \| f \|_{L^{p^*}_{\alpha_{p^*}}} \lesssim R^{\alpha_{p^*}} \| f \|_{L^{p^*}}. \]
    %
    But this means that
    %
    \[ \| I_{R,1} f \|_{L^{p^*}} \leq C_p(h) R^{1/p^* + \alpha_{p^*}} \| f \|_{L^{p^*}} = C_p(h) R^{(d-1)(1/2 - 1/p^*)} \| f \|_{L^{p^*}}. \]
    %
    This isn't a good enough bound to obtain what was required.

    \item We could write
    %
    \[ I_{R,1} = \sum_{1 \lesssim |k| \lesssim R} I_{R,k}, \]
    %
    where
    %
    \[ I_{R,k} = \int \eta(Rt - k) R \widehat{h}(Rt) e^{2 \pi i t \sqrt{-\Delta}} = \int a_{R,k}(t) e^{2 \pi i t \sqrt{-\Delta}}. \]
    %
    %\begin{align*}
    %    &\left( \sum_{1 \lesssim |k| \lesssim R} \| a_{R,k} \|_{L^p}^p \right)^{1/p}\\
    %    &\quad\quad \sim R^{1-(d-1)(1/p - 1/2)} \left( \int \sum_{1 \lesssim |k| \lesssim R} |\eta(Rt-k)|^p |\widehat{h}(Rt)|^p \right)^{1/p}\\
    %    &\quad\quad \sim R^{1/p^* - (d-1)(1/p - 1/2)} \left( \int_{|t| \sim 1} |\widehat{h}(t)|^p\; dt \right)^{1/p}\\
    %    &\quad\quad \lesssim R^{1/p^* - (d-1)(1/p - 1/2)} C_p(h).
    %\end{align*}
    %
    If we were able to justify that for all $R > 0$, and all $1 \lesssim k \lesssim R$,
    %
    \[ \left( R \int_{\frac{k}{R}}^{\frac{k + 1}{R}} \| e^{2 \pi i t \sqrt{-\Delta}} f \|_{L^{p^*}}^{p^*}\; dt \right)^{1/p^*} \lesssim \| f \|_{L^{p^*}_{\alpha_{p^*}}}, \]
    %
    then H\"{o}lder would imply that
    %
    \begin{align*}
        \| I_{R,k} f \|_{L^{p^*}(M)} &\lesssim \| a_{R,k} \|_{L^p} \left( \int_{|t| = k/R + O(1/R)} \| e^{2 \pi i t \sqrt{-\Delta}} f \|_{L^{p^*}}^{p^*}\; dt \right)^{1/p^*}\\
        &\lesssim \| a_{R,k} \|_{L^p} R^{(d-1)(1/2 - 1/p^*)-2/p^*} \| f \|_{L^{p^*}}.
    \end{align*}
    %
    %On the other hand, for $k \gtrsim 1$, we can apply the fixed time Sobolev estimates for the wave equation together with H\"{o}lder's inequality to conclude that
    % If a_{R,k} has height H_{R,k}
    % then (sum |k|^{(d-1)(1 - p/2)} H_{R,k}^p)^{1/p} << R^{1 - (d-1)(1/p - 1/2)}
    % for p <= r
    %\begin{align*}
    %    \| I_{R,k} f \|_{L^{p^*}(M)} &\lesssim \| a_{R,k} \|_{L^p} \| e^{2 \pi i t \sqrt{-\Delta}} f \|_{L^{p^*}_x(M) L^{p^*}_t[-1/R,1/R]}\\
    %    &\lesssim \| a_{R,k} \|_{L^p} R^{(d-1)(1/2 - 1/p^*)-1/p^*} \| f \|_{L^{p^*}}.
    %\end{align*}
    %
    Trivially applying the triangle inequality and H\"{o}lder's inequality yields that
    %
    \begin{align*}
        \sum_{1 \lesssim |k| \lesssim R} \| I_{R,k} f \|_{L^{p^*}(M)} &\lesssim R^{(d-1)(1/2 - 1/p^*)-2/p^*} \left( \sum_{1 \lesssim |k| \lesssim R} \| a_{R,k} \|_{L^p} \right) \| f \|_{L^{p^*}}\\
        &\lesssim C_p(h) \| f \|_{L^{p^*}},
    \end{align*}
    %
    This is a bound uniform in $R$.

    Unfortunately, we cannot expect to obtain a bound of the form
    %
    \[ \left( R \int_{\frac{k}{R}}^{\frac{k + 1}{R}} \| e^{2 \pi i t \sqrt{-\Delta}} f \|_{L^{p^*}}^{p^*}\; dt \right)^{1/p^*} \lesssim \| f \|_{L^{p^*}_{\alpha_{p^*}}} \]
    %
    which holds uniformly in $k$ and $R$, even if $f$ is a linear combination of eigenfunctions with eigenvalue about $R$. To see why, let $k = R/2$, and let $f$ be a sum of wave packets of the form
    %
    \[ f(x) = \sum e^{- 2 \pi i R \nu_\theta \cdot x} f_\theta(x), \]
    %
    where $\{ \theta \}$ is a collection of $O(1/\delta^{d-1})$ caps on an annulus of radius $\sim 1$ and thickness $\delta^2$, where the caps have dimension $\delta^2$ in the direction of the normal vector $\nu_\theta$ to $\theta$, and $\delta$ in tangential directions. The functions $\{ f_\theta \}$ are then bump functions adapted to $\theta$. By the uncertainty principle, we can make the frequency support of $f_\theta$ be mostly supported on $|\xi| \sim R$ provided that $\delta \gtrsim R^{-1/2}$. Let us set $\delta = R^{-1/2}$. Then we have
    %
    \[ |\nabla^\alpha f| \sim_\alpha \max(R^\alpha,\delta^{-2\alpha}) = R^\alpha. \]
    %
    Thus if $\delta \gtrsim R^{-1/2}$, then $|\nabla^\alpha f| \sim_\alpha R^\alpha$, and we get that
    %
    \[ \| f \|_{L^{p^*}_\alpha} \sim_\alpha R^\alpha \delta^{2/p^*} = R^{\alpha - 1/p^*}. \]
    %
    Now let us consider
    %
    \[ \int_{1/2}^{1/2 + 1/R} \| e^{2 \pi i t \sqrt{-\Delta}} f \|_{L^{p^*}}^{p^*}\; dt. \]
    %
    The frequency support of $e^{-2 \pi i R \nu_\theta} f_\theta$ is concentrated on the dual cap $\theta^*$ centered at $-  R \nu_\theta$. In particular, we have uncertainty $O(R)$ in the normal direction, and uncertainty $O(R^{1/2})$ in the tangential direction. But this means we have an angular uncertainty of about $O(1/R^{1/2})$, so that at each time $t \in [1/2, 1/2 + 1/R]$, $e^{2 \pi i t \sqrt{-\Delta}}$ has the majority of it's mass on a cap $\theta_t$, which is a dilation of $\theta$ by a constant factor.

    In particular, $\theta_t$ contains a ball of radius $1/R$, and on this ball, has real part $\gtrsim 1/2$. But this means that $e^{2 \pi i t \sqrt{-\Delta}} f$ has mass at least $O(1/\delta^{d-1})$ on this ball, and so
    %
    \[ \| e^{2 \pi i t \sqrt{-\Delta}} f \|_{L^{p^*}} \gtrsim \frac{R^{-d/p^*}}{\delta^{d-1}} = R^{ \frac{d-1}{2} - \frac{d}{p^*} }. \]
    %
    Thus if the bound held uniformly in $R$, we would have
    %
    \[ R^{ \frac{d-1}{2} - \frac{d}{p^*} } \lesssim R^{\alpha - 1/p^*}, \]
    %
    and thus we would have
    %
    \[ \alpha \geq \frac{d-1}{2} + \frac{1}{p^*} - \frac{d}{p^*} = (d-1) \left( 1/2 - 1/p^* \right). \]
    %
    In other words, this bound can \emph{only} hold as a fixed time estimate form an interval of length $1/R$ significantly away from the origin.

    If we instead perform this construction at some other $k \gtrsim 1$, we would get estimates of the form
    %
    \[ \| e^{2 \pi i t \sqrt{-\Delta}} f \|_{L^{p^*}} \gtrsim (k/R)^{d-1} \frac{R^{-d/p^*}}{\delta^{d-1}} = k^{d-1} R^{-\frac{d-1}{2} - \frac{d}{p^*} }. \]
    %
    Thus if we had a bound uniform in $R$ for this particular value of $k$, we would expect that
    %
    \[ k^{d-1} R^{- \frac{d-1}{2} - \frac{d}{p^*}} \lesssim R^{\alpha - 1/p}. \]
    %
    Thus
    %
    \begin{align*}
        \alpha &\geq (d-1) \log_R(k) + \frac{1}{p} - \frac{d-1}{2} - \frac{d}{p^*}\\
        &= (d-1)(\log_R(k) - 1) + (d-1)(1/2 - 1/p).
    \end{align*}
    %
    Thus it might still be possible to get a uniform bound in the range
    %
    \[ k \lesssim R^{1 - \frac{1}{(d-1) p^*}}. \]
    %
    TODO: Maybe look into seeing if this is provable, which would then allow us to restrict to a time interval of length $O(R^{-1/(d-1) p^*})$.
\end{itemize}

%Let us write $I_R = \sum_{|k| \lesssim R} I_{R,k}$, where
%
%
%The fact that $h$ has compact support implies that each of the functions $\{ a_{R,k} \}$ is morally a constant multiple of the indicator function supported on a time interval of length $\sim 1/R$, centered at the point $k/R$. Now in this case we also have that for $|k| \lesssim 1$,
%
%\[ \| a_{R,k} \|_{L^p} \lesssim R^{1/p^*} C_p(h) \]
%
%and
%
%\begin{align*}
%    &\left( \sum_{1 \lesssim |k| \lesssim R} \| a_{R,k} \|_{L^p}^p \right)^{1/p}\\
%    &\quad\quad \sim R^{1-(d-1)(1/p - 1/2)} \left( \int \sum_{1 \lesssim |k| \lesssim R} |\eta(Rt-k)|^p |\widehat{h}(Rt)|^p \right)^{1/p}\\
%    &\quad\quad \sim R^{1/p^* - (d-1)(1/p - 1/2)} \left( \int_{|t| \sim 1} |\widehat{h}(t)|^p\; dt \right)^{1/p}\\
%    &\quad\quad \lesssim R^{1/p^* - (d-1)(1/p - 1/2)} C_p(h).
%\end{align*}
%
%We break our analysis into these cases separately.

%To study $I_{R,k}$ for $|k| \lesssim 1$, we just take an inverse Fourier transform, i.e. writing
%
%\[ I_{R,k} = m_{k,R} \Big( \sqrt{-\Delta} \Big), \]
%
%where, if $\beta$ denotes the inverse Fourier transform of $\eta$, then
% eta(Rt - k)
% has inverse Fourier transform
% b(lambda) = int eta(Rt - k) e^{2 pi i t lambda}
% = 1/R * int eta(t - k) e^{2 pi i t lambda / R}
% = e^{2 pi i lambda k / R} 1/R int eta(t) e^{2 pi i lambda t / R}
% int eta(t) e^{- 2 pi i lambda (t + k)} beta(lambda / R)
%\[ m_{k,R}(\lambda) = \int h(\alpha) e^{- 2 \pi i \left( \frac{\lambda - \alpha}{R} \right) k} \beta \left( \frac{\lambda - \alpha}{R} \right)\; d\alpha. \]
%
%Then for $\lambda \geq 10$,
%
%\[ |D^N m_{k,R}(\lambda)| \lesssim_{N,M} \| h \|_{L^\infty(\RR)} R^{-N} \langle \lambda / R \rangle^{-M}. \]
%
%But this implies that $\{ m_{k,R} \}$ are a family of symbols of order zero, uniformly bouned in this symbol class for all $k$ and $R$. But this implies that $I_{R,k}$ is bounded in $L^p(M)$ for all $1 < p < \infty$. Thus $\sum_{|k| \lesssim 1} I_{R,k}$ is bounded in $L^p(M)$ for $1 < p < \infty$, and thus we immediately obtain $L^{p^*} \to L^{p^*}$ bounds which are uniform in $R$.

\begin{comment}
Using e.g. Kakeya like techniques, maybe we can shift the locations of the caps so that these caps have lots of overlap, i.e. so that if $\theta_t'$ denotes the closest $10\%$ of the cap $\theta_t$ to the origin, then for any $\varepsilon > 0$, if $R$ is suitably large,
%
\[ \bigcup \theta_t' \leq \varepsilon. \]
%
The wave packets have disjoint frequency support, so
%
\[ \| e^{2 \pi i t \sqrt{-\Delta}} f \|_{L^2(\RR^d)} \sim (1/\delta)^{\frac{d-1}{2}} \delta^{\frac{d+1}{2}} = \delta. \]
%
Applying dyadic pidgeonholing, this means that there exists a set $E$ with $|E| \leq \varepsilon$ such that
%
\[ |(e^{2 \pi i t \sqrt{-\Delta}} f)(x)| \sim \delta |E|^{-1/2} \]
%
for $x \in E$ and $t \in [1/2,1/2 + 1/R]$. But this means that
%
\[ \| e^{2 \pi i t \sqrt{-\Delta}} f \|_{L^{p^*}} \gtrsim \delta |E|^{1/p^* - 1/2} \]
%
But this means that
%
\[ |E|^{1/p^* - 1/2} \lesssim R^{\alpha - 1/p^* - 1/2} \]
% R^{alpha - 1/p^*}
Taking $\varepsilon \to 0$ gives that we must have $\alpha > 1/p^* + 1/2$.
\end{comment}

\section{Euclidean Case}

Can we prove a similar result for Euclidean multipliers? If we suppose the endpoint local smoothing conjecture held in Euclidean space at a particular exponent $2d/(d-1) < p^* < \infty$, is it then possible to prove the radial multiplier conjecture at the exponent $p$ (and by symmetry, also for $p^*$)? We do not have periodicity in this settings, so that `large times' are qualitatively different than small times. But we do have rescalability, i.e. so we do not need to introduce a parameter $R$ into the argument.

So let's consider a unit scale radial multiplier of the form
%
\[ T = h \Big( \sqrt{-\Delta} \Big), \]
%
To understand this operator, we consider the Fourier transform, writing
%
\[ Tf = \int_{-\infty}^\infty \widehat{h}(t) e^{2 \pi i t \sqrt{-\Delta}} f\; dt. \]
%
Let us assume that
%
\[ \int \langle t \rangle^{(d-1)(1 - p/2)} |\widehat{h}(t)|^p \; dt < \infty. \]
%
Suppose we set
%
\[ I f = \int \eta_0(t) \widehat{h}(t) e^{2 \pi i t \sqrt{-\Delta}}\; dt, \]
%
and
%
\[ II f = \sum_{|m| \gtrsim 1} T_m, \]
%
where
%
\[ T_m = \int \phi(t - m) \widehat{h}(t) e^{2 \pi i t \sqrt{-\Delta}}\; dt = \int a_m(t) e^{2 \pi i t \sqrt{-\Delta}}\; dt. \]
%
Then local smoothing (together with a dyadic decomposition if the local smoothing conjecture only is proved at a unit scale) implies that
%
\[ \| I f \|_{L^{p^*}(\RR^d)} \lesssim \| f \|_{L^{p*}(\RR^d)}. \]
%
On the other hand,
%
\[ \sum |m|^{(d-1)(1 - p/2)} \| a_m \|_{L^p[-1/2,1/2]}^p \lesssim 1. \]
%
H\"{o}lder's inequality thus implies that we have a pointwise bound of the form
%
\[ |T_m f| \lesssim \| a_m \|_{L^p[-1/2,1/2]} \left\| e^{2 \pi i t \sqrt{-\Delta}} f \right\|_{L^{p^*}[m-1/2,m+1/2]}. \]
%
Thus
%
\[ \| T_m f \|_{L^{p^*}(\RR^d)} \lesssim \| a_m \|_{L^p[-1/2,1/2]} \left\| e^{2 \pi i t \sqrt{-\Delta}} f \right\|_{L^{p^*}(\RR^d) L^{p^*}[m-1/2,m+1/2]}. \]
%
And so H\"{o}lder's inequality implies that
%
\[ \sum_{|m| \gtrsim 1} \| T_m f \|_{L^{p^*}(\RR^d)} \lesssim \left( \sum_{|m| \gtrsim 1} |m|^{-(d-1)(p^*/2 - 1)} \left\| e^{2 \pi i t \sqrt{-\Delta}} f \right\|_{L^{p^*}(\RR^d) L^{p^*}[m-1/2,m+1/2]}^{p^*} \right)^{1/p^*}. \]
%
Because of the choice of $p^*$, we \emph{could} replace the right hand side with a supremum in $m$. Thus our goal is to prove that
%
\[ \sup_{|m| \gtrsim 1} \left\| e^{2 \pi i t \sqrt{-\Delta}} f \right\|_{L^{p^*}(\RR^d) L^{p^*}[m-1/2,m+1/2]} \lesssim \| f \|_{L^{p^*}(\RR^d)}. \]
%
But this is a problem, since we cannot prove this bound, i.e. using a similar counterexample to the example considered before. We can also see why intuitively here by rescaling, i.e. writing this bound as an integral over the time interval $[ 1 - 1/2m, 1 + 1/2m ]$. For large $m$, it makes sense that this would quantity would behave like the pointwise wave equation.

%Suppose that
%
%\[ \sup_{R > 0} \int |\mathcal{F}(\varphi \text{Dil}_R m)(x)|\; dx < \infty. \]
%
%Let $m_k(\xi) = \phi(\xi/2^k) m(x)$. Then the condition says that the Fourier transform of $\text{Dil}_{1/2^k} m_k$ is uniformly bounded in $L^1$. But this tells us that
%
%\[ 2^{-kd} \int \text{Dil}_{2^k} \widehat{m_k}\; dx \lesssim 1, \]
%
%i.e. that
%
%\[ \int \widehat{m_k}(\xi)\; d\xi \lesssim 1. \]
%
%Thus the Fourier transform of $\widehat{m_k}$ is uniformly integrable in $k$, which tells us by Young's inequality that for $1 \leq p \leq \infty$, 
%
%\[ \| m_k(D) f \|_{L^p(\RR^d)} \lesssim \| f \|_{L^p(\RR^d)}. \]
%
%But now Littlewood-Paley theory tells us that for $1 < p < \infty$,
%
%\[ \| m(D) f \|_{L^p(\RR^d)} \lesssim_p \| f \|_{L^p(\RR^d)}. \]

\begin{comment}

\section{Inefficient Calculation}

% TODO: Maybe an L^infty norm is not tight here
% because the contribution from each slice of h^ is different
% The L^1 norm of H_R is equal to the L^1 norm of h^, which is uniform in R.
% The triangle inequality is the best we can do since each 1/R translate of h^ might look the same
% scaled down so that the integrability condition is satisfied
% We know that sum

\begin{align*}
    \left| \sum_{l = -\infty}^\infty \widehat{h}(R(t + l)) \right| &\lesssim \left( \sum_{l = -\infty}^\infty |\widehat{h}(R(t + l))|^p \langle R(t + l) \rangle^{(d-1)(1 - p/2)} \right)^{1/p}\\
    &\quad\quad\quad \left( \sum_{l \neq 0} \langle R(t + l) \rangle^{-(d-1)(p^*/2 - 1)} \right)^{1/p^*}\\
    &\lesssim C_p(h) R^{-(d-1)(1/2 - 1/p^*)}.
\end{align*}
%
TODO: Do we have mixed norm local smoothing estimates?

%
Thus we conclude that
%
\[ |H_R(t)| \lesssim C_p(h) R^{1-(d-1)(1/2 - 1/p^*)}. \]
%
This means that
%
\[ \left| \int H_R(t) e^{2 \pi i t \sqrt{-\Delta}} f \right| \lesssim C_p(h) R^{1-(d-1)(1/2 - 1/p^*)} \int_{-1/2}^{1/2} |e^{2 \pi i t \sqrt{-\Delta}} f |\; dt. \]



%
If $d(t,\mathbf{Z}) = s$ for some $s \in [0,1/2]$, then since $1 < p < 2d/(d+1)$, using the local constancy of $\widehat{h}$ given that $h$ is compactly supported, we have
%
\begin{align*}
    &\left( \sum_{l = -\infty}^\infty \langle R(t + l) \rangle^{-(d-1)(p^*/2 - 1)} \right)^{1/p^*} \sim \langle Rs \rangle^{-(d-1)(1/2 - 1/p^*)}.
%    &\quad \quad \sim \begin{cases} 1 &: t \leq 1/R, \\ (t/R) R^{-(d-1)(1/2 - 1/p^*)} t^{-(d-1)(1/2 - 1/p^*)} &: 1/R \leq t \leq 1/2. \end{cases}
\end{align*}
%
Thus
%
\[ |H_R(t)| \lesssim C_p(h) \Big( R \langle Rs \rangle^{-(d-1)(1/2 - 1/p^*)} \Big). \]
%
Write
%
\[ T_R f = \sum_{k = 0}^{O(\log R)} T_{R,k} f, \]
%
where
%
\[ T_{R,0} f = \int \tilde{\eta}(Rt) H_R(t) e^{2 \pi i t \sqrt{-\Delta}} f\; dt \]
%
and
%
\[ T_{R,k} f = \int \eta(R t / 2^k) H_R(t) e^{2 \pi i t \sqrt{-\Delta}} f\; dt. \]
%
% (R^{-1} alpha(t/R)) * (Sum_n h(n/R) delta_n)
% sum_n h(n/R) R^{-1} alpha((L - n)/R)
% Sum of O(R) terms,
% Bound by |h|_{l^1}
If $\alpha$ is the inverse Fourier transform of $\tilde{\eta}$, then $T_{R,0}$ corresponds to a multiplier operator $m_{R,0}(\sqrt{-\Delta})$, where
%
\[ m_{R,0}(\lambda) = R^{-1} \sum_n h(n/R) \alpha \Big( \frac{\lambda - n}{R} \Big). \]
%
But
%
\[ D^\beta m_{R,0}(\lambda) = R^{-1-\beta} \sum_n h(n/R) D^\beta \alpha \Big( \frac{\lambda - n}{R} \Big), \]
%
so that
%
\[ |D^\beta m_{R,0}(\lambda)| \lesssim_N R^{-\beta} \| h \|_{L^\infty} \langle \lambda/R \rangle^{-N} \lesssim \lambda^{-\beta} \| h \|_{L^\infty}. \]
%
Thus $m_{R,0}$ is a symbol of order zero, uniformly bounded by the $L^\infty$ norm of $h$, and so for $1 < p < \infty$,
%
\[ \| T_{R,0} f \|_{L^p(M)} \lesssim_p \| h \|_{L^\infty} \| f \|_{L^p(M)}. \]
%
A similar analysis gives that for $k > 0$,
%
\[ \| T_{R,k} f \|_{L^p(M)} \lesssim_p 2^k \| h \|_{L^\infty} \| f \|_{L^p(M)} \]
%
so we could reasonably use this bound for $k \lesssim 1$.

On the other hand, H\"{o}lder's inequality implies a pointwise bound of the form
%
\begin{align*}
    |T_{R,k} f| &\lesssim C_p(h) (R 2^{-k(d-1)(1/2 - 1/p^*)}) (2^k/R)^{1/p} \left( \int_{|t| \sim 2^k/R} |e^{2 \pi i t \sqrt{-\Delta}} f|^{p^*}\; dt \right)^{1/p^*}.
\end{align*}
%
Suppose the endpoint local smoothing conjecture held at the particular value $p^*$ we were consider, at all scales, i.e. so that
%
\[ \left\| \left( \int_{|t| \sim 2^k / R} |e^{2 \pi i t \sqrt{-\Delta}} f|^{p^*} \right)^{1/p^*} \right\|_{L^{p^*}(M)} \lesssim (2^k/R)^{1/p^*} \| f \|_{L^{p^*}_{\alpha_{p^*}}(M)} \]
%
where $\alpha_{p^*} = (d-1)(1/2 - 1/p^*) - 1/p^*$. Then, using the fact that
%
\[ \| f \|_{L^{p^*}_{\alpha_{p^*}}} \lesssim R^{\alpha_{p^*}} \| f \|_{L^{p^*}(M)}, \]
%
we conclude that
% d/p^* - (d-1)/2 < 0
%
\begin{align*}
    \| T_{R,k} f \|_{L^{p^*}(M)} &\lesssim C_p(h) (R 2^{-k(d-1)(1/2 - 1/p^*)} ) (2^k/R) R^{\alpha_{p^*}} \| f \|_{L^{p^*}(M)}\\
    &\lesssim C_p(h) R^{\alpha_{p^*}} 2^{k[ 1 - (d-1)(1/2 - 1/p^*) ]} \| f \|_{L^{p^*}(M)}.
%    \| T_{R,k} f \|_{L^{p^*}(M)} &\lesssim C_p(h) R^{1/p^* - (d-1)(1/2 - 1/p^*)} 2^{-k(d-1)(1/2 - 1/p^*)} (2^k / R)^{1/p^*} R^{\alpha_{p^*}} \| f \|_{L^{p^*}(M)}\\
%    &\lesssim C_p(h) 2^{k[d/p^* - (d-1)/2]} R^{-1/p^*} \| f \|_{L^{p^*}(M)}.
\end{align*}
%
% R^a 2^{-ak} <= O(1)
% a log R - ak log 2 <= O(1)
% k >= log_2 R - O(1/a)
% 2^k / R >= 1 / 100000
This bound is summable in $k$, resulting in a bound uniform in $R$, provided that $p \leq 2(d-1)/(d+1)$, and
%
\[ k \geq \frac{(d-1)(1/2 - 1/p^*) - 1/p^*}{(d-1)(1/2 - 1/p^*) - 1} \log R - O(1) \]


% 2(d-1)/(d-3) <= p^*
% (d-3)/2(d-1) >= 1 - 1/p
% p <= 2(d-1)/(d+1)
%
\[ k \geq \log_2(R) - O(1). \]
%
Thus we are left to analyze values $k$ with $O(1) \leq k \leq \log_2(R)$, which is equivalent, for instance, to analyzing times $100/R \leq |t| \leq 1/100$.

On the other hand, if we \emph{increase} the exponent in the integrability condition $C_p(h)$ by a $\varepsilon$, i.e. assuming the quantity
%
\[ C_{p,\varepsilon}(h) = \left( \int |\widehat{h}(t)|^p (1 + |t|)^{(d-1)(1 - p/2) + \varepsilon} \right)^{1/p} \]
%
is finite, then we have bounds of the form $|H_R(t)| \lesssim R \langle Rs \rangle^{-(d-1)(1/2 - 1/p^*) - \varepsilon}$, 

then the strategy above leads us to study the wave equation over a very very small set of times, i.e. $100/R \leq |t| \leq R^{-O(\varepsilon)}$.

Let's explore the analysis over these very very small set of times. Fix $\varepsilon > 0$, let $\phi \in C_c^\infty((0,\infty))$ and equal to one for $|t| \lesssim 1$, and consider an operator of the form
%
\[ T_R f = \int \phi( R^\varepsilon t) \left( \sum_{l = -\infty}^\infty R \widehat{h}(R(t + l)) \right) e^{2 \pi i t \sqrt{-\Delta}} f. \]
%
Let $a_R$ denote the inverse Fourier transform of $t \mapsto \phi(R^\varepsilon t) ( \sum_l R \widehat{h}(R(t + l)) )$. Then
% u . h_R
\[ a_R(\lambda) = R^{-\varepsilon} \sum_{\omega \in \ZZ} h \left( \frac{\lambda - \omega}{R} \right) \widehat{\phi}(\omega / R^\varepsilon) \]
%
In particular, for $N \geq 0$,
%
\[ D^N a_R(\lambda) = R^{-(N+1) \varepsilon} \sum_{\omega \in \ZZ} h \left( \frac{\lambda - \omega}{R} \right) D^N \widehat{\phi}(\omega / R^\varepsilon). \]
%
Thus we conclude that
%
\[ |D^N a_R(\lambda)| \lesssim_N R^{-N \varepsilon} \| h \|_{L^\infty} \]
%
In particular, if $|\lambda| \sim R$, then $|D^N a_R(\lambda)| \lesssim_N |\lambda|^{-N \varepsilon}$. In particular, if $\psi \in C_c^\infty(\RR)$ is supported on the annulus $|\lambda| \sim 1$, then the functions $\tilde{a}_R(\lambda) = \psi(\lambda / R) a_R(\lambda)$ uniformly lie in some symbol class $\mathcal{S}^0_\varepsilon(\RR)$, i.e. nonstandard symbols of order zero. We also have
%
\[ |D^N a_R(\lambda)| \lesssim_N R^{-(N + 1)\varepsilon} \left( \sup_{R'} \| \text{Dil}_{R'} h \|_{l^1(\ZZ)} \right), \]
%
which implies that under the assumption that the supremum in the inequality above is finite, then $\tilde{a}_R$ lies unformly in the symbol class $\mathcal{S}^{-\varepsilon}_\varepsilon(\RR)$. In the Euclidean setting, a bounded family of operators in $\mathcal{S}^m_\varepsilon(\RR^d)$ will be uniformly bounded in $L^p(\RR^d)$ for $m = - (1 - \varepsilon) d / 2$, so we should expect the condition is only sufficient if $\varepsilon \geq 1 - 2 / (d+2)$.

On the other hand, for smaller values of $k$, we might be able to get away with localizing, allowing us to reduce to an $L^2$ bound. The operators $T_{R,k}$ are `roughly' self-adjoint (since the wave equation is `roughly self adjoint' because of the time-reversal symmetry). Thus it suffices to prove an $L^p(M) \to L^p(M)$ estimate for these operators. If $f$ is supported on a ball of radius $2^k / R$, then we have
%
\[ \| T_{R,k} f \|_{L^p(M)} \lesssim (2^k / R)^{d(1/p - 1/2)} \| T_{R,k} f \|_{L^2(M)}. \]
%
Probably $L^2$ energy conservation should imply that we should use the estimate
%
\[ \| T_{R,k} f \|_{L^2(M)} \leq \int \eta(Rt/2^k) H_R(t) \| f \|_{L^2(M)} \lesssim 2^{k(1-(d-1)(1/p - 1/2))} \| f \|_{L^2(M)}. \]
%
Now using the Sobolev embedding and the fact that $f$ is, roughly speaking, supported at frequencies about $R$, we obtain that if $l = d(1/p - 1/2)$,
% k = 0
% p = 2
% l = ?
% q = p
% 1/2 = 1/p - l/d
% l/d = 1/p - 1/2
% l = d(1/p - 1/2)
\[ \| f \|_{L^2(M)} \lesssim \| f \|_{L^{l,p}(M)} \lesssim R^{d(1/p - 1/2)} \| f \|_{L^p(M)}. \]
%
Putting these bounds together, we conclude that
%
\begin{align*}
    \| T_{R,k} f \|_{L^p(M)} &\lesssim (2^k / R)^{d(1/p - 1/2)} 2^{k(1 - (d-1)(1/p - 1/2))} R^{d(1/p - 1/2)} \| f \|_{L^p(M)}\\
    &\lesssim 2^{k[ 1/2 + 1/p ]} \| f \|_{L^p(M)}.
\end{align*}
%
Summing over $1 \lesssim k \lesssim \log_2(R)$ gives a bound of the form $R^{1/2 + 1/p} \| f \|_{L^p(M)}$, which is not good enough, so this technique still doesn't help us. On the other hand, if we sum over $k \lesssim (1 - O(\varepsilon)) \log R$ gives a bound of the form $R^{(1 - O(\varepsilon))(1/2 + 1/p)}$ which is also not good enough.

\end{comment}

% R^a 2^{-(a + e) k } <= O(1)
% a log R - (a + e) k log 2 <= O(1)
% a log R - O(1) <= (a + e) k log 2
% k >= (1 - e/(a+e)) log_2 R - O(1)
% 2^k >= R^{}

\begin{comment}
%
Summing up gives
%
\[ \| T_{R,k} f \|_{L^{p^*}(M)} \lesssim R^{-1/p^*} \| f \|_{L^{p^*}(M)}. \]

Summing up this bound gives uniform bounds in $R \geq 1$. On the other hand, the ednpoint local smoothing conjecture also implies that
%
\[ \left\| \left( \int_{|t| \lesssim 1 / R} |e^{2 \pi i t \sqrt{-\Delta}} f|^{p^*} \right)^{1/p^*} \right\|_{L^{p^*}(M)} \lesssim R^{-1/p^*} \| f \|_{L^{p^*}_{\alpha_{p^*}}(M)}, \]
%
and so
%
\[ \| T_{R,0} f \|_{L^{p^*}(M)} \lesssim C_p(h) R^{\alpha_{p^*}} \| f \|_{L^{p^*}_{\alpha_{p^*}}(M)}. \]
%
This is problematic. To get the result required, we'd really like to have a `local local smoothing' estimate of the form
%
\[ \left\| \left( \int_{|t| \lesssim 1/R} |e^{2 \pi i t \sqrt{-\Delta}} f|^{p^*} \right)^{1/p^*} \right\|_{L^{p^*}(M)} \lesssim \| f \|_{L^{p^*}(M)} \]
%
as $R \to 0$.
\end{comment}

\begin{comment}

\section{Attempt \# 2: Unsuccessful?}

We break up $T_R f = \sum_{k = 0}^\infty T_{R,k} f$, where
%
\[ T_{R,0} f = \int_{-\infty}^\infty R \widehat{h}(Rt) \rho_0(Rt) e^{2 \pi i t \sqrt{-\Delta}} f\; dt \]
%
and for $k \geq 1$,
%
\[ T_{R,k} f = \int_{-\infty}^\infty R \widehat{h}(Rt) \rho(Rt / 2^k) e^{2 \pi i t \sqrt{-\Delta}} f\; dt. \]
%
Our goal is to obtain some exponential decay in $k$ so that we can use the triangle inequality to get uniform bounds on the whole operator.

First, let's deal with the $k = 0$ case. Let $m$ be the inverse Fourier transform of $\widehat{h} \cdot \rho_0$ then $T_{R,0} = m \left( \sqrt{-\Delta} / R \right)$. The function $m$ is smooth, i.e. it satisfies bounds of the form
%
\[ |\nabla^N m(\lambda)| \lesssim_{N,M} C_p(h) \langle \lambda \rangle^{-M} \]
%
for all $N$ and $M$. It follows by a manifold version of the H\"{o}rmander-Mikhlin theorem \cite{SeegerSogge} that $\| T_{R,0} f \|_{L^p(M)} \lesssim \| f \|_{L^p(M)}$ for all $1 < p < \infty$, uniformly in $R$, so this part of the operator is relatively trivial to analyze.

Now we deal with the $k > 0$ case. H\"{o}lder's inequality, combined with the trick of first multiplying and dividing by $(1 + |Rt|)^{(d-1)(1/p - 1/2)}$ implies that
% -(d-1)(p^*/2 - 1)
\begin{align*}
    |T_{R,k} f| &= \left| \int_{-\infty}^\infty R \widehat{h}(Rt) \rho(Rt / 2^k) e^{2 \pi i t \sqrt{-\Delta}} f\; dt \right|\\
    &\leq R \left( \int_{|t| \sim 2^k / R} \left( |\widehat{h}(Rt)| (1 + |Rt|)^{(d-1)(1/p - 1/2)} \right)^p\; dt \right)^{1/p}\\
    &\quad\quad\quad\left( \int_{|t| \sim 2^k / R} \left( |e^{2 \pi i t \sqrt{-\Delta}} f| (1 + |Rt|)^{-(d-1)(1/p - 1/2)} \right)^{p^*}\; dt \right)^{1/p^*}\\
    &\lesssim R^{1 - 1/p} 2^{-k(d-1)(1/2 - 1/p^*)} C_{p,k}(h) \left( \int_{|t| \sim 2^k / R} |e^{2 \pi i t \sqrt{-\Delta}} f|^{p^*}\; dt \right)^{1/p^*}.
\end{align*}
% -(d-1)(p^*/2 - p^* + 1))
% R^{1 - 1/p} 2^{-k(d-1)(p^* / 2 - 1)}
%
where
%
\[ C_{p,k}(h) = \left( \int_{|t| \sim 2^k} |\widehat{h}(t)|^p \cdot 2^{k(d-1)(1 - p/2)}\; dt \right)^{1/p} \]
%
Applying the periodicity of the wave equation, for $2^k \geq R$ we have
%
\[ \left( \int_{|t| \sim 2^k/R} |e^{2 \pi i t \sqrt{-\Delta}} f|^{p^*} \right)^{1/p^*} \lesssim (2^k/R)^{1/p^*} \left( \int_{|t| \lesssim 1} |e^{2 \pi i t \sqrt{-\Delta}} f|^{p^*} \right)^{1/p^*}. \]
%
If the endpoint local smoothing conjecture held, then we would have
%
\[ \left( \int_{|t| \lesssim 1} |e^{2 \pi i t \sqrt{-\Delta}} f|^{p^*} \right)^{1/p^*} \lesssim \| f \|_{L^{p^*}_{\alpha_p}}, \]
%
where
%
\[ \alpha_p = d(1/2 - 1/p) - 1/2. \]
%
But applying a Sobolev embedding, since $f$ is a sum of eigenfunctions of the Laplace-Beltrami operator with eigenvalue $\sim R$, we conclude that
%
\[ \| f \|_{L^{p^*}_{\alpha_p}} \lesssim R^{\alpha_p} \| f \|_{L^p} \]
%
Thus we conclude that
%
\begin{align*}
    \| T_{R,k} f \|_{L^p} &\lesssim R^{1 - 1/p} 2^{-k(d-1)(1/2 - 1/p^*)} (2^k/R)^{1/p^*} R^{d(1/p - 1/2) - 1/2} C_{p,k}(h) \| f \|_{L^p}\\
    &\lesssim R^{d(1/p - 1/2) - 1/2} 2^{k[d/p^* -(d-1)/2]} C_{p,k}(h) \| f \|_{L^p}.
\end{align*}
%
Provided that $p < 2d/(d+1)$, this bound is summable in $k$. And provided that $p \geq 2d/(d+1)$, the bound is uniform in $R \geq 1$. This indicates that we're dealing with is `precisely' at the endpoint in some sense.

In particular, if we do these calculations replacing $C_p(h)$ with $C_{p,\varepsilon}(h)$, where
%
\[ C_{p,\varepsilon}(h) = \left( \int_{-\infty}^\infty \left( |\widehat{h}(s)| (1 + |s|)^{(d-1)(1/p - 1/2) + \varepsilon} \right)^p \right)^{1/p}, \]
%
then for $2d/(d+1) \leq p < 2d/(d+1) + O(\varepsilon)$, we get uniform boundedness in $L^p$ as we vary $R$.
%then for $2d/(d+1) \leq p < 2d/(d + 1 - 2 \varepsilon) =$, we get uniform boundedness on $L^p$.

% 2d/(d+1-2e) - 2d/(d+1) = 2d(d+1) - 2d(d+1 - 2epsilon) / (d+1)(d + 1 - 2epsilon)

% 4d/(d+1)^2 epsilon
% 4/d

\end{comment}

\begin{comment}

If we choose a parameter $l$ such that
%
\[ 1/p^* - \alpha_p / d = 1/p - l/d \]
%
then 

% 1/p^* - alpha/d = 1/p
% 1 - alpha/d = 2/p
% d(1 - 2/p) = alpha
% d(1 - 2/p) >= d(1/2 - 1/p) - 1/2
% 1/2 >= d(1/p - 1/2)
% 1/p <= (1 + d)/2d
% p >= 2d/(d+1) = 2 - 2/(d+1).

Since $f$ is the span of eigenfunctions with eigenvalue $\lesssim R$, we have
%
\[ \| f \|{L^{p^*}_{\alpha_p}} \lesssim R^{\alpha_p} \| f \|_{L^{p^*}}. \]
%
Thus we conclude that
%
\[ \left( \int_{|t| \lesssim 1} |e^{2 \pi i t \sqrt{-\Delta}} f|^{p^*} \right)^{1/p^*} \lesssim R^{\alpha_p} \| f \|_{L^{p^*}}. \]
%
Putting all these bounds together yields that
%
\[ \| h(\sqrt{-\Delta} / R) f \|_{L^p} \lesssim R^{1 - 1/p} 2^{-k (d-1) (1/2 - 1/p^*)} C_p(h) (2^k / R)^{1/p^*} R^{\alpha_p} \| f \|_{L^{p^*}} \]




%
\[ \left( (1 + |Rt|)^{- (d-1)\frac{(2 - p)}{2(p-1)}} \right)^{1/p^*} \sim R^{-(d-1) \frac{(2-p)(1 - 1/p)}{2(p-1)}} |t|^{-(d-1) \frac{2-p}{2(p-1)}} \left( |Rt|^{- (d-1)\frac{(2 - p)}{2(p-1)}} \right)^{1/p^*}. \]
%
Applying the periodicity of the wave equation, we find that
%
\begin{align*}
    &\left( \int_{-\infty}^\infty |e^{2 \pi i t \sqrt{-\Delta}} f|^{p^*} (1 + |Rt|)^{- (d-1)\frac{(2 - p)}{2(p-1)}} \right)^{1/p^*}\\
    &\quad\quad s
\end{align*}

% the periodicity of the wave operators $\{ e^{2 \pi i t \sqrt{-\Delta}} \}$, and the fact that $1 < p < 2d/(d+1)$ allow us to obtain
%
\[  \]
%
\begin{align*}
    &\left| \int_{-\infty}^\infty R \widehat{h}(Rt) e^{2 \pi i t \sqrt{-\Delta}} f\; dt \right|\\
    &\quad\quad\lesssim R^{1/2 + d(1/2 - 1/p)}
\end{align*}
% R^{-(d-1)(2 - p)/2(p-1)} provided that p < 2d/(d+1) = 2/(1 + 1/d)
% \[ R^{1/2 + d(1/2 - 1/p)} \]


\emph{If} we are able to guarantee that $\widehat{h}(t) = 0$ for $t \lesssim 1$,

%

\[ R^{-(d-1)(2-p)/2(p-1)} \sum_{m = 1}^\infty |Rk|^{-(d-1)(2-p)/2(p-1)} \]
%
\end{comment}

\begin{comment}

The small time parameterix for the half-wave operator, combined with the composition calculus of Fourier integral operators, allows us to write, for $|t| \leq 1$,
%
\[ e^{2 \pi i t \sqrt{-\Delta}} f = T_t f + S^\infty_t f, \]
%
where $S^\infty_t$ is a \emph{smoothing operator}, i.e. an integral operator with
%
\[ S^\infty_t f(x) = \int K(t,x,y) f(y)\; dy, \]
%
where $K \in C^\infty([-1,1] \times S^n \times S^n)$, and where we can locally write
%
\[ T_t f(x) = \int_{\RR^n} a(t,x,y,\xi) e^{2 \pi i \Phi(x,y,\xi)} f(y)\; d\xi\; dy \]
% p(x,\xi) = \sqrt{-\Delta}
for some symbol $a \in S^0$, and some symbol $\Phi \in S^1$ satisfying
%
\[ \Phi(x,y,t,\xi) = (x - y) \cdot \xi + t g_y(\xi,\xi) + O(|x - y|^2 |\xi|). \]
%
To calculate the canonical relation of this operator, we look at the principal symbol of the operator $\sqrt{-\Delta}$. If $g$ is the metric of $S^n$, then the principal symbol will be
%
\[ p(x,\xi) = C \left( \sum g_{ij}(x) \xi^i \xi^j \right)^{1/2} = C |\xi| \]
%
for an appropriate constant $C$ (TODO: Do this calculation more precisely). One can calculate (see Remark in Section 4.1 Sogge's book) that the canonical relation of the family of operators $\{ T_t \}$ is given by
%
\[ \mathcal{C} \subset \{ (x,t,\xi,\tau,y,\eta) : (y,\eta) = \phi_t(x,\xi), \tau = p(x,\xi) \}, \]
%
where $t \mapsto \phi_t(x,\xi)$ is the geodesic travelling at a velocity of $2\pi$ which starts at $x$, and travels in the direction given by $\xi$. We claim this canonical relation satisfies the cinematic curvature condition. Indeed, the projections $\Pi_{y,\eta}: \mathcal{C} \to T^* S^n$ and $\Pi_{x,t}: \mathcal{C} \to T^* S^n$ are both submersions, so the Fourier integral operator is nondegenerate. For each pair $(x_0,t_0)$, the cone
%
\[ \mathcal{C}_{x_0,t_0} = \{ (\xi,\tau) : |\xi| = C^{-1} \tau \} \]
%
is an $n$ dimensional hypersurface in $T_{x_0,t_0}^*((-1,1) \times S^n)$, and it is easy to see this hypersurface is curved for all $t$. The endpoint local smoothing conjecture claims that if $f \in L^{p^*}(S^{n-1})$ for precisely the range of $p$ we care about in the radial multiplier conjecture, then $T f \in L^{p^*}_{-\alpha_p}((-1,1) \times S^{n-1})$, where $\alpha_p = n(1/2 - 1/p^*) - 1/2 = n(1/p - 1/2) - 1/2$. In particular, if we assume that Sobolev norms work on $S^n$ the same way they work on $\RR^n$, this means that if $f \in L^{p^*}(S^{n-1})$ has frequency supported on $|\xi| \leq L$, then
%
\[ \| T_R f \|_{L^{p^*}(\RR^d)} \lesssim \| f \|_{L^{p^*}(\RR^d)_{\alpha_p}} \lesssim L^{\alpha_p} \| f \|_{L^{p^*}(\RR^d)}. \]
%
Thus we see that local smoothing is pretty hopeless in proving the result we need to prove for general $f$.

The only non optimal inequality we applied here was H\"{o}lder's inequality, which would be tight if there exists a non-negative function $\gamma$ such that
%
\[ |e^{2 \pi i t \sqrt{-\Delta}} f|^{p^*} (1 + |Rt|)^{-(d-1)\frac{(2-p)}{2(p-1)}} = \gamma(x)^{p^*} |\widehat{h}(Rt)|^p (1 + |Rt|)^{(d-1)(1-p/2)}, \]
%
i.e. where
%
\[ |e^{2 \pi i t \sqrt{-\Delta}} f(x)| = \gamma(x) (1 + |Rt|)^{(d-1) \frac{(2-p)}{2}} |\widehat{h}(Rt)|^{1/p^*}. \]

TODO: Think about why local smoothing is useless. Is the theorem trivial if H\"{o}lder's inequality is applied?

%where $g$ is the Riemmanian metric of $S^n$. In particular, if we work with the coordinates $z_{\pm}$ in the strict upper and lower hemispheres given by
%
%\[ z_{\pm}^{-1}(t_1,\dots,t_n) = (t_1,\dots,t_n, \pm (1-t_1^2 - \dots - t_n^2)^{1/2}), \]
%
%then in coordinates we have
%
%\[ g_y(\xi,\xi) = |\xi|^2 - (1 - |y|^2)^{-1/2} |y \cdot \xi|^2, \]
%
%and so in these coordinates we have the explicit form
%
%\[ \Phi(x,y,s,\xi) = (x - y) \cdot \xi + t |\xi|^2 - t (1 - |y|^2)^{-1/2} |y \cdot \xi|^2 + O(|x - y|^2 |\xi|). \]
%
%TODO: IS THIS EXPLICIT FORM USEFUL?

\section{Junk}

Rescaling and applying H\"{o}lder's inequality, we have
%
\begin{align*}
    R \int_0^{2\pi} & \int_{\RR^d} \sum_{k = 0}^\infty w(R s + (2 \pi k) R) a(s,x,y,\xi) e^{2 \pi i \Phi(x,y,s/R,\xi)}\; d\xi\; ds\\
    &= \sum_{k = 0}^\infty \int_0^{2 \pi R} w(s + (2 \pi k) R) \int_{\RR^d} a(s/R,x,y,\xi) e^{2\ pi i \Phi(x,y,s/R,\xi)}\; d\xi\; ds\\
    &\leq \sum_{k = 0}^\infty \left( \int_0^{2\pi R} |w(s + (2 \pi k) R)|^q\; ds \right)^{1/q} \left( s \right)
\end{align*}


Now suppose that $\| w \|_{L^q(\RR^d, (1 + |x|)^{(d-1)(1 - q/2)})} < \infty$
%
\[ \int_0^{2\pi} \int_{\RR^d} \sum_{k = 0}^\infty b_t(R s + (2 \pi k) R) a(s,x,y,\xi) e^{2 \pi i \Phi(x,y,\xi)}\; d\xi\; ds, \]

Let us begin with the qualitative assumption that $h$ is compactly supported. Then, by breaking things up into a finite sum, we may assume that $h$ is supported on $[1/2,2]$. Fix a function $\chi \in C_c^\infty(\RR)$ equal to one on $[1/2,2]$, and vanishing outside of $[1/4,4]$. Write
%
\[ P_R = \chi \left( \sqrt{-\Delta} / R \right) = \sum \chi(\lambda / R) E_\lambda. \]
%
Then for any function $f \in C^\infty(S^n)$, $T_R f = T_R \{ P_R f \}$. Thus when bounding the behaviour of the operator $T_R$, we may assume inputs are linear combinations of eigenfunctions to $\sqrt{-\Delta}$ with eigenvalues $\lambda \sim R$.

\end{comment}







\chapter{Are Eigenfunctions of the Laplacian Locally Constant}

Suppose that $f \in C^\infty(M)$, and $\Delta f = - \lambda^2 f$. If $f$ was a function on $\RR^d$, the uncertainty principle would imply that $f$ was locally constant at a scale $1/\lambda$, i.e. such that for any $|x_0| \leq 1/\lambda$,
%
\[ |f(x_0)| \lesssim_N \lambda^d \int |f(x)| w_\lambda(x), \]
%
where $w_\lambda(x) = (1 + \lambda^2 |x|^2 )^{-N}$.

Does an analogous inequality hold for spherical harmonics? Perhaps we can use the Hecke-Funk formula. Namely, if $f \in V_R$, $\phi \in C_c^\infty(\RR)$ is non-negative, and equal to one on $1/2 \leq t \leq 2$, and
%
\[ P_R(t) = \sum_k \phi( k/R - 1 ) G_k(t), \]
%
then $f = P_R * f$. Experimental evidence (see code in GegenbauerSummationGraph.py) leads us to believe that $P_R$ is concentrated on a $1/R$ neighborhood of $t = 1$, and we should therefore expect to have a bound of the form
%
\[ |f(x)| \lesssim_N R^n \int_{S^n} w_N(R(x \cdot y)) |f(y)|\; dy, \]
%
where $w_N(t) = \langle t \rangle^{-N}$. Can we formally prove this is the case?









\chapter{Attempt Using Decoupling}

Let us try and attack our problem using decoupling. Fix a function $h: (0,\infty) \to \RR$, which is compactly supported on $\{ 1 < \lambda < 2 \}$, and use this function to induce a family of radial multiplier operators on the sphere $S^d$, of the form
%
\[ T_R f = \sum h(\lambda / R) \langle f, e_\lambda \rangle e_\lambda. \]
%
Our goal is to obtain bounds of the form
%
\[ \sup_{R > 0} \| T_R f \|_{L^p(S^d)} \lesssim \| f \|_{L^p(S^d)}. \]
%
Without loss of generality, as in the last chapter, because of the support of $h$, for a given $R$, we may assume that our inputs $f$ are a sum of eigenfunctions on the sphere with eigenvalue between $R$ and $2R$. We rewrite
%
\[ T_Rf = \int_{-\infty}^\infty R \widehat{h}(Rt) e^{2 \pi i t \sqrt{-\Delta}} f\; dt. \]
% We may assume h^ equals one in a neighborhood of the origin. Then
% Sum_{k = 0}^R R h^(Rk)
%
Since we are attempting to obtain bounds in the range that the local smoothing conjecture has been obtained, by the calculations in the last chapter, it suffices to analyze that part of the term above of the form
%
\[ T_Rf = \int_{1/R \lesssim |t| \lesssim 1} R \widehat{h}(Rt) e^{2 \pi i t \sqrt{-\Delta}} f\; dt. \]
%
%Using the fact that the wave equation is periodic of period one on $S^{d-1}$, we have
%
%\[ T_R f = \int_{-1/2}^{1/2} H_R(t) e^{2 \pi i t \sqrt{-\Delta}} f\; dt, \]
%
%where
%
% P = R
%\[ H_R(t) = R \sum_{l = -\infty}^\infty \widehat{h}(R(t + l)) = \sum_{R < l < 2R} h(l / R) e^{2 \pi i l t}. \]
%
In coordinates, modulo a smoothing operator, whose behaviour is negligible, we can write the kernel of $e^{2 \pi i t \sqrt{-\Delta}}$ as
%
\[ \int a(x,t,y,\xi) e^{2 \pi i \Phi(t,x,y,\xi)}\; d\xi, \]
%
where $a$ is a symbol of order zero with
%
\[ \text{supp}(a) \subset \{ (x,t,y,\xi): |x - y| \lesssim 1\ \text{and}\ |\xi| \gtrsim 1 \}, \]
%
and $\Phi(t,x,y,\xi) \approx (x - y) \cdot \xi + t |\xi|_g$. We can thus write the kernel of $T_R f$ as
%
\[ \int_{-1/2}^{1/2} \int R \widehat{h}(Rt) a(x,t,y,\xi) e^{2 \pi i \Phi(t,x,y,\xi)}\; d\xi. \]
%
We now decompose $T_R f$ in both frequency and time, writing
%
\[ T_R f = T_R^{\leq 0} + \sum_{k = 1}^{O(\log R)} \sum_{n = 1}^\infty T_{R,k}^{2^n} \]
%
where $T_{R,k}^{\lambda}$ has kernel
%
\[ \int_{-1/2}^{1/2} \int \beta(Rt/2^k) R \widehat{h}(Rt) a(x,t,y,\xi) \beta(\xi / \lambda) e^{2 \pi i \Phi(t,x,y,\xi)}\; d\xi\; dt, \]
%
and $T_R^{\leq 0}$ is supported on $|\xi| \leq 1$, and thus has the right $L^p$ bounds simply by applying the triangle inequality. By the choice of input $f$, we expect the majority of the contribution of the sum the operators $T_{R,k}^n$ should come from $n \approx \log R$. In the last chapter, we were able to obtain uniform bounds in $R$ by summing up $k \lesssim 1$ and $k \gtrsim \log R$, so it suffices to study the operators $T_R^{n,k}$ in the range $1 \lesssim k \lesssim \log R$.

We now try and apply the decoupling result of Beltran, Hickman, and Sogge; if we cover the unit sphere in $\RR^d_\xi$ by $O(\lambda^{-(d-1)/2})$ points $\Theta_\lambda$, consider an appropriate partition of unity $\{ \chi_\lambda^\nu : \nu \in \Theta_\lambda \}$, and thus write
%
\[ T_{R,k}^{\lambda} = \sum T_{R,k}^{\lambda,\nu}, \]
%
where $T_{R,k}^{\lambda,\nu}$ has kernel
%
\begin{align*}
    \int & \int \beta(Rt/2^k) R \widehat{h}(Rt) a(x,t,y,\xi) \beta(\xi / \lambda) \chi_\lambda^\nu(\xi) e^{2 \pi i \Phi(t,x,y,\xi)}\; d\xi\; dt\\
    &= \int R \widehat{h}(Rt) a_{\lambda,\nu,R,k}(x,t,y,\xi) e^{2 \pi i \Phi(t,x,y,\xi)}.
\end{align*}
%
Let us suppose that, using the techniques of their paper, we can show that
%
\[ \| T_{R,k}^\lambda f \|_{L^{p^*}(\RR^d)} \lesssim_{p,\varepsilon} \lambda^{\alpha_{p^*} + \varepsilon} \left( \sum_\nu \| T^{\lambda,\nu}_{R,k} f \|_{L^{p^*}(\RR^d)}^{p^*} \right)^{1/p^*}. \]
%
Thus it suffices to analyze the behaviour of each of the operators $T^{\lambda,\nu}_{R,k}$ separately. For each fixed $t$, energy conservation implies that
%
\begin{align*}
    \| T_{R,k}^{\lambda,\nu} f \|_{L^2(\RR^d)} &\lesssim \left( \int_{|t| \sim 2^k / R} R \widehat{h}(Rt)\; dt \right) \| f^\lambda_\nu \|_{L^2(\RR^d)} \\
    &\lesssim \int_{|t| \sim 2^k / R} R \langle Rs \rangle^{-(d-1)(1/2 - 1/p)} \| f^\lambda_\nu \|_{L^2(\RR^d)}\\
    &\lesssim (2^k)^{1 - (d-1)(1/2 - 1/p)} \| f^\lambda_\nu \|_{L^2(\RR^d)}.
\end{align*}
%
Thus $L^2$ orthogonality implies that
%
\[ \left( \sum_\nu \| T^{\lambda,\nu}_{R,k} f \|_{L^2(\RR^d)}^2 \right)^{1/2} \lesssim (2^k)^{1 - (d-1)(1/2 - 1/p)} \| f \|_{L^2(\RR^d)}. \]
%
On the other hand, to obtain an interpolation at $L^\infty$, we must understand the operator
%
\[ \sup_\nu \| T^{\lambda,\nu}_{R,k} f \|_{L^\infty(\RR^d)} \]
%
A stationary phase argument shows that, if we write
%
\[ T^{\lambda,\nu}_{R,k} f = \int_{|t| \sim 2^k / R} R \widehat{h}(Rt) T^{\lambda,t}_{R,\nu} f, \]
%
then the kernel $K^{\lambda,\nu,t}_{R}$ of $T^{\lambda,\nu,t}_{R}$ satisfies estimates of the form
%
\[ |K^{\lambda,\nu,t}_{R}(x,t;y)| \lesssim_N \frac{\lambda^{(d+1)/2}}{\Big\langle \lambda | \pi_\nu \nabla_\xi \Phi(x,y,t,\nu) | + \lambda^{1/2} | \pi_\nu^\perp \nabla_\xi \Phi(x,y,t,\nu) | \Big\rangle^N}. \]
%
Here we have $\nabla_\xi \Phi(x,y,t,\nu) = (x - y) + t \nu + O(|x - y|)$. Thus we conclude that, for a fixed $x$, this kernel has the majority of it's support on a cap centered at the point $x + t \nu$, with thickness $1/\lambda$ in the direction $\nu$, and thickness $1/\lambda^{1/2}$ in directions tangential to $\nu$. But this implies that the kernel of $K^{\lambda,\nu}_{R,k}$ is essentially supported on a $1/\lambda^{1/2}$ neighborhood of the line $\{ t \nu : |t| \sim 2^k/R \}$, and moreover, on that line we have
%
\[ |K^{\lambda,\nu}_{R,k}(x;t\nu)| \lesssim \int_{t - 1/\lambda}^{t + 1/\lambda} R \langle Rs \rangle^{-(d-1)(1/2 - 1/p^*)} \lambda^{(d+1)/2} \; ds \]
%
For $\lambda \geq R$, since $k \gtrsim 1$, and thus $|t| \geq 100/R$ we get that
%
\[ |K^\lambda_{R,\nu}(x;t\nu)| \lesssim \lambda^{-1} R^{1 - (d-1)(1/2 - 1/p^*)} t^{- (d-1)(1/2 - 1/p^*)}. \]
%
These same estimates hold replacing $t \nu$ with $t \nu + v$ for some $v$ perpendicular to $\nu$ with $|v| \leq \lambda^{-1/2}$. Thus we get that
%
\[ \int |K^\lambda_{R,\nu}(x;y)|\; dy \lesssim 2^k/R. \]
%
For $\lambda \leq R$, and $|t| \leq 10/\lambda$, we get that
%
\[ |K^\lambda_{R,\nu}(x;t\nu)| \lesssim \lambda^{(d+1)/2} \]
%
and for $|t| \geq 10/\lambda$, we get that
%
\[ |K^\lambda_{R,\nu}(x;t\nu)| \lesssim \lambda^{(d-1)/2} R^{1 - (d-1)(1/2-1/p^*)} t^{-(d-1)(1/2-1/p^*)}. \]
%
Integrating these results gives that
%
\[ \int |K^\lambda_{R,\nu}(x;y)|\; dy \lesssim 1, \]
%
the same bound as was obtained for $\lambda \geq R$. Thus Schur's Lemma gives
%
\[ \sup_\nu \| T^\lambda_{R,\nu} f \|_{L^\infty} \lesssim \| f \|_{L^\infty}. \]
%
%Using Poisson summation, we can write the kernel of the operator $T^\lambda_{R,\nu}$ as
%
%\begin{align*}
%    \sum_{R < l < 2R} &\int \int h(l/R) e^{2 \pi i l t} a_{\lambda,\nu}(x,t,y,\xi) e^{2 \pi i \Phi(t,x,y,\xi)}\; d\xi\; dt\\
%    &= \sum_{R < l < 2R} h(l/R) \int \int a_{\lambda,\nu}(x,t,y,\xi) e^{2 \pi i [\Phi(t,x,y,\xi) + t l]}.
    % Main contribution should occur for (x - y) + t xi / |xi| being <= 1/lambda
    % Then the nabla_t { Phi + tl } = |xi| + tl
%\end{align*}
% xi (x - y) + t(|xi| + l)
% The phase here should *always* be nonstationary in the t variable, so let's integrate by parts a bunch of times and see what happens.
%
% = sum_{R < l < 2R} h(l/R) 
%
%The function $H_R$ has decay away from $t = 0$, so to maximize this value we probably want the operator to maximize it's contribution near this value. Since we only care about the supremum over the values $\nu$, an extremizer will likely have $f = f^\lambda_\nu$. But this $f$ has Fourier support on a cap of thickness $\lambda$ and tangential thickness $\lambda^{1/2}$
%
Interpolating gives that
%
\[ \left( \sum \| T^\lambda_{R,\nu} f \|_{L^{p^*}}^{p^*} \right)^{1/p^*} \lesssim \| f \|_{L^{p^*}}, \]
%
and thus that
%
\[ \| T^\lambda_R f \|_{L^{p^*}} \lesssim_\varepsilon \lambda^{\alpha(p^*) + \varepsilon} \| f \|_{L^{p^*}}. \]








\chapter{Trying to Use Hadamard Parametrix}

Let
%
\[ T_R f = \sum_\lambda m(\lambda / R) \langle f, e_\lambda \rangle e_\lambda. \]
%
If we write
%
\[ M(t) = \int_0^\infty m(\lambda) \cos(2 \pi t \lambda)\; d\lambda, \]
%
then the inverse formula implies that
%
\[ m(\lambda / R) = \int M(t) \cos \left( \frac{2 \pi \lambda t}{R} \right)\; dt. \]
%
Thus we have
%
\[ T_R f = \int M(t) \cos \left( \frac{2 \pi \sqrt{-\Delta} t}{R} \right) f\; dt \]
%
Local smoothing implies that we need only analyze an integral of the form
%
\[ T_R f = R \int \eta(R t) M(R t) \cos \Big( 2 \pi \sqrt{-\Delta} \cdot t \Big) f\; dt, \]
%
where $\eta$ has support on an arbitrarily small, but fixed portion of the origin. Localizing the operator by a partition of unity, and then applying the Hadamard parametrix, we should expect to control the kernel $T_R f$ by a finite sum of functions of the form
%
\[ K_R(x,y) = R \int \int \frac{c(x,y) \eta(R t) M(R t)}{|\xi|^\nu} e^{2 \pi i (\xi \cdot d_g(x,y) + t |\xi|)}\; d\xi\; dR, \]
%
where $a$ is smooth and compactly supported, and $d_g$ denotes the geodesic distance on the manifold. Let us perform a frequency decomposition, writing $K_R = K_{R,0} + 2^{k(d - \nu)} \sum_{k = 1}^\infty K_{R,k}$, where
%
\[ K_{R,0}(x,t;y) = R \int M(Rt) \frac{c(x,y) \eta(R t) \psi_0(\xi)}{|\xi|^\nu} e^{2 \pi i (\xi \cdot d_g(x,y) + t |\xi|)}\; d\xi\; dt, \]
%
and for $k \geq 1$,
%
\[ K_{R,k}(x,t;y) = 2^{k(\nu - d)} R \int \int \frac{a(x,y) \eta(R t) M(R t) \tilde{\psi}( \xi / 2^k )}{|\xi|^\nu} e^{2 \pi i (\xi \cdot d_g(x,y) + t |\xi|)}\; d\xi\; dt. \]
%
Rescaling, and setting $a_\nu(x,t,y,\xi) =  a(x,y) \eta(t) \tilde{\psi}(\xi) / |\xi|^\nu$ gives that
%
\[ K_{R,k}(x,t;y) = \int M(t) \int a_\nu(x,t,y,\xi) e^{2 \pi i 2^k (\xi \cdot d_g(x,y) + (t / R) |\xi|)}\; d\xi\; dt. \]
%
Similarily,
%
\[ K_{R,0}(x,t;y) = \int M(t) a_{\nu,0}(x,y,t,\xi) e^{2 \pi i (\xi \cdot d_g(x,y) + (t/R) |\xi|)}\; d\xi\; dt. \]
%
Our goal is to obtain some $L^p$ estimates on this operator that are summable in $k$, thus obtaining bounds of the form
%
\[ \| T_{R,k} f \|_{L^p} \lesssim 2^{k(\nu - d - \varepsilon)} \]
%
for some $\varepsilon > 0$, which hold uniformly in $R$. The operator $T_{R,0}$ should not be an issue since one can just take in absolute values to obtain the required result.

Stationary phase tells us that the majority of the mass of the kernel $K_{R,k}$ should be concentrated on points $(x,t;y)$ where $|d_g(x,y) - t/R| \lesssim 2^{-k}$, a geodesic annulus of radius $t/R$, and thickness $2^{-k}$. If we are to try a decoupling result, let us split this annulus into a family $\Theta_k$ of sectors of aperture $2^{-k/2}$ with the finite itnersection property. We should require a set of sectors $\Theta_{R,k}$ with $\#(\Theta_k) \lesssim 2^{k(d-1)/2}$. If we consider a partition of unity $\{ \chi_\theta \}$ localizing the operator to these sectors, we can therefore write $T_{R,k} = \sum_\theta T_{R,k,\theta}$. Let us suppose a Wolff-type decoupling bound held for this operators, i.e. that
%
\[ \| T_{R,k} f \|_{L^p} \lesssim_\varepsilon 2^{k(\alpha(p) + \varepsilon)} \left( \sum_\theta \| T_{R,k,\theta} f \|_{L^p}^p \right)^{1/p}. \]
%
Let us thus analyze a particular one of these operators $T_{R,k,\theta}$, which has kernel
%
\[ K_{R,k,\theta}(x,t;y) = \int \int M(t) a_{\nu,\theta}(x,y,t,\xi) e^{2 \pi i 2^k (\xi \cdot d_g(x,y) + (t/R) |\xi|)}\; d\xi\; dt, \]
%
where $a_{\nu,\theta} = a_\nu \cdot \chi_\theta$. For each $t$, and $x$, nonstationary phase tells us that the mass of the kernel $K_{R,k,\theta}(x,t;\cdot)$ should be concentrated on a cap of long thickness $2^{-k}$ and short thicknesses $2^{-k/2}$, containing in the intersection of the sector $\theta$ and the $2^{-k}$ neighborhood of the annulus of radius $t/R$ centered at $y$. We should expect that on this cap the kernel should have amplitude equal to $O( M(t) 2^{-k(d-1)} )$. Thus we have
%
\[ \| T_{R,k,\theta} f \|_{L^\infty(\RR^d)} \lesssim C_p(m) 2^{-3k(d+1) / 2} \| f \|_{L^\infty} \]










\chapter{Attempt Using Heo-Nazarov-Seeger Technique}

Suppose $h$ is a radial multiplier with support on $\{ 1 \leq \lambda \leq 2 \}$, and let
%
\[ b(t) = 2 \int_0^\infty h(\lambda) \cos(2 \pi \lambda). \]
%
If $B_R(t) = R \sum_l b(R(t + l))$, our goal is to prove uniform $L^p$ bounds on the radial multiplier operator
%
\[ T_R = \int_1^2 B_R(t) e^{2 \pi i t \sqrt{-\Delta}}. \]
%
We may assume our input is a linear combination of eigenfunctions with eigenvalue between $R$ and $2R$. If we reduce to local coordinates, we can write
%
\[ e^{2 \pi i t \sqrt{-\Delta}} f(x) = \int_{\RR^d} a(x,t,y,\xi) e^{2 \pi i (\phi(x,y,\xi) - t |\xi|_g)} f(y)\; d\xi, \]
%
where $a$ is a compactly supported symbol of order zero, and where $\phi(x,y,\xi) \approx (x - y) \cdot \xi$ (this is only up to a smoothing operator, whose behaviour is irrelevant for the purposes of our argument since for such operator trivial $L^p$ estimates hold). Thus we have
%
\[ T_R f(x) = \int_{\RR^d} \int_{\RR^d} \int_1^2 B_R(t) a(x,t,y,\xi) e^{2 \pi i (\phi(x,y,\xi) - t)} f(y)\; dt\; dy\; dx. \]
%
Now let $\eta \in \mathcal{S}(\RR^d)$ be a Schwartz function vanishing to high order at the origin, and consider the operator
%
\[ \tilde{T}_R f(x) = \int_{\RR^d} \int_{\RR^d} \int_1^2 B_R(t) a(x,t,y,\xi) \eta( \xi ) e^{2 \pi i (\phi(x,y,\xi) - t |\xi|_g)} f(y)\; dt\; dy\; dx. \]
%
Then $\tilde{T}_R f = T_R \circ $

\[ \left\| \int_{\RR^d} \int_1^2 B_R(t) a(x,t,y,\xi) e^{2 \pi i (\phi(x,y,\xi) - t |\xi|_g)} f(y)\; d\xi\; dy\; dt \right\|. \]



% Hadamard Parametrix
% E_0 = |g|^{1/2} E_+(t, d_g(0,x))
% rho a_0(x) = 2 < x , nabla_x a_0 >,   a_0(0) = 1












\part{Papers I Don't Understand Yet}





\chapter{Seeger: Singular Convolution Operators in $L^p$ Spaces}

Let $m: \RR^d \to \CC$ be the symbol for a Fourier multiplier operator $m(D)$. If the resulting operator $m(D)$ was bounded from $L^p(\RR^d)$ to $L^p(\RR^d)$ with operator norm $A$, then the operator would also be bounded `at all scales'. That is, if we consider a littlewood Paley decomposition, i.e. taking
%
\[ f = \sum_{i = 0}^\infty f_i \]
%
where $\widehat{f_i} = \eta_i \widehat{f}$ is supported on $2^i \leq |\xi| \leq 2^{i+1}$ for $i \geq 1$, and $|\xi| \leq 2$ for $i = 0$, then we would have estimates of the form
%
\begin{equation} \label{piecewiseBound}
    \| m(D) f_i \|_{L^p(\RR^d)} \lesssim \| f_i \|_{L^p(\RR^d)} \lesssim \| f \|_{L^p(\RR^d)},
\end{equation}
%
where the implicit constant is uniform in $i$. The main focus of the paper in question is to determine whether a uniform bound of the form \eqref{piecewiseBound} implies $m(D)$ is bounded. More precisely, is it true that
%
\begin{equation} \label{operatorbound}
    \| m \|_{M^p(\RR^d)} \lesssim_p \sup\nolimits_{i \geq 0} \| m_i \|_{M^p(\RR^d)},
\end{equation}
%
where $m_i = \eta_i m$.

The Hilbert transform $H$ is a Fourier multiplier with symbol $m(\xi) = \text{sgn}(\xi)$. For each $i > 0$, $m_i(\xi) = \eta_i \text{sgn}(\xi)$, so that
%
\[ K_i(x) = \widehat{\eta_i \text{sgn}(\xi)} = 2^i H \eta(2^i x). \]
%
Thus
%
\[ \| K_i \|_{L^1(\RR)} = \| H \eta \|_{L^1(\RR)}. \]
%
TODO

It is clear that \eqref{operatorbound} is true for $p = 2$, since in this case the bound is equivalent to an inequality of the form
%
\[ \| m \|_{L^\infty(\RR^d)} \lesssim \sup\nolimits_{i \geq 0} \| m_i \|_{L^\infty(\RR^d)}, \]
%
which is true because the supports of the symbols $\{ m_i \}$ are almost all pairwise disjoint. On the other hand, \eqref{operatorbound} does not hold when $p = 1$ or $p = \infty$, which makes sense, since Littlewood-Paley runs into all kinds of problems for these values of $p$. Arguing more precisely, the condition would be equivalent to showing that for any $K: \RR^d \to \CC$,
%
\[ \| K \|_{L^1(\RR^d)} \lesssim \sup\nolimits_{i \geq 0} \| K * \widehat{\eta_i} \|_{L^1(\RR^d)}. \]
%
If
%
\[ K_N(x) = \int_{|\xi| \leq 2^N} e^{2 \pi i \xi \cdot x}\; d\xi \]
%
is the Dirichlet kernel, then $\| K_N \|_{L^1(\RR)} \sim N$. On the other hand, for $i \leq N-1$, we have $K_N * \widehat{\eta_i} = \widehat{\eta_i}$, so that
%
\[ \| K_N * \widehat{\eta_i} \|_{L^1(\RR)} = \| \widehat{\eta_i} \|_{L^1(\RR)} \lesssim 1. \]
%
For $i \geq N+1$, we have $K_N * \widehat{\eta_i} = 0$, so that
%
\[ \| K_N * \widehat{\eta_i} \|_{L^1(\RR)} = 0 \lesssim 1. \]
%
For $i = N$, we have
%
\[ (K_N * \widehat{\eta_N})(x) = 2^N \int_0^1 \eta(\xi) e^{2 \pi i 2^N (\xi \cdot x)} + \int_1^2 \eta(-\xi) e^{-2 \pi i 2^N (\xi \cdot x)}\; d\xi  \]
%
\[ \int |K_N * \widehat{\eta_i}| \]

whereas one
% 
%
\[ K_N * \widehat{\eta_i} = \begin{cases} \widehat{\eta_i} &: i \lesssim N \\ 0 &: i \gtrsim N \end{cases}, \]
%
and so $\| K_N * \widehat{\eta_i} \|_{L^1(\RR)} \lesssim 1$ uniformly in $N$ and $i$. We can then use Baire category techniques to find a kernel $K$ not in $L^1(\RR)$, but such that $\| K * \eta_i \|_{L^1(\RR)} \lesssim 1$, uniformly in $i$.

The result actually fails for $2 < p < \infty$, due to an examples of Triebel. For simplicity, let's work in $\RR$. If we fix a bump function $\phi \in C_c^\infty(\RR)$ supported in $[-1,1]$, and set
%
\[ m_N(\xi) = \sum_{k = N}^{2N} e^{2 \pi i (2^k \xi)} \phi(\xi - 2^k), \]
%
then $m_N(\xi) \eta_i(\xi) = m_{N,i}(\xi)$, where $m_{N,i}(\xi) = e^{2 \pi i (2^k \xi)} \phi(\xi - 2^k)$, and so $K_{N,i}(x) = \widehat{m_{N,i}}(x) = e^{2 \pi i 2^k(x - 2^k)} \widehat{\phi}(x - 2^k)$, hence
%
\[ \| m_{N,i}(D) f \|_{L^p(\RR^d)} = \| K_{N,i} * f \|_{L^p(\RR^d)} \leq \| \widehat{\phi} \|_{L^1(\RR)} \| f \|_{L^p(\RR)} \lesssim \| f \|_{L^p(\RR)}. \]
%
On the other hand, the operator norm of $m_N(D)$ from $L^p(\RR)$ to $L^p(\RR)$ is actually $\gtrsim_p N^{|1/p - 1/2|}$, and thus not bounded uniformly in $N$, so Baire category shows things don't work so well here.

This paper shows that one \emph{can} get uniform bounds assuming an additional, very weak smoothness condition, which rules out the example $m_N$ above. Under the most simple assumptions, if \eqref{piecewiseBound} holds, and $\| m_i \|_{\Lambda^\varepsilon} \lesssim 2^{-ik}$, where $\Lambda^\varepsilon$ is the $\varepsilon$-Lipschitz norm, then $\| m(D) f \|_{L^r(\RR^d)} \lesssim \| f \|_{L^r(\RR^d)}$ whenever $|1/r - 1/2| < |1/p - 1/2|$. Under slightly stronger smoothness assumptions, we can actually conclude $\| m(D) f \|_{L^p(\RR^d)} \lesssim \| f \|_{L^p(\RR^d)}$.

To prove the result, we rely on Littlewood-Paley theory and the Fefferman-Stein sharp maximal function. Without loss of generality we may assume that $2 < p < \infty$. We will actually show that if for all $i$ and $\omega \geq 0$,
%
\[ \int_{|x| \geq \omega} |K_i(x)|\; dx \leq B (1 + 2^i \omega)^{-\varepsilon}, \]
%
consistent with the fact that, if $m_i$ was smooth, the uncertainty principle would say that $K_i$ would live on a ball of radius $1/2^i$. We will then prove that $\| m(D) f \|_{L^p(\RR^d)} \leq A \widetilde{\log}(B/A)^{|1/2 - 1/p|}$. Our goal is to show that if
%
\[ S^\# f(x) = \sup_{x \in Q} \fint_Q \left( \sum_{i = 0}^\infty \left| m_i(D) f(y) - \fint_Q m_i(D) f(z)\; dz \right|^2 \right)^{1/2}\; dy, \]
%
then $\| S^\# f \|_{L^p(\RR^d)} \lesssim A \widetilde{\log}(B/A)^{1/2 - 1/p} \| f \|_{L^p(\RR^d)}$. It then follows by Littlewood-Paley theory implies
%
\begin{align*}
    \| m(D) f \|_{L^p(\RR^d)} &\lesssim_p \left\| \left( \sum_{k = 0}^\infty |m_i(D) f|^2 \right)^{1/2} \right\|_{L^p(\RR^d)}\\
    &\leq \left\| M \left[ \left( \sum_{k = 0}^\infty |m_i(D) f|^2 \right)^{1/2} \right] \right\|_{L^p(\RR^d)}\\
    &\lesssim \left\| S^\# \left( \sum_{k = 0}^\infty |m_i(D) f|^2 \right)^{1/2} \right\|_{L^p(\RR^d)}\\
    &\lesssim A \widetilde{\log}(B/A)^{1/2 - 1/p}.
\end{align*}
%
To bound $S^\#$, we linearize using duality, picking $Q_x$ for each $x$, and a family of functions $\chi_i(x,y)$ such that $\left( \sum |\chi_i(x,y)|^2 \right)^{1/2} \leq 1$, such that
%
\[ S^\# f(x) \approx \fint_{Q_x} \sum_{i = 0}^\infty \left( m_i(D) f(y) - \fint_{Q_x} m_i(D) f(z)\; dz \right) \chi_i(x,y)\; dy. \]
%
Thus $S^\# f = S_1 f + S_2 f$, where if $Q_x$ has sidelength $2^{l(x)}$,
%
\[ S_1 f(x) = \fint_{Q_x} \sum_{|i + l(x)| \leq \tilde\log(B/A)} \left( m_i(D) f(y) - \fint_{Q_x} m_i(D) f(z)\; dz \right) \chi_i(x,y)\; dy \]
%
and
%
\[ S_2 f(x) = \fint_{Q_x} \sum_{|i + l(x)| \geq \tilde\log(B/A)} \left( m_i(D) f(y) - \fint_{Q_x} m_i(D) f(z)\; dz \right) \chi_i(x,y)\; dy. \]
%
If $|i + l(x)| \lesssim 1$, then the uncertainty principle tells us that $m_i(D) f$ is roughly constant on squares on radius $Q_x$, up to some small error, so that we should expect
%
\[ \left| m_i(D) f(y) - \fint_{Q_x} m_i(D) f(z)\; dz \right| \lesssim \left| \fint_{Q_x} m_i(D) f(z)\; dz \right|. \]
%
Thus it is natural to use the bound, $|S_1 f(x)| \lesssim M(\sum_{i = 0}^\infty |m_i(D) f|^2)^{1/2}$, which implies
%
\begin{align*}
    \| S_1 f \|_{L^2(\RR^d)} &\lesssim \| M(\sum_{i = 0}^\infty |m_i(D) f|^2)^{1/2} \|_{L^2(\RR^d)}\\
    &\lesssim \left\| (\sum_{i = 0}^\infty |m_i(D) f|^2 )^{1/2} \right\|_{L^2(\RR^d)}\\
    &= \left( \sum_{i = 0}^\infty \| m_i(D) f \|_{L^2(\RR^d)}^2 \right)^{1/2}
\end{align*}
%
and
%
\begin{align*}
    \| S_1 f \|_{L^\infty(\RR^d)} &\leq \| M(\sum_{|i + l(x)| \leq \tilde{\log}(B/A)}^\infty |m_i(D) f|^2)^{1/2} \|_{L^\infty(\RR^d)}\\
    &\leq \left\| \left( \sum_{|i + l(x)| \leq \tilde{\log}(B/A)} |m_i(D) f|^2 \right)^{1/2} \right\|_{L^\infty(\RR^d)}\\
    &\lesssim \tilde{\log}(B/A)^{1/2} \sup_i  \| m_i(D) f \|_{L^\infty(\RR^d)}
\end{align*}
%
Interpolation gives $\| S_1 f \|_{L^p(\RR^d)} \lesssim \tilde{\log}(B/A)^{1/2 - 1/p} \| m_i(D) f \|_{L^p_x(l^p_i)}$. But now Littlewood-Paley theory shows that
%
\[ \| m_i(D) f \|_{L^p_x(l^p_i)} \leq A \left( \sum_{i = 0}^\infty \| P_i f \|_{L^p(\RR^d)} \right)^{1/p} \leq A \left( \sum_{i = 0}^\infty \| P_i f \|_{L^p(\RR^d)}^2 \right)^{1/2} \lesssim A \| f \|_{L^p}. \]
%
Thus $\| S_1 f \|_{L^p(\RR^d)} \lesssim A \tilde{\log}(B/A)^{1/2 - 1/p} \| f \|_{L^p(\RR^d)}$.

On the other hand, if $i$ is much smaller than $l(x)$, we should expect the error between $m_i(D) f(y)$ and $\fint_{Q_x} m_i(D) f(z)\; dz$ to be even smaller, and if $i$ is much bigger, then $m_i(D) f$ is no longer constant at this scale, and so the averages should be small, so $m_i(D) f(x)$ should dominate $\fint_{Q_x} m_i(D) f(z)$. Now since our assumption implues that $\| m(D) f \|_{L^2(\RR^d)} \lesssim \| f \|_{L^2(\RR^d)}$, it is not so difficult to prove that
%
\[ \| S_2 f \|_{L^2(\RR^d)} \lesssim A \| f \|_{L^2(\RR^d)} \sim A \left\| \left( \sum |P_i f|^2 \right)^{1/2} \right\|_{L^2(\RR^d)}. \]
%
The difficulty is proving $\| S_2 f \|_{L^\infty(\RR^d)} \lesssim A \| \left( \sum |P_i f|^{1/2} \right) \|_{L^\infty(\RR^d)}$, which we can interpolate into an inequality like above where we can apply Littlewood-Paley theory. To do this we perform another decomposition, writing
%
\[ S_2 f = I f + {II}f \]
%
where
%
\[ If(x) = \fint_{Q_x} \sum_{|i + l(x)| \geq \tilde\log(B/A)} \left( m_i(D) (\mathbf{I}_{2Q_x}f)(y) - \fint_{Q_x} m_i(D)(\mathbf{I}_{2Q_x} f)(z)\; dz \right) \chi_i(x,y)\; dy. \]
%
and
%
\[ If(x) = \fint_{Q_x} \sum_{|i + l(x)| \geq \tilde\log(B/A)} \left( m_i(D) (\mathbf{I}_{(2Q_x)^c}f)(y) - \fint_{Q_x} m_i(D)(\mathbf{I}_{(2Q_x)^c} f)(z)\; dz \right) \chi_i(x,y)\; dy. \]
%
Now
%
\[ \|If \|_{L^\infty} \leq \sup_x \fint_{Q_x} \left( \sum | m_i(D) (\mathbf{I}_{2Q_x} f) |^2 \right)^{1/2}\; dy \leq \sup_x |Q_x|^{-1/2} \left( \sum \| m_i(D) (\mathbf{I}_{2Q_x} f) \|_{L^2(\RR^d)}^2 \right)^{1/2} \lesssim A |Q_x|^{-1/2} \left( \| \mathbf{I}_{2Q_x} f \|_{L^2(\RR^d)}^2 \right)^{1/2} \lesssim A \left( \sum_{i = 0}^\infty \fint_{R_x} |f|^2 \right)^{1/2} \]




\part{Stuff to Read in More Detail}

\begin{itemize}
    \item Sogge, $L^p$ Estimates For the Wave Equation and Applications (1993).

    A survey of results on regularity results for the wave equation. In particular, reviews (without proof) the ideas of Mockenhaipt, Seeger, and Sogge which give local smoothing for Fourier integral operators satisfying the cone condition, as well as mixed norm estimates for non-homogeneous results on wave equations.

    \item In Sogge's Book, he mentions the main developments in harmonic / microlocal analysis he couldn't discuss in the book were the following:
    \begin{itemize}
        \item Bennett, Carbery, Tao, On the Multilinear Restriction and Kakeya Conjecture (2006).

        Introduction to multilinear methods in harmonic analysis.

        \item Bourgain, Guth, Bounds on Oscillatory Integral Operators Based on Multilinear Estimates (2010).

        Application of multilinear methods to bounding oscillatory integrals.

        \item Bourgain, Demeter, The Proof of the l2 Decoupling Conjecture (2014).

        Introduction to Decoupling.

        \item Peetre, New Thoughts on Besov-Spaces.

        Characterizes boundedness of Fourier multipliers on homogeneous Besov spaces.

        \item Johnson, Maximal Subspaces of Besov-Spaces Invariant Under Multiplication By Characters.

            Shows a Fourier multiplier operator is bounded in the $L^p$ norm if and only if it's translates are all localizably bounded as in Seeger.
    \end{itemize}

    \item For more background reading in microlocal analysis:
    \begin{itemize}
        \item H\"{o}rmander, The Analysis of Linear Partial Differential Operators, Volumes I-IV.
        \item Treves, Introduction to Pseudodifferential and Fourier Integral Operators, Volumes I-II.
        \item Taylor.
    \end{itemize}

%Chapter 4 describes the work of
%    - Hormander, The Spectral Function of an Elliptic Operator
%    - Avakumovic, Uber die Eigenfunktionen auf Geschlossenen Riemannschen Mannigfaltigkeiten
%    - Levitan, On the Asymptotic Behaviour of the Spectral Function of a Self-Adjoint Differential Equation of Second Order.

%- Read Hormander, Estimates for Translation Invariant Operators on Lp Spaces For More In Depth Foundations of Lp Boundedness of Multiplier Operators
%- See Strichartz [1] and Keel Tao [1], Ginibre Velo [1], Lindblad Sogge [1] for sharp embeding of wave operator using orthogonality argument introduced in introduction.
%- Seeger, Roos, Po Lam Yung. Maximal Functions for Families of Hilbert Transforms.
%- Guo, Oh, Wang. The Bochner-Riesz Problem: An Old Approach Revisited.
%- Find Stuff about the Transference Principle
%- Hickman, Guth, Illiopoulos. Sharp Estimates for Oscillatory Integral Operators via Polynomial Partitioning.

%- Fourier Restriction for Hypersurfaces in Three Dimensions and Newton Polyhedra

\end{itemize}










\bibliographystyle{plain}
\bibliography{RadialMultipliers}

\end{document}