\documentclass[12pt]{report}

\usepackage{amsmath}
\usepackage{amssymb}
\usepackage{amsthm}
\usepackage{amsopn}
\usepackage{kpfonts}
\usepackage{graphicx}
\usepackage{kbordermatrix}
\usepackage{tikz}
\usetikzlibrary{arrows, petri, topaths}%
\usepackage{tkz-berge}
\usepackage{multicol}

\usepackage{framed}
\usepackage{mathtools}
\usepackage{float}
\usepackage{subfig}
% \usepackage{cmbright}

\theoremstyle{plain}
\newtheorem{theorem}{Theorem}[chapter]
\newtheorem{lemma}[theorem]{Lemma}
\newtheorem{corollary}[theorem]{Corollary}
\newtheorem{prop}[theorem]{Proposition}
\newtheorem{exercise}{Exercise}[chapter]

\newtheorem*{example}{Example}
\newtheorem*{proof*}{Proof}

\theoremstyle{definition}
\newtheorem*{defi}{Definition}
\newenvironment{definition}
    {\begin{samepage}\begin{framed}\begin{defi}}
    {\end{defi}\end{framed}\end{samepage}}





\usepackage{hyperref} 
\hypersetup{
    colorlinks = true,
    linkcolor = black,
}

\makeatletter
\renewcommand*\env@matrix[1][*\c@MaxMatrixCols c]{%
  \hskip -\arraycolsep
  \let\@ifnextchar\new@ifnextchar
  \array{#1}}
\makeatother

\renewcommand*\contentsname{\hfill Table Of Contents \hfill}

\newcommand{\optionalsection}[1]{\section[* #1]{(Important) #1}}
\newcommand{\deriv}[3]{\left. \frac{\partial #1}{\partial #2} \right|_{#3}}

\title{Radial Multipliers}
\author{Jacob Denson}

\begin{document}

\maketitle

\tableofcontents

\newpage

\chapter{Heo, Nazarov, and Seeger}

In this chapter we give a description of the techniques of Heo, Nazarov, and Seeger's paper 2011 \emph{Radial Fourier Multipliers in High Dimensions} \cite{HeoNazrovSeeger2011}. Recall that if $m \in L^\infty(\RR^d)$ is the symbol of a Fourier multiplier operator $T_m$, then we let $\| m \|_{M^p(\RR^d)}$ denote the operator norm of $T_m$ from $L^p(\RR^d)$ to $L^p(\RR^d)$. The goal of this paper is to show that if $m \in L^\infty(\ZZ)$ is a radial function, $d \geq 4$, $1 < p < (2d - 2)/(d+1)$, and $\eta \in \mathcal{S}(\RR^d)$ is nonzero,
%
\[ \| m \|_{M^p(\RR^d)} \sim \sup_{t > 0} t^{d/p} \| T_m(\text{Dil}_t \eta) \|_{L^p(\RR^d)}. \]
%
where the implicit constant depends on $p$ and $\eta$. Since
%
\[ \sup_{t > 0} t^{d/p} \| T_m(\text{Dil}_t \eta) \|_{L^p(\RR^d)} \sim \sup_{t > 0} \frac{\| T_m(\text{Dil}_t \eta) \|_{L^p(\RR^d)}}{\| \text{Dil}_t \eta \|_{L^p(\RR^d)}} \]
%
we find that the boundedness of $T_m$ on $\mathcal{S}(\RR^d)$ is equivalent to it's boundedness on the family $\{ \text{Dil}_t \eta \}$.

In Garrig\'{o}s and Seeger's 2007 paper \emph{Characterizations of Hankel Multipliers}, it is proved that for $1 < p < 2d/(d+1)$, if $\eta \in \mathcal{S}(\RR^d)$ is a nonzero, radial Schwartz function, then
%
\[ \| m \|_{M^p_{\text{rad}}(\RR^d)} \sim \sup_{t > 0} t^{d/p}  \| T_m(\text{Dil}_t \eta) \|_{L^p(\RR^d)}, \]
%
where $M^p_{\text{rad}}(\RR^d)$ is the operator norm of $T_m$ from $L^p_{\text{rad}}(\RR^d)$ to $L^p(\RR^d)$. Thus, at least in the range $1 < p < (2d - 2)/(d+1)$, boundedness of $T_m$ on radial functions is equivalent to boundedness on all functions.

\section{Discretized Reduction}

It is obvious that
%
\[ \| m \|_{M^p(\RR^d)} \gtrsim_\eta \sup_{t > 0} t^{d/p} \| T_m(\text{Dil}_t \eta) \|_{L^p(\RR^d)}, \]
%
so it suffices to show that
%
\[ \| m \|_{M^p(\RR^d)} \lesssim_\eta \sup_{t > 0} t^{d/p} \| T_m(\text{Dil}_t \eta) \|_{L^p(\RR^d)}, \]
%
We will show this via a convolution inequality, which also proves local smoothing results for the wave equation.

Let $\sigma_r$ be the surface measure on the radius $r$ sphere centered at the origin in $\RR^d$. Also fix a nonzero Schwartz function $\psi \in \mathcal{S}(\RR^d)$. Given $x \in \RR^d$ and $r \geq 1$, define $f_{x r} = \text{Trans}_x (\sigma_r * \psi)$. Our goal is to prove the following inequality.

\begin{lemma} \label{lemma1}
    For any $a : \RR^d \times [1,\infty) \to \CC$, and $1 \leq p < (2d - 2)/(d+1)$,
    %
    \[ \left\| \int_{\RR^d} \int_1^\infty a_r(x) f_{x r}\; dx\; dr \right\|_{L^p(\RR^d)} \lesssim \left( \int_{\RR^d} \int_1^\infty |a_r(x)|^p r^{d-1} dr dx \right)^{1/p}. \]
    %
    where the implicit constant depends on $p$, $d$, and $\psi$.
\end{lemma}

Why is Lemma \ref{lemma1} useful? Suppose $m: \RR^d \to \CC$ is a radial multiplier given by some function $\tilde{m}: [1,\infty) \to \CC$, and we set $a_r(x) = \tilde{m}(r) f(x)$ for some $f: \RR^d \to \CC$. Then it is simple to check that
%
\[ \int_{\RR^d} \int_1^\infty a_r(x) f_{x r}\; dx\; dr = K * \psi * f \]
%
where $K(x) = |x|^{1-d} m(x)$. In this setting, Lemma \ref{lemma1} says that
%
\[ \| K * \psi * f \|_{L^p(\RR^d)} \lesssim \| m \|_{L^p(\RR^d)} \| f \|_{L^p(\RR^d)}, \]
%
which is clearly related to the convolution bound we want to show if $\psi = \widehat{\eta}$, provided that we are dealing with a multiplier $m$ that is supported away from the origin.

To understand Lemma \ref{lemma1} it suffices to prove the following discretized estimate.

\begin{theorem} \label{lemma2}
    Fix a finite family of pairs $\mathcal{E} \subset \RR^d \times [1,\infty)$, which is \emph{discretized} in the sense that $|(x_1,r_1) - (x_2,r_2)| \geq 1$ for each distinct pair $(x_1,r_1)$ and $(x_2,r_2)$ in $\mathcal{E}$. Then for any $a: \mathcal{E} \to \CC$ and $1 \leq p < (2d - 2)/(d+1)$, 
    %
    \[ \left\| \sum_{(x,r) \in \mathcal{E}} a_r(x) f_{x r} \right\|_{L^p(\RR^d)} \lesssim \left( \sum_{(x,r) \in \mathcal{E}} |a_r(x)|^p r^{p-1} \right)^{1/d}, \]
    %
    where the implicit constant depends on $p$, $d$, and $\psi$, but most importantly, is independant of $\mathcal{E}$.
\end{theorem}

\begin{proof}[Proof of Lemma \ref{lemma1} from Lemma \ref{lemma2}]
    For any $a: \RR^d \times [1,\infty) \to \CC$,
    %
    \[ \int_{\RR^d} \int_1^\infty a_r(x) f_{x r} = \int_{[0,1)^d} \int_0^1 \sum_{n \in \ZZ^d} \sum_{m \in \ZZ} \text{Trans}_{n,m} (a f_{rx})\; dr\; dx \]
    %
    Minkowski's inequality thus implies that
    %
    \begin{align*}
    \| \int_{\RR^d} \int_1^\infty a_r(x) f_{x r} \|_{L^p(\RR^d)} &\leq \int_{[0,1)^d} \int_0^1 \| \sum_{n \in \ZZ^d} \sum_{m \in \ZZ} \text{Trans}_{n,m} (a f_{rx}) \|_{L^p(\RR^d)}\; dr\; dx\\
    &\lesssim \int_{[0,1)^d} \int_0^1 \left( \sum_{n \in \ZZ^d} \sum_{m \in \ZZ} |a_r(x)|^p r^{p-1} \right)^{1/p}\; dr\; dx\\
    &\leq \left( \int_{[0,1)^d} \int_0^1 \sum_{n \in \ZZ^d} \sum_{m \in \ZZ} |a_r(x)|^p r^{p-1}\; dr\; dx \right)^{1/p}\\
    &= \left( \int_{\RR^d} \int_1^\infty |a_r(x)|^p r^{d-1} dr dx \right)^{1/p}. \qedhere
    \end{align*}
\end{proof}

Lemma \ref{lemma2} is further reduced by considering it as a bound on the operator $a \mapsto \sum_{(x,r) \in \mathcal{E}} a_r(x) f_{xr}$. In particular, applying real interpolation, it suffices for us to prove a restricted strong type bound. Given any discretized set $\mathcal{E}$, let $\mathcal{E}_k$ be the set of $(x,r) \in \mathcal{E}$ with $2^k \leq r < 2^{k+1}$. Then Lemma \ref{lemma2} is implied by the following Lemma.

\begin{lemma} \label{lemma3}
    For any $1 \leq p < (2d - 2)/(d+1)$ and $k \geq 1$,
    %
    \[ \| \sum_{(x,r) \in \mathcal{E}_k} f_{xr} \|_{L^p(\RR^d)} \lesssim 2^{k(d - 1)} \#(\mathcal{E}_k)^{1/p} = 2^k \cdot (2^{k(d-p-1)} \#(\mathcal{E}_k))^{1/p}. \]
\end{lemma}

\begin{proof}[Proof of Lemma \ref{lemma2} from Lemma \ref{lemma3}]
    Applying a dyadic interpolation result (Lemma 2.2 of the paper), Lemma \ref{lemma3} implies that
    %
    \[ \| \sum_{(x,r) \in \mathcal{E}} f_{xr} \|_{L^p(\RR^d)} \lesssim \left( \sum 2^{kp} 2^{k(d-p-1)} \#(\mathcal{E}_k) \right)^{1/p} = \left( \sum 2^{k(d-1)} \#(\mathcal{E}_k) \right)^{1/p} \]
    %
    This is a restricted strong type bound for Lemma \ref{lemma2}, which we can then interpolate.
\end{proof}

If $\psi$ is compactly supported, and $r$ is sufficiently large depending on the size of this support, then $f_{xr}$ is supported on an annulus with centre $x$, radius $r$, and thickness $O(1)$. Thus $\| f_{xr} \|_{L^p(\RR^d)} \sim r^{(d-1)/p}$, which implies that
%
\[ \| \sum_{(x,r) \in \mathcal{E}_k} f_{xr} \|_{L^p(\RR^d)} \gtrsim 2^{k(d-1)/p} \#(\mathcal{E}_k)^{1/p} \]
%
so this bound can only be true if $p \geq 1$, and becomes tight when $p = 1$, where we actually have
%
\[ \| \sum_{(x,r) \in \mathcal{E}_k} f_{xr} \|_{L^1(\RR^d)} \sim 2^{k(d-1)} \#(\mathcal{E}_k) \]
%
because there can be no constructive interference in the $L^1$ norm. Understanding the sum in Lemma \ref{lemma3} for $1 < p < (2d-2)/(d+1)$ will require an understanding of the interference patterns of annuli with comparable radius. We will use almost orthogonality principles to understand these interference patterns.

\begin{lemma} \label{lemma4}
    For any $N > 0$, $x_1,x_2 \in \RR^d$, and $r_1,r_2 > 1$,
    %
    \[ |\langle f_{x_1r_1}, f_{x_2r_2} \rangle| \lesssim_N (r_1r_2)^{(d-1)/2} (1 + |x_1 - x_2|)^{-(d-1)/2} \sum_{\pm,\pm} (1 + |r_1 \pm r_2 \pm |x_1 - x_2|)^{-N}. \]
\end{lemma}

\begin{remark}
    Lemma \ref{lemma4} implies that $\langle f_{x_1r_1}, f_{x_2r_2} \rangle$ is very small, except when $|x_1 - x_2|$ and $|r_1 - r_2|$ is small, and in addition:
    %
    \begin{itemize}
        \item $r_1 + r_2 - |x_1 - x_2| \approx 0$, which holds when the two annuli are `approximately' externally tangent to one another.

        \item $r_1 - r_2 + |x_1 - x_2| \approx 0$ or $r_1 - r_2 - |x_1 - x_2| \approx$, which holds when the two annuli are `approximately' internally tangent to one another.
    \end{itemize}
    %
    Heo, Nazarov, and Seeger do not exploit the tangency situations, though utilizing the tangencies seems important to improve the results they obtain.
\end{remark}

\begin{proof}
    Bessel function calculation.
\end{proof}

Lemma \ref{lemma4} implies that
%
\[ |\langle f_{x_1r_1}, f_{x_2r_2} \rangle| \lesssim \left( \frac{r_1r_2}{(1 + |x_1 - x_2|)(1 + |r_1 - r_2|)} \right)^{(d-1)/2}. \]
%
The exponent $(d-1)/2$ is too weak to apply almost orthogonality directly. 


\bibliographystyle{plain}
\bibliography{RadialMultipliers}

\end{document}