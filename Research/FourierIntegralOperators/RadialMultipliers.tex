    \documentclass[12pt, dvipsnames]{report}

\usepackage{amsmath}
\usepackage{algorithm}
%\usepackage{algorithmic}
\usepackage[noend]{algpseudocode}

\usepackage{amsmath}
\usepackage{amssymb}
\usepackage{amsthm}
\usepackage{amsopn}

\usepackage{txfonts}

\usepackage{tensor}

\usepackage{stmaryrd}


\usepackage{graphicx}

% Probably don't need this on notes anymore
%\usepackage{kbordermatrix}

% Standard tool for drawing diagrams.
\usepackage{tikz}
\usepackage{tkz-berge}
\usepackage{tikz-cd}
\usepackage{tkz-graph}
\usetikzlibrary{arrows,chains,matrix,positioning,scopes,calc}

\tikzset{
    right angle quadrant/.code={
        \pgfmathsetmacro\quadranta{{1,1,-1,-1}[#1-1]}     % Arrays for selecting quadrant
        \pgfmathsetmacro\quadrantb{{1,-1,-1,1}[#1-1]}},
    right angle quadrant=1, % Make sure it is set, even if not called explicitly
    right angle length/.code={\def\rightanglelength{#1}},   % Length of symbol
    right angle length=2ex, % Make sure it is set...
    right angle symbol/.style n args={3}{
        insert path={
            let \p0 = ($(#1)!(#3)!(#2)$) in     % Intersection
                let \p1 = ($(\p0)!\quadranta*\rightanglelength!(#3)$), % Point on base line
                \p2 = ($(\p0)!\quadrantb*\rightanglelength!(#2)$) in % Point on perpendicular line
                let \p3 = ($(\p1)+(\p2)-(\p0)$) in  % Corner point of symbol
            (\p1) -- (\p3) -- (\p2)
        }
    }
}

\usepackage{comment}

%
\usepackage{multicol}

%
\usepackage{framed}

%
\usepackage{mathtools}

%
\usepackage{float}

%
\usepackage{subfig}

%
\usepackage{wrapfig}

%
\let\savewideparen\wideparen
\let\wideparen\relax
\usepackage{mathabx}
\let\wideparen\savewideparen

% Used for generating `enlightening quotes'
\usepackage{epigraph}

% Forget what this is used for :P
\usepackage[utf8]{inputenc}

% Used for generating quotes.
\usepackage{csquotes}

% Allows what to generate links inside
% generated pdf files
\usepackage{hyperref}

% Allows one to customize theorem
% environments in mathematical proofs.
\usepackage{thmtools}

% Gives access to a proof
\usepackage{lplfitch}

% I forget what this is for.
\usepackage{accents}

% A package for drawing simple trees,
% as a substitute for unnesacary TIKZ code
\usepackage{qtree}

% Enables sequent calculus proofs
\usepackage{ebproof}

% For braket notation
\usepackage{braket}

% To change line spacing when using mathematical notations which require some height!
\usepackage{setspace}

%\usepackage[dvipsnames]{xcolor}

\usepackage{float}

% For block commenting
\usepackage{comment}

\usepackage{etoolbox}
\let\bbordermatrix\bordermatrix
\patchcmd{\bbordermatrix}{8.75}{4.75}{}{}
\patchcmd{\bbordermatrix}{\left(}{\left[}{}{}
\patchcmd{\bbordermatrix}{\right)}{\right]}{}{}




\setlength\epigraphwidth{8cm}

\usetikzlibrary{arrows, petri, topaths, decorations.markings}

% So you can do calculations in coordinate specifications
\usetikzlibrary{calc}
\usetikzlibrary{angles}

\theoremstyle{plain}
\newtheorem{theorem}{Theorem}[chapter]
\newtheorem{axiom}{Axiom}
\newtheorem{lemma}[theorem]{Lemma}
\newtheorem{corollary}[theorem]{Corollary}
\newtheorem{prop}[theorem]{Proposition}
\newtheorem{exercise}{Exercise}[chapter]
\newtheorem{fact}{Fact}[chapter]
\newtheorem{definition}{Definition}[chapter]

\newtheorem*{example}{Example}
\newtheorem*{proof*}{Proof}

\theoremstyle{remark}
\newtheorem*{exposition}{Exposition}
\newtheorem*{remark}{Remark}
\newtheorem*{remarks}{Remarks}

\theoremstyle{definition}
\newtheorem*{defi}{Definition}

\usepackage{hyperref}
\hypersetup{
    colorlinks = true,
    linkcolor = black,
}

\usepackage{textgreek}

\makeatletter
\renewcommand*\env@matrix[1][*\c@MaxMatrixCols c]{%
  \hskip -\arraycolsep
  \let\@ifnextchar\new@ifnextchar
  \array{#1}}
\makeatother

\renewcommand*\contentsname{\hfill Table Of Contents \hfill}

\newcommand{\optionalsection}[1]{\section[* #1]{(Important) #1}}
\newcommand{\deriv}[3]{\left. \frac{\partial #1}{\partial #2} \right|_{#3}} % partial derivative involving numerator and denominator.
\newcommand{\lcm}{\operatorname{lcm}}
\newcommand{\im}{\operatorname{im}}
\newcommand{\bint}{\mathbf{Z}}
\newcommand{\gen}[1]{\langle #1 \rangle}

\newcommand{\End}{\operatorname{End}}
\newcommand{\Mor}{\operatorname{Mor}}
\newcommand{\Id}{\operatorname{id}}
\newcommand{\visspace}{\text{\textvisiblespace}}
\newcommand{\Gal}{\text{Gal}}

\newcommand{\xor}{\oplus}
\newcommand{\ft}{\wedge}
\newcommand{\ift}{\vee}

\newcommand{\prob}{\mathbb{P}}
\newcommand{\expect}{\mathbb{E}}
\DeclareMathOperator{\Var}{\mathbb{V}}
\newcommand{\Ber}{\text{Ber}}
\newcommand{\Bin}{\text{Bin}}

\DeclareMathOperator{\sech}{sech}
\newcommand{\cadlag}{c\'{a}dl\'{a}g}
\newcommand{\caglad}{c\'{a}dl\'{a}d}

\newcommand{\loc}[1]{#1_{\text{loc}}}

%\newcommand{\widecheck}[1]{{#1}^{\ft}}

\DeclareMathOperator{\diam}{\text{diam}}

\DeclareMathOperator{\QQ}{\mathbb{Q}}
\DeclareMathOperator{\ZZ}{\mathbb{Z}}
\DeclareMathOperator{\RR}{\mathbb{R}}
\DeclareMathOperator{\HH}{\mathbb{H}}
\DeclareMathOperator{\BB}{\mathbb{B}}
\DeclareMathOperator{\CC}{\mathbb{C}}
\DeclareMathOperator{\AB}{\mathbb{A}}
\DeclareMathOperator{\PP}{\mathbb{P}}
\DeclareMathOperator{\MM}{\mathbb{M}}
\DeclareMathOperator{\VV}{\mathbb{V}}
\DeclareMathOperator{\TT}{\mathbb{T}}
\DeclareMathOperator{\LL}{\mathcal{L}}
\DeclareMathOperator{\DD}{\mathcal{D}}
\DeclareMathOperator{\SW}{\mathcal{S}}
\DeclareMathOperator{\EC}{\mathcal{E}}
\DeclareMathOperator{\AC}{\mathcal{A}}

\DeclareMathOperator{\EE}{\mathbb{E}}
\DeclareMathOperator{\NN}{\mathbb{N}}

\DeclareMathOperator{\II}{\mathbb{I}}

\DeclareMathOperator{\DQ}{\mathcal{Q}}

\DeclareMathOperator{\Ind}{\mathbb{I}}


\DeclareMathOperator{\IA}{\mathfrak{a}}
\DeclareMathOperator{\IB}{\mathfrak{b}}
\DeclareMathOperator{\IC}{\mathfrak{c}}
\DeclareMathOperator{\IP}{\mathfrak{p}}
\DeclareMathOperator{\IQ}{\mathfrak{q}}
\DeclareMathOperator{\IM}{\mathfrak{m}}
\DeclareMathOperator{\IN}{\mathfrak{n}}
\DeclareMathOperator{\IK}{\mathfrak{k}}
\DeclareMathOperator{\ord}{\text{ord}}
\DeclareMathOperator{\Ker}{\textsf{Ker}}
\DeclareMathOperator{\Coker}{\textsf{Coker}}
\DeclareMathOperator{\emphcoker}{\emph{coker}}
\DeclareMathOperator{\pp}{\partial}
\DeclareMathOperator{\tr}{\text{tr}}
\DeclareMathOperator{\Ree}{\text{Re}}


\DeclareMathOperator{\BL}{\text{BL}}

\DeclareMathOperator{\dstrike}{//}

\DeclareMathOperator{\supp}{\text{supp}}

\DeclareMathOperator{\codim}{\text{codim}}

\DeclareMathOperator{\minkdim}{\dim_{\mathbb{M}}}
\DeclareMathOperator{\hausdim}{\dim_{\mathbb{H}}}
\DeclareMathOperator{\sobdim}{\dim_{\mathbb{S}}}
\DeclareMathOperator{\lowminkdim}{\underline{\dim}_{\mathbb{M}}}
\DeclareMathOperator{\upminkdim}{\overline{\dim}_{\mathbb{M}}}
\DeclareMathOperator{\lhdim}{\underline{\dim}_{\mathbb{M}}}
\DeclareMathOperator{\lmbdim}{\underline{\dim}_{\mathbb{MB}}}
\DeclareMathOperator{\packdim}{\text{dim}_{\mathbb{P}}}
\DeclareMathOperator{\fordim}{\dim_{\mathbb{F}}}

\DeclareMathOperator{\CT}{ {{\otimes}^\wedge} }

\DeclareMathOperator{\msupp}{\text{$\mu$-supp}}
\DeclareMathOperator{\singsupp}{\text{sing-supp}}
\DeclareMathOperator{\Char}{\text{Char}}

\DeclareMathOperator*{\argmax}{arg\,max}
\DeclareMathOperator*{\argmin}{arg\,min}

\DeclareMathOperator{\ssm}{\smallsetminus}

\DeclarePairedDelimiter{\inner}{\langle}{\rangle}
\newcommand{\pder}[2]{\frac{\partial #1}{\partial #2}}
\newcommand{\tripnorm}[1]{{\left\vert\kern-0.25ex\left\vert\kern-0.25ex\left\vert #1 
    \right\vert\kern-0.25ex\right\vert\kern-0.25ex\right\vert}}

%\DeclareMathOperator{\span}{\text{span}}

\makeatletter
\newcommand*\bigcdot{\mathpalette\bigcdot@{.5}}
\newcommand*\bigcdot@[2]{\mathbin{\vcenter{\hbox{\scalebox{#2}{$\m@th#1\bullet$}}}}}
\makeatother

\title{Radial Multipliers}
\author{Jacob Denson}

\begin{document}

\maketitle

\tableofcontents

\newpage

\part{Review of Literature}

\chapter{General Introduction}

The question of the regularity of translation-invariant operators on $\RR^d$ has proved central to the development of modern harmonic analysis. Indeed, the regularity of scuh operators underpins any subtle understanding of the Fourier transform, since with essentially any such operator $T$, we can associate a tempered distribution $m: \RR^d \to \CC$, known as the \emph{symbol} of $T$, such that for any Schwartz function $f \in \mathcal{S}(\RR^d)$,
%
\[ Tf(x) = \int m(\xi) \widehat{f}(\xi) e^{2 \pi i \xi \cdot x}\; d\xi, \]
%
i.e. such that $\widehat{Tf} = m \cdot \widehat{f}$. This is why such operators are also called \emph{Fourier multipliers}. Using the spectral calculus of unbounded operators, one can also write this operator as $m(D)$, where $D = (2 \pi i)^{-1} \nabla$ is a self-adjoint normalization of the gradient operator. Thus the study of the boundedness of translation invariant operators is closely connected to the study of the interactions of the projections
%
\[ E_\xi f(x) = \widehat{f}(\xi) e^{2 \pi i \xi \cdot x}, \]
%
which act as projections onto the eigenspaces of the components of $D$, since we can write
%
\[ m(D) = \int m(\xi) E_\xi. \]
%
Thus $m(D)$ is represented as a weighted average of the operators $\{ E_\xi \}$.

The study of translation invariant operators emerges from many classical questions in analysis, like that of the convergence properties of Fourier series, or in mathematical physics, through the study of the heat, wave, and Schr\"{o}dinger equation. These operators also naturally have rotational symmetry, so it is natural to restrict our attention to translation-invariant operators which are also rotation-invariant. These operators are precisely those represented by symbols $m: \RR^d \to \CC$ which are \emph{radial}, i.e. such that
%
\[ m(\xi) = h(|\xi|) \]
%
for some function $h: [0,\infty) \to \CC$. This is the class of \emph{radial Fourier multipliers}. The spectral calculus again implies one can write $m(D) = h(\sqrt{-\Delta})$, where $\Delta$ is the Laplacian on $\RR^d$. Thus the study of radial multipliers is closely connected to interactions between the spherical projection operators
%
\[ E_\lambda f(x) = \int_{|\xi| = 1} \widehat{f}(\xi) e^{2 \pi i \xi \cdot x}, \]
%
for $0 < \lambda < \infty$, which are the projections onto the eigenspaces of $\sqrt{-\Delta}$, since we then have
%
\[ h(\sqrt{-\Delta}) = \int h(\lambda) E_\lambda. \]
%
Thus studying the regularity of radial Fourier multipliers allows us to understand the behaviour of weighted averages of the operators $\{ E_\lambda \}$.

Stated as above, we can extend the study of radial multipliers from $\RR^d$ to a general \emph{geodesically complete} Riemannian manifold $X$. On such a manifold we have a Laplace-Beltrami operator $\Delta$ which is an essentially self-adjoint unbounded operator on $L^2(X)$, and one can consider a spectral calculus. The operator $\sqrt{-\Delta}$ will be self-adjoint, and one can consider the study of operators of the form $h(\sqrt{-\Delta})$ for functions $h: [0,\infty) \to \CC$. Some techniques of analyzing radial multipliers on $\RR^d$ extend to the Riemannian case, whereas in other cases new tools are required.

This research project studies necessary and sufficient conditions to guarantee the $L^p$ boundedness of radial multiplier operators, both in the Euclidean setting, and also in the setting of Riemannian manifolds. % stimulated by recent developments which indicate lines of attack for three related problems in the field.


\section{Radial Multipliers on Euclidean Space}

The general study of the boundedness of Fourier multipliers was initiated in the 1960s. It was quickly realized that the most fundamental estimates were those of the form
%
\[ \| Tf \|_{L^q(\RR^d)} \lesssim \| f \|_{L^p(\RR^d)}, \]
%
for $1 \leq p \leq 2$, and $q \geq p$. It is therefore natural to introduce the spaces $M^{p,q}(\RR^d)$, consisting of all symbols $m$ which induce a Fourier multiplier operator $T$ bounded from $L^p(\RR^d)$ to $L^q(\RR^d)$. The space $M^{p,q}(\RR^d)$ is then naturally a Banach space under the operator norm
%
\[ \| m \|_{M^{p,q}(\RR^d)} = \sup \left\{ \frac{\| T f \|_{L^q(\RR^d)}}{\| f \|_{L^p(\RR^d)}} : f \in \mathcal{S}(\RR^d) \right\}. \]
%
For notational convenience, $M^{p,p}(\RR^d)$ is denoted by $M^p(\RR^d)$.

It was very simple to characterize the spaces $M^{1,q}(\RR^d)$, by virtue of the fact that the theory of boundedness of operators with domain $L^1(\RR^d)$ is often relatively trivial. For any symbol $m$, if $k = \widehat{m}$, then
%
\[ \| m \|_{M^{1,q}(\RR^d)} = \begin{cases} \| k \|_{L^q(\RR^d)} &: q > 1 \\ \| k \|_{M(\RR^d)} &: q = 1, \end{cases}. \]
%
where $M(\RR^d)$ is the space of finite signed Borel measures equipped with the total variation norm. Duality also allows us to characterize the spaces $M^{p,\infty}(\RR^d)$, since in general one has an isometric equivalence $M^{p,q}(\RR^d) = M^{q^*,p^*}(\RR^d)$, and so
%
\[ M^{p,\infty}(\RR^d) = M^{1,p^*}(\RR^d). \]
%
The fact that the Fourier transform is unitary also allowed the spaces $M^{p,2}(\RR^d)$ to be characterized, for $1 \leq p \leq 2$, by the identity
%
\[ \| m \|_{M^{p,2}(\RR^d)} = \| m \|_{L^q(\RR^d)}, \]
%
% | m f^ |_{L^2} <= | m |_q | f |_p
% If f^ is an extremizer for m
where $q = 2p/(2-p)$. However, characterizing the remaining spaces $M^{p,q}$, i.e. for $p \in (1,2]$ and $q \in [p, \infty) - \{ 2 \}$, proved more challenging. In the past 60 years there has been no tractable characterization of the spaces $M^{p,q}(\RR^d)$ for any other pair of exponents $p$ or $q$.

One tool that has proved useful outside of this range is \emph{Littlewood-Paley} theory, which makes it natural to restrict to the study of Fourier multipliers compactly suported on dyadic annuli. We fix a smooth bump function $\phi \in C_c^\infty(\RR^d)$ supported on $\{ |\xi| \sim 1 \}$, and such that $1 = \sum_j \text{Dil}_{2^j} \phi$.
\begin{comment}
    We can therefore introduce the Littlewood-Paley projection operators $P_j$, which are Fourier multipliers with symbol $\text{Dil}_{2^j} \phi$. Littlewood-Paley theory guarantees that for $1 < r < \infty$,
%
\[ \| f \|_{L^r(\RR^d)} \sim_r \left( \sum_n \| P_n f \|_{L^r(\RR^d)}^2 \right)^{1/2}. \]
%
We also must introduce a slightly thickened function $\tilde{\phi} \in C_c^\infty(\RR^d)$ supported on $\{ \xi \in \RR^d: 1/4 \leq |\xi| \leq 4 \}$, equal to one on the support of $\phi$, and with $1 = \sum_j \text{Dil}_{2^j} \tilde{\phi}$, then we can introuce the Littlewood-Paley projections $\tilde{P}_j$ with symbol $\text{Dil}_{2^j} \tilde{\phi}$. For $1 < q < \infty$, we thus have
%
\[ \| m(D) f \|_{L^q(\RR^d)} \sim_q \left( \sum_j \| P_j m(D) f \|_{L^q(\RR^d)}^2 \right)^{1/2} = \left( \| P_j m(D) \{ \tilde{P}_j f \} \|_{L^q(\RR^d)}^2 \right)^{1/2}. \]
%
Now
%
\[ P_j m(D) \{ \tilde{P}_j f \} = \text{Dil}_{1/2^j} \{ m_{2^j}(D) \circ \text{Dil}_{2^j} \{ \tilde{P}_j f \} \}, \]
%
and so
%
\[ \| P_j m(D) \{ \tilde{P}_j f \} \|_{L^q(\RR^d)} = 2^{-jd/q} \| m_{2^j}(D) \text{Dil}_{2^j} \{ \tilde{P}_j f \} \|_{L^q(\RR^d)}. \]
%
We thus have
%
\begin{align*}
    \| m(D) f \|_{L^q(\RR^d)} &\lesssim_q \left( \sum_j 4^{jd(1/p - 1/q)} \| m_{2^j} \|_{M^{p,q}(\RR^d)}^2 \| \tilde{P}_j f \|_{L^p(\RR^d)}^2 \right)^{1/2}\\
    &\leq \left( \sup_t t^{d(1/p - 1/q)} \| m_t \|_{M^{p,q}(\RR^d)} \right) \left( \sum_j \| \tilde{P}_j f \|_{L^p(\RR^d)}^2 \right)^{1/2}\\
    &\sim_p \left( \sup_t t^{d(1/p - 1/q)} \| m_t \|_{M^{p,q}(\RR^d)} \right) \| f \|_{L^p(\RR^d)}.
\end{align*}
%
Moreover, this inequality is tight.
\end{comment}
Then given a symbol $m$, we define
%
\[ m_t = (\text{Dil}_{1/t} m) \cdot \phi. \]
%
Thus $m_t$ describes the behaviour of the multiplier $m$ restricted to the annulus of frequencies $|\xi| \sim t$, rescaled so that this behaviour is now lying on the annulus $|\xi| \sim 1$. Littlewood-Paley theory implies that for $1 < p,q < \infty$, then
%
\[ \| m \|_{M^{p,q}(\RR^d)} \sim_{p,q} \sup_{t > 0} t^{d(1/p - 1/q)} \| m_t \|_{M^{p,q}(\RR^d)}. \]
%
In the study of multipliers we have not already characterized, it is therefore natural to restrict oneself to the study of multipliers with a compactly supported symbol; in $\RR^d$ we can rescale, so we can assume we are working with a multiplier supported on the annulus $1/2 \leq |\xi| \leq 2$. In the sequel, we will call these \emph{unit scale multipliers}.

One common heuristic to this theory is that the regularity of the symbol $m$, or equivalently, the decay of the convolution kernel $k$ away from the origin, implies some boundedness of the symbol, viewed as a multiplier. The most well known condition of this form for $1 < p < \infty$ is the H\"{o}rmander-Mikhlin multiplier theorem, which shows that for $1 < p < \infty$, and $\varepsilon > 0$, if $m$ is a unit scale multiplier, and $k$ is it's convolution kernel, then
%
\[ \| m \|_{M^p(\RR^d)} \lesssim_{p,\varepsilon} \int \frac{k(x)}{\langle x \rangle^{1 + \varepsilon}}\; dx. \]
%
This implies the slightly weaker inequality
%
\[ \| m \|_{M^p(\RR^d)} \lesssim_\varepsilon \| m \|_{L^2_{d/2 + \varepsilon}}. \]
%
All these results apply via a Paley-Wiener decomposition to general multipliers, i.e. with the right hand side replaced with a supremum of the quantities associated with the rescaled multipliers $\{ m_t \}$.

Conversely, some control over the mass of the convolution kernel $k$ is necessary in order to conclude that $m \in M^{p,q}(\RR^d)$ for some exponents $p$ and $q$. This is because if $k$ is the convolution kernel corresponding to a unit scale multiplier $m$, and $\phi \in C_c^\infty(\RR^d)$ has Fourier transform equal to one on the annulus $1/4 \leq |\xi| \leq 4$, then
%
\[ \| k \|_{L^q(\RR^d)} = \| k * \phi \|_{L^q(\RR^d)} \lesssim \| m \|_{M^{p,q}(\RR^d)}. \]
%
For a general, non compactly supported multiplier $m$, if $k_t$ is the convolution kernel associated with the multiplier $m_t$, then one obtains the condition
%
\[ \sup_{t > 0} \left\{ t^{d(1/p - 1/q)} \cdot \| k_t \|_{L^q(\RR^d)} \right\} \lesssim \| m \|_{M^{p,q}(\RR^d)}, \]
%
One can phrase this bound in terms of the homogeneous Besov spaces $\dot{B}^{p,q}_s(\RR^d)$, the space consisting of all distributions $u$ on $\RR^d$ such that the norm
%
\[ \| u \|_{\dot{B}^{p,q}_s(\RR^d)} = \left( \sum_{j = -\infty}^\infty \left( 2^{js} \| P_j u \|_{L^p(\RR^d)} \right)^q \right)^{1/q} = \| 2^{js} P_j u \|_{l^q(\ZZ) L^p(\RR^d)}, \]
%
is finite, where $P_j = \phi(D/2^j)$ is the Littlewood-Paley projection operator onto a dyadic frequency band of radius $2^j$. If $k$ is the convolution kernel of a multiplier $m$, one can rescale the condition above to read that
%
\[ \sup_{t > 0} \left\{ t^{-d/p^*} \| P_t k \|_{L^q(\RR^d)} \right\} \lesssim \| m \|_{M^{p,q}(\RR^d)}, \]
%
i.e. that
%
\[ \| k \|_{\dot{B}^{q,\infty}_{-d/p^*}} \lesssim \| m \|_{M^{p,q}(\RR^d)}. \]
%
Thus we conclude that $k$ must satisfy some (admittedly weak) regularity assumptions to be the convolution kernel of a bounded Fourier multiplier.

Despite the lack of a complete characterization of the classes $M^{p,q}(\RR^d)$, it is surprising that we \emph{can} conjecture a characterization of the subspace of $M^{p,q}(\RR^d)$ for \emph{radial symbols} in this class, for an appropriate range of exponents. This conjecture is best phrased in terms of the result of \cite{GarrigosandSeeger}, which concerned radial multipliers $m$ whose associated operator $T$ is bounded from the $L^p$ norm to the $L^q$ norm \emph{restricted to radial functions}, i.e. such that the norm
%
\[ \| m \|_{M^{p,q}_{\text{rad}}(\RR^d)} = \sup \left\{ \frac{\| Tf \|_{L^q(\RR^d)}}{\| f \|_{L^p(\RR^d)}} : f \in \mathcal{S}(\RR^d)\ \text{and $f$ is radial} \right\} \]
%
is finite. The main result of \cite{GarrigosandSeeger} was that if $d > 1$, if $1 < p < 2d/(d+1)$, and if $p \leq q < 2$, then $M^{p,q}_{\text{rad}}(\RR^d)$ is a subset of $L^1_{\text{loc}}(\RR^d)$, and for any unit scale, integrable, radial multiplier $m$,
%
\[ \| m \|_{M^{p,q}_{\text{rad}}(\RR^d)} \sim_{p,q,d} \| k \|_{L^q(\RR^d)}. \]
%
More generally, for any locally integrable radial symbol $m$,
%
\[ \| m \|_{M^{p,q}_{\text{rad}}(\RR^d)} \sim_{p,q,d} \sup_{t > 0} t^{d(1/p - 1/q)} \| k_t \|_{L^q(\RR^d)} = \| k \|_{\dot{B}^{q,\infty}_{-d/p^*}}. \]
%
Moreover, this condition give \emph{precisely the range} under which this characterization holds. It is natural to conjecture that the same constraint continues to hold when we remove the constraint that our inputs $f$ are radial, i.e. that for unit scale, integrable, radial symbols $m$, for $d > 1$, $1 < p < 2d/(d+1)$, and for $p \leq q < 2$,
%
\[ \| m \|_{M^{p,q}} \sim_{p,q,d} \| k \|_{L^q(\RR^d)} \]
%
and for general locally integrable symbols $m$,
%
\[ \| m \|_{M^{p,q}} \sim_{p,q,d} \| k \|_{\dot{B}_{-d/p^*}^{q,\infty}} \]
%
In the sequel, we call this the \emph{radial multiplier conjecture} in $\RR^d$. TODO: Look up counterexample which shows that these results cannot be obtained if $m$ is not radial.

\begin{remark}
    For $p = 1$, the radial multiplier conjecture is true \emph{for compactly supported multipliers}, and one does not even need to assume that the multiplier is radial in this case. Indeed, for any (not even compactly supported) multiplier $m$ we have
    %
    \[ \| m \|_{M^{1,q}(\RR^d)} = \| k \|_{L^q(\RR^d)}. \]
    %
    Littlewood-Paley says this quantity is proportional to
    %
    \[ \left( \sum \| P_j k \|_{L^q(\RR^d)}^2 \right)^{1/2} = \| k \|_{\dot{B}^{q,2}_0}. \]
    %
    On the other hand, for non compactly supported multipliers $m$ the radial multiplier conjecture should say that
    %
    \[ \| m \|_{M^{1,q}(\RR^d)} \sim \| k \|_{\dot{B}^{q,\infty}_0}, \]
    %
    so the result fails `in the second order exponents'.
%    For $p = 1$, the result fails, but only `barely'. 
%    `should say' that
    %
%    \[ \| k \|_{L^q} \sim \| m \|_{M^{1,q}(\RR^d)} \]

%    \[ \| k \|_{\dot{B}^{q,\infty}_0} \lesssim \| m \|_{M^{1,q}(\RR^d)}. \]
    %
%    On the other hand, the Littlewood-Paley inequality says that
    %
%    \[ \| m \|_{M^{1,q}(\RR^d)} = \| k \|_{L^q(\RR^d)} \sim_q \| k \|_{\dot{B}^{q,2}_0} \]
    %
%    Thus we have the double sided inequality
    % 
%    \[ \| k \|_{\dot{B}^{q,\infty}_0} \lesssim_d \| m \|_{M^{1,q}(\RR^d)} \lesssim_q \| m \|_{\dot{B}^{q,2}_0}, \]
    %
%    which shows that the radial multiplier conjecture `holds for $p = 1$', provided that we are willing to vary the second-order integrability parameter. TODO: Can one similarily obtain a result at the endpoint $p = 2d/(d+1)$, or even further past this range?
\end{remark}

% x has units M
% xi has units 1/M
% m is unitless
% Then k = m^ has units M^{-d}
% t also has units 1/M
% The projections P_t m^ have units M^{-d} as well.
% The L^p norm of P_t m^ has units M^{-d} * M^{d/p} = M^{-d/p^*}
% t^s | P_t m^ |_{L^p} has units A M^{-s-d/p^*}
% The L^q_t norm of t^s | P_t m^ |_{L^p} has units M^{-s-d/p^*-d/q}
% The Besov norm of m^ has units M^{-s-d/p^*-d/q}

% k has units M^{-d}
% For a dimensionless function f, k * f is dimensionless
% So | k * f |_{L^q} has units M^{d/q}
% And |f|_p has units M^{d/p}
% so the M^{p,q} norm of m, which is the supremum of quantities | k * f |_q / |f|_p, which have units M^{d/q-d/p}, also has units M^{d/q - d/p}.

% r has units M
% The projections P_r m are unitsless
% The L^p norm of P_r m has units M^{-d/p}
% r^s | P_r m^ |_{L^p} has units M^{s - d/p}
% The L^q_r norm of r^s |P_r m^ |_{L^p} has units M^{s - d/p + d/q}

% M^{1,q} norm has units M^{-d/q^*}
% The B^{p,2}_0 norm of m^ has units M^{-d/p^*-d/2}
% Suppose -d/q^* = -d/p^* - d/2
% 1/p^* - 1/q^* + 1/2 = 0
% Not quite matching up.

\begin{remark}
    Let $m(\xi) = h(|\xi|) \mathbf{I}(|\xi| \leq 1)$, where $h \in C_c^\infty(\RR)$ is supported on $1/4 \leq r \leq 4$, and is equal to one for $1/2 \leq r \leq 2$. Then $m$ differs from the `ball multiplier' by a compactly supported, smooth symbol, and thus $m$ has all the $L^p$ mapping properties that the ball multiplier has. In particular, it is a result of Fefferman (TODO: Cite) that the ball multiplier does not lie in $M^{p,q}(\RR^d)$ when $d > 1$ for any values of $p$ and $q$ except when $p = q = 2$, so the same is true of the multiplier $m$ given above. Now if $k$ is the convolution kernel of $m$, then polar coordinates gives that
    %
    \[ k(x) = \int m(\xi) e^{2 \pi i \xi \cdot x} = \frac{1}{|x|^{d/2 - 1}} \int_{1/4}^1 r^{d/2} h(r) J_{d/2-1}( r x ). \]
    %
    For $|x| \geq 1$, Bessel function asymptotics gives that there is some constant $a$, depending on $d$, such that
    %
    \[ k(x) = |x|^{-\frac{d-1}{2}} \int_{1/4}^1 r^{\frac{d-1}{2}} h(r) \cos(r |x| + a)\; dr + O \left( |x|^{-\frac{d+1}{2}} \right). \]
    %
    Integration by parts then gives that $|k(x)| \lesssim |x|^{- \frac{d+1}{2}}$. Since $k$ is bounded near the origin (i.e. by Paley-Wiener), we conclude that $k \in L^q(\RR^d)$ for $q > 2d/(d+1)$. In particular, this multiplier gives a counterexample to the radial multiplier conjecture for $q > 2d/(d+1)$. TODO: Is there a counterexmaple for $q = 2d/(d+1)$?
    %Then we can write
    %
    %\[ m(\xi) = m_0(\xi) + \sum_{j = 1}^\infty m_j(\xi) = m_0(\xi) + m_{\geq 1}(\xi). \]
    %
    %where $m_0(\xi)$ is smooth and compactly supported on $|\xi| \leq 1/2$, and where $m_j(\xi) = \phi(2^j (1 - x))$ lives at the frequency scale $\approx 2^j$ and is supported on an annulus of width $1/2^j$ and radius $\approx 1$, and is roughly constant on this annuli.
    % For j >> 0, P_j k = 0 since m is compactly supported
    % For j << 0, P_j k = P_j k~ where k~ is analytic.
    %For $j > 0$,
    %
    %\[ P_{-j}k(x) = \int \phi(2^j \xi) m(\xi) e^{2 \pi i \xi \cdot x}\; d\xi = \int \phi(2^j \xi) e^{2 \pi i \xi \cdot x}\; d\xi = 2^{-jd} \widehat{\phi}(x / 2^j). \]
    %
    %Thus
    %
    %\[ \| P_{-j} k \|_{L^q(\RR^d)} \sim 2^{jd(1/q - 1)} \]
    %
    %For $j > 0$, $P_j k = 0$. The only tricky case is $j = 0$. Assuming $\phi$ is radial, we can apply polar-coordinates / stationary phase to conclude that
    %
    %\[ |P_0 k(x)| \lesssim |x|^{- \frac{d-1}{2}} \]
    %
    %Thus $P_0 k \in L^q(\RR^d)$ for $q > 2d/(d+1)$. In particular, we conclude that for $q > 2d/(d+1)$,
    %
    %\begin{align*}
    %    \| k \|_{\dot{B}_s^{q,r}} &\sim \left( \sum_{j \geq 0} ( 2^{-js} \| P_{-j} k \|_{L^q(\RR^d)} )^r \right)^{1/r}\\
    %    &\sim \left( 1 + \sum_{j > 0} ( 2^{-js} 2^{jd(1/q - 1)} )^r \right)^{1/r}.
    %\end{align*}
    %
    %This quantity is finite for $s > d(1/q - 1)$ and for any exponent $r$, or for $s = d(1/q - 1)$ and with $r = \infty$. In particular, the convolution kernel of the ball multiplier lies in $\dot{B}^{q,\infty}_{-d/p^*}$ for $q \geq p$ and $q > 2d/(d+1)$. It is a result of Fefferman that $m$ does not lie in any of the spaces $M^{p,q}(\RR^d)$ when $d > 1$ except for the obvious bound $p = q = 2$. Thus we see that the radial multiplier conjecture cannot be true for $q > 2d/(d+1)$. TODO: Is it still possible for $p = q = 2d/(d+1)$?
\end{remark}

\begin{remark}
    Given a function $h$ on $[0,\infty)$, we define the \emph{$d$-dimensional Fourier-Bessel transform} of $h$ as
    %
    \[ \mathcal{B}_d h (r) = r^{- \frac{d-2}{2}} \int_0^\infty \rho^{d/2} h(\rho) J_{\frac{d-2}{2}}(\rho r)\; d\rho, \]
    %
    where $J_\alpha$ is the standard Bessel function of order $\alpha$. Then if $m(\xi) = h(|\xi|)$, then we have $\{ \mathcal{F}m \}(x) = \{ \mathcal{B}_d h \} (|x|)$. The condition in the radial multiplier conjecture for unit scale multipliers becomes that
    %
    \[ \| m \|_{M^{p,q}(\RR^d)} \sim \left( \int_0^\infty r^{d-1} |\mathcal{B}_d h(r)|^q\; dr \right)^{1/q}. \]
    %
    We therefore that, if we let $h_t = \phi \cdot \text{Dil}_{1/t} h$, then the condition in the radial multiplier conjecture becomes
    %
    \[ \sup_{t > 0} t^{d(1/p - 1/q)} \left( \int_0^\infty r^{d-1} |\{ \mathcal{B}_d h_t \}(r)|^q \right)^{1/q} < \infty. \]
    %
    One can also convert this into a statement involving the standard one-dimensional Fourier transform. One can use Bessel function asymptotics to convert this into a condition on the one-dimensional Fourier transform of $h$ (extended to an even function on $\RR$). Indeed, there exists a differential operator $L$ with constant coefficients and order at most $(d-1)/2$ such that we can write
    %
    \begin{align*}
        \mathcal{B}_d h(r) &= r^{- \frac{d-2}{2}} \int_{1/2}^2 \rho^{\frac{d}{2}} h(\rho) \left( \cos(\rho r + a) |\rho r|^{-1/2} + O(|\rho r|^{-3/2}) \right)\; d\rho\\
        &= r^{- \frac{d-1}{2}} \int_{1/2}^2 \rho^{\frac{d-1}{2}} h(\rho) \cos(\rho r + a)\; d\rho + O \left( r^{-\frac{d+1}{2}} \| h \|_{L^1} \right)\\
        &= r^{- \frac{d-1}{2}} \left( L \widehat{h}(r) + \int_{-\infty}^\infty \widehat{h}(r - s) s^{-\frac{d+1}{2}}\; ds \right) + O \left( r^{-\frac{d+1}{2}} \| h \|_{L^1} \right).
    \end{align*}
    %
    TODO: Finish this calculation; Theorem 1.2 of \cite{GarrigosandSeeger} guarantees the radial multiplier condition is equivalent to 
    %
    \[ \sup_{t > 0} t^{d(1/p - 1/q)} \left( \int (1 + |s|)^{(d-1)(1 - q/2)} |\widehat{h}_t(s)|^q\; ds \right)^{1/q} < \infty. \]
    %
    In particular, if $h$ is supported at a unit scale, then the condition is that
    %
    \[ \left( \int_0^\infty (1 + |s|)^{(d-1)(1 - q/2)} |\widehat{h}(s)|^q\; ds \right)^{1/q} < \infty. \]
    %
    Thus for most values of $s$, we have $|\widehat{h}(s)| \lesssim |s|^{-(d-1)(1/q - 1/2)}$.
%    If $m(\xi) = h(|\xi|)$ is radial, the condition that
    %
%    \[ \sup_{t > 0} t^{d(1/p - 1/q)} \| k_t \|_{L^q(\RR^d)} < \infty \]
    %
%    can be rephrased in terms involving the Fourier transform of $h$. Namely, we have
    %
%    \begin{align*}
%        & \sup_{t > 0} t^{d(1/p - 1/q)} \| k_t \|_{L^q(\RR^d)}\\
%        &\quad\quad \sim \sup_{t > 0} t^{d(1/p - 1/q)} \left( \int_{t/2 \leq |s| \leq 2t} |\widehat{h}(s)|^q (1 + |s|)^{(d-1)(1 - q/2)}\; dt \right)^{1/q}.
%    \end{align*}
    %
%    The weight inside the norm prevents us from easily converting this condition into a homogeneous Besov condition on the function $w$, but roughly speaking, we have $|\widehat{h}(s)| \lesssim s^{d(1/p-1/q)} \langle s \rangle^{-(d-1)(1-q/2)}$ for \emph{most} inputs $x$.
\end{remark}

We now know, by the results of \cite{HeoandNazarovandSeeger} that the radial multiplier conjecture is true when $d \geq 4$ and when $1 < p < (2d-2)/(d+1)$. When $d = 4$, this was improved by \cite{Cladek}, who showed that the conjecture is true here when $1 < p < 36/29$, where $36/29 \approx (2d - 1.79)/(d+1)$. When $d = 3$, \cite{Cladek} also established a \emph{restricted weak type} bound
% 
\[ \| Tf \|_{L^p(\RR^n)} \lesssim \| f \|_{L^{p,1}(\RR^n)} \]
% 13/12
when $d = 3$ and $1 < p < 13/12$, where $13/12 \approx (2d - 1.66)/(d+1)$ But the radial multiplier conjecture has not yet been completely resolved in any dimension $n$, we do not have any strong type $L^p$ bounds when $d = 3$, and we don't have any bounds whatsoever when $d = 2$. One goal of this research project is to investigate whether one can use modern research techniques to improve upon these bounds.

The full proof of the radial multiplier is likely far beyond current research techniques. Indeed, it remains a major open problem in harmonic analysis to determine the range of exponents for which \emph{specific} radial Fourier multipliers are bounded in the range where the conjecture would apply, such as the Fourier multiplier on $\RR^d$ with symbol $m_\lambda(\xi) = \left( 1 - |\xi| \right)^\lambda_+$, the family of \emph{Bochner-Riesz multipliers}. The radial multiplier conjecture characterizes the range of the Bochner-Riesz multipliers, and thus the conjecture would also imply the Kakeya and restriction conjectures. All three of these results are major unsolved problems in harmonic analysis. On the other hand, the Bochner Riesz conjecture is completely resolved when $d = 2$, while in contrast, no results related to the radial multiplier conjecture are known in this dimension at all. And in any dimension $d > 2$, the range under which the Bochner-Riesz multiplier is known to hold \cite{GuoandOhandWangandWuandZhang} is strictly larger than the range under which the radial multiplier conjecture is known to hold, even for the restricted weak-type bounds obtained in \cite{Cladek}. Thus it still seems within hope that the techniques recently applied to improve results for Bochner-Riesz problem, such as broad-narrow analysis \cite{BourgainandGuth}, the polynomial Wolff axioms \cite{KatzandRogers}, and methods of incidence geometry and polynomial partitioning \cite{Zahl2} can be applied to give improvements to current results characterizing the boundedness of general radial Fourier multipliers.

Our hopes are further emboldened when we consult the proofs in \cite{HeoandNazarovandSeeger} and \cite{Cladek}, which reduce the radial multiplier conjecture to the study of upper bounds of quantities of the form
%
\[ \left\| \sum_{(y,r) \in \mathcal{E}} F_{y,r} \right\|_{L^p(\RR^n)}, \]
%
where $\mathcal{E} \subset \RR^n \times (0,\infty)$ is a finite collection of pairs, and $F_{y,r}$ is an oscillating function supported on a $O(1)$ neighborhood of a sphere of radius $r$ centered at a point $y$. The $L^p$ norm of this sum is closely related to the study of the tangential intersections of these spheres, a problem successfully studied in more combinatorial settings using incidence geometry and polynomial partitioning methods \cite{Zahl}, which provides further estimates that these methods might yield further estimates on the radial multiplier conjecture.

When $d = 3$, the results of \cite{Cladek} are only able to obtain bounds on the $L^p$ sums in the last paragraph when $\mathcal{E}$ is a Cartesian product of two subsets of $(0,\infty)$ and $\RR^d$. This is why only restricted weak-type bounds have been obtained in this dimension. It is therefore an interesting question whether different techniques enable one to extend the $L^p$ bounds of these sums when the set $\mathcal{E}$ is \emph{not} a Cartesian product, which would allow us to upgrade the result of \cite{Cladek} in $d = 3$ to give strong $L^p$ bounds. This question also has independent interest, because it would imply new results for the `endpoint' local smoothing conjecture, which concerns the regularity of solutions to the wave equation in $\RR^d$. Incidence geometry has been recently applied to yield results on the `non-endpoint' local smoothing conjecture \cite{GuthandWangandZhang}, which again suggests these techniques might be applied to yield the estimates needed to upgrade the result of \cite{Cladek} to give strong $L^p$-type bounds.

\section{Multipliers on Riemannian Manifolds}

Fix a geodesically complete Riemannian manifold $X$. We can then define operators $h(\sqrt{-\Delta})$, which are analogues to the radial multipliers studied in the Euclidean setting. Just like multiplier operators on $\RR^n$ are crucial to an understanding of the interactions between the functions $e_\xi(x) = e^{2 \pi i \xi \cdot x}$ on $\RR^n$, understanding the operators $h(\sqrt{-\Delta})$ is crucial to understanding the interactions of eigenfunctions of the Laplace-Beltrami operator on $X$. We let $M^{p,q}(X, \sqrt{-\Delta} )$ denote the family of all symbols $h: \RR \to \CC$ such that the operator $T_h = h(\sqrt{-\Delta})$ is bounded from $L^p(X)$ to $L^q(X)$, with the analogous operator norm, though, when there is no ambiguity, we will overload notation and write this space as $M^{p,q}(X)$.

% M^{1,q}(X) should be related to the L^q norm of the Fourier transform of h
% h(sqrt{-Delta}) f = int k(t) e^{2 pi i t sqrt(-Delta)} f
% <= |k|_{L^q}

To avoid technicalities, we will focus on a compact Riemannian manifold $X$, which is automatically geodesically complete. For such manifolds, there is a problem which prevents a direct generalization of the radial multiplier conjecture. For such a manifold, there exists $0 = \lambda_1 < \lambda_2 \leq \dots$ with $\lambda_i \to \infty$, and an orthonormal family of eigenfunctions $\{ e_n \}$ in $C^\infty(X)$, forming a basis for $L^2(X)$, such that
%
\[ \Delta f = \sum - \lambda_n^2 \langle f, e_n \rangle \cdot e_n. \]
%
Thus for any function $h: [0,\infty) \to \CC$,
%
\[ h \left( \sqrt{-\Delta} \right) f = \sum h(\lambda_n) \langle f, e_n \rangle \cdot e_n. \]
%
The operators $h ( \sqrt{-\Delta} )$ act as an analogue of radial multipliers on $\RR^d$.

The study of multipliers on a Riemannian manifold has a certain technical problem, which the Euclidean case did not have. If $h$ has compact support, this sum will be finite, and thus by the triangle inequality, trivially bounded from $L^p(X)$ to $L^q(X)$ for any exponents $p$ and $q$. Thus $M^{p,q}(X)$ trivially contains all compactly supported radial multipliers. This trivializes the study of compactly supported radial multipliers in some sense, which is the complete opposite of the Euclidean case, where Littlewood-Paley allowed us to reduce the study of general multipliers to compactly supported radial multipliers. The key here is that Euclidean multipliers automatically have rescaling symmetries, whereas this is not present in the case of compact Riemannian manifolds. To get around this we add a rescaling into the definition of our operator norm, i.e. we study conditions that ensure we have a bound of the form
%
\[ \sup_{t > 0} t^{d(1/q - 1/p)} \| \text{Dil}_t h \|_{M^{p,q}(X)} < \infty. \]
%
We let $M^{p,q}_{\text{Dil}}(X)$ denote the family of all multipliers for which the inequality above holds, and give it the norm induced by the quantity on the left hand side. A transference principle of Mitjagin \cite{Mitjagin} shows that if $X$ is a compact Riemannian manifold, and $m: \RR^d \to \CC$ is radial, with $m(\xi) = h(|\xi|)$, then
%
\[ \| m \|_{M^{p,q}(\RR^d)} \lesssim_{X,p,q} \| h \|_{M^{p,q}_{\text{Dil}}(X)}. \]
%
Thus, in some sense, the dilation invariant Fourier multiplier problem on a compact manifold $X$ is at least as hard as it is on $\RR^n$. Another goal of this research project is to try and extend the radial multiplier conjecture to the setting of dilation invariant bounds for multipliers of the Laplacian on Riemannian manifolds.

As in the case $X = \RR^d$, the study of multipliers in $M^{2,2}_{\text{Dil}}(X)$ is trivial. Indeed, applying orthogonality, we calculate that
%
\begin{align*}
    \| h(\sqrt{-\Delta}) f \|_{L^2(X)} &= \left\| \sum_\lambda h(\lambda) E_\lambda f \right\|_{L^2(X)}\\
    &= \left( \sum_\lambda |h(\lambda)|^2 \| E_\lambda f \|_{L^2(X)}^2 \right)^{1/2}\\
    &\leq \left( \sup_{\lambda \in \sigma(\sqrt{-\Delta})} |h(\lambda)| \right) \left( \sum_\lambda \| E_\lambda f \|_{L^2(X)}^2 \right)^{1/2}\\
    &= \left( \sup_{\lambda \in \sigma(\sqrt{-\Delta})} |h(\lambda)| \right) \| f \|_{L^2(X)}.
\end{align*}
%
Taking $f$ to be an eigenfunction with eigenvalue $\lambda$ which maximizes the value of $|h(\lambda)|$ shows this inequality is tight, i.e. we have
%
\[ \| h \|_{M^{2,2}(X)} = \sup_{\lambda \in \sigma(\sqrt{-\Delta})} |h(\lambda)|. \]
%
Now applying an arbitrary dilation to $h$, we conclude that
%
\[ \| h \|_{M^{2,2}_{\text{Dil}} (X)} = \sup_{\lambda > 0} |h(\lambda)|. \]
%
Thus we have found a simple characterization of the space $M^{2,2}_{\text{Dil}}(X)$.

The spaces $M^{1,q}_{\text{Dil}}(X)$ are a little more tricky, since we do not have a precise theory of the Fourier transform in the setting of general Riemannian manifolds. To take a look at these bounds, we recall that $L^1 \to L^q$ bounds of an operator are characterized by Schur's test. If $\{ e_n \}$ is a $C^\infty(X)$ basis of eigenfunctions on $X$, with $\Delta e_n = - \lambda_n^2 e_n$, then
%
\[ h(\sqrt{-\Delta}) f(x) = \sum h(\lambda_n) \langle f, e_n \langle e_n(x) = \int \left( \sum_n h(\lambda_n) e_n(x) \overline{e_n(y)} \right) f(y)\; dy. \]
%
Thus the kernel of $h(\sqrt{-\Delta})$ is precisely $K(x,y) = \sum_n h(\lambda_n) e_n(x) \overline{e_n(y)}$, and we conclude by Schur's test that
%
\[ \| h \|_{M^{1,q}(X)} = \left\| \sum_n h(\lambda_n) e_n(x) \overline{e_n(y)} \right\|_{L^\infty_y L^q_x}. \]
%
In the case $X = \RR^n$, the analogous kernel is $K(x,y) = \int_{\RR^d} h(|\xi|) e^{2 \pi i \xi \cdot x} \overline{e^{2 \pi i \xi \cdot y}}$, which can be explicitly reduced to $K(x,y) = \mathcal{B}_d h(|x - y|)$, and the condition of being contained in $M^{1,q}(\RR^d)$ then becomes that
%
\[ \left( \int r^{d-1} |\mathcal{B}_d h(r)|^q\; dr \right)^{1/q} < \infty. \]
%
If $h$ is compactly supported, then this condition is equivalent to the condition that
%
\[ \left( \int (1 + |t|)^{(d-1)(1 - q/2)} |\widehat{h}(t)|^q\; dt \right)^{1/q} < \infty. \]
%
In the general setting we do not have quite as nice a formula, but we can still \emph{force} the Fourier transform into the equation to see if it can be used to understand these quantities (which will be necessary for studying the radial multiplier conjecture). We thus write
%
\begin{align*}
    \sum_n h(\lambda_n) e_n(x) \overline{e_n(y)} &= \sum_n \left( \int \widehat{h}(t) e^{2 \pi i t \lambda_n} e_n(x) \overline{e_n(y)}\; dt \right) \\
    &= \int \widehat{h}(t) \left( \sum_n e^{2 \pi i t \lambda_n} e_n(x) \overline{e_n(y)} \right)\; dt\\
    &= \int \widehat{h}(t) W_t(x,y)\; dt,
\end{align*}
%
where $W_t(x,y) = \sum_n e^{2 \pi i t \lambda_n} e_n(x) \overline{e_n(y)}$ is the kernel of the \emph{half-wave propogator} $e^{2 \pi i t \sqrt{-\Delta}}$ on $X$. The connection between radial multipliers on $X$ and the Fourier transform of their symbol is therefore closely related to the study of the half-wave equation $\partial_t = \sqrt{-\Delta}$ on $X$. We are therefore looking for an inequality of the form
%
\[ \left\| \int \widehat{h}(t) W_t(x,y)\; dt \right\|_{L^\infty_y L^q_x} \lesssim \left( \int (1 + |t|)^{(d-1)(1 - q/2)} |\widehat{h}(t)|^q\; dt \right)^{1/q} \]
%
to hold. TODO: What can we do here? TODO: In the model case $X = \RR^d$ can we prove this result? H\"{o}lder is \emph{not} good enough in this situation, since it leads to quantities of the form
%
\[ \| (1 + |t|)^{-(d-1)(1 - q/2)} W_t \|_{L^\infty_y L^q_x L^{q^*}_t}, \]
%
and the singularities of $W_t$ on the light cone mean the $L^{q^*}_t$ norm is infinite for all $x$ and $y$. But we should expect to do better than H\"{o}lder, since we only expect H\"{o}lder's inequality to be tight when $\widehat{h}$ is close to a scalar multiple to $t \mapsto W_t(x,y)$, and this cannot be true for all $x$ and $y$. TODO: Think about this more.

Directly translating the assumptions of the radial multiplier conjecture to this setting yields the following statement: If $h: [0,\infty) \to \RR$ is a function, and we define
%
\[ A_{p,q}(h) = \sup_{t > 0} t^{d(1/p - 1/q)} \left( \int |\widehat{h}_t(s)|^q (1 + |s|)^{(d-1)(1 - q/2)}\; ds \right)^{1/q}, \]
%
then for what values of $p$ and $q$ is is true that the inequality
%
\[ \| h \|_{M^{p,q}_{\text{Dil}}(X)} \lesssim A_{p,q}(h) \]
%
still holds. Mitjagin's result implies that we require $1 < p < 2d/(d+1)$ and $p \leq q < 2$, and we conjecture that, perhaps under appropriate assumptions on $X$, we can achieve similar ranges of exponents as have been obtained for the Euclidean radial multiplier conjecture.

On general compact manifolds, there are difficulties arising from a generalization of the radial multiplier conjecture, connected to the fact that analogues of the Kakeya / Nikodym conjecture are false in this general setting \cite{Minicozzi}. But these problems do not arise for constant curvature manifolds, like the sphere. The sphere also has over special properties which make it especially amenable to analysis, such as the fact that solutions to the wave equation on spheres are periodic. Best of all, there are already results which achieve the analogue of \cite{GarrigosandSeeger} on the sphere. Thus it seems reasonable that current research techniques can obtain interesting results for radial multipliers on the sphere, at least in the ranges established in \cite{HeoandNazarovandSeeger} or even \cite{Cladek}.

\section{Summary}

In conclusion, the results of \cite{HeoandNazarovandSeeger} and \cite{Cladek} indicate three lines of questioning about radial Fourier multiplier operators, which current research techniques place us in reach of resolving. The first question is whether we can extend the range of exponents upon which the conjecture of \cite{GarrigosandSeeger} is true, at least in the case $d = 2$ where Bochner-Riesz has been solved. The second is whether we can use more sophisticated arguments to prove the $L^p$ sum bounds obtained in \cite{Cladek} when $d = 3$ when the sums are no longer Cartesian products, thus obtaining strong $L^p$ characterizations in this settiong, as well as new results about the endpoint local smoothing conjecture. The third question is whether we can generalize these bounds obtained in these two papers to study radial Fourier multipliers on the sphere. 

\chapter{Notes on Bochner-Riesz}

The goal of this section is to compare and contrast approaches to understanding the Bochner-Riesz conjecture on Euclidean space and on compact Riemannian manifolds, in order to reflect on the differences in understanding multipliers on $\RR^d$ vs on a compact manifold $X$ before we attack the more general multiplier problem in this setting. We define the Riesz multipliers via symbols $r_\rho^\delta: [0,\infty] \to [0,\infty)$, defined for $\rho > 0$ and a real number $\delta$ by setting, for $\tau > 0$,
%
\[ r_\rho^\delta(\tau) = (1 - \tau / \rho)_+^\delta. \]
%
Here $s_+ = \max(s,0)$. The resulting radial multipliers on $\RR^n$, and on a compact Riemannian manifold $X$, will be denoted by
%
\[ R_\rho^\delta = r_\rho^\delta \left( \sqrt{-\Delta}\ \right). \]
%
The goal of the Bochner-Riesz conjecture is to determine bounds on the operators $\{ R_\rho^\delta \}$ invariant under dilation of the symbol.

\section{Euclidean Case}

Let's review a reduction of Bochner-Riesz to Tomas Stein:
%
\begin{itemize}
    \item First, we can \emph{rescale the problem}. If $r^\delta = r^\delta_1$, then
    %
    \[ r_\rho^\delta(\lambda) = r^\delta(\lambda / \rho). \]
    %
    Thus if $R^\delta = R^\delta_1$, then $R^\delta_\rho = R^\delta \circ \text{Dil}_{1/\rho}$, and so the operators $\{ R^\delta_\rho \}$ are uniformly bounded from $L^p$ to $L^p$ for all $\rho$ if and only if $R^\delta$ is bounded from $L^p$ to $L^p$.

    \item We now perform a \emph{spatial decomposition}. Let $k^\delta$ be the convolution kernel corresponding to the operator $R^\delta$. We break up the effects of the operator spatially into dyadic annuli, i.e. writing
    %
    \[ k^\delta(x) = \sum_{j = 0}^\infty k^\delta_j(2^j x), \]
    %
    where $k^\delta_0$ is supported on $|x| \leq 2$, and all of the other kernels $k^\delta_j$ are supported on the annuli $\{ 1/2 \leq |x| \leq 1 \}$, and can be written as
    %
    \[ k^\delta_j(x) = \phi \cdot \text{Dil}_{1/2^j} k^\delta \]
    %
    for some $\phi \in C_c^\infty$ supported on the annulus $\{ 1/2 \leq |x| \leq 2 \}$ and equal to one on the annulus $\{ 3/4 \leq |x| \leq 3/2 \}$. We analyze each of the convolution kernels separately and then collect up each of the bounds we obtain by applying the triangle inequality. Thus we let $R^\delta_j$ be the operator with convolution kernel $k^\delta_j$. Provided we can obtain a bound of the form
    %
    \[ \| R^\delta_j f \|_{L^p(\RR^d)} \lesssim 2^{- \varepsilon j} \| f \|_{L^p(\RR^d)} \]
    %
    for some $\varepsilon > 0$, and some implicit constant uniform in $j$, we can sum up the bounds using the triangle inequality to bound $R^\delta$.

    \item Spatial localization means that the operators $\{ R^\delta_j \}$ are \emph{local}, i.e. for any function $f$, the support of $R^\delta_j f$ is contained in a $O(1)$ neighborhood of the support of $f$. A decomposition argument, thus implies that it suffices to obtain a bound of the form
    %
    \[ \| R^\delta_j f \|_{L^p(\RR^d)} \lesssim 2^{- \varepsilon j} \| f \|_{L^p(\RR^d)} \]
    %
    for functions $f$ \emph{supported on balls of radius $1$}, since the general bound will follow from this.

    \item We \emph{reduce to $L^2$ bounds}: Now that $f$ is suported on a ball of radius 1, $R^\delta_j$ is supported on a ball of radius $O(1)$, and so for $p \leq 2$ we have
    %
    \[ \| R^\delta_j f \|_{L^p(\RR^d)} \lesssim \| R^\delta_j f \|_{L^2(\RR^d)}. \]
    %
    Thus it suffices to obtain a bound of the form $\| R^\delta_j f \|_{L^2(\RR^d)} \lesssim \| f \|_{L^p(\RR^d)}$. Switching from the $L^p$ norm to the $L^2$ norm is the most inefficient part of the proof, but it enables us to apply more powerful tools which we only have in $L^2(\RR^d)$. Getting around this reduction is key to improving the currently known Bochner-Riesz bounds.

    \item We reduce the problem to Tomas-Stein. Since we are now in $L^2(\RR^d)$, we can apply Plancherel. If $\psi^\delta_j$ is the Fourier transform of $k^\delta_j$, then we obtain that
    %
    \[ \| R^\delta_j f \|_{L^2(\RR^d)} = \| \psi^\delta_j \cdot \widehat{f} \|_{L^2(\RR^d)}. \]
    %
    A stationary phase calculation shows that $\psi^\delta_j$ has the majority of its mass on an annulus of radius $2^j$ and width $O(1)$, and has magnitude $O(2^{-j\delta})$ there, i.e.
    %
    \[ |\psi^\delta_j(\xi)| \lesssim_N 2^{-\delta j} \langle 2^j - |\xi| \rangle^{-N}. \]
    %
    Thus by Tomas-Stein, if $R_S$ denotes the restriction operator to the unit sphere $S$, we find that
    %
    \begin{align*}
        \| \psi^\delta_j \cdot \widehat{f} \|_{L^2(\RR^d)} &\lesssim_N 2^{-\delta j} \left( \int_0^\infty \langle 2^j |1 - r| \rangle^{-2N} \int_{|\xi| = 1} |\widehat{f}(r \xi)|^2\; d\sigma\; r^{d-1} dr \right)^{1/2}\\
        &\lesssim 2^{-\delta j} \left( \int_0^\infty \langle 2^j |1 - r| \rangle^{-2N} \| R_S \circ \text{Dil}_r f \|^2_{L^2(S^{n-1})} \frac{dr}{r} \right)^{1/2}\\
        &\lesssim  2^{-\delta j} \| f \|_{L^p(\RR^d)} \left( \int_0^\infty \langle 2^j |1 - r| \rangle^{-2N}  r^{2d/p - 1} dr \right)^{1/2}\\
        &\lesssim 2^{-\delta j} \| f \|_{L^p(\RR^d)}.
        % f(x/r) -> r^d f^(rx)
        % 1/2^{j+1} <= |1 - r| <= 1/2^j
    \end{align*}
    %
    This bound is summable in $j$, which yields the required result.
\end{itemize}
%
Let us end our discussion of the Euclidean case by expanding on the computation of the inequality
%
\[ |\psi^\delta_j(\xi)| \lesssim_N 2^{-j \delta} \langle 2^j - |\xi| \rangle^{-N}. \]
%
Before this, let's see why the result is \emph{intuitive}. The function $\psi^\delta_j$ is obtained by localizing the frequency multiplier $m^\delta$ on the spatial side and then rescaling. Thus our result is intuitively saying that the phase-portrait of the multiplier is concentrated on a neighborhood of the set
%
\[ \{ (x,\xi) : |\xi| \leq 1\ \text{and}\ ||\xi| - 1| = 1/|x| \}. \]
% (1 - tau)^delta
% k_delta^j = Dil_{1/2^j} { k } phi
% 2^{-jd} [Dil_{2^j} m] * phi^
% m^delta = sum 2^{jd} Dil_{1/2^j} psi^delta_j
%
This makes sense, since the `high frequency' components of $m^\delta$ should be distributed near the boundary of the unit ball, since this is where the symbol becomes singular; that the spatial part should be inversely proportional to the distance to the boundary can be detected by taking derivatives of $m$ in the frequency variable, i.e. noting that if $||\xi| - 1| \sim 1/2^j$, then
%
\[ |\nabla^N m^\delta(\xi)| \lesssim_{N,\delta} (1 - |\xi|)^{\delta - N} \sim 2^{-j \delta} 2^{jN}. \]
%
And we see the derivative grows in $N$ as a power of $2^j$, which is inversely propoertional to $||\xi| - 1|$. Working more precisely, we have
%
\[ \psi^\delta_j = 2^{-jd} \left[ \widehat{\phi} * \text{Dil}_{2^j} m^\delta \right]. \]
%
The function is the average of $\widehat{\phi}$ over a ball of radius $O(2^j)$ so we immediately obtain a bound by using the rapid decay of $\widehat{\phi}$, thus obtaining that
%
\[ |\psi^\delta_j(\xi)| \lesssim_N \langle 2^j - |\xi| \rangle^{-N}. \]
% 
Thus we see that $\psi^\delta_j$ has the majority of it's support on the ball of radius $2^j$. But we can do much better than this using the fact that $\widehat{\phi}$ is \emph{oscillatory}, since $\phi$ is supported away from the origin, and $m^\delta$ is \emph{mostly} smooth. More precisely, $\widehat{\phi}$ oscillates at frequencies $\sim 1$, so we should expect integration by parts to yield useful decay on a quantity $\widehat{\phi} * \text{Dil}_{2^j} f$ if we had a bound $|\nabla^N f| \ll 2^{Nj}$ for large $N > 0$. This is true of $m^\delta$ away from a thickness $O(2^{-j})$ annulus containing the unit ball. Thus we are motivated to define $m^\delta = a^\delta_j + b^\delta_j$, where
%
\[ a^\delta_j(\xi) = m^\delta(\xi) \eta( 2^j (1 - |\xi|)) \quad\text{and}\quad b^\delta_j(\xi) = m^\delta_j(\xi) ( 1 - \eta( 2^j (1 - |\xi|)) ) \]
%
where $\eta(t)$ is supported on $|t| \leq 1$ and equal to one for $|t| \leq 1/2$. The function $b^\delta_j$ is therefore supported on $|\xi| \leq 1 - 1/2^{j+1}$. For $N > 0$, we have
% 
\[ |\nabla^N m^\delta_j(\xi)| \lesssim_{N,\delta} (1 - |\xi|)^{\delta-N}. \]
%
By the product rule, $\nabla^N b^\delta_j$ is a sum of derivatives of $m^\delta_j$ and of derivatives of $1 - \eta(2^j (|\xi| - 1))$. The support of any derivative of the latter is supported on $|\xi| \geq 1 - 1/2^j$. Thus we have
%
\[ |\nabla^N b^\delta_j(\xi)| \lesssim_N (1 - |\xi|)^{\delta - N} \mathbf{I}(|\xi| \leq 1 - 1/2^{j+1}) + 2^{j(N - \delta)} \mathbf{I}(1 - 1/2^j \leq |\xi| \leq 1). \]
%
Since $\phi$ is supported away from the origin, we may antidifferentiate $\widehat{\phi}$ arbitrarily many times without any singular behaviour emerging. But now averaging the $N$th antiderivative of $\widehat{\phi}$, which is rapidly decaying, with the $N$th derivative of $\text{Dil}_{2^j} b^\delta_j$, which is rapidly decaying outside of an annulus of width $1$ and radius $2^j$, we find that
%
\[ |(\widehat{\phi} * \text{Dil}_{2^j} b^\delta_j)(\xi)| \lesssim 2^{j(d - \delta)} \langle 2^j - \xi \rangle^{-N}. \]
%
The multiplier $a^\delta_j$ is not so smooth, but it is supported on a very thin annulus of radius 1 and thickness $O(2^{-j})$, and $m^\delta$ has magnitude at most $2^{- \delta j}$ on this annulus, which gives that
%
\[ |(\widehat{\phi} * \text{Dil}_{2^j} a^\delta_j)(\xi)| \lesssim 2^{-j\delta} \int_{||\eta| - 2^j| \leq 1} |\widehat{\phi}(\xi - \eta)|\; d\eta \lesssim_N 2^{j(d - \delta)} \langle 2^j - \xi \rangle^{-N}. \]
%
Putting these results together gives the required bound.

%$K_\rho^\delta$ is the kernel of $R_\rho^\delta$, then there exists symbols $a_1$ and $a_2$, of order zero, with $|a_1(x)|, |a_2(x)| \gtrsim 1$ for $|x| \gtrsim 1$, such that
%
%\[ K_\rho^\delta(x) = a_1(x) \frac{e^{2 \pi i |x|}}{\langle x \rangle^{\frac{n+1}{2} + \delta}} + a_2(x) \frac{e^{- 2 \pi i |x|}}{\langle x \rangle^{\frac{n+1}{2} + \delta}} + O \left( \frac{1}{\langle x \rangle^{n+1}} \right) \]

%    &= a_1(x) \frac{e^{2 \pi i |x|}}{\langle x \rangle^{\frac{n+1}{2} + \delta}} + a_2(x) \frac{e^{- 2 \pi i |x|}}{\langle x \rangle^{\frac{n+1}{2} + \delta}} + O \left( \frac{1}{\langle x \rangle^{n+1}} \right),
%\end{align*}
%
%where $a_1$ and $a_2$ are symbols of order zero, with $|a_1(x)|, |a_2(x)| \gtrsim 1$ for $|x| \gtrsim 1$. TODO: Necessity of the $\delta(p)$ values.

%
%\begin{align*}
%    K_\rho^\delta(x) &= \int_0^\rho (1 - \lambda / \rho)^\delta \left( \int_{|\xi| = \lambda} e^{2 \pi i \xi \cdot x}\; d\xi \right)\; d\lambda\\
%    &= \rho^{d-1} \int_0^1 \lambda^{d-1} (1 - \lambda)^\delta \left( \int_{|\xi| = 1} e^{2 \pi i \rho \lambda \xi \cdot x}\; d\xi \right)\; d\lambda\\
%    &= (2 \pi) \rho^{d/2} |x|^{1-d/2} \int_0^1 \lambda^{d/2} (1 - \lambda)^\delta J_{d/2-1}(2 \pi \rho \lambda |x|),
%\end{align*}
%
%where $J_{d/2-1}$ is a Bessel function of the first kind, i.e. defined by the integral formula
% See https://dlmf.nist.gov/10.17
%\begin{align*}
%    J_\nu(s) &= \frac{1}{\pi^{1/2}} \frac{(s/2)^\nu}{\Gamma(\nu + 1/2)} \int_{-1}^1 e^{isx} (1 - x^2)^{\nu - 1/2}\; dx.
%\end{align*}
%
%We are interested with the asymptotic theory here: there exists constants $a_1(k)$ and $a_2(k)$ (See the Digital Library of Mathematical Functions, 10.17: Hankel's Expansions for more detail) such that
%
%\begin{align*}
%    K_\rho^\delta(x) \sim \sum_{k = 0}^\infty a_1(k) \frac{\rho^{d/2-2k}}{|x|^{d/2 - 1 + 2k}} \int_0^1 \lambda^{d/2-2k} (1 - \lambda)^\delta e^{2 \pi i \rho \lambda |x|}
%\end{align*}
%The infinite sum here converges rapidly, so we can interchange the summation with the integral and conclude that
%
%\begin{align*}
%    K_\rho^\delta(x) &= 2 \pi \sum_{k = 0}^\infty \rho^{d/2 + 2k} |x|^{1-d/2+2k} \frac{(-1)^k}{k!} \frac{\pi^{2k}}{\Gamma(d/2 + k)} \int_0^1 \lambda^{d/2+2k} (1 - \lambda)^\delta\; d\lambda\\
%    &= 2 \pi \sum_{k = 0}^\infty \frac{(-1)^k}{k!} \frac{\pi^{2k}}{\Gamma(d/2 + k)} \frac{\Gamma(d/2 + 2k + 1) \Gamma(\delta + 1)}{\Gamma(d/2 + 2k + \delta + 2)} \rho^{d/2+2k} |x|^{1-d/2+2k}
%\end{align*}

\section{Manifold Case}

The analogue of the Tomas Stein theorem on a compact Riemannian manifold $X$ is a result due to Sogge, so let's see if we can obtain a result for compact manifolds using similar techniques:
%
\begin{itemize}
    \item The first problem is that on a compact Riemannian manifold we do not have a rescaling symmetry which we can use to reduce the study of the Bochner-Riesz multipliers $R^\delta_\rho$ to the case $\rho = 1$. Thus we must analyze a general multiplier of the form $R^\delta_\rho$ for all $\rho > 0$. The case of small $\rho$ is easily dealt with using the triangle inequality, so we may assume that $\rho \gtrsim 1$ in what follows.

    \item Now we try and reduce to Sogge's spectral cluster bounds, which are analogous to the Tomas-Stein bounds in $\RR^d$. If we are able to justify that $K^\delta_{\rho,j}$ behaves like a spectral band projection operator, as in the Euclidean setting, we'd be able to apply this bound. Plancherel does not quite have an analogy to the $L^2$ setting on a manifold. But we can instead use the wave operator and it's parametrices, i.e. that
    %
    \begin{align*}
        R^\delta_\rho &= \sum_\lambda r^\delta(\lambda / \rho) E_\lambda\\
        &= \rho \int_0^\infty \widehat{r^\delta}(\rho t) e^{2 \pi i t \sqrt{-\Delta}}\; dt\\
        &= c_\delta \cdot \rho^{-\delta} \int_0^\infty e^{2 \pi i \rho t} (t + i0)^{-\delta - 1} e^{2 \pi i t \sqrt{-\Delta}}\; dt.
    \end{align*}
    %
    The singularity in the definition of this integral occurs at $t = 0$, so the operator should, for large $t$, be relatively well behaved.

    \item Since we expect the function is well behaved for large $t$, let's bound these terms so we may reduce to controlling the integral over $t \lesssim 1$. Fix $\alpha \in C_c^\infty(\RR)$ equal to one in a neighborhood of zero, and consider the behaviour of $R^\delta_\rho$ for large $t$, i.e. the operator
    %
    \[ R^\delta_\rho = c_\delta \cdot \rho^{-\delta} \int_0^\infty (1 - \alpha(t)) \cdot e^{- 2 \pi i \rho t} t^{-\delta - 1} e^{2 \pi i t \sqrt{-\Delta}}\; dt. \]
    %
    If $\psi$ is the inverse Fourier transform of $c_\delta t^{-\delta-1} (1 - \alpha(t))$, then $\psi$ is bounded and rapidly decreasing because all of the derivatives of it's Fourier transform are smooth and integrable. We thus can revert back to the multiplier setting and write
    %
    \[ R^\delta_\rho = \rho^{-\delta} \sum_\lambda \psi(\lambda - \rho) E_\lambda. \]
    %
    The rapid decay here means we can be fairly lazy in controlling this operator, for instance, employing the Sobolev embedding bound
    %
    \[ \| E_\lambda f \|_{L^2(X)} \lesssim \langle \lambda \rangle^{d(1/p - 1/2) - 1/2} \| f \|_{L^p(X)} \]
    %
    and the triangle inequality, using the rapid decay to obtain that
    %
    \[ \| R^\delta_\rho f \|_{L^p(X)} \lesssim \langle \rho \rangle^{-[\delta - d(1/p - 1/2) + 1/2]} \| f \|_{L^p(X)}, \]
    %
    which is better than what we need. Thus we now need only bound the operator
    %
    \[ \tilde{R}^\delta_\rho = c_\delta \cdot \rho^{-\delta} \int_0^\infty \alpha(t) e^{2 \pi i \rho t} (t + i0)^{-\delta - 1} e^{2 \pi i t \sqrt{-\Delta}}\; dt. \]
    %
    The advantage of doing this is because we only have understanding of the wave operator through Fourier integral operators (through the Lax parametrix) for times $t \lesssim 1$.

    \item We now `spatially localize' as in the Euclidean case, though things look different here since we are dealing with the wave equation. We choose $\beta$ such that
    %
    \[ 1 = \eta + \sum_{j = 1}^\infty \text{Dil}_{2^j} \beta. \]
    %
    We then write
    %
    \[ R^\delta_{\rho} = \sum_{j = 0}^{O(\log \rho)} R^\delta_{\rho,j} \]
    %
    where for $j > 0$
    %
    \[ R^\delta_{\rho,j} = c_\delta \cdot \rho^{-\delta} \int_0^\infty \alpha(t) (\text{Dil}_{2^j} \beta)(\rho t) e^{- 2 \pi i \rho t} t^{-\delta - 1} e^{2 \pi i \sqrt{-\Delta}}\; dt, \]
    %
    and
    %
    \begin{align*}
        R^\delta_{\rho,0} &= c_\delta \cdot \rho^{-\delta} \int_0^\infty \alpha(t) \eta(\rho t) e^{-2 \pi i \rho t} t^{-\delta - 1} e^{2 \pi i \sqrt{-\Delta}}\; dt\\
        &= c_\delta \cdot \rho^{-\delta} \int_0^\infty \eta(\rho t) e^{-2 \pi i \rho t} t^{-\delta - 1} e^{2 \pi i \sqrt{-\Delta}}\; dt,
    \end{align*}
    %
    where the last identity follows because the result the support of the integral is on $t \lesssim 1/\rho$, and we are assuming $\rho$ is large so that $\alpha$ may be assumed equal to one on the support of the integral. Thus $R^\delta_\rho$ is an integral over $t \sim 2^j / \rho$. This is analogous to the spatial decomposition we performed in the Euclidean setting, except now we have the wave equation involved, and the `pseudolocal' finite speed of propogation for the wave equation now must substitute for the explicit spatial localization we obtained in the Euclidean decomposition.

    \item Despite the singularity that occurs at the origin, the case $j = 0$ is simplest to deal with. If we define
    %
    \[ m(\lambda) = c_\delta (\widehat{\eta} * r_\delta) \]
    %
    then $R^\delta_{\rho,0}$ is a multiplier operator with symbol
    %
    \[ m_\rho(\lambda) = \rho^{-\delta} \text{Dil}_\rho m. \]
    %
    We have estimates of the form
    %
    \[ | \nabla^N m(\lambda) | \lesssim_N \langle \lambda \rangle^{-M}. \]
    %
    Thus
    %
    \[ | \nabla^N m_\rho(\lambda)| \lesssim_N \rho^{-\delta-N} \langle \lambda / \rho \rangle^{-M}. \]
    %
    In particular, taking $M = N$ and $M = 0$ yields that
    %
    \[ | \nabla^N m_\rho(\lambda)| \lesssim_N \rho^{-\delta} \langle \lambda \rangle^{-N}. \]
    %
    Thus  $\{ \rho^\delta m_\rho \}$ are a uniformly bounded family of symbols of order zero. Thus (TODO: Review estimates for multipliers given by a symbol) we can obtain that
    %
    \[ \| m_\rho(\sqrt{-\Delta}) f \|_{L^p(X)} \lesssim \rho^{-\delta} \| f \|_{L^p(X)} \lesssim \| f \|_{L^p(X)}. \]
    %
    TODO: Check there isn't an error here since the $\rho^{-\delta}$ terms helps us out, but shouldn't our bounds be scale invariant?

    \item Now we deal with the $j > 0$ terms, and we must use the pseudolocal finite speed of propogation of the wave equation as a substitute for explicit localization. Since we have localized to times $t \lesssim 1$. We deal with this by using the Lax parametrix for the wave equation, but first we must ensure the remainder terms from employing the parametrix are well behaved. For $t \lesssim 1$, we can write $e^{2 \pi i t \sqrt{-\Delta}} = Q(t) + R(t)$, where $Q(t)$ is a Fourier integral operator supported on a $O(1)$ neighborhood of the diagonal $\Delta = \{ (x,x): x \in X \}$, and with kernel given in coordinates by
    %
    \[ (x,y) \mapsto \int e^{2 \pi i [\phi(x,y,\xi) + t |\xi|]} q(t,x,y,\xi)\; d\xi \]
    %
    where $q$ is a symbol of order zero, and $\phi$ is homogeneous of order one in $\xi$, with $\phi(x,y,\xi) \approx (x - y) \cdot \xi$, in the sense that
    %
    \[ |\nabla_\xi^N [ \phi(x,y,\xi) - (x - y) \cdot \xi ]| \lesssim_N |x - y|^2 |\xi|^{1-N} \]
    %
    for all $N > 0$. The operators $\{ R(t) \}$ are smoothing, i.e. with a joint kernel $A$ uniformly in $C^\infty([-1,1] \times X \times X)$. Thus we write
    %
    \begin{align*}
        R^\delta_{\rho,j} &= c_\delta \cdot \rho^{-\delta} \int_0^\infty \alpha(t) (\text{Dil}_{2^j} \beta)(\rho t) e^{- 2 \pi i \rho t} t^{-\delta - 1} ( Q(t) + R(t) )\; dt\\
        &= R^\delta_{\rho,j,Q} + R^\delta_{\rho,j,R}.
    \end{align*}
    %
    Let's control the $R(t)$ term. Computing the integral of the kernel defining $R^\delta_{\rho,j,R}$ leads to a term of the form
    %
    \[ c_\delta 2^j \rho^{-1-\delta} ( \widehat{\alpha A} * \text{Dil}_{\rho / 2^j} \widehat{\beta} * r^\delta)(\rho). \]
    %
    The function $\alpha A$ is smooth and compactly supported in the $t$ variable, so it's Fourier transform is rapidly decaying. The same is true of $\widehat{\beta}$, except it is rescaled so we can imagine the majority of it's mass occurs on $|\lambda| \lesssim \rho / 2^j$. Finally, $r^\delta$ is concentrated on $|\lambda| \lesssim 1$. Thus the kernel is pointwise bound from above by a constant times
    %
    \[ 2^j \rho^{-1-\delta} \int_{\rho - O(1)}^{\rho + O(1)} ( \widehat{\alpha A} * \text{Dil}_{\rho / 2^j} \widehat{\beta} )(\lambda)\; d\lambda. \]
    %
    Taking advantage of the oscillation of $\widehat{\beta}$, and the smoothness of $\widehat{\alpha A}$, i.e. integrating by parts, one can shows that for $|\lambda - \rho| \lesssim 1$
    %
    \[  |( \widehat{\alpha A} * \text{Dil}_{\rho / 2^j} \widehat{\beta} )(\lambda)| \lesssim_{N,M} (\rho / 2^j)^N \cdot \rho^{-M} \cdot (\rho / 2^j) = \rho^{1+N-M} 2^{-(N+1)j}, \]
    %
    Taking $N = M$ gives that the kernel is bounded above by
    %
    \[ 2^j \rho^{-1-\delta} \left( \rho 2^{-(N+1)j} \right) = 2^{-Nj}. \]
    %
    But now trivial estimates, e.g. using Schur's lemma implies that
    %
    \[ \| R^\delta_{\rho,j,R} f \|_{L^p(X)} \lesssim_N \rho^{-\delta} 2^{-Nj} \| f \|_{L^p(X)}, \]
    %
    a bound that can be summed in $j$ by taking, e.g. $N = 1$. Thus we are now reduced to the study of the oscillatory integral operators $R^\delta_{\rho,j,Q}$.

    \item Now let's localize. First off, the condition that $K_{\rho,j}$ is supported on the diagonal, and the compactness of $X$, means we need only prove the result restricted to a single coordinate chart. Let $K_{\rho,j,t}$ be the kernel of the operator $R^\delta_{\rho,j,Q}$. Intuitively, the wave equation travels at unit speed, so, since $R^\delta_{\rho,j,Q}$ involves the wave equation localized to times $t \sim 2^j / \rho$, we should expect this kernel to be localized to $|x - y| \lesssim 2^j / \rho$. In fact, we will show that the restricted kernel
    %
    \[ K'_{\rho,j,t}(x,y) = K_{\rho,j,t}(x,y) \cdot \mathbf{I}(|x - y| \geq 2^{j(1 + \varepsilon)} / \rho) \]
    %
    has $L^\infty_y L^1_x$ and $L^\infty_x L^1_y$ bounds of the form $O_{\varepsilon,N}(2^{-jN})$, so that Schur's lemma implies that if we write $(R^\delta_{\rho,j,Q})'$ as the operator with kernel $K'_{\rho,j,t}$, then
    %
    \[ \| (R^\delta_{\rho,j,Q})' f \|_{L^p(X)} \lesssim_N 2^{-jN} \| f \|_{L^p(X)}. \]
    %
    This reduces us to proving localized estimates of the following form: for some $\varepsilon > 0$, and for any function $f$ supported on a ball of radius $2^j / \rho$, we have a bound
    %
    \[ \| R^\delta_{\rho,j,Q} f \|_{L^p( O(2^j / \rho) )} \lesssim 2^{-j \varepsilon} \| f \|_{L^p(X)}. \]
    %
    Notice the localization we get here is slightly weaker than in the Euclidean setting (the operators are localized to balls of radius $O(2^{j(1 + \varepsilon)} / \rho)$ for any $\varepsilon > 0$ rather than localized to balls of radius $O(2^j / \rho)$) which means our bounds here need the slightly greater decay in $j$ (the $O(2^{-j \varepsilon})$ bound above) rather than a bound independent of $j$.

    To prove the bounds for the restricted kernel $K'_{\rho,j,t}$ above, we just apply the principle of nonstationary phase to the integral representation, which says that for $|x - y| \gtrsim 2^{j(1+\varepsilon)} / \rho$ we have, taking the Fourier inversion formula in the $t$ variable,
    %
    \begin{align*}
        K'_{\rho,j,t} &= c_\delta \rho^{-\delta} \int_0^\infty \int \alpha(t) (\text{Dil}_{2^j} \beta)(\rho t) (t + i0)^{-\delta - 1} q(t,x,y,\xi) e^{2 \pi i [\phi(x,y,\xi) + t |\xi| - \rho t]}\; d\xi\; dt\\
        &= \int a^\delta_{\rho,j}(x,y,\xi,|\xi| - \rho) e^{2 \pi i \phi(x,y,\xi)}\; d\xi,
    \end{align*}
    %
    where
    %
    \[ a^\delta_{\rho,j}(x,y,\xi,\cdot) = c_\delta 2^j \rho^{-1-\delta} (\widehat{\alpha q(\cdot, x,y,\xi)} * \text{Dil}_{\rho / 2^j} \beta * r^\delta * q_{\cdot} (x,y,\xi)) \]
    %
    and therefore TODO satisfies estimates of the form
    %
    \[ |\nabla^n_t \nabla^m_\xi a^\delta_{\rho,j}| \lesssim_{n,m,N} 2^{-j \delta} (2^j / \rho)^n \langle 2^j \tau / \rho \rangle^{-N} \langle\xi \rangle^{-m} \]
    %
    Nonstationary phase TODO thus gives the required bounds.

    % Suppose the operator K1 = K I(|x - y| >= 2^{e j} ) has operator norm O_{N,e}( 2^{-Nj} ) for all N > 0
    % Given f, write f = sum f_Q, where f_Q is supported on a cube Q with sidelength 1
    % Then |f|_{Lp}^p = sum |f_Q|_{Lp}^p

    % We write Kf_Q = (K1)f_Q + (1 - K1)f_Q
    
    % Then (1 - K1)f_Q is supported on 2^{e j} * Q, which overlaps with at most O(2^{(e d) j}) other cubes. Thus
    % | (1 - K1) f |_{Lp}^p = | sum (1 - K1)f_Q |_{L^p}^p << 2^{ d e j (p/ p^*)} sum | (1 - K1) f_Q |_p^p
    % If we could show that | (1 - K1) f_Q |_p << 2^{-d e j / p^*}
    % then we would conclude that | (1 - K1) f |_{Lp} << sum |f_Q|_p^p = |f|_{Lp}^p
    
    % Often e is arbitrary, so we need only show any bound of the form | (1 - K1) f_Q |_p << 2^{- ej} to get the required result.

    \item It now suffices to show that for some $\varepsilon > 0$, and for any function $f$ supported on a ball $B$ of radius $2^j / \rho$, we have a bound
    %
    \[ \| R^\delta_{\rho,j,Q} f \|_{L^p( O(1) \cdot B )} \lesssim 2^{-j \varepsilon} \| f \|_{L^p(X)}. \]
    %
    Since we are localized, we can now, like in the Euclidean case, reduce to an $L^2$ bound, i.e. writing
    %
    \[ \| R^\delta_{\rho,j,Q} f \|_{L^p( O(1) \cdot B)} \lesssim (2^j / \rho)^{d(1/p - 1/2)} \| R^\delta_{\rho,j,Q} f \|_{L^2(O(1) \cdot B)}. \]
    %
    It now suffices to note TODO that $R^\delta_{\rho,j,Q}$ is a Fourier multiplier operator with symbol which is pointwise bouned by $O_N(2^{- j \delta}  \langle 2^j \tau / \rho \rangle^{-N})$, so we can now TODO apply Sogge's version of Tomas Stein on manifolds summed over geometric intervals to yield the required bounds.
\end{itemize}

%\section{Proof Involving Carleson-Sj\"{o}lin}

%On $\RR^d$, we can take Fourier transforms, applying stationary phase to determine that if $K_{\rho,\delta}$ is the convolution kernel corresponding to $R_\rho^\delta$, then
%
%\begin{align*}
%    K_\delta(x) &= \int_0^1 \lambda^{n-1} (1 - \lambda)^\delta e^{2 \pi i \xi \cdot x}\; d\lambda\\
%    &= a_1(x) \frac{e^{2 \pi i |x|}}{\langle x \rangle^{\frac{n+1}{2} + \delta}} + a_2(x) \frac{e^{- 2 \pi i |x|}}{\langle x \rangle^{\frac{n+1}{2} + \delta}} + O \left( \frac{1}{\langle x \rangle^{n+1}} \right),
%\end{align*}
%
%where $a_1$ and $a_2$ are symbols of order zero, with $|a_1(x)|, |a_2(x)| \gtrsim 1$ for $|x| \gtrsim 1$. TODO: Necessity of the $\delta(p)$ values.

%\begin{lemma}
%    If $a \in C_c^\infty(\RR^n \times \RR^n)$, and $a(x,y) = 0$ unless $1/2 \leq |x - y| \leq 2$, then
    %
%    \[ \left\| \int e^{2 \pi i \rho |x - y|} a(x,y) f(y) \right\|_{L^q(\RR^n)} \lesssim \rho^{n/q} \| f \|_{L^p(\RR^n)} \]
    %
%    if $q = [(n+1)/(n-1)] p^*$ and $1 \leq p \leq 2$.
%\end{lemma}
%\begin{proof}
%    This is a non homogeneous oscillatory integral operator with wavefront set
    %
%    \[ \left\{ \left( x, y ; \frac{x}{|x - y|}, \frac{y}{|x - y|} \right) \right\} \]
    %
%    which, because of the assumption of the support of $a$, satisfies the Carleson Sj\"{o}lin conditions, and thus the result follows.
%\end{proof}

%The required bounds now follows by applying a dyadic spatial decomposition, rescaling, and applying the result above, hich can be applied because of our explicit computation of the kernel $K_\delta$ above. TODO: Go over this argument in more detail to make sure it actually works.





\chapter{Heo, Nazarov, and Seeger: Initial Radial Conjecture Results}

In this chapter we give a description of the techniques of Heo, Nazarov, and Seeger's 2011 paper \emph{Radial Fourier Multipliers in High Dimensions} \cite{HeoNazrovSeeger2011}. One of the main goals of this paper is to verify the radial multiplier conjecture in $\RR^d$ for $d \geq 4$, and $1 < p < p_d$, where $p_d = 2(d-1)/(d+!)$, i.e. that if $m \in L^\infty(\ZZ)$ is a radial function, $d \geq 4$, and $\eta \in \mathcal{S}(\RR^d)$ is nonzero, then
%
\[ \| m \|_{M^p(\RR^d)} \sim \sup_{t > 0} t^{d/p} \| T_m(\text{Dil}_t \eta) \|_{L^p(\RR^d)} \quad \text{for}\ p \in \left(1, \frac{2(d-1)}{d+1} \right), \]
%
where the implicit constant depends on $p$ and $\eta$. We have
%
\[ \sup_{t > 0} t^{d/p} \| T_m(\text{Dil}_t \eta) \|_{L^p(\RR^d)} \sim \sup_{t > 0} \frac{\| T_m(\text{Dil}_t \eta) \|_{L^p(\RR^d)}}{\| \text{Dil}_t \eta \|_{L^p(\RR^d)}}. \]
%
Thus we find that the boundedness of $T_m$ on $\mathcal{S}(\RR^d)$ is equivalent to it's boundedness on the family of inputs $\{ \text{Dil}_t \eta \}$. If we make the assumption that $m$ is compactly supported, then the assumption is equivalent to the fact that the convolution kernel $k$ associated with $m$ is in $L^p(\RR^n)$.

Another consequence of the techniques of this paper is that an `endpoint' result for local smoothing. Namely, the techniques of the paper imply that if $d \geq 4$, and $q > 2 + 4/(d-3)$, then
%
\[ \frac{1}{2L} \int_{-L}^L \| e^{it \sqrt{-\Delta}} f \|_{L^q(\RR^d)}^q\; dt \lesssim_{q,d} \| (I - L^2 \Delta)^{\alpha/2} f \|_{L^q(\RR^d)}^q, \]
%
where $\alpha = d(1/2 - 1/q) - 1/2$. TODO: Why is this an `endpoint result', i.e. is it because it works for an arbitrarily $L$, rather than a unit time interval like local smoothing normally deals with?

\section{Discretized Reduction}

It is obvious that
%
\[ \| m \|_{M^p(\RR^d)} \gtrsim_\eta \sup_{t > 0} t^{d/p} \| T_m(\text{Dil}_t \eta) \|_{L^p(\RR^d)}, \]
%
so it suffices to show that
%
\[ \| m \|_{M^p(\RR^d)} \lesssim_\eta \sup_{t > 0} t^{d/p} \| T_m(\text{Dil}_t \eta) \|_{L^p(\RR^d)}, \]
%
We will show this via a discrete convolution inequality, which can also be used to prove local smoothing results for the wave equation.

Let $\sigma_r$ be the surface measure for the sphere of radius $r$ centered at the origin in $\RR^d$. Also fix a nonzero, radial, compactly supported function $\psi \in \mathcal{S}(\RR^d)$ whose Fourier transform is non-negative, and vanishes to high order at the origin. Given $x \in \RR^d$ and $r \geq 1$, define $\chi_{x r} = \text{Trans}_x (\sigma_r * \psi)$, which we view as a smooth function oscillation on a thickness $\approx 1$ annulus of radius $r$ centered at $x$. Our goal is to prove the following inequality.

\begin{lemma} \label{lemma1}
    For any $a : \RR^d \times [1,\infty) \to \CC$, and $1 \leq p < p_d$,
    %
    \[ \left\| \int_{\RR^d} \int_1^\infty a(x,r) \chi_{x, r}\; dx\; dr \right\|_{L^p(\RR^d)} \lesssim \left( \int_{\RR^d} \int_1^\infty |a(x,r)|^p r^{d-1} dr dx \right)^{1/p}. \]
    %
    The implicit constant here depends on $p$, $d$, and $\psi$.
\end{lemma}

How does Lemma \ref{lemma1} prove the required result? Suppose $m: \RR^d \to \CC$ is a radial multiplier, so we can consider it's convolution kernel $k: \RR^d \to \CC$, which is also radial. Let $k(x) = b(|x|)$ for some function $b: [0,\infty) \to \CC$. If we set $a(x,r) = g(x) b(r)$ for any function $g: \RR^d \to \CC$, then the function
%
\[ F(x) = \int_{\RR^d} \int_1^\infty a(x',r) \chi_{x', r}\; dx'\; dr, \]
%
is equal to $k * \psi * g$. In this setting, Lemma \ref{lemma1} says that
%
\[ \| k * \psi * g \|_{L^p(\RR^d)} \lesssim \| k \|_{L^p(\RR^d)} \| g \|_{L^p(\RR^d)}. \]
%
The left hand side is a Fourier multiplier operator applied to $g$, with symbol equal to $\widehat{\psi} \cdot m$, which is clearly related to the bounds we want to show. In particular, if $m$ is compactly supported away from the origin, let's say, on the annulus $1/2 \leq |\xi| \leq 2$. If we chose $\psi$ so that $\widehat{\psi}$ is nonvanishing on the annulus $1/4 \leq |\xi| \leq 2$, then the multiplier $1/\widehat{\psi}$ is smooth on the support of $m$, and so satisfies $L^p \to L^p$ bounds for all $1 < p < \infty$ restricted to functions with Fourier support on $m$. In particular, we conclude that $m$ is bounded from $L^p$ to $L^p$ if it's Fourier transform lies in $L^p(\RR^d)$. We can then use other tools (Hardy space technology and the like) to study more general multipliers that aren't compactly supported.

To prove Lemma \ref{lemma1}, it suffices to prove the following discretized estimate where we replace integrals with sums.

\begin{theorem} \label{lemma2}
    Fix a finite family of pairs $\mathcal{E} \subset \RR^d \times [1,\infty)$, which is \emph{discretized} in the sense that for any $(x_1,r_1)$ and $(x_2,r_2)$ in$ \mathcal{E}$, one either has $x_1 = x_2$, or $|x_1 - x_2| \geq 1$, and one either has $r_1 = r_2$, or $|r_1 - r_2| \geq 1$. Then for any $a: \mathcal{E} \to \CC$ and $1 \leq p < 2(d - 1)/(d+1)$, 
    %
    \[ \left\| \sum_{(x,r) \in \mathcal{E}} a(x,r) \chi_{x, r} \right\|_{L^p(\RR^d)} \lesssim \left( \sum_{(x,r) \in \mathcal{E}} |a(x,r)|^p r^{p-1} \right)^{1/p}, \]
    %
    where the implicit constant depends on $p$, $d$, and $\psi$, but most importantly, is independent of $\mathcal{E}$.
\end{theorem}

\begin{proof}[Proof of Lemma \ref{lemma1} from Lemma \ref{lemma2}]
    For any $a: \RR^d \times [1,\infty) \to \CC$, if we consider the vector-valued function $\mathbf{a}(x,r) = a(x,r) \chi_{x,r}$, then
    %
    \[ \int_{\RR^d} \int_1^\infty \mathbf{a}(x,r)\; dr\; dx = \int_{[0,1)^d} \int_0^1 \sum\nolimits_{n \in \ZZ^d} \sum\nolimits_{m > 0} \text{Trans}_{n,m} \mathbf{a}(x,r)\; dr\; dx \]
    %
    Minkowski's inequality thus implies that
    %
    \begin{align*}
    \left\| \int_{\RR^d} \int_1^\infty \mathbf{a}(x,r)\; dr\; dx \right\|_{L^p(\RR^d)} &\leq \int_{[0,1)^d} \int_0^1 \left\| \sum\nolimits_{n \in \ZZ^d} \sum\nolimits_{m > 0} \text{Trans}_{n,m} \mathbf{a}(x,r) \right\|_{L^p(\RR^d)}\; dr\; dx\\
    &\lesssim \int_{[0,1)^d} \int_0^1 \left( \sum\nolimits_{n \in \ZZ^d} \sum\nolimits_{m > 0} |a(x - n, r + m)|^p r^{d-1} \right)^{1/p}\; dr\; dx\\
    &\leq \left( \int_{[0,1)^d} \int_0^1 \sum\nolimits_{n \in \ZZ^d} \sum\nolimits_{m > 0} |a(x - n, r + m)|^p r^{d-1}\; dr\; dx \right)^{1/p}\\
    &= \left( \int_{\RR^d} \int_1^\infty |a(x,r)|^p r^{d-1} dr dx \right)^{1/p}. \qedhere
    \end{align*}
\end{proof}

Lemma \ref{lemma2} is further reduced by considering it as a bound on the operator $a \mapsto \sum_{(x,r) \in \mathcal{E}} a(x,r) \chi_{x,r}$. In particular, applying real interpolation, it suffices for us to prove a restricted strong type bound. Given any discretized set $\mathcal{E}$, let $\mathcal{E}_k$ be the set of $(x,r) \in \mathcal{E}$ with $2^k \leq r < 2^{k+1}$. Then Lemma \ref{lemma2} is implied by the following Lemma.

\begin{lemma} \label{lemma3}
    For any $1 \leq p < 2(d - 1)/(d+1)$ and $k \geq 1$,
    %
    \[ \left\| \sum_{(x,r) \in \mathcal{E}} \chi_{x,r} \right\|_{L^p(\RR^d)} \lesssim_p \left( \sum_{k \geq 1} 2^{k(d-1)} \#(\mathcal{E}_k) \right)^{1/p}. \]
\end{lemma}

\begin{remark}
    Note that if $2^k \leq r \leq 2^{k+1}$, then because $\| \chi_{x,r} \|_{L^p(\RR^d)} \sim 2^{k(d-1)/p}$, we can write this as
    %
    \[ \left\| \sum_{(x,r) \in \mathcal{E}} \chi_{x,r} \right\|_{L^p(\RR^d)} \lesssim_p \left( \sum_{(x,r) \in \mathcal{E}} \| \chi_{x,r} \|_{L^p(\RR^d)}^p \right)^{1/p}. \]
    %
    Thus we are proving a kind of $l^p L^p$ decoupling for the functions $\chi_{x,r}$. This is strictly weaker than an $l^2 L^p$ decoupling bound. TODO: Could we possibly get an $l^2 L^p$ decoupling bound here?
\end{remark}

\begin{comment}
\begin{proof}[Proof of Lemma \ref{lemma2} from Lemma \ref{lemma3}]
    Let
    %
    \[ F = \sum_{(x,r) \in \mathcal{E}} \chi_{x,r} \]
    %
    and then for $k \geq 1$, let
    %
    \[ F_k = \sum_{(x,r) \in \mathcal{E}_k} \chi_{x,r}. \]
    %
    Then $F = \sum_k F_k$, and. Applying a dyadic interpolation result (Lemma 2.2 of that paper), the bound
    %
    \[ \| F_k \|_{L^r(\RR^d)} \lesssim 2^k (2^{k(d-r-1)} \#(\mathcal{E}_k)^{1/r}) \]
    %
    which holds for $r$ to the left and right of $p$, can be interpolated to yield that
    %
    \[ \| F \|_{L^p(\RR^d)} \lesssim \left( \sum_k 2^{kp} ( 2^{k(d-r-1)} ) \right)^{1/p} \]


    Applying a dyadic interpolation result (Lemma 2.2 of the paper), Lemma \ref{lemma3} implies that
    %
    \[ \left\| \sum_{(x,r) \in \mathcal{E}} \chi_{x,r} \right\| \]

    %
    \[ \left\| \sum_{(x,r) \in \mathcal{E}} \chi_{x,r} \right\|_{L^p(\RR^d)} \lesssim \left( \sum 2^{kp} 2^{k(d-p-1)} \#(\mathcal{E}_k) \right)^{1/p} = \left( \sum 2^{k(d-1)} \#(\mathcal{E}_k) \right)^{1/p} \]
    %
    This is a restricted strong type bound for Lemma \ref{lemma2}, which we can then interpolate.
\end{proof}
\end{comment}

\section{Density Decomposition}

To control these sums, we apply a `density decomposition', somewhat analogous to a Calderon Zygmund decomposition, which will enable us to obtain $L^2$ bounds. We say a 1-separated set $\mathcal{E}$ in $\RR^d \times [R,2R)$ is of \emph{density type} $(u,R)$ if
%
\[ \#(B \cap \mathcal{E}) \leq u \cdot \diam(B) \]
%
for each ball $B$ in $\RR^{d+1}$ with diameter $\leq R$.

%A covering argument then shows that for any ball $B$,
%
%\[ \#(B \cap \mathcal{E}) \lesssim_d u \cdot \left( 1 + \frac{\diam(B)}{R} \right)^d \cdot \diam(B). \]
%
%(NOTE: WE MIGHT BE ABLE TO DO BETTER USING THE FACT THAT $\mathcal{E} \subset \RR^d \times [R,2R)$, USING THE VALUE $R$).

\begin{theorem} \label{DecompositionTheorem}
    For any 1-separated set $\mathcal{E}_k \subset \RR^d \times [2^k,2^{k+1})$, we can consider a disjoint union $\mathcal{E}_k = \bigcup_{m = 1}^\infty \mathcal{E}_k(2^m)$ with the following properties:
    %
    \begin{itemize}
        \item For each $m$, $\mathcal{E}_k(2^m)$ has density type $(2^m,2^k)$.

        \item If $B$ is a ball of radius $r \leq 2^k$ containing at least $2^m \cdot r$ points of $\mathcal{E}_k$, then
        %
        \[ B \cap \mathcal{E}_k \subset \bigcup_{m' \geq m} \mathcal{E}_k(2^{m'}). \]

        \item For each $m$, there are disjoint balls $\{ B_i \}$, with radii $\{ r_i \}$, each at most $2^k$, such that
        %
        \[ \sum_i r_i \leq \frac{\#(\mathcal{E}_k)}{2^m} \]
        %
        such that $\bigcup B_i^*$ covers $\bigcup_{m' \geq m} \mathcal{E}_k(2^{m'})$, where $B_i^*$ denotes the ball with the same center as $B_i$ but 5 times the radius.
    \end{itemize}
\end{theorem}
\begin{proof}
    Define a function $M: \mathcal{E}_k \to [0,\infty)$ by setting
    %
    \[ M(x,r) = \sup \left\{ \frac{\#(\mathcal{E}_k \cap B)}{\text{rad}(B)} : (x,r) \in B\ \text{and}\ \text{rad}(B) \leq 2^k \right\}. \]
    %
    We can establish a kind of weak $L^1$ estimate for $M$ using a Vitali type argument. Let
    %
    \[ \widehat{\mathcal{E}}_k(2^m) = \{ (x,r) \in \mathcal{E}_k : M(x,r) \geq 2^m \}. \]
    %
    We can therefore cover $\widehat{\mathcal{E}}_k(2^m)$ by a family of balls $\{ B \}$ such that $\#(\mathcal{E}_k \cap B) \geq 2^m \text{rad}(B)$. The Vitali covering lemma allows us to find a disjoint subcollection of balls $B_1,\dots,B_N$ such that $B_1^* ,\dots, B_N^*$ covers $\widehat{\mathcal{E}}_k(2^m)$. We find that
    %
    \[ \#(\mathcal{E}_k) \geq \sum_i \#(B_i \cap \mathcal{E}_k) \geq 2^m \sum_i \text{rad}(B_i), \]
    %
    Setting $\mathcal{E}_k = \widehat{\mathcal{E}}_k(2^m) - \bigcup_{k' > k} \widehat{\mathcal{E}}_{k'}(2^m)$ thus gives the required result.
\end{proof}

To prove Lemma \ref{lemma3}, we perform a decomposition of $\mathcal{E}_k$ for each $k$, into the sets $\mathcal{E}_k(2^m)$, and then define $\mathcal{E}^m = \bigcup_{k \geq 1} \mathcal{E}_k^m$. For appropriate exponents, we will prove $L^p$ bounds on the functions
%
\[ F^m = \sum_{(x,r) \in \mathcal{E}^m} \chi_{x,r} \]
%
which are exponentially decaying in $m$, i.e. that
%
\[ \| F^m \|_{L^p(\RR^d)} \lesssim m \cdot 2^{-m(1/p - 1/p_d)} \left( \sum_k 2^{k(d-1)} \#(\mathcal{E}_k) \right)^{1/p}. \]
%
Thus summing in $m$ using the triangle inequality gives a bound on $F = \sum_m F^m$, in the range $1 < p < p_d$, i.e. that
%
\[ \| F \|_{L^p(\RR^d)} \lesssim \left( \sum_k 2^{k(d-1)} \#(\mathcal{E}_k) \right)^{1/p}, \]
%
proving Lemma \ref{lemma3}. To get the bound on $F^m$, we interpolate being an $L^2$ bound for $F^m$, and an $L^0$ bound (i.e. a bound on the measure of the support of $F^m$). First, we calculate the support of $F^m$.

\begin{lemma} \label{lemma5}
    For each $k$,
    %
    \[ |\text{supp}(F^m_k)| \lesssim 2^{-m} 2^{k(d-1)} \# \mathcal{E}_k. \]
    %
    Thus we have
    %
    \[ |\text{supp}(F^m)| \leq \sum_k |\text{supp}(F^m_k)| \lesssim \sum_k 2^{-m} 2^{k(d-1)} \# \mathcal{E}_k. \]
\end{lemma}
\begin{proof}
    We recall that for each $k$ and $m$, we can find disjoint balls $B_1,\dots,B_N$ with radii $r_1,\dots,r_N \leq 2^k$ such that
    %
    \[ \sum_{i = 1}^N r_i \leq 2^{-m} \# \mathcal{E}_k, \]
    %
    where $\mathcal{E}_k(2^m)$ is covered by the expanded balls $B_1^* \cup \dots \cup B_N^*$. If we write
    %
    \[ F^m_{k,i} = \sum_{(x,r) \in \mathcal{E}_k(2^m) \cap B_i^*} \chi_{x,r}, \]
    %
    then $\text{supp}(F^m_k) \subset \bigcup_i \text{supp}(F^m_{k,i})$. For each $(x,r) \in B_i^* \cap \mathcal{E}_k(2^m)$, the support of $\chi_{x,r}$, an annulus of thickness $O(1)$ and radius $r$, is contained in an annulus of thickness $O(r_i)$ and radius $O(2^k)$ with the same centre as $B_i$. Thus we conclude that
    %
    \[ |\text{supp}(F^m_{k,i})| \lesssim r_i 2^{k(d-1)}, \]
    %
    and it follows that
    %
    \[ |\text{supp}(F^m_k)| \leq \sum_i r_i 2^{k(d-1)} \leq 2^{-m} 2^{k(d-1)} \# \mathcal{E}_k. \qedhere \]
\end{proof}

From interpolation, it therefore suffices to prove the following $L^2$ estimate on the function $F^m$.

\begin{lemma} \label{lemma6}
    Suppose $\mathcal{E} = \bigcup_k \mathcal{E}_k$ is a one-separated set, where $\mathcal{E}_k \subset \RR^d \times [2^k,2^{k+1})$ is a set of density type $(2^m, 2^k)$. Then
    %
    \[ \left\| \sum_{(x,r) \in \mathcal{E}} \chi_{x,r} \right\|_{L^2(\RR^d)} \lesssim \sqrt{m} \cdot 2^{ \frac{m}{d-1} } \left( \sum_k 2^{k(d-1)} \#(\mathcal{E}_k) \right)^{1/2}. \]
\end{lemma}

% The L2 norms of the chi_{x,r} are equal to 2^{k(d-1)/2}, so the
% triangle ienquality implies that the LHS is bounded by sum_k 2^{k(d-1)/2} \#(E_k)

Note that this bound gets worse and worse as $m$ grows, whereas the support bound gets better and better, since annuli are concentrating in a small set, which is bad from the perspective of constructive interference, but absolutely fine from the perspective of a support bound. Interpolation gives a bound exponentially decaying in $m$ for $1 < p < p_d$.

\begin{comment}
\begin{proof}[Proof of Lemma \ref{lemma3} from Lemma \ref{lemma6}]
    Write $F = \sum_{(x,r) \in \mathcal{E}_k} \chi_{x,r}$, and then perform a decomposition $\mathcal{E}_k = \bigcup_{m \geq 0} \mathcal{E}_k(2^m)$, and thus define $F = \sum_{m \geq 0} F_m$, where
    %
    \[ F_m = \sum_{(x,r) \in \mathcal{E}(2^m)} \chi_{x,r}. \]
    %
    We have
    %
    \[ \| F_m \|_{L^2(\RR^d)} \lesssim 2^{\frac{m}{d-1} + \frac{k(d-1)}{2}} \log(2 + 2^m)^{1/2} \cdot \#(\mathcal{E}_k)^{1/2}. \]
    %
    If we interpolate this bound with the support bound for $F_m$, a kind of $L^0$ norm estimate, we conclude that for $0 < p \leq 2$,
    % ( int |F_m|^p )^{1/p} <= |S|^{1/pq^*} int |F_m|^{pq} )^{1/pq}
    % pq = 2
    % Then q = 2/p so 1/q^* = 1 - 1/q = 1 - p/2 = (2 - p)/2
    % so q^* = 2/(2-p)
    % 1/pq^* = (2-p)/2p = (1/p - 1/2)
    \begin{align*}
        \| F_m \|_{L^p(\RR^d)} &\leq |\text{Supp}(F_m)|^{1/p - 1/2} \| F_m \|_{L^2(\RR^d)}\\
        &\lesssim ( 2^{k(d-1) - m})^{1/p - 1/2} 2^{\frac{m}{d-1} + \frac{k(d-1)}{2}} \log(2 + 2^m)^{1/2} \cdot \#(\mathcal{E}_k)^{1/p} \\
        &\lesssim 2^{m(1/p_d - 1/p)} \log(2 + 2^m)^{1/2} 2^{\frac{k(d-1)}{p}} \#(\mathcal{E}_k)^{1/p}.
    \end{align*}
    % int |F_m|^p <= |S|^{1/p-1/2} ( int |F_m|^2 )^{1/2}
    %
    where $p_d = 2(d-1)/(d+1)$. This bound is summable in $m$ for $p < p_d$, which enables us to conclude that
    %
    \[ \| F \|_{L^p(\RR^d)} \lesssim 2^{\frac{k(d-1)}{p}} \#(\mathcal{E}_k)^{1/p}. \]
    %
    Thus for $1 \leq p < p_d$, we obtain the bound stated in Lemma \ref{lemma3}.
\end{proof}
\end{comment}

\section{$L^2$ Bounds}

Proving \ref{lemma6} is where the weak-orthogonality bounds from Lemma \ref{lemma4} come into play. Indeed, we can write the inequality as
%
\[ \left\| \sum_{(x,r) \in \mathcal{E}} \chi_{x,r} \right\|_{L^2(\RR^d)} \lesssim \sqrt{m} \cdot 2^{ \frac{m}{d-1} } \left( \sum_{(x,r) \in \mathcal{E}} \| \chi_{x,r} \|_{L^2(\RR^d)}^2 \right)^{1/2}, \]
%
and if we had perfect orthogonality, or even almost orthogonality, then we could replace the $\sqrt{m} \cdot 2^{\frac{m}{d-1}}$ term with a constant.

To prove this $L^2$ bound, we require an analysis of the interference patterns of the functions $\chi_{x,r}$, which are supported on various annuli, but oscillate on these annuli. We will use almost orthogonality principles to understand these interference patterns which work the best now we have reduced our analysis to $L^2$ bounds.

%If $\psi$ is compactly supported, and $r$ is sufficiently large depending on the size of this support, then $\chi_{x,r}$ is supported on an annulus with centre $x$, radius $r$, and thickness $O(1)$. Thus $\| \chi_{x,r} \|_{L^p(\RR^d)} \sim r^{(d-1)/p}$, which implies that
%
%\[ \left\| \sum_{(x,r) \in \mathcal{E}_k} \chi_{x,r} \right\|_{L^p(\RR^d)} \gtrsim 2^{k(d-1)/p} \#(\mathcal{E}_k)^{1/p}. \]
%
%Thus this bound can only be true if $p \geq 1$, and becomes tight when $p = 1$, where we actually have
%
%\[ \left\| \sum_{(x,r) \in \mathcal{E}_k} \chi_{x,r} \right\|_{L^1(\RR^d)} \sim 2^{k(d-1)} \#(\mathcal{E}_k) \]
%

\begin{lemma} \label{lemma4}
    For any $N > 0$, $x_1,x_2 \in \RR^d$ and $r_1,r_2 \geq 1$,
    %with $|x_1 - x_2| \geq 1$ or $x_1 = x_2$, and $r_1,r_2 > 1$,
    %
    \begin{align*}
        |\langle \chi_{x_1,r_1}, \chi_{x_2,r_2} \rangle| &\lesssim_N (r_1r_2)^{(d-1)/2} (1 + |r_1 - r_2| + |x_1 - x_2|)^{-(d-1)/2}\\
        &\quad\quad\quad\sum_{\pm,\pm} (1 + ||x_1 - x_2| \pm r_1 \pm r_2|)^{-N}.
    \end{align*}
    %
    In particular,
    %
    \[ |\langle \chi_{x_1,r_1}, \chi_{x_2,r_2} \rangle| \lesssim \left( \frac{r_1r_2}{|(x_1,r_1) - (x_2,r_2)|} \right)^{(d-1)/2} \]
\end{lemma}

\begin{remark}
    Suppose $r_1 \leq r_2$. Then Lemma \ref{lemma4} implies that $\chi_{x_1,r_1}$ and $\chi_{x_2,r_2}$ are roughly uncorrelated, except when $|x_1 - x_2|$ and $|r_1 - r_2|$ is small, and in addition, one of the following two properties hold:
    %
    \begin{itemize}
        \item $r_1 + r_2 \approx |x_1 - x_2|$, which holds when the two annuli are `approximately' externally tangent to one another.

        \item $r_2 - r_1 \approx |x_1 - x_2|$, which holds when the two annuli are `approximately' internally tangent to one another.
    \end{itemize}
    %
    Heo, Nazarov, and Seeger do not exploit the tangency information, though utilizing the tangencies seems important to improve the results they obtain. Laura Cladek's paper exploits this tangency information, to some extent, to obtain the improved result in her paper.
\end{remark}

\begin{proof}
%    We may assume $|x_1 - x_2| \geq 1$, for otherwise the inequality holds trivially since unless $|r_1 - r_2| \lesssim 1$, $f_{x_1r_1}$ and $f_{x_2r_2}$ have disjoint support, and if $|r_1 - r_2| \lesssim 1$ then Cauchy Schwartz implies that
    %
%    \begin{align*}
%        |\langle f_{x_1r_1}, f_{x_2r_2} \rangle| &\lesssim (r_1 r_2)^{(d-1)/2}\\
%        &\lesssim_{N,d} (r_1r_2)^{(d-1)/2} (1 + |r_1 - r_2| + |x_1 - x_2|)^{-(d-1)/2} \sum_{\pm,\pm} (1 + ||x_1 - x_2| \pm r_1 \pm r_2|)^{-N}
%    \end{align*}
%
    We write
    %
    \begin{align*}
        \langle \chi_{x_1 r_1}, \chi_{x_2 r_2} \rangle &= \left\langle \widehat{\chi}_{x_1 r_1}, \widehat{\chi}_{x_2 r_2} \right\rangle\\
        &= \int_{\RR^d} \widehat{\sigma_{r_1} * \psi}(\xi) \cdot \overline{\widehat{\sigma_{r_2} * \psi}(\xi)} e^{2 \pi i (x_2 - x_1) \cdot \xi}\; d\xi\\
        &= (r_1 r_2)^{d-1} \int_{\RR^d} \widehat{\sigma}(r_1 \xi) \overline{\widehat{\sigma}(r_2 \xi)} |\widehat{\psi}(\xi)|^2 e^{2 \pi i (x_2 - x_1) \cdot \xi}\; d\xi.
    \end{align*}
    %
    Define functions $A$ and $B$ such that $B(|\xi|) = \widehat{\sigma}(\xi)$, and $A(|\xi|) = |\widehat{\psi}(\xi)|^2$. Then
    %
    \[ \langle \chi_{x_1, r_1}, \chi_{x_2, r_2} \rangle = C_d (r_1r_2)^{d-1} \int_0^\infty s^{d-1} A(s) B(r_1 s) B(r_2 s) B(|x_2 - x_1| s)\; ds. \]
    %
    Using well known asymptotics for the Fourier transform for the spherical measure, we have
    %
    \[ B(s) = s^{-(d-1)/2} \sum_{n = 0}^{N-1} (c_{n,+} e^{2 \pi i s} + c_{n,-} e^{-2 \pi i s}) s^{-n} + O_N(s^{-N}). \]
    %
    But now substituting in, assuming $A(s)$ vanishes to order $100N$ at the origin, we conclude that
    %
    \begin{align*}
        \langle \chi_{x_1 r_1}, \chi_{x_2 r_2} \rangle &= C_d \left( \frac{r_1r_2}{|x_1 - x_2|} \right)^{(d-1)/2} \sum_{n,\tau} c_{n,\tau}  r_1^{-n_1} r_2^{-n_2} |x_2 - x_1|^{-n_3}\\
        &\quad\quad\quad \Bigg\{ \int_0^\infty A(s) s^{-(d-1)/2}  s^{-n_1-n_2-n_3} e^{2 \pi i (\tau_1 r_1 + \tau_2 r_2 + \tau_3 |x_2 - x_1|) s}\; ds \Bigg\}\\
        &\lesssim_N \left( \frac{r_1r_2}{|x_1 - x_2|} \right)^{\frac{d-1}{2}} \left(1 + \frac{1}{|x_1 - x_2|^N} \right) \sum_{\tau} \left( 1 + |\tau_1 r_1 + \tau_2 r_2 + \tau_3 |x_2 - x_1|| \right)^{-5N}\\
        &\lesssim_N \left( \frac{r_1r_2}{|x_1 - x_2|} \right)^{\frac{d-1}{2}} \left(1 + \frac{1}{|x_1 - x_2|^N} \right) \sum_\tau \left( 1 + |\tau_1 \tau_3 r_1 + \tau_2 \tau_3 r_2 + |x_2 - x_1|| \right)^{-5N}.
    \end{align*}
    %
    This gives the result provided that $1 + |x_1 - x_2| \geq |r_1 - r_2| / 10$ and $|x_1 - x_2| \geq 1$. If $1 + |x_1 - x_2| \leq |r_1 - r_2| / 10$, then the supports of $\chi_{x_1,r_1}$ and $\chi_{x_2,r_2}$ are disjoint, so the inequality is trivial. On the other hand, if $|x_1 - x_2| \leq 1$, then the bound is trivial by the last sentence unless $|r_1 - r_2| \leq 10$, and in this case the inequality reduces to the simple inequality
    %
    \[ \langle \chi_{x_1,r_1}, \chi_{x_2,r_2} \rangle \lesssim_N (r_1 r_2)^{(d-1)/2}. \] 
    %
    But this follows immediately from the Cauchy-Schwartz inequality.
\end{proof}

The exponent $(d-1)/2$ in Lemma \ref{lemma4} is too weak to apply almost orthogonality directly to obtain $L^2$ bounds on $\sum_{(x,r) \in \mathcal{E}_k} \chi_{xr}$ on it's own, but together with the density decomposition assumption we will be able to obtain Lemma \ref{lemma6}.

\begin{proof}[Proof of Lemma \ref{lemma6}]
    Without loss of generality, we may assume that the $k$ such that $\mathcal{E}_k \neq \emptyset$ is $10$-separated. Write
    %
    \[ F = \sum_{(x,r) \in \mathcal{E}} \chi_{x,r} \]
    %
    and $F_k = \sum_{(x,r) \in \mathcal{E}_k} \chi_{x,r}$. First, we deal with $F_{\lesssim m} = \sum_{k \leq 10 m} F_k$ trivially, i.e. writing
    %
    \begin{align*}
        \| F \|_{L^2(\RR^d)} &\lesssim m^{1/2} \left( \sum_{k \leq 10m} \| F_k \|_{L^2(\RR^d)}^2 + \| \sum_{k > 10m} F_k \|_{L^2(\RR^d)} \right)^{1/2}.
    \end{align*}
    %
    We then decompose
    %
    \[ \| \sum_{k > 10 m} F_k \|_{L^2(\RR^d)}^2 \leq \sum_{k > 10 m} \| F_k \|_{L^2(\RR^d)}^2 + 2 \sum_{k' > k > 10m} |\langle F_k, F_{k'} \rangle|. \]
    %
    Let us analyze $\langle F_k, F_{k'} \rangle$. The term will become a sum of the form $\langle \chi_{x,r}, \chi_{y,s} \rangle$, where $r \sim 2^k$ and $s \sim 2^{k'}$. Because of our assumption of being 10-separated, we have $r \leq s / 2^{10}$. If $\langle \chi_{x,r}, \chi_{y,s} \rangle \neq 0$, then since the support of $\chi_{y,s}$ is an annulus of radius $s$ centered at $y$, with thickness $O(1)$, and $\chi_{x,r}$ has support on an annulus of radius $r$ centered at $x$, with thickness $O(1)$, the fact that $r$ is comparatively smaller than $s$ implies that $(x,r)$ must be contained in the annulus of radius $s$ centered at $y$, with thickness $O(2^k)$. Such an annulus is covered by $O( 2^{(k'-k)(d-1)} )$ balls of radius $2^k$. Each ball can only contain $2^{k + m}$ points $(x,r)$, and so there can be at most
    %
    \[ O(2^{k'(d-1)} 2^{-k(d-1)} 2^{k+m} ) = O( 2^{k'(d-1) - k(d-2) + m} ). \]
    %
    pairs $(x,r) \in \mathcal{E}_k$ for which $\langle \chi_{x,r}, \chi_{y,s} \rangle \neq 0$. For such pairs we have
    %
    \[ |\langle \chi_{x,r}, \chi_{y,s} \rangle| \lesssim \left( \frac{2^k 2^{k'}}{2^{k'}} \right)^{\frac{d-1}{2}} = 2^{\frac{k(d-1)}{2}}. \]
    %
    Thus we conclude that
    %
    \[ |\langle F_k, \chi_{y,s} \rangle| \lesssim 2^{-k ( \frac{d-3}{2} ) + k'(d-1) + m }. \]
    %
    Summing over $10m < k < k'$, we conclude that since $d \geq 4$,
    %
    \[ \sum_{10m < k < k'} |\langle F_k, \chi_{y,s} \rangle| \lesssim 2^{k'(d-1) + m} \sum_{10m < k < k'} 2^{-k \frac{d-3}{2}} \lesssim 2^{k'(d-1) + m} 2^{-5m} \lesssim 2^{k'(d-1)}. \]
    %
    But this means that
    %
    \[ \sum_{10m < k < k'} |\langle F_k, F_{k'} \rangle| \lesssim 2^{k'(d-1)} \cdot \# ( \mathcal{E}_{k'} ). \]
    %
    This means that
    %
    \[ \| \sum_{k > 10m} F_k \|_{L^2(\RR^d)}^2 \lesssim \sum_{k > 10m} \| F_k \|_{L^2(\RR^d)}^2 + \sum_{k'} 2^{k'(d-1)} \# (\mathcal{E}_{k'}), \]
    %
    and it now suffices to deal with estimates the $\| F_k \|_{L^2(\RR^d)}$, i.e. the interactions of functions supported on radii of comparable magnitude. To deal with these, we further decompose the radii, writing $[2^k,2^{k+1})$ as the disjoint union of intervals $I_{k,\mu} = [2^k + (\mu - 1) 2^{am}, 2^k + \mu 2^{am}]$, for some $a$ to be chosen later. These interval induces a decomposition $\mathcal{E}_k = \bigcup_\mu \mathcal{E}_{k,\mu}$. Again, incurring a constant loss at most, we may assume that the $\mu$ such that $\mathcal{E}_{k,\mu} \neq \emptyset$ are $10$ separated. We write $F_k = \sum F_{k,\mu}$, and we have
    %
    \[ \| F_k \|_{L^2(\RR^d)}^2 = \sum_\mu \| F_{k,\mu} \|_{L^2(\RR^d)}^2 + \sum_{\mu < \mu'} |\langle F_{k,\mu}, F_{k,\mu'} \rangle|. \]
    %
    We now consider $\chi_{x,r}$ and $\chi_{y,s}$ with $r \in I_{k,\mu}$ and $s \in I_{k',\mu}$. Then we must have $|x - y| \lesssim 2^k$ and $2^{am} \leq |r - s| \lesssim 2^k$, and so we have
    %
    \begin{align*}
        |\sum_{\mu < \mu'} \langle F_{k,\mu}, \chi_{y,s} \rangle| &\lesssim 2^{k(d-1)} \sum_{\substack{(x,r) \in \mathcal{E}_k\\ 2^{am} \leq |(x,r) - (y,s)| \lesssim 2^k}} |(x,r) - (y,s)|^{- \frac{d-1}{2}}\\
        &\lesssim 2^{k(d-1)} \sum_{am \leq l \leq k} 2^{-l(d-1)/2} \# \{ (x,r) \in \mathcal{E}_k: |(x,r) - (y,s)| \sim 2^l \}.
    \end{align*}
    %
    Using the density assumption,
    %
    \[ \# \{ (x,r) \in \mathcal{E}_k: |(x,r) - (y,s)| \sim 2^l \} \lesssim 2^{l + m} \]
    %
    and so we obtain that, again using the assumption that $d \geq 4$,
    %
    \[ |\sum_{\mu < \mu'} \langle F_{k,\mu}, \chi_{y,s} \rangle| \lesssim 2^{k(d-1)} 2^{m(1-a(d-3)/2)}. \]
    %
    Now summing over all $(y,s)$, we obtain that
    %
    \[ |\sum_{\mu < \mu'} \langle F_{k,\mu}, F_{k,\mu'} \rangle| \lesssim 2^{k(d-1)} 2^{m(1 - a(d-3)/2)} \#(\mathcal{E}_{k,\mu'}). \]
    %
    and now summing over $\mu'$ gives that
    %
    \[ \| F_k \|_{L^2(\RR^d)}^2 \lesssim \sum_\mu \| F_{k,\mu} \|_{L^2(\RR^d)}^2 + 2^{k(d-1)} 2^{m(1 - a(d-3)/2)} \# \mathcal{E}_k, \]
    %
    which is a good enough bound if we pick $a$ to be large enough. Now we are left to analyze $\| F_{k,\mu} \|_{L^2(\RR^d)}$, i.e. analyzing interactions between annuli which have radii differing from one another by at most $O(2^{am})$. Since the family of all possible radii are discrete, the set $\mathcal{R}_{k,\mu}$ of all possible radii has cardinality $O(2^{am})$. We do not really have any orthogonality to play with here, so we just apply Cauchy-Schwartz, writing $F_{k,\mu} = \sum_{r \in \mathcal{R}_{k,\mu}} F_{k,\mu,r}$, to write
    %
    \[ \| F_{k,\mu} \|_{L^2(\RR^d)}^2 \lesssim 2^{am} \sum_r \| F_{k,\mu,r} \|_{L^2(\RR^d)}^2. \]
    %
    Recall that $\chi_{x,r} = \text{Trans}_x(\sigma_r * \psi)$, where $\psi$ is a compactly supported function whose Fourier transform is non-negative and vanishes to high order at the origin. In particular, we now make the additional assumption that $\psi = \psi_{\circ} * \psi_{\circ}$ for some other compactly function $\psi_{\circ}$ whose Fourier transform is non-negative and vanishes to high order at the origin. Then we find that $F_{k,\mu,r}$ is equal to the convolution of the function
    %
    \[ A_r = \sum_{(x,r) \in \mathcal{E}} \text{Trans}_x \psi_{\circ} \]
    %
    with the function $\sigma_r * \psi_{\circ}$. Using the standard asymptotics for the Fourier transform of $\sigma_r$, i.e. that for $|\xi| \geq 1$,
    %
    \[ |\widehat{\sigma_r}(\xi)| \lesssim r^{d-1} (1 + r |\xi|)^{- \frac{d-1}{2}}, \]
    %
    and since $|\widehat{\psi_\circ}(\xi)| \lesssim_N |\xi|^N$, we get that if $r \geq 1$, then for $|\xi| \leq 1/r$,
    %
    \[ |\widehat{\sigma_r}(\xi) \widehat{\psi_\circ}(\xi)| \lesssim_N r^{d-1-N} \]
    %
    and for $|\xi| \geq 1/r$,
    %
    \[ |\widehat{\sigma_r}(\xi) \widehat{\psi_\circ}(\xi)| \lesssim_N r^{\frac{d-1}{2}} |\xi|^{-N}. \]
    %
    Thus in particular,the $L^\infty$ norm of the Fourier transform of $\sigma_r * \psi_\circ$ is $O(r^{(d-1)/2})$. Now the functions $\psi_{\circ}$ are compactly supported, so since the set of $x$ such that $(x,r) \in \mathcal{E}$ is one-separated, we find that
    %
    \[ \| A_r \|_{L^2(\RR^d)} \lesssim \# \{ x : (x,r) \in \mathcal{E} \}^{1/2}. \]
    %
    But this means that
    %
    \[ \| F_{k,\mu,r} \|_{L^2(\RR^d)} = \| A_r * (\sigma_r * \psi_{\circ}) \|_{L^2(\RR^d)} \lesssim r^{\frac{d-1}{2}} \# \{ x : (x,r) \in \mathcal{E} \}^{1/2}. \]
    %
    Thus we have that
    %
    \[ \| F_{k,\mu} \|_{L^2(\RR^d)}^2 = 2^{am} \cdot \# \mathcal{E}_{k,\mu} \cdot 2^{k(d-1)}. \]
    %
    Summing over $\mu$ gives that
    %
    \[ \| F_k \|_{L^2(\RR^d)}^2 = 2^{k(d-1)} \# \mathcal{E}_k (2^{am}  + 2^{m(1 - a(d-3)/2)}). \]
    %
    Picking $a = 2/(d-1)$ optimizes this bound, giving
    %
    \[ \| F_k \|_{L^2(\RR^d)} \lesssim 2^{m/(d-1)} 2^{k(d-1)/2} (\# \mathcal{E}_k)^{1/2}. \]
    %
    Plugging this into the estimates we got for $F$ gives the required bound.
\end{proof}

\chapter{Cladek: Improvements Using Incidence Geometry}

The results of Heo, Nazarov, and Seeger only apply when $d \geq 4$. Cladek found a method to get an initial radial multiplier conjecture result in $\RR^3$, and an improvmeent of the bounds obtained by Heo, Nazarov, and Seeger when $d = 3$. The idea is to exploit the fact that one need only prove a version of \ref{lemma2} for a set $\mathcal{E} = \mathcal{E}_X \times \mathcal{E}_R$, where $\mathcal{E}_X$ is a one-separated family of points, and $\mathcal{E}_R$ are a family of radii. One can then exploit this Cartesian product structure when analyzing functions of the form
%
\[ F = \sum_{(x,r) \in \mathcal{E}} \chi_{x,r}, \]
%
in particular, improving upon the result of \cite{HeoandNazarovandSeeger}.

\section{Result in 3 Dimensions}

As in \cite{HeoandNazarovandSeeger}, Cladek first performs a density decomposition, i.e. writing
%
\[ F = \sum F_k^m \]
%
where
%
\[ F_k^m = \sum_{(x,r) \in \mathcal{E}_k(2^m)} \chi_{x,r}. \]
%
Cladek then interpolates between an $L^0$ bound and an $L^2$ bound on the resulting functions. The $L^0$ bound is exactly the same bound used in \cite{HeoandNazarovandSeeger}.

\begin{theorem}
    For the function $F$, we have
    %
    \[ |\text{supp}(F_k^m)| \lesssim 2^{-m} 4^k \# \mathcal{E}_k \]
    %
    and thus
    %
    \[ |\text{supp}(F^m)| \lesssim \sum_k 2^{-m} 4^k \# \mathcal{E}_k. \]
\end{theorem}

The $L^2$ bound is improved upon, which is what allows us to obtain a new result in three dimensions.

\begin{lemma} \label{cladeksl2}
    Suppose $\mathcal{E} = \bigcup_k \mathcal{E}_k$ is a one-separated set, where $\mathcal{E}_k \subset \RR^d \times [2^k,2^{k+1})$ is a set of density type $(2^m, 2^k)$. Then
    %
    \[ \left\| \sum_{(x,r) \in \mathcal{E}} \chi_{x,r} \right\|_{L^2(\RR^d)} \lesssim_\varepsilon 2^{[(11/13) + \varepsilon] m} \sum_k 4^k \# \mathcal{E}_k. \]
\end{lemma}

Interpolation thus yields that for a set of density type $2^m$ as in this Lemma,
%
\[ \| \sum_{(x,r) \in \mathcal{E}} \chi_{x,r} \|_{L^p(\RR^d)} \lesssim_\varepsilon 2^{-m(1/p - 12/13 - \varepsilon)} ( \sum_k 4^k \# \mathcal{E}_k )^{1/p}. \]
%
If $1 < p < 13/12$, this sum is favorable in $m$, and may be summed without harm to prove the radial multiplier conjecture for unit scale radial multipliers in this range.

\begin{proof} [Proof of Lemma \ref{cladeksl2}]
    Write
    %
    \[ F_k = \sum_{(x,r) \in \mathcal{E}_k} \chi_{x,r}. \]
    %
    As before, we can throw away terms for $k \leq 10 m$, i.e. obtaining that
    %
    \[ \| \sum F_k \|_{L^2(\RR^d)} \lesssim m^{1/2} \left( \sum_k \| F_k \|_{L^2(\RR^d)}^2 + \sum_{10m < k < k'} |\langle F_k, F_{k'} \rangle| \right)^{1/2}. \]
    %
    Our proof thus splits into two cases: where the radii are incomparable, and where the radii are comparable.

    TODO:
\end{proof}

\section{Results in 4 Dimensions}

TODO





\chapter{Mockenhaupt, Seeger, and Sogge: Exploiting Wave-Equation Periodicity}

The main goal of the paper \emph{Local Smoothing of Fourier Integral Operators and {C}arleson-{S}j\"{o}lin Estimates} is to prove local regularity theorems for a class of Fourier integral operators in $I^\mu(Z,Y;\mathcal{C})$, where $Y$ is a manifold of dimension $n \geq 2$, and $Z$ is a manifold of dimension $n+1$, which naturally arise from the study of wave equations. A consequence of this result will be a local smoothing result for solutions to the wave equation, i.e. that if $2 < p < \infty$, then there is $\delta$ depending on $p$ and $n$, such that if $T: Y \to Y \times \RR$ is the solution operator to the wave equation, and $Y$ is a compact manifold whose geodesics are periodic, then $T$ is continuous from from $L^p_c(Y)$ to $L^p_{\alpha,\text{loc}}(Y \times \RR)$ for $\alpha \leq -(n-1)|1/2 - 1/p| + \delta$. Such a result is called local smoothing, since if we define $Tf(t,x) = T_tf(x)$, then the operator $T_t$ is, for each $t$, a Fourier integral operator of order zero, with canonical relation
%
\[ \mathcal{C}_t = \{ (x,y;\xi,\xi) : x = y + t \widehat{\xi} \}, \]
%
where $\widehat{\xi} = \xi / |\xi|$ is the normalization of $\xi$. Standard results about the regularity of hyperbolic partial differential equations show that each of the operators $T_t$ is continuous from $L^p_c(Y)$ To $L^p_{\alpha,\text{loc}}(Y \times \RR)$ for $\alpha \leq -(n-1)|1/2 - 1/p|$, and that this bound is sharp. Thus $T$ is \emph{smoothing} in the $t$ variable, so that for any $f \in L^p$, the functions $T_t f$ `on average' gain a regularity of $\delta$ over the worst case regularity at each time. The local smoothing conjecture states that this result is true for any $\delta < 1/p$.

The class of Fourier integral operators studied are those satisfying the following condition: as is standard, the canonical relation $\mathcal{C}$ is a conic Lagrangian manifold of dimension $2n + 1$. The fact that $\mathcal{C}$ is Lagrangian implies $\mathcal{C}$ is locally parameterized by $(\nabla_\zeta H(\zeta, \eta), \nabla_\eta H(\zeta, \eta),\zeta,\eta)$, where $H$ is a smooth, real homogeneous function of order one. If we assume $\mathcal{C} \to T^* Y$ is a submersion, then $D_\xi [\nabla_\eta H(\zeta,\eta)]$ has full rank, which implies $D_\eta [\nabla \xi H(\zeta, \eta)] = (D_\xi [\nabla_\eta H(\zeta, \eta)])^T$ has full rank, and thus the projection $\mathcal{C} \to T^* Z$ is an immersion. We make the further assumption that the projection $\mathcal{C} \to Z$ is a submersion, from which it follows that for each $z$ in the image of this projection, the projection of points in $\mathcal{C}$ onto $T^*_z Z$ is a conic hypersurface $\Gamma_z$ of dimension $n$. The final assumption we make is that all principal curvatures of $\Gamma_z$ are non-vanishing.

\begin{remark}
    The projection properties of $\mathcal{C}$ imply that, in $T^* (Z \times Y)$, there exists a smooth phase $\phi$ defined on an open subset of $Z \times T^* Y$, homogeneous in $T^* Y$, such that locally we can write $\mathcal{C}$ as $(z, \nabla_z \phi(z,\eta), \nabla_\eta \phi(z,\eta), \eta)$ for $\eta \neq 0$. Then, working locally on conic sets,
    %
    \[ \Gamma_z = \{ (\nabla_z \phi(z,\eta)) \}, \]
    %
    and the curvature condition becomes that the Hessian $H_{\eta \eta} \langle \nabla_z \phi, \nu \rangle$ has constant rank $n-1$, where $\nu$ is the normal vector to $\Gamma_z$. This is a natural homogeneous analogue of the Carleson-Sj\"{o}lin condition for non-homogeneous oscillatory integral operators, i.e. the Carleson-Sj\"{o}lin condition is allowed to assume $H_{\eta \eta} \phi$ has rank $n$, which cannot be possible in our case, since $\phi$ is homogeneous here. An approach using the analytic interpolation method of Stein or the Strichartz / Fractional Integral approach generalizes the Carleson-Sj\"{o}lin theorem to show that for any smooth, non-homogeneous phase function $\Phi: \RR^{n+1} \times \RR^n \to \RR$, and any compactly supported smooth amplitude $a$ on $\RR^{n+1} \times \RR^n$. Consider the operators
    %
    \[ T_\lambda f(z) = \int a(z,y) e^{2 \pi i \lambda \Phi(z,y)} f(y)\; dy. \]
    %
    If the associated canonical relation $\mathcal{C}$, if $\mathcal{C}$ projects submersively onto $T^* \RR^n$, so that for each $z \in \RR^{n+1}$ in the image of the projection map $\mathcal{C}$, the set $S_z \subset \RR^{n+1}$ obtained from the inverse image of the projection of $\mathcal{C} \to Z$ at $z$ is a $n$ dimensional hypersurface with $k$ non-vanishing curvatures. Then for $1 \leq p \leq 2$,
    %
    \[ \| T_\lambda f \|_{L^q(\RR^{n+1})} \lesssim \lambda^{-(n+1)/q} \| f \|_{L^p(\RR^n)}. \]
    %
    where $q = p^*(1 + 2/k)$.
\end{remark}

\begin{remark}
    We can also see these assumptions as analogues in the framework of cinematic curvature, splitting the $z$ coordinates into `time-like' and 'space-like' parts. Working locally, because $\mathcal{C} \to T^* Y$ is a submersion, we can consider coordinates $z = (x,t)$ so that, with the phase $\phi$ introduced above, $D_x (\nabla_\eta \phi)$ has full rank $n$, and that $\partial_t \phi(x,t,\eta) \neq 0$. Then for each $z = (x,t)$, we can locally write $\partial_t \phi(x,t,\eta) = q(x,t,\nabla_x \phi(x,t,\eta))$, homogeneous in $\eta$, and then
    %
    \[ \mathcal{C} = \{ (x,t,y;\xi,\tau,\eta) : (x,\xi) = \chi_t(y,\eta), \tau = q(x,t,\xi) \}, \]
    %
    where $\chi_t$ is a canonical transformation. Our curvature conditions becomes that $H_{\xi \xi} q$ has full rank $n-1$. This is the cinematic curvature condition introduced by Sogge. %TODO: READ SOGGE, PROPOGATION OF SINGULARITIES AND MAXIMAL FUNCTIONS IN THE PLANE, WHICH INTRODUCES CINEMATIC CURVATURE?
\end{remark}

Under these assumptions, the paper proves that any Fourier integral operator $T$ in $I^{\mu - 1/4}(Z,Y;\mathcal{C})$ maps $L^2_c(Y)$ to $L^q_{\text{loc}}(Z)$ if
%
\[ 2 \left( \frac{n+1}{n-1} \right) \leq q < \infty \quad\text{and}\quad \mu \leq - n (1/2 - 1/q) + 1/q. \]
%
and maps $L^p_c(Y)$ to $L^p_{\text{loc}}(Z)$ if
%
\[ p > 2 \quad\text{and}\quad \mu \leq -(n-1)(1/2 - 1/p) + \delta(p,n). \]
%
If we introduce time and space variables locally as in the remark above, any operator in $I^{\mu - 1/4}(Z,Y;\mathcal{C})$ can be written locally as a finite sum of operators of the form
%
\[ Tf(x) = \int_{-\infty}^\infty T_t f(x), \]
%
where
%
\[ T_t f(x) = \int a(t,x,\eta) e^{2 \pi i \phi(x,t,y,\eta)} f(y)\; dy\; d\eta. \]
%
is a Fourier integral operator whose canonical relation is a locally a canonical graph, then the general theory implies that each of the maps $T_t$ maps $L^2_c(Y)$ to $L^q_{\text{loc}}(X)$ if
%
\[ 2 \leq q \leq \infty \quad\text{and}\quad \mu \leq -n(1/2 - 1/q) \]
%
so that here we get local smoothing of order $1/q$, and also maps $L^p_c(Y)$ to $L^p_{\text{loc}}(X)$ if
%
\[ 1 < p < \infty \quad\text{and}\quad \mu \leq -(n-1)|1/p - 1/2| \]
%
so we get $\delta(p,n)$ smoothing. A consequence of the smoothing, via Sobolev embedding, is a maximal theorem result for the operator $T_t$, i.e. that for any finite interval $I$, the operator
%
\[ Mf = \sup_{t \in I} |T_t f| \]
%
maps $L^p_c(Y)$ to $L^p_{\text{loc}}(X)$ if $\mu < -(n-1)(1/2 - 1/p) - (1/p - \delta(p,n))$. If the local smoothing conjecture held, we would conclude that, except at the endpoint $T^*$ has the same $L^p_c(Y)$ to $L^p_{\text{loc}}(X)$ mapping properties as each of the operators $T_t$. We also get square function estimates, such that for any finite interval $I$, if we consider
%
\[ Sf(x) = \left( \int_I |T_t f(x)|^2\; dt \right)^{1/2}, \]
%
then for
%
\[ 2 \frac{n+1}{n-1} \leq q < \infty \quad\text{and}\quad \mu \leq -n(1/2 - 1/q) + 1/2, \]
%
the operator $S$ is bounded from $L^2_c(Y)$ to $L^q_{\text{loc}}(X)$.

Our main reason to focus on this paper is the results of the latter half of the paper applying these techniques to radial multipliers on compact manifolds with periodic geodesics. Thus we consider a compact Riemannian manifold $M$, such that the geodesic flow is periodic with minimal period $2 \pi \cdot \Pi$. We consider $m \in L^\infty(\RR)$, such that $\sup_{s > 0} \| \beta \cdot \text{Dil}_s m \|_{L^2_\alpha(\RR)} = A_\alpha$ is finite for some $\alpha > 1/2$ and some $\beta \in C_c^\infty(\RR)$. We define a `radial multiplier' operator
%
\[ Tf = \sum_\lambda m(\lambda) E_\lambda f \]
%
where $E_\lambda$ is the projection of $f$ onto the space of eigenfunctions for the operator $\sqrt{-\Delta}$ on $M$ with eigenvalue $\lambda$. We can also write this operator as $m(\sqrt{-\Delta})$. Then the wave propogation operator $e^{2 \pi i t \sqrt{-\Delta}}$ is periodic of period $\Pi$. The Weyl formula tells us that the number of eigenvalues of $\sqrt{-\Delta}$ which are smaller than $\lambda$ is equal to $V(M) \cdot \lambda^n + O(\lambda^{n-1})$.

\begin{theorem}
    Let $m \in L^2_\alpha(\RR)$ be supported on $(1,2)$, and assume $\alpha > 1/2$, then for $2 \leq p \leq 4$, $f \in L^p(M)$, and for any integer $k$,
    %
    \[ \left\| \sup_{2^k \leq \tau \leq 2^{k+1}} |\text{Dil}_\tau m(\sqrt{-\Delta}) f| \right\|_{L^p(M)} \lesssim_\alpha \| m \|_{L^2_\alpha(M)} \| f \|_{L^p(M)}. \]
\end{theorem}
\begin{proof}
    To understand the radial multipliers we apply the Fourier transform, writing
    %
    \[ T_\tau f = (\text{Dil}_\tau m)(\sqrt{-\Delta}) f = m(\sqrt{-\Delta} / \tau) f = \int_{-\infty}^\infty \tau \widehat{m}(t \tau) e^{2 \pi i t \sqrt{-\Delta}} f\; dt. \]
    %
    If we define $\beta \in C_c^\infty((1/2,8))$ such that $\beta(s) = 1$ for $1 \leq s \leq 4$, and set $L_k f = \text{Dil}_{2^k} \beta(\sqrt{-\Delta}) f$, then for $2^k \leq \tau \leq 2^{k+1}$
    %
    \[ T_\tau f = (\text{Dil}_\tau m)(\sqrt{-\Delta}) f = (\text{Dil}_\tau m \cdot \text{Dil}_{2^k} \beta)(\sqrt{-\Delta}) = T_\tau L_k f. \]
    %
    so Cauchy-Schwartz implies that
    %
    \begin{align*}
        |T_\tau f(x)| &= \left| \int_{-\infty}^\infty \tau \widehat{m}(\tau) e^{2 \pi i t \sqrt{-\Delta}} L_k f(x)\; dt \right|\\
        &\leq \| m \|_{L^2_\alpha(M)} \left( \int_{-\infty}^\infty \frac{\tau}{(1 + |t \tau|^2)^\alpha} |e^{2 \pi i t \sqrt{-\Delta}} L_k f(x)|^2 \right)^{1/2}\\
        &\leq \| m \|_{L^2_\alpha(M)} \left( \int_{-\infty}^\infty \frac{2^k}{(1 + |2^k t|^2)^\alpha} |e^{2 \pi i t \sqrt{-\Delta}} L_k f(x)|^2 \right)^{1/2}
    \end{align*}
    %
    Because of periodicity, if we set $w_k(t) = 2^k / (1 + |2^k t|^2)^\alpha$, it suffices to prove that for $\alpha > 1/2$,
    %
    \[ \left\| \left( \int_0^\Pi w_k(t) |e^{2 \pi i t \sqrt{-\Delta}} L_k f(x)|^2\; dt \right)^{1/2} \right\|_{L^p(M)} \lesssim_{\alpha,p} \| f \|_{L^p(M)}. \]
    %
    This is a weighted combination of the wave propogators, roughly speaking, assigning weight $2^k$ for $t \lesssim 1/2^k$, and assigning weight $1/t$ to values $t \gtrsim 1/2^k$.

    For a fixed $0 < \delta$, we can split this using a partition of unity into a region where $t \gtrsim \delta$ and a region where $t \lesssim \delta$, where $\delta$ is independent of $k$. For each $t$, the wave propogation $e^{2 \pi i t \sqrt{-\Delta}}$ is a Fourier integral operator of order zero (we have an explicit formula for small $t$, and the composition calculus for Fourier integral operators can then be used to give a representation of the propogation operators for all times $t$, such that the symbols of these operators are locally uniformly bounded in $S^0$). Thus the square function estimate above can be applied in the region where $t \gtrsim \delta$, because the weighted square integral above has weight $O_\delta(1)$ uniformly in $k$.

    Next, we move onto the region $t \lesssim 1/2^k$. The symbol of the operator $e^{2 \pi i t \sqrt{-\Delta}}$

    Finally we move onto the region $1/2^k \lesssim t \lesssim \delta$. On this region we have $w_k(t) \sim 1/t$, which hints we should try using dyadic estimates. In particular, suppose that for $\gamma \leq \delta$, we have a family of dyadic estimates of the form
    %
    \[ \left\| \left( \int_\gamma^{2\gamma} |e^{2 \pi i t \sqrt{-\Delta}} L_k f|^2\; dt \right)^{1/2} \right\|_{L^p(M)} \lesssim \gamma^{1/2} (1 + \gamma 2^k)^\varepsilon \cdot \| f \|_{L^p(M)}. \]
    %
    Summing over the $O(k)$ dyadic numbers between $1/2^k$ and $\delta$ gives
    %
    \[ \left\| \left( \int_{1/2^k \lesssim t \lesssim \delta} |e^{2 \pi i t \sqrt{-\Delta}} L_k f|^2\; \frac{dt}{t} \right)^{1/2} \right\|_{L^p(M)} \lesssim 2^{\varepsilon k} \| f \|_{L^p(M)} \]






    If we were able to obtain this inequality for some $\varepsilon > 0$, then we could bound


     that for all $0 < \gamma < \Pi/2$


    If we localize near $t \lesssim 1/2^k$ by multiplying by $\phi(2^k t)$ for some compactly supported smooth $\phi$ supported on $|t| \lesssim 1$, then for $t$ on the support of $\phi(2^k t)$ we have a weight proportional to $2^k$, and rescaling shows that it suffices to bound the quantities
    %
    \[ \left\| \left( \int \phi(t) |e^{2 \pi i (t/2^k) \sqrt{-\Delta}} L_k f(x)|^2\; dt \right)^{1/2} \right\| \]

     the family of functions
    %
    \[ \left\| \left( \int |\phi(t) e^{2 \pi i (t / 2^k) \sqrt{-\Delta}} L_k f(x)|^2\; Dt \right)^{1/2} \right\|_{L^p_x} \lesssim \sup \| e^{2 \pi i (t / 2^k) \sqrt{-\Delta}} L_k f \|_{L^p_x} \]


    $a_k(t) = 2^{-k/2} \widehat{\phi}(t/2^k) \beta(\tau/2^k)$

    it suffices to uniformly bound quantities of the form
    %
    \[ \left\| \left( \int 2^k \phi(2^k t) |e^{2 \pi i \sqrt{-\Delta}} L_k f(x)|^2\; dt \right)^{1/2} \right\|_{L^p(M)} \lesssim_{\alpha,p} \| f \|_{L^p(M)} \]
    %
    We now apply a dyadic decomposition to deal with the smaller values of $t$. Let us assume for simplicity of notation that $\delta < 1$, and then consider a partition of unity $1 = \sum_{j = 1}^\infty \phi(2^j t)$ for $0 \leq t \leq 1$, and such that $\phi$ is localized near $1/4 \leq t \leq 2$, then our goal is to bound the quantities
    %
    \[ \left\| \left( \int_{-\infty}^\infty \phi(2^j t) \frac{2^k}{(1 + |2^k t|^2)^\alpha} |A_t L_k f(x)|^2\; dt \right)^{1/2} \right\|_{L^p(M)}, \]
    %
    which are each proportional to
    \[ s \]
\end{proof}







\chapter{Lee and Seeger: Decomposition Arguments For Estimating Fourier Integral Operators}

Let's now discuss a paper \cite{LeeSeeger} entitled \emph{Lebesgue Space Estimates For a Class of Fourier Integral Operators Associated With Wave Propogation}. In this paper, Lee and Seeger prove a variable coefficient version of the result of Heo, Nazarov, and Seeger, i.e. generalizing that result as it applies to sharp local smoothing on $\RR^d$ to the local smoothing of Fourier integral operators satisfying the cinematic curvature condition.

We consider a localized Fourier integral operator $T: \mathcal{D}(Y) \to \mathcal{D}^*(Z)$ of order $\mu - 1/4$, where $\dim(Y) = d$ and $\dim(Z) = d + 1$, with a canonical relation $\mathcal{C}$ (which must be a $2d + 1$ dimensional submanifold of $T^* Z \times T^* Y$ by virtue of the fact it is Lagrangian), satisfying the following properties:
%
\begin{itemize}
    \item The projection map $\pi_{T^* Y}: \mathcal{C} \to T^* Y$ is a submersion. It follows that around any point $(z_0,y_0;\zeta_0,\eta_0)$ we can choose coordinate systems $y$ on $Y$ and $z = (x,t)$ on $Z$ centered at $z_0$ and $y_0$ such that $\zeta_0 = dx_1$, $\eta_0 = dy_1$, and the tangent plane to $\mathcal{C}$ at this point is given by
    %
    \[ dx = dy \quad\text{and}\quad d\xi = d\eta \quad\text{and}\quad d\tau = 0. \]
    %
    In particular, it follows that $\pi_Z : \mathcal{C} \to Z$ is a submersion, and we can locally find a function $\phi(z,\eta)$, homogeneous in $\eta$, such that, locally,
    %
    \[ \mathcal{C} = \{ (z, \nabla_\eta \phi(z,\eta) ; \nabla_z \phi(z,\eta), \eta) \}. \]
    %
    By assumption on the tangent space of $\mathcal{C}$,
    %
    \[ \nabla_\eta \phi(0,e_1) = 0 \quad\text{and}\quad \nabla_z \phi(0,e_1) = e_1. \]
    %
    The equivalence of phase theorem implies we can find a symbol $a(x,t,y,\eta)$ of order $\mu$ such that, after appropriately localizing the operator $T$, we have
    %
    \[ Tf(x,t) = \int a(x,t,y,\eta) e^{2 \pi i [\phi(x,t,\eta) - y \cdot \eta]} f(y)\; d \eta\; dy. \]

    \item The last assumption implies that for each $z_0$, $\Sigma_{z_0} \pi_Z^{-1}(z_0)$ is a $d$ dimensional submanifold of $\mathcal{C}$. Moreover, our choice of coordinates makes it easy to see that the natural map $\Sigma_{z_0} \to T^*_{z_0} Z$ is an immersion, whose image is the immersed hypersurface $\Gamma_{z_0}$ of $T^*_{z_0}$. Indeed, the tangent plane to $\Sigma_{z_0}$ at the point above is given in coordinates by
    %
    \[ dx = dy = dt = d\tau = 0 \quad\text{and}\quad d\xi = d\eta. \]
    %
    And this is projected injectively to the plane defined by $d\tau = 0$ in $T^*_{z_0} Z$. Our other assumption we make about $\mathcal{C}$ is an assumption on \emph{cinematic curvature}. We assume that for each $z_0$, the hypersurface $\Sigma_{z_0}$ is a cone with $l$ nonvanishing principal curvatures, for some $1 \leq l \leq d-1$. Since
    %
    \[ \Sigma_{z_0} = \{ (z_0; \nabla_z \phi(z_0,\eta_0) \}. \]
    %
    The projection assumptions imply that the $(d+1) \times d$ matrix $D_\eta \nabla_z \phi$ has full rank, and the curvature assumptions imply that the Hessian matrix $H_\eta \{ \partial \phi / \partial t \}$ has rank at least $l$ in a neighborhood of our initial point.
\end{itemize}
%
Given these assumptions, the following result is obtained.

\begin{theorem}
    If
    %
    \[ l \geq 3 \quad\text{and}\quad \frac{2l}{l-2} < q < \infty \quad\text{and}\quad \mu \leq \frac{d}{q} - \frac{d-1}{2}, \]
    %
    then $T$ maps $L^q(Y)$ into $L^q(Z)$.
\end{theorem}

If we take $l = d-1$, we get the full assumption of `cinematic curvature' and we can use this to get results about local smoothing of the wave equation on compact Riemannian manifolds, which recovers the local smoothing result of Heo, Nazarov, and Seeger obtained in their paper on radial Fourier multipliers.

\begin{theorem}
    Consider a finite interval $I$, as well as
    %
    \[ d \geq 4 \quad\text{and}\quad \frac{2(d-1)}{d - 3} < q < \infty. \]
    %
    If $M$ is a compact Riemannian manifold, and $\alpha = (d-1)/2 - d/q$, then
    %
    \[ \| e^{it \sqrt{-\Delta}} f \|_{L^q_t(I) L^q_x(M)} \lesssim_I \| f \|_{L^q_\alpha(M)}. \]
\end{theorem}
\begin{proof}
    For any compact time interval $I$, the Lax parametrix construction allows one, for any suitably small coordinate system, to find a phase function
    %
    \[ \phi(x,y,\xi) \approx (x - y) \cdot \xi \]
    %
    and a symbol $a$ of order zero such that
    %
    \[ \text{supp}(a) \subset \{ (x,t,y,\eta) : |x - y| \lesssim 1\ \text{and}\ |\eta| \gtrsim 1 \}, \]
    %
    such that if $\Phi(t,x,y,\eta) = \phi(x,y,\eta) + t |\eta|_g$, then, modulo smoothing operators, for $|t| \lesssim 1$ we have
    %
    \[ (e^{it \sqrt{-\Delta}} f)(x) = \int a(x,t,y,\eta) e^{2 \pi i \Phi(t,x,y,\eta)} f(y)\; d\eta\; dy. \]
    %
    Now define
    %
    \[ Tf(x,t) = \int a(x,t,y,\xi) e^{2 \pi i \Phi(t,x,y,\xi)} f(y)\; d\xi\; dy, \]
    %
    and set
    %
    \[ Sf = T \{ (1-\Delta)^{-\alpha/2} f \}. \]
    %
    Then 
    %
    \[ Sf(x) = \int \left[ a(x,t,y',\eta) (1 + |\xi|^2)^{-\alpha / 2} \right] e^{2 \pi i (\Phi(t,x,y',\eta) + \xi \cdot (y' - y))} f(y)\; d\eta\; dy'\; dy. \]
    %
    % Symbol of order - alpha
    % integrating over N = 2d variables
    % Kernel is defined on n = d + (d+1) = 2d + 1 dimensional space.
    % (mu-1/4) - N/2 + n/4 = -alpha
    % (mu-1/4) - d + (2d + 1)/4 = -alpha
    % (mu-1/4) = (2d - 1)/4 - alpha
    % mu = d/2 - alpha
    Then $S$ is a Fourier integral operator of order $\mu - 1/4$, where $\mu = - \alpha$. The canonical relation of $S$ (as with $T$) is
    %
    \[ \mathcal{C} = \{ (x,t,y; \eta, \omega, \eta) : x = \exp_y(t \xi / |\xi|)\ \text{and}\ \omega = |\xi|_g \}. \]
    %
    One immediately sees that the projection condition is satisfied, and if we are working on a coordinate system localized smaller than the injectivity radius of $M$, for each $z_0 = (x_0,t_0)$,
    %
    \[ \Gamma_{z_0} = \{ (\xi, \omega) : \omega = |\xi|_g \} \]
    %
    is a spherical cone, and thus has $d-1$ nonvanishing principal curvatures. Applying the result, we see that for
    %
    \[ d \geq 4 \quad\text{and}\quad \frac{2(d-1)}{d-3} < q < \infty \quad\text{and}\quad \alpha \geq \frac{d-1}{2} - \frac{d}{q}. \]
    %
    the operator $S$ maps $L^q(M)$ into $L^q(M \times \RR)$, which is equivalent to $T$ mapping $L^q_\alpha(M)$ into $L^q(M \times \RR)$.
\end{proof}

\section{Frequency Localization and Discretization}

Let us describe the idea of the proof. Let $K(z,y)$ denote the kernel of $T$, i.e.
%
\[ K(x,t,y) = \int a(x,t,y,\eta) e^{2 \pi i [\phi(x,t,\eta) - y \cdot \eta]}\; d\eta. \]
%
Without loss of generality, we may assume that $a$ is supported on $|\eta| \geq 1$, since integrals over small frequencies give a smoothing operator. We perform a frequency decomposition, writing
%
\[ K(z,y) = \sum_{i = 1}^\infty 2^{i \mu} K_i(z,y) \]
%
where
%
\[ K_i(z,y) = \int \chi_i(z,y,2^{-i} \eta) e^{2 \pi i [\phi(x,t,\eta) - y \cdot \eta]}\; d\eta \]
%
for some family of functions $\{ \chi_i \}$ supported on a common compact subset of $Z \times Y \times \Xi$, and satisfy estimates of the form $|\nabla^N \chi_i| \lesssim_N 1$, for all $N \geq 0$, uniformly in $i$. We can set
%
\[ \chi_i(z,y,\eta) = 2^{-i \mu} a(z,y, 2^i \eta) \chi(\eta), \]
%
since then $\chi_i$ is supported on $|\eta| \sim 1$ and by virtue of the fact $a$ is a symbol, we have the required estimates. By performing another decomposition, we may assume $\Xi$ is an arbitrarily small neighborhood of $e_1$, such that for $z \in Z$ and $\eta \in \Xi$,
%
\[ \nabla_z \phi(z,\eta) \approx e_1 \quad\text{and}\quad D_z \nabla_\eta \phi(z,\eta) \approx \begin{pmatrix} I_d \\ 0 \end{pmatrix} \]
%
and $H_\eta \{ \partial \phi / \partial t \}$ has rank at least $l$. In this section, we analyze each of these operators separately. If we write $T_i$ for the operator with kernel $K_i$, then here we will prove that
%
\[ \| T_i f \|_{L^p} \lesssim 2^{i \left( \frac{d-1}{2} - \frac{d}{q} \right)} \| f \|_{L^p}. \]
%
for $q > 2l/(l-2)$. In Seeger, Sogge, and Stein, it is proved that
%
\[ \| T_i f \|_{L^\infty} \lesssim 2^{i \frac{d-1}{2}} \| f \|_{L^\infty}. \]
%
By interpolation, it thus suffices to prove a restricted weak type inequality of the form
%
\[ \| T_i \chi_E \|_{L^{q_l,\infty}} \lesssim 2^{i(d/l - 1/2)} |E|^{1/q_l} \]
%
where
%
\[ q_l = 2 + 4/(l-2). \]
%
By duality, it suffices to show that for
%
\[ p_l = 2 - 4/(l+2), \]
%
we have
%
\[ \| T_i^* \chi_E \|_{L^{p_l,\infty}} \lesssim 2^{i(2d-l)/2l} |E|^{1/p_l}, \]
%
which is equivalent to prove that for $t > 0$, the measure of the set
%
\[ \{ y \in M : |T_i^* \chi_E(y)| \geq t \} \]
%
is bounded by $O(t^{-p_l} 2^{i(2d-l)/(l+2)} |E| )$. The operator $T_i^*$ has kernel
%
\[ K_i^*(y,z) = \overline{K_i(z,y)} = \int \chi_i(z,y, 2^{-i} \eta) e^{2 \pi i (y \cdot \eta - \phi(z,\eta))}\; dz \]
%
so we still have a Fourier integral operator, but with a reversed canonical relation. We will obtain these bounds by proving an analogous discretized result at a scale $1/2^i$.

We consider $\mathcal{Z}_i = 2^{-i} \ZZ^{d+1} \cap [-\varepsilon^2,\varepsilon^2]^{d+1}$, for some small constant $\varepsilon > 0$. For each $\mathfrak{z} \in \mathcal{Z}_i$ we consider a function $a_{i,\mathfrak{z}}$ supported on frequencies $|\eta| \sim 2^i$ which make an angle $O(\varepsilon^2)$ with $e_1$, so that
%
\[ |\partial_\eta^\alpha a_{i,\mathfrak{z}}(\eta)| \leq 2^{-i |\alpha|} \]
%
for $|\alpha| \lesssim 1$. Set
%
\[ S_{i,\mathfrak{z}}(y) = \int a_{i,\mathfrak{z}}(\eta) e^{2 \pi i (y \cdot \eta - \phi(\mathfrak{z},\eta))}\; d\eta. \]
%
Our job is to understand the sums $\sum S_{i,\mathfrak{z}}$.

\begin{lemma}
    For each $\mathcal{E} \subset \mathcal{Z}_i$, the measure of the set of $y$ such that
    %
    \[ | \sum_{\mathfrak{z} \in \mathfrak{E}} S_{i,\mathfrak{z}}(y) | \geq t \]
    %
    is
    %
    \[ \lesssim 2^{i \frac{dl - 2}{(l+2)}} t^{- p_l} \cdot \#(\mathcal{E}). \]
\end{lemma}

How does one reduce our problem to this setting? First, suppose we assume that
%
\[ \chi_i(z,y,2^{-i} \xi) = \eta_i(z, 2^{-i} \xi) \cdot \chi_\circ(y) \]
%
for some $\chi_0 \in C_c^\infty(\RR^d)$ supported on a small neighborhood of the origin, and where $\eta_i$ is supported on a set of diameter $O(\varepsilon^2)$ near $(z,\xi) = (0,e_1)$ and with uniformly bounded derivatives in $i$ up to a suitably high order. Then one may set
%
\[ a_{i,\mathfrak{z}}(\xi) = 2^{(i+1)d} \int_{Q_{\mathfrak{z}}} \eta_i(z, 2^{-i} \xi) e^{2 \pi i (\phi(\mathfrak{z},\xi) - \phi(z,\xi))}. \]
%
The phase function has derivatives $O(2^{-i})$, which gives the required results. To get a general result, we apply a Fourier series, writing
%
\[ \chi_i(z,y,2^{-i} \xi) = \sum_{\nu \in \ZZ^d} c_{k,\nu} \eta_{i,\nu}(z, 2^{-i} \xi) e^{2 \pi i y \cdot \eta} \]
%
where the constants, and the derivatives of $\eta_{i,\nu}$, are rapidly decaying in $\nu$.

\section{$L^1$ Estimates}

To understand the individual pieces $S_z$, we consider a maximal $2^{-i/2}$ separated set $\Theta$ covering the unit sphere, and perform a further decomposition
%
\[ a_z(\eta) = \sum_{\theta \in \Theta} a_{z,\theta}(\eta), \]
%
where $a_{z,\theta}$ is supported in a cone with aperture $O(2^{-i/2})$ centered at $\theta$. Then $a_{z,\theta}$ is roughly speaking, supported on a set with length $2^{i/2}$ tangent to the radial direction, and with length $2^i$ in the radial direction. Thus differentiating in the radial direction no longer leads to quite as good derivative estimates, namely, if $u_1,\dots,u_{M}$ are unit vectors tangent to $\theta$, and $M + N \lesssim 1$, then
%
\[ (\theta \cdot \nabla_\eta)^N \prod_{i = 1}^M (u_i \cdot \nabla_\eta) \{ a_{z,\theta} \} \lesssim 2^{-k N - k M / 2}. \]
%
The decomposition of $a_z$ of course leads to a decomposition $S_z = \sum S_{z,\theta}$.

Now because each component of $\nabla_\eta \phi$ is homogeneous of degree $0$, Euler's homogeneous function theorem says that
%
\[ H_\eta \phi(x,\eta) \cdot \eta = 0. \]
%
Integration by parts (TODO: How? Also is there a typo?) yields that
%
\[ |S_{z,\theta}(y)| \lesssim 2^{i \frac{d+1}{2}} \left( 1 + 2^i |(\nabla_\xi \phi(z,\theta) - y) \cdot \theta| + 2^{k/2} | \Pi_{\theta^\perp}( \nabla_\xi \phi(z,\theta) - y ) | \right)^{-O(1)}. \]
%\begin{align*}
%    S_{z,\theta} = \frac{1}{2 \pi i} \int \frac{a_{z,\theta}(\xi)}{|y - \nabla_\xi \phi(z,\xi)|^2} \left[ (y - \nabla_\xi \phi(z,\xi)) \cdot \nabla_\xi e^{2 \pi i (y \cdot \xi - \phi(z,\xi))} \right]\; d\xi\\
%    &= 
%\end{align*}
%
Roughly speaking, this inequality says that, roughly speaking, $S_{z,\theta}$ has magnitude $2^{i \frac{d+1}{2}}$, and is concentrated on a tube centered at $\nabla_\xi \phi(z,\theta)$, with thickness $2^{-i}$ in the radial direction, and thickness $2^{-i/2}$ in the tangential direction to $\theta$. In particular, we find that
%
\[ \| S_{z,\theta} \|_{L^1} \lesssim 1. \]
%
The triangle inequality (probably the best we can do in general in the $L^1$ setting) implies that
%
\[ \| S_z \|_{L^1} \lesssim 2^{i(d-1)/2}. \]
%
This is the bound we will use in $L^1$.

To get more interesting bounds in other $L^p$ spaces, we look at the orthogonality of the functions $\{ S_z \}$. On the Fourier side of things, we have
%
\[ \widehat{S_z}(\eta) = a_z(\eta) e^{- 2 \pi i \phi(z,\eta)}. \]
%
Thus by Parsevel, we have
%
\[ \langle S_z, S_w \rangle = \langle \widehat{S_z}, \widehat{S_w} \rangle = \int a_z(\eta) a_w(\eta) e^{2 \pi i [ \phi(z,\eta) - \phi(w,\eta) ]}. \]
%
TODO: Expand on rest of argument.

\section{Adapting the Argument to Fourier Multipliers}

Let $T = m(-\sqrt{\Delta})$ be a radial multiplier on $\RR^n$, i.e. such that
%
\[ T f(x) = \int m(|\xi|) e^{2 \pi i \xi \cdot (x - y)} f(y)\; d\xi\; dy. \]
%
If $m$ is a symbol, then we can interpret $T$ directly as a Pseudodifferential Operator. But Heo, Nazarov, and Seeger's result discuss families of multipliers $m$ that are not even necessarily smooth, but do satisfy certain integrability conditions. To fix this, we assume a priori that we have applyied a decomposition argument, so we may assume $m$ is compactly supported away from the origin. Then (by Paley-Wiener) $\widehat{m}$ is a smooth symbol of some finite order satisfying some integrability properties, which indicates how we might apply the theory of Fourier integral operators, i.e. by taking the Fourier transform of $m$, we get that
%
\[ Tf(x) = \int \widehat{m}(\rho) e^{2 \pi i [\rho |\xi| + \xi \cdot (x-y)]} f(y)\; d\rho\; d\xi\; dy. \]
%
This is `almost' a Fourier integral operator, except the phase is not smooth unless $\widehat{m}$ is supported away from the origin (fixed by a decomposition argument), and the phase is non-homogeneous. To fix the non-homogeneity, we just isolate the operator in $\rho$, writing
%
\[ Tf(x) = \int_{-\infty}^\infty \widehat{m}(\rho) T_\rho f(x)\; d\rho, \]
%
where
%
\[ T_\rho f(x) = e^{2 \pi i \rho \sqrt{-\Delta}} f(x) = \int e^{2 \pi i [\rho |\xi| + \xi \cdot (x - y)]} f(y)\; d\xi\; dy \]
%
is the propogation operator for the half-wave equation $\partial_t u = \sqrt{-\Delta} \cdot u$. It has phase $\phi(x,y,\xi) = \rho |\xi| + \xi \cdot (x - y)$, and thus we have a stationary frequency value when $x = y - \rho \widehat{\xi}$, where $\widehat{\xi} = \xi / |\xi|$ is the normalization of $\xi$. This has canonical relation




\chapter{Relations to Local Smoothing}

Let us now try and prove certain special cases of the radial multiplier conjecture on the sphere $S^n$. Thus we fix a symbol $h$, and study operators of the form
%
\[ T_R = h \left( \sqrt{-\Delta} / R \right) = \sum h(\lambda / R) E_\lambda, \]
%
where $E_\lambda$ is the projection operator onto the eigenspace corresponding to the eigenvalue $\lambda$. In particular, we wish to characterize the boundedness properties of the operators $T_{h,R}$, in terms of appropriate control of the Fourier transform of the function $h$. For simplicity, let us assume that $\text{supp}(h)$ is contained in $1/2 \leq \lambda \leq 2$, with the hope that things will generalize to non compactly supported values using the appropriate dyadic decomposition technology. For an exponent $1 \leq p < 2d/(d+1)$, we then assume that the quantity
%
\[ A_p(h) = \left( \int_0^\infty (1 + |s|)^{(d-1)(1 - p/2)} |\widehat{h}(s)|^p\; ds \right)^{1/p} \]
%
is finite, which is a necessary condition for the multiplier $h(\sqrt{-\Delta})$ to be bounded on $L^p(\RR^d)$ or $L^{p^*}(\RR^d)$, and thus by a transference principle of Mitjagin, necessary for the family of operators $\{ T_R : R > 0 \}$ to be uniformly bounded in $R$ on $L^p(S^n)$ or $L^{p^*}(S^n)$. The triangle inequality gives uniform boundedness for $R \leq 1$, so we may assume in what follows that $R \geq 1$.

Our goal is to show that, uniformly in $R$, we have
%
\[ \| T_R f \|_{L^p} \lesssim \| f \|_{L^p}. \]
%
Since $T_R$ is a multiplier with symbol supported on $R/2 \leq \lambda \leq 2R$, we may assume that $f$ is also supported on this frequency range, i.e. is in the span of eigenfunctions to $\sqrt{-\Delta}$ with eigenvalue $R/2 \leq \lambda \leq 2R$. In particular, this implies that we have derivative estimates of the form
%
\[ \| f \|_{L^p_\alpha} \lesssim_\alpha (1 + R)^\alpha \| f \|_{L^p}, \]
%
for any $\alpha \geq 0$, i.e. a kind of Bernstein's inequality.

\section{Attempt \# 1: Successful?}

To exploit the fact that $A_p(h)$ is finite, we apply the Fourier transform to the sum defining $T_R$, writing $T_R f$ as the vector valued integral
%
\[ T_R f = \int_{-\infty}^\infty R \widehat{h}(R t) e^{2 \pi i t \sqrt{-\Delta}} f\; dt. \]
%
where $\{ e^{2 \pi i t \sqrt{-\Delta}} \}$ are the solution operators to the half wave equation
%
\[ \frac{\partial}{\partial t} = 2 \pi i \sqrt{-\Delta}. \]
%
Using the periodicity of the wave equation, we write
%
\[ T_R f = \int_{-1/2}^{1/2} \left( \sum_{l = -\infty}^\infty R \cdot \widehat{h}(R(t + l)) \right) e^{2 \pi i t \sqrt{-\Delta}} f\; dt \]
%
Applying H\"{o}lder's inequality, we have
%
\begin{align*}
    \left| \sum_{l = -\infty}^\infty \widehat{h}(R(t + l)) \right| &\lesssim \left( \sum_{l = -\infty}^\infty |\widehat{h}(R(t + l))|^p \langle R(t + l) \rangle^{(d-1)(1 - p/2)} \right)^{1/p}\\
    &\quad\quad\quad \left( \sum_{l = -\infty}^\infty \langle R(t + l) \rangle^{-(d-1)(p^*/2 - 1)} \right)^{1/p^*}.
\end{align*}
%
If $d(t,\mathbf{Z}) = s$ for some $s \in [0,1/2]$, then since $1 < p < 2d/(d+1)$, we have
%
\begin{align*}
    &\left( \sum_{l = -\infty}^\infty \langle R(t + l) \rangle^{-(d-1)(p^*/2 - 1)} \right)^{1/p^*} \sim \langle Rt \rangle^{-(d-1)(1/2 - 1/p^*)}.
%    &\quad \quad \sim \begin{cases} 1 &: t \leq 1/R, \\ (t/R) R^{-(d-1)(1/2 - 1/p^*)} t^{-(d-1)(1/2 - 1/p^*)} &: 1/R \leq t \leq 1/2. \end{cases}
\end{align*}
%
Write
%
\[ T_R f = \sum_{k = 0}^{O(\log R)} T_{R,k} f, \]
%
where
%
\[ T_{R,0} f = \int_{|t| \leq 1/R} \left( \sum_{l = -\infty}^\infty R \cdot \widehat{h}(R(t + l)) \right) e^{2 \pi i t \sqrt{-\Delta}} f\; dt \]
%
and
%
\[ T_{R,k} f = \int_{|t| \sim 2^k/R} \left( \sum_{l = -\infty}^\infty R \cdot \widehat{h}(R(t + l)) \right) e^{2 \pi i t \sqrt{-\Delta}} f\; dt. \]
%
H\"{o}lder's inequality applied to the inner sums implies that
%
\begin{align*}
    |T_{R,0} f| \leq R \int_{|t| \leq 1/R} \left( \sum_{l = -\infty}^\infty |\widehat{h}(R(t + l))|^p \langle R(t + l) \rangle^{(d-1)(1 - p/2)} \right)^{1/p} |e^{2 \pi i t \sqrt{-\Delta}} f|\ dt.
\end{align*}
%
Another application of H\"{o}lder's inequality, this time to the integral, gives us that
%
\[ |T_{R,0} f| \lesssim A_p(h) R^{1/p^*} \left( \int_{|t| \leq 1/R} |e^{2 \pi i t \sqrt{-\Delta}} f|^{p^*}\; dt \right)^{1/p^*} \]
%
A similar analysis gives that
%
\[ |T_{R,k} f| \lesssim A_p(h) R^{1/p^*} 2^{-k(d-1)(1/2 - 1/p^*)} \left( \int_{|t| \sim 2^k/R} |e^{2 \pi i t \sqrt{-\Delta}} f|^{p^*}\; dt \right)^{1/p^*}. \]
%
Let us analyze $T_{R,k}$ first. Suppose the endpoint local smoothing conjecture held at the particular value $p^*$ we were consider, at all scales, i.e. so that
%
\[ \left\| \left( \int_{|t| \sim 2^k / R} |e^{2 \pi i t \sqrt{-\Delta}} f|^{p^*} \right)^{1/p^*} \right\|_{L^{p^*}(M)} \lesssim (2^k/R)^{1/p^*} \| f \|_{L^{p^*}_{\alpha_{p^*}}(M)} \]
%
where $\alpha_{p^*} = (d-1)(1/2 - 1/p^*) - 1/p^*$. Then, using the fact that
%
\[ \| f \|_{L^{p^*}_{\alpha_{p^*}}} \lesssim R^{\alpha_{p^*}} \| f \|_{L^{p^*}(M)}, \]
%
we conclude that
% d/p^* - (d-1)/2 < 0
%
\begin{align*}
    \| T_{R,k} f \|_{L^{p^*}(M)} &\lesssim A_p(h) R^{1/p^*} 2^{-k(d-1)(1/2 - 1/p^*)} (2^k / R)^{1/p^*} R^{\alpha_{p^*}} \| f \|_{L^{p^*}(M)}\\
    &\lesssim A_p(h) R^{\alpha_{p^*}} 2^{-k((d-1)/2 - d/p^*)} \| f \|_{L^{p^*}(M)}.
%    \| T_{R,k} f \|_{L^{p^*}(M)} &\lesssim A_p(h) R^{1/p^* - (d-1)(1/2 - 1/p^*)} 2^{-k(d-1)(1/2 - 1/p^*)} (2^k / R)^{1/p^*} R^{\alpha_{p^*}} \| f \|_{L^{p^*}(M)}\\
%    &\lesssim A_p(h) 2^{k[d/p^* - (d-1)/2]} R^{-1/p^*} \| f \|_{L^{p^*}(M)}.
\end{align*}
%
% R^a 2^{-ak} <= O(1)
% a log R - ak log 2 <= O(1)
% k >= log_2 R - O(1/a)
% 2^k / R >= 1 / 100000
This bound is uniform in $R$, and summable in $k$, provided that
%
\[ k \geq \log_2(R) - O(1). \]
%
Thus we are left to analyze $k \leq \log_2(R) - O(1)$, i.e. we are left to studying the wave equation over very small times, i.e. for $|t| \lesssim 1/100$. On the other hand, if we \emph{increase} the exponent in the integrability condition $A_p(h)$ by a $\varepsilon$, i.e. assuming the quantity
%
\[ A_{p,\varepsilon}(h) = \left( \int |\widehat{h}(t)|^p (1 + |t|)^{(d-1)(1 - p/2) + \varepsilon} \right)^{1/p} \]
%
is finite, then the strategy above leads us with only needing to study the wave equation over a very very small set of times, i.e. $|t| \lesssim R^{-O(\varepsilon)}$.

Let's explore the analysis over these very very small set of times. Fix $\varepsilon > 0$, let $\phi \in C_c^\infty((0,\infty))$ and equal to one for $|t| \lesssim 1$, and consider an operator of the form
%
\[ T_R f = \int \phi( R^\varepsilon t) \left( \sum_{l = -\infty}^\infty R \widehat{h}(R(t + l)) \right) e^{2 \pi i t \sqrt{-\Delta}} f. \]
%
Let $a_R$ denote the inverse Fourier transform of $t \mapsto \phi(R^\varepsilon t) ( \sum_l R \widehat{h}(R(t + l)) )$. Then
% u . h_R
\[ a_R(\lambda) = R^{-\varepsilon} \sum_{\omega \in \ZZ} h \left( \frac{\lambda - \omega}{R} \right) \widehat{\phi}(\omega / R^\varepsilon) \]
%
In particular, for $N \geq 0$,
%
\[ D^N a_R(\lambda) = R^{-(N+1) \varepsilon} \sum_{\omega \in \ZZ} h \left( \frac{\lambda - \omega}{R} \right) D^N \widehat{\phi}(\omega / R^\varepsilon). \]
%
Thus we conclude that
%
\[ |D^N a_R(\lambda)| \lesssim_N R^{-N \varepsilon} \| h \|_{L^\infty} \]
%
In particular, if $|\lambda| \sim R$, then $|D^N a_R(\lambda)| \lesssim_N |\lambda|^{-N \varepsilon}$. In particular, if $\psi \in C_c^\infty(\RR)$ is supported on the annulus $|\lambda| \sim 1$, then the functions $\tilde{a}_R(\lambda) = \psi(\lambda / R) a_R(\lambda)$ uniformly lie in some symbol class $\mathcal{S}^0_\varepsilon(\RR)$, i.e. nonstandard symbols of order zero. We also have
%
\[ |D^N a_R(\lambda)| \lesssim_N R^{-\varepsilon - N\varepsilon} \left( \sup_{R'} \| \text{Dil}_{R'} h \|_{l^1(\ZZ)} \right), \]
%
which implies that under the assumption that the supremum in the inequality above is finite, then $\tilde{a_R}$ lies unformly in the symbol class $\mathcal{S}^{-\varepsilon}_\varepsilon(\RR)$. In the Euclidean setting, a bounded family of operators in $\mathcal{S}^m_\varepsilon(\RR)$ will be uniformly bounded in $L^p(\RR^d)$ for $m = - (1 - \varepsilon) d / 2$, so we should expect the condition is only sufficient if $\varepsilon \geq 1 - 2 / (d+2)$.

% R^a 2^{-(a + e) k } <= O(1)
% a log R - (a + e) k log 2 <= O(1)
% a log R - O(1) <= (a + e) k log 2
% k >= (1 - e/(a+e)) log_2 R - O(1)
% 2^k >= R^{}

\begin{comment}
%
Summing up gives
%
\[ \| T_{R,k} f \|_{L^{p^*}(M)} \lesssim R^{-1/p^*} \| f \|_{L^{p^*}(M)}. \]

Summing up this bound gives uniform bounds in $R \geq 1$. On the other hand, the ednpoint local smoothing conjecture also implies that
%
\[ \left\| \left( \int_{|t| \lesssim 1 / R} |e^{2 \pi i t \sqrt{-\Delta}} f|^{p^*} \right)^{1/p^*} \right\|_{L^{p^*}(M)} \lesssim R^{-1/p^*} \| f \|_{L^{p^*}_{\alpha_{p^*}}(M)}, \]
%
and so
%
\[ \| T_{R,0} f \|_{L^{p^*}(M)} \lesssim A_p(h) R^{\alpha_{p^*}} \| f \|_{L^{p^*}_{\alpha_{p^*}}(M)}. \]
%
This is problematic. To get the result required, we'd really like to have a `local local smoothing' estimate of the form
%
\[ \left\| \left( \int_{|t| \lesssim 1/R} |e^{2 \pi i t \sqrt{-\Delta}} f|^{p^*} \right)^{1/p^*} \right\|_{L^{p^*}(M)} \lesssim \| f \|_{L^{p^*}(M)} \]
%
as $R \to 0$.
\end{comment}

\section{Attempt \# 2: Unsuccessful?}

We break up $T_R f = \sum_{k = 0}^\infty T_{R,k} f$, where
%
\[ T_{R,0} f = \int_{-\infty}^\infty R \widehat{h}(Rt) \rho_0(Rt) e^{2 \pi i t \sqrt{-\Delta}} f\; dt \]
%
and for $k \geq 1$,
%
\[ T_{R,k} f = \int_{-\infty}^\infty R \widehat{h}(Rt) \rho(Rt / 2^k) e^{2 \pi i t \sqrt{-\Delta}} f\; dt. \]
%
Our goal is to obtain some exponential decay in $k$ so that we can use the triangle inequality to get uniform bounds on the whole operator.

First, let's deal with the $k = 0$ case. Let $m$ be the inverse Fourier transform of $\widehat{h} \cdot \rho_0$ then $T_{R,0} = m \left( \sqrt{-\Delta} / R \right)$. The function $m$ is smooth, i.e. it satisfies bounds of the form
%
\[ |\nabla^N m(\lambda)| \lesssim_{N,M} A_p(h) \langle \lambda \rangle^{-M} \]
%
for all $N$ and $M$. It follows by a manifold version of the H\"{o}rmander-Mikhlin theorem \cite{SeegerSogge} that $\| T_{R,0} f \|_{L^p(M)} \lesssim \| f \|_{L^p(M)}$ for all $1 < p < \infty$, uniformly in $R$, so this part of the operator is relatively trivial to analyze.

Now we deal with the $k > 0$ case. H\"{o}lder's inequality, combined with the trick of first multiplying and dividing by $(1 + |Rt|)^{(d-1)(1/p - 1/2)}$ implies that
% -(d-1)(p^*/2 - 1)
\begin{align*}
    |T_{R,k} f| &= \left| \int_{-\infty}^\infty R \widehat{h}(Rt) \rho(Rt / 2^k) e^{2 \pi i t \sqrt{-\Delta}} f\; dt \right|\\
    &\leq R \left( \int_{|t| \sim 2^k / R} \left( |\widehat{h}(Rt)| (1 + |Rt|)^{(d-1)(1/p - 1/2)} \right)^p\; dt \right)^{1/p}\\
    &\quad\quad\quad\left( \int_{|t| \sim 2^k / R} \left( |e^{2 \pi i t \sqrt{-\Delta}} f| (1 + |Rt|)^{-(d-1)(1/p - 1/2)} \right)^{p^*}\; dt \right)^{1/p^*}\\
    &\lesssim R^{1 - 1/p} 2^{-k(d-1)(1/2 - 1/p^*)} A_{p,k}(h) \left( \int_{|t| \sim 2^k / R} |e^{2 \pi i t \sqrt{-\Delta}} f|^{p^*}\; dt \right)^{1/p^*}.
\end{align*}
% -(d-1)(p^*/2 - p^* + 1))
% R^{1 - 1/p} 2^{-k(d-1)(p^* / 2 - 1)}
%
where
%
\[ A_{p,k}(h) = \left( \int_{|t| \sim 2^k} |\widehat{h}(t)|^p \cdot 2^{k(d-1)(1 - p/2)}\; dt \right)^{1/p} \]
%
Applying the periodicity of the wave equation, for $2^k \geq R$ we have
%
\[ \left( \int_{|t| \sim 2^k/R} |e^{2 \pi i t \sqrt{-\Delta}} f|^{p^*} \right)^{1/p^*} \lesssim (2^k/R)^{1/p^*} \left( \int_{|t| \lesssim 1} |e^{2 \pi i t \sqrt{-\Delta}} f|^{p^*} \right)^{1/p^*}. \]
%
If the endpoint local smoothing conjecture held, then we would have
%
\[ \left( \int_{|t| \lesssim 1} |e^{2 \pi i t \sqrt{-\Delta}} f|^{p^*} \right)^{1/p^*} \lesssim \| f \|_{L^{p^*}_{\alpha_p}}, \]
%
where
%
\[ \alpha_p = d(1/2 - 1/p) - 1/2. \]
%
But applying a Sobolev embedding, since $f$ is a sum of eigenfunctions of the Laplace-Beltrami operator with eigenvalue $\sim R$, we conclude that
%
\[ \| f \|_{L^{p^*}_{\alpha_p}} \lesssim R^{\alpha_p} \| f \|_{L^p} \]
%
Thus we conclude that
%
\begin{align*}
    \| T_{R,k} f \|_{L^p} &\lesssim R^{1 - 1/p} 2^{-k(d-1)(1/2 - 1/p^*)} (2^k/R)^{1/p^*} R^{d(1/p - 1/2) - 1/2} A_{p,k}(h) \| f \|_{L^p}\\
    &\lesssim R^{d(1/p - 1/2) - 1/2} 2^{k[d/p^* -(d-1)/2]} A_{p,k}(h) \| f \|_{L^p}.
\end{align*}
%
Provided that $p < 2d/(d+1)$, this bound is summable in $k$. And provided that $p \geq 2d/(d+1)$, the bound is uniform in $R \geq 1$. This indicates that we're dealing with is `precisely' at the endpoint in some sense.

In particular, if we do these calculations replacing $A_p(h)$ with $A_{p,\varepsilon}(h)$, where
%
\[ A_{p,\varepsilon}(h) = \left( \int_{-\infty}^\infty \left( |\widehat{h}(s)| (1 + |s|)^{(d-1)(1/p - 1/2) + \varepsilon} \right)^p \right)^{1/p}, \]
%
then for $2d/(d+1) \leq p < 2d/(d+1) + O(\varepsilon)$, we get uniform boundedness in $L^p$ as we vary $R$.
%then for $2d/(d+1) \leq p < 2d/(d + 1 - 2 \varepsilon) =$, we get uniform boundedness on $L^p$.

% 2d/(d+1-2e) - 2d/(d+1) = 2d(d+1) - 2d(d+1 - 2epsilon) / (d+1)(d + 1 - 2epsilon)

% 4d/(d+1)^2 epsilon
% 4/d

\begin{comment}

If we choose a parameter $l$ such that
%
\[ 1/p^* - \alpha_p / d = 1/p - l/d \]
%
then 

% 1/p^* - alpha/d = 1/p
% 1 - alpha/d = 2/p
% d(1 - 2/p) = alpha
% d(1 - 2/p) >= d(1/2 - 1/p) - 1/2
% 1/2 >= d(1/p - 1/2)
% 1/p <= (1 + d)/2d
% p >= 2d/(d+1) = 2 - 2/(d+1).

Since $f$ is the span of eigenfunctions with eigenvalue $\lesssim R$, we have
%
\[ \| f \|{L^{p^*}_{\alpha_p}} \lesssim R^{\alpha_p} \| f \|_{L^{p^*}}. \]
%
Thus we conclude that
%
\[ \left( \int_{|t| \lesssim 1} |e^{2 \pi i t \sqrt{-\Delta}} f|^{p^*} \right)^{1/p^*} \lesssim R^{\alpha_p} \| f \|_{L^{p^*}}. \]
%
Putting all these bounds together yields that
%
\[ \| h(\sqrt{-\Delta} / R) f \|_{L^p} \lesssim R^{1 - 1/p} 2^{-k (d-1) (1/2 - 1/p^*)} A_p(h) (2^k / R)^{1/p^*} R^{\alpha_p} \| f \|_{L^{p^*}} \]




%
\[ \left( (1 + |Rt|)^{- (d-1)\frac{(2 - p)}{2(p-1)}} \right)^{1/p^*} \sim R^{-(d-1) \frac{(2-p)(1 - 1/p)}{2(p-1)}} |t|^{-(d-1) \frac{2-p}{2(p-1)}} \left( |Rt|^{- (d-1)\frac{(2 - p)}{2(p-1)}} \right)^{1/p^*}. \]
%
Applying the periodicity of the wave equation, we find that
%
\begin{align*}
    &\left( \int_{-\infty}^\infty |e^{2 \pi i t \sqrt{-\Delta}} f|^{p^*} (1 + |Rt|)^{- (d-1)\frac{(2 - p)}{2(p-1)}} \right)^{1/p^*}\\
    &\quad\quad s
\end{align*}

% the periodicity of the wave operators $\{ e^{2 \pi i t \sqrt{-\Delta}} \}$, and the fact that $1 < p < 2d/(d+1)$ allow us to obtain
%
\[  \]
%
\begin{align*}
    &\left| \int_{-\infty}^\infty R \widehat{h}(Rt) e^{2 \pi i t \sqrt{-\Delta}} f\; dt \right|\\
    &\quad\quad\lesssim R^{1/2 + d(1/2 - 1/p)}
\end{align*}
% R^{-(d-1)(2 - p)/2(p-1)} provided that p < 2d/(d+1) = 2/(1 + 1/d)
% \[ R^{1/2 + d(1/2 - 1/p)} \]


\emph{If} we are able to guarantee that $\widehat{h}(t) = 0$ for $t \lesssim 1$,

%

\[ R^{-(d-1)(2-p)/2(p-1)} \sum_{m = 1}^\infty |Rk|^{-(d-1)(2-p)/2(p-1)} \]
%
\end{comment}

\begin{comment}

The small time parameterix for the half-wave operator, combined with the composition calculus of Fourier integral operators, allows us to write, for $|t| \leq 1$,
%
\[ e^{2 \pi i t \sqrt{-\Delta}} f = T_t f + S^\infty_t f, \]
%
where $S^\infty_t$ is a \emph{smoothing operator}, i.e. an integral operator with
%
\[ S^\infty_t f(x) = \int K(t,x,y) f(y)\; dy, \]
%
where $K \in C^\infty([-1,1] \times S^n \times S^n)$, and where we can locally write
%
\[ T_t f(x) = \int_{\RR^n} a(t,x,y,\xi) e^{2 \pi i \Phi(x,y,\xi)} f(y)\; d\xi\; dy \]
% p(x,\xi) = \sqrt{-\Delta}
for some symbol $a \in S^0$, and some symbol $\Phi \in S^1$ satisfying
%
\[ \Phi(x,y,t,\xi) = (x - y) \cdot \xi + t g_y(\xi,\xi) + O(|x - y|^2 |\xi|). \]
%
To calculate the canonical relation of this operator, we look at the principal symbol of the operator $\sqrt{-\Delta}$. If $g$ is the metric of $S^n$, then the principal symbol will be
%
\[ p(x,\xi) = C \left( \sum g_{ij}(x) \xi^i \xi^j \right)^{1/2} = C |\xi| \]
%
for an appropriate constant $C$ (TODO: Do this calculation more precisely). One can calculate (see Remark in Section 4.1 Sogge's book) that the canonical relation of the family of operators $\{ T_t \}$ is given by
%
\[ \mathcal{C} \subset \{ (x,t,\xi,\tau,y,\eta) : (y,\eta) = \phi_t(x,\xi), \tau = p(x,\xi) \}, \]
%
where $t \mapsto \phi_t(x,\xi)$ is the geodesic travelling at a velocity of $2\pi$ which starts at $x$, and travels in the direction given by $\xi$. We claim this canonical relation satisfies the cinematic curvature condition. Indeed, the projections $\Pi_{y,\eta}: \mathcal{C} \to T^* S^n$ and $\Pi_{x,t}: \mathcal{C} \to T^* S^n$ are both submersions, so the Fourier integral operator is nondegenerate. For each pair $(x_0,t_0)$, the cone
%
\[ \mathcal{C}_{x_0,t_0} = \{ (\xi,\tau) : |\xi| = C^{-1} \tau \} \]
%
is an $n$ dimensional hypersurface in $T_{x_0,t_0}^*((-1,1) \times S^n)$, and it is easy to see this hypersurface is curved for all $t$. The endpoint local smoothing conjecture claims that if $f \in L^{p^*}(S^{n-1})$ for precisely the range of $p$ we care about in the radial multiplier conjecture, then $T f \in L^{p^*}_{-\alpha_p}((-1,1) \times S^{n-1})$, where $\alpha_p = n(1/2 - 1/p^*) - 1/2 = n(1/p - 1/2) - 1/2$. In particular, if we assume that Sobolev norms work on $S^n$ the same way they work on $\RR^n$, this means that if $f \in L^{p^*}(S^{n-1})$ has frequency supported on $|\xi| \leq L$, then
%
\[ \| T_R f \|_{L^{p^*}(\RR^d)} \lesssim \| f \|_{L^{p^*}(\RR^d)_{\alpha_p}} \lesssim L^{\alpha_p} \| f \|_{L^{p^*}(\RR^d)}. \]
%
Thus we see that local smoothing is pretty hopeless in proving the result we need to prove for general $f$.

The only non optimal inequality we applied here was H\"{o}lder's inequality, which would be tight if there exists a non-negative function $\gamma$ such that
%
\[ |e^{2 \pi i t \sqrt{-\Delta}} f|^{p^*} (1 + |Rt|)^{-(d-1)\frac{(2-p)}{2(p-1)}} = \gamma(x)^{p^*} |\widehat{h}(Rt)|^p (1 + |Rt|)^{(d-1)(1-p/2)}, \]
%
i.e. where
%
\[ |e^{2 \pi i t \sqrt{-\Delta}} f(x)| = \gamma(x) (1 + |Rt|)^{(d-1) \frac{(2-p)}{2}} |\widehat{h}(Rt)|^{1/p^*}. \]

TODO: Think about why local smoothing is useless. Is the theorem trivial if H\"{o}lder's inequality is applied?

%where $g$ is the Riemmanian metric of $S^n$. In particular, if we work with the coordinates $z_{\pm}$ in the strict upper and lower hemispheres given by
%
%\[ z_{\pm}^{-1}(t_1,\dots,t_n) = (t_1,\dots,t_n, \pm (1-t_1^2 - \dots - t_n^2)^{1/2}), \]
%
%then in coordinates we have
%
%\[ g_y(\xi,\xi) = |\xi|^2 - (1 - |y|^2)^{-1/2} |y \cdot \xi|^2, \]
%
%and so in these coordinates we have the explicit form
%
%\[ \Phi(x,y,s,\xi) = (x - y) \cdot \xi + t |\xi|^2 - t (1 - |y|^2)^{-1/2} |y \cdot \xi|^2 + O(|x - y|^2 |\xi|). \]
%
%TODO: IS THIS EXPLICIT FORM USEFUL?

\section{Junk}

Rescaling and applying H\"{o}lder's inequality, we have
%
\begin{align*}
    R \int_0^{2\pi} & \int_{\RR^d} \sum_{k = 0}^\infty w(R s + (2 \pi k) R) a(s,x,y,\xi) e^{2 \pi i \Phi(x,y,s/R,\xi)}\; d\xi\; ds\\
    &= \sum_{k = 0}^\infty \int_0^{2 \pi R} w(s + (2 \pi k) R) \int_{\RR^d} a(s/R,x,y,\xi) e^{2\ pi i \Phi(x,y,s/R,\xi)}\; d\xi\; ds\\
    &\leq \sum_{k = 0}^\infty \left( \int_0^{2\pi R} |w(s + (2 \pi k) R)|^q\; ds \right)^{1/q} \left( s \right)
\end{align*}


Now suppose that $\| w \|_{L^q(\RR^d, (1 + |x|)^{(d-1)(1 - q/2)})} < \infty$
%
\[ \int_0^{2\pi} \int_{\RR^d} \sum_{k = 0}^\infty b_t(R s + (2 \pi k) R) a(s,x,y,\xi) e^{2 \pi i \Phi(x,y,\xi)}\; d\xi\; ds, \]

Let us begin with the qualitative assumption that $h$ is compactly supported. Then, by breaking things up into a finite sum, we may assume that $h$ is supported on $[1/2,2]$. Fix a function $\chi \in C_c^\infty(\RR)$ equal to one on $[1/2,2]$, and vanishing outside of $[1/4,4]$. Write
%
\[ P_R = \chi \left( \sqrt{-\Delta} / R \right) = \sum \chi(\lambda / R) E_\lambda. \]
%
Then for any function $f \in C^\infty(S^n)$, $T_R f = T_R \{ P_R f \}$. Thus when bounding the behaviour of the operator $T_R$, we may assume inputs are linear combinations of eigenfunctions to $\sqrt{-\Delta}$ with eigenvalues $\lambda \sim R$.

\end{comment}





\part{Papers I Don't Understand Yet}





\chapter{Seeger: Singular Convolution Operators in $L^p$ Spaces}

Let $m: \RR^d \to \CC$ be the symbol for a Fourier multiplier operator $m(D)$. If the resulting operator $m(D)$ was bounded from $L^p(\RR^d)$ to $L^p(\RR^d)$ with operator norm $A$, then the operator would also be bounded `at all scales'. That is, if we consider a littlewood Paley decomposition, i.e. taking
%
\[ f = \sum_{i = 0}^\infty f_i \]
%
where $\widehat{f_i} = \eta_i \widehat{f}$ is supported on $2^i \leq |\xi| \leq 2^{i+1}$ for $i \geq 1$, and $|\xi| \leq 2$ for $i = 0$, then we would have estimates of the form
%
\begin{equation} \label{piecewiseBound}
    \| m(D) f_i \|_{L^p(\RR^d)} \lesssim \| f_i \|_{L^p(\RR^d)} \lesssim \| f \|_{L^p(\RR^d)},
\end{equation}
%
where the implicit constant is uniform in $i$. The main focus of the paper in question is to determine whether a uniform bound of the form \eqref{piecewiseBound} implies $m(D)$ is bounded. More precisely, is it true that
%
\begin{equation} \label{operatorbound}
    \| m \|_{M^p(\RR^d)} \lesssim_p \sup\nolimits_{i \geq 0} \| m_i \|_{M^p(\RR^d)},
\end{equation}
%
where $m_i = \eta_i m$.

The Hilbert transform $H$ is a Fourier multiplier with symbol $m(\xi) = \text{sgn}(\xi)$. For each $i > 0$, $m_i(\xi) = \eta_i \text{sgn}(\xi)$, so that
%
\[ K_i(x) = \widehat{\eta_i \text{sgn}(\xi)} = 2^i H \eta(2^i x). \]
%
Thus
%
\[ \| K_i \|_{L^1(\RR)} = \| H \eta \|_{L^1(\RR)}. \]
%
TODO

It is clear that \eqref{operatorbound} is true for $p = 2$, since in this case the bound is equivalent to an inequality of the form
%
\[ \| m \|_{L^\infty(\RR^d)} \lesssim \sup\nolimits_{i \geq 0} \| m_i \|_{L^\infty(\RR^d)}, \]
%
which is true because the supports of the symbols $\{ m_i \}$ are almost all pairwise disjoint. On the other hand, \eqref{operatorbound} does not hold when $p = 1$ or $p = \infty$, which makes sense, since Littlewood-Paley runs into all kinds of problems for these values of $p$. Arguing more precisely, the condition would be equivalent to showing that for any $K: \RR^d \to \CC$,
%
\[ \| K \|_{L^1(\RR^d)} \lesssim \sup\nolimits_{i \geq 0} \| K * \widehat{\eta_i} \|_{L^1(\RR^d)}. \]
%
If
%
\[ K_N(x) = \int_{|\xi| \leq 2^N} e^{2 \pi i \xi \cdot x}\; d\xi \]
%
is the Dirichlet kernel, then $\| K_N \|_{L^1(\RR)} \sim N$. On the other hand, for $i \leq N-1$, we have $K_N * \widehat{\eta_i} = \widehat{\eta_i}$, so that
%
\[ \| K_N * \widehat{\eta_i} \|_{L^1(\RR)} = \| \widehat{\eta_i} \|_{L^1(\RR)} \lesssim 1. \]
%
For $i \geq N+1$, we have $K_N * \widehat{\eta_i} = 0$, so that
%
\[ \| K_N * \widehat{\eta_i} \|_{L^1(\RR)} = 0 \lesssim 1. \]
%
For $i = N$, we have
%
\[ (K_N * \widehat{\eta_N})(x) = 2^N \int_0^1 \eta(\xi) e^{2 \pi i 2^N (\xi \cdot x)} + \int_1^2 \eta(-\xi) e^{-2 \pi i 2^N (\xi \cdot x)}\; d\xi  \]
%
\[ \int |K_N * \widehat{\eta_i}| \]

whereas one
% 
%
\[ K_N * \widehat{\eta_i} = \begin{cases} \widehat{\eta_i} &: i \lesssim N \\ 0 &: i \gtrsim N \end{cases}, \]
%
and so $\| K_N * \widehat{\eta_i} \|_{L^1(\RR)} \lesssim 1$ uniformly in $N$ and $i$. We can then use Baire category techniques to find a kernel $K$ not in $L^1(\RR)$, but such that $\| K * \eta_i \|_{L^1(\RR)} \lesssim 1$, uniformly in $i$.

The result actually fails for $2 < p < \infty$, due to an examples of Triebel. For simplicity, let's work in $\RR$. If we fix a bump function $\phi \in C_c^\infty(\RR)$ supported in $[-1,1]$, and set
%
\[ m_N(\xi) = \sum_{k = N}^{2N} e^{2 \pi i (2^k \xi)} \phi(\xi - 2^k), \]
%
then $m_N(\xi) \eta_i(\xi) = m_{N,i}(\xi)$, where $m_{N,i}(\xi) = e^{2 \pi i (2^k \xi)} \phi(\xi - 2^k)$, and so $K_{N,i}(x) = \widehat{m_{N,i}}(x) = e^{2 \pi i 2^k(x - 2^k)} \widehat{\phi}(x - 2^k)$, hence
%
\[ \| m_{N,i}(D) f \|_{L^p(\RR^d)} = \| K_{N,i} * f \|_{L^p(\RR^d)} \leq \| \widehat{\phi} \|_{L^1(\RR)} \| f \|_{L^p(\RR)} \lesssim \| f \|_{L^p(\RR)}. \]
%
On the other hand, the operator norm of $m_N(D)$ from $L^p(\RR)$ to $L^p(\RR)$ is actually $\gtrsim_p N^{|1/p - 1/2|}$, and thus not bounded uniformly in $N$, so Baire category shows things don't work so well here.

This paper shows that one \emph{can} get uniform bounds assuming an additional, very weak smoothness condition, which rules out the example $m_N$ above. Under the most simple assumptions, if \eqref{piecewiseBound} holds, and $\| m_i \|_{\Lambda^\varepsilon} \lesssim 2^{-ik}$, where $\Lambda^\varepsilon$ is the $\varepsilon$-Lipschitz norm, then $\| m(D) f \|_{L^r(\RR^d)} \lesssim \| f \|_{L^r(\RR^d)}$ whenever $|1/r - 1/2| < |1/p - 1/2|$. Under slightly stronger smoothness assumptions, we can actually conclude $\| m(D) f \|_{L^p(\RR^d)} \lesssim \| f \|_{L^p(\RR^d)}$.

To prove the result, we rely on Littlewood-Paley theory and the Fefferman-Stein sharp maximal function. Without loss of generality we may assume that $2 < p < \infty$. We will actually show that if for all $i$ and $\omega \geq 0$,
%
\[ \int_{|x| \geq \omega} |K_i(x)|\; dx \leq B (1 + 2^i \omega)^{-\varepsilon}, \]
%
consistent with the fact that, if $m_i$ was smooth, the uncertainty principle would say that $K_i$ would live on a ball of radius $1/2^i$. We will then prove that $\| m(D) f \|_{L^p(\RR^d)} \leq A \widetilde{\log}(B/A)^{|1/2 - 1/p|}$. Our goal is to show that if
%
\[ S^\# f(x) = \sup_{x \in Q} \fint_Q \left( \sum_{i = 0}^\infty \left| m_i(D) f(y) - \fint_Q m_i(D) f(z)\; dz \right|^2 \right)^{1/2}\; dy, \]
%
then $\| S^\# f \|_{L^p(\RR^d)} \lesssim A \widetilde{\log}(B/A)^{1/2 - 1/p} \| f \|_{L^p(\RR^d)}$. It then follows by Littlewood-Paley theory implies
%
\begin{align*}
    \| m(D) f \|_{L^p(\RR^d)} &\lesssim_p \left\| \left( \sum_{k = 0}^\infty |m_i(D) f|^2 \right)^{1/2} \right\|_{L^p(\RR^d)}\\
    &\leq \left\| M \left[ \left( \sum_{k = 0}^\infty |m_i(D) f|^2 \right)^{1/2} \right] \right\|_{L^p(\RR^d)}\\
    &\lesssim \left\| S^\# \left( \sum_{k = 0}^\infty |m_i(D) f|^2 \right)^{1/2} \right\|_{L^p(\RR^d)}\\
    &\lesssim A \widetilde{\log}(B/A)^{1/2 - 1/p}.
\end{align*}
%
To bound $S^\#$, we linearize using duality, picking $Q_x$ for each $x$, and a family of functions $\chi_i(x,y)$ such that $\left( \sum |\chi_i(x,y)|^2 \right)^{1/2} \leq 1$, such that
%
\[ S^\# f(x) \approx \fint_{Q_x} \sum_{i = 0}^\infty \left( m_i(D) f(y) - \fint_{Q_x} m_i(D) f(z)\; dz \right) \chi_i(x,y)\; dy. \]
%
Thus $S^\# f = S_1 f + S_2 f$, where if $Q_x$ has sidelength $2^{l(x)}$,
%
\[ S_1 f(x) = \fint_{Q_x} \sum_{|i + l(x)| \leq \tilde\log(B/A)} \left( m_i(D) f(y) - \fint_{Q_x} m_i(D) f(z)\; dz \right) \chi_i(x,y)\; dy \]
%
and
%
\[ S_2 f(x) = \fint_{Q_x} \sum_{|i + l(x)| \geq \tilde\log(B/A)} \left( m_i(D) f(y) - \fint_{Q_x} m_i(D) f(z)\; dz \right) \chi_i(x,y)\; dy. \]
%
If $|i + l(x)| \lesssim 1$, then the uncertainty principle tells us that $m_i(D) f$ is roughly constant on squares on radius $Q_x$, up to some small error, so that we should expect
%
\[ \left| m_i(D) f(y) - \fint_{Q_x} m_i(D) f(z)\; dz \right| \lesssim \left| \fint_{Q_x} m_i(D) f(z)\; dz \right|. \]
%
Thus it is natural to use the bound, $|S_1 f(x)| \lesssim M(\sum_{i = 0}^\infty |m_i(D) f|^2)^{1/2}$, which implies
%
\begin{align*}
    \| S_1 f \|_{L^2(\RR^d)} &\lesssim \| M(\sum_{i = 0}^\infty |m_i(D) f|^2)^{1/2} \|_{L^2(\RR^d)}\\
    &\lesssim \left\| (\sum_{i = 0}^\infty |m_i(D) f|^2 )^{1/2} \right\|_{L^2(\RR^d)}\\
    &= \left( \sum_{i = 0}^\infty \| m_i(D) f \|_{L^2(\RR^d)}^2 \right)^{1/2}
\end{align*}
%
and
%
\begin{align*}
    \| S_1 f \|_{L^\infty(\RR^d)} &\leq \| M(\sum_{|i + l(x)| \leq \tilde{\log}(B/A)}^\infty |m_i(D) f|^2)^{1/2} \|_{L^\infty(\RR^d)}\\
    &\leq \left\| \left( \sum_{|i + l(x)| \leq \tilde{\log}(B/A)} |m_i(D) f|^2 \right)^{1/2} \right\|_{L^\infty(\RR^d)}\\
    &\lesssim \tilde{\log}(B/A)^{1/2} \sup_i  \| m_i(D) f \|_{L^\infty(\RR^d)}
\end{align*}
%
Interpolation gives $\| S_1 f \|_{L^p(\RR^d)} \lesssim \tilde{\log}(B/A)^{1/2 - 1/p} \| m_i(D) f \|_{L^p_x(l^p_i)}$. But now Littlewood-Paley theory shows that
%
\[ \| m_i(D) f \|_{L^p_x(l^p_i)} \leq A \left( \sum_{i = 0}^\infty \| P_i f \|_{L^p(\RR^d)} \right)^{1/p} \leq A \left( \sum_{i = 0}^\infty \| P_i f \|_{L^p(\RR^d)}^2 \right)^{1/2} \lesssim A \| f \|_{L^p}. \]
%
Thus $\| S_1 f \|_{L^p(\RR^d)} \lesssim A \tilde{\log}(B/A)^{1/2 - 1/p} \| f \|_{L^p(\RR^d)}$.

On the other hand, if $i$ is much smaller than $l(x)$, we should expect the error between $m_i(D) f(y)$ and $\fint_{Q_x} m_i(D) f(z)\; dz$ to be even smaller, and if $i$ is much bigger, then $m_i(D) f$ is no longer constant at this scale, and so the averages should be small, so $m_i(D) f(x)$ should dominate $\fint_{Q_x} m_i(D) f(z)$. Now since our assumption implues that $\| m(D) f \|_{L^2(\RR^d)} \lesssim \| f \|_{L^2(\RR^d)}$, it is not so difficult to prove that
%
\[ \| S_2 f \|_{L^2(\RR^d)} \lesssim A \| f \|_{L^2(\RR^d)} \sim A \left\| \left( \sum |P_i f|^2 \right)^{1/2} \right\|_{L^2(\RR^d)}. \]
%
The difficulty is proving $\| S_2 f \|_{L^\infty(\RR^d)} \lesssim A \| \left( \sum |P_i f|^{1/2} \right) \|_{L^\infty(\RR^d)}$, which we can interpolate into an inequality like above where we can apply Littlewood-Paley theory. To do this we perform another decomposition, writing
%
\[ S_2 f = I f + {II}f \]
%
where
%
\[ If(x) = \fint_{Q_x} \sum_{|i + l(x)| \geq \tilde\log(B/A)} \left( m_i(D) (\mathbf{I}_{2Q_x}f)(y) - \fint_{Q_x} m_i(D)(\mathbf{I}_{2Q_x} f)(z)\; dz \right) \chi_i(x,y)\; dy. \]
%
and
%
\[ If(x) = \fint_{Q_x} \sum_{|i + l(x)| \geq \tilde\log(B/A)} \left( m_i(D) (\mathbf{I}_{(2Q_x)^c}f)(y) - \fint_{Q_x} m_i(D)(\mathbf{I}_{(2Q_x)^c} f)(z)\; dz \right) \chi_i(x,y)\; dy. \]
%
Now
%
\[ \|If \|_{L^\infty} \leq \sup_x \fint_{Q_x} \left( \sum | m_i(D) (\mathbf{I}_{2Q_x} f) |^2 \right)^{1/2}\; dy \leq \sup_x |Q_x|^{-1/2} \left( \sum \| m_i(D) (\mathbf{I}_{2Q_x} f) \|_{L^2(\RR^d)}^2 \right)^{1/2} \lesssim A |Q_x|^{-1/2} \left( \| \mathbf{I}_{2Q_x} f \|_{L^2(\RR^d)}^2 \right)^{1/2} \lesssim A \left( \sum_{i = 0}^\infty \fint_{R_x} |f|^2 \right)^{1/2} \]




\part{Stuff to Read in More Detail}

\begin{itemize}
    \item Sogge, $L^p$ Estimates For the Wave Equation and Applications (1993).

    A survey of results on regularity results for the wave equation. In particular, reviews (without proof) the ideas of Mockenhaipt, Seeger, and Sogge which give local smoothing for Fourier integral operators satisfying the cone condition, as well as mixed norm estimates for non-homogeneous results on wave equations.

    \item In Sogge's Book, he mentions the main developments in harmonic / microlocal analysis he couldn't discuss in the book were the following:
    \begin{itemize}
        \item Bennett, Carbery, Tao, On the Multilinear Restriction and Kakeya Conjecture (2006).

        Introduction to multilinear methods in harmonic analysis.

        \item Bourgain, Guth, Bounds on Oscillatory Integral Operators Based on Multilinear Estimates (2010).

        Application of multilinear methods to bounding oscillatory integrals.

        \item Bourgain, Demeter, The Proof of the l2 Decoupling Conjecture (2014).

        Introduction to Decoupling.

        \item Peetre, New Thoughts on Besov-Spaces.

        Characterizes boundedness of Fourier multipliers on homogeneous Besov spaces.

        \item Johnson, Maximal Subspaces of Besov-Spaces Invariant Under Multiplication By Characters.

            Shows a Fourier multiplier operator is bounded in the $L^p$ norm if and only if it's translates are all localizably bounded as in Seeger.
    \end{itemize}

    \item For more background reading in microlocal analysis:
    \begin{itemize}
        \item H\"{o}rmander, The Analysis of Linear Partial Differential Operators, Volumes I-IV.
        \item Treves, Introduction to Pseudodifferential and Fourier Integral Operators, Volumes I-II.
        \item Taylor.
    \end{itemize}

%Chapter 4 describes the work of
%    - Hormander, The Spectral Function of an Elliptic Operator
%    - Avakumovic, Uber die Eigenfunktionen auf Geschlossenen Riemannschen Mannigfaltigkeiten
%    - Levitan, On the Asymptotic Behaviour of the Spectral Function of a Self-Adjoint Differential Equation of Second Order.

%- Read Hormander, Estimates for Translation Invariant Operators on Lp Spaces For More In Depth Foundations of Lp Boundedness of Multiplier Operators
%- See Strichartz [1] and Keel Tao [1], Ginibre Velo [1], Lindblad Sogge [1] for sharp embeding of wave operator using orthogonality argument introduced in introduction.
%- Seeger, Roos, Po Lam Yung. Maximal Functions for Families of Hilbert Transforms.
%- Guo, Oh, Wang. The Bochner-Riesz Problem: An Old Approach Revisited.
%- Find Stuff about the Transference Principle
%- Hickman, Guth, Illiopoulos. Sharp Estimates for Oscillatory Integral Operators via Polynomial Partitioning.

%- Fourier Restriction for Hypersurfaces in Three Dimensions and Newton Polyhedra

\end{itemize}










\bibliographystyle{plain}
\bibliography{RadialMultipliers}

\end{document}