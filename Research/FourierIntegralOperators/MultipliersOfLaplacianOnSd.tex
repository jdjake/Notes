\documentclass[dvipsnames,letterpaper,12pt]{article}

\usepackage[margin = 1in]{geometry}
\usepackage{amsmath,amssymb,graphicx,mathabx,accents}
\usepackage{enumerate,mdwlist}

\usepackage{tikz}

%\setlist[enumerate]{label*={\normalfont(\Alph*)},ref=(\Alph*)}

%\numberwithin{equation}{section}

\usepackage{amsthm}

\usepackage{hyperref}

\usepackage{verbatim}

\usepackage{nag}

\DeclareMathOperator{\minkdim}{\dim_{\mathbb{M}}}
\DeclareMathOperator{\hausdim}{\dim_{\mathbb{H}}}
\DeclareMathOperator{\lowminkdim}{\underline{\dim}_{\mathbb{M}}}
\DeclareMathOperator{\upminkdim}{\overline{\dim}_{\mathbb{M}}}
\DeclareMathOperator{\fordim}{\dim_{\mathbb{F}}}

\DeclareMathOperator{\lhdim}{\underline{\dim}_{\mathbb{M}}}
\DeclareMathOperator{\lmbdim}{\underline{\dim}_{\mathbb{MB}}}

\DeclareMathOperator{\RR}{\mathbb{R}}
\DeclareMathOperator{\ZZ}{\mathbb{Z}}
\DeclareMathOperator{\QQ}{\mathbb{Q}}
\DeclareMathOperator{\TT}{\mathbb{T}}
\DeclareMathOperator{\CC}{\mathbb{C}}

\DeclareMathOperator{\B}{\mathcal{B}}

\newtheorem{theorem}{Theorem}
%\newtheorem{lemma}{Lemma}
%\newtheorem{corollary}{Corollary}
\newtheorem{lemma}[theorem]{Lemma}
\newtheorem{corollary}[theorem]{Corollary}
%\newtheorem{prop}[theorem]{Proposition}
\newtheorem{remark}[theorem]{Remark}
\newtheorem{remarks}[theorem]{Remarks}
\newtheorem*{remarksaboutresults}{Remarks About The Results Stated}
%\newtheorem*{concludingremarks}{Concluding Remarks}

\DeclareMathOperator{\EE}{\mathbb{E}}
\DeclareMathOperator{\PP}{\mathbb{P}}

\DeclareMathOperator{\DQ}{\mathcal{Q}}
\DeclareMathOperator{\DR}{\mathcal{R}}

\newcommand{\psitwo}[1]{\| {#1} \|_{\psi_2(L)}}
\newcommand{\TV}[2]{\| {#1} \|_{\text{TV}({#2})}}



\title{Necessary and Sufficient Conditions For Boundedness of Functions of Laplace-Beltrami Multipliers on $S^d$}
\author{Jacob Denson\footnote{University of Madison Wisconsin, Madison, WI, jcdenson@wisc.edu}}

\begin{document}

\maketitle

Consider the unit sphere $S^d$ in $\RR^{d+1}$, viewed as a Riemannian manifold. Then we can consider the Laplace-Beltrami operator $\Delta$ on $S^d$. In this paper, we study multipliers of the operator $P = \sqrt{-\Delta}$. We fix a function $h: (0,\infty) \to \CC$ with $\text{supp}(h) \subset \{ t: 1/2 \leq |t| \leq 2 \}$, and study operators of the form
%
\[ h \left( P / R \right) = \sum\nolimits_\lambda h(\lambda / R) \mathcal{P}_\lambda, \]
%
where $\mathcal{P}_\lambda$ is the orthogonal projection operator onto the eigenspace of $P$ corresponding to the eigenvalue $\lambda$. We study necessary and sufficient conditions to ensure the $L^p$ boundedness of the operators $h(P/R)$ in the range
%
\[ 1 < p < \frac{2d}{d+1}, \]
%
in terms of appropriate control of the Fourier transform of the function $h$. For $1 \leq q \leq \infty$, define $s_q = (d-1)|1/q - 1/2|$. Let us assume that
%
\[ C_p(h) = \left( \int_0^\infty \left[ \langle t \rangle^{s_p} |\widehat{h}(t)| \right]^p\; dt \right)^{1/p} \]
%
is finite. The main result of this paper is that the operators $\{ h(P/R) : R > 0 \}$ are uniformly bounded on $L^p(S^d)$ if and only if $C_p(h)$ is finite.

\begin{theorem} \label{MainSphereTheorem}
    Suppose $1 \leq p < \frac{2(d-1)}{d+1}$, and $\text{supp}(h) \subset \{ t : 1/2 \leq |t| \leq 2 \}$. Then
    %
    \[ \| h(P/R) f \|_{L^p(S^d)} \lesssim C_p(h) \| f \|_{L^p(S^d)}, \]
    %
    where the implicit constant is uniform over all $f \in L^p(S^d)$, over all $h: (0,\infty) \to \CC$, and over all $R > 0$.
\end{theorem}

\begin{remarks}
    \ 
\begin{enumerate}
    \item By a transference principle of Mitjagin \cite{Mitjagin}, the uniform boundedness of the operators $\{ h(P/R) : R > 0 \}$ on $L^p(S^d)$ implies that the Fourier multiplier operator on $\RR^d$ with symbol $h(|\cdot|)$ is bounded in $L^p(\RR^d)$. As discussed in \cite{HeoandNazarovandSeeger}, this can only be true if $C_p(h) < \infty$, so the finiteness of $C_p(h)$ is necessary for the operators $\{ h(P/R) \}$ to be uniformly bounded in $R$. The boundedness of $C_p(h)$ has already been proved sufficient provided the inputs functions $f \in L^p(\RR^d)$ are restricted to \emph{zonal} functions \cite{Alladi}. The main novelty of the result of this paper is that we can extend this bound to general $f \in L^p(\RR^d)$.

    \item If $1 \leq p < 2d/(d+1)$ and $C_p(h) < \infty$, then \cite{HeoandNazarovandSeeger} implies that the operator $h(|\cdot|)$ is bounded as a Fourier multiplier operator on $L^p(\RR^d)$ with
    %
    \[ \| h(|\cdot|) \|_{M^p(\RR^d)} \lesssim C_p(h). \]
    %
    Interpolation and duality (see Section 2.5.5 of \cite{Grafakos} for more details) implies that the operator is also a Fourier multiplier operator on $L^2(\RR^d)$, and so we conclude that
    %
    \[ \| h \|_{L^\infty(\RR)} = \| h( |\cdot| ) \|_{M^2(\RR^d)} \lesssim C_p(h). \]

    \item The projection operators $\{ \mathcal{P}_\lambda \}$ are each individually smoothing, though not uniformly as $\lambda \to \infty$. They thus individually satisfy bounds of the form $\| \mathcal{P}_\lambda f \|_{L^p(S^d)} \lesssim_\lambda \| f \|_{L^p(M)}$ for all $1 \leq p \leq \infty$. It thus follows trivially from the triangle inequality, and that there are finitely many eigenvalues for $P$ in $[0,100]$, that for any $R \leq 100$,
    %
    \begin{align*}
        \| M_R f \|_{L^p(S^d)} &\leq \sum_\lambda |h(\lambda/R)| \| \mathcal{P}_\lambda f \|_{L^p(S^d)}\\
        &\leq \| h \|_{L^\infty(0,200)} \sum\nolimits_{\lambda \in [0,200]} \| \mathcal{P}_\lambda f \|_{L^p(S^d)}\\
        &\lesssim \| h \|_{L^\infty(0,\infty)} \| f \|_{L^p(S^d)}.
    \end{align*}
    %
    Thus, in the analysis that follows, we will always assume that $R \geq 100$.

    \item Because $h$ is a unit scale multiplier, if we fix a smooth bump function $\beta$ supported on $\{ 1/4 \leq |t| \leq 4 \}$, and equal to one on $\{ 1/2 \leq |t| \leq 2 \}$, set $\beta_R(\lambda) = \beta(\lambda / R)$, and set $Q_R = \beta(P/R)$, then
    %
    \[ h(P/R) = Q_R \circ h(P/R) \circ Q_R. \]
    %
    By including the operators $\{ Q_R \}$ in our analysis, we essentially reduce our analysis to the study to inputs and outputs lying in the range of the operators $Q_R$, which is equal to the finite dimensional subspace $V_R$ of $C^\infty(S^d)$ spanned by eigenfunctions of $P$ with eigenvalue in $R/4 \leq \lambda \leq 4R$. Since $P$ is positive-semidefinite and self-adjoint, it is often useful to use the heuristic that an element of $V_R$ should behave like a function on $\RR^d$ with Fourier support on $\{ R/4 \leq |\xi| \leq 4R \}$.

    A particular application of this heuristic is an analogue of Bernstein's inequality on $\RR^d$ (see \cite{Wolff},  Proposition 5.1), but for functions on a Riemannian manifold lying in $V_R$. This analogue states that for $1 < r < \infty$, uniformly for $R \geq 100$ and $f \in V_R$ we have
    %
    \begin{equation} \label{ManifoldBernsteinInequality}
        \| f \|_{L^r_s(S^d)} \lesssim_{r,s} R^s \| f \|_{L^r(S^d)}.
    \end{equation}
    %
    See Section 3.3 of \cite{Sogge} for a proof.

    Another useful inequality follows from the fact that the family of functions $\{ \beta_R \}$ form a uniformly bounded subset of the Fr\'{e}chet space $\mathcal{S}^0$, i.e. satisfying estimates of the form
    %
    \[ |\partial_\lambda^n \{ \beta_R \}(\lambda)| \lesssim_n \langle \lambda \rangle^{-n} \quad\text{uniformly in $R > 0$}. \]
    %
    It follows that the operators $\{ Q_R \}$ are pseudodifferential operators of order zero, uniformly bounded as operators on $L^r_s(M)$ for all $1 < r < \infty$, i.e. satisfying
    %
    \begin{equation}
        \| Q_R f \|_{L^r_s(S^d)} \lesssim_{r,s} \| f \|_{L^r_s(S^d)}.
    \end{equation}
    %
    See Corollary 4.3.2 of \cite{Sogge} for more details.
\end{enumerate}
\end{remarks}

\begin{comment}

Our remarks reduce Theorem \ref{MainSphereTheorem} to the following statement.

\begin{theorem} \label{RestrictedMainSphereTheorem}
    Suppose $1 \leq p < 2d/(d+1)$. If $R \geq 100$, then for $f \in V_R$,
    %
    \[ \| M_R f \|_{L^p(S^d)} \lesssim C_p(h) \| f \|_{L^p(S^d)}, \]
    %
    where the implicit constant is uniform in $f$, $h$, and $R$.
\end{theorem}

\begin{proof}[Proof that Theorem \ref{RestrictedMainSphereTheorem} implies Theorem \ref{MainSphereTheorem}]
    Remark 2 shows that the analysis of $\{ M_R \}$ for $R \leq 100$ is trivial. For any $f \in L^p(S^d)$, $Q_R f \in V_R$, we conclude from Remark 3 that
    %
    \[ \| M_R f \|_{L^p(S^d)} = \| M_R (Q_R f) \|_{L^p(S^d)} \lesssim C_p(h) \| Q_R f \|_{L^p(S^d)} \lesssim C_p(h) \| f \|_{L^p(S^d)}. \]
    %
    We have thus obtained the required bound for general inputs.
\end{proof}

\end{comment}

%In microlocal analysis, it is often handy to work with Schwartz operators $T$ \emph{modulo smoothing operators}, i.e. modulo Schwartz operators whose kernels are smooth. This simplifies many calculations, including those involving $L^p$ estimates, since the set of all pairs $(p,q)$ such that $T$ maps $L^p_c$ into $L^q_{\text{loc}}$ is stable modulo the smoothing operators. Since we are proving estimates parameterized by a family $R$, we augment this terminology slightly. We will say a family of operators $\{ T_R \}$ is \emph{uniformly smoothing} if the kernels $\{ K_R \}$ satisfy estimates of the form
%
%\[ |\partial_x^\alpha \partial_y^\beta K_R(x,y)| \lesssim_{\alpha,\beta} 1 \]
%
%where the implicit constant is independent of $R$. The set of all pairs $(p,q)$ such that $\{ T_R \}$ maps $L^p_c$ uniformly into $L^q_{\text{loc}}$ is then preserved modulo uniformly smoothing operators. If, we have the stronger condition that
%
%\[ |\partial_x^\alpha \partial_y^\beta K_R(x,y)| \lesssim_{\alpha,\beta,N} R^{-N} \]
%
%for all $N \geq 0$, we say the operators $\{ T_R \}$ are \emph{uniformly negligible}. Such a bound becomes more useful than mere uniform smoothness in arguments where we decompose the operator $T_R$ into a sum of $O(R^\alpha)$ different operators, since modifying each of these operators by a uniformly negligible 

%In order to deal with certain remainder terms that arise in our calculations, it will be useful to keep some naive bounds on the $L^p$ norm of $M_R f$ handy.

%\begin{lemma} \label{SmoothInputBounds}
%    If $s > 2d$, then for $1 \leq p \leq \infty$, and for any function $j: [0,\infty) \to \CC$, and set $J_R = j(P / R)$. Then
    %
%    \begin{equation} \label{SmoothInputInequality1}
%        \| J_R f \|_{L^p(S^d)} \lesssim_s \| j \|_{L^\infty([0,\infty))} \| f \|_{L^2_s(S^d)}.
%    \end{equation}
    %
%    If $h: [0,\infty) \to \CC$ is a unit scale multiplier, and $M_R = h(P/R)$, then
    %
%    \begin{equation} \label{SmoothInputInequality2}
%        \| M_R f \|_{L^p(S^d)} \lesssim_s R^s\; \| h \|_{L^\infty} \| f \|_{L^p(S^d)}.
%    \end{equation} 
%\end{lemma}
%\begin{proof}
%    Without loss of generality, since $S^d$ has finite measure, in the proof of \eqref{SmoothInputInequality1} we may assume $p = \infty$. Fix $0 < \varepsilon < s - 2d$. Using Sobolev embeddings, if $f \in C^\infty(S^d)$ we calculate that
    %
%    \begin{align*}
%        \| \mathcal{P}_\lambda f \|_{L^\infty(S^d)} &\lesssim \| \mathcal{P}_\lambda f \|_{L^2_{d+\varepsilon}(S^d)}\\
%        &\lesssim \| (1 + P)^{d+\varepsilon} \mathcal{P}_\lambda f \|_{L^2(S^d)}\\
%        &\lesssim \lambda^{d+\varepsilon} \| \mathcal{P}_\lambda f \|_{L^2(S^d)} \lesssim \lambda^{d+\varepsilon-s} \| f \|_{L^2_s(S^d)}.
%    \end{align*}
    %
%    The Weyl law implies that for $k \geq 1$, there are $O(2^{kd})$ eigenvalues of $P$ in the interval $[2^k,2^{k+1}]$, and so
    %
%    \begin{align*}
%        \left\| J_R f \right\|_{L^\infty(S^d)} &\lesssim \| j \|_{L^\infty(\RR)} \left( \| \mathcal{P}_0 f \|_{L^\infty(S^d)} + \sum_{k = 1}^\infty \sum_{2^k \leq \lambda \leq 2^{k+1}} \| \mathcal{P}_\lambda f \|_{L^\infty(S^d)} \right)\\
%        &\lesssim \| j \|_{L^\infty(\RR)} \Big( 1 + \sum_{k = 1}^\infty 2^{kd} 2^{k(d+\varepsilon - s)} \Big) \| f \|_{L^2_s(S^d)}\\
%        &\lesssim \| j \|_{L^\infty(\RR)} \| f \|_{L^2_s(S^d)}.
%    \end{align*}
    %
%    Thus we have proved \eqref{SmoothInputInequality1}. To prove \eqref{SmoothInputInequality2}, we again apply the Weyl law, i.e. that there are $O(R^d)$ eigenvalues of $P$ on the support of $h(\cdot / R)$, which gives
    %
%    \[ \| M_R f \|_{L^\infty(S^d)} \lesssim \| h \|_{L^\infty(\RR)} \| f \|_{L^2_s(S^d)}. \]
    %
%    For $f \in V_R$, Bernstein's inqeuality implies that $\| f \|_{L^2_s(S^d)} \lesssim R^s \| f \|_{L^2(S^d)}$, which yields
    %
%    \[ \| M_R f \|_{L^\infty(S^d)} \lesssim R^s \| f \|_{L^2(S^d)}. \]
    %
%    For general inputs, the same inequality follows by the fact that $M_R = M_R \circ Q_R$, and that the operators $\{ Q_R \}$ are uniformly bounded on $L^2(S^d)$.

%    we start by making the assumption that $1 \leq p \leq 2$. If we use \eqref{SmoothInputInequality1}, we conclude that for $f \in L^p(S^d)$, with $g = Q_R f$,
    %
%    \begin{align*}
%        \| M_R f \|_{L^p(S^d)} = \| M_R g \|_{L^p(S^d)} &\lesssim \| h \|_{L^\infty} \| g \|_{L^2_s(S^d)}\\
%        &\lesssim R^s \| h \|_{L^\infty} \| g \|_{L^2(\RR^d)} \lesssim R^s \| h \|_{L^\infty} \| f \|_{L^2(\RR^d)}.
%    \end{align*}
    %
%    Now that the result has been proved for $1 \leq p \leq 2$, the result for $2 \leq p \leq \infty$ is immediately implied by duality, since $h(P)^* = \overline{h}(P)$, and $\| \overline{h} \|_{L^\infty} = \| h \|_{L^\infty}$.
%    \end{comment}
%\end{proof}

% Later on in our analysis, we will reduce our analysis to the study of various Fourier integral operators. The composition calculus of Fourier integral operators will show that this assumption also allows us to show the behaviour of these operators is mainly determined by the behaviour of the integral defining the operators over values of the phase variables roughly proportional to $R$.

To exploit the fact that $C_p(h)$ is finite in the proof, we must employ the Fourier transform of $h$ in some way. A standard method is to apply the Fourier inversion formula to write
%
\[ h(P/R) = \int_{-\infty}^\infty R \widehat{h}(Rt) e^{2 \pi i t P}\; dt, \]
%
where
%
\[ e^{2 \pi i t P} = \sum_\lambda e^{2 \pi i t \lambda} \mathcal{P}_\lambda \]
%
is the multiplier operator on $S^d$ which, as $t$ varies, gives solutions to the half-wave equation
%
\[ \partial_t - i P = 0. \]
%
%
%\[ M_R = 2 \int_0^\infty R \widehat{h}(Rt) \cos(2 \pi t P)\; dt, \]
%
%where
%
%\[ \cos(2 \pi t P) = \sum_\lambda \cos(2 \pi t \lambda) \mathcal{P}_\lambda \]
%
%is the Fourier multiplier giving solutions to the wave equation
%
%\[ \partial_t^2 u + (2 \pi P)^2 = 0. \]
%
%
%
%We begin by listing some fixed time bounds for the propogators $\{ e^{2 \pi i t P} \}$. The propogators are unitary, so for $t \in \RR$,
%
%\[ \| e^{2 \pi i t P} f \|_{L^2(S^d)} = \| f \|_{L^2(S^d)}. \]
%
%For $1 < p < \infty$, we have
%
%\[ \| e^{2 \pi i t P} f \|_{L^p(S^d)} \lesssim_{p,t_0} \| f \|_{L^p_{s_p}(S^d)} \quad\text{uniformly for $t > 0$}, \]
%
%which follows from Corollary 6.2.3 of \cite{Sogge}, and the periodicity of the wave propogators $\{ e^{2 \pi i t P} \}$.
%
Our goal is thus to study the regularity properties of averages of the half-wave operator.

Consider a cover
%
\[ \{ |t| < 100/R \} \cup \{ 50/R < |t| < 1/100 \} \cup \{ 1/200 < |t| < \infty \} \]
%
of $\RR$, and find a smooth partition of unity $\chi_{I,R}$, $\chi_{II,R}$, and $\chi_{III,R}$ adapted to these sets. Without loss of generality, we may assume all three functions are even, that $\chi_{I,R}(t) = \chi_I(Rt)$, for some smooth, compactly supported function $\chi_I$ adapted to the open set $\{ |t| < 1 \}$, and also assume that $\chi_{III,R} = \chi_{III}$ is independent of $R$. Given this partition of unity, we now write
%
\[ h(P/R) = I_R + II_R + III_R, \]
%
where, for $\Pi \in \{ I, II, III \}$, the operators
%
\[ \Pi_R = \int \chi_{\Pi,R}(t) R \widehat{h}(Rt) e^{2 \pi i t P}\; dt \]
%
isolate the study of $h(P/R)$ to the behaviour of the half-wave propogators on three different time intervals. The remainder of the argument will consist of separately obtaining $L^p$ boundedness for each of the three operators $Q_R \circ \Pi_R \circ Q_R$, since then the triangle inequality gives the $L^p$ boundedness of
%
\begin{align*}
    (Q_R \circ I_R \circ Q_R) + (Q_R \circ II_R \circ Q_R) + &(Q_R \circ II_R \circ Q_R) = h(P/R).
\end{align*}
%
The study of the operators $\{ I_R \}$ will reduce to a study of pseudodifferential operators, we will be able to apply the endpoint local smoothing inequality of \cite{LeeSeeger} to control the operators $\{ III_R \}$, and the study of the operators $\{ II_R \}$ will be given by generalizations of the methods of \cite{HeoandNazarovandSeeger} to a variable coefficient setting.




\section{Analysis of $I_R$}

Let us analyze
%
\[ I_R = \int \chi_I(Rt) R \widehat{h}(Rt) e^{2 \pi i t P}\; dt. \]
%
We are analyzing inputs to $I_R$ coming from the composition of a general element of $C^\infty(S^d)$ with $Q_R$, which heuristically localizes the `frequency support' of this function to a band of frequencies with magnitude $\sim R$. Thus, by uncertainty principle heuristics, such functions are locally constant at a scale $1/R$. The half-wave equation propogates a majority of the mass of it's input at a unit speed, and since the operators $\{ I_R \}$ are obtained by averaging the half-wave equation over times $\lesssim 1/R$, we should expect that the behaviour of the operators $\{ I_R \}$ to behave in a localized manner. In fact, the following analysis will show that the operators $\{ I_R \}$ are pseudodifferential operators, which will allow us to conclude these operators are uniformly bounded in $L^p(S^d)$.

\begin{comment}

This result follows from the following Lemma.

\begin{lemma}
    For each $m \in \mathcal{S}^0(\RR)$, and each $R > 0$, the operator $m(P/R)$ is a pseudodifferential operator of order $0$. Moreover, if we fix a coordinate chart $x: U \to V$ from an open subset $U \subset M$ of our manifold $M$ onto $V \subset \RR^d$, and let $s_R \in \mathcal{S}^0(T^* V)$ denote it's symbol in the coordinate chart, i.e. the symbol of the pseudodifferential operator $(x^{-1})^* \circ m(P/R) \circ x^*$ on $\DD(V)$, then the family of symbols $\{ s_R : R \geq 1 \}$ forms a bounded subset of $\mathcal{S}^0(T^* V)$.
\end{lemma}
\begin{proof}
    We apply the Hadamard parametrix. Let $0 < \varepsilon_0 < T_{\text{max}}$ such that for each $\sigma \in \{ +, - \}^2$, the interval $[0, 10 \varepsilon_0]$ is disjoint from $\text{supp}_\tau a_\sigma$. Consider an even function $\rho \in \mathcal{S}(\RR)$ such that
    %
    \[ \text{supp} \left( \widehat{\rho} \right) \subset [0,\varepsilon_0], \]
    %
    and such that $\widehat{\rho}(t) = 1$ for $|t| \leq \varepsilon_0 / 2$. Set $m_R(\lambda) = m(\lambda / R)$. Then
    %
    \[ m_R(P) = (m_R * \rho)(P) + (m_R * (\delta - \rho))(P). \]
    %
    We will analyze each of these operators separately.

    Let $r_R = m_R * (\delta - \rho)$. Since $m$ is a symbol of order zero, we have
    %
    \[ |D^\alpha_t \widehat{m}(t)| \lesssim_{\alpha,N} |t|^{-1-\alpha-N} \quad\text{for all $N \geq 0$}. \]
    %
    Since
    %
    \[ \widehat{r}_R(t) = \widehat{m}_R(t) (1 - \widehat{\rho}(t)) = [R \widehat{m}(Rt)] (1 - \widehat{\rho}(t)), \]
    %
    which vanishes for $|t| \leq \varepsilon_0 / 2$, we conclude from the estimates above and the product rule that
    %
    \[ |D^M_t \{ \widehat{r}_R(t) \} | \lesssim_N R^{-N} \langle t \rangle^{-1 - M - N} \quad\text{for all $N \geq 0$}. \]
    %
    But inverting the Fourier transform thus gives that
    %
    \[ |r_R(\lambda)| \lesssim_{N,M} R^{- N} \lambda^{-M} \quad\text{for all $M + N > 0$}. \]
    %
    If $\phi$ is a normalized bump function supported on $[-1,+1]$, then for $0 \leq T \leq \varepsilon$,
    %
    \[ \int \widehat{r}_R(t) \phi(t / T)\; dt = \int [R\widehat{m}(Rt)] (1 - \widehat{\rho}(t)) \phi(t/T)\; dt = 0, \]
    %
    and for $\varepsilon \leq T \leq 1$, and for $N > 1$,
    %
    \begin{align*}
        \left| \widehat{r}_R(t) \phi(t/T)\; dt \right| &= \left| \int [R\widehat{m}(Rt)] (1 - \widehat{\rho}(t)) \phi(t/T)\; dt \right|\\
        &\lesssim_\varepsilon \left| \int m_R(\lambda) \widehat{\phi}(T \lambda) \right| + \left| \int m_R(\lambda) \int \rho(\lambda - \tau) \widehat{\phi}(T \tau) \right|\\
        &\lesssim_N \int |m(\lambda / R)| \langle \lambda \rangle^{-N}\; d\lambda \leq \| m \|_{L^\infty}.
    \end{align*}
    %
    The uniformity here in $T$ and $R$ implies (see Stein, Chapter 6, Proposition 3) that the family $\{ r_R : R > 0 \}$ are a family of symbols of order zero, uniformly bounded in $\mathcal{S}^0(\RR)$, so that in particular, we also have the estimate
    %
    \[ |r_R(t)| \lesssim_N 1. \]
    %
    Let $X_R$ be the space spanned by eigenfunctions to $P$ with eigenvalues in $[0,R]$. We split our proof into two cases.
    %
    \begin{itemize}
        \item For $R \geq 1$, $\dim(X_R) \lesssim R^d$. Consider any orthonormal basis $\{ e_{R,j} \}$ to $X_R$ with $P e_{R,j} = \lambda_{R,j} e_{R,j}$. Crude Sobolev embedding techniques imply that for each $f \in X_R$,
        %
        \[ \| D^\alpha_x f \|_{L^\infty(M)} \lesssim_\alpha R^{|\alpha| + d/2 + 1} \| f \|_{L^2(M)}. \]
        %
        But from this, we conclude that for $M > d$,
        %
        \begin{align*}
            |D^\alpha_x D^\beta_y r_R(P)(x,y)| &\leq \sum_j |r_R(\lambda_{R,j})| |D^\alpha_x e_j(x)| |D^\beta_y e_j(y)|\\
            &\lesssim_{N,M,\alpha,\beta} R^{M - N} \sum_j \lambda_{R,j}^{-M} R^{|\alpha| + d/2 + 1} R^{|\beta| + d/2 + 1}\\
            &\lesssim R^{d + 2 + M - N + |\alpha| + |\beta|} \sum_j \lambda_{R,j}^{-M}\\
            &\lesssim R^{d + 2 + M - N + |\alpha| + |\beta|}.
        \end{align*}
        %
        Taking $N > d + 2 + M + |\alpha| + |\beta|$ gives that
        %
        \[ |D^\alpha_x D^\beta_y r_R(P)(x,y)| \lesssim_{\alpha,\beta} 1. \]

        \item If $R \leq 1$, then $\dim(X_R) \lesssim 1$. Then for any $f \in X_R$,
        %
        \[ \| D^\alpha_x f \|_{L^\infty(M)} \lesssim \| f \|_{L^2(M)}. \]
        %
        But this implies, as above, instead using the uniform bound $|r_R(t)| \lesssim 1$, that
        %
        \[ |D^\alpha_x D^\beta_y r_R(P)(x,y)| \lesssim_{\alpha,\beta} 1. \]
    \end{itemize}
    %
    Thus $\{ r_R : R > 0 \}$ is a family of smoothing operators, uniformly in $R$.





    Now we discuss $(m_R * \rho)(P)$, which is now frequency localized to within the scope where we can apply the Lax parametrix to write
    %
    \[ (m_R * \rho)(P) = \sum_\sigma T_{R,\sigma} + S_R, \]
    %
    where
    %
    \[ T_\sigma = \int [R \widehat{m}(Rt)] \widehat{\rho}(t) L_{t,\sigma}\; dt, \]
    %
    and where
    %
    \[ S_R = \int [R \widehat{m}(Rt)] \widehat{\rho}(t) A_t\; dt. \]
    %
    We're analyze each of these terms separately.

    Firstly, $\{ S_R \}$ is a family of smoothing operators, uniformly in $R$. If
    %
    \[ b(x,y,\lambda) = \int \widehat{\rho}(t) A_t(x,y) e^{2 \pi i \lambda t}\; dt, \]
    %
    then integration by parts yields that
    %
    \[ |D^\alpha_x D^\beta_y b(x,y,\lambda)| \lesssim_{\alpha,\beta,N} \langle \lambda \rangle^{-N}. \]
    %
    The multiplication formula for the Fourier transform implies we can write
    %
    \[ S_R(x,y) = \int m_R(\lambda) b(x,y,\lambda)\; d\lambda. \]
    %
    This yields estimates of the form
    %
    \[ |D^\alpha_x D^\beta_y S_R(x,y)| \lesssim_{\alpha,\beta,N} \int \langle \lambda \rangle^{-N} \| m \|_{L^\infty_\lambda}\; d\lambda \lesssim \| m \|_{L^\infty_\lambda}. \]
    %
    Thus the operators $\{ S_R \}$ are smoothing, uniformly in $R$.

    Now we look at the operators $\{ T_\sigma \}$. Set
    %
    \begin{align*}
        c_{\sigma,R}(x,y,\tau) &= \int_0^\infty [R \widehat{m}(Rt)] \widehat{\rho}(t) a_\sigma(x,y,\tau,t) s_\sigma(\tau d_g(x,y)) e^{2 \pi i \tau \sigma_1 t}\; dt\\
        &= s_\sigma(\tau d_g(x,y)) \int_{-\infty}^\infty (m_R * \rho)(\sigma_1 \tau - \lambda) (\mathcal{F}_t a_\sigma)(x,y,\tau, \lambda)\; d\lambda.
    \end{align*}
    %
    Then
    %
    \[ T_\sigma = \int c_{\sigma,R}(x,y,\tau) e^{2 \pi i \sigma_2 \tau d_g(x,y)}\; d\tau. \]
    %
    This gives $T_\sigma$ as an oscillatory integral whose phase has canonical relation on the diagonal. If we can show that the family of functions $\{ c_{\sigma,R} \}$ are uniformly bounded in a given symbol class, then the equivalence of phase theorem implies we can write
    %
    \[ T_\sigma = \int \tilde{c}_{\sigma,R}(x,y,\xi) e^{2 \pi i \xi \cdot (x - y)}\; d\xi, \]
    %
    where the symbols $\{ \tilde{c}_{\sigma,R}(x,y,\xi) \}$ are uniformly bounded, which proves what was required to be shown. To obtain these symbol bounds, we note that for any multi-indices $\alpha$, $\beta$, and $\kappa$, the product rule and differentiation under convolution implies that $D^\alpha_x D^\beta_y D^\kappa_\tau c_{\sigma,R}(x,y,\tau)$ is a finite linear combination of terms of the form
    %
    \[ \tau^j R^{-\kappa_1} \phi_\circ(x,y) \cdot s_\circ(\tau d_g(x,y)) \int_{-\infty}^\infty (m_{\circ,R} * \rho_\circ)(\sigma_1 \tau - \lambda) (\mathcal{F}_t a_\circ)(x,y,\tau, \lambda)\; d\lambda, \]
    %
    where:
    %
    \begin{itemize}
        \item $s_\circ$ is a symbol of order $- (d-1)/2 - j - \kappa_0$.

        \item the functions $\phi_\circ$ and $\rho_\circ$ are smooth with $\text{supp}_\tau \rho_\circ \subset [-\varepsilon_0,\varepsilon_0]$.

        \item $m_\circ$ is a symbol of order $-\kappa_1$, and $m_{\circ,R}(\lambda) = m_\circ(\lambda/R)$.

        \item $a_\circ$ is a symbol of order $d-\kappa_2$ in the $\tau$ variable, with $\text{supp}_\tau a_\circ \subset \text{supp}_\tau a_\sigma$.

        \item $\kappa_0 + \kappa_1 + \kappa_2 = \kappa$, and $0 \leq j \leq |\alpha| + |\beta|$.
    \end{itemize}

        Because $\text{supp}_t a_\sigma \subset I_{\text{max}}$, and
    %
    \[ |D^\alpha_t a_\circ(x,y,\tau,t)| \lesssim \tau^d, \]
    %
    we conclude that
    %
    \[ |(\mathcal{F}_t a_\sigma)(x,y,\tau, \lambda)| \lesssim_N \tau^d \lambda^{- N}. \]
    %
    Thus we conclude that
    %
    \[ \int_{|\lambda| \geq \varepsilon_0} (m_R * \rho_\circ)(\sigma_1 \tau - \lambda) (\mathcal{F}_t a_\circ)(x,y,\tau, \lambda)\; d\lambda \lesssim \| m \|_{L^\infty} \tau^d. \]
    %
    But if $|\lambda| \leq \varepsilon_0$,  then the fact that $m$ is a symbol of order zero, and because any $\tau \in \text{supp}_\tau a_\circ$ implies that
    %
    \[ |(m_R * \rho_\circ)(\sigma_1 \tau - \lambda)| \lesssim 1, \]
    %
    and so we also have that
    %
    \[ \int_{|\lambda| < \varepsilon_0} (m_R * \rho_\circ)(\sigma_1 \tau - \lambda) (\mathcal{F}_t a_\circ)(x,y,\tau,\lambda)\; d\lambda \lesssim 1 \lesssim \| m \|_{L^\infty} \tau^d \]
    %
    But this implies that
    %
    \begin{align*}
        &\left| \tau^i \phi_\circ(x,y) \cdot s_\circ(\tau d_g(x,y)) \int_{-\infty}^\infty (m_R * \rho_\circ)(\sigma_1 \tau - \lambda) (\mathcal{F}_t a_\circ)(x,y,\tau, \lambda)\; d\lambda \right|\\
        &\quad\quad \lesssim \tau^{i+d} \langle \tau d_g(x,y) \rangle^{- \frac{d-1}{2} - i} \lesssim \tau^{d - \frac{d-1}{2}} d_g(x,y)^{ - \frac{d-1}{2} - i}.
    \end{align*}

    But we also have
    %
    \[ |(m_R * \rho)(\tau)| \lesssim \| m \|_{L^\infty}, \]
    %
    which implies
    %
    \[ |c_{\sigma,R}(x,y,\tau)| \lesssim_N \| m \|_{L^\infty} \tau^{-N}. \]


    
    % Triangle Inequality Bounds
    % Sum over k, of the integral on the region t ~ 2^k / R for k = -inf to k = log R
    %   On the interval corresponding to k,
    %       [R m^(Rt)] has magnitude about R 2^{-k}
    %       rho^ has magnitude about 1
    %       a(x,y,tau,t) has magnitude tau^d
    %   So integrating over this interval gives a quantity tau^d, which cannot be summed over k.
    % 2^k \int \chi(t) [m^(2^k t)] a(x,y,tau,2^k t / R)


    % 2^k int chi(t) m^(2^k t) rho^(2^k t / R) a(x,y,tau,2^k t / R) s(tau d_g(x,y)) e^{2 pi i (2^k tau / R) t} dt

    For any multi-indices $\alpha$, $\beta$, and $\kappa$, applying the product rule to the integral expression for $c_{\sigma,R}$ yields that the function $D^\alpha_x D^\beta_y D^\kappa_\tau c_{\sigma,R}$ is a finite sum of terms of the form
    %
    \[ \int [R \widehat{m}(Rt)] \widehat{\rho}(t) [t^{j + |\kappa|} \tilde{a}(x,y,\eta,t)] e^{2 \pi i t p(y,\eta)}\; dt \]

    Since
    %
    \[ |D^N m(\lambda)| \lesssim_N \langle \lambda \rangle^{-N} \]
    %
    \[ |D^N \lambda^M m(\lambda)| = \sum \lambda^{M-N+k} \langle \lambda \rangle^{-k} \]
    % lambda^{M-N} for lambda <= 1
    % lambda^{M-N} for lambda >= 1
    we find that for $N > M+1$,
    %
    \[ |D^M \widehat{m}(t)| \lesssim |t|^{-N} \]

    TODO: Switch back to other parametrix, it's better for this purpose.

    \[ c_{R,W}(x,y,\eta) = \int \widehat{\rho}(t) R \widehat{m}(Rt) a(x,y,\eta,t) e^{2 \pi i t p(y,\eta)}\; dt. \]




    %

    %
    where $a_j$ is a symbol of order $j$, and where $- |\kappa| \leq j \leq |\alpha|$. Now if we set
    %
    \[ g_j(x,y,\eta,\lambda) = \int \widehat{\rho}(t) a_j(x,y,\eta,t) e^{2 \pi i t[p(y,\eta) - \lambda]}\; dt, \]
    %
    then the multiplication formula for the Fourier transform yields that
    %
    \[ c_{R,W,j}(x,y,\eta) = (1/R)^{j + |\kappa|} \int (D^{j + |\kappa|} m)(\lambda / R) g_j(x,y,\eta,\lambda)\; d\lambda. \]
    %
    Integration by parts yields that
    % Symbol of order 0 + |beta|
    \[ |g_j(x,y,\eta,\lambda)| \lesssim_M |\eta|^j \cdot \langle p(y,\eta) - \lambda \rangle^{-M}. \]
    %
    Since $P$ is elliptic, $|p(y,\eta)| \sim |\eta|$, and so we get that
    %
    \[ (1/R)^{j + |\kappa|} \int_{|\lambda| \ll |p(y,\eta)|} (D^{j + |\kappa|} m)(\lambda / R) g_j(x,y,\eta,\lambda)\; d\lambda \lesssim_{j,M} |\eta|^{-M} \]
    %
    and
    %
    \[ (1/R)^{j + |\kappa|} \int_{|\lambda| \gg |p(y,\eta)|} (D^{j + |\kappa|} m)(\lambda / R) g_j(x,y,\eta,\lambda)\; d\lambda \lesssim_{j,M} |\eta|^{-M}. \]
    %
    Finally, we get that
    %
    \begin{align*}
        & (1/R)^{j + |\kappa|} \int_{|\lambda| \sim |p(y,\eta)|} (D^{j + |\kappa|} m)(\lambda / R) g_j(x,y,\eta,\lambda)\; d\lambda\\
        &\quad \lesssim (1/R)^{j + |\kappa|} \left\langle \frac{|\eta|}{R} \right\rangle^{\mu - j - |\kappa|} |\eta|^j \int_{C^{-1} |p(y,\eta)|}^{C |p(y,\eta)|} \langle p(y,\eta) - \lambda \rangle^{-M}\; d\lambda\\
        &\quad \lesssim (1/R)^{j + |\kappa|} \left\langle \frac{|\eta|}{R} \right\rangle^{\mu - j - |\kappa|} |\eta|^j.
    \end{align*}
    %
    For $|\eta| \leq R$, we thus get that
    %
    \[ |c_{R,W,j}(x,y,\eta)| \lesssim |\eta|^{-M} + (1/R)^{j + |\kappa|} |\eta|^j \lesssim |\eta|^{-M} + |\eta|^{-|\kappa|} \lesssim |\eta|^{-|\kappa|}. \]
    %
    For $|\eta| \geq R$, we get that
    %
    \[ |c_{R,W,j}(x,y,\eta)| \lesssim |\eta|^{-M} + R^{- \mu} |\eta|^{\mu - |\kappa|}. \]
    %
    Applying the equivalence of phase theorem, this proves the result if $\mu \geq 0$, but there are problems if $\mu < 0$.
\end{proof}

\end{comment}

\begin{lemma} \label{SmallTimeInequalityLemma}
    For all $f \in C^\infty(S^d)$,
    %
    \[ \| I_R f \|_{L^p(S^d)} \lesssim \| h \|_{L^\infty(\RR)} \| f \|_{L^p(S^d)} \lesssim C_p(h) \| f \|_{L^p(S^d)}, \] 
    %
    where the implicit constant is uniformly bounded in $R \geq 1$ and $h$, for $1 < p < \infty$. Thus in particular,
    %
    \[ \| (Q_R \circ I_R \circ Q_R) f \|_{L^p(S^d)} \lesssim C_p(h) \| f \|_{L^p(S^d)}. \]
\end{lemma}
\begin{proof}
    Let $a$ be the inverse Fourier transform of the function $t \mapsto \chi_I(t) \widehat{h}(t)$. Then $I_R = a(P/R)$. If $\psi$ denotes the inverse Fourier transform of $\chi_I$, then we can write
    %
    \[ a(\lambda) = \int h(\alpha) \psi(\lambda - \alpha)\; d\alpha. \]
    %
    The fact that $h$ is a unit scale multiplier, and $\psi$ is Schwartz, implies that
    %
    \[ |\partial^\alpha a(\lambda)| \lesssim_{\alpha,N} \| h \|_{L^\infty(\RR)} \langle \lambda \rangle^{-N}. \]
    %
    If we set $a_R(\lambda) = a(\lambda / R)$, then
    %
    \[ |\partial^\alpha a_R(\lambda)| \lesssim_\alpha \| h \|_{L^\infty(\RR)} \langle \lambda \rangle^{-|\alpha|}, \]
    %
    with an implicit constant independent of $R$ for $R \geq 1$. Thus the family of symbols $\{ a_R : R \geq 1 \}$ form a uniformly bounded subset of the Fre\'{c}het space $\mathcal{S}^0(\RR)$ of order zero symbols, and so the operators $I_R$ are pseudodifferential operators of order zero, and uniformly bounded in the $L^p(S^d)$ norm for all $1 < p < \infty$, which yields the required claim.
\end{proof}

\section{Analysis of $III_R$}

We now show the uniform boundedness of the operators $\{ III_R \}$ on $L^p(S^d)$ in the range of $p$ we are considering in this problem, by a reduction to an endpoint local smoothing inequality. This might seem unintuitive, since the operators $III_R$ are obtained by averaging the wave equation over large times $|t| \gtrsim 1$, whereas local smoothing gives bounds for averages of the wave equation over times $|t| \lesssim 1$. We are able to reduce large times to small times by exploiting the \emph{periodicity} of the half-wave equation on the sphere.

\begin{lemma} \label{LocalSmoothingLargeTimesTheorem}
    Fix $1 < p < 2d/(d+1)$, let $q$ be the H\"{o}lder conjugate to $p$, and let $I = [-1/2, 1/2]$. Suppose that the sharp local smoothing inequality
    %
    \[ \left\| e^{2 \pi i t P} f \right\|_{L^q(S^d) L^q_t(I)} \lesssim \| f \|_{L^q_{s_q-1/q}(S^d)} \]
    %
    holds for all $f \in C^\infty(S^d)$. Then the operators $\{ III_R \}$ satisfy a bound
    %
    \[ \| (III_R \circ Q_R) f \|_{L^p(S^d)} \lesssim C_p(h) \| f \|_{L^p(S^d)}, \]
    %
    with the implicit constant uniformly bounded in $R$. In particular,
    %
    \[ \| (Q_R \circ III_R \circ Q_R) f \|_{L^p(S^d)} \lesssim C_p(h) \| f \|_{L^p(S^d)}, \]
\end{lemma}
\begin{proof}
    For each $R$, the \emph{class} of operators of the form $\{ III_R \}$ formed from a multiplier $h$ satisfying the hypothesis of Theorem \ref{MainSphereTheorem} is closed under taking adjoints. Indeed, if $III_R$ is obtained from $h$, then $III_R^*$ is obtained from the multiplier $\overline{h}$. Because of this self-adjointness, if we can prove that for any multiplier $h$ satisfying the assumptions of the theorem, the operators $\{ III_R \}$ are uniformly bounded in $L^q(S^d)$, where $q$ is the H\"{o}lder conjugate to $p$, then it follows by duality that for any such $h$, it is also true that the operators $\{ III_R \}$ are uniformly bounded back in the original $L^p(S^d)$ norm. In this argument we will prove such $L^q$ estimates, because we will exploit \emph{local smoothing} inequalities, which tend to work better with large Lebesgue exponents, precisely because Lebesgue norms with large exponents are more sensitive to functions with sharp peaks, something explicitly prevented by obtaining control over the smoothness of a function.

    We begin by noting that for a pair of H\"{o}lder conjugates $p$ and $q$, $s_q = s_p$. Using the periodicity of the wave equation on $S^d$, i.e. that
    %
    \[ e^{2 \pi i (t + n) P} = e^{2 \pi i t P} \quad\quad\quad\text{for $n \in \ZZ$ and $t \in \RR$}, \]
    %
    we can write
    %
    \[ III_R = \int_{-1/2}^{1/2} H_R(t) e^{2 \pi i t P}\; dt, \]
    %
    where
    %
    \[ H_R(t) = \sum_l \chi_{III}(t) R \widehat{h}(R(t + l)) = \sum_l H_{R,l}(t). \]
    %
    Now
    %
    \begin{align*}
        &\left( \sum_{l \neq 0} \left( |Rl|^{s_q} \| H_{R,l} \|_{L^p[-1/2,1/2]} \right)^p \right)^{1/p}\\
        &\quad\quad \sim R \left( \int_{-1/2}^{1/2} \sum_l \left( |R(t + l)|^{s_q} |\widehat{h}(R(t + l))| \right)^p\; dt \right)^{1/p}\\
        &\quad\quad\sim R \left( \int_{|t| \geq 1/2} \left( |Rt|^{s_q} |\widehat{h}(Rt)| \right)^p\; dt \right)^{1/p} \\
        &\quad\quad\lesssim R^{1/q} C_p(h).
    \end{align*}
    %
    and
    %
    \begin{align*}
        \| H_{R,0} \|_{L^p[-1/2,1/2]} &= \left( \int_{-1/2}^{1/2} |\chi_{III}(t) R \widehat{h}(Rt)|^p \right)^{1/p}\\
        &\leq \left( \int_{1/200 \leq |t| \leq 1/2} |R \widehat{h}(Rt)|^p \right)^{1/p}\\
        &= R^{1/q} \left( \int_{R/3}^{R/2} |\widehat{h}(t)|^p \right)^{1/p}\\
        &\lesssim R^{1/q - s_q} C_p(h).
    \end{align*}
    %
    Since the family of functions $\{ H_{R,l} \}$ could in general be chosen arbitrarily, they can be quite correlated, and so we should expect H\"{o}lder's inequality should be efficient, in the worst case. Thus we conclude that, provided $p < 2d/(d+1)$, so that $q > 2d/(d-1)$, and thus
    %
    \[ q s_q = (d-1)(q/2 - 1) > 1, \]
    %
    so we can use H\"{o}lder's inequality to conclude that
    % (d-1)(q/2 - 1) > 1
    % q > 2d/(d-1)
    \begin{align*}
        \| H_R \|_{L^p[-1/2,1/2]} &\leq \sum_l \| H_{R,l} \|_{L^p[-1/2,1/2]}\\
        &= \| H_{R,0} \|_{L^p[-1/2,1/2]} + \sum_{l \neq 0} \left( |Rl|^{s_q} \| H_{R,l} \|_{L^p[-1/2,1/2]} \right) |Rl|^{-s_q} \\
        &\lesssim R^{-s_q + 1/q} C_p(h) + ( R^{1/q} C_p(h) ) \left( \sum_{l \neq 0} |Rl|^{- s_q q} \right)^{1/q}\\
        &= R^{-s_q + 1/q} C_p(h) \left( 1 + \left( \sum_{l \neq 0} |l|^{-s_q q} \right)^{1/q} \right)\\
        &= R^{-s_q + 1/q} C_p(h) \left( 1 + \left( \sum_{l \neq 0} |l|^{- s_q q} \right)^{1/q} \right) \\
        &\lesssim_p R^{-s_q + 1/q} C_p(h).
    \end{align*}
    %
    A further application of H\"{o}lder's inequality shows that
    %
    \begin{align*}
        | III_R | &= \left| \int_{-1/2}^{1/2} H_R(t) e^{2 \pi i t P}\; dt \right|\\
        &\lesssim C_p(h) R^{-s_q + 1/q} \left( \int_{-1/2}^{1/2} |e^{2 \pi i P}|^{q}\; dt \right)^{1/q}.
    \end{align*}
    %
    Applying the endpoint local smoothing inequality, we conclude that
    %
    \begin{align*}
        \| (III_R \circ Q_R)f \|_{L^{q}(M)} &\lesssim C_p(h) R^{-s_q + 1/q} \| e^{2 \pi i P} (Q_R f) \|_{L^{q}_t L^{q}_x}\\
        &\lesssim C_p(h) R^{-s_q + 1/q} \| Q_R f \|_{L^{q}_{s_q - 1/q}(M)},
    \end{align*}
    %
    Applying Bernstein's inequality gives
    %
    \[ \| Q_R f \|_{L^{q}_{s_q - 1/q}(M)} \lesssim R^{s_q - 1/q} \| f \|_{L^q(M)}. \]
    %
    Thus we conclude that
    %
    \[ \| (III_R \circ Q_R) f \|_{L^{q}(M)} \lesssim C_p(h) \| f \|_{L^{q}(M)}. \]
    %
    We have therefore bounded $III_R$ uniformly in $R$.
\end{proof}

Corollary 1.2 of \cite{LeeSeeger} establishes that the sharp local smoothing inequality holds for $p < 2(d-1)/(d+1)$, which covers the range of parameters studied in this paper. Thus we have obtained uniform bounds on the operators $\{ III_R \}$.

\begin{comment}

As we have already made notice, for $f \in V_R$, $Q_R f = f$. Thus proving Theorem \ref{RestrictedMainSphereTheorem} is equivalent to proving bounds of the form
%
\begin{equation} \label{MainEquationWithQ}
    \| (M_R \circ Q_R) f \|_{L^p(S^d)} \lesssim \| f \|_{L^p(S^d)}
\end{equation}
%
for $f \in V_R$. If we combine Lemma \ref{LargeTimeQBounds} with Lemma \ref{SmoothInputBounds}, we conclude that
%
\[ \| (M_R \circ Q_R^{\text{High}}) f \|_{L^p(S^d)} \lesssim_s R^s \| Q_R^{\text{High}} f \|_{L^p(S^d)} \lesssim_N R^{s-N} \| f \|_{L^p(S^d)}. \]
%
Taking $N \geq s$ yields
%
\begin{equation} \label{HighInequality}
    \| (M_R \circ Q_R^{\text{High}}) f \|_{L^p(S^d)} \lesssim \| f \|_{L^p(S^d)}.
\end{equation}
%
Lemma \ref{LargeTimeQBounds} (for $N = 0$) and the bounds $\| Q_R f \|_{L^p(S^d)} \lesssim \| f \|_{L^p(S^d)}$ imply via the triangle inequality that
%
\begin{equation} \label{LowUniformBounds}
    \| Q_R^{\text{Low}} f \|_{L^p(S^d)} \lesssim \| f \|_{L^p(S^d)}.
\end{equation}
%
Combining \eqref{LowUniformBounds} with Lemmas \ref{SmallTimeInequalityLemma} and \ref{LocalSmoothingLargeTimesTheorem} also yields bounds of the form
%
\begin{equation} \label{HighInequality}
    \| (I_R \circ Q_R^{\text{Low}}) f \|_{L^p(S^d)} \lesssim \| f \|_{L^p(S^d)} \ \ \text{and}\ \ \| (III_R \circ Q_R^{\text{Low}}) f \|_{L^p(S^d)} \lesssim \| f \|_{L^p(S^d)}.
\end{equation}
%
Combining \eqref{HighInequality} and \eqref{HighInequality}, we see that \eqref{MainEquationWithQ} will have been proved provided we can obtain a bound of the form
%
\begin{equation} \label{IIInequality}
    \| (II_R \circ Q_R^{\text{Low}}) f \|_{L^p(S^d)} \lesssim \| f \|_{L^p(S^d)}.
\end{equation}
%
We will prove this inequality in the next section, which will complete the proof. notice that $II_R$ and $Q_R^{\text{Low}}$ are both defined only in terms of the wave propogators $\{ e^{2 \pi i t P} \}$ for $|t| \leq 1/100$, which will enable us to be able to use the Lax-H\"{o}rmander parametrix for the half-wave equation in order to obtain a more robust understanding of the operator.

\end{comment}

\begin{comment}

\section{Notation For the Parametrix}

We now recall the Lax parametrix construction, which we will make heavy use of in the sequel in the analysis of the operators $\{ II_R \}$. For $|t| \lesssim 1$, we can write
%
\[ e^{2 \pi i t P} = T_{A,t} + \sum_{W \in \mathcal{W}} T_{W,t}, \]
%
where $T_{A,t}$ has kernel $A(t,\cdot,\cdot)$, and $T_{W,t}$ has kernel $W(t,\cdot,\cdot)$, with the following properties:
%
\begin{itemize}
    \item $A \in C^\infty(\RR \times M \times M)$.

    \item For each $W \in \mathcal{W}$, we can find a compact set $K_0 \subset \RR^d$, contained in an open subset $\Omega$ of $\RR^d$, together with a diffeomorphism $G: \Omega \to S^d$, such that $W$ is supported on $G(X_0) \times \RR \times G(X_0)$.

    \item For $f \in C^\infty(M)$, and $x \in X$, we can write
    %
    \[ (T_W f)(G(x)) = \int_\Omega \int_{\RR^d} a(x,y,\eta,t) e^{2 \pi i [ \varphi(x,y,\eta) + t p(y,\eta) ]} f(G(y))\; d\eta\; dy, \]
    %
    where:
    %
    \begin{itemize}
        \item The function $p$ is the principal symbol of the pseudodifferential operator $P$.

        \item The function $\varphi$ is smooth away from $\eta = 0$, homogeneous of degree one, and satisfies $\varphi(x,y,\eta) \approx (x - y) \cdot \eta$ in the sense that
        %
        \[ |\partial^\beta_\eta \{ \varphi(x,y,\eta) - (x - y) \cdot \eta \}| \lesssim_\beta |x - y|^2 |\eta|^{1 - \beta} \quad\text{for all $\beta$}. \]
        %
        We write $\phi(x,t,y,\eta) = \varphi(x,y,\eta) + t p(y,\eta)$.

        \item The amplitude function $a$ is a symbol of order zero with
        %
        \[ \text{supp}(a) \subset \{ (x,y,\eta,t) \in K_0 \times K_0 \times \RR^d \times \RR: |x - y| \lesssim 1\ \text{and}\ |\eta| \gtrsim 1 \}. \]
        %
        In particular, we may choose $\text{supp}_{x,y}(a)$ to be close enough to the diagonal so that one has
        %
        \[ |\nabla_\eta \varphi(x,y,\eta)| \gtrsim |x - y| \quad\text{and}\quad |\nabla_x \varphi(x,y,\eta)| \gtrsim |\eta| \]
        %
        for $(x,y) \in \text{supp}_{x,y}(a)$.
    \end{itemize}
\end{itemize}
%
Thus, modulo smoothing operators, locally solutions to the half-wave equation have expressions as oscillatory integrals.


























We now recall the Hadamard parametrix construction, which we will make heavy use of in the sequel. We recall that on a compact Riemannian manifold $M$, the natural identification of $\DD(M)$ at a subset of $\DD(M)^*$ is given by applying the bilinear pairing
%
\[ (\phi, \psi) = \int \phi(x) \psi(x) dV_g(x). \]
%
Consider the function $s: \RR_t \times T^* M \to \RR$ given by
%
\[ s(t,\xi_x) = \frac{1}{2\pi} \frac{\sin(2 \pi t |\xi|_g)}{|\xi|_g}. \]
%
Then the map $t \mapsto s(t,\cdot)$ is a smooth map from $\RR_t$ into $\mathcal{S}^1(T^* M)$. We can therefore defines a smooth map $t \mapsto E_+(t,\cdot)$ from $\RR_t$ into the family of all distributions of order $d$ on $M$ by considering the oscillatory integral distribution given by taking the weak limit of the
%
\[ \langle E_+(t,\cdot), \phi \rangle = H(t) \lim_{R \to \infty} \int \int \chi \left( \frac{|\xi|_g}{R} \right) \phi(x)  s(t,\xi_x) e^{2 \pi i \xi \cdot x}\; d\xi\; dx. \]


Here is the form of the parametrix we use here:
%
\begin{itemize}
    \item There exists a sequence of smooth functions $\{ \alpha_k : k \geq 0 \}$, lying in $C^\infty(M \times M)$, and a series of distributions $\{ E_\nu : k \geq 0 \}$, lying in $\DD(M \times M)$, such that for any $N \geq 0$,
    %
    \[ \int (\partial_t^2 - \Delta_x) \left\{ \sum_{\nu \geq 0} a_\nu E_\nu \right\} = \delta(  ) \]
\end{itemize}


Normally, this parametrix is written, modulo elements of $C^\infty(M \times M)$, by a distribution $E(x,y,t)$ given by an integral in normal coordinates about the point $y$ by an equation of the form
%
\[ E(t,x,y) = \int_{T^*_y M} a(x,y, |\xi|_g ) e^{2 \pi i [ \xi \cdot (x - y) + t |\xi|_g ]}\; d\xi, \]
%
where $a$ is a classical symbol of order zero, which vanishes for $|\xi|_g \leq 100$. Note that because we are using normal coordinates, $|x - y| = d_g(x,y)$.
%by a $C^\infty(M \times M)$ linear combination of the distributions $E_k(x,y,t)$, given by testing against a function $u \in \DD(M \times M)$ by setting
%
%\[ \int \chi(\xi) \]
%\[ \int E_k(x,y,t) u(x,y)\; dy = \int_{T^* M} e^{2 \pi i [\xi \cdot (x - y) + t |\xi|_g]} |\xi|_g^{-k} u(x,y)\; d\xi\; dy. \]
%
Applying polar coordinates means we can write the distribution as
%
\[ \int_0^\infty a(x,y,\tau) B_d(\tau d_g(x,y)) e^{2 \pi i t \tau}\; d\tau, \]
%
where $a$ is now a classical symbol of order $d-1$ vanishing for $|\tau| \leq 1$.

Consider the distribution
%
\[ \int_0^\infty a(x,y,\tau) \chi(\tau d_g(x,y)) e^{2 \pi i t \tau}\; d\tau, \]
%
where $\chi$ is smooth and $\chi(\rho)$ vanishes for $|\rho| \geq 1$. Because $a$ vanishes for $|\tau| \leq 1$, the support of this distribution is thus contained in $\{ (x,y): d_g(x,y) \leq 1/100 \}$. Taking inverse Fourier transforms shows this quantity is also equal to
%
\[ d_g(x,y)^{-1} \int \widehat{a}(x,y, t - s ) \widehat{\chi} \left( \frac{s}{d_g(x,y)} \right)\; ds. \]
%
\[ 2^k  \int \eta( s ) \widehat{a}(x,y, t - d_g(x,y) 2^k s) \widehat{\chi}(2^k s)\; ds \]

Now
%
\[ \int_{T^* M} e^{2 \pi i [\xi \cdot (x - y) + t |\xi|_g]} |\xi|_g^{-k} = \int_M \int_0^\infty \rho^{d-1-k} B_d(\rho d_g(x,y)) e^{2 \pi i t \rho}\; d\rho. \]
%
These all differ, modulo $C^\infty(M \times M)$, from a distribution of the form
%
\[ \int_0^\infty \chi(\rho) \rho^{d-1-k} B_d(\rho d_g(x,y)) e^{2 \pi i t \rho} \; d\rho, \]
%
where $\text{supp}(\chi) \subset \{ |\rho| \geq 100 \}$, and equal to one on $\{ |\rho| \geq 2 \}$. Let us consider a distribution of the form
%
\[ \int_0^\infty \chi(\rho) \tilde{\chi}(\rho d_g(x,y)) \rho^{d-1-k} e^{2 \pi i t \rho}\; d\rho = \int_0^\infty \chi(\rho) \tilde{\chi}(\rho d_g(x,y)) \langle \rho \rangle^{d-1-k} e^{2 \pi i t \rho}\; d\rho, \]
%
where $\tilde{\chi}$ is compactly supported on $|\rho| \leq 1$, smooth, and equal to one on $|\rho| \leq 1/2$. In order for this quantity to be non-zero, we need $d_g(x,y) \leq 1 / 100$. This integral ranges over the region $[100, d_g(x,y)^{-1}]$. Let us understand this quantity via an integration by parts. The $N$th derivative of $\chi(\rho) \tilde{\chi}(\rho d_g(x,y) \langle \rho \rangle^{d-1-k}$ is a linear combination of terms of the form
%
\[ f_{N_1,N_2,N_3}(x,y,\rho) = D^{N_1} \chi(\rho) d_g(x,y)^{N_2} D^{N_2} \tilde{\chi}(\rho d_g(x,y)) \langle \rho \rangle^{d-1-k-N_3} \]
%
where $N_1 + N_2 + N_3 = N$. If $N_1 > 0$, then the term is supported on $1 \leq |\rho| \leq 2$, and we conclude that
%
\[ \| D^\alpha_x D^\beta_y f_{N_1,N_2,N_3} \|_{L^1_\rho} \lesssim_{\alpha,\beta} 1. \]
%
If $N_1 = 0$, but $N_2 > 0$, then the term is supported on $1 / 2 \leq \rho d_g(x,y) \leq 1$, and has magnitude $\lesssim d_g(x,y)^{N_2} d_g(x,y)^{N_3 + k + 1 - d}$, and so has $L^1$ norm $d_g(x,y)^{N + k - d}$. Finally, if $N_1 = N_2 = 0$, then the term has $L^1$ norm $O(1)$ if $N \geq k - d + 1$, and otherwise $d_g(x,y)^{N + k - d}$.

Then we can find a symbol $a$ of order $-d$, a smooth, rapidly decaying function $b$, and a symbol $c$ of order $k + 1$ such that the distribution is written as
%
\[ \int a(t - d_g(x,y) r - s) b(r) c(s)\; dr\; ds. \]


If $b_1$ and $b_2$ are the Fourier transforms of $\chi$ and $\tilde{\chi}$, and
%
\[ b(r,x,y) = \int b_1(r - d_g(x,y) s) b_2(s)\; ds \]


Then when $\chi(\rho d_g(x,y)) \neq 0$, $\chi(\rho) = 1$, so we can write the distribution as
%
\[ \int_0^\infty \chi(\rho d_g(x,y)) \rho^{d-1-k} e^{2 \pi i t \rho}\; d\rho, \]
%
which we can compute as
%
\[ d_g(x,y)^{k-d} \int_0^\infty \chi(\rho) \rho^{d-1-k} e^{2 \pi i (t / d_g(x,y)) \rho}\; d\rho. \]
% For k <= d-1
For $k \leq d-1$, this becomes a constant multiple of
%
\[ (D^{d-1-k} \widehat{\chi})( t / d_g(x,y) ). \]
%
For $k > d-1$,

% rho <= 1 / d_g(x,y)



%
\[ E_{j,k}(x,y) = \int_{T^*_y M} e^{2 \pi i [\xi \cdot x + t |\xi|_g]} |\xi|_g^{ - 1 - k}\; d\xi \]

%\begin{align*}
%    \int a(\xi) e^{2 \pi i \xi \cdot x \pm it |\xi|}\; d\xi &= \int_0^\infty a(\rho) \rho^d B_d(\rho \cdot d_g(x,y)) e^{\pm 2 \pi i \rho t}\; d\rho.
%\end{align*}

\begin{theorem}
    Let $M$ be a compact manifold, and let $P$ be a classical, self-adjoint elliptic pseudodifferential operator of order one on $M$. Then there exists a time $T_{\text{max}} > 0$, such that for $t$ lying in the interval $I_{\text{max}} = [0, T_{\text{max}}]$, we can write
    %
    \[ \cos(2 \pi P t) = \sum_{\sigma \in \{ -, + \}^2 } L_{t,\sigma} + A_t, \]
    %
    where the right hand side is a sum of operators satisfying the following properties:
    %
    \begin{itemize}
        \item For each $\sigma$, there exists a classical symbol
        %
        \[ a_\sigma: M_x \times M_y \times I \times [0,\infty)_\tau \to \CC, \]
        %
        of order $d$, with
        %
        \[ \text{supp}_\tau \{ a_\sigma \} \subset \{ \tau \gtrsim 1 \}, \]
        %
        and a classical symbol
        %
        \[ s_\sigma: [0,\infty) \to \CC, \]
        %
        of order $- \frac{d-1}{2}$, such that
        %
        \[ L_{t,\sigma} f(x) = \int_0^\infty a_\sigma(x,y,t,\tau) s_\sigma(\tau d_g(x,y)) e^{2 \pi i \tau \phi_\sigma(t,d_g(x,y))} f(y)\; d\tau\; dy, \]
        %
        where $\phi_\sigma(t,d_g(x,y)) = \sigma_1 t + \sigma_2 d_g(x,y)$.

        \item The kernel of $A_t$ is given by a smooth function $M \times M \times I_{\text{max}} \to \CC$.
    \end{itemize}
\end{theorem}

%Firstly, let's discuss the properties of the operators $\{ L_{t,\kappa} \}$:
    %
%    \begin{itemize}
%        \item For each $\kappa \in \mathcal{K}$, there exists an open subset $U_\kappa \subset M$, such that the kernel of $L_{t,\kappa}$ is supported on a compact subset of $U_\kappa \times U_\kappa$.

%        \item There exists a bounded open subset $\tilde{U}^\kappa \subset \RR^d$, and a diffeomorphism $G_\kappa: \tilde{U}^\kappa \to U^\kappa$, together with a symbol $a_\kappa: \tilde{U}^\kappa_x \times \tilde{U}^\kappa_y \times \RR^d_\xi \to \CC$ of order zero with $\text{supp}_{x,y}(a_\kappa)$ forming a compact subset of $\tilde{U}^\kappa \times \tilde{U}^\kappa$, and with $\text{supp}_\xi(a_\kappa)$ contained in $\{ |\xi| \geq 1 \}$, such that for $f \in \DD(M)$,
        %
%        \[ L_{t,\kappa} f(G(x)) = \text{sgn}(t) \int E_\nu(|t|,d_g(x,y)) f(G(y))\; dy \]
%    \end{itemize}

\end{comment}





\section{Analysis of $II_R$: Density Decompositions}

%Accounting for Lemmas \ref{SmallTimeInequalityLemma} and \ref{LocalSmoothingLargeTimesTheorem}, it

It finally remains to analyze the operator $Q_R \circ II_R \circ Q_R$, where
%
\[ II_R = \int \chi_{II}(t) R \widehat{h}(Rt) e^{2 \pi i t P}\; dt \] % \quad\text{and} \quad Q_R^{\text{Low}} = \int c_R(t) e^{2 \pi i t P}\; dt, \]
%
is obtained by integrating the wave propogators over times $100/R \leq |t| \leq 0.01$ respectively. To prevent notation from growing too cumbersome later on, let us eschew uses of the subscript $R$ in our operators in this section, e.g. writing $II_R$ as
%
\[ II = \int b(t) e^{2 \pi i t P}\; dt, \]
%
where $b(t) = \chi_{II}(t) R \widehat{h}(Rt)$. We then have
%
\[ \| b(t) \langle t \rangle^{s_p} \|_{L^p(\RR)} \lesssim R^{1 - 1/p - s_p} C_p(h). \]
%
Bounding $II$ requires a more subtle analysis of the geometric behaviour of the wave-propogator operators, and we will begin by converting our problem in coordinates on $S^d$, where the kernels have more explicit representations in oscillatory integrals.
\begin{comment}

To prevent notation from growing too cumbersome later on, let us eschew uses of the subscript $R$ in our operators for now, e.g. writing the operator $II_R$ as
%
\[ II = \int b(t) e^{2 \pi i (t/R) P}\; dt. \]


It will be more natural to rescale this operator, writing
%
\[ II_R = \int b_R(t) e^{2 \pi i (t/R) P}\; dt \]
%
with
%
\[ b_R(t) = \chi_{II}(t / R) \widehat{h}(t) \]
%
supported on $|t| \leq R / 100$. Our assumptions on $h$ give that
%%
%We must prove bounds of the form
%
%\begin{equation} \label{IIRVRBound}
%    \| II_R f \|_{L^p(S^d)} \lesssim C_p(h) \| f \|_{L^p(S^d)} \quad\text{for $f \in V_R$}.
%\end{equation}
%
%This will require a more detailed analysis than was required for the operators $\{ I_R \}$ and $\{ III_R \}$, involving computations of Fourier integrals in various coordinate systems. In this setting, it is difficult to directly exploit the fact that we are restricted our inputs to the space $V_R$. We therefore switch to an inequality equivalent to \eqref{IIRVRBound}, namely, that
%
%\begin{equation} \label{IIRBound}
%    \| (II_R \circ Q_R) f \|_{L^p(S^d)} \lesssim C_p(h) \| f \|_{L^p(S^d)} \quad\text{for $f \in L^p(S^d)$}.
%\end{equation}
%
%This is because we can do more explicit computations with the operators $\{ Q_R \}$ and their kernels than we can with general elements of $V_R$.
To prevent notation from growing too cumbersome later on, let us eschew uses of the subscript $R$ in our operators for now, e.g. writing the operator $II_R$ as
%
\[ II = \int b(t) e^{2 \pi i (t/R) P}\; dt. \]
%
Essentially, we know the following facts about the function $b$:
%
\begin{itemize}
    \item We have $\| t^{s_p} b_R \|_{L^p(\RR)} \leq C_p(h)$.

    % b_R(t) = chi(t) R h^(Rt)
    % R^{-1} | <t>^{s_p} b_R(t/R) |_{L^p_t}
    % | b_{R,0} |_{L^p} << R^{1 - 1/p}
    \item We have
    %
    \[ \widehat{b}(\tau) = \int [R\widehat{\chi}_{II}(R \omega)] \cdot h \left( \tau - \omega \right)\; d\omega, \]
    %
    so the Fourier support of $b$ remains concentrated on $\{ |\tau| \sim 1 \}$, i.e. for $|\tau| \leq 1/4$ we have
    %
    \[ |\partial_\tau^\alpha \big\{ \widehat{b} \big\}(\tau)| \lesssim_{\alpha,N} \| h \|_{L^\infty(\RR)}\; R^{-N}, \]
    %
    for $1/4 \leq |\tau| \leq 4$, we have
    %
    \[ \left| \partial_\tau^\alpha \big\{ \widehat{b} \big\}(\tau) \right| \lesssim \| h \|_{L^\infty(\RR)}, \]
    %
    and for $|\tau| \geq 4R$, we have
    %
    \[ \Big| \partial_\tau^\alpha \big\{ \widehat{b} \big\} (\tau) \Big| \lesssim_{\alpha,N} \| h \|_{L^\infty(\RR)}\; \tau^{-N}. \]
    %
    Uncertainty principle heuristics thus imply that we should expect $b$ to be locally constant at a scale $\sim 1$. If we let $\mathcal{T}$ be all points in the lattice $\ZZ / 10$ lying in the set $\{ |t| \leq R / 99 \}$, and for each $t_0 \in \mathcal{T}$, write
    %
    \[ \chi_{II} = \sum_{t_0 \in \mathcal{T}} \chi_{t_0}, \]
    %
    where for each $t_0 \in \mathcal{T}$, $\chi_{t_0}$ is smooth and adapted to the sidelength $1$ interval centered at $t_0$, i.e. satisfying estimates of the form
    %
    \[ | (\partial_t^\alpha \chi_{t_0})(t) | \lesssim_\alpha 1. \]
    %
    We can thus decompose $b = \sum b_{t_0}$, where
    %
    \[ b_{t_0}(t) = \chi_{t_0}(t) b(t). \]
    %
    For such a decomposition, the functions $\{ b_{t_0} \}$ satisfy
    %
    \[ | \partial_\tau^\alpha \{ \widehat{b}_{t_0} \}(\tau) | \lesssim \| h \|_{L^\infty} \langle \tau \rangle^{-N}. \]
    %
    Thus we should still think of $\widehat{b}_{t_0}$ as being Fourier localized to frequencies lying in the ball of radius $O(1)$ about the origin, though the localization has introduced non-negligible frequencies near the origin, which is to be expected given the heuristic constancy of the function $b$ at the scale we are localizing.
\end{itemize}

\end{comment}

We will employ some restricted weak type bounds, together with interpolation, to obtain $L^p$ estimates on the operators $Q \circ II \circ Q$. We thus introduce a set $E \subset S^d$ and try to obtain $L^{p,\infty}$ bounds on the function $S = (Q \circ II_W \circ Q) \{ E \}$. Given that $Q$ already acts, heuristically, by localizing the behaviour of it's inputs to the frequency $R$, despite the choice of the set $E$, the uncertainty principle implies $Q \{ E \}$ should be locally constant at a scale $1/R$, and so it is natural to discretize at this scale. Consider a maximal $1/2R$ separated subset $\mathcal{X}$ of $S^d$. Then break $E$ down into a disjoint union of sets $\{ E_{x_0}: x_0 \in \mathcal{X} \}$, where for $x_0 \in \mathcal{X}$, the set $E_{x_0}$ is supported on the geodesic ball of radius $1/R$ about $x_0$. Similarily, let $\mathcal{T}$ be all points in the lattice $\ZZ / 10 R$ lying in the set $\{ 100/R \leq |t| \leq 1 \}$, and write
%
\[ b = \sum_{t \in \mathcal{T}} u(t) b_t, \]
%
where for each $t \in \mathcal{T}$, $u(t) = \| b \|_{L^\infty[t - 10/R, t + 10/R]}$, and $b_t$ is a smooth function, compactly supported on the sidelength $1/R$ interval centered at $t$, satisfying
%
\[ |\partial^\alpha b_t| \lesssim_\alpha R^{|\alpha|}, \]
%
with implicit constants uniform in $b$ and $t$. By the Plancherel-Polya theorem, 
%
\[ \| u(t) \langle t \rangle^{s_p} \|_{l^p(\mathcal{T})} \lesssim R^{1 - s_p}. \]
%\[ \left( \sum_t \left( \| b_t \|_{L^p(\RR)} t^{s_p} \right)^p \right)^{1/p} \lesssim R^{-s_p} C_p(h). \]
%\[ \left( \sum_t \left( \left( |u_t| t^{s_p} \right)^p \right) \right)^{1/p} \lesssim R^{1/p-s_p} C_p(h) \]
%
%
% int | R h^(Rt)|^p <Rt>^{ps_p}
% |u_t|^p R^{-1} (Rt)^{ps_p} = int_{t' - 1/R}^{t' + 1/R} (Rt)^{ps_p} |R h^(Rt)|^p
% ( Sum ( |u_t| t^{s_p} )^p R^{(d-1)(1 - p/2) - 1} )^{1/p}
%
We can then write
%
\[ S = \sum |E_{x_0}| {S\!}_{x_0,t_0} \quad\text{where}\quad {S\!}_{x_0,t_0} = \int |E_{x_0}|^{-1} b_{t_0}(t) (Q \circ e^{2 \pi i t P} \circ Q) \{ E_{x_0} \}\; dt. \]
%
%For simplicity, let us assume that if
%
%\[ \mathcal{X}_0 = \{ x_0: E_{x_0} \neq \emptyset \}, \]
%
%then $|E_{x_0}| \sim 2^{-l} R^{-d}$ for all $x_0 \in \mathcal{X}_0$. 
Our computation would be complete if we could show that for any coefficients $\{ c(x_0,t_0) : x_0 \in \mathcal{X}, t_0 \in \mathcal{T} \}$,
%
\[ \left\| \sum_{x_0,t_0} c(x_0,t_0) t_0^{\frac{d-1}{2}} {S\!}_{x_0,t_0} \right\|_{L^p(S^d)} \lesssim R^{s_p - 1 + d(1 - 1/p)} \left( \sum_{x_0,t_0} |c(x_0,t_0)|^p t_0^{d-1} \right)^{1/p}. \]
%
Indeed, we set $c(x_0,t_0) = |E_{x_0}| u(t_0) t_0^{- \frac{d-1}{2}}$ and apply H\"{o}lder's inequality, then the inequality above gives exactly that
% R^{1 - s_p}
\[ \| S \|_{L^p(S^d)} \lesssim C_p(h) |E|^{1/p}, \]
%
For $p = 1$, this follows from applying the triangle inequality, and using the pointwise estimates
%
\[ |{S\!}_{x_0,t_0}(x)| \lesssim_M \frac{R^{d-1}}{(R d_g(x,x_0))^{\frac{d-1}{2}}} \Big\langle R \big| t_0 - d_g(x,x_0) \big| \Big\rangle^{-M}. \]
%
Applying interpolation, for $p > 1$ we need only prove a restricted weak type version of this inequality. In other words, we can restrict $c$ to be the indicator function of a set $\mathcal{E}$, and take $L^{p,\infty}$ norms on the left hand side. If we write $\mathcal{E} = \bigcup_k \mathcal{E}_k$, where $\mathcal{E}_k$ is the set of $(x,t) \in \mathcal{E}$ with $|t| \sim 2^k / R$, then the inequality reads that
%
\[ \left\| \sum_{k = 1}^\infty 2^{k \frac{d-1}{2}} \sum_{(x_0,t_0) \in \mathcal{E}_k} {S\!}_{x_0,t_0} \right\|_{L^{p,\infty}(S^d)}^p \lesssim R^{(d-1)p - d} \left( \sum_{k = 1}^\infty 2^{k(d-1)} \# \mathcal{E}_k \right). \]%
% When p = 1 the inequality says
% Sum_k 2^{k (d-1)/2} |S_{x_0,t_0}|_{L^1} << R^{-1} sum_k 2^{k(d-1)} #(E_k)
%
% d_g(x,x_0) ~ 2^k / R
%
% Has height R^{d-1} 2^{-k (d-1)/2}
%
% on an annulus of thickness 1/R, and radius 2^k/R
%
% So has L^1 norm R^{-1} 2^{k(d-1)/2}
%
% SO WE DO GET THE p = 1 INEQUALITY!
%
% So for weak type estimate, by interpolation,
% we only have to consider large lambda superlevel sets
%
% LOW DENSITY PART: DEALT WITH USING L2 ESTIMATE
% HIGH DENSITY PART: Since lambda is large,
%           can use essential support of function.
%
This is equivalent to showing that for any $\lambda > 0$,
%
\[ \left| \left\{ x: \left|\sum_k 2^{k \frac{d-1}{2}} {S\!}_{x_0,t_0}(x)\right| \geq \lambda \right\} \right| \lesssim \lambda^{-p} R^{(d-1)p - d} \sum_k 2^{k(d-1)} \# \mathcal{E}_k. \]
%
The case $\lambda \lesssim R^{d-1}$ follows from the $L^1$ boundedness we've already proved, so we may assume $\lambda \gtrsim R^{d-1}$ in the sequel.

To obtain this bound, we employ the method of density decompositions, introduced in \cite{HeoandNazarovandSeeger}. Let
%
\[ A = \left( \frac{\lambda}{R^{d-1}} \right)^{(d-1)(1 - p/2)} \log \left( \frac{\lambda}{R^{d-1}} \right)^{O(1)}. \]
%
Then for each $k$, consider the collection $\mathcal{B}_k(\lambda)$ of all balls $B$ with radius at most $2^k / R$ such that $\# \mathcal{E}_k \cap B \geq R A \text{rad}(B)$. Applying the Vitali covering lemma, we can find a disjoint family of balls $\{ B_1, \dots, B_N \}$ in $\mathcal{B}_k$ such that the balls $\{ B_1^*, \dots, B_N^* \}$ obtained by dilating the balls by 5 cover $\bigcup \mathcal{B}_k(\lambda)$. Then
%
\[ \sum \text{rad}(B_j) \leq R^{-1} A^{-1} \# \mathcal{E}_k, \]
%
and the set $\widehat{\mathcal{E}}_k = \mathcal{E}_k - \bigcup \mathcal{B}_k(\lambda)$ has density type $(R A, 2^k / R)$. Then we conclude that, using the quasi-orthogonality estimates below,
%
\[ \left\| \sum_k \sum_{(x_0,t_0) \in \widehat{\mathcal{E}}_k} 2^{k \frac{d-1}{2}} {S\!}_{x_0,t_0} \right\|_{L^2(S^d)}^2 \lesssim_p R^{d-2} \log(A) A^{\frac{2}{d-1}} \sum_k 2^{k(d-1)} \# \mathcal{E}_k. \]
%
Appling Chebyshev's inequality, and utilizing the choice of $A$ above, we conclude that
%
\begin{align*}
    \left| \left\{ x: \left|\sum_k \sum_{(x_0,t_0) \in \widehat{\mathcal{E}}_k} 2^{k \frac{d-1}{2}} {S\!}_{x_0,t_0}(x)\right| \geq \lambda / 2 \right\} \right| &\lesssim R^{d-2} \log(A) A^{\frac{2}{d-1}} \sum_k 2^{k(d-1)} \# \mathcal{E}_k\\
    &\lesssim \lambda^{-p} R^{(d-1)p - d} \sum_k 2^{k(d-1)} \# \mathcal{E}_k.
\end{align*}
% 
Conversely, we exploit the clustering of the sets $\mathcal{E}_k - \widehat{\mathcal{E}}_k$ to bound
%
\[ \left| \left\{ x: \left|\sum_k \sum_{(x_0,t_0) \in \mathcal{E}_k - \widehat{\mathcal{E}}_k} 2^{k \frac{d-1}{2}} {S\!}_{x_0,t_0}(x)\right| \geq \lambda / 2 \right\} \right| \]
%
That is, we have found balls $B_1^*< \dots, B_N^*$, each with radius $O(2^k / R)$, such that
%
\[ \sum \text{rad}(B_j) \leq R^{-1} A^{-1} \# \mathcal{E}_k. \]
%
Let $(x_j,t_j)$ denote the center of the ball $B_j$. Then the function
%
\[ \sum_{(x_0,t_0) \in B_j} {S\!}_{x_0,t_0} \]
%
has mass concentrated on the geodesic annulus $\text{Ann}_j \subset S^d$ with radius $t_j$ and thickness $O( \text{rad}(B_j) )$, a set with measure $(2^k / R)^{d-1} \text{rad}(B_j)$. For $(x_0,t_0) \in B_j$, we calculate using the pointwise bounds that
%
\begin{align*}
    \int_{\text{Ann}_j^c} |{S\!}_{x_0,t_0}(x)|\; dx &\lesssim R^{d-1} \int_{\text{rad}(B_j) \lesssim |t_j - d_g(x,x_0)| \leq 1} \langle R |t_0 - d_g(x,x_0)| \rangle^{-M}\\
    &\lesssim R^{d-1} \int_{\text{rad}(B_j) \leq |t_j - s| \leq 1} s^{d-1} \langle R |t_0 - s| \rangle^{-M}\; ds\\
    &\lesssim 2^{k(d-1)} R^{d-1} ( R \text{rad}(B_j) )^{-M}.
\end{align*}
%
Because the set of points in $\mathcal{E}_k$ is $1/R$ separated, there can only be at most $O(R \text{rad}(B_j))^{d+1}$ values of $(x_0,t_0)$, and so applying the triangle inequality gives that the sum of the $L^1$ norm outside of $\text{Ann}_j$ is
%
\[ \lesssim 2^{k(d-1)}R^{d-1} ( R\; \text{rad}(B_j) )^{d+1-M} \]
%
Note that since $\# \mathcal{E}_k \cap B_j \geq R A \text{rad}(B_j)$, and because $\mathcal{E}_k$ is $1/R$ discretized,
%
\[ \text{rad}(B_j) \geq (A / R)^{\frac{1}{d-1}}, \]
%
and this, together with Markov's inequality, is enough to justify the required bound. Conversely, since $1 < p < 2(d-1)/(d+1)$, we have
%
\begin{align*}
    \sum |\text{Ann}_j| \lesssim (2^k / R)^{d-1} \sum_j \text{rad}(B_j)\\
    &\lesssim (2^k / R)^{d-1} R^{-1} (L / R^{d-1} )^{-(d-1)(1 - p/2)} log(L / R^{d-1})^{O(1)}\\
    &\lesssim \lambda^{-p} R^{(d-1)p - d} 2^{k(d-1)} \# \mathcal{E}_k,
\end{align*}
%
Summing over $k$ completes the analysis.
% The sum of S_{x_0,t_0}, where (x_0,t_0) range over some ball B_j
% has measure concentrated on a radius ~ 2^k / R annulus with thickness rad(B_j), a set with measure (2^k / R)^{d-1} rad(B_j).

%
% Picking A = (L / R^{d-1})^{ps_p}
% gives slightly too large a bound,
% by a factor log(L / R^{d-1})
%
% If we can replace ps_p by ps_p - epsilon
% we're probably good.
%
% Or even (L / R^{d-1})^{ps_p} log ( L / R^{d-1} )^{-O(1)}
%
% Then
%
% We're left with dealing with the concentrated case
% points are covered by balls B_1 ... B_N with radius at most 2^k / R
% and with sum rad(B_i) <= R^{-1} A^{-1} #E_k
%
% <= R^{-1} (L / R^{d-1})^{-(d-1)(1 - p/2)} log(L / R^{d-1})^{O(1)}
%
% The sum of S_{x_0,t_0}, where (x_0,t_0) range over some ball B_j
% has measure concentrated on a radius ~ 2^k / R annulus with thickness rad(B_j), a set with measure (2^k / R)^{d-1} rad(B_j).
%
% Summing over j, the total measure of these annulus are
%
% << (2^k / R)^{d-1} R^{-1} (L / R^{d-1} )^{-(d-1)(1 - p/2)} log(L / R^{d-1})^{O(1)}
% 2^{k(d-1)} R^{-d + (d-1)p} L^{-p} #(E_k)
%

% This is good provided that
% R^{d-2} L^{2p/(d-1) - 2} <= L^{-p} R^{(d-1)p - d}
% R^{(2-p)(d-1)} <= L^{2-p - 2p/(d-1)}
% R^{(2-p)(d-1)} <= L^{2 - p( (d+1)/(d-1) )}
% L >= R^{(d-1) [ (2-p)/(2 - p(d+1)/(d-1))]} = R^{(d-1) + eps_p}
% The right exponent exceeds d-1 in the range we are considering
% Shouldn't this be bad?

% s(lambda))^{2p/(d-1)} <= lambda^{2-p} R^{-(2-p)(d-1)}
% s(lambda) <= lambda^{ s_p } R^{ - (d-1)s_p } = ( lambda / R^{d-1} )^{s_p}
% p = 2 (d-1)/ (d+1)

% (2 - p)(d-1) / 2p > 1 in the range we are considering

% log_2(s(lambda)) s(lambda) <= lambda^{(2-p)(d-1)/2p} R^{[(d-1)^2(p-1) - d(d-1)]/2p}


% \log_2(\lambda) lambda^{(d+1)/(d-1)[p - 2(d-1)/(d+1)]} <= R^{2 + (d-1)p - 2d}
% if
% R^{(2-p)(d-1)} <= lambda^{2 - p(d+1)/(d-1)}
% which holds iff
% lambda >= R^{(2-p)(d-1)/[2 - p(d+1)/(d-1)]}
% L^{(2p)/(d-1) - 2 + p} <= R^{(d-1)p - d + 2 - d} 

% d > (p+2)/(2-p)
% 

\section{Analysis of $II_R$: Quasi-Orthogonality}

Our first goal will be to understand how orthogonal the functions $\{ {S\!}_{x_0,t_0} \}$ are to one another, which will give $L^2$ estimates for $S$, that can be interpolated with $L^1$ estimates to obtain the required $L^p$ estimates. The rest of this section will be devoted to proving the following inner product estimate, which, together with a density decomposition argument, introduced in \cite{HeoandNazarovandSeeger}, can be used to obtain $L^2$ estimates, which we can then interpolate to obtain $L^p$ estimates for the function $S$.

\begin{lemma} \label{mainOrthogonalityLemma}
    %
    \[ \left| \langle {S\!}_{x_0,t_0}, {S\!}_{x_1,t_1} \rangle \right| \lesssim_M \frac{R^{d-2}}{( R d_g(x_0,x_1) )^{\frac{d-1}{2}}} \sum_{\pm} \Big\langle R \big| (t_0 - t_1) \pm d_g(x_1,x_0) \big| \Big\rangle^{-M}. \]
\end{lemma}

\begin{comment}

We now employ the \emph{Lax Parametrix} construction, writing, for $|t| \leq 0.01$,
%
\[ e^{2 \pi i t P} = W(t) + A(t). \]
%
Here $A(t)$ is a Schwartz operator, whose kernel $A(t,x,y)$ is a smooth function of $t$, $x$, and $y$, and $W(t)$ is a Schwartz operator, with kernel satisfying
%
\[ \text{supp}(W) \subset \{ (t,x,y) : d(x,y) \leq t + 0.01 \}, \]
%
and such that, if we consider Schwartz operators $\{ W_\alpha(t) \}$ with kernel
%
\[ W_\alpha(t,x,y) = \tilde{\eta}_\alpha(x) W(t,x,y) \tilde{\eta}_\alpha(y), \]
%
then $W(t) = \sum W_\alpha(t)$, and each operator $W_\alpha(t)$ can, in the coordinate system $U_\alpha$, be given as a Fourier integral with canonical relation supported in the geodesic light cone, i.e. for each $\alpha$, there exists a symbol $s$ of order zero, and a phase function $\Phi$ such that the pushforward operator $W_\alpha^{z_\alpha}(t)$ has kernel
%
\[ W_\alpha^{z_\alpha}(t,x,y) = \int s(t,x,y,\xi) e^{2 \pi i \Phi(t,x,y,\xi)}\; d\xi, \]
%
where $\Phi(t,x,y,\xi) = \phi(x,y,\xi) + t |\xi_y|_g$, and where $\exp$ denotes the geodesic flow in cotangent space. Let us use this decomposition to write $II = II_W + II_A$, where
%
\[ II_W = \int b(t) W(t)\; dt \quad\text{and}\quad II_A = \int b(t) A(t)\; dt. \]
%
%Similarily, we can write the operator $Q^{\text{Low}}$ as $Q^{\text{Low}}_W + Q^{\text{Low}}_A$, where
%
%\[ Q^{\text{Low}}_W = \int c(t) W(t)\; dt \quad\text{and}\quad Q^{\text{Low}}_A = \int c(t) A(t)\; dt. \]
%
The kernel of $II_A$ is non-singular, which makes it simple to deal with.

\begin{lemma} \label{IIALemma}
    Fix $1 \leq p \leq 2d/(d+1)$. Then for any $1 \leq q \leq \infty$,
    %
    \[ \| II_A f \|_{L^q(M)} \lesssim C_p(h) \| f \|_{L^q(M)} \quad\text{for $f \in C^\infty(M)$,} \]
    %
    where the implicit constant is uniform in $R$. From the uniform boundedness of $\{ Q_R \}$, it immediately follows that
    %
    \[ \| (II_A \circ Q_R) f \|_{L^q(M)} \lesssim C_p(h) \| f \|_{L^q(M)}. \]
\end{lemma}
\begin{proof}
    Let $K$ denote the kernel of $II_A$, i.e. so we have
    %
    \[ K(x,y) = \int b(t) A(t,x,y)\; dt. \]
    %
    Then H\"{o}lder's inequality, combined with the fact that $b$ is supported on the interval $I = \{ 50/R \leq |t| \leq 2/5 \}$ implies that
    %
    \begin{align*}
        \| K \|_{L^\infty_x(S^d) L^\infty_y(S^d)} &\lesssim \| b (R t)^{s_p} \|_{L^p_t(I)} \| A \cdot (R t)^{-s_p} \|_{L^{p^*}_t(I)}\\
        &\lesssim C_p(h) R^{1/p^* - s_p}\\
        &\lesssim C_p(h),
    \end{align*}
    %
    where we used the fact that $1/p^* - s_p < 0$ for $1 \leq p < 2d/(d+1)$, i.e. the range of the radial multiplier conjecture. But since $S^d$ has finite measure, we can now apply Schur's Lemma to conclude that $II_A$ is bounded from $L^1(M)$ to $L^1(M)$ and $L^\infty(M)$ to $L^\infty(M)$, with an operator norm bounded by a constant multiple of $C_p(h)$. Interpolation completes the proof.
\end{proof}

\end{comment}

\begin{comment}

%The bounds $\| Q f \|_{L^p(S^d)} \lesssim \| f \|_{L^p(S^d)}$ and $\| Q^{\text{High}} f \|_{L^p(S^d)} \lesssim \| f \|_{L^p(S^d)}$ imply that $\| Q^{\text{Low}} f \|_{L^p(S^d)} \lesssim \| f \|_{L^p(S^d)}$, and so we conclude that
%%\[ \| (II_A \circ Q^{\text{Low}}) f \|_{L^p(S^d)} \lesssim \| f \|_{L^p(S^d)}. \]
%
%The next lemma gives a bound on $II_W \circ Q^{\text{Low}}_A$, so that we have reduced ourselves to the study of $II_W \circ Q^{\text{Low}}_W$.

\begin{lemma}
    We have
    %
    \[ \| (II_W \circ Q^{\text{Low}}_A) f \|_{L^{p^*}(M)} \lesssim C_p(h) \| f \|_{L^{p^*}(M)}, \]
    %
    where the implicit constant is uniform in $R$.
\end{lemma}
\begin{proof}
    The smoothness of $A$ implies that the function
    %
    \[ Z(t,s,x,y) = \iint W(t,x,z) A(s,z,y)\; dz \]
    %
    is smooth (a distribution applied to a smooth function is smooth). We can then write the kernel of $II_W \circ Q^{\text{Low}}_A$ as
    %
    \[ K(x,y) = \iint b(t) c(s) Z(t,s,x,y)\; dt\; ds. \]
    %
    As in Lemma \ref{IIALemma}, applying H\"{o}lder's inequality to separate out the $b$ variable from the other variables gives that for $1 \leq p < 2d/(d+1)$,
    %
    \[ \| K \|_{L^\infty_x(S^d) L^\infty_y(S^d)} \lesssim C_p(h) R^{1/p^* - s_p} \lesssim C_p(h), \]
    %
    and we can again apply Schur's Lemma.
\end{proof}

\end{comment}

\begin{comment}
It is natural to break this operator over dyadic multiples of $1/R$. Thus we write
%
\[ II = \sum_{1 \lesssim k \lesssim R} II_k, \]
%
where
%
\[ II_k = \int b_k(t) e^{2 \pi i t P}\; dt, \]
%
where $b_k(t) = \chi_{II,k}(t) R \widehat{h}(Rt)$. Thus $\text{supp} \{ a_k \} \subset \{ |t| \sim R / 2^k \}$, and
%
\[ \sum_k 2^{ks_p} \| b_k \|_{L^p(\RR)} \lesssim C_p(h) R^{1 - 1/p}. \]
% chi( R t / 2^k ) R h^(Rt)
% Derivative
More generally, we have
%
\[ \sum_k 2^{ks_p} \| \partial^\alpha b_k \|_{L^p(\RR)} \lesssim C_p(h) R^{|\alpha| + 1 - 1/p}, \]
%
which reflects the uncertainty principle based heuristic that since $h$ is supported on frequencies $\lesssim 1$, it's Fourier transform is locally constant at a scale $1$, and thus the functions $b_k$ are locally constant at a scale $1/R$.
\end{comment}

Let us proceed with the proof. To begin with, we can use the self-adjointness of the operators $Q$, the semigroup structure of $\{ e^{2 \pi i t P} \}$, and the fact that multipliers commute, to write
%
\begin{align*}
    \langle {S\!}_{x_0,t_0}, {S\!}_{x_1,t_1} \rangle &= \int \frac{b_{t_0}(t) \overline{b_{t_1}(s)}}{|E_{x_0}| |E_{x_1}|}\; \Big\langle (Q \circ e^{2 \pi i t P} \circ Q) \{ E_{x_0} \}, (Q \circ e^{2 \pi i s P} \circ Q) \{ E_{x_1} \} \Big\rangle\; dt\; ds\\
    &= \int \frac{b_{t_0}(t) \overline{b_{t_1}(s)}}{|E_{x_0}| |E_{x_1}|}\; \Big\langle (Q^2 \circ e^{2 \pi i (t - s) P} \circ Q^2) \{ E_{x_0} \}, E_{x_1} \Big\rangle\\
    &= \int \frac{c_{t_0,t_1}(t)}{|E_{x_0}||E_{x_1}|} \Big\langle (Q^2 \circ e^{2 \pi i t P} \circ Q^2) \{ E_{x_0} \}, E_{x_1} \Big\rangle,
\end{align*}
%
\begin{comment}

The wave propogators $\{ e^{2 \pi i t P} \}$ form a unitary semigroup. As a result, the Lax parametrix $\{ W(t) \}$ also satisfy the semigroup property \emph{modulo smoothing operators}, i.e. there exists a Schwartz operator $B(t,s)$ whose kernel $B(t,s,x,y)$ is smooth, such that
%
\[ W(s)^* \circ W(t) = W(t - s) + B(t,s). \]
%
This smoothness implies that
%
\[ \| \mathcal{P}_\lambda \{ B(t,s,\cdot,y) \} \|_{L^\infty(S^d)} \lesssim_N \lambda^{-N}, \]
%
and this together with the Weyl Law and the triangle inequality is sufficient to conclude that the kernel $K(t,s,x,y)$ of $Q \circ B(t,s)$ satisfies bounds of the form
%
\[ |K(t,s,x,y)| = \left| \sum \beta(\lambda / R) \mathcal{P}_\lambda \{ B(t,s,\cdot,y) \} \right| \lesssim_N R^{-N}. \]
%
Schur's Lemma, applied to $Q \circ B(t,s)$, together with the uniform boundedness of $Q$, implies that
%
\[ \| (Q \circ B(t,s) \circ Q) \{ E_{x_0} \} \|_{L^2(S^d)} \lesssim_N R^{-N} |E_{x_0}|^{1/2} \]
%
and so Cauchy-Schwartz implies that
%
\begin{align*}
    &\left| \int b_{t_0}(t) \overline{b_{t_1}(s)} \langle (Q \circ B(t,s) \circ Q) \{ E_{x_0} \}, E_{x_1} \rangle\; dt\; ds \right|\\
    &\quad \lesssim_N R^{-N} \int |b_{t_0}(t)| |b_{t_1}(s)| |E_{x_0}|^{1/2} |E_{x_1}|^{1/2}\; dt\; ds\\
    &\quad\lesssim_N C_p(h)^2 R^{-N}.
\end{align*}

\end{comment}
%
where
%
\[ c_{t_0,t_1}(t) = \int b_{t_0}(u) \overline{b_{t_1}}(u - t)\; dt, \]
%
is the convolution of $b_{t_0}$ with the reflection of $\overline{b_{t_1}}$ about the $y$-axis. Thus $c_{t_0,t_1}$ is supported on the length $2/R$ interval centered at $t_0 - t_1$, and has $L^1$ norm $O(1/R^2)$ by Young's convolution inequality.

We next perform a decomposition of the inner product into various coordinate systems. Cover $S^d$ by a finite family of sets $\{ V_\alpha \}$, chosen such that for each $V_\alpha$, there is a coordinate chart $U_\alpha$ such that the neighbourhood $N(V_\alpha, 0.5)$ is contained in $U_\alpha$. Let $\{ \eta_\alpha \}$ be a partition of unity subordinate to $\{ V_\alpha \}$. It will also be convenient to define $V_\alpha^* = N(V_\alpha, 0.1)$. We can then write
%
\[ \langle S_{t_0,x_0}, S_{t_1,x_1} \rangle = \sum_\alpha \int \frac{c_{t_0,t_1}(t)}{|E_{x_0}||E_{x_1}|} \big\langle (Q^2 \circ e^{2 \pi i t P} \circ Q^2) \{ \eta_\alpha E_{x_0} \}, E_{x_0} \big\rangle\; dt. \]
%
We will bound each of the terms on the right separately from one another, by working with each inner product in the coordinate systems $\{ U_\alpha \}$.

The next Lemma allows us to approximate the operator $Q$, and the propogators $e^{2 \pi i t P}$, with operators which have more explicit representations in the coordinate system $U_\alpha$, by an error term which is negligible to our analysis. It utilizes the \emph{Lax-H\"{o}rmander parametric} for the half-wave equation over small times, which expresses $e^{2 \pi i t P}$ in coordinates as a Fourier integral operator.

%The next Lemma allows us to replace $Q$ with an operator more amenable to analysis in the coordinate systems $\{ U_\alpha \}$. Similar proofs are given as Lemma 2.4 of (TODO: Sogge, Seeger, 1989), or (TODO: Find Taylor reference), or (TODO: Sogge).

\begin{lemma} \label{pseudodifferentialCoordinateLemma}
    For each $\alpha$, and $|t| \leq 1/100$, there exists Schwartz operators $Q_\alpha$ and $W_\alpha(t)$, each with kernel supported on $U_\alpha \times V^*_\alpha$, such that the following properties hold:
    %
    \begin{itemize}
        \item For $f \in C^\infty(S^d)$ with $\text{supp}(f) \subset V^*_\alpha$,
        %
        \[ \text{supp}(Q_\alpha f) \subset N(\text{supp}(f), 0.1) \quad\text{and}\quad \text{supp}(W_\alpha(t) f) \subset N(\text{supp}(f), 0.1). \]
        %
        Moreover,
        %
        \[ \| (Q^2 - Q_\alpha) f \|_{L^2(S^d)} \lesssim_N R^{-N} \| f \|_{L^2(S^d)} \]
        %
        and
        %
        \[ \bigg\| \Big(Q_\alpha \circ \Big( e^{2 \pi i t P} - W_\alpha(t) \Big) \circ Q_\alpha \Big) \{ f \} \bigg\|_{L^2(S^d)} \lesssim_N R^{-N} \| f \|_{L^2(S^d)}. \]

        \item In the coordinate system of $U_\alpha$, $Q_\alpha$ is a pseudodifferential operator of order zero given by a symbol $\sigma(x,\xi)$, where
        %
        \[ \text{supp}(\sigma) \subset \{ |\xi| \sim R \}, \]
        %
        and $\sigma$ satisfies derivative estimates of the form
        %
        \[ |\partial^\beta_x \partial^\kappa_\xi \sigma(x,\xi)| \lesssim_{\beta,\kappa} R^{-|\kappa|}. \]

        \item In the coordinate system $U_\alpha$, the operator $W_\alpha(t)$ has a kernel $W_\alpha(t,x,y)$ with an oscillatory integral representation
        %
        \[ W_\alpha(t,x,y) = \int s(t,x,y,\xi) e^{2 \pi i [ \phi(x,y,\xi) + t |\xi|_y ]}\; d\xi, \]
        %
        where $s$ has compact support in it's $x$ and $y$ coordinates, with
        %
        \[ \text{supp}(s) \subset \{ |\xi| \sim R \}, \]
        %
        where $s$ satisfies derivative estimates of the form
        %
        \[ | \partial_{t,x,y}^\beta \partial_\xi^\kappa s | \lesssim_{\beta, \kappa} R^{- |\kappa|}, \]
        %
        and where $| \cdot |_y$ denotes the norm on $\RR^n_\xi$ induced by the Riemannian metric on $S^d$ on the contangent space $T^*_y S^d$.
    \end{itemize}
\end{lemma}

\begin{comment}

\begin{proof}
    Introduce some $\rho \in C_c^\infty(\RR)$ with compact support near the origin, and use the Lax parametrix to write
    %
    \begin{align*}
        Q &= \int R \widehat{\beta}(Rt) e^{2 \pi i t P}\; dt\\
        &= \int \rho(t) R \widehat{\beta}(Rt) e^{2 \pi i t P}\; dt + \int (1 - \rho(t)) R \widehat{\beta}(Rt) e^{2 \pi i t P}\; dt\\
        &= \int \rho(t) R \widehat{\beta}(Rt) W(t) + \int \rho(t) R \widehat{\beta}(Rt) A(t)\\
        &\quad\quad + \int (1 - \rho(t)) R \widehat{\beta}(Rt) e^{2 \pi i t P}\; dt.
    \end{align*}
    %
    Write these three terms as $T_1$, $T_2$, and $T_3$. We can write $T_3 = r(P)$, where $r$ is the inverse Fourier transform of $(1 - \rho(t)) R \widehat{\beta}(Rt)$. The rapid decay of $\widehat{\beta}$ implies that
    %
    \[ |(\partial_\lambda^\alpha r)(\lambda)| \lesssim_{\alpha,N} R^{-N} \langle \lambda \rangle^{-M} \quad\text{for all $N,M \geq 0$}, \]
    %
    and so the Weyl Law and Schur's Lemma implies that
    %
    \[ \| T_3 f \|_{L^2(S^d)} \lesssim_N R^{-N} \| f \|_{L^2(S^d)}. \]
    %
    Since the kernel of $A$ is smooth, if $\psi(\lambda,x,y)$ denotes the Fourier transform of $\rho(t) A(t,x,y)$ in the $t$-variable, then $|\psi(\lambda,x,y)| \lesssim_N \langle \lambda \rangle^{-N}$, and so the multiplication formula implies that
    %
    \[ \left| \int \rho(t) R \widehat{\beta}(Rt) A(t, x, y)\; dt \right| = \left| \int \beta(\lambda/R) \psi(t,x,y)\; dt \right| \lesssim_N R^{-N}. \]
    %
    We can now again apply Schur's Lemma to conclude that
    %
    \[ \| T_2 f \|_{L^2(S^d)} \lesssim_N R^{-N} \| f \|_{L^2(S^d)} \quad\text{for $f \in C^\infty(S^d)$}. \]
    %
    Next, we note that for $f \in C^\infty_c(\tilde{V}_\alpha)$. We now abuse notation, working in coordinates, and identifying an operator with it's pushforward in the coordinate system $z_\alpha$. Since $W(t) f = W_\alpha(t) f$, the operator $T_1$ has a kernel representation
    %
    \begin{align*}
        T_1(x,y) &= \int \rho(t) R \widehat{\beta}(Rt) W(t,x,y)\; dt\\
        &= \int \int \rho(t) R \widehat{\beta}(Rt) s(t,x,y,\xi) e^{2 \pi i [\phi(x,y,\xi) + t |\xi_y|_g]}\; d\xi,
    \end{align*}
    %
    where $s$ is supported on $\{ d_g(x,y) \leq 0.01 \}$. If we let $\tilde{s}(\lambda,x,y,\xi)$ denote the inverse Fourier transform of $\rho(t) R \widehat{\beta}(Rt) a(t,x,y,\xi)$ in the $t$ variable, then we can write
    %
    \[ T_1(x,y) = \int \tilde{s}(|\xi_y|_g,x,y,\xi) e^{2 \pi i \phi(x,y,\xi)}\; d\xi. \]
    %
    If we write $\tilde{s} = s_1 + s_2$, where
    %
    \begin{align*}
        & s_1(\lambda,x,y,\xi) = \chi(\lambda/R) \tilde{s}(\lambda,x,y,\xi)\\
        &\quad\quad\quad\text{and}\quad s_2(\lambda,x,y,\xi) = (1 - \chi(\lambda/R)) \tilde{s}(\lambda,x,y,\xi),
    \end{align*}
    %
    where $\chi$ has support on the band $\{ |\lambda| \sim_\alpha 1 \}$, and is equal to one on a slightly thinner band, then
    % beta_R * psi
    % D^s (beta_R * psi)
    % = R^{-N} ((D^s beta_R) * psi)
    % R^{-M} lambda^{-M}
    % For 
    \[ |(\partial_x^\alpha \partial_y^\beta \partial_\xi^\kappa \partial_\lambda^\gamma s_1)(\lambda,x,y,\xi)| \lesssim_{\alpha,\beta,\kappa,\gamma} \langle \lambda \rangle^{-\gamma} \langle \xi \rangle^{-\kappa}, \]
    %
    and
    %
    \[ |(\partial_x^\alpha \partial_y^\beta \partial_\xi^\kappa \partial_\lambda^\gamma s_2)(\lambda,x,y,\xi)| \lesssim_{\alpha,\beta,\kappa,\gamma,N,M} R^{-N} \langle \lambda \rangle^{-M} |\xi|^{-\kappa}. \]
    %
    Write $T_1 = S_1 + S_2$, where for $j \in \{ 1, 2 \}$, the operator $S_j$ has kernel
    %
    \[ S_j(x,y) = \int s_j(|\xi_y|_g, x,y, \xi) e^{2 \pi i \phi(x,y,\xi)}\; d\xi. \]
    %
    Write $\tilde{s}_j(x,y,\xi) = s_j(|\xi_y|_g, x, y, \xi)$. Then
    %
    \[ |(\partial_x^\alpha \partial_y^\beta \partial_\xi^\kappa) \{ \tilde{s}_2 \}(x,y,\xi)| \lesssim_{\alpha, \beta, \kappa, N} R^{-N} |\xi|^{-\kappa}. \]
    %
    Applying the theory of pseudodifferential operators of order zero, we conclude that
    %
    \[ \left\| S_2 f \right\|_{L^2(S^d)} \lesssim_N R^{-N} \| f \|_{L^2(S^d)} \quad\text{for all $f \in C_c^\infty(\tilde{V}_\alpha)$}. \]
    %
    On the other hand, we have
    %
    \[ |(\partial_x^\alpha \partial_y^\beta \partial_\xi^\kappa) \{ \tilde{s}_1 \}(x,y,\xi)| \lesssim_{\alpha,\beta,\kappa} |\xi|^{-\kappa}. \]
    %
    The theory of pseudodifferential operators thus implies that $S_1$ is a pseudodifferential operator of order zero, and, because the above estimates are uniform in $R$, the symbol $\mathfrak{s}(x,\xi)$ of $S_1$ satisfies estimates of the form
    %
    \[ |(\partial_x^\alpha \partial_y^\beta \partial_\xi^\kappa \mathfrak{s})(x,\xi)| \lesssim_{\alpha,\beta,\kappa} |\xi|^{-\kappa} \]
    %
    uniformly in $R$. We can also calculate the symbol via the oscillatory integral formula
    %
    \[ \mathfrak{s}(x,\xi) = \int \tilde{s}_1(x,y,\eta) e^{2 \pi i [\phi(x,y,\eta) + \xi \cdot (y - x)]}\; d\eta\; dy. \]
    %
    Using the fact that $|\eta| \sim R$ on the support of $\tilde{s}_1$, unless $|\xi| \sim R$, the integral will be non-stationary in the $y$-variable, and we can use this to show that for such values of $\xi$,
    %
    \[ |(\partial_x^\alpha \partial_\xi^\beta \mathfrak{s})(x,\xi)| \lesssim_{\alpha,\beta, N,M} R^{-N} |\xi|^{-M}. \]
    %
    We thus see that setting $Q_\alpha$ to be the pseudodifferential operator with symbol $\tilde{\chi}(\xi / R) \mathfrak{s}(x,\xi)$ completes the proof, where $\tilde{\chi}$ is smooth, compactly supported on $\{ |\xi| \sim R \}$, and equal to one on a slightly narrower band.
\end{proof}

\end{comment}

%To work with the wave propogators in coordinates, we employ the \emph{Lax-H\"{o}rmander Parametrix}, localized to frequencies $\sim R$, which is sufficient to approximate the solution to the wave equations with inputs given by the range of the operators $Q$.

%\begin{lemma}
%    For any $\alpha$, and for $|t| \lesssim 1$, we can find a Schwartz operator $W_t$ such that:
    %
%    \begin{itemize}
        % \item For $f \in C^\infty(\tilde{V}_\alpha)$, $e^{2 \pi i t P} f = W_t f + A_t f$.

%        \item In the coordinate system $U_\alpha$, the operator $W_t$ has a kernel $W_t(x,y)$ given by an oscillatory integral of the form
        %
%        \[ W_t(x,y) = \int s(t,x,y,\xi) e^{2 \pi i [ \phi(x,y,\xi) + t |\xi|_y ]}\; d\xi, \]
        %
%        where
        %
%        \[ \text{supp}(s) \subset \{ |\xi| \sim R \} \cap \{ d_g(x,y) \leq t + 0.01 \}, \]
        %
%        where $s$ satisfies derivative estimates of the form
        %
%        \[ |\partial_{t,x,y}^\beta \partial_\xi^\kappa s| \lesssim_{\beta,\kappa} R^{- |\kappa|}, \]
        %
%        and where $|\cdot|_y$ denotes the norm on the cotangent space of $y$ induced by the Riemannian metric on $S^d$.

%        \item For any $N \geq 0$,
        %
%        \begin{align*}
%         \Bigg| &\int c_{t_0,t_1}(t) \big\langle (Q \circ e^{2 \pi i t P} \circ Q) \{ \phi_\alpha E_{x_0} \}, E_{x_1} \big\rangle\; dt\\
%         &\quad - \int c_{t_0,t_1}(t) \big\langle (Q_\alpha \circ W_t \circ Q_\alpha) \{ \phi_\alpha E_{x_0} \}, E_{x_1} \big\rangle\; dt \Bigg| \lesssim_N R^{-N} C_p(h)^2.
%        \end{align*}
%    \end{itemize}
%\end{lemma}

% write $e^{2 \pi i t P} = W(t) + A(t)$, where $A(t)$ has a kernel $A(t,x,y)$ which is a smooth function of $t$, $x$, and $y$, and $W(t)$ is a Schwartz operator, with kernel
%
%\[ \text{supp}(W(t)) \subset \{ (t,x,y) : d(x,y) \leq t + 0.01 \}, \]
%
%and such that in the coordinate system of $U_\alpha$, the kernel of $W(t)$ is given by an oscillatory integral of the form
%
%\[ W(t,x,y) = \int s(t,x,y,\eta) e^{2 \pi i [ \phi(x,y,\eta) + |\eta|_y ]}\; d\eta, \]
%
%where $|\cdot|_y$ gives the norm induced on the contangent space at $y$ by the metric on $S^{d-1}$.
%\[ W(t,x,y) = \int s(t,x,y,\eta) e^{2 \pi i [ \phi(t,y,\eta) + x \cdot \eta ]}\; d\eta. \]
%
% Stationary in eta variable when x = exp_t((y, eta))
% and then [x,(nabla_x phi)(t,x,eta)] = exp_t((y,eta))
%  = eta
% Nabla_x phi(t,x,eta)
%
%where
%
%\[ | \partial_\xi^\alpha \{ \phi(x,y,\xi) - \xi \cdot (x - y) \} | \lesssim_\alpha |x - y|^2 |\xi|^{1 - \alpha}, \]
%
%and $|\nabla_\xi \phi(x,y,\xi)| \gtrsim |x - y|$ for $(x,y,\xi) \in \text{supp}(s)$.

%\begin{proof}
%    The standard Lax-Parametrix construction (see TODO) implies that for each $\alpha$, we can find operators $\tilde{W}_t$ and $\tilde{A}_t$ such that for $x \in U_\alpha$ and $y \in V_\alpha^*$,
    %
%    \[ e^{2 \pi i t P}(x,y) = \tilde{W}_t(x,y) + \tilde{A}(t,x,y), \]
    %
%    where:
    %
%    \begin{itemize}
%        \item For $f \in C^\infty(V_\alpha^*)$, $\text{supp}(\tilde{W}_t f) \subset N(\text{supp}(f), t + 0.01)$.

%        \item In the coordinate system $U_\alpha$, we can write
        %
%        \[ \tilde{W}_t(x,y) = \int \tilde{s}(t,x,y,\xi) e^{2 \pi i [ \phi(x,y,\xi) + t |\xi|_y ]}\; d\xi, \]
        %
%        where $s$ satisfies derivative estimates of the form
        %
%        \[ |\partial_{t,x,y}^\beta \partial_\xi^\kappa \tilde{s}| \lesssim_{\beta,\kappa} R^{- |\kappa|}. \]
%    \end{itemize}
    %
%    In other words, $\tilde{W}_t$ satisfies all our required assumptions, except that $\tilde{s}$ is not supported on $\{ |\xi| \sim R \}$. But we define $s(t,x,y,\xi) = \tilde{s}(t,x,y,\xi) \chi( \xi / R )$ for some fixed $\chi \in C_c^\infty(\RR)$.
    %
%    \[ W_t = \chi( \xi / R ) \]


%    in the coordinate system $U_\alpha$,
%    %
%    \[ \tilde{W}_t(x,y) = \int \tilde{s}(t,x,y,\xi) e^{2 \pi i [ \phi(x,y,\xi) + t |\xi|_y ]}, \]
    %


%    and such that $\tilde{A}(t,x,y)$ is a $C^\infty$ function in the $x$ and $y$ variables.


%    Lemma TODO implies
    %
%    \[ \left| \int c_{t_0,t_1}(t) \big\langle (Q \circ e^{2 \pi i t P} \circ (Q - Q_\alpha)) \{ \phi_\alpha E_{x_0} \}, E_{x_1} \big\rangle\; du \right| \lesssim_N C_p(h)^2 R^{-N}. \]
    %
%    The smoothness of the kernel of $A(t)$ implies that
    %
%    \[ \| \mathcal{P}_\lambda \{ A(t, \cdot, y) \} \|_{L^\infty(S^d)} \lesssim_N \lambda^{-N}, \]
    %
%    and this together with the Weyl Law and the triangle inequality is sufficient to argue that the kernel of $Q \circ A(u)$ has $L^\infty$ norm $O_N(R^{-N})$ for all $N > 0$. Using Schur's Lemma, we conclude that (TODO: DETAILS)
    %
%    \[ \left| \int c_{t_0,t_1}(t) \big\langle (Q \circ A(t) \circ Q_\alpha) \{ \phi_\alpha E_{x_0} \}, E_{x_1} \big\rangle\; du \right| \lesssim_N C_p(h)^2 R^{-N}, \]
    %
%    another negligible error. Finally, using Lemma \ref{pseudodifferentialCoordinateLemma} again, we have
    %
%    \[ \left| \int c_{t_0,t_1}(t) \big\langle (Q - Q_\alpha) \circ W_t \circ Q_\alpha) \{ \phi_\alpha E_{x_0} \}, E_{x_1} \big\rangle\; du \right| \lesssim_N C_p(h)^2 R^{-N}. \]
    %
%    TODO: THE REST.

%    Fix a large constant $C > 0$, and consider a function $\chi \in C_c^\infty(\RR^d)$ equal to one on $\{ C^{-1} \leq |\xi| \leq C \}$. Consider
    %
%    \[ \tilde{W}(t) = \int \chi(\xi/R) s(t,x,y,\xi) e^{2 \pi i [ \phi(t,y,\xi) + x \cdot \xi ]}\; d\xi. \]
    %
%    We first claim that $(Q_\alpha \circ (W_t - \tilde{W}(t))(x,y)$ has $L^\infty$ norm $O_N(R^{-N})$. Indeed, we can write this integral as
    %
%    \begin{align*}
%        &\int \sigma(x,\theta) (1 - \chi(\xi / R)) s(t,w,y,\xi)\\
%        &\quad\quad\quad\quad e^{2 \pi i [ \theta \cdot (x - w) + w \cdot \xi - \phi(t,y,\xi) ]}\; dw\; d\xi\; d\theta.
%    \end{align*}
    %
%    Notice that
    %
%    \[ \nabla_w \{ \theta \cdot (x - w) + w \cdot \xi - \phi(t,y,\xi) \} = \theta - \xi. \]
    %
%    On the support of the integral above, we have $|\theta| \sim R$, and we also either have $|\xi| \leq C^{-1} R$ or $|\xi| \geq C R$. In the former case, if $C$ is sufficiently large, we have
    %
%    \[ |\theta - \xi| \geq |\theta| - |\xi| \gtrsim R, \]
    %
%    and in the latter case,
    %
%    \[ |\theta - \xi| \geq |\xi| - |\theta| \gtrsim |\xi|. \]
    %
%    In either case, first integrating by parts arbitrarily many times in the $w$ variable, and then taking in absolute values and integrating in all variables yields the required result.
%\end{proof}

Thus, ignoring errors negligible to our analysis, we need only analyze
%
\[ \left| \int \frac{c_{t_0,t_1}(t)}{|E_{x_0}| |E_{x_1}|}\; \big\langle (Q_\alpha \circ W_\alpha(t) \circ Q_\alpha) \{ \phi_\alpha E_{x_0} \}, E_{x_1} \big\rangle\; du \right|. \]
%
The behaviour of all operations in this expression are now completely localized to $U_\alpha$ for inputs supported on $V^*_\alpha$; in particular, this expression is equal to zero unless $E_{x_0}$ and $E_{x_1}$ are both compactly contained in $U_\alpha$, So we can now naturally work with the kernels of the operators in coordinates to upper bound the inner product, which will complete the required estimate of the inner product.

%\begin{lemma}
%    Modulo errors of order $O_N(R^{-N})$, the adjoint operator $Q_\alpha^*$ agrees with the pseudodifferential oeprator $Q_\alpha'$ with symbol $\sigma^*(x,\xi)$, supported on $\{ |\xi| \sim R \}$ and with uniform derivative bounds.
%\end{lemma}
%\begin{proof}
%    The symbol of the operator $Q_\alpha^*$ can be computed by the formula
    %
%    \[ (x,\xi) \mapsto \int \sigma(y,\eta) e^{2 \pi i (\xi + \eta) \cdot (y - x)}\; d\eta\; dy. \]
    %
%    If $|\xi| \leq R$, we may integrate by parts in the $y$ variable arbitrarily many times to conclude this integral is $O_N(R^{-N})$. If $|\xi| \gtrsim R$, we may integrate by parts in the $y$-variable arbitrarily many times to conclude this integral is $O_N(|\xi|^{-N})$. TODO: Clean this up.
%\end{proof}

\begin{lemma}
    Let $c$ be an integrable function supported on the length $10/R$ interval centered at a value $t^*$ with $|t| \leq 1/100$. Then
    %
    \begin{align*}
        & \left| \int \frac{c(t)}{|E_{x_0}| |E_{x_1}|} \big\langle (Q_\alpha \circ W_\alpha(t) \circ Q_\alpha) \{ \phi_\alpha E_{x_0} \}, E_{x_1} \big\rangle\; dt \right|\\
        &\quad \lesssim_M R^d \frac{\| c \|_{L^1(\RR)}}{( R d_g(x_1,x_0) )^{\frac{d-1}{2}}} \sum_{\pm} \Big\langle R \big| t^* \pm d_g(x_1,x_0) \big| \Big\rangle^{-M}
    \end{align*}
\end{lemma}
\begin{proof}
%    Note that $\phi$ satisfies the Eikonal equation
    %
%    \[ |\nabla_w \phi(w,y,\xi)|_w = |\xi|_y, \]
    %
 %   and so $|\nabla_w \phi(w,y,\xi)| \sim |\xi|$.
% On the support of the integral above, we have $|\theta| \sim R$, and we also either have $|\xi| \leq C^{-1} R$ or $|\xi| \geq C R$. In the former case, if $C$ is sufficiently large, we have
    %
 %   \[ |\nabla_w \{ \theta \cdot (x - w) + \phi(w,y,\xi) + t |\xi|_y \}| \geq |\theta| - |\nabla_w \phi(w,y,\xi)| \gtrsim R, \]
    %
  %  and in the latter,
    %
   % \[ |\nabla_w \{ \theta \cdot (x - w) + \phi(w,y,\xi) + t |\xi|_y \}| \geq |\nabla_w \phi(w,y,\xi)| - |\theta| \gtrsim |\xi|. \]
    %
    %In either case, first integrating by parts arbitrarily many times in the $w$ variable, and then taking in absolute values and integrating in all variables yields the required result.
    We write the integral as
    %
    \begin{align*}
        & \int \frac{c(t)}{|E_{x_0}| |E_{x_1}|} (\eta_\alpha E_{x_1})(w)  \sigma(w,\theta) e^{2 \pi i \theta \cdot (w - x)}\\
        &\quad\quad\quad s(t,x,y,\xi) e^{2 \pi i [ \phi(x,y,\xi) + t |\xi|_y ]} \sigma(y,\eta) e^{2 \pi i \eta \cdot (y - z)} E_{x_0}(z)\\
        &\quad\quad\quad\quad\quad \; dt\; dx\; dy\; dz\; dw\; d\theta\; d\xi\; d\eta.
    \end{align*}
    %
    The integral looks complicated, but can be simplified considerably by noticing that all the spatial variables are highly localized. To begin with, we use the fact that $s$ is smooth and compactly supported in all it's variables, so $s$ should roughly behave like a linear combination of tensor products; using Fourier series, we can write
    %
    \[ s(t,x,y,\xi) = \sum_{n \in \ZZ^d} s_{n,1}(x) s_{n,2}(t,y,\xi), \]
    %
    where $s_{n,1}(x) = e^{2 \pi i n \cdot x}$, and where
    %
    \[ |\partial_{t,y}^\alpha \partial_\xi^\kappa \{ s_{n,2} \}| \lesssim_{\alpha,\kappa,N} |n|^{-N} R^{- |\kappa|} \]
    %
    If we define $a_n(\xi) = a_{n,1}(R \xi) a_{n,2}(R \xi)$, where
    %
    \begin{align*}
        a_{n,1}(\xi) &= |E_{x_1}|^{-1} \int (\eta_\alpha E_{x_1})(w) \sigma(w,\theta) s_{n,1}(x) e^{2 \pi i [\theta \cdot (w - x) + (x - x_1) \cdot \xi ]}\; d\theta\; dw\; dx \intertext{\text{and}}
        a_{n,2}(\xi) &= |E_{x_0}|^{-1} \int c(t) s_{n,2}(t,y,\xi) \sigma(y,\zeta) E_{x_0}(z) e^{2 \pi i [ (\phi(t^*, x_0, \xi) - \phi(t,y,\xi)) + \zeta \cdot (y - z)]}\; d\zeta\; dt\; dy\; dz,
    \end{align*}
    %
    % a_{n,1}(xi) = int e(x) s_{n,1}(x) e^{2 \pi i (x - x_1) * xi}
    %       Here e has L^1 norm 1, and should decay
    %       rapidly away from a 1/R neighborhood of x_1
    %       So this a_{n,1} should have L^infty norm 1
    %
    % a_{n,2}(xi)
    %
    then, rescaling, we can write the required integral as
    %
    \[ R^d \sum_{n \in \ZZ^d} \int a_n(\xi) e^{2 \pi i R [\phi(x_1,x_0,\xi) + t^* |\xi|_{x_0}]}\; d\xi. \]
    %
    Notice that $\text{supp}(a_n) \subset \{ |\xi| \sim 1 \}$, and
    %
    \[ |(\nabla_\xi^\kappa a_n)(\xi)| \lesssim_{\kappa,N} |n|^{-N} \| c \|_{L^1(\RR)}. \]
    %
    To obtain an efficient upper bound on this oscillatory integral, it will be convenient to change coordinate systems in a way better respecting the Riemannian metric at $x_0$, i.e. finding a smooth family of diffeomorphisms $\{ F_{x_0}: S^{d-1} \to S^{d-1} \}$ such that $|F_{x_0}|_{x_0} = 1$. We can choose this function such that $F_{x_0}(-x) = - F_{x_0}(x)$. Then if $\tilde{a}_n(\rho, \eta) = a_n( \rho F_{x_0}(\eta) ) JF_{x_0}(\eta)$, then a change of variables gives that
    % 
    \[ \int a_n(\xi) e^{2 \pi i R [ \phi(x_1,x_0,\xi) + t^* |\xi|_{x_0} ]} = \int_0^\infty \rho^{d-1} \int_{|\eta| = 1} \tilde{a}_n(\rho,\eta) e^{2 \pi i R \rho [ \phi(x_1, x_0, F_{x_0}(\eta)) + t^* ]}\; d\eta\; d\rho. \]
    %
    For each fixed $\rho$, we claim that the phase has exactly two stationary points in the $\eta$ variable, at the values $\pm \eta_0$, where $x_1$ lies on the geodesic passing through $x_0$ tangent to the vector $\eta_0^\sharp$ (here we are using the musical isomorphism to map the cotangent vector $\eta_0$ to a tangent vector). Moreover, at these values,
    %
    \[ \phi(x_1,x_0, F_{x_0}(\pm \eta_0)) = \pm d_g(x_1,x_0), \]
    %
    and the Hessian at $\pm \eta_0$ is (positive / negative) definite, with each eigenvalue having magnitude exceeding a constant multiple of $d_g(x_1,x_0)$. It follows from the principle of stationary phase, that the integral above can be written as
    %
    \[ \frac{1}{\left( R d_g(x_1,x_0) \right)^{\frac{d-1}{2}}} \sum_{\pm} \int_0^\infty \rho^{\frac{d-1}{2}} f_{n,\pm}(\rho) e^{2 \pi i R \rho [ t^* \pm d_g(x_1,x_0)]}\; d\rho, \]
    %
    where $f_{n,\pm}$ is supported on $|\rho| \sim 1$, and
    %
    \[ |\partial_\rho^m f_{n,\pm}| \lesssim_{m,N} |n|^{-N} \| c \|_{L^1(\RR)}. \]
    %
    Integrating by parts in the $\rho$ variable if $\pm d_g(x_1,x_0) + t^*$ is large, and then taking in absolute values, we conclude that
    %
    \[ \left| \int a_n(\xi) e^{2 \pi i R [ \phi(x_1,x_0,\xi) + t^* |\xi|_{x_0} ]} \right| \lesssim_{N,M} |n|^{-N} \frac{\| c \|_{L^1(\RR)}}{( R d_g(x_1,x_0) )^{\frac{d-1}{2}}} \sum_{\pm} \langle R |t^* \pm d_g(x_1,x_0)| \rangle^{-M}. \]
    %
    Taking $N \geq d + 1$, and summing in the $n$ variable, we conclude that
    %
    \[ \left| \sum_n \int a_n(\xi) e^{2 \pi i R [ \phi(x_1,x_0,\xi) + t' |\xi|_{x_0} ]} \right| \lesssim_M \frac{\| c \|_{L^1(\RR)}}{( R d_g(x_1,x_0) )^{\frac{d-1}{2}}} \sum_{\pm} \langle R | t^* \pm d_g(x_1,x_0)| \rangle^{-M}. \]
    %
    But this is precisely an estimate for the quantity we wished to estimate. $\qedhere$


%    TODO: REPLACE? To finish off our calculation, we calculate
    %
%$$    \[ \int a(\xi) e^{2 \pi i R [ \phi(x_1,x_0,\xi) + (t_0 - t_1) |\xi|_{x_0} ]}\; d\xi. \]
    %
%    We work in polar coordinates, computing
    %
%    \[ \int \rho^{d-1} e^{2 \pi i R \rho (t_1 - t_0)} \int_{|\xi| = 1} a(\rho \xi) e^{2 \pi i R \rho \phi(x_1,x_0,\xi)}\; d\xi. \]
    % (x_1 - x_0) * xi + O(|x_1 - x_0|^2 |xi|)
%    HOW TO FORMALLY PROVE THAT STATIONARY POINT OCCURS WHEN $\xi$ is pointing in the direction of $x_1$?


%    To finish off our calculation, we will calculate
    %
%    \[ \int a(\xi) e^{2 \pi i R[ x_1 \cdot \xi - \phi(t_0 - t_1, x_0, \xi) ]}\; d\xi. \]
    %
%    We must now consider the behaviour of the function $\phi$ in more detail.
    % Phase function is x * xi - phi(t,y,xi)
    %
    % Stationary when x = phi_xi(t,y,xi)
    %
    % x-derivative is xi
    % y-derivative is - phi_y(t,y,xi)
    %
    % So Canonical relation of this equation is
    % { (phi_xi, y; xi, phi_y)  }
    %
    % And this should be the canonical relation of the half-wave equation,
    %
    % so (y, phi_y) = Phi_t(phi_xi, xi / |xi|)
    %
    % In the Euclidean case:
    %
    %   y = phi_xi + t xi / |xi|
    %   phi_y = t xi
    %
    %   phi = t |xi| - y * xi
    %
    %   phi_xi = t xi / |xi| - y
    %   phi_y = - xi
    %
    %   y = (t xi / |xi| - y) + t xi
%    The function $\phi$ is constructed via an eikonal equation, and satisfies the partial differential equation
    %
%    \[ (\phi_\xi,\xi) = \Phi_t(y,\phi_y), \]
    %
    % phi_xi = y + t phi_y / |phi_y|
    % xi = phi_y
%    where $\Phi_t$ gives the unit speed geodesic flow at time $t$ in the cotangent space, with respect to the metric induced from $S^{d-1}$. In particular, in the oscillatory integral above, the values of $\xi$ that are stationary satisfy the equation
    %
%    \[ (x_1, \xi) = \Phi_{t_0 - t_1}(x_0, \phi_y). \]
    %
%    and the set of such values of $\xi$ form a ray through the origin. So we consider polar coordinates, writing the integral as
    %
%    \[ \int_0^\infty \int_{|\xi| = 1} \rho^{d-1} a(\rho \xi) e^{2 \pi i \rho R [ x_1 \cdot \xi - \phi(t_0 - t_1, x_0, \xi) ]}. \]
    %
%    As mentioned above, there is a unit vector $\nu = \nu(x_0,x_1)$ such that the unique non-degenerate stationary point in the integral above in the $\xi$ variable occurs when $\xi = \nu$. It follows that the integral above is
    %
%    \[ R^{- \frac{d-1}{2}} \int_{1/2}^2 \rho^{\frac{d-1}{2}} \tilde{a}(\rho)\; d\rho \lesssim R^{- \frac{d-1}{2}}. \]
    %
%    But this means the overall integral is $O(R^{-\frac{3d - 1}{2}})$.

%    TODO: CLEAN UP THE REST OF THIS MESSY CALCULATION BELOW.


%    The uniqueness of geodesic flow tells us that there is a unique ray through the origin, given by the equation $\xi = \rho \nu(x_0,x_1,t_0)$

%     whose values give stationary values of $\xi$.

%    In particular, in the oscillatory integral above, the stationary values of $\xi$ come when $( x_1, \xi ) = \exp( x_0, (t_0 - t_1) \phi_y )$
    


    % In the flat case the integral is (x - y) * xi + t |xi|
    % So phi(t,y,xi) = t |xi| - y * xi
    % [t xi / |xi| - y, xi] = exp( y, - t xi )

%     The canonical relation of the wave equation allows us to determine some of it's behaviour. Namely, we have
    %
%    \[ \nabla_\xi \{ x \cdot \xi - \phi(t,y,\xi) \} = 0 \]
    %
%    if and only if $(\nabla_\xi \phi)(t,y,\xi) = x$, and in such a situation we find that
    %
%    \[ \nabla_x \{ x \cdot \xi - \phi(t,y,\xi) \} = \xi \]
%    \[ \nabla_y \{ x \cdot \xi - \phi(t,y,\xi) \} = - \nabla_y \phi(t,y,\xi), \]
    %
%    and so we conclude that
    %
%    \[ ( - \phi_\xi, \xi ) = \exp( y, \phi_y ) \]

%    The function is obtained from an Eikonal equation,
    %
%    \[ \nabla_\xi \{ x_1 \cdot \xi + \phi(t_0 - t_1, x_0, \xi) \} = x_1 + \nabla_\xi \]

%    This phase has a stationary point precisely when $(x_1,\xi) = \exp( x_0, (t_0 - t_1) \xi)$

%    We have thus reduced our analysis to
%    %
%    \[ \int c_{t_0,t_1}(t)\; \big\langle (Q_\alpha \circ W_t \circ Q_\alpha) \{ \phi_\alpha E_{x_0} \}, E_{x_1} \big\rangle\; dt. \]
    %
%    s    


%    Using the adjoint calculation above, modulo negligible errors, we can write this integral as
    %
%    \[ \int c_{t_0,t_1}(t) \big\langle (\tilde{W}(t) \circ Q_\alpha) \{ \phi_\alpha E_{x_0} \}, Q_\alpha' E_{x_1} \big\rangle\; du. \]
    %
%    Let
    %
%    \[ a(t,x,y,\xi) = \chi(\xi) s(t,x,y,R \xi), \]
    %
%    and consider
    %
%    \[ R^d \int c_{t_0,t_1}(t) a(t,x,w,\xi) \sigma(w,\theta) \sigma^*(x,\zeta) e^{2 \pi i R [ \phi(t,w,\xi) + \xi \cdot x + \theta \cdot (w - y) - \zeta \cdot (x - z)]}\; d\xi\; d\theta\; d\zeta\; dw\; dz\; dt \]
%    %
%    Write $a(t,x,w,\xi) = a_1(x) a_2(t,w) a_3(\xi)$. If
    %
%    \[ f(\xi) = a_3(\xi) \int c_{t_0,t_1}(t) a_2(t,w) \sigma(w,\theta) (\eta_\alpha E_{x_0})(y) e^{2 \pi i R [ [\phi(t,w,\xi) - \phi(t_0 - t_1,x_0,\xi)] + \theta \cdot (w - y) ]}\; dw\; dy\; dt\; d\theta, \]
    %
%    and
    %
%    \[ g(\xi) = \int a_1(x) \sigma^*(x,\zeta) E_{x_1}(x) e^{2 \pi i R \zeta \cdot (x - z) + [ \xi \cdot (x - x_1) ]}\; d\zeta\; dz\; dx. \]
    %
%    Then the integral above is
    %
%    \[ \int f(\xi) \overline{g(\xi)} e^{2 \pi i R [ \phi(t_0 - t_1, x_0, \xi) + \xi \cdot x_1]}\; d\xi. \]
    %
%    Now $\nabla_\xi \{ \phi(t_0 - t_1, x_0, \xi) + \xi \cdot x_1 \} = 0$ at a unique value $\xi_1$, where $(x_1,\xi_1) = \exp( x_0, (t_0 - t_1) \nabla_y \phi(t_0 - t_1, x_0, \xi) )$, and so we get that the integral above is
    %
%    \[ R^{d/2} f( \xi_1 ) \overline{g(\xi_1)}. \]

%    Taking Fourier transforms, 

%    \[ \int a_2(w) \sigma(w,\theta) e^{2 \pi i R [ \phi(t,w,\xi) - \phi(t_0,x_0,\xi) ]}\; dw \]


%    \[ f(x;t,\xi) = e^{2 \pi i R[\phi(t_0 - t_1,x_0,\xi) - \phi(t,x,\xi)]} Q_\alpha \{ \phi_\alpha E_{x_0} \}(x), \]
    %
%    and let
    %
%    \[ g(x) = Q_\alpha' \{ E_{x_1} \}(x). \]
    %
%    Using the compactness of the support of $a$, and taking a Fourier series, let us assume that
    %
%    \[ a(t,x,y,\xi) \]

%    Then we can write
    %
%    \begin{align*}
%        &\int c_{t_0,t_1}(t) \langle \tilde{W}(t) Q_\alpha E_{x_0}, Q_\alpha' E_{x_1} \rangle\; dt\\
%        &\quad\quad = \int c_{t_0,t_1}(t) a(t,x,y,\xi / R) f(y;t,\xi / R) e^{2 \pi i [ x \cdot \xi - \phi(t_0 - t_1, x_0, \xi)]} \overline{g(x)}\; d\xi dx\; dy\; dt\\
%        &\quad\quad = \int c_{t_0,t_1}(t) (\mathcal{F} a)(t,\eta,y,\xi / R) e^{-2 \pi i R \phi(t_0 - t_1, x_0, \xi)} f(y;t,\xi) \overline{ \mathcal{F} g(\eta - R \xi)}\; d\xi\; d\eta\; dy\; dt.
%    \end{align*}

%    \[ \int a(t,x,y,\xi / R) f(y;t,\xi) e^{2 \pi i x \cdot (\xi - \eta)}\; d\xi\; dx \]
    %
%    The equation $\phi$ satisfies the following Eikonal equation:
    %
%    \[ ( \nabla_\xi \phi(t,y,\xi), \xi ) = \exp ( y, t \nabla_y \phi(t,y,\xi) ). \]


%    $x = - a(t,y,\xi)$ gives a
    %
%    \[ \nabla_\xi \phi(t,y,\xi) = a \]
    %
%    then $x = -a$, gives a stationary point in the $\xi$ variable, and for this point
    %
%    \[ \nabla_x \{  \} \]

%     we know we must have
    %
%    \[ \nabla_x \]

%    \[ -x = \nabla_\xi \phi(t,y,\xi) \]
    %
%    Then $\nabla_y \phi($
    % x = exp_y(xi)
%    \[ \nabla_y \phi(t,y,\xi) \]

%    WHAT EIKONAL EQUATION DOES THIS $\phi$ SATISFY?

%    \[ \int e^{2 \pi i R x \cdot \xi} f(x;,t,\xi) \]
    %


%    \[ \int a(x,\eta) e^{2 \pi i R[ \phi(x,x_0,\eta) + t_0 |\eta|_{x_0} - \xi \cdot x ]}\; d\eta\; dx\]
    %
    % Probabily only a isngle stationary point, when (x,xi) = exp(x_0, t eta_0)

    %W(t) = int e^{2 pi i ( phi(t,y,eta) - x * eta)}


%    \[  \int a(x,\eta) e^{2 \pi i R [ \phi(x_0,t_0,\eta) - x \cdot \eta ]} \overline{g(x)}\; d\eta\; dx\; dy\; dt \]

%    For each $y$, consider a diffeomorphism $\exp_y$ that expresses $x$ in geodesic normal coordinates, and consider a polar coordinate system in the $\theta$ variables with respect to the metric on $S^{d-1}$, i.e. writing $\theta = \rho \tilde{\theta}$ with $|\theta|_w = \rho$. Letting $a'$ be the product of $a$ with the relevant Jacobians and $\rho^{d-1}$, we can write the quantity above as
    %
%    \[ R^d \int c_{t_0,t_1}(t) a' e^{2 \pi i \rho R \tilde{\Phi}}, \]
    %
%    where
    %
%    \[ \tilde{\Phi}(t,x,y,\rho, \tilde{\theta}) = \phi(\exp_y(x),y,\tilde{\theta}) + t = \tilde{\phi}(x,y,\tilde{\theta}) + t. \]
    %
%    s


%    In the $\tilde{\theta}$ variable, the oscillatory integral has only two stationary points, which we denote by $v_+(x,y)$ and $v_-(x,y)$, and each is non-degenerate. Thus we can write the required integral as
    %
%    \[ R^{\frac{d+1}{2}} \sum_{\pm} \int c_{t_0,t_1}(t) a_{\pm}(t,x,y,\rho v_{\pm}(x,y)) e^{2 \pi i R [ \phi(x,y,v_{\pm}(x,y)) + t |v_{\pm}(x,y)| ]} \]

%    One nice feature of this coordinate change is that\footnote{TODO: DOUBLE CHECK THIS WITH AN EXPERT} $\phi(\exp_w(z), w, \tilde{\theta}) = z \cdot \tilde{\theta}$. In particular, this implies that, in the $\tilde{\theta}$ variable, this oscillatory integral has only two stationary points, i.e. when
    %
%    \[ \tilde{\theta} = \pm \frac{z}{|z|}. \]
    %
%    The principle of stationary phase thus implies that the integral above
    %
%    \begin{align*}
%        \tilde{U}(t,x,y) &= R^{3d - \frac{d-1}{2}} \sum_{\pm} \int a_{\pm}(t,z,w,\xi,\rho,\xi)\\
%    &\quad\quad\quad\quad\quad\quad\quad\quad e^{2 \pi i R [ \xi \cdot (x - \exp_w(z)) + \rho (t \pm |z|) + \zeta \cdot ( w - y) ]}\; dz\; dw\; d\xi\; d\rho\; d\zeta,
%    \end{align*}
    %
%    where $a_+$ and $a_-$ are compactly supported bump functions satisfying analogous conditions to the function $a'$. Integrating in the $\rho$ variable, we can write $\tilde{U}(t,x,y)$ as
    %
%    \begin{align*}
%        R^{3d - \frac{d-1}{2}} \sum_{\pm} \int (\mathcal{F}^{-1}_{\rho,\xi,\zeta} \widehat{a}_{\pm})(t,z,w,R(x - \exp_w(z)),R (t \pm |z|),R(w - y))\; dz\; dw.
%    \end{align*}
    %
%    Taking absolute values in and integrating gives that this integral is
    %
%    \[ R^{\frac{d+1}{2}} \langle R ( t \pm d_g(x,y) ) \rangle^{-N}, \]
    %
%    i.e. the integral is concentrated on $\{ (x,y): |t \pm d_g(x,y)| \lesssim 1/R \}$, on which has height $O(R^{\frac{d+1}{2}})$. Integrating in the $x$ and $y$ variables over $E_{x_0}$ and $E_{x_1}$, we conclude that
    %
%    \[ |\langle (Q \circ W(t) \circ Q) \{ E_{x_0} \}, E_{x_1} \rangle| \lesssim_N R^{\frac{1-3d}{2}} \langle R ( t \pm d_g(x_0,x_1) ) \rangle^{-N}. \]
    %
%    Now

%    Now using Parseval's inequality, we obtain that
    %
%    \begin{align*}
%        &\int c_{t_0,t_1}(t) \langle \tilde{W}(t) Q_\alpha E_{x_0}, Q_\alpha' E_{x_1} \rangle\; dt\\
%        &\quad\quad\quad = \int c_{t_0,t_1}(t) v(t,y,\xi) e^{2 \pi i [ \phi(t,x_0,\eta) + x \cdot \eta ]} f(x) \overline{g}_\alpha(x)\; d\eta\; dx\; dt\\
%        &\quad\quad\quad = \int c_{t_0,t_1}(t) v(t,y,\xi) e^{2 \pi i \phi(t,x_0,\eta)} \widehat{f}(\xi - \eta) \overline{\widehat{g}(\xi)}\; d\eta\; d\xi\; dt
%    \end{align*}


%    Let $U(t) = Q_\alpha \circ W(t) \circ Q_\alpha$, and consider it's kernel $U(t,x,y)$. We have justified that we can write this kernel as an oscillatory integral, i.e. we have
    %
%    \begin{align*}
%        U(t,x,y) &= \int \sigma(x,\xi) s(t,z,w,\theta) \sigma(w,\zeta) e^{2 \pi i [ \xi \cdot (x - z) + \phi(t,w,\theta) - z \cdot \theta + \zeta \cdot (w - y) ]}\; dz dw d\xi d\theta d\zeta.
%    \end{align*}
    %
%    On the support of this integral, $|\xi| \sim R$, and we can use this to justify that the only significant contribution occurs when $|\theta| \sim R$. Indeed, if we write the phase as
    %%
%    \[ \Phi(t,x,z,w,y,\xi,\theta,\zeta) = \xi \cdot (x - z) + \phi(t,z,\theta) - w \cdot \theta + \zeta \cdot (w - y), \]
    %
%    then, using the fact that $\nabla_w \Phi = \zeta - \theta$, then for $C > 0$ chosen appropriately large, on the support of the amplitude function, if $|\theta| \leq C^{-1} R$, then
    %
%    \[ |(\nabla_w \Phi)(t,x,z,w,y,\xi,\theta,\zeta)| = |\zeta - \theta| \gtrsim R, \]
    %
%    and for $|\theta| \geq CR$, we have
    %
%    \[ |(\nabla_z \Phi)(t,x,z,w,y,\xi,\theta,\zeta)| \gtrsim |\theta|. \]
    %
%    If $\eta$ is equal to one on $C^{-1} \leq |\theta| \leq C$, and vanishing for $(2C)^{-1} \leq |\theta| \leq 2C$, and we define the operator $\tilde{U}(t)$ to have kernel
    %
%    \begin{align*}
%        \tilde{W}(t,x,y) &= \int \sigma(x,\xi) \chi(\theta / R) s(t,z,w,\theta) \sigma(w,\eta) e^{2 \pi i \Phi(t,x,z,w,y,\xi,\theta,\zeta)}\; dz\; dw\; d\xi\; d\theta\; d\zeta,
%    \end{align*}
    %
%    then it follows by integrating by parts enough times in the $w$ variable that
    %
%    \[ |U(t,x,y) - \tilde{U}(t,x,y)| \lesssim_N R^{-N}. \]
    %
%    Now use duality to write, modulo errors of order $O_N(R^{-N})$,
    %
%    \[ \big\langle (Q_\alpha \circ W(t) \circ Q_\alpha) \{ \phi_\alpha E_{x_0} \}, E_{x_1} \big\rangle = \big\langle (Q_\alpha \circ W(t) \circ Q_\alpha) \{ \phi_\alpha E_{x_0} \}, E_{x_1} \big\rangle \]

%    If we define
    %
%    \[ a(t,x,z,w,\xi,\theta,\zeta) = \sigma(x,\xi) \left\{ \chi(\theta / R) s(t,z,w,\theta) \right\} \sigma(w,\eta), \]
    %
%    then $a$ is a smooth, compactly supported function, uniformly in $R$, i.e. supported on $|x|,|z|,|w| \lesssim 1$, and $|\xi|, |\theta|, |\zeta| \sim 1$, and with
    %
%    \[ |\nabla_{t,x,z,w,\xi,\theta,\zeta}^N \{ a \}| \lesssim_N 1 \quad\text{for all $N > 0$}. \]
    %
%    We can rescale the integral defining $\tilde{U}$ in the $\xi$, $\theta$, and $\zeta$ variables, and write
    %
%    \[ \tilde{U}(t,x,y) = R^{3d} \int a\; e^{2 \pi i R \Phi}\; dz\; dw\; d\xi\; d\theta\; d\zeta. \]
    %
    


    %
    % C_p(h) = ( int [ <t>^{s_p} |h^(t)| ]^p dt ] )^{1/p}
    %
    % |b_{t_0}|_{L^p}^p = R^p int chi_{t_0}(t) |h^(Rt)|^p dt
    %
    % Sum [(Rt)^{s_p} |b_{t_0}|_{L^p}]^p = R^p int [(Rt)^{s_p} |h^(Rt)|]^p dt
    %                                    = R^{p-1} int [ t^{s_p} |h^(t)| ]^p dt
    %
    % | (Rt_0)^{s_p} |b_{t_0}|_{L^p} |_{l^p_{t_0}} ~ R^{1 - 1/p} C_p(h)
    % 
    % |b_{t_0}|_{L^p} << R^{1 - 1/p} (Rt_0)^{-s_p}, in a summable sense
    %       H_{t_0} << R (Rt_0)^{-s_p}
    %       So L^1 norm is (Rt_0)^{-s_p)
    % So L^infty norm of c_{t_0t_1} is (Rt_0)^{-s_p} (Rt_1)^{-s_p}
    %
    % Integrating the required bound in t gives that
    % R^{-(3d + 1)/2} (Rt_0)^{-s_p} (Rt_1)^{-s_p}
    %
    % integrating the bound above in t, we conclude that the inner product is
    % 
    % R^{-0.5 - 1.5d - (d-1)(2/p - 1)}
    % R^{-(d + 3)/2 - (2/p)(d - 1)}
    % R^{(-d/2)(4/p + 1) + (2/p - 1.5) }


    % Locally Constant
    % Intuitively, b_{t_0} has height H = R^{(d+1) [ 1/p - 1/2 ]} t_0^{(d-1)(1/p - 1/2)}
    % on a width 1/R segment.
    % So the convolution has height R^{2 (d+1) [ 1/p - 1/2 ] - 1 } t_0^{(d-1)(1/p - 1/2)} t_1^{(d-1)(1/p - 1/2)}
    % on a width 1/R segment
    % So intuitively, the inner product is
    % <<_N R^{(1 - 3d)/2 + 2(D+1)(1/p - 1/2) - 2} t_0^{(d-1)(1/p - 1/2)} t_1^{(d-1)(1/p - 1/2)} < R (t_0 - t_1 +/- d_g(x_0,x_1))  >
    % (d-1)(1/p - 1/2) > (d-1)/2


    % R h^(Rt)
    % If we let b_{t_0}(t) = R chi_{t_0}(t) h^(Rt)
    % Then |b_{t_0}|_{L^p}^p = R^p int chi_{t_0}(t) |h^(Rt)|^p dt
    % Sum (Rt_0)^{p s_p} |b_{t_0}|_{L^p}^p = R^p int (Rt)^{p s_p} |h^(Rt)|^p\; dt
    %           = R^{p-1} int t^{p s_p} |h^(t)|^p; 
    %           = R^{p-1} C_p(h)^p
    % ( Sum R^{1 - p + p s_p} t_0^{p s_p} |b_{t_0}|_{L^p}^p )^{1/p} = C_p(h)
    % ( Sum R^{d - (d+1)/2 * p} t_0^{(d-1)(1 - p/2)} |b_{t_0}|_{L^p}^p )
    % So |b_{t_0}|_{L^p} << R^{d/p - (d+1)/2} t_0^{(d-1)(1/p - 1/2)},
    % in a sense summable in t_0
    % s_p = (d-1)(1/p - 1/2)
    % 1 - p + (d-1)(1 - p/2)
    % d - (d+1) p / 2
\end{proof}

\begin{comment}

\begin{remark}
    It might be enlightening to compare this bound to a rescaled version of the analogous bound in the Euclidean setting, obtained in Lemma 3.3 of \cite{HeoandNazarovandSeeger}. Consider a radial multiplier $m: \RR^d \to \CC$, i.e. with $m(\xi) = h(|\xi|)$. The Fourier multiplier $M_R$ with symbol $m( \cdot / R)$ can then be given by convolution against the kernel
    %
    \[ k_R(x) = \int_0^\infty [ R \widehat{h}(Rt) ] k_t(x)\; dt, \]
    %
    where $k_t$ is the Fourier transform of $e^{2 \pi i t |\xi|}$. The dilation symmetry of Euclideans pace implies that
    %
    \[ k_t = t^{-d}\; \text{Dil}_t k, \]
    %
    where we let $k = k_1$. We can compute that
    %
    \begin{align*}
        k(x) &= \int e^{2 \pi i [|\xi| + \xi \cdot x]}\; d\xi\\
        &= |x|^{- \frac{d-2}{2}} \int_0^\infty r^{d/2} e^{2 \pi i r} J_{\frac{d-2}{2}}( 2 \pi r |x| ).
    \end{align*}
    %
    Consider a cutoff for $r \leq 2 / |x|$, i.e. letting
    %
    \[ k_L(x) = |x|^{- \frac{d-2}{2}} \int_0^\infty \chi(r |x|) r^{d/2} e^{2 \pi i r} J_{\frac{d-2}{2}}(2 \pi r |x|). \]
    %
    We can rescale this quantity, writing it as
    %
    \[ |x|^{-d} \int_0^\infty \chi(r) r^{d/2} e^{2 \pi i r / |x|} J_{\frac{d-2}{2}}(2 \pi r). \]
    %
    The quantity $r^{d/2} J_{\frac{d-2}{2}}(r)$ has a zero of order $d-1$ at $r = 0$. Integrating by parts $d$ times thus gives that
    %
    \[ k_L(x) = C_d + \int_0^\infty \tilde{\chi}(r) e^{2 \pi i r / |x|} \]
    %
    for some constant $C_d$ and some smooth, compactly supported function $\tilde{\chi}$. This is sufficient to conclude that $k_L$ is a symbol of order zero. Next, we analyze the integral
    %
    \[ k_H(x) = |x|^{- \frac{d-2}{2}} \int_0^\infty [1 - \chi(r |x|)] r^{d/2} e^{2 \pi i r} J_{\frac{d-2}{2}}(2 \pi r |x|). \]
    %
    Bessel function asymptotics allow us to write
    %
    \[ J_{\frac{d-2}{2}}(s) = \sum_{\pm} s^{-1/2} e^{\pm 2 \pi i s} a_{\pm}(s), \]
    %
    where $a_{\pm}$ are symbols of order zero. Thus, rescaling and introducing this asymptotic, we can write the integral above as
    %
    \[ k_H(x) = |x|^{-d} \sum_{\pm} \int_0^\infty [1 - \chi(r)] r^{\frac{d-1}{2}} e^{2 \pi i r [ 1/|x| \pm 1]} a_{\pm}(r)\; dr. \]
    %
    If we write $a_{\pm,0}(r) = [1 - \chi(r)] r^{\frac{d-1}{2}} a_{\pm}(r)$, then $a_{\pm,0}$ is a symbol of order $(d-1)/2$, and the required integral can then be expressed as
    % -(d+1)/2
    \[ |x|^{-d} \sum_{\pm} \widehat{a}_{\pm,0}(1/|x| \pm 1), \] 
    %
    where $\widehat{a}_{\pm,0}$ has a singularity of order $-(d+1)/2$ at the origin, and rapidly decays away from this singularity. In other words, for $|x| \leq 1/2$,
    %
    \[ |\partial_{\text{Rad}}^\alpha k(x)| \lesssim_{\alpha} 1, \]
    %
    for $1/2 \leq |x| \leq 2$,
    % |1 - |x||^{-(d+1)/2}
    \[ |\partial_{\text{Rad}}^\alpha k(x)| \lesssim_\alpha |1 - |x||^{- \frac{d+1}{2} - \alpha}, \]
    %
    and for $|x| \geq 2$,
    %
    \[ |\partial_{\text{Rad}}^\alpha k(x)| \lesssim_\alpha |x|^{-d - \alpha}. \]
    %
    BLEH

    % Contributions from |x| >= 2 are O(1), rapidly decaying for |xi| >> 1
    % The same for for |x| <= 1/2
    % For

    % We might expect the ~1 Frequency Cutoff to make k
    % behave like 
    % 2^j s + 2^{-j} (r/s) |xi|
    % s^2 = 4^{-j} r |xi|
    % s = 2^{-j} r^{1/2} |xi|^{1/2}

    The function $k$ should behave like
    %
    \[ \sum_{k = 1}^\infty 2^{k \frac{d+1}{2}} \chi( 2^k ||x| - 1| ). \]
    %
    Now $\chi( 2^k ||\cdot| - 1|)$ is a Bochner Riesz bump function, and thus it's Fourier transform behaves like $2^{-k} |\xi|^{- \frac{d-1}{2}} e^{2 \pi i |\xi|}$. Once localized to $|\xi| \sim 1$, it's Fourier transform looks like $2^{-k} \chi(\xi) e^{2 \pi i |\xi|}$,

    The function $k$ should behave like
    %
    \[ \sum_{k = 1}^\infty 2^{k \frac{d+1}{2}} \chi( 2^k ||x| - 1| ) \]
    \[ \chi(\xi) \int_0^\infty \chi( 2^k |r - 1| ) r^{d/2} e^{2 \pi i r} J_{\frac{d-2}{2}}(\lambda r)\; dr \]
    \[ \chi(\xi) \int_0^\infty \chi( 2^k |r - 1| ) r^{\frac{d-1}{2}} e^{2 \pi i r [1 \pm \lambda]}\; dr \]


    \[ \chi(\xi) \int_0^\infty r^{d/2} \chi(|r - 1| / 2^k) e^{2 \pi i r} J_{\frac{d-2}{2}}(r) \]


    How about when we average the half-wave equation now over time?
    % Wave EQUATIONS SHOULD BE L1 Normalized
    \begin{align*}
        &\int [ R \widehat{h}(R(t - t_0)) ] t^{-d} k_L(x/t)\; dt\\
        &\quad = \int \widehat{h}(t - t_0) t^{-d}\; dt
    \end{align*}
    \[ \int [R \widehat{h}(R(t - t_0))] t^{-d} k_L(x/t)\; dt \]

    This quantity can then be written as
    %
    \[ |x|^{-d} |x|^{-(d+1)/2} |1/|x| +/- 1|^{-(d+1)/2} \]

    As $|x| \to 1$, this quantity has a singularity of order $-(d+1)/2$

    % ~a are symbols of order (d-1)/2, supported away from the origin
    \[ |x|^{-d} \sum_{\pm} \int_0^\infty \tilde{a}_{\pm}(r) e^{} \]

    % |x|^{-d} int chi(r) r^{d/2} e^{2 pi i r LAMBDA} J(r)
    % 
    This is the integral of a function compactly supported near the origin, and so a Taylor expansion argument (TODO) shows that this quantity is $|x|^{d/2}$.

    %if we apply a cutoff
    %
    %\[ |x|^{-d} \int_0^\infty \chi(r) r^{d/2 + \alpha + 2k} e^{2 \pi i r / |x|} \]
    %
    %If we integrate by parts in the $r$ variable $d/2 + \alpha + 2k$ times, we end up with
    %
    %\[ |x|^{\alpha + 2k - d/2} \int_0^\infty \tilde{\chi}(r) e^{2 \pi i r / |x|}, \]
    %
    %where $\tilde{\chi}(0) \neq 0$. And so it follows that this quantity is equal to
    %
    %\[ |x|^{\alpha + 2k - d/2} \psi(|x|) \]
    %
    % for a function $\psi$, analytic in $|x|$.

    % int_0^infty a(x) b(y) = a(0) B(0) - int_0^infty a'(x) B(x)
    % So no boundary terms provided that a(0) = 0
    % 

    \[ \int_0^\infty \chi(|x| r) r^{d/2} e^{2 \pi i r} (r x)^{\alpha + 2k} \]


    %
    We consider the term
    %
    \begin{align*}
        &\int_0^\infty \chi(|x| r) r^{\frac{d}{2}} e^{2 \pi i r} J_{\frac{d-2}{2}}( 2 \pi r |x| )\; dr\\
        &\quad = |x|^{-1-d/2} \int_0^\infty r^{\frac{d}{2}} e^{2 \pi i r / |x|} \tilde{\chi}(r)\; dr.
    \end{align*}
    %
    Integrating by parts gives that this quantity is can be written as
    %
    \[ |x|^{d/2 - 1} \chi(|x|), \]
    %
    for some smooth function $\chi$, nonvanishing at the origin.

    The function $\tilde{\chi}$ has a zero of order $d/2 - 1$ at the origin, which implies that this
    %
    \[ |x|^{d/2 - 1} [ A  ] \int_0^\infty \partial_r^{d-1} \{ r^{d/2} \tilde{\chi}(r) \} e^{2 \pi i r / |x|} \]
    % Integration by parts will work d-1 times - on the dth time, we pick up a boundary term
    % int x f(x)
    % = - int F(x) dx


    % Near the origin, chi has a zero of order d/2 - 1
    % So together with r^{d/2}, chi has a zero of order d - 1


    We consider a cutoff function $\chi(|x| r)$

    TODO: Asymptotics for $r \leq 1/|x|$.

    Using Bessel asymptotics, we write
    %
    \[ k(x) = |x|^{- \frac{d-2}{2}} \sum_{\pm} \int_0^\infty r^{\frac{d-1}{2}} a_{\pm}(r |x|) e^{2 \pi i r [1 \pm |x|]}\; dr. \]


    %
    \[ k_R(x) = \int_0^\infty [R \widehat{h}(Rt) ]\; \mathcal{F}_\xi \{ e^{2 \pi i t |\xi|} \}(x)\; dt, \]
    %
    Then for $|x| \geq 2/t$,
    %
    \[ |k_{\text{HW}}(x)| \lesssim_d t^{-\frac{d+1}{2}} |x|^{- \frac{d-1}{2}} \]
    % NOt the regime we're dealing with in our approach


    \[ |k_{\text{HW}}(tx)| \lesssim_d | 1 - |x| |^{-\frac{d+1}{2}} = t^{-d} \int e^{2 \pi i [|\xi| + \xi \cdot x]} d\xi \]
    %
    which is a constant multiple of
    % 
    \[ |x|^{- \frac{d-2}{2}} \int_0^\infty \int_0^\infty [ R \widehat{h}(Rt) ] r^{d/2} e^{2 \pi i t r} J_{\frac{d-2}{2}}( 2 \pi |x| r )\; dr\; dt. \]
    \[ |x|^{-d/2} \sum_{\pm} \int_0^\infty \int_0^\infty [ R \widehat{h}(Rt) ] r^{\frac{d-1}{2}} a_{\pm}(|x| r) e^{2 \pi i r(t \pm |x|)}\; dr\; dt \]
    \[ |x|^{-d/2-1} \sum_{\pm} \int_0^\infty [ R \widehat{h}(Rt) ] (\partial^{\frac{d-1}{2}} A_{\pm})(t / |x| \pm 1)\; dt \]
    \[ |x|^{2d - 3/2} \sum_{\pm} \int_0^\infty [R \widehat{h}(Rt)] (t \pm |x|)^{(1-3d)/2}\; dt \]
    % t = 0, |x|^{d/2 - 1}
    % Homogeneous of order -d-(d-1)/2

    \[ |x|^{-d/2} \sum_{\pm} \int_0^\infty \int_0^\infty \widehat{h}(t) r^{\frac{d-1}{2}} a_{\pm}(|x| r) e^{2 \pi i r(t / R \pm |x|)}\; dr\; dt \]
    \[ R^{\frac{d+1}{2}} |x|^{-d/2} \sum_{\pm} \int_0^\infty h(r) r^{\frac{d-1}{2}} a_{\pm}(R r |x|) e^{\pm 2 \pi i r R |x|}\; dr \]
    %
    If we pretend that $a_{\pm} = 1$, then we get
    %
    \[ R^{\frac{d+1}{2}} |x|^{-d/2} \langle R |x| \rangle^{-N} \]


     can then be written as
    %
    \[ (M_R f)(x) = \int_0^\infty \int_{\RR^d} [R \widehat{h}(R t)] e^{2 \pi i t |\xi|} \widehat{f}(\xi)\; d\xi\; dt. \]

     given by a convolution by $R^d\; k( R \cdot )$, where $k$ is the Fourier transform of $m$.

    \[ \int R \widehat{h}(Rt) e^{2 \pi i t P} \]
\end{remark}

\end{comment}

\begin{comment}

Each of the operators $Q_\alpha$, $W(u)$, and $Q_\alpha$ is given by an oscillatory integral in the coordinates $z_\alpha$, 

%
\[ Q \circ A(u) \]

This smoothness implies that
%
\[ \| \mathcal{P}_\lambda \{ B(t,s,\cdot,y) \} \|_{L^\infty(S^d)} \lesssim_N \lambda^{-N}, \]
%
and this together with the Weyl Law and the triangle inequality is sufficient to conclude that the kernel $K(t,s,x,y)$ of $Q \circ B(t,s)$ satisfies bounds of the form
%
\[ |K(t,s,x,y)| = \left| \sum \beta(\lambda / R) \mathcal{P}_\lambda \{ B(t,s,\cdot,y) \} \right| \lesssim_N R^{-N}. \]
%
Schur's Lemma, applied to $Q \circ B(t,s)$, together with the uniform boundedness of $Q$, implies that
%
\[ \| (Q \circ B(t,s) \circ Q) \{ E_{x_0} \} \|_{L^2(S^d)} \lesssim_N R^{-N} |E_{x_0}|^{1/2} \]
%
and so Cauchy-Schwartz implies that
%
\begin{align*}
    &\left| \int b_{t_0}(t) \overline{b_{t_1}(s)} \langle (Q \circ B(t,s) \circ Q) \{ E_{x_0} \}, E_{x_1} \rangle\; dt\; ds \right|\\
    &\quad \lesssim_N R^{-N} \int |b_{t_0}(t)| |b_{t_1}(s)| |E_{x_0}|^{1/2} |E_{x_1}|^{1/2}\; dt\; ds\\
    &\quad\lesssim_N C_p(h)^2 R^{-N}.
\end{align*}




$t$, $x$, and $y$, and $W(t)$ is a Schwartz operator, with kernel satisfying
%
\[ \text{supp}(W) \subset \{ (t,x,y) : d(x,y) \leq t + 0.01 \}, \]
%
and such that, if we consider Schwartz operators $\{ W_\alpha(t) \}$ with kernel
%
\[ W_\alpha(t,x,y) = \tilde{\eta}_\alpha(x) W(t,x,y) \tilde{\eta}_\alpha(y), \]
%
then $W(t) = \sum W_\alpha(t)$, and each operator $W_\alpha(t)$ can, in the coordinate system $U_\alpha$, be given as a Fourier integral with canonical relation supported in the geodesic light cone, i.e. for each $\alpha$, there exists a symbol $s$ of order zero, and a phase function $\Phi$ such that the pushforward operator $W_\alpha^{z_\alpha}(t)$ has kernel
%
\[ W_\alpha^{z_\alpha}(t,x,y) = \int s(t,x,y,\xi) e^{2 \pi i \Phi(t,x,y,\xi)}\; d\xi, \]
%
where $\Phi(t,x,y,\xi) = \phi(x,y,\xi) + t |\xi_y|_g$, and where $\exp$ denotes the geodesic flow in cotangent space. Let us use this decomposition to write $II = II_W + II_A$, where
%
\[ II_W = \int b(t) W(t)\; dt \quad\text{and}\quad II_A = \int b(t) A(t)\; dt. \]

We now consider quantities of the form
%
\[ \sum_{\alpha,\beta} \int c_{t_0,t_1}(u) \langle (Q \circ W_\alpha(u) \circ Q) \{ \phi_\beta \cdot E_{x_0} \}, E_{x_1} \rangle \]



It thus follows that if we define $\widetilde{W}(u) = \tilde{Q} \circ W(u) \circ \tilde{Q}$, then (TODO: Double Check Calculation)
%
\begin{equation} \label{firstreductionorthogonal}
    \left| \langle \ess_{x_0,t_0}, \ess_{x_1,t_1} \rangle - \int c_{t_0,t_1}(u) \langle \widetilde{W}(u) \{ E_{x_0} \}, E_{x_1} \rangle\; du \right| \lesssim_N C_p(h)^2 R^{-N},
\end{equation}
%
We note, for later use, that $c_{t_0,t_1}$ is supported in a $O(1/R)$ neighborhood of $t_0 - t_1$.

Now let's perform some calculations in each coordinate system $\{ U_\alpha \}$. For each $E_{x_0}$, our reductions show that $|\langle \ess_{x_0,t_0}, \ess_{x_1,t_1} \rangle| \lesssim_N C_p(h)^2 R^{-N}$ unless $d(E_{x_0}, E_{x_1}) \leq 1/50$. For any $\alpha$ such that $(W_\alpha(u) \circ \tilde{Q}) \{ E_{x_0} \} \neq 0$, it therefore follows that $E_{x_1} \subset U_\alpha$. And to determine the $L^2$ product, it thus suffices to find a representation of $\tilde{Q} \circ W_\alpha \circ \tilde{Q}$ in the coordinate system $U_\alpha$.


%
\[ \int c_{t_0,t_1}(u) \langle \tilde{W}_\alpha(u) E_{x_0}, E_{x_1} \rangle\; du. \]
%


We now wish to perform some computations in the coordinate system $U_\alpha$. In order to do this, we need an expression for $Q$ in the coordinate system, and to do this we rely on the following Lemma, 

It follows from this lemma and \eqref{firstreductionorthogonal} (TODO: Check Calculation) that
%
\[ \left| \langle \ess_{x_0,t_0}, \ess_{x_1,t_1} \rangle - \int c_{t_0,t_1}(u) \langle (\tilde{Q} \circ W(t) \circ \tilde{Q}) \{ E_{x_0} \}, E_{x_1} \rangle\; du \right| \lesssim_N C_p(h)^2 R^{-N}. \]

If $d(E_{x_0}, E_{x_1}) \geq 2/100$, then we can use the pseudolocality of the operators $Q$, together with the finite propogation speed of the Lax parametrix, to show the interactions between $\ess_{t_0,x_0}$ and $\ess_{t_1,x_1}$ are completely negligible. TODO: Get $O_N(R^{-N} C_p(h)^2)$ error.

On the other hand, if $d(E_{x_0}, E_{x_1}) \leq 2/100$, then we can find a single coordinate system compactly containing $E_{x_0}$ and $E_{x_1}$.

We now wish to make the assumption that $E_{x_0}$ and $E_{x_1}$ lie in a single coordinate system, and to work with the Lax parametrix in this coordinate system. Let us begin by choosing finitely many coordinate charts $\{ \theta \}$ such that for any $x_0,x_1 \in \mathcal{X}$, if $d(E_{x_0}, E_{x_1}) \leq 1/10$, then $E_{x_0}$ and $E_{x_1}$ both are compactly contained in the domain of one of the coordinate charts $\theta$. For any $t_0, t_1 \in \mathcal{T}$, the support of $c_{t_0,t_1}$ is contained in $\{ s : |s - (t_0 - t_1)| \leq 1/100 \}$, and since $|t_0|, |t_1| \leq 1/100$, contained in $\{ |s| \leq 1/49 \}$. The fact that $W(u)$ is supported on $\{ (x,y): d(x,y) \leq u + 1/100 \}$, together with the pseudolocality of $Q$, means it is natural to expect that $\langle (Q \circ W(u) \circ Q) \{ E_{x_0} \}, E_{x_1} \rangle$ will only be non-negligible when $d(E_{x_0}, E_{x_1}) \leq 1/10$. But the operator $Q$ is not local, so we must verify this precisely.

In this calculation, it will be helpful in that analysis to use the methods of this section to characterize the `large time' behaviour of the projection operators $Q$ introduced in Remark 3 of the introduction. We write $Q = Q^{\text{Low}} + Q^{\text{High}}$, where
%
\[ Q^{\text{Low}} = \int \chi(t) \widehat{\beta}_R(t) e^{2 \pi i t P}\; dt \quad\text{and}\quad Q^{\text{High}} = \int (1 - \chi(t)) \widehat{\beta}_R(t) e^{2 \pi i t P}\; dt \]
%
for some $\chi \in C_c^\infty(\RR)$ equal to one for $|t| \leq 1/200$ and vanishing for $|t| \leq 1/100$. We rely on a periodization argument, like in our analysis of the operators $\{ III_R \}$, to show the behaviour of the operator $Q^{\text{High}}$ can be safely disregarded.

\begin{lemma} \label{LargeTimeQBounds}
    The operators $\{ Q^{\text{High}} \}$ are uniformly negligible.
%    The kernel $Q_R^{\text{High}}(x,y)$ of $Q_R^{\text{High}}$ satisfies
%    %
%    \[ |\partial_x^\alpha \partial_y^\beta Q_R^{\text{High}}(x,y)| \lesssim_{\alpha,\beta,N} R^{-N} \quad\text{for all $N \geq 0$}. \]
\end{lemma}
\begin{proof}
    For $t \in [-1/2,1/2]$, define
    %
    \[ H(t) = \sum_l (1 - \chi(t + l)) R \widehat{\beta}(R (t + l)). \]
    %
    Unlike $h$, the function $\beta$ is smooth and compactly supported, and thus $\widehat{\beta}$ is rapidly decaying, which allows us to conclude that for each $k$, we have $|\partial_t^k H(t)| \lesssim_{k,N} R^{-N}$ for all $N \geq 0$. Now
    %
    \[ (\partial_x^\alpha \partial_y^\beta Q^{\text{High}})(x,y) = \int H_R(t) (\partial_x^\alpha \partial_y^\beta e^{2 \pi i t P})(x,y)\; dt \]
    %
    If $M$ is a sufficiently large integer, we `integrate by parts', writing
    %
    \[ \int H(t) e^{2 \pi i t P}\; dt = \int \langle D_t \rangle^{M/2} \{ H \} \langle D_t \rangle^{-M/2} \{ e^{2 \pi i t P} \}\; dt, \]
    %
    where we view $\langle D_t \rangle = (1 + D_t^2)^{1/2}$ is a positive definite self-adjoint pseudodifferential operator of order $1$. The derivative estimates on $H$ imply that we have $|\langle D_t \rangle^{M/2} \{ H \}| \lesssim_{N,M} R^{-N}$ for any $N > 0$. On the other hand, $t \mapsto \partial_x^\alpha \partial_y^\beta e^{2 \pi i t P}$ is a smooth parameterization of a family of Fourier integral operators of order $|\alpha| + |\beta|$ on $S^d$. The composition calculus for Fourier integral operators thus implies that $t \mapsto \langle D_t \rangle^{-M/2} \{ e^{2 \pi i t P} \}$ is a parameterized family of Fourier integral operators of order $|\alpha| + |\beta| - M$. If $M$ is sufficiently large, we conclude that the kernels of these Fourier integral operators are bounded, uniformly in $|t| \leq 1/2$ by virtue of the smoothness of the parameterization. But this implies the required kernel estimates, namely
    %
    \[ |(\partial_x^\alpha \partial_y^\beta Q^{\text{High}})(x,y)| \lesssim_N R^{-N} \int_{|t| \leq 1/2} |\langle D_t \rangle^{-M/2} \{ e^{2 \pi i t P} \}(x,y)| \lesssim_N R^{-N}. \qedhere \]
\end{proof}

It follows that $Q \circ W(u) \circ Q$ differs from $Q^{\text{Low}} \circ W(u) \circ Q^{\text{Low}}$ by a uniformly negligible family of operators. Now we have done this, we can use the Lax parametrix to understand $Q^{\text{Low}}$, namely, writing $Q^{\text{Low}} = Q^{\text{Low}}_W + Q^{\text{Low}}_A$, where
%
\[ Q^W = \int \widehat{\beta}_R(t) W(t)\; dt \quad\text{and}\quad Q^A = \int \widehat{\beta}_R(t) A(t), \]
%
for smoothing operators $\{ A(t) \}$. We claim that $Q^{\text{Low}} \circ W(u) \circ Q^{\text{Low}}$ also differs from $Q^W \circ W(u) \circ Q^W$ by a uniformly negligible family of operators.

\begin{lemma}
    The operator $Q^{\text{Low}} \circ W(u) \circ Q^{\text{Low}}$ differs from $Q^W \circ W(u) \circ Q^W$ by a uniformly negligible family of operators.
\end{lemma}
\begin{proof}
    First off, $A(t) \circ W(u)$ has a kernel which is a smooth function of $t$ and $u$. It follows, as with the analysis of the smoothing operators $B(t,s)$ applied to $Q$, that $A(t) \circ W(u) \circ Q$ has a kernel which is pointwise $O_N(R^{-N})$. But since $Q^{\text{High}}$ is self-adjoint, we also know that $A(t) \circ W(u) \circ Q^{\text{High}}$ has a kernel which is pointwise $O_N(R^{-N)}$, and so $A(t) \circ W(u) \circ Q^{\text{Low}}$ has a kernel which is $O_N(R^{-N})$. This shows $Q^A \circ W(u) \circ Q^{\text{Low}}$ is uniformly negligible, and it remains to analyze $Q^W \circ W(u) \circ Q^A$ is uniformly negligible. But we can write this operator as
    %
    \[ \int \widehat{\beta}_R(t) Q^W \circ W(u) \circ A(t)\; du\; dt. \]
    %
    Taking Fourier transforms in the $t$ variable, we can write this as
    %
    \[ \int \beta(\lambda / R) Q^W \circ W(u) \circ \widehat{A}(\lambda)\; du\; d\lambda. \]
    %
    The fact that $A$ is smooth, and compactly supported in the $t$ variable, implies that $\widehat{A}$ is rapidly decaying in $\lambda$, which gives the require $O_N(R^{-N})$ estimates given that $\beta(\lambda / R)$ is supported on $|\lambda| \sim R$.
\end{proof}

Modulo errors of order $O_N(C_p(h)^2 R^{-N})$ for arbtrarily large $N$, we have therefore reduced our analysis of $\langle \ess_{x_0,t_0}, \ess_{x_1,t_1} \rangle$ to quantities of the form $\langle T E_{x_0}, E_{x_1} \rangle$, where
%
\[ T = \int c(u) (Q^W \circ W(u) \circ Q^W)\; du. \]
%
Because $Q^W$ is now defined in terms of the Lax parametrix, this operator now has support on $\{ (x,y): d(x,y) \leq 1/50 \}$, which implies that $\langle T E_{x_0}, E_{x_1} \rangle = 0$ unless $d(E_{x_0}, E_{x_1}) \leq 1/10$. Thus when this quantity is non-zero, we can find a coordinate system $\theta$ from the finite family we have selected, such that $E_{x_0}$ and $E_{x_1}$ are compactly contained in the domain of $\theta$. We will now compute quantities associated with $T$ in this coordinate system.

First, let's study the behaviour of $(Q^W)^\theta$. This operator is, modulo neglible terms, the projection operator onto eigenvalues of the Laplacian of $S^d$ with eigenvalue $\sim R$. Given ellipticity, we should expect that, in coordinates, modulo some extra error terms, the same is true.

\begin{lemma}
    Modulo uniformly negligible terms, the operator $(Q^W)^\theta$ is a pseudodifferential operator on $\RR^d$, and we can write it's symbol as $s(x, \xi / R)$, where $\text{supp}_\xi(s) \subset \{ 1/5 \leq |\xi| \leq 5 \}$, and $|\partial^\alpha_x \partial^\beta_\xi s(x,\xi)| \lesssim_{\alpha,\beta} 1$, uniformly in $R$.
\end{lemma}
\begin{proof}
    TODO: Did calculation in notes.
\end{proof}

Since $s(x,D)$ behaves like a variable-coefficient Fourier cutoff at a scale $R$, we should be able to use this to isolate the behaviour of the Lax parametrix $W(u)$ at this same scale.

\begin{lemma}
    Modulo uniformly negligible terms, the operator $s(x,D) \circ W(u)$ is equal to $s(x,D) \circ \tilde{W}(u)$, where $\tilde{W}(u)$ is expressed as a Fourier integral of the form
    %
    \[ \int a(u,x,y,\xi / R) e^{2 \pi i \phi(u,x,y,\xi)}\; d\xi, \]
    %
    where $a$ satisfies estimates of the form
    %
    \[ |\partial_u^\alpha \partial_x^\beta \partial_y^\delta \partial_\xi^\epsilon a(u,x,y,\xi)| \lesssim_{\alpha,\beta,\delta,\epsilon} 1, \]
    %
    and $\text{supp}_\xi(a) \subset \{ |\xi| \sim 1 \}$.
\end{lemma}

TODO: Calculations

TODO: The $L^1$ norm of $S^k(2^m)$ is
% Can cover support of S^k(2^m) by balls
% B1, ..., BN with
% sum r_i <= #(E_k) / 2^m
% and B_i contains <= 2^m r_i points to which we evaluate.
% Majority of support of Sum_{Bi} lies on an annulus of thickness ri
% and radius 2^k, i.e. it's size is sum_t 

% 1/R neighborhood of radius t sphere
% Volume is t^{d-1} / R
% So we should expect the L^1 norm to be sum_t A(t) R^{(d-5)/2} t^{d-1}
% Density 2^m
% t is about 2^k
\[ \lesssim \sum_t A(t) R^{\frac{d-3}{2}} \langle R (t + d(x,y)) \rangle^{-N} \]

\begin{lemma}
    The kernel of $T$ is upper bounded by
    %
    \[ A(t_0) A(t_1) R^{\frac{d-3}{2}} \langle R (t_0 + d(x,y)) \rangle^{-N} \]
    %
    for all $N \geq 0$.
\end{lemma}

\end{comment}

\begin{comment}

by reducing our analysis to a study of the operator in coordinates. We cover $S^d$ by a family of coordinate charts $\{ \theta \}$, and then perform a decomposition $II_W = \sum II_{W,i}$ and $Q^{\text{Low}}_W = \sum Q_{W,j}$ by applying a partition of unity to the kernels, such that the following property holds: for any two indices $i$ and $j$, such that $II_{W,i} \circ Q_{W,j} \neq 0$, there exists a coordinate system $\theta$ such that the kernels of $II_{W,i}$ and $Q_{W,j}$ both have support on a compact subset of $\theta$ (here we are using the fact that we are only looking at the \emph{small time} behaviour of the operators with respect to the wave equation, so that the kernels of the operators are supported on a small neighborhood of the diagonal). Applying the fact that the sum $\sum II_{W,i} \circ Q^{\text{Low}}_{W,j}$ is finite, it suffices to bound each term individually, and we do this by pushing forward our operator into coordinates. Let
%
\[ \mathsf{II} = II_{W,i}^\theta \quad\text{and}\quad \mathsf{Q} = Q^{\text{Low},\theta}_{W,j}. \]
%
be the pushforward operators, now defined mpaping functions on $\RR^d$ to functions on $\RR^d$. It will suffice to obtain $L^p$ estimates for $\mathsf{II} \circ \mathsf{Q}$ that are independent of $R$.

We will be aided in this analysis by an expression for the Lax parametrix in the coordinate system $\theta$.


 We will obtain bounds for this operator given some restricted bounds. So let us consider an analysis of functions of the form $S = (II_W \circ Q^{\text{Low}}_W) \{ E \}$, for some fixed measurable set $E \subset S^d$.  The pushforward operator $T_W^\theta$ has a kernel $K_\theta$ defined by an oscillatory integral distribution
%
\[ K_\theta(x,y) = a_\theta(x,y,\xi) e^{2 \pi i [ \varphi_\theta(x,y,\xi) + t |\xi_y|_{g(y)} ]}\; d\xi, \]
%
where the amplitude $a_\theta$ is a symbol of order zero, with compact $(x,y)$ support, the phase $\varphi$ is smooth, and we have
%
\[ \varphi_\theta(x,y,\xi) \approx (x - y) \cdot \xi, \]
%
in the sense that
%
\[ |\partial_x^\alpha \partial_y^\beta \partial_\xi^\lambda \{ \varphi_\theta(x,y,\xi) - (x - y) \cdot \xi \}| \lesssim |x - y|^2 |\xi|^{2 - \lambda}, \]
%
and where $|\xi|_{g(y)} = (\sum g^{ij}(y) \xi_i \xi_j )^{1/2}$ gives the length of the covector $\xi$ with respect to the Riemannian metric $g$. Let us write $\phi_\theta$ for $\varphi_\theta + t |\xi|_g$.

Becuase $II_W \circ Q_W$ breaks down into a finite sum of terms of the form $II_{W,i} \circ Q_{W,j}$, it suffices to bound each of these terms separately, and to do this, without loss of generality it suffices to bound $\mathcal{II} \circ \mathcal{Q}$, where $\mathcal{II} = II_{W,i}^\theta$, and $\mathcal{Q} = Q_{W,j}^\theta$. We will obtain bounds for this operator using restricted bounds. We consider $E \subset \RR^d$, and try and find bounds on the function $S = \mathcal{II} \circ \mathcal{Q} \{ E \}$. Since the operators have compact kernels, we may assume $E$ is contained in a large ball about the origin of radius $O(1)$.




%
and so
%
\begin{align*}
    \langle S_{x_0,t_0}, S_{x_1,t_1} \rangle &= \int b_{t_0}(t) \overline{b}_{t_1}(t') \Big\langle T_W(t) \{ E_{x_0} \}, T_W(t') \{ E_{x_1} \} \Big\rangle\; dt\; dt'\\
    &= \int b_{t_0}(t) \overline{b}_{t_1}(t') \langle T_W^*(t') \circ T_W(t) \{ E_{x_0} \}, E_{x_1} \rangle\; dt\; dt'\\
    &= \int b_{t_0}(t) \overline{b}_{t_1}(t') \Big[ \langle T_W(t-t') E_{x_0}, E_{x_1} \rangle + \langle T_{A'}(t,t') E_{x_0}, E_{x_1} \rangle \Big]\; dt\; dt'\\
    &= \int (b_{t_0} * \overline{b}_{t_1})(t) \langle T_W(t) E_{x_0}, E_{x_1} \rangle\; dt\\
    &\quad\quad+ \int b_{t_0}(t) \overline{b}_{t_1}(t') \langle T_{A'}(t,t') E_{x_0}, E_{x_1} \rangle\; dt\; dt'.
\end{align*}
%
Write
%
\[ G_{x_0,x_1}(t,t') = \langle T_{A'}(t,t') E_{x_0}, E_{x_1} \rangle. \]
%
Then $G_{x_0,x_1}$ satisfies estimates of the form
%
\[ \Big| \partial_t^k \partial_{t'}^l G_{x_0,x_1} \Big| \lesssim_{k,l} 1 \]


\[ \int b_{t_0}(t) \overline{b}_{t_1}(t') \langle T_{A'}(t,t') E_{x_0}, E_{x_1} \rangle\; dt\; dt' \]


%
Smoothness should guarantee that the term corresponding to $\langle T_R E_{x_0}, E_{x_1} \rangle$ is negligible, so we conclude that
%
\begin{align*}
    \langle f_{x_0,t_0}, f_{x_1,t_1} \rangle &\approx \int b_{t_0}(t) \overline{b_{t_1}(t')} \langle T_W(t-t') E_{x_0}, E_{x_1} \rangle\; dt\; dt'\\
    &= \int (b_{t_0} * b_{t_1}^*)(s) \langle T_W(s) E_{x_0}, E_{x_1} \rangle\; ds.
\end{align*}
%
an integral supported for $s$ in a $O(1/R)$ neighborhood of $t_0-t_1$. Let us write $c_{t_0,t_1} = b_{t_0} * b_{t_1}^*$. We have good control over the $L^p$ norms of $b_{t_0}$ and $b_{t_1}$, and so by Young's inequality, we have good control over the $L^q$ norm of $c_{t_0,t_1}$, where $q = p/(2-p)$; As we get close to the optimal range, $q$ increases until it gets arbitrarily close to $d$. However, since $c_{t_0,t_1}$ has compact support, we can decrease $q$ as a tradeoff for powers of $R$, and we can increase $q$ as a tradeoff against powers of $R$ since $c_{t_0,t_1}$ has Fourier transform supported on a ball of radius $R$.

To further study this quantity, we have
%
\[ \langle T_W(t-t') E_{x_0}, E_{x_1} \rangle = \sum_\theta \langle T_{W,\theta}(t-t') E_{x_0}, E_{x_1} \rangle, \ \langle T_W(t) E_{x_0}, E_{x_1} \rangle\; dt + \int b_{t_0}(t) \overline{b}_{t_1}(t') \langle T_{A'} E_{x_0}, E_{x_1} \rangle \]
%
where $\theta$ ranges over all coordinate charts whose domains contain $E_{x_0}$ and $E_{x_1}$ (the other charts we may ignore, by the support properties of the operators making up the Lax parametrix).  If we let $E_{x_0}^\theta$ and $E_{x_0}^\theta$ denote the images of these two sets in the coordinate chart, then we can write
%
\[ \langle T_{W,\theta}(s) E_{x_0}, E_{x_1} \rangle = \int_{E_{x_0}^\theta} \int_{E_{x_1}^\theta} \int a_\theta(s,x,y,\xi) \chi(\xi / R) e^{2 \pi i \phi_\theta(s,x,y,\xi)}\; d\xi\; dx\; dy. \]
%
For each $y$, consider a smooth function $F_y$ defined on $\text{supp}_x a_\theta$, such that
%
\[ \exp_y(F_y^{-1}(x)) = x, \]
%
where $\exp_y$ is the geodesic flow emerging from $y$. If
%
\[ a'_\theta(s,z,y,\xi) = |\det(DF_y(z))| a_\theta(s,F_y(z),y,\xi), \]
%
then we can rewrite the oscillatory sum above as
%
\[ R^d \int_{E_{x_0}} \int_{F_y^{-1}(E_{x_0})} \int a'_\theta(s,z,y,R \xi) \chi(\xi) e^{2 \pi i R \phi_\theta(s,F_y(z),y,\xi)}\; d\xi\; dz\; dy. \]
%
Let us now perform a polar coordinate decomposition in $\xi$, i.e. writing $\xi = \rho \omega$, where $\rho = |\xi|_{g(y)}$ and $\omega$ ranges over $\Sigma_y$. Writing $\alpha(y,\omega) = \langle n(y,\omega), \omega \rangle$, then
%
\[ a''_\theta(s,z,y,\rho,\omega) = \alpha(y,\omega) a'_\theta(s,z,y,\rho \omega), \]
%
then we obtain that the sum is
%
\[ R^d \int_{E_{x_0}} \int_{F_y^{-1}(E_{x_0})} \int \int \rho^{d-1} a''_\theta(s,z,y,R \rho, \omega) \chi(\rho \omega) e^{2 \pi i R \rho \phi_\theta(s,F_y(z),y,\omega)}\; d\omega\; d\rho\; dz\; dy. \]
%
Applying the principle of nonstationary phase in the $\omega$ variable, which for any value of $y$, $z$, and $\rho$, is only stationary when $\omega = z / |z|$, we conclude that there is a symbol $\tilde{a}_\theta$ of order $-(d-1)/2$ such that the oscillatory integral is
%
\[ R^d \int_{E_{x_0}} \int_{F_y^{-1}(E_{x_0})} \int \rho^{d-1} \tilde{a}_\theta(s,z,y,R \rho) \chi(\rho) e^{2 \pi i R \rho (s + |z|)}\; ds\; dz\; dy. \]
%
Imagining that $\tilde{a}_\theta(s,z,y, R \rho) \approx \eta(s,z,y) (R \rho)^{-(d-1)/2}$, we conclude that the integral should behave like
%
\[ R^{\frac{d+1}{2}} \int_{E_{x_0}} \int_{F_y^{-1}(E_{x_0})} \eta(s,z,y) (\partial^{\frac{d-1}{2}} \widehat{\chi})(R(s + |z|))\; dz\; dy \]
%
TODO: Integrate by parts in $z$ now? IS THE FACT THAT WE ABSORBED THE SYMBOL INTO $\eta$ BAD, i.e. so that differentiating in $z$ orthogonally behaves differently? If we now reintroduce the integral in time, we have to now analyze
%
\[ R^{\frac{d+1}{2}} \int \int_{E_{x_0}} \int_{F_y^{-1}(E_{x_0})} c_{t_0,t_1}(s) \eta(s,y,z) (\partial^{\frac{d-1}{2}} \widehat{\chi} )(R(s + |z|))\; dz\; dy\; ds. \]
%
TODO: Does integrating by parts in the $t$ and $t'$ variable do anything for us? Is this where we need oscillation. If $|t_0 - t_1 + d_g(x_0,x_1)| \geq 10/R$, then we conclude this quantity is
%
\begin{align*}
    & O_N \Big( R^{\frac{d+1}{2}-N} \int \int_{E_{x_0}} \int_{F_y^{-1}(E_{x_0})} |c_{t_0,t_1}(s)| |\eta(s,y,z)| (s + |z|)^{-N}\; dz\; dy\; ds \Big)\\
    &\quad\quad O_N \Big( R^{\frac{d+1}{2} - N} |t_0 - t_1 + d_g(x_0,x_1)|^{-N} |E_{x_0}| |E_{x_1}| \| c_{t_0,t_1} \|_{L^1} \Big)
\end{align*}
%
If $|E_{x_0}|$ and $|E_{x_1}|$ are both proportional to $1/R^d$, then this quantity is
%
\begin{align*}
    & O_N \Big( R^{-\frac{3d-1}{2} - N} |t_0 - t_1 + d_g(x_0,x_1)|^{-N} \| c_{t_0,t_1} \|_{L^1} \Big)\\
    &\quad = O_N \Big( R^{-\frac{3d-1}{2} - 1/p^* - N} (t_0 - t_1 + d_g(x_0,x_1))^{-N} \| b_{t_0} \|_{L^p} \| b_{t_1} \|_{L^p} \Big)\\
    &\quad = O_N \Big( t_0^{s_p} t_1^{s_p} R^{-\frac{3d-1}{2} - 1/p^* - N} (t_0 - t_1 + d_g(E_{x_0}, E_{x_1}))^{-N} \Big)
    % -3d/2 + 1/2 - N
    % s_p = (d-1)(1/p - 1/2)
    % 
\end{align*}
% << (3d/2 + N - d + 1) = ((d+1)/2) + N)
% |c| << |c|_{L^q} R^{2(1/p-1)}

\end{comment}

Now applying this Lemma with $c = c_{t_0,t_1}$, and then summing in $\alpha$, we complete the proof of Lemma \ref{mainOrthogonalityLemma}.


\section{Analysis of $II_R$: $L^2$ Estimates}

Lemma \ref{mainOrthogonalityLemma} of the last section implies two functions ${S\!}_{x_0,t_0}$ and ${S\!}_{x_1,t_1}$ can only be correlated in $L^2$ if $d_g(x_0,x_1) \approx |t_0 - t_1|$. We now exploit this geometry to obtain some $L^2$ estimates for sums of the functions ${S\!}_{x_0,t_0}$.

\begin{lemma}
    Fix $u \geq 1$. Consider a set $\mathcal{E} \subset \mathcal{X} \times \mathcal{T}$. Write
    %
    \[ \mathcal{E} = \bigcup_{k = 0}^\infty \mathcal{E}_k, \]
    %
    where $\mathcal{E}_k = \{ (x,t) \in \mathcal{E}: |t| \sim 2^k / R \}$. Suppose that each of the sets $\mathcal{E}_k$ has density type $(Ru,2^k / R)$, i.e. for any set $B \subset \mathcal{X} \times \mathcal{T}$ with $\text{diam}(B) \leq 2^k / R$,
    %
    \[ \#( \mathcal{E}_k \cap B ) \leq Ru\; \text{diam}(B). \]
    %
    Then
    %
    \[ \left\| \sum_k \sum_{(x_0,t_0) \in \mathcal{E}_k} 2^{k \frac{d-1}{2}} {S\!}_{x_0,t_0} \right\|_{L^2(S^d)}^2 \lesssim R^{d-2} \log_2(u) u^{\frac{2}{d-1}} \sum_k 2^{k(d-1)} \# \mathcal{E}_k. \]
\end{lemma}

Write $F = \sum F_k$, where
%
\[ F_k = 2^{k \frac{d-1}{2}} \sum_{(x_0,t_0) \in \mathcal{E}} {S\!}_{x_0,t_0}. \]
%
Applying Cauchy-Schwartz, we have
%
\[ \| F \|_{L^2(S^d)}^2 \lesssim \log_2(u) \left( \sum_{k \lesssim \log_2(u)} \| F_k \|_{L^2(S^d)}^2 + \| \sum_{k \gtrsim \log_2(u)} F_k \|_{L^2(S^d)}^2 \right). \]
%
Without loss of generality by increasing the implicit constant, we can assume that $\{ k : \mathcal{E}_k \neq \emptyset \}$ is $10$-separated, and that all values of $t$ with $(x,t) \in \mathcal{E}$ are positive (the case where all values of $t$ being negative being treated analogously). Thus if $F_k$ and $F_{k'}$ are both nonzero, then $k = k'$ or $|k - k'| \geq 10$. For $k \geq k' + 10$, let us estimate $\langle F_k, F_{k'} \rangle$. We can decompose this inner product into a sum of quantities of the form $2^{k \frac{d-1}{2}} 2^{k' \frac{d-1}{2}} \langle {S\!}_{x_0,t_0}, {S\!}_{x_1,t_1} \rangle$, where $t_0 \sim 2^k / R$ and $t_1 \sim 2^{k'} / R$. Now consider the two sets
%
\[ \mathcal{G}_{x_0,t_0,\text{low}} = \{ (x_1,t_1) \in \mathcal{E}_{k'} : |d_g(x_0,x_1) - (t_0 - t_1)| \lesssim 2^{k' + 10} / R \} \]
%
and for $l \geq k' + 10$, consider the set
%
\[ \mathcal{G}_{x_0,t_0,l} = \{ (x_1,t_1) \in \mathcal{E}_{k'} : |d_g(x_0,x_1) - (t_0 - t_1)| \sim 2^l / R \}. \]
%
Let us use the density properties of $\mathcal{E}$ to control the size of these index sets. First, note that for any $(x_0,t_0) \in \mathcal{E}_k$ and $(x_1,t_1) \in \mathcal{E}_{k'}$, $t_0 - t_1$ lies in a radius $O(2^{k'} / R)$ interval centered at $t_0$:
%
\begin{itemize}
    \item Let us first estimate interactions between the functions ${S\!}_{x_0,t_0}$ and ${S\!}_{x_1,t_1}$ with $(x_1,t_1) \in \mathcal{G}_{x_0,t_0,\text{low}}$. If $(x_1,t_1) \in \mathcal{G}_{x_0,t_0,\text{low}}$, then $x_1$ must lie in a width $O(2^{k'} / R)$ and radius $O(2^k / R)$ annulus centered at $x_0$. Thus $\mathcal{G}_{x_0,t_0,\text{low}}$ is covered by $O( 2^{(k-k')(d-1)} )$ balls of radius $2^{k'} / R$. The density properties of $\mathcal{E}_{k'}$ implies that
    %
    \[ \# \mathcal{G}_{x_0,t_0,l} \lesssim Ru\; 2^{(k-k')(d-1)} (2^{k'} / R) = u 2^{(k-k')(d-1) + k'}. \]
    %
    Together with Lemma \ref{mainOrthogonalityLemma}, we conclude that
    %
    \[ 2^{k \frac{d-1}{2}} 2^{k' \frac{d-1}{2}} \sum_{(x_1,t_1) \in \mathcal{G}_{x_0,t_0,\text{low}}} |\langle {S\!}_{x_0,t_0}, {S\!}_{x_1,t_1} \rangle| \lesssim_M R^{d-2} 2^{k \frac{d-1}{2}} 2^{k' \frac{d-1}{2}} \Big( u 2^{(k-k')(d-1) + k'} \Big) \Big( 2^{-k \frac{d-1}{2}} \Big). \]
    %
    We can now sum over $\log_2(u) \lesssim k' \leq k - 10$ and $(x_0,t_0) \in \mathcal{E}_k$ to find
    % d = 2: get an extra O( log k ) factor
    \[ 2^{k \frac{d-1}{2}} 2^{k' \frac{d-1}{2}} \sum_{(x_0,t_0) \in \mathcal{E}_k} \sum_{k' \leq k - 10} \sum_{(x_1,t_1) \in \mathcal{G}_{x_0,t_0,\text{low}}} |\langle {S\!}_{x_0,t_0}, {S\!}_{x_1,t_1} \rangle| \lesssim R^{d-2} 2^{k (d-1)} \# \mathcal{E}_k. \]

    \item Next, let's estimate interactions between the functions ${S\!}_{x_0,t_0}$ and ${S\!}_{x_1,t_1}$ with $(x_1,t_1) \in \mathcal{G}_{x_0,t_0,l}$ with $k' + 10 \leq l \leq k - 5$. If $(x_1,t_1) \in \mathcal{G}_{x_0,t_0,l}$, then $x_1$ must lie in one of two geodesic annuli centered at $x_0$, each width $O(2^l/R)$ and radii $O(2^k / R)$. Thus $\mathcal{G}_{x_0,t_0,l}$ is covered by $O( 2^{(l-k')} 2^{(k-k')(d-1)} )$ balls of radius $2^{k'} / R$, and the density of $\mathcal{E}_{k'}$ implies that
    %
    \[ \# \mathcal{G}_{x_0,t_0,l} \lesssim Ru\; 2^{(l-k')} 2^{(k-k')(d-1)} 2^{k'} / R = u 2^{l} 2^{(k-k')(d-1)}. \]
    %
    Together with Lemma \ref{mainOrthogonalityLemma}, we conclude that
    %
    \[ 2^{k \frac{d-1}{2}} 2^{k' \frac{d-1}{2}} \sum_{(x_1,t_1) \in \mathcal{G}_{x_0,t_0,l}} |\langle {S\!}_{x_0,t_0}, {S\!}_{x_1,t_1} \rangle| \lesssim_M R^{d-2} 2^{k \frac{d-1}{2}} 2^{k' \frac{d-1}{2}} \Big( u 2^{l} 2^{(k-k')(d-1)} \Big) \Big( 2^{-k \frac{d-1}{2}} 2^{-lM} \Big). \]
    %
    Picking $M > 1$, we can sum over $k' + 10 \leq l \leq k - 5$, $\log_2(u) \lesssim k' \leq k - 10$, and $(x_0,t_0) \in \mathcal{E}_k$ to find
    %
    \[ \sum_{(x_0,t_0) \in \mathcal{E}_k} \sum_{k' \leq k - 10} \sum_{k'+10 \leq l \leq k - 5} \sum_{(x_1,t_1) \in \mathcal{G}_{x_0,t_0,l}} 2^{k \frac{d-1}{2}} 2^{k' \frac{d-1}{2}} |\langle {S\!}_{x_0,t_0}, {S\!}_{x_1,t_1} \rangle| \lesssim R^{d-2} 2^{k (d-1)} \# \mathcal{E}_k. \]

    \item Now let's estimate the interactions between the functions ${S\!}_{x_0,t_0}$ and ${S\!}_{x_1,t_1}$ with $(x_1,t_1) \in \mathcal{G}_{x_0,t_0,l}$, for $k + 10 \leq l \leq \log_2 R$, then $x_1$ must lie in a geodesic ball of radius $O(2^l/R)$ centered at $x_0$. Such a ball is covered by $O( 2^{(l-k')d} )$ balls of radius $2^{k'}/R$, and the density of $\mathcal{E}_{k'}$ implies that
    %
    \[ \# \mathcal{G}_{x_0,t_0,l} \lesssim Ru\; 2^{(l-k')d} (2^{k'}/R) = u 2^{(l-k')d} 2^{k'}. \]
    %
    Together with Lemma \ref{mainOrthogonalityLemma}, we conclude that
    %
    \[ 2^{k \frac{d-1}{2}} 2^{k' \frac{d-1}{2}} \sum_{(x_1,t_1) \in \mathcal{G}_{x_0,t_0,l}} |\langle {S\!}_{x_0,t_0}, {S\!}_{x_1,t_1} \rangle| \lesssim_M R^{d-2} 2^{k \frac{d-1}{2}} 2^{k' \frac{d-1}{2}} \Big( u 2^{(l-k')d} 2^{k'} \Big) \Big( 2^{-lM} \Big). \]
    %
    Picking $M > d$, we can sum over $k - 5 \leq l \lesssim \log R$, $\log_2(u) \lesssim k' \leq k - 10$, and $(x_0,t_0) \in \mathcal{E}_k$ to conclude that
    %
    \[ 2^{k \frac{d-1}{2}} 2^{k' \frac{d-1}{2}} \sum_{(x_0,t_0) \in \mathcal{E}_k} \sum_{k' \leq k - 10} \sum_{k-5 \leq l \lesssim \log R} \sum_{(x_1,t_1) \in \mathcal{G}_{x_0,t_0,l}} R^{d-2} |\langle {S\!}_{x_0,t_0}, {S\!}_{x_1,t_1} \rangle| \lesssim R^{d-2}. \]
\end{itemize}
%
Putting these three bounds together, we conclude that
%
\[ \sum_{\log_2(u) \lesssim k' < k} |\langle F_k, F_{k'} \rangle| \lesssim R^{d-2} \sum_k 2^{k (d-1)} \# \mathcal{E}_k. \]
%
In particular, we have
%
\[ \| F \|_{L^2(S^d)}^2 \lesssim \log_2(u) \left( \sum_k \| F_k \|_{L^2(S^d)}^2 + R^{d-2} \sum_k 2^{k (d-1)} \# \mathcal{E}_k \right). \]
%
Next, let us fix some parameter $a$, and decompose $[2^k/R, 2^{k+1}/R]$ into the disjoint union of length $u^a$ intervals
%
\[ I_{k,\mu} = [ 2^k / R + (\mu - 1) u^a / R, 2^k / R + \mu u^a / R] \quad\text{for $1 \leq \mu \leq 2^k/u^a$}, \]
%
and thus considering a further decomposition $\mathcal{E}_k = \bigcup \mathcal{E}_{k,\mu}$ and $F_k = \sum F_{k,\mu}$. As before, increasing the implicit constant in the Lemma, we may assume without loss of generality that the set $\{ \mu: \mathcal{E}_{k,\mu} \neq \emptyset \}$ is $10$-separated. We now estimate
%
\[ \sum_{\mu \geq \mu' + 10} |\langle F_{k,\mu}, F_{k,\mu'} \rangle|. \]
%
For $(x_0,t_0) \in \mathcal{E}_{k,\mu}$ and $l \geq 1$, define
%
\[ \mathcal{H}_{x_0,t_0,l} = \left\{ (x_1,t_1) \in \mathcal{E}_{k,\mu'} : \max(d_g(x_0,x_1), t_0 - t_1) \sim 2^l u^a / R \right\}. \]
%
Then $\bigcup_{l \geq 1} \mathcal{H}_{x_0,t_0,l}$ covers $\bigcup_{\mu \geq \mu' + 10} \mathcal{E}_{k,\mu'}$. The density properties of $\mathcal{E}_{k,\mu'}$ imply that provided that $l \leq k - a \log_2 u + 10$ (so that $2^l u^a / R \leq 2^k / R$),
%
\[ \# \mathcal{H}_{x_0,t_0,l} \lesssim (R u) (2^l u^a / R) = u^{a+1} 2^l \]
%
For $(x_1,t_1) \in \mathcal{H}_{x_0,t_0,l}$, we claim that
%
\[ 2^{k(d-1)} |\langle {S\!}_{x_0,t_0}, {S\!}_{x_1,t_1} \rangle| \lesssim R^{d-2} 2^{k(d-1)} (2^l u^a)^{- \frac{d-1}{2}}. \]
%
Indeed, for such tuples we have
%
\[ d_g(x_0,x_1) \gtrsim 2^l u^a / R \quad\text{or}\quad |d_g(x_0,x_1) - (t_0 - t_1)| \gtrsim 2^l u^a / R, \]
%
and the estimate follows from Lemma \ref{mainOrthogonalityLemma} in either case. Since $d \geq 4$,
%
\begin{align*}
    \sum_{1 \leq l \leq k - a \log_2 u + 10} \sum_{(x_1,t_1) \in \mathcal{H}_{x_0,t_0,l}} 2^{k(d-1)} |\langle {S\!}_{x_0,t_0}, {S\!}_{x_1,t_1} \rangle| &\lesssim R^{d-2} \sum_{1 \leq l \leq k - a \log_2 u + 10} (2^{k(d-1)}) (2^l u^a)^{- \frac{d-1}{2}} (u^{a+1} 2^l)\\
    &\lesssim R^{d-2} \sum_{1 \leq l \leq k - a \log_2 u + 10} 2^{k(d-1)} 2^{-l \frac{d-3}{2}} u^{1 - a \left( \frac{d-3}{2} \right)}\\
    &\lesssim R^{d-2} 2^{k(d-1)} u^{1 - a \left( \frac{d-3}{2} \right)}.
\end{align*}
%
For $l > k - a \log_2 u + 10$, a tuple $(x_1,t_1)$ lies in $\mathcal{H}_{x_0,t_0,l}$ if and only if $d_g(x_0,x_1) \sim 2^l u^a / R$, since we always have
%
\[ |t_0 - t_1| \lesssim 2^k / R \ll 2^l u^a / R. \]
%
We conclude from Lemma \ref{mainOrthogonalityLemma} that
%
\[ 2^{k(d-1)} |\langle {S\!}_{x_0,t_0}, {S\!}_{x_1,t_1} \rangle| \lesssim_M R^{d-2} 2^{k(d-1)} (2^l u^a)^{- M}. \]
%
Now $\mathcal{H}_{x_0,t_0,l}$ is covered by $O( (2^{l-k} u^a)^d )$ balls of radius $2^k / R$, and the density properties of $\mathcal{E}_k$ imply that
%
\[ \# \mathcal{H}_{x_0,t_0,l} \lesssim (Ru) (2^{l-k} u^a)^d ( 2^k / R ) \lesssim u^{1 + ad} 2^{ld} 2^{-k(d-1)}. \]
%
Thus, picking $M > \max(d,1+ad)$, we conclude that
%
\begin{align*}
    \sum_{l \geq k - a \log_2 u + 10} \sum_{(x_1,t_1) \in \mathcal{H}_{x_0,t_0,l}} 2^{k(d-1)} |\langle {S\!}_{x_0,t_0}, {S\!}_{x_1,t_1} \rangle| &\lesssim R^{d-2} \sum_{l \geq k - a \log_2 u + 10} (2^{k(d-1)}) (2^l u^a)^{-M} u^{1 + ad} 2^{ld} 2^{-k(d-1)}\\
    &\lesssim R^{d-2}.
\end{align*}
%
Putting these two bounds together, and then summing over the tuples $(x_0,t_0)$, we conclude that
%
\[ \sum_{\mu \geq \mu' + 10} |\langle F_{k,\mu}, F_{k,\mu'} \rangle| \lesssim R^{d-2} \left( 1 + 2^{k(d-1)} u^{1 - a \left( \frac{d-3}{2} \right)} \right) \# \mathcal{E}_{k,\mu}. \]
%
Now summing in $\mu$, we conclude that
%
\[ \| F_k \|_{L^2(S^d)}^2 \lesssim \sum_\mu \| F_{k,\mu} \|_{L^2(S^d)}^2 + R^{d-2} \left( 1 + 2^{k(d-1)} u^{1 - a \left( \frac{d-3}{2} \right)} \right) \# \mathcal{E}_k. \]
\begin{comment}

But this means that
%
\[ \sum_{(x_1,t_1) \in \mathcal{H}_{x_0,t_0,l}} |\langle {S\!}_{x_0,t_0}, {S\!}_{x_1,t_1} \rangle| \lesssim (2^l u^a)^{- \frac{d-1}{2}} (u^{a+1} 2^l) = u^{1 - a \left( \frac{d-3}{2} \right)} 2^{- l \left( \frac{d-3}{2} \right)}. \]
%
Since $d \geq 4$, we can sum to conclude that
%
\[ \sum_{1 \leq l \leq k - a \log_2 u} \]
Summing over $1 \leq l \leq k - a \log_2 u$, 

\[ \lesssim \frac{1}{(R d_g(x_0,x_1))^{\frac{d-1}{2}}} \langle R | d_g(x_0,x_1) - (t_0 - t_1) | \rangle^{-M} \]


 and $0 \leq l \lesssim R u^{-a}$, define
%
\[ \mathcal{H}_{x_0,t_0,l} = \left\{ (x_1,t_1) \in \mathcal{E}_{k,\mu'} : l(u^a / R) \leq d_g(x_0,x_1) \leq (l+1)(u^a/R) \right\}. \]
%
Note that for $(x_0,t_0) \in \mathcal{E}_{k,\mu}$ and $(x_1,t_1) \in \mathcal{E}_{k,\mu'}$, $t_0 - t_1$ lies in a radius $O(u^a / R)$ interval centered at $(\mu - \mu') (u^a / R)$:
%
\begin{itemize}
    \item For $0 \leq l \leq (\mu - \mu') / 2$, if $(x_1,t_1) \in \mathcal{H}_{x_0,t_0,l}$, then $x_1$ lies in a thickness $O(u^a / R)$, radius $l (u^a / R)$ geodesic annulus centered at $x_0$. Thus $\mathcal{H}_{x_0,t_0,l}$ is covered by $O( \langle l \rangle ^{d-1})$ balls of radius $u^a / R$. Note that we incurred the logarithmic error in $u$ so that we can assume $u^a \leq 2^k$, so that the density properties of $\mathcal{E}_k$ imply that
    %
    \[ \# \mathcal{H}_{x_0,t_0,l} \lesssim (Ru) \langle l^{d-1} \rangle (u^a / R) = u^{a+1} \langle l \rangle^{d-1}. \]
    % We use the logaorithmic error to get u^a <= 2^k
    %
    Together with Lemma \ref{mainOrthogonalityLemma}, we conclude that
    %
    \[ \sum_{(x_1,t_1) \in \mathcal{H}_{x_0,t_0,l}} |\langle {S\!}_{x_0,t_0}, {S\!}_{x_1,t_1} \rangle| \lesssim_M ( u^{a+1} \langle l \rangle^{d-1} ) \left( \langle l u^a \rangle^{- \frac{d-1}{2}} \left( (\mu - \mu') u^a \right)^{-M} \right). \]
    %
    Picking $M \gtrsim d$ and summing over $0 \leq l \leq (\mu - \mu') / 2$, $\mu' \leq \mu - 10$, and $(x_0,t_0) \in \mathcal{E}_{k,\mu}$ gives that
    %
    \[ \sum_{(x_0,t_0) \in \mathcal{E}_{k,\mu}} \sum_{\mu' \leq \mu - 10} \sum_{0 \leq l \leq (\mu - \mu') / 2} \sum_{(x_1,t_1) \in \mathcal{H}_{x_0,t_0,l}'} |\langle {S\!}_{x_0,t_0}, {S\!}_{x_1,t_1} \rangle| \lesssim \# \mathcal{E}_{k,\mu}. \]

    \item For $(\mu - \mu') / 2 \leq l \leq 2 (\mu - \mu')$, if $(x_1,t_1) \in \mathcal{H}_{x_0,t_0,l}$, then $x_1$ lies in a thickness $O(u^a / R)$, radius $O((\mu - \mu') (u^a / R))$ geodesic annulus centered at $x_0$. Thus $\mathcal{H}_{x_0,t_0,l}$ is covered by $O( (\mu - \mu')^{d-1} )$ balls of radius $u^a / R$. The density properties of $\mathcal{E}_k$ imply that
    %
    \[ \# \mathcal{H}_{x_0,t_,l} \lesssim Ru (\mu - \mu')^{d-1} (u^a/R) = u^{a+1} (\mu - \mu')^{d-1}. \]
    %
    Together with Lemma \ref{mainOrthogonalityLemma}, we conclude that
    %
    \begin{align*}
        &\sum_{(x_1,t_1) \in \mathcal{H}_{x_0,t_0,l}} |\langle {S\!}_{x_0,t_0}, {S\!}_{x_1,t_1} \rangle|\\
        &\quad\quad \lesssim_M (u^{a+1} (\mu - \mu')^{d-1}) \left( ((\mu - \mu') u^a )^{- \frac{d-1}{2}} \langle (l - (\mu - \mu')) u^a \rangle^{-M} \right)\\
        &\quad\quad = u^{1 - a \left( \frac{d-3}{2} \right)} (\mu - \mu')^{\frac{d-1}{2}} \langle (l - (\mu - \mu')) u^a \rangle^{-M}.
    \end{align*}
    %
    Picking $M \gtrsim d$ and summing over $(\mu - \mu') / 2 \leq l \leq 2 (\mu - \mu')$, $\mu' \leq \mu - 10$, and $(x_0,t_0) \in \mathcal{E}_{k,\mu}$ gives that
    %
    \[ \sum_{(x_0,t_0) \in \mathcal{E}_{k,\mu}} \sum_{\mu' \leq \mu - 10} \sum_{(\mu - \mu')/2 \leq l \leq 2(\mu - \mu')} \sum_{(x_1,t_1) \in \mathcal{H}_{x_0,t_0,l}'} |\langle {S\!}_{x_0,t_0}, {S\!}_{x_1,t_1} \rangle| \lesssim u^{1 - a \left( \frac{d-3}{2} \right)} \mu^{\frac{d+1}{2}} \# \mathcal{E}_{k,\mu}. \]

    \item For $l > 2(\mu - \mu')$, if $(x_1,t_1) \in \mathcal{H}_{x_0,t_0,l}$, then $x_1$ lies in a geodesic annulus with thickness $O(u^a/R)$ and radius $\gtrsim l u^a / R$, centered at $x_0$. Thus $\mathcal{H}_{x_0,t_0,l}$ is covered by $O(l^{d-1})$ balls of radius $u^a / R$. The density properties of $\mathcal{E}_k$ imply that
    %
    \[ \# \mathcal{H}_{x_0,t_0,l} \lesssim (Ru) l^{d-1} (u^a/R) = u^{a+1} l^{d-1}. \]
    %
    Together with Lemma \ref{mainOrthogonalityLemma}, we conclude that
    %
    \[ \sum_{(x_1,t_1) \in \mathcal{H}_{x_0,t_0,l}} |\langle {S\!}_{x_0,t_0}, {S\!}_{x_1,t_1} \rangle| \lesssim_M (u^{a+1} l^{d-1}) (l u^a)^{-M}. \]
    %
    Picking $M \gtrsim d$ and summing over $l > 2 (\mu - \mu')$, $\mu' \leq \mu - 10$, and $(x_0,t_0) \in \mathcal{E}_{k,\mu}$ gives that
    %
    \[ \sum_{(x_0,t_0) \in \mathcal{E}_{k,\mu}} \sum_{\mu' \leq \mu - 10} \sum_{(\mu - \mu')/2 \leq l} \sum_{(x_1,t_1) \in \mathcal{H}_{x_0,t_0,l}} |\langle {S\!}_{x_0,t_0}, {S\!}_{x_1,t_1} \rangle| \lesssim \# \mathcal{E}_{k,\mu}. \]
\end{itemize}
%
Thus we have, using that $\mu \lesssim 2^k / u^a$, we conclude that
%
\begin{align*}
    \| F_k \|_{L^2(S^d)}^2 &\lesssim \sum_\mu \| F_{k,\mu} \|_{L^2(S^d)}^2 + \sum_\mu \left( 1 + u^{1 - a \left( \frac{d-3}{2} \right)} \mu^{\frac{d+1}{2}} \right) \# \mathcal{E}_{k,\mu}\\
    &= \sum_\mu \| F_{k,\mu} \|_{L^2(S^d)}^2 + \left( 1 + 2^{k \frac{d+1}{2}} u^{1 - a (d-1)} \right) \# \mathcal{E}_k.
\end{align*}
%
\end{comment}
The functions in the sum defining $F_{k,\mu}$ are highly coupled, and it is difficult to use anything except Cauchy-Schwartz to break them apart. Since $\# ( \mathcal{T} \cap I_{k,\mu}) \sim u^a$, if we set $F_{k,\mu} = \sum_{t \in \mathcal{T} \cap I_{k,\mu}} F_{k,\mu,t}$, then we find
%
\[ \| F_{k,\mu} \|_{L^2(S^d)}^2 \lesssim u^a \sum_{t \in \mathcal{T} \cap I_{k,\mu}} \| F_{k,\mu,t} \|_{L^2(S^d)}^2. \]
%
Fortunately, since $\mathcal{X}$ is 1-separated, the functions in $F_{k,\mu,t}$ are quite orthogonal to one another, and so
%
\[ \| F_{k,\mu,t} \|_{L^2(S^d)}^2 \lesssim R^{d-2} 2^{k(d-1)} \# (\mathcal{E}_k \cap (S^d \times \{ t \})). \]
%
But this means that
%
\[ u^a \sum_t \| F_{k,\mu,t} \|_{L^2(S^d)}^2 \lesssim R^{d-2} 2^{k(d-1)} u^a \# \mathcal{E}_{k,\mu}. \]
%
and so
%
\begin{align*}
    \| F_k \|_{L^2(S^d)}^2 &\lesssim \sum_\mu \| F_{k,\mu} \|_{L^2(S^d)}^2 + R^{d-2} \left( 1 + 2^{k(d-1)} u^{1 - a \left( \frac{d-3}{2} \right)} \right) \# \mathcal{E}_k\\
    &\lesssim R^{d-2} \left( 2^{k(d-1)} u^a + (1 + 2^{k(d-1)} u^{1 - a \left( \frac{d-3}{2} \right)} \right) \# \mathcal{E}_k.
\end{align*}
% u^a = u^{1 - a(d-3)/2}
% a ( (d-1)/2 ) = 1
%
Picking $a = 2 / (d-1)$, we conclude that
%
\[ \| F_k \|_{L^2(S^d)}^2 \lesssim R^{d-2} 2^{k(d-1)} u^{\frac{2}{d-1}} \# \mathcal{E}_k. \]
%
Thus, returning to our bound for $F$, we conclude that
%
\[ \| F \|_{L^2(S^d)}^2 \lesssim R^{d-2} \log_2(u) u^{\frac{2}{d-1}} \sum_k  2^{k(d-1)} \# \mathcal{E}_k. \]  
%
This completes the proof of the $L^2$ density bound.

\bibliographystyle{amsplain}
\bibliography{MultipliersOfLaplacianOnSd}

\end{document}
