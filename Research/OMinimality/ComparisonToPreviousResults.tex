\documentclass[12pt]{report}

\usepackage{amsmath}
\usepackage{amssymb}
\usepackage{amsthm}
\usepackage{amsopn}
\usepackage{kpfonts}
\usepackage{graphicx}
\usepackage{kbordermatrix}
\usepackage{tikz}
\usetikzlibrary{arrows, petri, topaths}%
\usepackage{tkz-berge}
\usepackage{multicol}

\usepackage{framed}
\usepackage{mathtools}
\usepackage{float}
\usepackage{subfig}
% \usepackage{cmbright}

\theoremstyle{plain}
\newtheorem{theorem}{Theorem}[chapter]
\newtheorem{lemma}[theorem]{Lemma}
\newtheorem{corollary}[theorem]{Corollary}
\newtheorem{prop}[theorem]{Proposition}
\newtheorem{exercise}{Exercise}[chapter]

\newtheorem*{example}{Example}
\newtheorem*{proof*}{Proof}

\theoremstyle{definition}
\newtheorem*{defi}{Definition}
\newenvironment{definition}
    {\begin{samepage}\begin{framed}\begin{defi}}
    {\end{defi}\end{framed}\end{samepage}}





\usepackage{hyperref} 
\hypersetup{
    colorlinks = true,
    linkcolor = black,
}

\makeatletter
\renewcommand*\env@matrix[1][*\c@MaxMatrixCols c]{%
  \hskip -\arraycolsep
  \let\@ifnextchar\new@ifnextchar
  \array{#1}}
\makeatother

\renewcommand*\contentsname{\hfill Table Of Contents \hfill}

\newcommand{\optionalsection}[1]{\section[* #1]{(Important) #1}}
\newcommand{\deriv}[3]{\left. \frac{\partial #1}{\partial #2} \right|_{#3}}

\title{How does BGZZ Compare to Previous Techniques}

\begin{document}

\maketitle

\chapter{BNW - Convex Hypersurfaces and Fourier Transforms (1988)}

In this paper, Bruna, Nagel, and Wainger prove Fourier-decay estimates for the surface measure $\sigma$ of a $d$-dimensional, smooth, compact hypersurface $S$ of finite type, under the assumption that $S$ is the boundary of an open convex subset of $\RR^{d+1}$. Under the convexity and finite type assumptions, for each $\xi \in \RR^{d+1}$, there are exactly two points $x_-$ and $x_+$ on $S$ such that $\xi$ is normal to the tangent planes of $S$ at $x_-$ and $x_+$, which can be obtained by taking the maximum and minimum values of the function $x \mapsto \xi \cdot x$ on $S$. Fix a smooth, compactly supported function $\chi$ equal to one on a neighborhood of the origin, and use this function to generate a family of bump functions $\{ \chi_{x_0} \}$, where $\chi_{x_0}(x) = \chi(x - x_0)$ is a bump function at $x_0$. Define
%
\[ H(x_0, \lambda) = \widehat{\chi_{x_0} \sigma}(\lambda n(x_0)), \]
%
where $n(x_0)$ is the outward pointing unit vector of $S$ at $x_0$. Using a partition of unity type argument, one can show that
%
\[ \widehat{\sigma}(\xi) = H(x_+, \lambda) + H(x_-, - \lambda) + R(\xi), \]
%
where $R$ is a Schwartz function. The main focus of this paper is proving that
%
\[ H(x_0, \lambda) = a(x_0,\lambda) e^{2 \pi i \lambda n(x_0) \cdot x_0}, \]
%
where $a$ satisfies `symbol estimates' of the form
%
\[ \left| (\partial^\alpha_x \partial^\beta_\lambda a)(x_0, \pm \lambda)\right| \lesssim_{\alpha,\beta} \lambda^{-\beta} \sigma\left( B \left(x_0, 1/\lambda \right) \right). \]
%
This bound uniformly incorporates a number of different results; if $x_0$ is a point of order $k$ on $S$, then $\sigma(B(x_0,1/\lambda)) \lesssim \lambda^{-d/k}$, and so we conclude that $a$ is a symbol of order $-d/k$. But $x_0$ might only be a point of order $k$ because of a single line on which the surface vanishes, and in this case we might still have $B(x_0,1/\lambda) \lesssim \lambda^{-1/k - (d-1)}$, in which case we can conclude that $a$ is a symbol of order $-1/k - (d-1)$. This is a good example of why stationary set estimates are useful over standard stationary phase results, i.e. because they can more robustly handle the geometry of higher dimensional singularities.

How is the result proved? We first reduce the study of the $n$-dimensional oscillatory integral to a one dimensional oscillatory integral by focusing on the radial oscillation first (i.e. integrating in polar coordinates in the radial direction). To control this radial oscillation, we apply an approximation argument to reduce the study to general convex functions of finite type $m$ to the study of convex polynomials of order at most $m$, and then apply a compactness argument (this space of polynomials is finite dimensional) to show that various quantities associated with these polynomials (e.g. the sum of absolute values of the coefficients) upper and lower bound various other quantities -- in particular, the size of the ball $B(x_0,1/\lambda)$.

I wonder how the exploitation of this compactness compares to the exploitation of o-minimality. To present this argument, we can perhaps first focus on convex functions. I'm also interested in what kinds of functions are well approximated by definable functions (e.g. to generalize the argument which approximates a general convex function of finite type with a convex polynomial).

\end{document}