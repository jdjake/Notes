\documentclass[12pt]{report}

\usepackage{amsmath}
\usepackage{amssymb}
\usepackage{amsthm}
\usepackage{amsopn}
\usepackage{kpfonts}
\usepackage{graphicx}
\usepackage{kbordermatrix}
\usepackage{tikz}
\usetikzlibrary{arrows, petri, topaths}%
\usepackage{tkz-berge}
\usepackage{multicol}

\usepackage{framed}
\usepackage{mathtools}
\usepackage{float}
\usepackage{subfig}
% \usepackage{cmbright}

\theoremstyle{plain}
\newtheorem{theorem}{Theorem}[chapter]
\newtheorem{lemma}[theorem]{Lemma}
\newtheorem{corollary}[theorem]{Corollary}
\newtheorem{prop}[theorem]{Proposition}
\newtheorem{exercise}{Exercise}[chapter]

\newtheorem*{example}{Example}
\newtheorem*{proof*}{Proof}

\theoremstyle{definition}
\newtheorem*{defi}{Definition}
\newenvironment{definition}
    {\begin{samepage}\begin{framed}\begin{defi}}
    {\end{defi}\end{framed}\end{samepage}}





\usepackage{hyperref} 
\hypersetup{
    colorlinks = true,
    linkcolor = black,
}

\makeatletter
\renewcommand*\env@matrix[1][*\c@MaxMatrixCols c]{%
  \hskip -\arraycolsep
  \let\@ifnextchar\new@ifnextchar
  \array{#1}}
\makeatother

\renewcommand*\contentsname{\hfill Table Of Contents \hfill}

\newcommand{\optionalsection}[1]{\section[* #1]{(Important) #1}}
\newcommand{\deriv}[3]{\left. \frac{\partial #1}{\partial #2} \right|_{#3}}

\renewcommand\thesection{}

\begin{document}

Fix $q > q_k$. Our goal is to show that
%
\[ \int_{\RR^d} |E_k 1(x)|^q\; dx < \infty. \]
%
Since $\| E_k 1 \|_{L^\infty} \leq 1$, we can write
%
\[ \int_{\RR^d} |E_k 1(x)|^q\; dx = \sum_{j = 1}^\infty \int_{1/2^j \leq |E_k 1(x)| \leq 1/2^{j-1}} |E_k 1(x)|^q\; dx. \]
%
Without the $L^\infty$ norm, we would need to sum over all $j \in \ZZ$ rather than just $j \geq 1$. Using (3.2), the set $\{ x : 1/2^j \leq |E_k 1(x)| \leq 1/2^{j-1} \}$ has measure at most
%
\[ |\log(1 + 2^j)| 2^{jq_k} \lesssim j\; 2^{jq_k}. \]
%
And on the set $\{ x : 1/2^j \leq |E_k 1(x)| \leq 1/2^{j-1} \}$,
%
\[ |E_k 1(x)|^q \leq (1/2^{j-1})^q \lesssim_q 2^{-qj}. \]
%
And so
%
\[ \int_{1/2^j \leq |E_k 1(x)| \leq 1/2^{j-1}} |E_k 1(x)|^q\; dx \lesssim (j\; 2^{jq_k}) 2^{-qj} = j\; 2^{-j(q - q_k)}. \]
%
But for $q > q_k$, we conclude that
%
\[ \int_{\RR^d} |E_k 1(x)|^q\; dx \lesssim \sum_{j = 1}^\infty j\; 2^{-j(q - q_k)} < \infty. \]

\chapter*{Motivation Plus Comparison to Previous Techniques}

	Now that we've proved Theorem 1.1, let's provide some motivation as to why this result is needed. In particular, we'll provide an explanation of the background of the convergence exponent of the `Tarry problem' the paper discusses, and review why this method might potentially work well for solving such a problem.

	\section{Tarry's Problem}

	To understand the Tarry problem, I mainly used (Wooley, 2017), as well as a discussion of Tarry's problem in (Arkhipov, Karatsuba, Chubarikov, 1980) and L. Schoenfeld's review of (Hua, 1952) as a reference, to understand what was meant by the following sentence in (BGZZ, 2021):
	%
	\begin{quote}
		The quantity $\| E_{[0,1]}^{(1,k)} \|_{L^p(\RR^N)}$ appears in the leading coefficient of the asymptotic expansion of Vinogradov's Mean Value$\dots$
	\end{quote}
	%
	Let's start by discussing how these quantities arise in number theory.

	One very useful technique for estimating the number of solutions to certain integer equations in additive number theory, is to utilize orthogonality to reduce such estimates to the study of properties of certain exponential sums. A rudimentary example is obtained by trying to count the number $S(N)$ of solutions to the equation
	%
	\begin{equation} \label{simpleadditiveequation}
		n_1^2 + n_2^2 = m_1^2 + m_2^2
	\end{equation}
	%
	with $x_1,x_2,y_1,y_2 \in [1,N]$. For an integer $n$, let $e_n: [0,1] \to [0,1]$ be the function
	%
	\[ e_n(\xi) = e^{2 \pi i n \xi}. \]
	%
	The main reason such functions are useful in additive number theory is because of three useful properties:
	%
	\[ e_n e_m = e_{n+m}, \quad\quad \overline{e_n} = e_{-n}, \quad\text{and}\quad \int_0^1 e_n(\xi)\; d\xi = \mathbb{I}(n = 0). \]
	%
	Therefore, for any integers $n_1,n_2,m_1,m_2$,
	%
	\[ \int_0^1 e_{n_1^2}(\xi) e_{n_2^2}(\xi) \overline{e_{m_1^2}(\xi)} \overline{e_{m_2^2}(\xi)}\; d\xi = \int_0^1 e_{n_1^2 + n_2^2 - m_1^2 - m_2^2}(\xi)\; d\xi = \mathbb{I}(n_1^2 + n_2^2 = m_1^2 + m_2^2).  \]
	%
	But this means that
	%
	\begin{align*}
		S(N) &= \sum_{n_1,n_2,m_1,m_2 = 1}^N \mathbb{I}(n_1^2 + n_2^2 = m_1^2 + m_2^2)\\
		&= \sum_{n_1,n_2,m_1,m_2 = 1}^N \int_0^1 e_{n_1^2}(\xi) e_{n_2^2}(\xi) \overline{e_{m_1^2}(\xi) e_{m_2^2}(\xi)}\; d\xi\\
		&= \int_0^1 \left( \sum_{n_1 = 1}^N e_{n_1^2}(\xi) \right) \left( \sum_{n_2 = 1}^N e_{n_2^2}(\xi) \right) \overline{\left( \sum_{m_1 = 1}^N e_{m_1^2}(\xi) \right) \left( \sum_{m_2 = 1}^N e_{m_2^2}(\xi) \right)}\; d\xi\\
		&= \int_0^1 \left( \sum_{n = 1}^N e_{n^2}(\xi) \right) \left( \sum_n^N e_{n^2}(\xi) \right) \overline{\left( \sum_{n = 1}^N e_{n^2}(\xi) \right) \left( \sum_{n = 1}^N e_{n^2}(\xi) \right)}\; d\xi.
	\end{align*}
	%
	We can then write this integral as
	%
	\[ \int_0^1 \left| f_N(\xi) \right|^4\; d\xi \quad\text{where}\ f_N = \sum_{n = 1}^N e_{n^2}. \]
	%
	Thus computing the number of solutions to the equation \eqref{simpleadditiveequation} reduces to the study of the $L^4$ norm of a certain exponential sum $f_N$.

	\emph{Tarry's problem} involves studying a more complicated equation, namely the number $J_{s,k}(N)$ of integer tuples $(n_1,\dots,n_s,m_1,\dots,m_s)$ of solutions to the system of equations
	%
	\begin{align*}
		n_1 + \dots + n_s &= m_1 + \dots + m_s\\
		n_1^2 + \dots + n_s^2 &= m_1^2 + \dots + m_s^2\\
		\cdots \cdots \cdots \cdots & \cdots \cdots \cdots \cdots \cdots\\
		n_1^k + \dots + n_s^k &= m_1^k + \dots + m_s^k.
	\end{align*}
	%
	Such equations can be dealt with similarily; Defining $f_N: [0,1]^k \to \CC$ by setting
	%
	\[ f_N(\xi) = \sum_{n = 1}^N e^{2 \pi i (n\xi_1 + n^2 \xi_2 + \dots + n^k \xi_k)}, \]
	%
	then we find that
	%
	\begin{equation} \label{discretJskEquation}
		J_{s,k}(N) = \int_{[0,1]^k} |f_N(\xi)|^{2s}\; d\xi.
	\end{equation}
	%
	Exponential sums are often harder to deal with than exponential integrals. Fortunately, using the Hardy-Littlewood circle method (e.g. performing a decomposition into major and minor arcs), one can asymptotically replace the exponential sum above with an oscillatory integral. These calculations can be found in the proof of Corollary 1.3 of (Wooley, 2017), though these calculations will be terse if you're unfamiliar with the circle method, and since the method doesn't really impact an understanding of (BGZZ, 2021), it's probably best for the purpose of this study guide to treat the method as a black box. The main consequence is that as $N \to \infty$, we have a formal identity
	%
	\[ J_{s,k}(N) = \mathfrak{J}_s(k) \mathfrak{S}_s(k) \cdot N^{2s - \frac{k(k+1)}{2}} + o \left( N^{2k - \frac{n(n+1)}{2}} \right), \]
	%
	where $\mathfrak{S}_s$ is a formal trigonometric series (hence the use of the Fraktur $S$), and $\mathfrak{J}_s$ is defined in terms of an $L^p$ norm of an oscillatory integral over the infinite region $\RR^s$. Thus neither function is defined for all values of $k$\footnote{TODO: What happens to the asymptotics in these circumstances}. We'll ignore $\mathfrak{S}_s$, since it requires different techniques to deal with than a study of oscillatory integrals\footnote{TODO: The convergence properties of $\mathfrak{S}_s$ are completely known, in Hua 1959, but what about the higher dimensional variants?}. Here we're interested in the convergence properties of $\mathfrak{J}_s$, because it is equal to
	%
	\begin{equation} \label{JsSingularEquation}
		\int_{\RR^k} \left| \int_0^1 e^{2 \pi i (\xi_1 x + \xi_2 x^2 + \dots + \xi_k x^k)}\; dx \right|^{2s}\; d\xi,
	\end{equation}
	%
	Notice that the cost of replacing the exponential sum in $n$ in \eqref{discretJskEquation} with an oscillatory integral in $x$ in \eqref{JsSingularEquation} is that instead of integrating $\xi$ over $[0,1]^k$, we must now integrate $\xi$ over $\RR^k$. Determining whether \eqref{JsSingularEquation} is finite thus requires us to obtain good bounds on the family of oscillatory integrals
	%
	\begin{equation} \label{oscillatoryEquation}
		I(\xi) = \int_0^1 e^{2 \pi i (\xi_1 x + \dots + \xi_k x^k)}\; dx.
	\end{equation}
	%
	Namely, this problem is equivalent to showing that $I \in L^p(\RR^k)$ for $p = 2s$. The problem is classical, and has been completely solved in (Arkhipov, Karatsuba, Chubarikov, 1980), so we're interested in a higher dimensional variant, namely, for a given $k$ and $d$, to determine which $L^p$ spaces the function
	%
	\begin{equation} \label{higherdimensionaloscillatoryequation}
		I(\xi) = \int_{[0,1]^d} e^{2 \pi i \phi_\xi(x)}\; dx
	\end{equation}
	%
	lies in, where $\xi = \{ \xi_\beta : |\beta| \leq k \}$ is a $D$-dimensional vector of values, which we take as coefficients to the polynomial phase function $\phi_\xi(x) = \sum \xi_\beta x^\beta$. This probably has consequences for determining the number of solutions to the system of equations
	%
	\[ \{ n_1^\alpha + \dots + n_s^\alpha = m_1^\alpha + \dots + m_s^\alpha : |\alpha| \leq k \}, \]
	%
	where $n_1,\dots,n_s,m_1,\dots,m_s$ are $d$-dimensional integer vectors, but I couldn't find a source on this, and I'm not sure how well the Hardy-Littlewood circle method generalizes to handling higher dimensional oscillatory sums.

	The difficulty in understanding $I$ is that we must obtain bounds for oscillatory integrals with polynomial phases which are stable under an arbitrary pertubation of the coefficients of the polynomials. The methods and heuristics we normally use to study such oscillatory integrals do not often behave as nicely with respect to such a pertubation; our heuristics are normally developed on integrals of the form
	%
	\[ J(\lambda) = \int_{[0,1]^d} e^{2 \pi i \lambda \phi(x)}\; dx \]
	%
	for a \emph{fixed phase} $\phi$, only varying the value of $\lambda$ to try and obtain an asymptotic understanding of the integral as $\lambda \to \infty$. We can relate the more general pertubation in \eqref{higherdimensionaloscillatoryequation} with this formulation by using polar coordinates, i.e. we can write \eqref{higherdimensionaloscillatoryequation} as
	%
	\[ \int_0^\infty \int_{|\xi| = 1}  \lambda^{D-1}|I(\lambda \xi)|^{2s}\; d\xi\; d\lambda. \]
	%
	Thus, if for each $|\xi_0| = 1$, we can obtain a good understanding of the asymptotic behaviour of $I(\lambda \xi)$ as $\lambda \to \infty$ for $|\xi| = 1$ in a small neighborhood of $\xi_0$, we can use the compactness of the sphere $\{ |\xi| = 1 \}$ to get good control over the overall integral above.

	\section{Pertubations of Oscillatory Integrals}

	Notice that the difficulty in understanding the behaviour of $I$ is that we are understanding a family of oscillatory integrals parameterized by $\xi$, whose phase ranges over \emph{all degree $k$ polynomials}. We must obtain certain uniform bounds over $\xi$, and thus our oscillatory integral bounds \emph{can only rely on properties of polynomials remaining stable under a pertubation of the coefficients}.

	A basic example of the kinds of problems that can occur can be illustrated by trying to naively analyze the family of oscillatory integrals
	%
	\[ I(r,\lambda) = \int_0^1 e^{2\pi i \lambda (rx + x^3)}\; dx. \]
	%
	For $r > 0$, the integral has no stationary points, and so we would expect
	%
	\[ |I(r,\lambda)| \sim \lambda^{-1}. \]
	%
	For $r < 0$, the integral has two stationary points, each non-degenerate, and so stationary phase yields that
	%
	\[ |I(r,\lambda)| \sim \lambda^{-1/2}. \]
	%
	For $r = 0$, the integral has a single stationary point, which is degenerate of order three. Thus stationary phase yields that
	%
	\[ |I(r,\lambda)| \sim \lambda^{-1/3}. \]
	%
	As $r \to 0$, the fact that the decay in $\lambda$ changes qualitatively means that the implicit constants in these equations must become singular as $r \to 0$, and so our heuristics run into problems.

	In this case we can use the Van der Corput Lemma to obtain a more precise bound, namely that for $0 < r < 1$ and $\lambda \geq 1$,
	%
	\[ I(r,\lambda) \lesssim \min \left( \frac{1}{r\lambda}, \frac{1}{\lambda^{1/3}} \right) \]
	%
	and
	%
	\[ I(-r,\lambda) \lesssim \min \left( \frac{1}{r\lambda} + \frac{1}{r^{1/4} \lambda^{1/2}}, \frac{1}{\lambda^{1/3}} \right). \]
	%
	However, the Van der Corput Lemma is \emph{not sharp} for oscillatory integrals in dimensions $d > 1$, so we cannot necessarily fix our techniques in the same way we did here for a one-dimensional oscillatory integral.
	% 3x^2 - r
	% For x << r^{1/2}, 1st derivative is >> r
	% 		So r^{1/2} r^{-3/2} lambda^{-1} = r^{-1} lambda^{-1}
	% And 3rd derivative is >> 1
	% 		So lamdba^{-1/3}
	%
	% For x ~ r^{1/2}, 2nd derivative is >> r^{1/2}
	% 		So r^{1/2} ( r^{-3/4} lambda^{-1/2} ) = r^{-1/4} lambda^{-1/2}
	% 				   3rd derivative is >> 1
	%		So << lambda^{-1/3}
	% For x >> r^{1/2}, 1st derivative is >> 1
	% 		So << lambda^{-1}
	%

	Let's review some of the fundamental techniques to analyzing oscillatory integrals, and how they're not quite equipped to deal with the problem at hand:
	%
	\begin{itemize}
		\item The most basic method of understanding oscillatory integrals occurs when the oscillatory integral has finitely many isolated stationary points. If each of these stationary points is non-degenerate, then using the Morse Lemma, we can obtain asymptotically sharp bounds of the form
		%
		\[ |I(\lambda \xi)| \lesssim \lambda^{-d/2}. \]
		%
		Moreover, this bound is stable under $C^{d/2 + \varepsilon}$ pertubations of the phase for any $\varepsilon > 0$. However, in dimensions greater than two, it is difficult to extend this argument to polynomials with degenerate stationary points, even if these stationary points are isolated, because the pertubations can change the geometry of the stationary set dramatically.

		A more stable approach is obtained by decompoing the region of integration, and applying the Van der Corput Lemma in various different ways on different regions. This approach is rather useful in one dimension, as we saw about for the phase $rx + x^3$ above, and yields the optimal range of $L^p$ estimates for $I$ when $d = 1$ (Arkhipov, Karatsuba, Chubarikov, 1980). But in higher dimension the Van der Corput Lemma gives suboptimal decay rates in general; (Arkhipov, Karatsuba, Chubarikov, 1980) uses variants of these techniques to study the case $d > 1$, but does not obtain optimal bounds using these methods.

		%For this problem, an analysis of the $L^p$ properties of $I$ is classical; the calculations can be found in the section of (Arkhipov, Karatsuba, Chubarikov, 1980) entitled ``Exponent of convergence of singular integrals in the Tarry problem''. They show that $I$ lies in $L^p(\RR^k)$ for $p \in (\frac{k(k+1)}{2} + 1, \infty]$, and does not lie in $L^p(\RR^k)$ for $p \in [1, \frac{k(k+1)}{2}]$. At a first glance, it seems there methods are essentially more robust versions of the Van der Corput Lemma (though there's some polynomial interpolation trickery I don't understand). My heuristic here is that Van der Corput works here because derivative bounds \emph{are stable under pertubation}.

		\item For any real analytic phase $\phi$ defined in a neighborhood of the origin, one can find the order of growth of the oscillatory integral by performing a `resolution of singularities'. Consider the \emph{Newton polyhedron} of $\phi(x)$, obtained by performing a Taylor expansion $\phi(x) = \sum c_\alpha x^\alpha$ about the origin (this will only be a finite sum in the polynomial case), and defining the Newton polyhedron $N$ to be the convex hull of the union of the sets $Q_\alpha = [\alpha_1, \infty) \times \dots \times [\alpha_,d \infty)$ for each $\alpha$ with $c_\alpha \neq 0$. If $\beta$ is the smallest value such that $(\beta,\dots,\beta)$ lies in $N$, then we have, as $\lambda \to \infty$,
		%
		\[ \left| \int \chi(x) e^{2 \pi i \lambda \phi(x)}\; dx \right| \lessapprox \lambda^{-1/\beta}. \]
		%
		I'm not sure how well this result works when $\chi$ is replaced with an indicator function (the assumption in Stein's Harmonic analysis book is that $\chi$ salready must have small support near a particular point). Furthermore, there are examples that show that, in general $\beta$ is \emph{not} stable under polynomial pertubations\footnote{TODO: Think about the example on page 2 of (Gressman, 2018)}, though there are hints that under certain conditions, the bounds are stable (Phong, Stein, Sturm, 1999).
	\end{itemize}
	%
	Instead of these methods, the technique discussed in this paper is a \emph{stationary set method}. Roughly, this means obtaining a bound of the form
	%
	\[ \int e^{2 \pi i \lambda \phi(x)}\; dx \lesssim \max_{\mu} \Big| \{ x : \mu \leq \phi(x) \leq \mu + 1/\lambda \} \Big|, \]
	%
	where the implicit constant is stable under pertubation of $\phi$. In general, obtaining good control on the right hand side can be tricky (I tried using a bound like this to obtain estimates on a phase $\phi(x) = rx + x^3$, and it got very calculation heavy). But the results obtained seem robust, and there are various results, like the Esseen concentration theorem (see Section 7.3 of Tao, Additive Combinatorics for a further discussion), which says that for each $\xi$,
	%
	\[ \max_\mu \Big| \{ x : \mu \leq \phi(x) \leq \mu + 1/\lambda \} \Big| \lesssim_d \int_0^{\lambda} |I(\tau \xi)|\; d\tau, \]
	%
	where the implicit constant is uniform in the phase. The result proved in Section 1.1 of (BGZZ, 2021) seems to be a form of this equation. Thus we might expect that stationary set estimates are sharp, \emph{especially when dealing with means of oscillatory integrals}, like on the right hand side of the equation above. A major advantage is that the right hand side is not `committed' to any decay of the form $\lambda^{-t}$ for any fixed $t$, unlike the Newton polyhedron method discussed above, which can help in the analysis of pertubations of phases that cause different decay exponents to show up, like in the analysis of $rx + x^3$ we did above.

	In it's discussion of stationary set estimates, (BGZZ,2021) lists one other example of a stationary set method, the result of (Bruna, Nagel, and Wainger, 1988) which we'll now discuss. In this paper, Bruna, Nagel, and Wainger prove Fourier-decay estimates for the surface measure $\sigma$ of a $d$-dimensional, smooth, compact hypersurface $S$ of finite type, under the assumption that $S$ is the boundary of an open convex subset of $\RR^{d+1}$. Under the convexity and finite type assumptions, for each $\xi \in \RR^{d+1}$, there are exactly two points $x_-$ and $x_+$ on $S$ such that $\xi$ is normal to the tangent planes of $S$ at $x_-$ and $x_+$, which can be obtained by taking the maximum and minimum values of the function $x \mapsto \xi \cdot x$ on $S$. Fix a smooth, compactly supported function $\chi$ equal to one on a neighborhood of the origin, and use this function to generate a family of bump functions $\{ \chi_{x_0} \}$, where $\chi_{x_0}(x) = \chi(x - x_0)$ is a bump function at $x_0$. Define
	%
	\[ H(x_0, \lambda) = \widehat{\chi_{x_0} \sigma}(\lambda n_{x_0}), \]
	%
	where $n_{x_0}$ is the outward pointing unit vector of $S$ at $x_0$. Using a partition of unity type argument, one can show that
	%
	\[ \widehat{\sigma}(\xi) = H(x_+, \lambda) + H(x_-, - \lambda) + R(\xi), \]
	%
	where $R$ is a Schwartz function, which is treated as neglible for determining the behaviour of the Fourier transform for large $\xi$. The main focus of this paper is proving that
	%
	\[ H(x_0, \lambda) = a(x_0,\lambda) e^{2 \pi i \lambda n_{x_0} \cdot x_0}, \]
	%
	where $a$ satisfies `symbol estimates'\footnote{Can BGZZ obtain symbol estimates, or only decay estimates?} of the form
	%
	\[ \left| (\partial^\alpha_x \partial^\beta_\lambda a)(x_0, \pm \lambda)\right| \lesssim_{\alpha,\beta} \lambda^{-\beta} \sigma\left( B \left(x_0, 1/\lambda \right) \right). \]
	%
	The right hand side is a form of a stationary set estimate in disguise; if we can write $S$ as $\{ x_n = \psi(x') \}$ for some convex function $\psi$ with $(\nabla \psi)(0) = 0$, where $x' = (x_1,\dots,x_{d-1})$ (which is always locally possible after a rotation and translation of the coordinate system), we can write
	%
	\[ H(x_0,\lambda) = \int \tilde{\chi}(x') e^{-2 \pi i \lambda \psi_{x_0}(x')}\; dx', \]
	%
	where $\tilde{\chi}$ is some smooth bump function, and the fact that $(\nabla \psi)(0) = 0$ implies that the maximum measure of the stationary set $\{ x : \mu \leq \psi(x') \leq \mu + 1/\lambda \}$ will be proportional to $\sigma \left( B(x_0,1/\lambda) \right)$. As we vary $x_0$, $\psi_{x_0}$ is perturbed do a different convex function, but the results here are stable under such pertubations, depending only on the convexity of the phase, uniform upper bounds on the derivatives of $\psi$, and lower bounds on $|\sum_{|\alpha| \leq m} (\partial^\alpha \psi)(0)|$ for a fixed $m$.

	%This bound uniformly incorporates a number of different results; if $x_0$ is a point of order $k$ on $S$, then $\sigma(B(x_0,1/\lambda)) \lesssim \lambda^{-d/k}$, and so we conclude that $a$ is a symbol of order $-d/k$. But $x_0$ might only be a point of order $k$ because of a single line on which the surface vanishes, and in this case we might still have $B(x_0,1/\lambda) \lesssim \lambda^{-1/k - (d-1)}$, in which case we can conclude that $a$ is a symbol of order $-1/k - (d-1)$. This is a good example of why stationary set estimates are useful over standard stationary phase results, i.e. because they can more robustly exploit the geometry of phase functions which may qualitatively change as we vary a given parameter (this mainly occurs in the presence of singularities).	

	%How is the result proved. We first fix $m > 0$, and a small $\delta > 0$, and we study convex one-dimensional functions $\varphi: \RR \to \RR$ which have the property that their Taylor series at the origin has the form
	%
	%\[ \varphi(x) = \sum_{n = 2} a_n x^n + O(x^{n+1}). \]
	%
	%where the remainder term is bounded uniformly by 

	How is the result proved? We first reduce the study of the $n$-dimensional oscillatory integral to a one dimensional oscillatory integral by focusing on the radial oscillation first (i.e. integrating in polar coordinates in the radial direction). To control this radial oscillation, we apply an approximation argument to reduce the study to general convex functions of finite type $m$ to the study of convex polynomials of order at most $m$, and then apply a compactness argument (this space of polynomials is finite dimensional) to show that various quantities associated with these polynomials, like the sum of absolute values of the coefficients, and the size of the ball $B(x_0,1/\lambda)$, are all equivalent norms on this finite dimensional space \footnote{TODO: I wonder how the exploitation of this compactness compares to the exploitation of o-minimality, i.e. which gives finitely monotonic intervals}.


	%To present this argument, we can perhaps first focus on convex functions. I'm also interested in what kinds of functions are well approximated by definable functions (e.g. to generalize the argument which approximates a general convex function of finite type with a convex polynomial).	

	\begin{comment}


	\chapter{AKC - Trigonometric Integrals (1980)}

	Let's list the results of AKC, which give some Van der Corput type estimates for one dimensional integrals:
	%
	\begin{itemize}
		\item If $n > 1$ and $|\partial^n \phi| \geq A$, then
		%
		\[ J(1,\phi) \leq \min(|I|, 6nA^{-1/n}). \]
		%
		If $a$ is piecewise monotonic and continuous (on at most $p$ pieces), with $L^\infty$ norm at most $H$< then
		%
		\[ J(a,\phi) \leq H \min(1, 24pn A^{-1/n}). \]

		\item If $n \geq 1$, and $\phi(x) = \alpha_n x^n + \dots + \alpha_1 x$, and
		%
		\[ H = \min_{x \in I} \sum_r \frac{|(\partial^r \phi)(x)|}{r!}, \]
		%
		then
		%
		\[ J(1,\phi) \leq \min(|I|,6en^3H^{-1}). \]
	\end{itemize}

	\end{comment}








	\end{document}

	% 

	In harmonic analysis, we often wish to show that a `rapidly oscillating integral' of the form
	%
	\[ J(a,\phi) = \int a(x) e^{2 \pi i \lambda \phi(x)}\; dx. \]
	%
	is small\footnote{Often, we also want to find \emph{asymptotic expansions} for some parameterized family of these integrals, but here we mainly focus on tight upper bounds for the time being}. An often useful intuition to analyze such oscillatory integrals is to utilize the smoothness of $\phi$ in combination with it's \emph{nonstationarity}. This is often exploited by some form of integration by parts, which works over the region where $\phi'$ is non-vanishing. It is then often the case that the magnitude of the oscillatory integral is dominated by the behaviour of the integral near points where the integral is stationary, and a decomposition method is then often employed, analyzing the behaviour of the integral on regions away from these `stationary points' and analyzing the behaviour of the integral in a small neighborhood of the stationary points. This is the \emph{principle of stationary phase}.



	Such a method is often sufficient to obtain tight bounds for a large range of oscillatory integrals. But the bounds obtained often depend heavily on the behaviour of the oscillatory integral near the stationary points, and thus give rise to various qualitatively different types of bounds depending on the point analyzed. For instance, consider the family of integrals
	%
	\[ I(r,\lambda) = \int_{-\infty}^\infty a(t) e^{2 \pi i \lambda (rt + t^3)}\; dt, \]
	%
	where $a$ is smooth and compactly supported. A variation of the method of stationary phase (the Van der Corput Lemma) tells us that
	%
	\[ I(0,\lambda) \sim \lambda^{-1/3} \quad\text{as $\lambda \to \infty$}. \]
	%
	For a fixed $r < 0$, a similar analysis shows that
	%
	\[ I(r,\lambda) \sim \lambda^{-1/2} \quad\text{as $\lambda \to \infty$}. \]
	%
	For a fixed $r > 0$, we have that for any $N > 0$,
	%
	\[ I(r,\lambda) \lesssim_N \lambda^{-N} \quad\text{as $\lambda \to \infty$}. \]
	%
	Thus we see three qualitatively different decay behaviours as $\lambda \to \infty$ depending on the value of $r$. As a result, these three different bounds \emph{must depend on $r$}. In other words, as $r \to 0$ from above and below, the implicit constants in the inequalities must blow up when $\lambda$ is large. However, if we can prove a stationary phase estimate
	%
	\[ I(r,\lambda) \lesssim \sup_{x_0} |\{ t \in I : |(rt + t^3) - x_0| \leq 1/\lambda \}|, \]
	%
	which is uniform in $r$, then we can combine these three estimates into a uniform estimate. 

	% { mu - 1 / lambda <= tr + t^3 <= mu + 1/ lambda }
	%
	% For r > 0, tr + t^3 is monotonically increasing, and
	% the magnitude of the derivative is increasing away
	% from the origin. Thus the value of mu which gives the
	% largest such interval is mu = 0.
	%
	% By symmetry, this is twice the value of s such that
	% 	rs + s^3 == lambda^{-1}
	% so
	%
	\[ I(r,\lambda) \lesssim \frac{(9 / \lambda + \sqrt{3} \sqrt{27/\lambda^2 + 4r^3})^{2/3} - 2^{2/3} 3^{1/3} r}{(9/\lambda + \sqrt{3} \sqrt{27/\lambda^2 + 4r^3})^{1/3}} \]
	%
	For $\lambda \gtrsim r^{-3/2}$,
	%
	\[ I(r,\lambda) \lesssim \frac{1}{\lambda r}, \]
	%
	and for $\lambda \lesssim r^{-3/2}$ (TODO: IMPROVE THIS),
	%
	\[ I(r,\lambda) \lesssim \lambda^{-1/3}. \]
	%
	The bounds here are uniform in $r$ and $\lambda$. Thus we get a uniform bound that varies between the various bounds obtained above.

	% TODO: Is the Esseen Concentration Inequality True for Smooth Amplitudes?

	% For r > 0, tr + t^3 is increasing
	% If [r(t + s) + (t + s)^3] - [rt + t^3]  1/lambda
	% 	Then
	% 		s = -t - 2^{1/3}r/A + A/3*2^{1/3}(
	% 	where A = (27/L + 27rt + 27t^3 + sqrt(108r^3 + (27/L + 27rt + 27t^3)^2))^{1/3}
	%
	% 	When t = 0, s = - r/(27/L + sqrt(108r^3 + (27/L)^2))^{1/3}
	%	For r >> L^{-2/3}, s = -1


	%		27/L >= 27rt when t <= 1/Lr
	% 		27/L >= 27t^3 when t <= 1/L^{1/3}
	% 		27rt >= 27t^3 when t <= r^{1/2}
	%
	% When t <= min(1/Lr, 1/L^{1/3})


	% When t is close to zero, tr dominates for t <= r^{1/2}
	% Then t^3 starts to dominate past this range.

	Such a method is often sufficient to obtain tight bounds for a large range of oscillatory integrals. For instance, in one dimension, the Van der Corput Lemmas often gives tight estimates. A tight version of this Lemma is as follows: (Arhipov, Karacupa, Cubarikov, 1980) show that if $n > 1$, and $|\partial^n \phi| \geq A > 0$ then
	%
	\[ |J(\phi)| \leq \min(|I|, 6n A^{-1/n}), \]
	%
	and if $n = 1$, and in addition, $\phi$ is \emph{piecewise monotonic} and continuous 



	i.e. integrating by parts on regions of the integral where $\phi'$ is non-vanishing. But nonstationarity in this sense cannot be the ultimate method for analyzing general oscillatory integral. Indeed, we can always rearrange integrals; given any measure preserving transform $F: I \to I$, we have
	%
	\[ \int_I a(x) e^{2 \pi i \phi(x)}\; dx = \int_I a(F(x)) e^{2 \pi i \phi(F(x))}\; dx. \]
	%
	Smoothness is \emph{not} preserved under measure preserving transformations. What is always preserved are the measures of the sets
	%
	\[ E(y_0,\varepsilon) = \{ x \in I: |\phi(x) - y_0| \leq \varepsilon \}, \]
	%
	which measure the ability for the oscillatory integral to concentrate about $x_0$.	