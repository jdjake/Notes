\documentclass{article}

\usepackage{amsthm}
\usepackage{amsmath}
\usepackage{multicol}
\usepackage{amssymb}
\usepackage{mathabx}
\usepackage{accents}
\usepackage[margin=0.5in]{geometry}

\theoremstyle{plain}
\newtheorem{lemma}{Lemma}
\newtheorem{prop}{Proposition}
\newtheorem*{example}{Example}
\newtheorem*{fact}{Fact}
\newtheorem*{corollary}{Corollary}

\usepackage{algorithm}
\usepackage[noend]{algpseudocode}

\theoremstyle{plain}
\newtheorem{theorem}{Theorem}
\newtheorem{proposition}[theorem]{Proposition}
\newtheorem*{remark}{Remark}

\title{Fractals Avoiding Fractal Sets}

\author{
  Jacob Denson\\
  \and
  Malabika Pramanik\\
  \and
  Josh Zahl
}

\begin{document}

\maketitle

\begin{multicols}{2}

When looking at the stability of various functional operators in modern analysis, it is often the case that one can boil down the calculations into studying how the operator acts on the characteristic functions of various sets. The existence of various characteristics in the set often gives insight into the behaviour of the operator's output, and the boundedness of the operator then rests on how large a set can be possessing these characteristics.

In this paper, we look at general methods to determine how large a set can be when the characteristic in question is the existence of fine-scale patterns in the set. Important examples include affine configurations, such as the existnece of sets not containing vertices of equilateral triangles, set not containing three term arithmetic progressions, and sets not generating particular families of angles. For these examples, the Lebesgue density theorem immediately implies any set of positive measure contains these patterns; thus we instead quantify the size of sets in terms of their Hausdorff dimension.

There are two approaches to the pattern avoidance problem. We can either upper bound the problem by proving any set with a sufficiently large Hausdorff dimension contains such patterns, or we can lower bound the problem by constructing sets with large Hausdorff dimension not containing these configurations. In this paper, we give general methods to solve the {\it construction problem} for configurations. 

There are already various generic configuration avoidance methods in the literature, discussed in section A. But these rely on the assumption that the configurations are specified in a regular manner, be they related to an affine variety, or a smooth hypersurface. The novel of the method developed here is that we are able to construct sets avoiding points on an {\it arbitrary} fractal set, which still have Hausdorff dimensions comparable to the results of these other methods. In order to do this, we introduce a new geometric framework in section B for thinking about pattern avoidance problems, which should aid in finding further methods in the field. Once this is done, a simple combinatorial argument, described in section C, can be exploited repeatedly to construct the configuration avoiding set.

\section{A Fractal Avoidance Framework}

A common thing to notice about the 

\section{Discrete Building Block}

We now develop a discrete technique used to construct solutions to the fractal avoidance problem. It depends very little on the Euclidean structure of the plane. As such, we rephrase the construction as a combinatorial problem on graphs.

Recalling definitions, we say an {\it $n$ uniform hypergraph} is a collection of {\it vertices} and {\it hyperedges}, where a hyperedge is a set of $n$ distinct vertices. An {\it independent set} is a subset of vertices containing no complete set of vertices in any hyperedge of the graph. A {\it colouring} is a partition of the vertex set into finitely many independent sets, each of which we call a {\it colour}. Such a colouring is {\it $K$ uniform} if each colour class has $K$ elements.

The next lemma is a variant of Tur\'{a}n's theorem on independent sets. For technical reasons, we need an extra restriction on the independant set so it is `uniformly' chosen over the graph. This is why colorings are introduced.

\begin{lemma}
	Let $G$ be an $n$ uniform hypergraph with a $K$ uniform coloring. Then there is an independent set $W$ containing elements from all but $|E|/K^n$ colors.
\end{lemma}
\begin{proof}
	Let $U$ be a random vertex set chosen by selecting a vertex of each color uniformly randomly. Every vertex occurs in $U$ with probability $1/K$. For any edge $e = (v_1, \dots, v_n)$, the vertices $v_i$ all have different colors. Thus they have an independent chance of being added to $U$, and we calculate
	%
	\begin{align*}
		\mathbf{P}(v_1 \in U, \dots, v_n \in U) = \mathbf{P}(v_1 \in U) \dots \mathbf{P}(v_n \in U) = 1/K^n
	\end{align*}
	%
	If we let $E'$ denote the edges $e = (u_1, \dots, u_n)$ with $u_1, \dots, u_n \in U$, then
	%
	\[ \mathbf{E}|E'| = \sum_{e \in E} \mathbf{P}(e \in E') = \sum_{e \in E} 1/K^n = \frac{|E|}{K^n} \]
	%
	This means we may choose a {\it particular}, nonrandom $U$ for which $|E'| \leq |E|/K^n$. If we form a vertex set $W \subset U$ by removing, for each $e \in E'$, a vertex in $U$ adjacent to the edge, then $W$ is an independent set containing all but $|E'| \leq |E|/K^n$ colors.
\end{proof}

\begin{corollary}
	If $|V| \gtrsim N^a$, $|E| \lesssim N^b$, and $K \gtrsim N^c$, where $b < a + c(n-1)$, then as $N \to \infty$ we can find an independent set containing all but a fraction $o(1)$ of the colors.
\end{corollary}
\begin{proof}
	A simple calculation on the quantities of the previous lemma yields
	%
	\begin{align*}
		\frac{\# ( \text{colors removed} )}{\# ( \text{all colors} )} = \frac{|E|/K^n}{|V|/K} = \frac{|E|}{|V|K^{n-1}} \lesssim \frac{N^b}{N^{a + c(n-1)}}
	\end{align*}
	%
	This is $o(1)$ if $b < a + c(n-1)$.
\end{proof}

We now apply these constructions to a problem clearly related to the fractal avoidance problem. It will form our key method to construct fractal avoidance solutions. Given an integer $N$, we subdivide $\mathbf{R}^d$ into a lattice of sidelength $1/N$ cubes with corners on $\mathbf{Z}^d/N$, the collection of such cubes we will denote by $\mathcal{B}(1/N)$. This grid is used to granularize configuration avoidance.

\begin{theorem}
	Suppose $\mathcal{I}_1, \dots, \mathcal{I}_n$ are disjoint collections of cubes in $\mathcal{B}(1/N)$, with $|\mathcal{I}_i| \gtrsim N^d$. We assume the lower Minkowski dimension of $Y$ is bounded above by $\alpha$, and $\beta > d(n-1)/(n-\alpha)$. Then there exists arbitrarily large $N$ and collections of cubes $\mathcal{J}_1, \dots, \mathcal{J}_n \in \mathcal{B}(1/N^\beta)$ with each cube in $\mathcal{J}_1 \times \dots \times \mathcal{J}_n$ disjoint from $Y$, and as $N \to \infty$, each $\mathcal{J}_i$ contains cubes in all but a fraction $o(1)$ of cubes in $\mathcal{I}_i$.
\end{theorem}
\begin{proof}
	If $\mathcal{K} \subset \mathcal{B}(1/N^\beta)^n$ is the collection of all cubes in a sidelength $1/N^\beta$ lattice intersecting $Y$, then $|\mathcal{K}| \lesssim N^{\alpha \beta}$. We then let $\mathcal{I}'_i$ be all cubes in $\mathcal{B}(1/N^\beta)$ contained in $\mathcal{I}_i$. Considering these cubes as vertices gives us an $n$ uniform hypergraph $G$ with a hyperedge between $I_1 \in \mathcal{I}'_1, \dots, I_n \in \mathcal{I}'_n$ if $I_1 \times \dots \times I_n \in \mathcal{K}$. We say two cubes in $G$ are the same color if they are contained in a common cube in $\mathcal{I}_i$.

	Using the fact that a sidelength $1/N$ cube contains $N^{d(\beta - 1)}$ sidelength $1/N^\beta$ cubes, we conclude that $G$ has $\sum |\mathcal{I}_i| = N^{d(\beta - 1)} \sum |\mathcal{I}_i| \gtrsim N^{d \beta}$ vertices. The number of edges in $G$ is bounded by $|\mathcal{K}| \lesssim N^{\alpha \beta}$. Finally, the coloring is $N^{d(\beta - 1)}$ uniform. Thus in the terminology of the previous corollary, $a = d \beta$, $b = \alpha \beta$, and $c = d(\beta - 1)$, and the inequality in the hypothesis of this theorem is then equivalent to the inequality in the hypothesis of the corollary. Applying the corollary gives the required result.
\end{proof}

The value $d(n-1)/(n-\alpha)$ in the theorem is directly related to the dimension $(n-\alpha)/(n-1)$ we obtain in our main result. Any improvement on this bound for special cases of the fractal avoidance problem immediately will lead to improvements on the Hausdorff dimension of the set constructed. The fact that our hypergraph result is tight indicates that for the general fractal avoidance problem, our construction gives tight bounds.

% TODO: Include Tightness Calculation?

\section{A Fractal Avoiding Set}

Our solutions $X$ to fractal avoidance problems will be obtained by breaking the problem down into a sequence of discrete configuration problems. The central idea was first used in \cite{MalabikaRob}. We construct $X$ as a limit $\lim X_N$, where $X_N$ is a disjoint union of sidelength $L_N$ cubes, and $X_{N+1}$ is obtained from $X_N$ by subdividing the cubes into length $R_N$ cubes, then further subdividing these cubes into cubes of smaller sidelength $L_{N+1}$, and removing a portion of them.

At each step $N$, we consider a disjoint collection of sidelength $R_N$ cubes $\mathcal{I}_1(N), \dots, \mathcal{I}_n(N) \subset \mathcal{B}(R_N)$, each cube contained in $X_N$. The main result of the previous section allows us to find a collection of sidelength $L_{N+1} = R_N^{\beta_N}$ cubes $\mathcal{J}_i(N) \subset \mathcal{I}_i(N)$ with all cubes in $\mathcal{J}_1(N) \times \dots \times \mathcal{J}_n(N)$ disjoint from $Y$, and where $\beta_N$ converges to $\beta = d(n-1)/(n-\alpha)$ from above. We then form $X_{N+1}$ from $X_N$ by removing the parts of cubes in $\mathcal{I}_i(N)$ which are not contained in the cubes in $\mathcal{J}_i(N)$. Once parameters are fixed, and an initial set $X_0$ is chosen, which we might as well assume to be $[0,1]^d$, we obtain a sequence $X_0, X_1, \dots$ converging to a set $X$. A simple constraint detailed below is all that is required to ensure that $X$ is a solution to the fractal avoidance problem.

\begin{lemma}
	Suppose that for any choice of distinct $x_1, \dots, x_n \in X$, there exists $N$ such that each $x_i$ is contained in a cube in $\mathcal{I}_i(N)$. Then $X^d \cap Y \subset \Delta$.
\end{lemma}
\begin{proof}
	For then $x_1, \dots, x_n \in X_{N+1}$, so $x_1 \in \mathcal{J}_1(N), \dots, x_n \in \mathcal{J}_n(N)$, and so the tuple $(x_1, \dots, x_n)$ is contained in a cube in $\mathcal{J}_1(N) \times \dots \times \mathcal{J}_n(N)$, which is disjoint from $Y$. Taking contrapositives of this argument shows that if $y \in X^d \cap Y$, then there must be some $i$ and $j$ for which $y_i = y_j$, so $y \in \Delta$.
\end{proof}

% TODO: Include diagram of construction of queueing.

We achieve the constraint in the lemma by dynamically choosing parameters subject to a queueing process. The queue will consist of an ordered sequence of tuples $(I_1, \dots, I_n)$, where $I_1 ,\dots, I_n$ are disjoint cubes. At stage $N$ of the construction, we take off the front tuple $(I_1, \dots, I_n)$ from the queue, and set $\mathcal{I}_i(N)$ to be the set of all cubes in $\mathcal{B}(R_N)$ which are a subset of both $I_i$ and $X_N$. We then subdivide $X_N$ using these parameters to form the set $X_{N+1}$ as a union of length $L_{N+1} = R_N^\beta$ intervals. After this, for {\it any} ordered choice of distinct intervals $I_1, \dots, I_n \in \mathcal{B}(L_{N+1})$, with each interval $I_i$ a subset of $X_{N+1}$, we add the tuple $(I_1, \dots, I_n)$ to the end of the queue.

Provided that $L_N \to 0$, which will of course be the case, then for any distinct choice of $x_1, \dots, x_n \in X$, there exists $N$ and $L_N$ such that $|x_i - x_j| \geq 2 L_N$ for all $i \neq j$. Thus at stage $N$ of the construction, a tuple $(I_1, \dots, I_n)$ is added to the end of the queue with $x_i \in I_i$, and at a {\it much} {\it much} later stage $M$ of the construction, this tuple is popped off the front of the queue, and so each $x_i$ is contained in a cube in $\mathcal{I}_i(M)$. Thus we conclude that $X$ is a solution to the fractal avoidance problem.

\section{Dimension Bounds}

To complete the proof, it suffices to choose the parameters $R_N$ and $\beta_N$ which lead to the correct Hausdorff dimension bound on $X$. The actual choice of $\beta_N$ doesn't matter, only that it is an increasing sequence converging to $\beta$ in the limit. We also fix a decreasing sequence $\lambda_N$ such that $\lambda_N \beta_N > d$, to be used later on in our argument. Since $\beta_N$ converges to $\beta$ from above, we can let $\lambda_N$ tend to $\lambda = (dn - \alpha)/(n - 1)$ from below. The fact that the dissection of $X_{N+1}$ for $X_N$ occurs uniformly over the will aid us in annihilating the superexponentially increasing constants which inherently occur from the exponentially decreasing values of $L_N$ we are forced to choose.

We rely on the mass distribution principle to construct a probability measure $\mu$ supported on $X$. This enables us to calculate the Hausdorff dimension of $X$ using Frostman's lemma. We begin by putting the uniform probability measure $\mu_0$ on $X_0 = [0,1]^d$. Then, at each stage of the construction, we construct $\mu_{N+1}$ from $\mu_N$ by taking the mass on a certain sidelength $L_N$ cube in $X_N$, and uniformly distributing it's mass over the sidelength $L_{N+1}$ cubes in $I \cap X_{N+1}$. Using the weak compactness of the unit ball in $L^1(\mathbf{R}^d)^*$, we obtain a weak limit $\mu = \lim \mu_n$. The fact that $\mu_n$ is supported on $X_n$ for each $n$ implies $\mu$ is supported on $X$.

It is intuitive that the mass on $\mu$ will be distributed more thinly at each stage the fatter the cubes that are kept. Quantifying this precisely allows us to apply Frostman's lemma. More precisely, we will prove that for each length $L$ interval $I$, $\mu(I) \lesssim_N L^{\lambda_N}$. Thus Frostman's lemma guarantees that $\dim_{\mathbf{H}}(X) \geq \lambda_N$, and taking $\lambda_N \to \lambda$ will complete the proof.

\begin{lemma}
	For $R_N \gg 0$, if $I \in \mathcal{B}(L_{N+1})$ and $J \in \mathcal{B}(L_N)$,
	%
	\[ \mu(I) \leq 2 (R_N/L_N)^d \mu(J)\ \ \ \mu(I) \leq 2^N R_0^{d - \beta_0} \dots R_N^{d - \beta_N} R_N^d \]
\end{lemma}
\begin{proof}
	If $I$ is not a cube in $X_{N+1}$, then $\mu(I) = \mu_{N+1}(I) = 0$, so the inequality is obviously true. Otherwise, we can find a cube $J \in \mathcal{B}(L_N)$ in $I \cap X_N$. $J$ contains $(L_N/R_N)^d$ sidelength $R_N$ cubes. Our main discrete result implies that $X_{N+1}$ contains a sidelength $L_{N+1}$ cube in all but a fraction $o(1)$ of these cubes,. In particular, if we choose $R_N$ sufficiently large, then we know that we keep a sidelength $L_N$ portion of at least half of these cubes. Thus
	%
	\begin{align*}
		\mu(I) &= \mu_{N+1}(I) \leq \frac{\mu_N(J)}{(L_N/R_N)^d/2}\\
		&= 2 \mu_N(J) (R_N/L_N)^d = 2 \mu(J) (R_N/L_N)^d
	\end{align*}
	%
	completing the calculation. Applying this calculation iteratively, we conclude
	%
	\begin{align*}
		\mu(I) &\leq 2^N (R_0/L_0)^d (R_1/L_1)^d \dots (R_N/L_N)^d\\
		&= 2^N R_0^{d - \beta_0} \dots R_{N-1}^{d - \beta_{N-1}} R_N^d
	\end{align*}
	%
	completing the calculation.
\end{proof}

\begin{corollary}
	If $R_N$ is large, we can force $\mu(I) \leq L_N^{\lambda_N}$ for $I \in \mathcal{B}(L_N)$.
\end{corollary}
\begin{proof}
	We write the inequality in the last problem as
	%
	\[ \mu(I) \leq [2^N R_0^{d - \beta_0} \dots R_{N-1}^{d - \beta_{N-1}} R_N^{d - \lambda_N \beta_N}] L_{N+1}^{\lambda_N} \]
	%
	Since $\lambda_N \beta_N > d$, the quantity in the square brackets is $o(1)$ as $R_N \to \infty$. Thus for sufficiently large $R_N$, we conclude that $\mu(I) \leq L_N^{\lambda_N}$.
\end{proof}

This is almost the required inequality, except we have only proven it for intervals at particular scales. To obtain a general inequality, we use the fact that our construction is obtained uniformly across all intervals.

\begin{theorem}
	If $R_N$ is chosen large enough that the previous inequalities hold, then we have $\mu(I) \leq 2^{d+1} L^{\lambda_N}$ for all intervals $I$ with sidelength $L \leq L_N$.
\end{theorem}
\begin{proof}
	We break our analysis into three cases, depending on the size of $L$:
	%
	\begin{itemize}
		\item If $R_N \leq L \leq L_N$, we can cover $I$ by at most $2^d(L/R_N)^d$ cubes in $\mathcal{B}(R_N)$. For each such cube, we know that the mass on each sidelength $R_N$ cube is at most $2(R_N/L_N)^d$ times the mass on an element of $\mathcal{B}(L_N)$. Thus we obtain a bound
		%
		\[ \mu(I) \leq [2^d(L/R_N)^d] [2(R_N/L_N)^d] [L_N^{\lambda_N}] \leq \frac{2^{d+1} L^d}{L_N^{d - \lambda_N}} \leq 2^{d+1} L^{\lambda_N} \]
		%
		which gives the required result.

		\item If $L_{N+1} \leq L \leq R_N$, we can cover $L$ by at most $2^d$ cubes in $\mathcal{B}(R_N)$. Each cube in $\mathcal{B}(R_N)$ contains at most one cube in $\mathcal{B}(L_{N+1})$ which is also contained in $X_{N+1}$, so the bound in the last corollary gives that $\mu(I) \leq 2^d L_{N+1}^{\lambda_N} \leq 2^d L^{\lambda_N}$.

		\item If $L \leq L_{N+1}$, there certainly exists $M$ such that $L_{M+1} \leq L \leq L_M$, and one of the previous cases yields that $\mu(I) \leq 2^{d+1} L^{\lambda_M} \leq 2^{d+1} L^{\lambda_N}$.
	\end{itemize}
	%
	This covers all possible situations, completing the proof.
\end{proof}

To employ Frostman's lemma, we need the result $\mu(I) \lesssim L^{\lambda_N}$ for an {\it arbitrary} interval, not just one with $L \leq L_N$. But this is no trouble; it is only the behaviour of the measure on arbitrarily small scales that matters. This is because if $L \geq L_N$, then $\mu(I)/L^{\lambda_N} \leq 1/L_N^{\lambda_N} \lesssim_N 1$, so $\mu(I) \lesssim_N L^{\lambda_N}$ holds automatically for all sufficiently large intervals. Thus all problems with the Hausdorff dimension argument are complete, and we have proven that there is a choice of parameters which constructs a set $X$ with Hausdorff dimension no less than $(nd - \alpha)/(n-1)$.

\end{multicols}

\begin{thebibliography}{9}

\bibitem{RuzsaSetsWithoutSquares}
I. Z. Ruzsa
\textit{Difference Sets Without Squares}

\bibitem{KeletiDimOneSet}
Tam\'{a}s Keleti
\textit{A 1-Dimensional Subset of the Reals that Intersects Each of its Translates in at Most a Single Point}

\bibitem{MalabikaRob}
Robert Fraser, Malabika Pramanik
\textit{Large Sets Avoiding Patterns}

\bibitem{Sudakov}
B. Sudakov, E. Szemer\'{e}di, V.H. Vu
\textit{On a Question of Erd\"{o}s and Moser}

\bibitem{Mathe}
A. Ma\'{t}h\'{e}
\textit{Sets of Large Dimension Not Containing Polynomial Configurations}

\end{thebibliography}

%- Berend's counterexample is a discrete version of Kaleti's continuous counterexample for 3APs
%- Look up Wisewell functions
%- (x_2 - x_1)^2 = x_3 - x_1: Comes up in Bourgain & Chang

%- Principal character gives main term, rest should be thrown into the error term.   

\end{document}