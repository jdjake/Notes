\documentclass{article}

\usepackage{amsthm}
\usepackage{amsmath}
\usepackage{multicol}
\usepackage{amssymb}
\usepackage{mathabx}
\usepackage{accents}
\usepackage[margin=1in]{geometry}

\theoremstyle{plain}
\newtheorem{lemma}{Lemma}
\newtheorem{prop}{Proposition}
\newtheorem*{example}{Example}
\newtheorem*{fact}{Fact}
\newtheorem*{corollary}{Corollary}

\usepackage{algorithm}
\usepackage[noend]{algpseudocode}

\theoremstyle{plain}
\newtheorem{theorem}{Theorem}
\newtheorem{proposition}[theorem]{Proposition}
\newtheorem*{remark}{Remark}

\title{Fractals Avoiding Fractal Sets}
\author{Jacob Denson}

\begin{document}

\maketitle

\begin{multicols}{2}

\section{A Discrete Building Block}

We now develop a discrete technique used to construct solutions to the fractal avoidance problem. It depends very little on the Euclidean structure of the plane. As such, we rephrase the construction as a combinatorial problem on graphs.

An {\it $n$ uniform hypergraph} is a collection of {\it vertices} and {\it hyperedges}, where a hyperedge is a set of $n$ distinct vertices. An {\it independent set} is a subset of vertices containing no complete set of vertices in any hyperedge of the graph. A {\it colouring} is a partition of the vertex set into finitely many independent sets, each of which we call a {\it colour}. Such a colouring is {\it $K$ uniform} if each colour class has $K$ elements.

The next lemma is a variant of Tur\'{a}n's theorem on independent sets. For technical reasons, we need an extra restriction on the independant set so it is `uniformly' chosen over the graph. This is why colorings are introduced.

\begin{lemma}
	Let $G$ be an $n$ uniform hypergraph  with a $K$ uniform coloring. Then there exists an independent set $W$ containing all but $|E|/K^n$ colors.
\end{lemma}
\begin{proof}
	Let $U$ be a random vertex set chosen by selecting a vertex of each color uniformly randomly. Every vertex occurs in $U$ with probability $1/K$. For any edge $e = (v_1, \dots, v_n)$, the vertices $v_i$ all have different colors. Thus they have an independent chance of being added to $U$, and we calculate
	%
	\begin{align*}
		&\mathbf{P}(v_1 \in U, \dots, v_n \in U)\\
		&\ \ \ \ = \mathbf{P}(v_1 \in U) \dots \mathbf{P}(v_n \in U) = 1/K^n
	\end{align*}
	%
	If we let $E'$ denote the set of all edges $e = (u_1, \dots, u_n)$ with $u_1, \dots, u_n \in U$, then
	%
	\[ \mathbf{E}|E'| = \sum_{e \in E} \mathbf{P}(e \in E') = \sum_{e \in E} 1/K^n = \frac{|E|}{K^n} \]
	%
	This means we may choose a {\it particular}, nonrandom $U$ for which $|E'| \leq |E|/K^n$. If we form a vertex set $W \subset U$ by removing, for each $e \in E'$, a vertex in $U$ adjacent to the edge, then $W$ is an independent set containing all but $|E'| \leq |E|/K^n$ colors.
\end{proof}

\begin{corollary}
	If $|V| \gtrsim N^a$, $|E| \lesssim N^b$, and $K \gtrsim N^c$, where $b < a + c(n-1)$, then as $N \to \infty$ we can find an independent set containing all but a fraction $o(1)$ of the colors.
\end{corollary}
\begin{proof}
	A simple calculation on the quantities of the previous lemma yields
	%
	\begin{align*}
		&\frac{\# ( \text{colors removed} )}{\# ( \text{all colors} )} = \frac{|E|/K^n}{|V|/K}\\
		&\ \ \ \ = \frac{|E|}{|V|K^{n-1}} \lesssim \frac{N^b}{N^{a + c(n-1)}}
	\end{align*}
	%
	This is $o(1)$ if $b < a + c(n-1)$.
\end{proof}

We now apply these constructions to a problem clearly related to the fractal avoidance problem. It will form our key method to construct fractal avoidance solutions. Given an integer $N$, we subdivide $\mathbf{R}^d$ into a lattice of sidelength $1/N$ cubes with corners on $\mathbf{Z}^d/N$, the collection of such cubes we will denote by $\mathcal{B}(N)$. This grid is used to granularize configuration avoidance.

\begin{theorem}
	Suppose $\mathcal{I}_1, \dots, \mathcal{I}_n$ are disjoint collections of cubes in $\mathcal{B}(N)$, with $|\mathcal{I}_i| \gtrsim N^d$. We assume the lower Minkowski dimension of $Y$ is bounded above by $\alpha$, and $\beta > d(n-1)/(n-\alpha)$. Then there exists arbitrarily large $N$ and collections of cubes $\mathcal{J}_1, \dots, \mathcal{J}_n \in \mathcal{B}(N^\beta)$ with each cube in $\mathcal{J}_1 \times \dots \times \mathcal{J}_n$ disjoint from $Y$, and as $N \to \infty$, each $\mathcal{J}_i$ contains cubes in all but a fraction $o(1)$ of cubes in $\mathcal{I}_i$.
\end{theorem}
\begin{proof}
	If $\mathcal{K} \subset \mathcal{B}(N^\beta)^n$ is the collection of all cubes in a sidelength $1/N^\beta$ lattice intersecting $Y$, then $|\mathcal{K}| \lesssim N^{\alpha \beta}$. We then let $\mathcal{I}'_i$ be all cubes in $\mathcal{B}(N^\beta)$ contained in $\mathcal{I}_i$. Considering these cubes as vertices gives us an $n$ uniform hypergraph $G$ with a hyperedge between $I_1 \in \mathcal{I}'_1, \dots, I_n \in \mathcal{I}'_n$ if $I_1 \times \dots \times I_n \in \mathcal{K}$. We say two cubes in $G$ are the same color if they are contained in a common cube in $\mathcal{I}_i$.

	Using the fact that a sidelength $1/N$ cube contains $N^{d(\beta - 1)}$ sidelength $1/N^\beta$ cubes, we conclude that $G$ has $\sum |\mathcal{I}_i| = N^{d(\beta - 1)} \sum |\mathcal{I}_i| \gtrsim N^{d \beta}$ vertices. The number of edges in $G$ is bounded by $|\mathcal{K}| \lesssim N^{\alpha \beta}$. Finally, the coloring is $N^{d(\beta - 1)}$ uniform. Thus in the terminology of the previous corollary, $a = d \beta$, $b = \alpha \beta$, and $c = d(\beta - 1)$, and the inequality in the hypothesis of this theorem is then equivalent to the inequality in the hypothesis of the corollary. Applying the corollary gives the required result.
\end{proof}

The value $d(n-1)/(n-\alpha)$ in the theorem is directly related to the dimension $(n-\alpha)/(n-1)$ we obtain in our main result. Any improvement on this bound for special cases of the fractal avoidance problem immediately leads to improvements on the Hausdorff dimension of the set constructed. The fact that our hypergraph result is tight indicates that for the general fractal avoidance problem, our construction gives tight bounds.

% Include Tightness?

\section{A Fractal Avoiding Set}

Our solutions $X$ to fractal avoidance problems will be obtained by breaking the problem down into a sequence of discrete configuration problems. The central idea was first used in \cite{MalabikaRob}. We construct $X$ as a limit $\lim X_N$, where $X_N$ is a disjoint union of sidelength $L_N$ cubes, and $X_{N+1}$ is obtained from $X_N$ by taking a subset of cubes in $X_N$, subdividing the cube into cubes of a smaller sidelength, and removing a portion of them.

We will associate with each $N$ a disjoint collection of sidelength $L_N$ cubes $\mathcal{I}_1(N), \dots, \mathcal{I}_n(N)$, each cube contained in $X_N$. The main result of the previous section immediately allows us to find a collection of sidelength $L_N^{\beta_n}$ cubes $\mathcal{J}_i(N) \subset \mathcal{I}_i(N)$ with all cubes in $\mathcal{J}_1(N) \times \dots \times \mathcal{J}_n(N)$ disjoint from $Y$, and where $\beta_n$ converges to $d(n-1)/(n-\alpha)$ from above. We then form $X_{N+1}$ from $X_N$ by removing the parts of cubes in $\mathcal{I}_i(N)$ which are not contained in the cubes in $\mathcal{J}_i(N)$. Once parameters are fixed, and an initial set $X_0$ is chosen, we obtain a sequence $X_0, X_1, \dots$ converging to a set $X$. A simple constraint is all that is required to ensure that $X$ is a solution to the fractal avoidance problem.

\begin{lemma}
	Suppose that for any choice of distinct $x_1, \dots, x_n \in X$, there exists $N$ such that each $x_i$ is contained in a cube in $\mathcal{I}_i(N)$. Then $X^d \cap Y \subset \Delta$.
\end{lemma}
\begin{proof}
	For then $x_1, \dots, x_n \in X_{N+1}$, so $x_1 \in \mathcal{J}_1(N), \dots, x_n \in \mathcal{J}_n(N)$, and so the tuple $(x_1, \dots, x_n)$ is contained in a cube in $\mathcal{J}_1(N) \times \dots \times \mathcal{J}_n(N)$, which is disjoint from $Y$. Taking contrapositives of this argument shows that if $y \in X^d \cap Y$, then there must be some $i$ and $j$ for which $y_i = y_j$, so $y \in \Delta$.
\end{proof}

% TODO: Include diagram of construction of queueing.

We achieve the constraint in the lemma by dynamically choosing parameters subject to a queueing process. The queue will consist of an ordered sequence of tuples $(I_1, \dots, I_n)$, where $I_1 ,\dots, I_n$ are disjoint cubes. At stage $N$ of the construction, we take off the front tuple $(I_1, \dots, I_n)$ from the queue, and set $\mathcal{I}_i(N)$ to be the set of all cubes in $\mathcal{B}(L_N)$ which are a subset of both $I_i$ and $X_N$. We then subdivide $X_N$ using these parameters to form the set $X_{N+1}$ as a union of length $L_N^\beta$ intervals. Then for {\it any} ordered choice of distinct intervals $I_1, \dots, I_n \in \mathcal{B}(L_N^\beta)$, with each interval $I_i$ a subset of $X_{N+1}$, we add the tuple $(I_1, \dots, I_n)$ to the end of the queue. Provided that $L_N \to 0$, for any distinct choice of $x_1, \dots, x_n \in X$, there exists $N$ and $L_N$ such that $|x_i - x_j| \geq 2 L_N$ for all $i \neq j$. Thus at stage $N$, a tuple $(I_1, \dots, I_n)$ is added to the end of the queue with $x_i \in I_i$, and at a {\it much} {\it much} later stage $M$ of the construction, this tuple is considered, and so $x_i \in \mathcal{I}_i(M)$. Thus we conclude that $X$ is a solution to the fractal avoidance problem.

\section{Dimension Bounds}

To complete the proof, it suffices to choose the parameters $L_N$ and $\beta_N$ which lead to the correct Hausdorff dimension bound on $X$. The `uniformity' result present in our discrete construction will aid us in eliminating the superexponentially increasing constants which emerge from the exponentially decrasing values of $L_N$ we are forced to pick to eliminate the inherent multiplicative constants which occur in our construction.

To prove the dimension bounds on $X$, we rely on the mass distribution principle to construction a probability measure $\mu$ on $X$ from which we can apply Frostman's lemma. We begin by putting the uniform probability measure $\mu_0$ on $X_0 = [0,1]$. Then, at each stage of the construction, we construction $\mu_{N+1}$ from the measure $\mu_N$ supported on $X_N$ by taking the mass of $\mu_N$ supported on a certain length $L_{N-1}$ interval in $X_N$, and uniformly distributing it over the length $L_N$ intervals contained with this interval which remain in $X_{N+1}$. Then we just use the weak compactness of the unit ball in $L^1(\mathbf{R}^d)^*$ to construct a weak limit $\mu = \lim \mu_n$, for which $\mu$ is supported on $X$. It should be intuitive that the mass will be distributed more thinly at each stage the fatter the intervals that are kept at each stage, and thus Frostman's lemma will obtain a higher Hausdorff dimension bound.

\begin{lemma}
	If $I$ is an interval of length $L_N$ in $X_N$, and $J \subset I$ is an interval of length $L_{N+1}$ kept at the next stage in $X_{N+1}$, then
\end{lemma}

\end{multicols}

\begin{thebibliography}{9}

\bibitem{RuzsaSetsWithoutSquares}
I. Z. Ruzsa
\textit{Difference Sets Without Squares}

\bibitem{KeletiDimOneSet}
Tam\'{a}s Keleti
\textit{A 1-Dimensional Subset of the Reals that Intersects Each of its Translates in at Most a Single Point}

\bibitem{MalabikaRob}
Robert Fraser, Malabika Pramanik
\textit{Large Sets Avoiding Patterns}

\bibitem{Sudakov}
B. Sudakov, E. Szemer\'{e}di, V.H. Vu
\textit{On a Question of Erd\"{o}s and Moser}

\end{thebibliography}

%- Berend's counterexample is a discrete version of Kaleti's continuous counterexample for 3APs
%- Look up Wisewell functions
%- (x_2 - x_1)^2 = x_3 - x_1: Comes up in Bourgain & Chang

%- Principal character gives main term, rest should be thrown into the error term.   

\end{document}