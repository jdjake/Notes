\documentclass{article}

\usepackage{amsthm}
\usepackage{amsmath}
\usepackage{multicol}
\usepackage{amssymb}
\usepackage{mathabx}
\usepackage{accents}
\usepackage[margin=1in]{geometry}

\theoremstyle{plain}
\newtheorem{lemma}{Lemma}
\newtheorem{prop}{Proposition}
\newtheorem*{example}{Example}
\newtheorem*{fact}{Fact}
\newtheorem*{corollary}{Corollary}

\usepackage{algorithm}
\usepackage[noend]{algpseudocode}

\theoremstyle{plain}
\newtheorem{theorem}{Theorem}
\newtheorem{proposition}[theorem]{Proposition}
\newtheorem*{remark}{Remark}

\title{Fractals Avoiding Fractal Sets}
\author{Jacob Denson}

\begin{document}

\maketitle

\begin{multicols}{2}

\section{A Discrete Building Block}

We now develop a technique which we will use repeatedly to construct solutions to the fractal avoidance problem. It depends very little on the Euclidean structure of the plane, and as such, we rephrase the construction as a combinatorial problem on graphs. Recall that an {\it $n$ uniform hypergraph} if a collection of {\it vertices} and {\it hyperedges}, where a hyperedge is a set of $n$ distinct vertices. We say such a graph is {\it hypartite} if we can partition the vertex set into $n$ sets $V_1, \dots, V_n$, such that each edge in the graph contains exactly one vertex from each set. An {\it independant set} is a subset of vertices containing no complete set of vertices in any edge. A {\it coloring} is a partition of the vertex set into finitely many classes, and we call each such class a {\it color}. The next lemma is a variant of Tur\'{a}n's theorem on the construction of independant sets, generalized for hypergraphs. For technical reasons, we need an extra restriction on the vertex set so it is `uniformly' chosen over the graph, which is the reason for the coloring.

\begin{lemma}
	Let $G$ be an $n$ uniform hypergraph together with a coloring partitioning each vertex set into size $A$ independant sets. Then we can find an independent set $U$ containing all but $(n/A^n)|E|$ of the colors.
\end{lemma}
\begin{proof}
	Let $U$ be a vertex set chosen by randomly selecting a vertex of each color. Then each vertex in the graph is selected with probability $1/A$. For each edge $e = (v_1, \dots, v_n)$ in the graph, we know that the vertices $v_i$ all have different colors, so they are added to $U$ independantly, and so the chance that this edge connects vertices in the set we have constructed is
	%
	\begin{align*}
		&\mathbf{P}(v_1 \in U, \dots, v_n \in U)\\
		&\ \ \ \ = \mathbf{P}(v_1 \in U) \dots \mathbf{P}(v_n \in U) = 1/A^n
	\end{align*}
	%
	Thus if we let $E'$ denote the set of all edges $e = (u_1, \dots, u_n)$ with $u_1, \dots, u_n \in U$, then
	%
	\[ \mathbf{E}|E'| = \sum_{e \in E} \mathbf{P}(e \in E') = \sum_{e \in E} 1/A^n = \frac{|E|}{A^n} \]
	%
	In particular, we may choose a {\it particular}, nonrandom choice $U$ for which $|E'| \leq |E|/A^n$. If we form a vertex set $W \subset U$ by removing all vertices $u \in U$ which are a vertex in some edge in $E'$, then $W$ is an independant set containing all but $n |E'| \leq (n/A^n) |E|$ colors.
\end{proof}

\begin{corollary}
	If $|V| \gtrsim N^a$, $|E| \lesssim N^b$, and $A \gtrsim N^c$, where $b < a + c(n-1)$, then as $N \to \infty$ we can find an independent set containing all but a fraction $o(1)$ of the colors.
\end{corollary}
\begin{proof}
	A simple calculation on the quantities of the previous lemma yields
	%
	\begin{align*}
		&\frac{\# ( \text{colors removed} )}{\# ( \text{total colors} )} = \frac{(n/A^n)|E|}{|V|/A}\\
		&\ \ \ \ = \frac{n|E|}{|V|A^{n-1}} \lesssim \frac{N^b}{N^{a + c(n-1)}}
	\end{align*}
	%
	This is $o(1)$ if $b < a + c(n-1)$.
\end{proof}

We now apply these constructions on graphs to a problem which is quite clearly related to the fractal avoidance problem, and will form our key building block to constructing solutions to the problem. Given a fixed integer $N$, we subdivide $\mathbf{R}^d$ into a grid of sidelength $1/N$ cubes, the collection of such cubes we will denote by $\mathcal{I}$.

\begin{theorem}
	Suppose that $\text{dim}_{\mathbf{M}}(Y) < \alpha$. If $\mathcal{I}_1, \dots, \mathcal{I}_n$ are disjoint collections of length $1/N$ cubes in $\mathcal{I}$, with $|\mathcal{I}_i| \gtrsim N^d$ for each $i$, then provided $\beta > d(n-1)/(n-\alpha)$, we can find a collection of length $1/N^\beta$ cubes $\mathcal{J}_1, \dots, \mathcal{J}_n$ with $\mathcal{J}_1 \times \dots \times \mathcal{J}_n$ disjoint from $Y$ and as $N \to \infty$, each $\mathcal{J}_i$ contains a cube in all but $o(1)$ of the cubes in $\mathcal{I}_i$.
\end{theorem}
\begin{proof}
	For sufficiently large $N$, if we partition $\mathbf{R}^{nd}$ into a grid of length $1/N^\beta$ cubes, and if $\mathcal{K}$ is the collection of all these cubes intersecting $Y$, then $|\mathcal{K}| \lesssim N^{\alpha \beta}$. Similarily, we partition each $\mathcal{I}_i$ into a grid of length $1/N^\beta$ cubes $\mathcal{I}'_i$, using these cubes as vertices in a hypartite graph $G$ with a hyperedge between $I_1 \in \mathcal{I}'_1, \dots, I_n \in \mathcal{I}'_n$ if $I_1 \times \dots \times I_n \in \mathcal{K}$. We say two cubes in $G$ are the same color if they are contained in a common cube in $\mathcal{I}_i$. Using the fact that a sidelength $1/N$ cube contains $N^{d(\beta - 1)}$ sidelength $1/N^\beta$ cubes, we conclude
	%
	\begin{itemize}
		\item The number of vertices in $G$ is
		%
		\[ \sum |\mathcal{I}'_i| = N^{d(\beta - 1)} \sum |\mathcal{I}_i| \gtrsim N^{d \beta} \]

		\item The number of edges is bounded by $|\mathcal{K}| \lesssim N^{\alpha \beta}$.

		\item Each color in $G$ contains $N^{d(\beta - 1)}$ vertices.
	\end{itemize}
	%
	Thus in the terminology of the previous corollary, $a = d \beta$, $b = \alpha \beta$, and $c = d(\beta - 1)$, and the inequality in the hypothesis of this theorem is then equivalent to the inequality in the hypothesis of the corollary.
\end{proof}

The value $d(n-1)/(n-\alpha)$ is directly related to the Hausdorff dimension $(n-\alpha)/(n-1)$ we obtain for solutions to the fractal avoidance problem in our main result. Any improvement on this bound for any special case of the fractal avoidance problem would immediately lead to improvements on the Hausdorff dimension of the resulting set. However, since this hypergraph result is tight in general, we believe that for the class of problems we consider, our construction is tight.

% Include Tightness?

\section{A Fractal Avoidance Set}

We construct our solution $X$ to the fractal avoidance problem by breaking down the problem into a sequence of discrete configuration problems on disecting cubes which lead to the complete fractal avoidance problem in the limit. The central idea of this construction was first used by Pramanik and Fraser (TODO: Insert Citation) in their general constructions to configuration avoidance problems. We construct $X = \lim X_N$, where each $X_N$ is a union of cubes of a fixed length, and $X_{N+1}$ is obtained from $X_N$ by taking a certain subset of cubes in $X_N$, and dissecting this subset, subdividing the cube into cubes of a smaller sidelength and removing a portion of them. We will associate with each $N$ a disjoint collection of sidelength $L_N$ cubes $\mathcal{I}_1(N), \dots, \mathcal{I}_n(N)$, with all such cubes contained in $X_N$. The previous section immediately allows us to find a collection of sidelength $L_N^{\beta_n}$ cubes $\mathcal{J}_1(N) \subset \mathcal{I}_1(N), \dots, \mathcal{J}_n(N) \subset \mathcal{I}_n(N)$ with $\mathcal{J}_1(N) \times \dots \times \mathcal{J}_n(N)$ disjoint from $Y$, with $\beta_n$ converging to $d(n-1)/(n-\alpha)$ from above. We then form $X_{N+1}$ from $X_N$ by removing each part of an cubes in $\mathcal{I}_i(N)$ which is not contained in an cubes in $\mathcal{J}_i(N)$, for each index $i$. We choose $X_0 = [0,1]$ as an initial to start off our construction.

There is only a simple constraint required on the parameters to this construction to ensure that $X$ is a solution to the fractal avoidance problem: For any choice of distinct $x_1, \dots, x_n \in X$, there exists $N$ such that for each $i$, $x_i$ is contained in a cube in $\mathcal{I}_i(N)$. Since we surely know $x_1, \dots, x_n \in X_{N+1}$, it then follows that $x_1 \in \mathcal{J}_1(N), \dots, x_n \in \mathcal{J}_n(N)$, and so the tuple $(x_1, \dots, x_n)$ are contained in a cube in $\mathcal{J}_1(N) \times \mathcal{J}_n(N)$, which is disjoint from $Y$.

% TODO: Include diagram of construction of queueing.

We achieve  the constraint to the construction by choosing our parameters subject to a dynamically changing queue consisting $(I_1, \dots, I_n)$, where $I_1 ,\dots, I_n$ are disjoint cubes. To get the process tarted, we can initialize the queue to begin with the tuple $([0,1/n], [1/n,2/n], \dots, [(n-1)/n])$. At each step $N$ of our process, we take off the front tuple $(I_1, \dots, I_n)$, subdivide $X_N$ into a grid of length $L_N$ cubes, and for each $i$, set $\mathcal{I}_i(N)$ to be the set of all such length $L_N$ cubes which are contained in $I_i$. After this is done, we have a subdivision of $X_{N+1}$ into length $L_N^\beta$ cubes, and we add each choice of $n$ length $L_N^\beta$ disjoint intervals in $X_{N+1}$ in this subdivision to the end of the queue. The queue grows inconcievably fast over time, but in the limit, every subdivision is processed. Provided that $L_N \to 0$, for any distinct $x_1, \dots, x_n \in X$ there is $L_N$ with $|x_i - x_j| \geq 2 L_N$, and so on the step $N$, we will add intervals $I_1, \dots, I_n$ with $x_1 \in I_1, \dots, x_n \in I_n$ to the end of the queue, and so eventually considered much further on in the construction. Thus we conclude that $X$ is a solution to the fractal avoidance problem.

\section{Dimension Bounds}

To complete the proof, it suffices to choose the parameters $L_N$ and $\beta_N$ which lead to the correct Hausdorff dimension bound on $X$. The `uniformity' result present in our discrete construction will aid us in eliminating the superexponentially increasing constants which emerge from the exponentially decrasing values of $L_N$ we are forced to pick.

To prove the dimension bounds on $X$, we rely on the mass distribution principle to construction a probability measure $\mu$ on $X$ from which we can apply Frostman's lemma. We begin by putting the uniform probability measure $\mu_0$ on $X_0 = [0,1]$. Then, at each stage of the construction, we construction $\mu_{N+1}$ from the measure $\mu_N$ supported on $X_N$ by taking the mass of $\mu_N$ supported on a certain length $L_{N-1}$ interval in $X_N$, and uniformly distributing it over the length $L_N$ intervals contained with this interval which remain in $X_{N+1}$. Then we just use the weak compactness of the unit ball in $L^1(\mathbf{R}^d)^*$ to construct a weak limit $\mu = \lim \mu_n$, and $\mu$ is supported on $X$. It should be intuitive that the mass will be distributed more thinly at each stage the fatter the intervals that are kept at each stage, and thus Frostman's lemma will obtain a higher Hausdorff dimension bound.

\begin{lemma}
	s
\end{lemma}

\end{multicols}

\begin{thebibliography}{9}

\bibitem{RuzsaSetsWithoutSquares}
I. Z. Ruzsa
\textit{Difference Sets Without Squares}

\bibitem{KeletiDimOneSet}
Tam\'{a}s Keleti
\textit{A 1-Dimensional Subset of the Reals that Intersects Each of its Translates in at Most a Single Point}

\bibitem{MalabikaRob}
Robert Fraser, Malabika Pramanik
\textit{Large Sets Avoiding Patterns}

\bibitem{Sudakov}
B. Sudakov, E. Szemer\'{e}di, V.H. Vu
\textit{On a Question of Erd\"{o}s and Moser}

\end{thebibliography}

%- Berend's counterexample is a discrete version of Kaleti's continuous counterexample for 3APs
%- Look up Wisewell functions
%- (x_2 - x_1)^2 = x_3 - x_1: Comes up in Bourgain & Chang

%- Principal character gives main term, rest should be thrown into the error term.   

\end{document}