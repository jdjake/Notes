\documentclass{article}

\usepackage{amsthm}
\usepackage{amsmath}
\usepackage{multicol}
\usepackage{amssymb}
\usepackage{mathabx}
\usepackage{accents}
\usepackage[margin=1in]{geometry}

\theoremstyle{plain}
\newtheorem{lemma}{Lemma}
\newtheorem{prop}{Proposition}
\newtheorem*{example}{Example}
\newtheorem*{fact}{Fact}
\newtheorem*{corollary}{Corollary}

\usepackage{algorithm}
\usepackage[noend]{algpseudocode}

\theoremstyle{plain}
\newtheorem{theorem}{Theorem}
\newtheorem{proposition}[theorem]{Proposition}
\newtheorem*{remark}{Remark}

\title{Fractals Avoiding Fractal Sets}
\author{Jacob Denson}

\begin{document}

\maketitle

\begin{multicols}{2}

Our proof depends very little on the Euclidean structure of the plane, and as such, we rephrase the construction as a combinatorial problem on graphs. Recall that an {\it $n$ uniform hypergraph} if a collection of {\it vertices} and {\it hyperedges}, where a hyperedge is a set of $n$ distinct vertices. We say such a graph is {\it hypartite} if we can partition the vertex set into $n$ sets $V_1, \dots, V_n$, such that each edge in the graph contains exactly one vertex from each set. An {\it independant set} is a subset of vertices containing no complete set of vertices in any edge. A {\it coloring} is a partition of the vertex set into finitely many classes, and we call each such class a {\it color}. The next lemma is a variant of Tur\'{a}n's theorem on the construction of independant sets, generalized for hypergraphs. For technical reasons, we need an extra restriction on the vertex set so it is `uniformly' chosen over the graph, which is the reason for the coloring.

\begin{lemma}
	Let $G$ be an $n$ uniform graph together with a coloring partitioning each vertex set into size $A$ independant sets. Then we can find an independent set $U$ containing all but $(n/A^n)|E|$ of the colors.
\end{lemma}
\begin{proof}
	Let $U$ be a vertex set chosen by randomly selecting a vertex of each color. Then each vertex in the graph is selected with probability $1/A$. For each edge $e = (v_1, \dots, v_n)$ in the graph, we know that the vertices $v_i$ all have different colors, so they are added to $U$ independantly, and so the chance that this edge connects vertices in the set we have constructed is
	%
	\begin{align*}
		&\mathbf{P}(v_1 \in U, \dots, v_n \in U)\\
		&\ \ \ \ = \mathbf{P}(v_1 \in U) \dots \mathbf{P}(v_n \in U) = 1/A^n
	\end{align*}
	%
	Thus if we let $E'$ denote the set of all edges $e = (u_1, \dots, u_n)$ with $u_1, \dots, u_n \in U$, then
	%
	\[ \mathbf{E}|E'| = \sum_{e \in E} \mathbf{P}(e \in E') = \sum_{e \in E} 1/A^n = \frac{|E|}{A^n} \]
	%
	In particular, we may choose a {\it particular}, nonrandom choice $U$ for which $|E'| \leq |E|/A^n$. If we form a vertex set $W \subset U$ by removing all vertices $u \in U$ which are a vertex in some edge in $E'$, then $W$ is an independant set containing all but $n |E'| \leq (n/A^n) |E|$ colors.
\end{proof}

\begin{corollary}
	If $|V| \gtrsim N^a$, $|E| \lesssim N^b$, and $A \gtrsim N^c$, where $b < a + c(n-1)$, then as $N \to \infty$ we can find an independent set containing all but a fraction $o(1)$ of all colors.
\end{corollary}
\begin{proof}
	A simple calculation on the quantities of the previous lemma yields
	%
	\begin{align*}
		&\frac{\# ( \text{colors removed} )}{\# ( \text{total colors} )} = \frac{(n/A^n)|E|}{|V|/A}\\
		&\ \ \ \ = \frac{n|E|}{|V|A^{n-1}} \lesssim \frac{N^b}{N^{a + c(n-1)}}
	\end{align*}
	%
	This is $o(1)$ if $b < a + c(n-1)$.
\end{proof}

We now apply these constructions on graphs to a problem which is quite clearly related to the fractal avoidance problem, and will form our key building block to constructing solutions to the problem. Given a fixed integer $N$, we subdivide $\mathbf{R}^d$ into a grid of sidelength $1/N$ cubes, the collection of such cubes we will denote by $\mathcal{I}$.

\begin{theorem}
	Suppose that $\text{dim}_{\mathbf{M}}(Y) < \alpha$. If $\mathcal{I}_1, \dots, \mathcal{I}_n$ are disjoint collections of length $1/N$ cubes in $\mathcal{I}$, with $|\mathcal{I}_i| \gtrsim N^d$ for each $i$, then provided $\beta < (n-1)/(n-\alpha)$, we can find a collection of length $1/N^\beta$ cubes $\mathcal{J}_1, \dots, \mathcal{J}_n$ with $\mathcal{J}_1 \times \dots \times \mathcal{J}_n$ disjoint from $Y$ and as $N \to \infty$, each $\mathcal{J}_i$ contains a cube in all but $o(1)$ of the cubes in $\mathcal{I}_i$.
\end{theorem}
\begin{proof}
	For sufficiently large $N$, if we partition $\mathbf{R}^{nd}$ into a grid of length $1/N^\beta$ cubes, and if $\mathcal{K}$ is the collection of all these cubes intersecting $Y$, then $|\mathcal{K}| \lesssim N^{\alpha \beta}$. Similarily, we partition each $\mathcal{I}_i$ into a grid of length $1/N^\beta$ cubes $\mathcal{I}'_i$, using these intervals as vertices in a hypartite graph $G$ with a hyperedge between $I_1 \in \mathcal{I}'_1, \dots, I_n \in \mathcal{I}'_n$ if $I_1 \times \dots \times I_n \in \mathcal{K}$. We say two cubes in $G$ are the same color if they are contained in a common cube in $\mathcal{I}_i$. Since each sidelength $1/N$ cube contains $N^{\beta d}/N^d = N^{(\beta - 1)d}$ sidelength $1/N^\beta$ cubes, each color in $G$ contains $N^{(\beta - 1)d}$ vertices. Each vertex set $V_i$ in $G$ contains $|\mathcal{I}'_i| = N^{\beta - 1} |\mathcal{I}_i| \gtrsim N^{\beta + (d-1)}$ vertices. Finally the number of edges is bounded above by $|\mathcal{K}| \lesssim N^{\alpha \beta}$. Thus in the parameters of the previous corollary, $a = \beta + (d-1)$, $b = \alpha \beta$, and $c = d(\beta - 1)$, and in particular, we can find an independent set $\mathcal{J}_1 \subset \mathcal{I}_1, \dots, \mathcal{J}_n \subset \mathcal{I}_n$ containing all but $o(1)$ of the colors provided that $\alpha \beta < \beta + (d-1) + (\beta - 1)d(n-1)$, which can be rearranged to give the inequality in the hypothesis.
\end{proof}

s

\end{multicols}

\begin{thebibliography}{9}

\bibitem{RuzsaSetsWithoutSquares}
I. Z. Ruzsa
\textit{Difference Sets Without Squares}

\bibitem{KeletiDimOneSet}
Tam\'{a}s Keleti
\textit{A 1-Dimensional Subset of the Reals that Intersects Each of its Translates in at Most a Single Point}

\bibitem{MalabikaRob}
Robert Fraser, Malabika Pramanik
\textit{Large Sets Avoiding Patterns}

\bibitem{DickmanK}
Karl Dickman
\textit{On the Frequency of Numbers Containing Prime Factors of a Certain Relative Magnitude}

\bibitem{Sudakov}
B. Sudakov, E. Szemer\'{e}di, V.H. Vu
\textit{On a Question of Erd\"{o}s and Moser}

\end{thebibliography}

%- Berend's counterexample is a discrete version of Kaleti's continuous counterexample for 3APs
%- Look up Wisewell functions
%- (x_2 - x_1)^2 = x_3 - x_1: Comes up in Bourgain & Chang

%- Principal character gives main term, rest should be thrown into the error term.   

\end{document}