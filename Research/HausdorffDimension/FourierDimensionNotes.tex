\documentclass[12pt,reqno]{article}

\usepackage{amsmath}
\usepackage{amsthm}
\usepackage{amssymb}

\usepackage{comment}
\usepackage{mathtools}
\usepackage{MnSymbol}

\usepackage[margin=1.5in]{geometry}

\newtheorem{theorem}{Theorem}
\newtheorem{lemma}[theorem]{Lemma}
\newtheorem{corollary}[theorem]{Corollary}
\newtheorem{prop}[theorem]{Proposition}
\newtheorem{exercise}[theorem]{Exercise}
\newtheorem{remark}[theorem]{Remark}

\DeclareMathOperator{\RR}{\mathbf{R}}
\DeclareMathOperator{\ZZ}{\mathbf{Z}}
\DeclareMathOperator{\EE}{\mathbf{E}}
\DeclareMathOperator{\PP}{\mathbf{P}}

\DeclareMathOperator{\DQ}{\mathcal{Q}}
\DeclareMathOperator{\DR}{\mathcal{R}}

\newcommand{\psitwo}[1]{\| {#1} \|_{\psi_2(L)}}
\newcommand{\TV}[2]{\| {#1} \|_{\text{TV}({#2})}}








\title{Salem Sets Avoiding Rough Configurations}
\author{Jacob Denson}

\begin{document}

\maketitle

Recall that a set $X \subset \RR^d$ is a \emph{Salem set} of dimension $s$ if it has Hausdorff dimension $s$, and for every $\varepsilon > 0$, there exists a probability measure $\mu_\varepsilon$ supported on $X$ such that for all $\xi \in \RR^d$,
%
\[ \sup_{\xi \in \RR^d} |\xi|^{s - \varepsilon} |\widehat{\mu_\varepsilon}(\xi)| < \infty. \]
%
Our goal in these notes is to obtain, for each set $Z \subset \RR^{dn}$ with Minkowski dimension $s$, a Salem set $X \subset \RR^d$ with dimension
%
\[ \frac{nd - s}{s}, \]
%
such that for each set of $n$ distinct elements $x_1, \dots, x_n \in X$, $(x_1, \dots, x_n) \not \in Z$. We hope that we can rely on the random selection approach of our paper on rough configurations to obtain such a result.

\section{Concentration Inequalities}

Define a convex function $\psi_2: [0,\infty) \to [0,\infty)$ by $\psi_2(t) = e^{t^2} - 1$, and a corresponding Orlicz norm on the family of scalar valued random variables $X$ over a probability space by setting
%
\[ \psitwo{X} = \inf \left\{ A \in (0,\infty) : \EE(\psi_2(|X|/A)) \leq 1 \right\}. \]
%
The family of random variables $\psi_2(L)$ are known as \emph{subgaussian random variables}. Here are some important properties:
%
\begin{itemize}
	\item (Gaussian Tails): If $\psitwo{X} \leq A$, then for each $t \geq 0$,
	%
	\[ \PP \left( |X| \geq t \right) \leq 10 \exp \left( -t^2/10A^2 \right). \]

	\item (Bounded Variables are Subgaussian): For any random $X$,
	%
	\[ \psitwo{X} \leq 10 \| X \|_{L^\infty}. \]

	%\item (Centering) For any random variable $X$,
	%
	%\[ \psitwo{X - \EE(X)} \lesssim \psitwo{X}. \]
	
	\item (Union Bound) If $X_1, \dots, X_N$ are random variables, then
	%
	\[ \psitwo{X_1 + \dots + X_N} \leq \psitwo{X_1} + \dots + \psitwo{X_N}. \]
	
	\item (Hoeffding's Inequality): If $X_1, \dots, X_N$ are \emph{independent} random variables, then
	%
	\[ \psitwo{X_1 + \dots + X_N} \leq 10 \left( \psitwo{X_1}^2 + \dots + \psitwo{X_N}^2 \right)^{1/2}. \]
\end{itemize}
%
The Orlicz norm is a convenient notation to summarize calculations involving the principle of concentration of measure. Roughly speaking, we can think of a random variable $X$ with $\psitwo{X} \leq A$ as essentially always lying in the interval $[-A,A]$, very rarely deviating outside this interval.

\section{A Family of Cubes}

Fix sequences of integers $\{ K_m : m \geq 1 \}$ and $\{ M_m : m \geq 1 \}$, and set $N_m = K_m M_m$. We then define two sequences of real numbers $\{ l_m : m \geq 0 \}$ and $\{ r_m : m \geq 0 \}$, by
%
\[ l_m = \frac{1}{N_1 \dots N_m} \quad\text{and}\quad r_m = \frac{1}{N_1 \dots N_{m-1} M_m}. \]
%
For each $m, d \geq 0$, we define two collections of strings
%
\[ \Sigma_m^d = \ZZ^d \times [M_1]^d \times [K_1]^d \times \dots \times [M_m]^d \times [K_m]^d \]
%
and
%
\[ \Pi_m^d = \ZZ^d \times [M_1]^d \times [K_1]^d \times \dots \times [K_{m-1}]^d \times [M_m]^d. \]
%
For each string $i \in \Sigma_m^d$, we define a vector $a_i \in (l_m \ZZ)^d$ by setting
%
\[ a_i = i_0 + \sum_{k = 1}^m i_{2k-1} \cdot r_k + i_{2k} \cdot l_k \]
%
Then each string $i \in \Sigma_m^d$ can be identified with the sidelength $l_m$ cube
%
\[ Q_i = \prod_{j = 1}^d \left[ a_{ij}, a_{ij} + l_m \right]. \]
%
centered at $a_i$. Similarily, for each string $i \in \Pi_m^d$, we define a vector $a \in (r_m \ZZ)^d$ by setting, for each $1 \leq j \leq d$,
%
\[ a_i = i_0 + \left( \sum_{k = 1}^{m-1} i_{2k-1} \cdot r_k + i_{2k} \cdot l_k \right) + i_{2m-1} \cdot r_m, \]
%
and then define a sidelength $r_m$ cube
%
\[ R_i = \prod_{j = 1}^d \left[ a_{ij}, a_{ij} + r_m \right]. \]
%
We let $\DQ_m^d = \{ Q_i : i \in \Sigma_m^d \}$, and $\DR_m^d = \{ R_i : i \in \Pi_m^d \}$. Here are some important properties of this collection of cubes:
%
\begin{itemize}
	\item For each $m$, $\DQ_m^d$ and $\DR_m^d$ are covers of $\RR^d$.

	\item If $Q_1,Q_2 \in \bigcup_{m = 0}^\infty \DQ_m^d$, then either $Q_1$ and $Q_2$ have disjoint interiors, or one cube is contained in the other. Similarily, if $R_1,R_2 \in \bigcup_{m = 1}^\infty \DR_m^d$, then either $R_1$ and $R_2$ have disjoint interiors, or one cube is contained in the other.

	\item For each cube $Q \in \DQ_m$, there is a unique cube $Q^* \in \DR_m$ with $Q \subset Q^*$. We refer to $Q^*$ as the \emph{parent cube} of $Q$. Similarily, if $R \in \DR_m$, there is a unique cube in $R^* \in \DQ_{m-1}$ with $R \subset R^*$, and we refer to $R^*$ as the \emph{parent cube} of $R$.
\end{itemize}

We say a set $E \subset \RR^d$ is $\DQ_m$ discretized if it is a union of cubes in $\DQ_m^d$, and we then let $\DQ_m(E) = \{ Q \in \DQ_m^d : Q \subset E \}$. Similarily, we say a set $E \subset \RR^d$ is $\DR_m$ discretized if it is a union of cubes in $\DR_m^d$, and we then let $\DR_m(E) = \{ R \in \DR_m^d : R \subset E \}$. We set $\Sigma_m(E) = \{ i \in \Sigma_m^d : Q_i \in \DQ_m(E) \}$, and $\Pi_m(E) = \{ i \in \Pi_m^d : R_i \in \DR_m(E) \}$. We say a cube $Q_1 \times \dots \times Q_n \in \DQ_m^{dn}$ is \emph{strongly non diagonal} if there does not exist two distinct indices $i,j$, and a third index $k \in \Pi_m^d$, such that $R_k \cap Q_i, R_k \cap Q_j \neq \emptyset$.

\section{A Family of Mollifiers}

We now consider a family of mollifiers, which we will use to smooth out the Fourier transform of the measures we study.

\begin{lemma} \label{lemma19020941290}
    There exists a non-negative, $C^\infty$ function $\psi$ supported on $[-1,1]^d$ such that
    %
    \begin{equation} \label{equation1249015901590190}
        \int_{\RR^d} \psi(x)\; dx = 1,
    \end{equation}
    %
    and for each $x \in \RR^d$,
    %
    \begin{equation} \label{equation50914902149012}
        \sum_{n \in \ZZ^d} \psi(x + n) = 1.
    \end{equation}
\end{lemma}
\begin{proof}
    Let $\alpha$ be a non-negative, $C^\infty$ function compactly supported on $[0,1]$, such that $\alpha(1/2 + x) = \alpha(1/2 - x)$ for all $x \in \RR$, $\alpha(x) = 1$ for $x \in [1/3,2/3]$, and $0 \leq \alpha(x) \leq 1$ for all $x \in \RR$. Then define $\beta$ to be the non-negative, $C^\infty$ function supported on $[-1/3,1/3]$ defined for $x \in [-1/3,1/3]$ by
    %
    \[ \beta(x) = 1 - \alpha(|x|). \]
    %
    Symmetry considerations imply that $\int \alpha(x) + \beta(x) = 1$, and for each $x \in \RR$,
    %
    \begin{equation} \label{equation129410294910}
        \sum_{m \in \ZZ} \alpha(x + m) + \beta(x + m) = 1.
    \end{equation}
    %
    If we set
    %
    \[ \psi_0(x) = \alpha(x) + \beta(x), \]
    %
    then $\psi_0$ satisfies the required constraints, at least in the one dimensional case. In general, define
    %
    \[ \psi(x_1, \dots, x_d) = \psi_0(x_1) \dots \psi_0(x_d). \qedhere \]
\end{proof}

Fix some choice of $\psi$ given by Lemma \ref{lemma19020941290}. Since $\psi$ is $C^\infty$ and compactly supported, then for each $t \in [0,\infty)$, we conclude
%
\begin{equation} \label{equation682928418931289}
	\sup_{\xi \in \RR^d} |\xi|^t |\widehat{\psi}(\xi)| < \infty.
\end{equation}
%
Now we rescale the mollifier. For each $m > 0$, we let
%
\[ \psi_m(x) = l_m^{-d} \psi(l_m \cdot x). \]
%
Then $\psi_m$ is supported on $[-l_m,l_m]^d$. Equation \eqref{equation1249015901590190} implies that for each $x \in \RR^d$,
%
\begin{equation} \label{equation19204910490190190}
	\int_{\RR^d} \psi_m = 1.
\end{equation}
%
Equation \eqref{equation50914902149012} implies
%
\begin{equation} \label{equation990249012409129041290} \sum_{n \in \ZZ^d} \psi(x + l_m \cdot n) = l_m^{-d}. \end{equation}
%
An important property of the rescaling in the frequency domain is that for each $\xi \in \RR^d$,
%
\begin{equation} \label{equation12901902419209012}
    \widehat{\psi_m}(\xi) = \widehat{\psi}(l_m \xi),
\end{equation}
%
In particular, \eqref{equation12901902419209012} implies that for each $t \geq 0$,
%
\begin{equation}
    \sup_{|\xi| \in \RR^d} |\widehat{\psi_m}(\xi)| |\xi|^t = l_m^{-t} \sup_{|\xi| \in \RR^d} |\widehat{\psi}(\xi)| |\xi|^t.
\end{equation}
%
Thus, uniformly in $m$, $\widehat{\psi_m}$ decays sharply outside of the box $[-l_m^{-1}, l_m^{-1}]^d$, a manifestation of the Heisenberg uncertainty principle.

\section{Discrete Lemma}

We now consider a discrete form of the Fourier bound argument, which we can apply iteratively to obtain a Salem set avoiding configurations.

\begin{comment}

\begin{lemma}
    Fix $s \in [1,dn)$ and $\varepsilon \in [0,(dn-s)/4)$. Let $T \subset [0,1]^d$ be a non-empty, $\DQ_m$ discretized set, and let $\mu_T$ be a smooth probability measure compactly supported on $T$, together with a constant $C > 0$ such that for each $m \in \ZZ^d$,
    %
    \[ |\widehat{\mu_T}(m)| \leq C |m|^{c\varepsilon-\frac{dn-s}{2n}}. \]
    %
    where
    %
    \begin{equation} \label{equation12904901249104910}
        c \geq 2 + \frac{2s + 3}{2n(dn-s)}.
    \end{equation}
    %
    Let $B \subset \RR^{dn}$ be a non-empty, $\DQ_{m+1}$ discretized set such that
    %
    \begin{equation} \label{equation1290419204912090120912}
       \#(\DQ_{m+1}(B)) \leq (1/l_{m+1})^{s + \varepsilon}.
    \end{equation}
    %
    Then there exists constants $C_1(\mu_T,n,d,s,\varepsilon)$ and $C_2(\mu_T,n,d,s,\varepsilon)$ such that if
    %
    % Must depend on d and \mu
    \begin{equation} \label{equation1095121284102}
        K_{m+1} \geq C_1(\mu_T,n,d,s,\varepsilon),
    \end{equation}
    %
    and
    %
    % Must depend on l_m, d, n, and s
    \begin{equation} \label{equation5890129048128941891}
        K_{m+1} \geq C_2(\mu_T,n,d,s,\varepsilon) M_{m+1}^{\frac{s+\varepsilon}{dn - s - \varepsilon}},
    \end{equation}
    %
    then there exists a $\DQ_{m+1}$ discretized set $S \subset T$ together with a smooth probability measure supported on $S$ such that
    %
    \begin{enumerate}
        \item[(A)] For any strongly non-diagonal cube
        %
        \[ Q = Q_1 \times \dots \times Q_n \in \DQ_{m+1}(B), \]
        %
        there exists $i$ such that $Q_i \not \in \DQ_{m+1}(S)$.

        \item[(B)] For any $m \in \ZZ^d$,
        %
        \[ |\widehat{\mu}(m)| \leq \max(1, (C + 10^d l_{m+1}^\varepsilon)) |m|^{\varepsilon-s/2}. \]
    \end{enumerate}
    %
    More precisely, the constants $C_0$ and $C_1$ must be chosen large enough so that \eqref{equation1095121284102} and \eqref{equation5890129048128941891} imply
    %
    \begin{equation} \label{equation10491249012}
        M_{m+1} \geq \left( 3^d \sqrt{d} \cdot l_m \| \nabla \mu \|_{L^\infty(\RR^d)} \right)^2,
    \end{equation}
    %
    \begin{equation} \label{equation19020912390129031290312}
        %N_{m+1} \geq \left( \frac{3^{dn}}{10 l_m^{s + \varepsilon}} \right)^{\frac{1}{dn - s - \varepsilon}} M_{m+1}^{\frac{s + \varepsilon}{dn - s - \varepsilon}}.
        K_{m+1} \geq \left( \frac{3^{dn}}{10 l_m^{dn}} \right)^{\frac{2}{dn - s}} M_{m+1}^{\frac{s + \varepsilon}{dn - s - \varepsilon}},
    \end{equation}
    %
    \begin{equation} \label{equation109519029012}
        K_{m+1} \leq M_{m+1}^{\frac{2dn}{dn - s}},
    \end{equation}
    %
    \begin{equation} \label{equation1940129041}
        M_{m+1} \geq \exp \left( \frac{10^7 (3dn - s) d^2}{dn - s} \right),
    \end{equation}
    %
    \begin{equation} \label{equation129041902412901290129}
        M_{m+1}^d K_{m+1}^d \geq A(d + 1 + s/2) 2^{1 + d + s/2} l_m^d,
    \end{equation}
    %
    \begin{equation} \label{equation190512919204912901}
    \begin{split}
        K_{m+1}^d M_{m+1}^d \geq \frac{l_m^d 8^d A(d + 1 + s/2)}{1 + s/2},
    \end{split}
    \end{equation}
    %
    \begin{equation} \label{equation68943893493849}
        K_{m+1}^d M_{m+1}^d \geq l_m^d 2^{3d + s/2 + 1} B(d + s/2 + 1),
    \end{equation}
    %
    where $A(t) = \sup |\widehat{\mu_T}(\xi)| |\xi|^{t}$, and $B(t) = \sup |\widehat{\psi}(\xi)| |\xi|^t$ for each $t \geq 0$.
\end{lemma}
\begin{proof}
	For each $i \in \Pi_{m+1}^d$, let $j_i$ be a random integer vector chosen from $[K_{m+1}]^d$, such that the family $\{ j_i : i \in \Pi_{m+1}^d \}$ is independent. We define a measure $\nu_S$ by setting, for each $x \in \RR^d$,
	%
	\[ d\nu_S(x) = r_{m+1}^d \sum_{i \in \Pi_{m+1}^d} \psi_{m+1}(x - a_{ij_i}) d\mu_T(x). \]
	%
	We then normalize, defining
	%
	\[ \mu_S = \frac{\nu_S}{\nu_S(\RR^d)}. \]
	%
	If we set
	%
	\[ S = \bigcup \{ Q \in \DQ_{m+1}^d : \mu_S(Q) > 0 \}, \]
	%
	then by definition, $S$ is a $\DQ_{m+1}$ discretized set, $\mu_S$ is supported on $S$, and $S \subset T$. It now suffices to show that with nonzero probability, $\DQ_m(S^n)$ is disjoint from the strongly non diagonal cubes in $\DQ_m(B)$, and $\mu_S$ has the required Fourier decay.

    In our calculations, it will be helpful to decompose the measure $\nu_S$. For each $i \in \Pi_{m+1}(T)$, define a random measure $\nu_i$ by setting
	%
	\[ d\nu_i(x) = r_{m+1}^d \psi_{m+1}(x - a_{ij_i}) d\mu_T(x). \]
	%
	Then $\nu_S = \sum_{i \in \Pi_{m+1}(T)} \nu_i$.

    First, let's show that $\nu_S$ only needs to be slightly perturbed to become a probability measure. Note that if $j_0,j_1 \in [K_{m+1}]^d$, then
	%
	\[ |a_{ij_0} - a_{ij_1}| = |j_0 - j_1| \cdot l_{m+1} \leq \sqrt{d} \cdot K_{m+1} l_{m+1} = \sqrt{d} \cdot r_{m+1}, \]
	%
	which, together with \eqref{equation19204910490190190}, implies
	%
	\begin{equation} \label{equation92941294129412919}
	\begin{split}
		&\left| r_{m+1}^d \int_{\RR^d} \psi_{m+1}(x - a_{ij_0}) \mu_T(x) - r_{m+1}^d \int_{\RR^d} \psi_{m+1}(x - a_{ij_1}) \mu_T(x) \right|\\
		&\ \ \ \ \ \ \ \ \leq r_{m+1}^d \int_{\RR^d} \psi_{m+1}(x) \left| \mu_T(x + a_{ij_0}) - \mu_T(x + a_{ij_1}) \right|\\
		&\ \ \ \ \ \ \ \ \leq \sqrt{d} \cdot r_{m+1}^{d+1} \int_{\RR^d} \psi_{m+1}(x) \| \nabla \mu_T \|_{L^\infty(\RR^d)} = \sqrt{d} \cdot r_{m+1}^{d+1} \| \nabla \mu \|_{L^\infty(\RR^d)}.
	\end{split}
	\end{equation}
	%
	%For each $i \in \Sigma_{m+1}^d$, if we set
	%
	%\[ A_i = \EE(\nu_i(\RR^d)) = \frac{1}{N_{n+1}^d} \sum_{j \in [N_{n+1}]^d} \int_{\RR^d} \psi_{m+1}(x - a_{ij}) \mu_T(x)\; dx, \]
	%
	Thus $\eqref{equation92941294129412919}$ implies that for each $i$,
	%
	\begin{equation} \label{equation491040912491}
		\| \nu_i(\RR^d) - \EE(\nu_i(\RR^d)) \|_{L^\infty} \leq \sqrt{d} \cdot r_{m+1}^{d+1} \| \nabla \mu \|_{L^\infty(\RR^d)}.
	\end{equation}
	%
	Furthermore, \eqref{equation990249012409129041290} implies
	%
	\begin{equation} \label{9921490124912}
	\begin{split}
		\sum_{i \in \Pi_{m+1}^d} \EE(\nu_i(\RR^d)) &= r_{m+1}^d \sum\nolimits_{(i,j) \in \Sigma_{m+1}^d} \PP(j_i = j) \int_{\RR^d} \psi_{m+1}(x - a_{ij}) \mu_T(x)\; dx\\
		&= \frac{r_{m+1}^d}{K_{m+1}^d} \int_{\RR^d} \left( \sum\nolimits_{(i,j) \in \Sigma_{m+1}^d} \psi_{m+1}(x - a_{ij}) \right) \mu_T(x)\; dx\\
		&= \frac{r_{m+1}^d l_{m+1}^{-d}}{K_{m+1}^d} = 1.
	\end{split}
	\end{equation}
	%
	For all but at most $3^d \cdot r_{m+1}^{-d}$ indices $i$, $\nu_i = 0$ almost surely. Thus we can apply the triangle inequality together with \eqref{equation491040912491} and \eqref{9921490124912} to conclude that
	%
	\begin{equation} \label{equation42214124124102412}
	\begin{split}
		\| \nu_S(\RR^d) - 1 \|_{L^\infty} &= \| \sum\nolimits_{i \in \Pi_{m+1}^d} \left[ \nu_i(\RR^d) - \EE(\nu_i(\RR^d)) \right] \|_{L^\infty} \\
		&\leq \sum\nolimits_{i \in \Pi_{m+1}^d} \| \nu_i(\RR^d) - \EE(\nu_i(\RR^d)) \|_{L^\infty}\\
		&\leq 3^d \sqrt{d} \cdot r_{m+1}^{-d} r_{m+1}^{d+1} \| \nabla \mu \|_{L^\infty(\RR^d)}\\
		&= 3^d \sqrt{d} \cdot r_{m+1} \| \nabla \mu \|_{L^\infty(\RR^d)}\\
        &= \frac{3^d \sqrt{d} \cdot l_m \| \nabla \mu \|_{L^\infty(\RR^d)}}{M_{m+1}}.
	\end{split}
	\end{equation}	
	%
    Thus \eqref{equation42214124124102412} and \eqref{equation10491249012} imply that $\| \nu_S(\RR^d) - 1 \|_{L^\infty} \leq M_{m+1}^{-1/2}$, or, in other words, that it is almost surely true that
    %
    \begin{equation} \label{equation86912904129041290}
        1 - M_{m+1}^{-1/2} \leq \nu_S(\RR^d) \leq 1 + M_{m+1}^{-1/2}.
    \end{equation}
    %
    This means that for a suitably large choice of $M_{m+1}$, normalizing this measure to a probability measure is neglible to the asymptotics we consider.

	For each $i \in \Pi_{m+1}^d$, let
	%
	\[ S_i = \bigcup \{ Q \in \DQ_{m+1}^d: \nu_i(Q) > 0 \}. \]
	%
	Then $S = \bigcup_{i \in \Pi_{m+1}^d} S_i$. Because $j_i$ is selected uniformly from $[K_{m+1}]^d$ for each $i$, for any $j \in [K_{m+1}]^d$,
    %
    \begin{equation} \label{equation129412904912090}
        \PP(j_i = j) = K_{m+1}^{-d}.
    \end{equation}
    %
    Since $\psi_{m+1}$ is supported on $[-l_{m+1},l_{m+1}]^d$,
	%
	\[ S_i \subset \bigcup \{ R_{i_0} : R_{i_0} \cap R_i \neq \emptyset \}. \]
	%
	For any cube $Q_{ij} \in \Sigma_{m+1}^d$, there are at most $3^d$ pairs $(i_0,j_0)$ such that $Q_{i_0j_0} \cap Q_{ij} \neq \emptyset$, and so a union bound together with \eqref{equation129412904912090} gives
	%
    \begin{equation} \label{equation5901490129129012409}
        \PP(Q_{ij} \in \DQ_{m+1}(S)) \leq \sum\nolimits_{Q_{i_0j_0} \cap Q_{ij} \neq \emptyset} \PP(j_{i'} = j') \leq 3^d K_{m+1}^{-d}.
    \end{equation}
	%
	Without loss of generality, removing cubes from $B$ if necessary, we may assume all cubes in $B$ are strongly non-diagonal. Let $Q = Q_{i_1j_1} \times \dots \times Q_{i_nj_n} \in \DQ_{m+1}(B)$ be such a cube. Since $Q$ is strongly diagonal, the events $\{ Q_{i_kj_k} \in S \}$ are independent from one another, which together with \eqref{equation5901490129129012409} implies that
	%
    \begin{equation} \label{equation190589012590812892189}
	   \PP(Q \in \DQ_{m+1}(S^n)) = \PP(Q_{i_1j_1} \in S) \dots \PP(Q_{i_nj_n} \in S) \leq 3^{dn} K_{m+1}^{-dn}.
    \end{equation}
	%
	Taking expectations over all cubes in $B$, and applying \eqref{equation1290419204912090120912} and \eqref{equation190589012590812892189} gives
	%
    \begin{equation} \label{equation129041289589128921891289}
	\begin{split}
		\EE(\#(\DQ_{m+1}(B) \cap \DQ_{m+1}(S^n))) &\leq 3^{dn} \#(\DQ_{m+1}(B)) \cdot K_{m+1}^{-dn}\\
		&\leq 3^{dn} l_{m+1}^{-(s + \varepsilon)} K_{m+1}^{- dn}\\
		&= \frac{3^{dn} M_{m+1}^{s + \varepsilon} l_m^{-(s + \varepsilon)}}{K_{m+1}^{dn - s - \varepsilon}}.
	\end{split}
    \end{equation}
    %
    We can then apply Markov's inequality with \eqref{equation129041289589128921891289} and \eqref{equation19020912390129031290312} to conclude
    %
    \begin{equation} \label{fourierdim2}
    \begin{split}
        \mathbf{P}(\DQ_{k+1}(B) \cap \DQ_{k+1}(S^n) \neq \emptyset) &= \mathbf{P}(\# (\DQ_{k+1}(B) \cap \DQ_{k+1}(S^n)) \geq 1)\\
        &\leq \EE(\#(\DQ_{m+1}(B) \cap \DQ_{m+1}(S^n)))\\
        &\leq 1/10.
    \end{split}
    \end{equation}
    %
    Thus $\DQ_{k+1}(S^n)$ is disjoint from $\DQ_{k+1}(B)$ with high probability.

    Now we analyze the Fourier transform of the measure $\nu$ over frequencies with small magnitude. For each $i \in \Pi_{m+1}^d$, and $m \in \ZZ$, define $X_{im} = \widehat{\nu_i}(m) - \widehat{\EE(\nu_i)}(m)$. Applying \eqref{equation50914902149012} gives 
    %
    \begin{equation} \label{equation891248921894128942189}
    \begin{split}
        \sum_{i \in \Pi_{m+1}^d} \widehat{\EE(\nu_i)}(m) &= \sum_{i \in \Pi_{m+1}^d} l_{m+1}^d \sum_{j \in [K_{m+1}]^d} \int_{\RR^d} e^{- 2 \pi i m \cdot x} \psi_{m+1}(x - a_{ij}) d\mu_T(x)\\
        &= \int_{\RR^d} e^{-2 \pi i m \cdot x} d\mu_T(x) = \widehat{\mu_T}(m).
    \end{split}
    \end{equation}
    %
    For each $i$ and $m$, the standard $(L^1,L^\infty)$ bound on the Fourier transform, combined with \eqref{equation491040912491}, shows
    %
    \begin{equation} \label{equation12904912049012}
    \begin{split}
        \psitwo{X_{im}} &\leq 10 \| X_{im} \|_{L^\infty}\\
        &\leq 10[\| \nu_i(\RR^d) \|_{L^\infty} + \EE(\nu_i)(\RR^d)]\\
        &\leq 10^2 \left( \EE(\nu_i)(\RR^d) + r_{m+1}^{d+1} \| \nabla \mu_T \|_{L^\infty(\RR^d)} \right).
    \end{split}
    \end{equation}
    %
    For a fixed $m$, the family of random variables $\{ X_{im} \}$ are independent. Furthermore, $\sum X_{im} = \widehat{\nu}(m) - \widehat{\EE(\nu)}(m)$. Equations \eqref{equation990249012409129041290} and \eqref{equation129412904912090} imply that
    %
    \begin{equation} \label{equation19241902490129021}
    \begin{split}
        \EE(\widehat{\nu_S}(m)) &= \frac{r_{m+1}^d}{K_{m+1}^d} \int_{\RR^d} e^{-2 \pi i m \cdot x} \left( \sum_{(i,j) \in \Sigma_{m+1}^d} \psi_{m+1}(x - a_{ij}) \right) d\mu_T(x)\\
        &= \frac{r_{m+1}^d l_{m+1}^{-d}}{K_{m+1}^d} \int_{\RR^d} e^{-2 \pi i m \cdot x} d\mu_T(x)\\
        &= \frac{r_{m+1}^d l_{m+1}^{-d}}{K_{m+1}^d} \widehat{\mu_T}(m) = \widehat{\mu_T}(m).
    \end{split}
    \end{equation}
    %
    Hoeffding's inequality, together with \eqref{equation12904912049012} and \eqref{equation19241902490129021}, imply that
    %
    \begin{equation} \label{equation190219024901290129041}
    \begin{split}
        & \psitwo{\widehat{\nu}(m) - \widehat{\mu_T}(m)}\\
        &\ \ \ \ \ \ \ \ \ \ \leq 10^3 \sqrt{d} \left( \left( \sum \EE(\nu_i)(\RR^d)^2 \right)^{1/2} + r_{m+1}^{d/2+1} \| \nabla \mu_T \|_{L^\infty(\RR^d)} \right).
    \end{split}
    \end{equation}
    %
    Equation \eqref{equation19204910490190190} shows
    %
    \begin{equation} \label{equation129401924901290412}
    \begin{split}
        \EE(\nu_i)(\RR^d) &= l_{m+1}^d \sum_{j \in [K_{m+1}]^d} \int \psi_{m+1}(x - a_{ij}) d\mu_T(x)\\
        &\leq r_{m+1}^d \| \mu_T \|_{L^\infty(\RR^d)}.
    \end{split}
    \end{equation}
    %
    Combining \eqref{equation190219024901290129041} and \eqref{equation129401924901290412} gives
    %
    \begin{equation} \label{equation10491204901290}
        \psitwo{\widehat{\nu}(m) - \widehat{\mu_T}(m)} \leq 10^3 \sqrt{d} \left[ \| \mu_T \|_{L^\infty(\RR^d)} + \| \nabla \mu_T \|_{L^\infty(\RR^d)} \right] r_{m+1}^{d/2}.
    \end{equation}
    %
    We can then apply a union bound over $D = \{ m \in \ZZ^d : |m| \leq 10 l_{m+1}^{-1} \}$, which has cardinality at most $10^{d+1} l_{m+1}^{-d}$, together with \eqref{equation10491204901290} to conclude that
    %
    \begin{equation} \label{equation1290421941021290}
    \begin{split}
        \PP& \left(\| \widehat{\nu} - \widehat{\mu_T} \|_{L^\infty(D)} \geq r_{m+1}^{d/2} \log(M_{m+1}) \right)\\
        &\ \ \ \ \ \leq 10^{d+2} \cdot l_{m+1}^{-d} \exp \left( - \frac{\log(M_{m+1})^2}{10^7 d} \right)\\
        &\ \ \ \ \ = 10^{d+2} l_m^{-d} \exp \left( d \log(M_{m+1} K_{m+1}) - \frac{\log(M_{m+1})^2}{10^7 d} \right).
    \end{split}
    \end{equation}
    %
    Combined with \eqref{equation109519029012} and \eqref{equation1940129041}, \eqref{equation1290421941021290} implies
    %
    \begin{equation} \label{equation0148912489128}
        \PP \left(\| \widehat{\nu} - \widehat{\mu_T} \|_{L^\infty(D)} \geq r_{m+1}^{d/2} \log(M_{m+1}) \right) \leq 1/10.
    \end{equation}
    %
    Thus $\widehat{\nu}$ and $\widehat{\mu_T}$ are highly likely to differ only by a neglible amount over small frequencies.

    It remains to analyze values of $\widehat{\nu_S}(m)$ when $|m| \geq 10 l_{m+1}^{-1}$. If we define a random measure
    %
    \[ \alpha = r_{m+1}^d \sum\nolimits_{\substack{i \in \Pi_{m+1}^d\\d(a_i,T) \leq 2 r_{m+1}^{-1}}} \delta_{a_{ij_i}}, \]
    %
    then $\nu_S = (\alpha * \psi_{m+1}) \mu_T$.
    %Since $\mu_T$ is supported on $T$, we can really truncate the sum defining $\eta$ to $O_d(r_{m+1}^{-d})$ terms such that the convolution identity remains valid.
    Thus we have $\widehat{\nu_S} = (\widehat{\alpha} \cdot \widehat{\psi_{m+1}}) * \widehat{\mu_T}$. Since $\mu_T$ is compactly supported, we can define, for each $t > 0$,
    %
    \[ A(t) = \sup |\widehat{\mu_T}(\xi)| |\xi|^t < \infty. \]
    %
    In light of \eqref{equation12901902419209012}, if we define, for each $t > 0$,
    %
    \[ B(t) = \sup |\widehat{\psi}(\xi)| |\xi|^t, \]
    %
    then
    %
    \[ \sup |\widehat{\psi_{m+1}}(\xi)| |\xi|^t = l_{m+1}^{-t} B(t). \]
    %
    We now aim to obtain asymptotic bounds on
    %
    \[ \widehat{\nu_S}(\eta) = \int \widehat{\mu_T}(\eta - \xi) \widehat{\alpha}(\xi) \widehat{\psi_{m+1}}(\xi)\; d\xi, \]
    %
    assuming $\eta \geq 10 l_{m+1}^{-1}$. The measure $\alpha$ is the sum of at most $2^d r_{m+1}^{-d}$ delta functions, so we have $\alpha(\RR^d) \leq 2^d$, so $\| \widehat{\alpha} \|_{L^\infty(\RR^d)} \leq 2^d$. Thus
    %
    \begin{equation} \label{equation9942941924912912}
        |\widehat{\nu_S}(\eta)| \leq 2^d \int |\widehat{\mu_T}(\eta - \xi)| |\widehat{\psi_{m+1}}(\xi)|\; d\xi.
    \end{equation}
    %
    If $|\xi| \leq |\eta|/2$, $|\eta - \xi| \geq |\eta|/2$, so for all $t > 0$, and since \eqref{equation19204910490190190} implies $\| \widehat{\psi_{m+1}} \|_{L^\infty(\RR^d)} \leq 1$, we find
    %
    \begin{equation} \label{equation999941204192049012}
        \int_{0 \leq |\xi| \leq |\eta|/2} |\widehat{\mu_T}(\eta - \xi)| |\widehat{\psi_{m+1}}(\xi)|\; d\xi \leq \frac{A(t) 2^{t-d}}{|\eta|^{t-d}}.
    \end{equation}
    %
    Set $t = d + 1 + s/2$. Equation \eqref{equation999941204192049012}, together with \eqref{equation129041902412901290129}, implies
    %
    \begin{equation} \label{equation1111902491209012}
    \begin{split}
        \int_{0 \leq |\xi| \leq |\eta|/2} |\widehat{\mu_T}(\eta - \xi)| |\widehat{\psi_{m+1}}(\xi)|\; d\xi &\leq \frac{A(d + 1 + s/2) 2^{1 + s/2} |\eta|^{-1}}{|\eta|^{s/2}}\\
        &\leq \frac{A(d + 1 + s/2) 2^{1 + s/2} l_{m+1}}{|\eta|^{s/2}}\\
        &\leq \frac{1}{10 \cdot 2^d} \frac{1}{|\eta|^{s/2}}.
    \end{split}
    \end{equation}
    %
    Conversely, if $|\xi| \geq 2|\eta|$, then $|\eta - \xi| \geq |\xi|/2$, so for each $t > d$,
    %
    \begin{equation} \label{equation5551092491022109}
    \begin{split}
        \int_{|\xi| \geq 2|\eta|} |\widehat{\mu_T}(\eta - \xi)| |\widehat{\psi}_{m+1}(\xi)|\; d\xi &\leq \int_{|\xi| \geq 2|\eta|} \frac{A(t) 2^t}{|\xi|^t}\\
        &\leq 2^d \int_{2|\eta|}^\infty r^{d-1 - t} A(t) 2^t\\
        &\leq \frac{4^d A(t)}{t - d} |\eta|^{d-t}.
    \end{split}
    \end{equation}
    %
    Set $t = d + 1 + s/2$. Equation \eqref{equation190512919204912901}, applied to \eqref{equation5551092491022109}, allows us to conclude
    %
    \begin{equation} \label{equation123123123}
        \int_{|\xi| \geq 2|\eta|} |\widehat{\mu_T}(\eta - \xi)| |\widehat{\psi}_{m+1}(\xi)|\; d\xi \leq \frac{1}{10 \cdot 2^d} \frac{1}{|\eta|^{s/2}}.
    \end{equation}
    %
    Finally, if $t > 0$, we use the fact that $\| \widehat{\mu_T} \|_{L^\infty(\RR^d)} \leq 1$ to conclude that
    %
    \begin{equation} \label{equation59010491092}
    \begin{split}
        \int_{|\eta|/2 \leq |\xi| \leq 2|\eta|} |\widehat{\mu_T}(\eta - \xi)| |\widehat{\psi_{m+1}}(\xi)|\; d\xi &\leq \frac{2^{d+t} B(t)}{|\eta|^{t-d}}.
    \end{split}
    \end{equation}
    %
    Set $t = d + s/2 + 1$. Then \eqref{equation59010491092} and \eqref{equation68943893493849} imply
    %
    \begin{equation} \label{equation2194129}
        \int_{|\eta|/2 \leq |\xi| \leq 2|\eta} |\widehat{\mu_T}(\eta - \xi)| |\widehat{\psi_{m+1}}(\xi)|\; d\xi \leq \frac{1}{10 \cdot 2^d} \frac{1}{|\xi|^{s/2}}.
    \end{equation}
    %
    Summing up \eqref{equation1111902491209012}, \eqref{equation123123123}, and \eqref{equation2194129}, we conclude from \eqref{equation9942941924912912} that if $|\eta| \geq 10 l_{m+1}^{-1}$, then
    %
    \begin{equation} \label{equation12904120941290419029012}
        |\widehat{\nu_S}(\eta)| \leq \frac{1}{2|\eta|^{s/2}}.
    \end{equation}
    %
    Thus, almost surely, $\widehat{\nu_S}(\eta)$ has fast Fourier decay for $|\eta| \geq 10 l_{m+1}^{-1}$.

    Let us put all our calculations together. In light of \eqref{fourierdim2} and \eqref{equation0148912489128}, there exists some choice of $j_i$ for each $i$, and a resultant non-random pair $(\nu_{S_0}, S_0)$ such that $S_0$ satisfies Property (A) of the Lemma, and for any $m \in \ZZ^d$ with $|m| \leq 10 l_{m+1}^{-1}$,
    %
    \begin{equation} \label{equation129419024129}
        |\widehat{\nu_{S_0}}(m) - \widehat{\mu_T}(m)| \leq r_{m+1}^{d/2} \log(M_{m+1}).
    \end{equation}
    %
    Equation \eqref{equation12904901249104910}, \eqref{equation1095121284102}, and \eqref{equation5890129048128941891} show
    %
    \[ r_{m+1}^{d/2} \log(M_{m+1}) \leq l_{m+1}^{-\frac{dn - s}{2n} - c\varepsilon + \varepsilon}. \]
    %
    % r_{m+1}^{d/2} \leq l_{m+1}^{(dn - s)/2n - c*eps}
    % l_m^{s/2n + c*eps} K_{m+1}^{(dn - s)/2n - c*eps} \leq M_{m+1}^{s/2n + c*eps}
    % A(l_m) K^{(dn - s)/2n - c*eps} <= M^{s/2n + c*eps}
    %
    % A(l_m) K^{- c * eps} K^{(dn - s)/2n)} <= M^{s/2n + c*eps}
    % B(l_m) K^{- c * eps} M^{(dn - s)(s + eps)/2n(dn - s - eps)} <= M^{s/2n + c*eps}
    % B(l_m) K^{- c * eps} <= M^{s/2n + c*eps}
    %
    % s + 2nc*eps - [1 + eps/(dn - s - eps)] (s + eps) >= 0
    % c >= [2s + 3]/2n(dn-s)
    %
    Thus we conclude that if $|m| \leq 10 l_{m+1}^{-1}$,
    %
    \begin{align*}
        |\widehat{\nu_{S_0}}(m)| &\leq r_{m+1}^{d/2} \log(M_{m+1}) + C |m|^{c\varepsilon - \frac{dn-s}{2n}}\\
        &\leq l_{m+1}^{-\frac{dn-s}{2n}-c\varepsilon + \varepsilon} + C|m|^{c\varepsilon - \frac{dn-s}{2n}}\\
        &\leq [C + 10^d l_{m+1}^\varepsilon] |m|^{c\varepsilon - \frac{dn-s}{2n}}.
    \end{align*}
    %
    Since $|\widehat{\nu_{S_0}}(m)| \leq |m|^{-\frac{dn-s}{2n}}$ holds automatically for $|m| \geq 10l_{m+1}^{-1}$, this completes the proof.
\end{proof}

\end{comment}

\begin{lemma} \label{discreteLemma}
    Fix $s \in [1,dn)$ and $\varepsilon \in [0,(dn-s)/2)$. Let $T \subset [0,1]^d$ be a non-empty, $\DQ_m$ discretized set, and let $\mu_T$ be a smooth probability measure compactly supported on $T$, together with a constant $A \geq 1$ such that for each $m \in \ZZ^d$,
    %
    \[ |\widehat{\mu_T}(m)| \leq A \cdot |m|^{a\varepsilon-\frac{dn-s}{2n}}. \]
    %
    where
    %
    \[ a = \frac{3d + 2dn - 2s}{dn}. \]
    %
    Let $B \subset \RR^{dn}$ be a non-empty, $\DQ_{m+1}$ discretized set such that
    %
    \begin{equation} \label{equation1290419204912090120912}
       \#(\DQ_{m+1}(B)) \leq (1/l_{m+1})^{s + \varepsilon}.
    \end{equation}
    %
    Then there exists a large constant $C(\mu_T,n,d,s,\varepsilon)$, such that if
    %
    % Must depend on d and \mu
    \begin{equation} \label{equation1095121284102}
        K_{m+1}, M_{m+1} \geq C(\mu_T,n,d,s,\varepsilon),
    \end{equation}
    %
    and
    %
    % Must depend on l_m, d, n, and s
    \begin{equation} \label{equation5890129048128941891}
        M_{m+1}^{\frac{s}{dn-s} + c\varepsilon} \leq K_{m+1} \leq 2 M_{m+1}^{\frac{s}{dn-s} + c \varepsilon},
    \end{equation}
    %
    where
    %
    \[ c = \frac{6dn}{(dn - s)^2}, \]
    %
    then there exists a $\DQ_{m+1}$ discretized set $S \subset T$ together with a smooth probability measure supported on $S$ such that
    %
    \begin{enumerate}
        \item[(A)] For any strongly non-diagonal cube
        %
        \[ Q = Q_1 \times \dots \times Q_n \in \DQ_{m+1}(B), \]
        %
        there exists $i$ such that $Q_i \not \in \DQ_{m+1}(S)$.

        \item[(B)] For any $m \in \ZZ^d$,
        %
        \[ |\widehat{\mu}(m)| \leq (1 + M_{m+1}^{-1/2}) [A + 10^d M_{m+1}^{-\varepsilon}] |m|^{c\varepsilon - \frac{dn-s}{2n}}. \]
    \end{enumerate}
\end{lemma}

\begin{proof}[Proof of Lemma \ref{discreteLemma}]
    \renewcommand{\qedsymbol}{}
    First, we describe the construction of the set $S$, and the measure $\mu_S$. For each $i \in \Pi_{m+1}^d$, let $j_i$ be a random integer vector chosen from $[K_{m+1}]^d$, such that the family $\{ j_i : i \in \Pi_{m+1}^d \}$ is independent. Then it is certainly true for any $j \in [K_{m+1}]^d$ that
    %
    \begin{equation} \label{equation129412904912090}
        \PP(j_i = j) = K_{m+1}^{-d}.
    \end{equation}
    %
    We define a measure $\nu_S$ such that, for each $x \in \RR^d$,
    %
    \[ d\nu_S(x) = r_{m+1}^d \sum\nolimits_{i \in \Pi_{m+1}^d} \psi_{m+1}(x - a_{ij_i}) d\mu_T(x). \]
    %
    We then normalize, defining
    %
    \[ \mu_S = \frac{\nu_S}{\nu_S(\RR^d)}. \]
    %
    If we set
    %
    \[ S = \bigcup \{ Q \in \DQ_{m+1}^d : \mu_S(Q) > 0 \}, \]
    %
    then $S$ is $\DQ_{m+1}$ discretized, $\mu_S$ is supported on $S$, and $S \subset T$. Our goal is to show, with non-zero probability, some choice of $\{ j_i \}$ yields a set $S$ satisfying Properties (A) and (B) of the Lemma.

    In our calculations, it will help us to decompose the measure $\nu_S$ into components roughly supported on sidelength $r_{m+1}^d$ cubes. For each $i \in \Pi_{m+1}(T)$, define a measure $\nu_i$ such that for each $x \in \RR^d$,
    %
    \[ d\nu_i(x) = r_{m+1}^d \psi_{m+1}(x - a_{ij_i}) d\mu_T(x). \]
    %
    Then $\nu_S = \sum_{i \in \Pi_{m+1}^d(T)} \nu_i$. We shall split the proof of the statement into several, more managable lemmas.
\end{proof}

\begin{lemma} \label{nuNormalizationLemma}
    If
    %
    \begin{equation} \label{equation10491249012}
        M_{m+1} \geq \left( 3^d \sqrt{d} \cdot l_m \| \nabla \mu \|_{L^\infty(\RR^d)} \right)^2,
    \end{equation}
    %
    then almost surely, $|\nu(\RR^d) - 1| \leq M_{m+1}^{-1/2}$.
\end{lemma}
\begin{proof}
    If $j_0, j_1 \in [K_{m+1}]^d$, then
    %
    \[ |a_{ij_0} - a_{ij_1}| = |j_0 - j_1| \cdot l_{m+1} \leq (\sqrt{d} K_{m+1}) \cdot l_{m+1} = \sqrt{d} \cdot r_{m+1}, \]
    %
    which, together with \eqref{equation19204910490190190}, implies
    %
    \begin{equation} \label{equation92941294129412919}
    \begin{split}
        &\left| r_{m+1}^d \int_{\RR^d} \psi_{m+1}(x - a_{ij_0}) \mu_T(x) - r_{m+1}^d \int_{\RR^d} \psi_{m+1}(x - a_{ij_1}) \mu_T(x) \right|\\
        &\ \ \ \ \ \ \ \ \leq r_{m+1}^d \int_{\RR^d} \psi_{m+1}(x) \left| \mu_T(x + a_{ij_0}) - \mu_T(x + a_{ij_1}) \right|\\
        &\ \ \ \ \ \ \ \ \leq \sqrt{d} \cdot r_{m+1}^{d+1} \cdot \| \nabla \mu_T \|_{L^\infty(\RR^d)} \int_{\RR^d} \psi_{m+1}(x) = \sqrt{d} \cdot r_{m+1}^{d+1} \cdot \| \nabla \mu_T \|_{L^\infty(\RR^d)}.
    \end{split}
    \end{equation}
    %
    %For each $i \in \Sigma_{m+1}^d$, if we set
    %
    %\[ A_i = \EE(\nu_i(\RR^d)) = \frac{1}{N_{n+1}^d} \sum_{j \in [N_{n+1}]^d} \int_{\RR^d} \psi_{m+1}(x - a_{ij}) \mu_T(x)\; dx, \]
    %
    Thus $\eqref{equation92941294129412919}$ implies that almost surely, for each $i$,
    %
    \begin{equation} \label{equation491040912491}
        |\nu_i(\RR^d) - \EE(\nu_i(\RR^d))| \leq \sqrt{d} \cdot r_{m+1}^{d+1} \| \nabla \mu \|_{L^\infty(\RR^d)}.
    \end{equation}
    %
    Furthermore, \eqref{equation990249012409129041290} implies
    %
    \begin{equation} \label{9921490124912}
    \begin{split}
        \sum_{i \in \Pi_{m+1}^d} \EE(\nu_i(\RR^d)) &= r_{m+1}^d \sum\nolimits_{(i,j) \in \Sigma_{m+1}^d} \PP(j_i = j) \int_{\RR^d} \psi_{m+1}(x - a_{ij}) \mu_T(x)\; dx\\
        &= \frac{r_{m+1}^d}{K_{m+1}^d} \int_{\RR^d} \left( \sum\nolimits_{(i,j) \in \Sigma_{m+1}^d} \psi_{m+1}(x - a_{ij}) \right) \mu_T(x)\; dx\\
        &= \frac{r_{m+1}^d l_{m+1}^{-d}}{K_{m+1}^d} = 1.
    \end{split}
    \end{equation}
    %
    For all but at most $3^d \cdot r_{m+1}^{-d}$ indices $i$, $\nu_i = 0$ almost surely. Thus we can apply the triangle inequality together with \eqref{equation491040912491} and \eqref{9921490124912} to conclude that almost surely,
    %
    \begin{equation} \label{equation42214124124102412}
    \begin{split}
        |\nu_S(\RR^d) - 1| &= \| \sum\nolimits_{i \in \Pi_{m+1}^d} \left[ \nu_i(\RR^d) - \EE(\nu_i(\RR^d)) \right] \|_{L^\infty} \\
        &\leq \sum\nolimits_{i \in \Pi_{m+1}^d} \| \nu_i(\RR^d) - \EE(\nu_i(\RR^d)) \|_{L^\infty}\\
        &\leq 3^d \sqrt{d} \cdot r_{m+1}^{-d} r_{m+1}^{d+1} \| \nabla \mu \|_{L^\infty(\RR^d)}\\
        &= 3^d \sqrt{d} \cdot r_{m+1} \| \nabla \mu \|_{L^\infty(\RR^d)}\\
        &= \frac{3^d \sqrt{d} \cdot l_m \| \nabla \mu \|_{L^\infty(\RR^d)}}{M_{m+1}}.
    \end{split}
    \end{equation}  
    %
    Thus \eqref{equation42214124124102412} and \eqref{equation10491249012} imply that almost surely, $|\nu_S(\RR^d) - 1| \leq M_{m+1}^{-1/2}$.
\end{proof}

\begin{lemma} \label{propertyALemma}
    If
    %
    \begin{equation} \label{equation194012904129009}
        M_{m+1} \geq \left( 10 \cdot 3^{dn} \cdot l_m^{-(s + \varepsilon)} \right)^{1/\varepsilon},
    \end{equation}
    %
    then
    %
    \[ \PP \left(S\ \text{\normalfont{does not satisfies Property (A)}} \right) \leq 1/10. \]
\end{lemma}
\begin{proof}
    For any cube $Q_{ij} \in \Sigma_{m+1}^d$, there are at most $3^d$ pairs $(i_0,j_0) \in \Sigma_{m+1}^d$ such that $Q_{i_0j_0} \cap Q_{ij} \neq \emptyset$, and so a union bound together with \eqref{equation129412904912090} gives
    %
    \begin{equation} \label{equation5901490129129012409}
        \PP(Q_{ij} \in \DQ_{m+1}(S)) \leq \sum\nolimits_{Q_{i_0j_0} \cap Q_{ij} \neq \emptyset} \PP(j_{i'} = j') \leq 3^d K_{m+1}^{-d}.
    \end{equation}
    %
    Without loss of generality, removing cubes from $B$ if necessary, we may assume all cubes in $B$ are strongly non-diagonal. Let $Q = Q_{i_1j_1} \times \dots \times Q_{i_nj_n} \in \DQ_{m+1}(B)$ be such a cube. Since $Q$ is strongly diagonal, the events $\{ Q_{i_kj_k} \in S \}$ are independent from one another, which together with \eqref{equation5901490129129012409} implies that
    %
    \begin{equation} \label{equation190589012590812892189}
       \PP(Q \in \DQ_{m+1}(S^n)) = \PP(Q_{i_1j_1} \in S) \dots \PP(Q_{i_nj_n} \in S) \leq 3^{dn} K_{m+1}^{-dn}.
    \end{equation}
    %
    Taking expectations over all cubes in $B$, and applying \eqref{equation1290419204912090120912} and \eqref{equation190589012590812892189} gives
    %
    \begin{equation} \label{equation129041289589128921891289}
    \begin{split}
        \EE(\#(\DQ_{m+1}(B) \cap \DQ_{m+1}(S^n))) &\leq \#(\DQ_{m+1}(B)) \cdot (3^{dn} K_{m+1}^{-dn})\\
        &\leq l_{m+1}^{-(s + \varepsilon)} (3^{dn} K_{m+1}^{- dn})\\
        &= 3^{dn} l_m^{-(s + \varepsilon)} \frac{M_{m+1}^{s + \varepsilon}}{K_{m+1}^{dn - s - \varepsilon}}.
    \end{split}
    \end{equation}
    % M^{(s + e)/(dn - s - e)} <= K
    % (s + e)/(dn - s - e) \leq (s + e[s + 3])/(dn - s)
    %
    Since $\varepsilon \leq (dn - s)/2$, we conclude
    %
    \begin{align*}
        \left( dn - s - \varepsilon \right) \left( \frac{s}{dn - s} + c\varepsilon \right) &= s + \varepsilon \left( c(dn - s - \varepsilon) - \frac{s}{dn - s} \right)\\
        &\geq s + \varepsilon \left( \frac{c(dn - s)}{2} - \frac{s}{dn - s} \right)\\
        &= s + \varepsilon \frac{3dn - s}{dn - s} \geq s + 2\varepsilon.
    \end{align*}
    %
    Applying \eqref{equation5890129048128941891} together with this bound, we conclude that
    %
    \begin{align*}
        K_{m+1}^{dn - s - \varepsilon} &\geq M_{m+1}^{(dn - s - \varepsilon) \left( \frac{s}{dn - s} + c\varepsilon \right)} \geq M_{m+1}^{s + 2 \varepsilon}.
    \end{align*}
    %
    Combined with \eqref{equation194012904129009}, we conclude that
    %
    \begin{equation} \label{equation1290419024190}
        \frac{3^{dn} l_m^{-(s + \varepsilon)} M_{m+1}^{s + \varepsilon}}{K_{m+1}^{dn - s - \varepsilon}} \leq \frac{3^{dn} l_m^{-(s + \varepsilon)}}{M_{m+1}^\varepsilon} \leq 1/10.
    \end{equation}
    %
    We can then apply Markov's inequality with \eqref{equation129041289589128921891289} and \eqref{equation1290419024190} to conclude
    %
    \begin{equation} \label{fourierdim2}
    \begin{split}
        \mathbf{P}(\DQ_{k+1}(B) \cap \DQ_{k+1}(S^n) \neq \emptyset) &= \mathbf{P}(\# (\DQ_{k+1}(B) \cap \DQ_{k+1}(S^n)) \geq 1)\\
        &\leq \EE(\#(\DQ_{m+1}(B) \cap \DQ_{m+1}(S^n)))\\
        &\leq 1/10. \qedhere
    \end{split}
    \end{equation}
    %
%    Thus $\DQ_{k+1}(S^n)$ is disjoint from $\DQ_{k+1}(B)$ with high probability.
\end{proof}

\begin{lemma} \label{deviationLemma}
    Set $D = \{ m \in \ZZ^d: |m| \leq 10l_{m+1}^{-1} \}$. Then if
    %
    \begin{equation} \label{equation109519029012}
        K_{m+1} \leq M_{m+1}^{\frac{2dn}{dn - s}},
    \end{equation}
    %
    and
    %
    \begin{equation} \label{equation1940129041}
        M_{m+1} \geq \exp \left( \frac{10^7 (3dn - s) d^2}{dn - s} \right),
    \end{equation}
    %
    then
    %
    \[ \PP \left( \| \widehat{\nu_T} - \widehat{\mu_T} \|_{L^\infty(D)} \geq r_{m+1}^{d/2} \log(M_{m+1}) \right) \leq 1/10 \]
\end{lemma}
\begin{proof}
    For each $i \in \Pi_{m+1}^d$, and $m \in \ZZ$, define $X_{im} = \widehat{\nu_i}(m) - \widehat{\EE(\nu_i)}(m)$. Applying \eqref{equation50914902149012} gives 
    %
    \begin{equation} \label{equation891248921894128942189}
    \begin{split}
        \sum_{i \in \Pi_{m+1}^d} \widehat{\EE(\nu_i)}(m) &= \sum_{i \in \Pi_{m+1}^d} l_{m+1}^d \sum_{j \in [K_{m+1}]^d} \int_{\RR^d} e^{- 2 \pi i m \cdot x} \psi_{m+1}(x - a_{ij}) d\mu_T(x)\\
        &= \int_{\RR^d} e^{-2 \pi i m \cdot x} d\mu_T(x) = \widehat{\mu_T}(m).
    \end{split}
    \end{equation}
    %
    For each $i$ and $m$, the standard $(L^1,L^\infty)$ bound on the Fourier transform, combined with \eqref{equation491040912491}, shows
    %
    \begin{equation} \label{equation12904912049012}
    \begin{split}
        \psitwo{X_{im}} &\leq 10 \| X_{im} \|_{L^\infty}\\
        &\leq 10[\| \nu_i(\RR^d) \|_{L^\infty} + \EE(\nu_i)(\RR^d)]\\
        &\leq 10^2 \left( \EE(\nu_i)(\RR^d) + r_{m+1}^{d+1} \| \nabla \mu_T \|_{L^\infty(\RR^d)} \right).
    \end{split}
    \end{equation}
    %
    For a fixed $m$, the family of random variables $\{ X_{im} \}$ are independent. Furthermore, $\sum X_{im} = \widehat{\nu}(m) - \widehat{\EE(\nu)}(m)$. Equations \eqref{equation990249012409129041290} and \eqref{equation129412904912090} imply that
    %
    \begin{equation} \label{equation19241902490129021}
    \begin{split}
        \EE(\widehat{\nu_S}(m)) &= \frac{r_{m+1}^d}{K_{m+1}^d} \int_{\RR^d} e^{-2 \pi i m \cdot x} \left( \sum_{(i,j) \in \Sigma_{m+1}^d} \psi_{m+1}(x - a_{ij}) \right) d\mu_T(x)\\
        &= \frac{r_{m+1}^d l_{m+1}^{-d}}{K_{m+1}^d} \int_{\RR^d} e^{-2 \pi i m \cdot x} d\mu_T(x)\\
        &= \frac{r_{m+1}^d l_{m+1}^{-d}}{K_{m+1}^d} \widehat{\mu_T}(m) = \widehat{\mu_T}(m).
    \end{split}
    \end{equation}
    %
    Hoeffding's inequality, together with \eqref{equation12904912049012} and \eqref{equation19241902490129021}, imply that
    %
    \begin{equation} \label{equation190219024901290129041}
    \begin{split}
        & \psitwo{\widehat{\nu}(m) - \widehat{\mu_T}(m)}\\
        &\ \ \ \ \ \ \ \ \ \ \leq 10^3 \sqrt{d} \left( \left( \sum \EE(\nu_i)(\RR^d)^2 \right)^{1/2} + r_{m+1}^{d/2+1} \| \nabla \mu_T \|_{L^\infty(\RR^d)} \right).
    \end{split}
    \end{equation}
    %
    Equation \eqref{equation19204910490190190} shows
    %
    \begin{equation} \label{equation129401924901290412}
    \begin{split}
        \EE(\nu_i)(\RR^d) &= l_{m+1}^d \sum_{j \in [K_{m+1}]^d} \int \psi_{m+1}(x - a_{ij}) d\mu_T(x)\\
        &\leq r_{m+1}^d \| \mu_T \|_{L^\infty(\RR^d)}.
    \end{split}
    \end{equation}
    %
    Combining \eqref{equation190219024901290129041} and \eqref{equation129401924901290412} gives
    %
    \begin{equation} \label{equation10491204901290}
        \psitwo{\widehat{\nu}(m) - \widehat{\mu_T}(m)} \leq 10^3 \sqrt{d} \left[ \| \mu_T \|_{L^\infty(\RR^d)} + \| \nabla \mu_T \|_{L^\infty(\RR^d)} \right] r_{m+1}^{d/2}.
    \end{equation}
    %
    We can then apply a union bound over $D = \{ m \in \ZZ^d : |m| \leq 10 l_{m+1}^{-1} \}$, which has cardinality at most $10^{d+1} l_{m+1}^{-d}$, together with \eqref{equation10491204901290} to conclude that
    %
    \begin{equation} \label{equation1290421941021290}
    \begin{split}
        \PP& \left(\| \widehat{\nu} - \widehat{\mu_T} \|_{L^\infty(D)} \geq r_{m+1}^{d/2} \log(M_{m+1}) \right)\\
        &\ \ \ \ \ \leq 10^{d+2} \cdot l_{m+1}^{-d} \exp \left( - \frac{\log(M_{m+1})^2}{10^7 d} \right)\\
        &\ \ \ \ \ = 10^{d+2} l_m^{-d} \exp \left( d \log(M_{m+1} K_{m+1}) - \frac{\log(M_{m+1})^2}{10^7 d} \right).
    \end{split}
    \end{equation}
    %
    Combined with \eqref{equation109519029012} and \eqref{equation1940129041}, \eqref{equation1290421941021290} implies
    %
    \begin{equation} \label{equation0148912489128}
        \PP \left(\| \widehat{\nu} - \widehat{\mu_T} \|_{L^\infty(D)} \geq r_{m+1}^{d/2} \log(M_{m+1}) \right) \leq 1/10.
    \end{equation}
    %
    Thus $\widehat{\nu}$ and $\widehat{\mu_T}$ are highly likely to differ only by a neglible amount over small frequencies.
\end{proof}

\begin{lemma} \label{largeFrequencyLemma}
    If
    %
    \begin{equation} \label{equation129041902412901290129}
        M_{m+1}^d K_{m+1}^d \geq A(d + 1 + s/2) 2^{1 + d + s/2} l_m^d,
    \end{equation}
    %
    \begin{equation} \label{equation190512919204912901}
    \begin{split}
        K_{m+1}^d M_{m+1}^d \geq \frac{l_m^d 8^d A(d + 1 + s/2)}{1 + s/2},
    \end{split}
    \end{equation}
    %
    and
    %
    \begin{equation} \label{equation68943893493849}
        K_{m+1}^d M_{m+1}^d \geq l_m^d 2^{3d + s/2 + 1} B(d + s/2 + 1),
    \end{equation}
    %
    then almost surely, if $|m| \geq 10l_{m+1}^{-1}$,
    %
    \[ |\widehat{\nu_T}(m)| \leq \frac{1}{2|\eta|^{s/2}}. \]
\end{lemma}
\begin{proof}
    Define a random measure
    %
    \[ \alpha = r_{m+1}^d \sum\nolimits_{\substack{i \in \Pi_{m+1}^d\\d(a_i,T) \leq 2 r_{m+1}^{-1}}} \delta_{a_{ij_i}}. \]
    %
    Then $\nu_S = (\alpha * \psi_{m+1}) \mu_T$.
    %Since $\mu_T$ is supported on $T$, we can really truncate the sum defining $\eta$ to $O_d(r_{m+1}^{-d})$ terms such that the convolution identity remains valid.
    Thus we have $\widehat{\nu_S} = (\widehat{\alpha} \cdot \widehat{\psi_{m+1}}) * \widehat{\mu_T}$. Since $\mu_T$ is compactly supported, we can define, for each $t > 0$,
    %
    \[ A(t) = \sup |\widehat{\mu_T}(\xi)| |\xi|^t < \infty. \]
    %
    In light of \eqref{equation12901902419209012}, if we define, for each $t > 0$,
    %
    \[ B(t) = \sup |\widehat{\psi}(\xi)| |\xi|^t, \]
    %
    then
    %
    \[ \sup |\widehat{\psi_{m+1}}(\xi)| |\xi|^t = l_{m+1}^{-t} B(t). \]
    %
    The measure $\alpha$ is the sum of at most $2^d r_{m+1}^{-d}$ delta functions, scaled by $r_{m+1}^d$, so $\| \widehat{\alpha} \|_{L^\infty(\RR^d)} \leq \alpha(\RR^d) \leq 2^d$. Thus
    %
    \begin{equation} \label{equation9942941924912912}
        |\widehat{\nu_S}(\eta)| \leq 2^d \int |\widehat{\mu_T}(\eta - \xi)| |\widehat{\psi_{m+1}}(\xi)|\; d\xi.
    \end{equation}
    %
    If $|\xi| \leq |\eta|/2$, $|\eta - \xi| \geq |\eta|/2$, so for all $t > 0$, and since \eqref{equation19204910490190190} implies $\| \widehat{\psi_{m+1}} \|_{L^\infty(\RR^d)} \leq 1$, we find
    %
    \begin{equation} \label{equation999941204192049012}
        \int_{0 \leq |\xi| \leq |\eta|/2} |\widehat{\mu_T}(\eta - \xi)| |\widehat{\psi_{m+1}}(\xi)|\; d\xi \leq \frac{A(t) 2^{t-d}}{|\eta|^{t-d}}.
    \end{equation}
    %
    Set $t = d + 1 + s/2$. Equation \eqref{equation999941204192049012}, together with \eqref{equation129041902412901290129}, implies
    %
    \begin{equation} \label{equation1111902491209012}
    \begin{split}
        \int_{0 \leq |\xi| \leq |\eta|/2} |\widehat{\mu_T}(\eta - \xi)| |\widehat{\psi_{m+1}}(\xi)|\; d\xi &\leq \frac{A(d + 1 + s/2) 2^{1 + s/2} |\eta|^{-1}}{|\eta|^{s/2}}\\
        &\leq \frac{A(d + 1 + s/2) 2^{1 + s/2} l_{m+1}}{|\eta|^{s/2}}\\
        &\leq \frac{1}{10 \cdot 2^d} \frac{1}{|\eta|^{s/2}}.
    \end{split}
    \end{equation}
    %
    Conversely, if $|\xi| \geq 2|\eta|$, then $|\eta - \xi| \geq |\xi|/2$, so for each $t > d$,
    %
    \begin{equation} \label{equation5551092491022109}
    \begin{split}
        \int_{|\xi| \geq 2|\eta|} |\widehat{\mu_T}(\eta - \xi)| |\widehat{\psi}_{m+1}(\xi)|\; d\xi &\leq \int_{|\xi| \geq 2|\eta|} \frac{A(t) 2^t}{|\xi|^t}\\
        &\leq 2^d \int_{2|\eta|}^\infty r^{d-1 - t} A(t) 2^t\\
        &\leq \frac{4^d A(t)}{t - d} |\eta|^{d-t}.
    \end{split}
    \end{equation}
    %
    Set $t = d + 1 + s/2$. Equation \eqref{equation190512919204912901}, applied to \eqref{equation5551092491022109}, allows us to conclude
    %
    \begin{equation} \label{equation123123123}
        \int_{|\xi| \geq 2|\eta|} |\widehat{\mu_T}(\eta - \xi)| |\widehat{\psi}_{m+1}(\xi)|\; d\xi \leq \frac{1}{10 \cdot 2^d} \frac{1}{|\eta|^{s/2}}.
    \end{equation}
    %
    Finally, if $t > 0$, we use the fact that $\| \widehat{\mu_T} \|_{L^\infty(\RR^d)} \leq 1$ to conclude that
    %
    \begin{equation} \label{equation59010491092}
    \begin{split}
        \int_{|\eta|/2 \leq |\xi| \leq 2|\eta|} |\widehat{\mu_T}(\eta - \xi)| |\widehat{\psi_{m+1}}(\xi)|\; d\xi &\leq \frac{2^{d+t} B(t)}{|\eta|^{t-d}}.
    \end{split}
    \end{equation}
    %
    Set $t = d + s/2 + 1$. Then \eqref{equation59010491092} and \eqref{equation68943893493849} imply
    %
    \begin{equation} \label{equation2194129}
        \int_{|\eta|/2 \leq |\xi| \leq 2|\eta} |\widehat{\mu_T}(\eta - \xi)| |\widehat{\psi_{m+1}}(\xi)|\; d\xi \leq \frac{1}{10 \cdot 2^d} \frac{1}{|\xi|^{s/2}}.
    \end{equation}
    %
    Summing up \eqref{equation1111902491209012}, \eqref{equation123123123}, and \eqref{equation2194129}, we conclude from \eqref{equation9942941924912912} that if $|\eta| \geq 10 l_{m+1}^{-1}$, then
    %
    \begin{equation} \label{equation12904120941290419029012}
        |\widehat{\nu_S}(\eta)| \leq \frac{1}{2|\eta|^{s/2}}. \qedhere
    \end{equation}
\end{proof}

\begin{proof}[Proof of Lemma \ref{discreteLemma}, Continued]
    Let us now put all our calculations together. In light of Lemma \ref{propertyALemma} and Lemma \ref{deviationLemma}, there exists some choice of $j_i$ for each $i$, and a resultant non-random pair $(\nu_S, S)$ such that $S$ satisfies Property (A) of the Lemma, and for any $m \in \ZZ^d$ with $|m| \leq 10 l_{m+1}^{-1}$,
    %
    \begin{equation} \label{equation129419024129}
        |\widehat{\nu_{S}}(m) - \widehat{\mu_T}(m)| \leq r_{m+1}^{d/2} \log(M_{m+1}).
    \end{equation}
    %
    Now
    %
    \begin{align*}
        r_{m+1}^{d/2} \log(M_{m+1}) &= \left( l_{m+1}^{a\varepsilon - \frac{dn-s}{2n}} \cdot r_{m+1}^{d/2} \log(M_{m+1}) \right) l_{m+1}^{\frac{dn - s}{2n} - a \varepsilon}.
    \end{align*}
    %
    Equation \eqref{equation5890129048128941891} implies
    %
    \begin{align*}
        &l_{m+1}^{a\varepsilon - \frac{dn-s}{2n}} \cdot r_{m+1}^{d/2} \log(M_{m+1})\\
        &\ \ \ \ \ = \frac{l_m^{a\varepsilon - \frac{dn-s}{2n} + d/2} \log(M_{m+1}) K_{m+1}^{\frac{dn-s}{2n} - a\varepsilon}}{M_{m+1}^{a\varepsilon + \frac{s}{2n}}}\\
        &\ \ \ \ \ \leq \left[ l_m^{a\varepsilon - \frac{dn-s}{2n} + d/2} 2^{\frac{dn-s}{2n} - a \varepsilon} \right] \log(M_{m+1}) M_{m+1}^{\left( \frac{s}{dn-s} + c\varepsilon \right)\left(\frac{dn-s}{2n} - a\varepsilon\right) - a\varepsilon - \frac{s}{2n}}.
    \end{align*}
    %
    Now
    %
    \begin{align*}
        \left( \frac{s}{dn - s} + c\varepsilon \right) \left( \frac{dn - s}{2n} - a\varepsilon \right) - a\varepsilon &\leq \left[ \frac{(dn-s)c}{2n} - \left( \frac{s}{dn-s}+1 \right)a \right] \varepsilon\\
        &= \left[ \frac{d(3 - na)}{(dn - s)} \right] \varepsilon\\
        &\leq -2 \varepsilon.
    \end{align*}
    %
    Thus, if we assume that
    %
    \begin{equation} \label{equation1290412904129049102}
        A(l_m) \leq X^\varepsilon / \log(X)
    \end{equation}
    %
    then we conclude
    %
    \begin{align*}
        r_{m+1}^{d/2} \log(M_{m+1}) &\leq \left[ l_m^{a\varepsilon - \frac{dn-s}{2n} + d/2} 2^{\frac{dn-s}{2n} - a \varepsilon} \log(M_{m+1}) M_{m+1}^{-\varepsilon} \right] M_{m+1}^{-\varepsilon} l_{m+1}^{\frac{dn-s}{2n} - a\varepsilon}\\
        &\leq M_{m+1}^{-\varepsilon} l_{m+1}^{\frac{dn-s}{2n} - a \varepsilon}
    \end{align*}
    %
    Thus we conclude that if $|m| \leq 10 l_{m+1}^{-1}$,
    %
    \begin{align*}
        |\widehat{\nu_{S}}(m)| &\leq |\widehat{\nu_{S}}(m) - \widehat{\mu_T}(m)| + |\widehat{\mu_T}(m)|\\
        &\leq r_{m+1}^{d/2} \log(M_{m+1}) + A |m|^{c\varepsilon - \frac{dn-s}{2n}}\\
        &\leq l_{m+1}^{-\frac{dn-s}{2n}-c\varepsilon + \varepsilon} + A|m|^{c\varepsilon - \frac{dn-s}{2n}}\\
        &\leq [A + 10^d M_{m+1}^{-\varepsilon}] |m|^{c\varepsilon - \frac{dn-s}{2n}}.
    \end{align*}
    %
    Since $|\widehat{\nu_{S}}(m)| \leq |m|^{-\frac{dn-s}{2n}} \leq |m|^{c\varepsilon - \frac{dn-s}{2n}}$ holds automatically for $|m| \geq 10l_{m+1}^{-1}$, we conclude that for all $m \in \ZZ^d$,
    %
    \[ |\widehat{\nu_S(m)}| \leq (A + 10^d M_{m+1}^{-\varepsilon}) |m|^{c\varepsilon - \frac{dn-s}{2n}}. \]
    %
    Applying Lemma \ref{nuNormalizationLemma}, we conclude that for all $m \in \ZZ^d$,
    %
    \begin{align*}
        |\widehat{\mu_S(m)}| &\leq \frac{A + 10^d M_{m+1}^{-\varepsilon}}{1 - M_{m+1}^{-1/2}} |m|^{c\varepsilon - \frac{dn-s}{2n}}\\
        &\leq (1 + M_{m+1}^{-1/2}) [A + 10^d M_{m+1}^{-\varepsilon}] |m|^{c\varepsilon - \frac{dn-s}{2n}}. \qedhere
    \end{align*}
\end{proof}

\section{Construction of the Salem Set}

Let us now construct our configuration avoiding set. First, we fix some preliminary parameters. Write $Z \subset \bigcup_{i = 1}^\infty Z_i$, where $Z_i$ has lower Minkowski dimension at most $s$ for each $i$. Then choose an infinite sequence $\{ i_m : m \geq 1 \}$ which repeats every positive integer infinitely many times. Also, choose an arbitrary, decreasing sequence $\{ \varepsilon_m : m \geq 1 \}$, with $\varepsilon_m < (dn - s)/2$ for each $m$.

We choose our parameters $\{ M_m \}$ and $\{ K_k \}$ inductively. First, set $X_0 = [0,1]^d$, and $\mu_0$ an arbitrary smooth probability measure supported on $X_0$. At the $m$'th step of our construction, we have found a set $X_{m-1}$ and a measure $\mu_{m-1}$. We then choose $K_m$ and $M_m$ such that
%
\[ K_m, M_m \geq C(\mu_{m-1}, n, d, s, \varepsilon_m), \]
%
such that
%
\[ M_m^{\frac{s}{dn-s} + c\varepsilon_m} \leq K_m \leq 2 M_m^{\frac{s}{dn-s} + c\varepsilon}, \]
%
and such that the set $Z_{i_m}$ is covered by at most $l_m^{-(s + \varepsilon_m)}$ cubes in $\DQ_m$, the union of which, we define to be equal to $B_m$. We can then apply Lemma \ref{discreteLemma} with $\varepsilon = \varepsilon_m$, $T = X_{m-1}$, $\mu_T = \mu_{m-1}$, and $B = B_m$. This produces a $\DQ_{m+1}$ discretized set $S \subset T$, and a measure $\mu_S$ supported on $S$. We define $X_m = S$, and $\mu_m = \mu_S$.

The preceding paragraph recursively generates an infinite sequence $\{ X_m \}$. We set $X = \bigcap X_m$, and pick an arbitrary measure $\mu$, and some subsequence $\mu_{i_k}$, such that $\mu_{i_k} \to \mu$ weakly. It then follows from pointwise convergence of the Fourier transform that for each $m \in \ZZ^d$, and each $\varepsilon > 0$,
%
\[ \sup_{m \in \ZZ^d} |\widehat{\mu}(m)| |m|^{\frac{dn-s}{2n} - \varepsilon} \leq \limsup_{i \to \infty} \sup_{m \in \ZZ^d} |\widehat{\mu_i}(m)| |m|^{\frac{dn - s}{2n} - \varepsilon}. \]
%
Fix $\varepsilon > 0$. For each $m$, define
%
\[ A_{m,\varepsilon} = \sup |\widehat{\mu_M}(m)| |m|^{\frac{dn-s}{2n} - \varepsilon}. \]
%
Since each measure $\mu_M$ is smooth, all these quantities are finite. Since $\varepsilon_m \to 0$, there is $M$ such that if $m \geq M$, then $a \varepsilon_m \leq \varepsilon$. Property (B) of Lemma \eqref{discreteLemma} implies that for each $m \geq M$,
%
\[ A_{m+1,\varepsilon} \leq (1 + M_{m+1}^{-1/2})(A_{m,\varepsilon} + 10^d M_{m+1}^{-\varepsilon_{m+1}}). \]
%
If the sequence $\{ M_m \}$ increases rapidly enough, this recursive relationship guarantees that $\sup_{m \to \infty} A_{m,\varepsilon} < \infty$. Thus, for each $\varepsilon > 0$,
%
\[ |\widehat{\mu}(m)| \lesssim_\varepsilon |m|^{\varepsilon - \frac{dn-s}{2n}}.  \]

\end{document}