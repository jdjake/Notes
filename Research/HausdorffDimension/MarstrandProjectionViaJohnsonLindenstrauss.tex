\documentclass[dvipsnames,letterpaper,12pt]{article}

\usepackage[margin = 1.0in]{geometry}
\usepackage{amsmath,amssymb,graphicx,mathabx,accents}
\usepackage{enumerate,mdwlist}

\usepackage{tikz}

%\setlist[enumerate]{label*={\normalfont(\Alph*)},ref=(\Alph*)}

\numberwithin{equation}{section}

\usepackage{amsthm}

\usepackage{hyperref}

\usepackage{verbatim}

\usepackage{nag}

\DeclareMathOperator{\minkdim}{\dim_{\mathbb{M}}}
\DeclareMathOperator{\hausdim}{\dim_{\mathbb{H}}}
\DeclareMathOperator{\lowminkdim}{\underline{\dim}_{\mathbb{M}}}
\DeclareMathOperator{\upminkdim}{\overline{\dim}_{\mathbb{M}}}
\DeclareMathOperator{\fordim}{\dim_{\mathbb{F}}}

\DeclareMathOperator{\lhdim}{\underline{\dim}_{\mathbb{M}}}
\DeclareMathOperator{\lmbdim}{\underline{\dim}_{\mathbb{MB}}}

\DeclareMathOperator{\RR}{\mathbb{R}}
\DeclareMathOperator{\ZZ}{\mathbb{Z}}
\DeclareMathOperator{\QQ}{\mathbb{Q}}
\DeclareMathOperator{\TT}{\mathbb{T}}
\DeclareMathOperator{\CC}{\mathbb{C}}

\DeclareMathOperator{\B}{\mathcal{B}}

\newtheorem{theorem}{Theorem}
%\newtheorem{lemma}{Lemma}
%\newtheorem{corollary}{Corollary}
\newtheorem{lemma}[theorem]{Lemma}
\newtheorem{corollary}[theorem]{Corollary}
%\newtheorem{prop}[theorem]{Proposition}
\newtheorem{remark}[theorem]{Remark}
\newtheorem{remarks}[theorem]{Remarks}
%\newtheorem*{concludingremarks}{Concluding Remarks}
\numberwithin{theorem}{section}

\DeclareMathOperator{\EE}{\mathbb{E}}
\DeclareMathOperator{\PP}{\mathbb{P}}

\DeclareMathOperator{\DQ}{\mathcal{Q}}
\DeclareMathOperator{\DR}{\mathcal{R}}

\newcommand{\psitwo}[1]{\| {#1} \|_{\psi_2(L)}}
\newcommand{\TV}[2]{\| {#1} \|_{\text{TV}({#2})}}








\title{Marstrand Projection Theorem Via Marstrand Projection Theorem}
\author{Jacob Denson\footnote{University of Madison Wisconsin, Madison, WI, jcdenson@wisc.edu}}

\begin{document}

\maketitle

\begin{abstract}
    TODO
\end{abstract}

Recall the classic Marstrand Projection Theorem.

\begin{theorem}
    Suppose $E \subset \RR^n$ has Hausdorff dimension $s$. If $s < m$, then for almost every $\pi \in G(n,m)$, $\hausdim(E) = s$, and if $s \geq m$, $\hausdim(E) = m$.
\end{theorem}

The goal of this paper is to discuss the connection between Marstrand's projection theorem, and the following result from metric geometry.

\begin{theorem}
    Fix $0 < \delta < 1$, let $X$ be a set of $N$ points in $\RR^n$, and suppose $m > 8 \ln(N) / \delta^2$. Then with probability greater than or equal to $1 - 2 \exp(-c \delta^2 m)$, a random projection $\pi \in G(n,m)$ will satisfy
    %
    \[ (1 - \delta)(m/n)^{1/2} |x - y| \leq |\pi(x) - \pi(y)| \leq (1 + \delta) (m/n)^{1/2} |x - y|, \]
    %
    i.e. $(n/m)^{1/2} \pi$, restricted as a map from $X$ to $\RR^m$, will be an approximate isometry.
\end{theorem}

Let us recall some notation, introduced by Katz and Tao, and modified by Hera, Schmerkin, and Yavicoli. Fix some small quantity $\varepsilon_0 \ll 1$:
%
\begin{itemize}
    \item A \emph{hyper-dyadic} number will be a number of the form $2^{-\lfloor (1 + \varepsilon_0)^k \rfloor}$ for some $k \geq 0$. A \emph{hyper-dyadic cube} is a cube with hyper-dyadic sidelengths. We note that for any $N$, there are $O_{\varepsilon_0}(\log N)$ hyper-dyadic numbers between $\delta$ and $\delta^N$ for any $N > 0$, which is much less than the $O_{\varepsilon_0}(N \log(1/\delta))$ many dyadic numbers between $\delta$ and $\delta^N$, which depends on $\delta$.

    \item A family of sets $\{ X_\alpha \}$ \emph{strongly covers} a set $X$ if each point in $X$ is contained in infinitely many of the sets $\{ X_\alpha \}$.

    \item A set $E$ is \emph{$\delta$ discretized} if it is the union of $\delta$ balls.

    \item A set $E \subset \RR^n$ is a $(\delta,s)$ set if $E$ is a $\delta$ discretized subset of $B(0,2)$, and for all $\delta \leq r \leq 2$,
    %
    \[ |E \cap B(x,r)| \lesssim \delta^{n-\varepsilon} (r/\delta)^s. \]

    \item $|E| \gtrsim \delta^{n-s}$.
\end{itemize}

\begin{comment}
\begin{lemma}
    Suppose $E$ is a $(\delta,s)$ set. If $r \lessapprox \delta$ then $N(E,r)$ is a $(\delta,s)$ set.
\end{lemma}
\begin{proof}
    Suppose $E$ is a $(\delta,s)$ set. Then there exists a family of balls $\{ B_i \}$, where $B_i$ has center $x_i$ and radius $r_i \lessapprox \delta$, such that $E = \bigcup B_i$. Using the Vitali covering lemma, we may find a disjoint family of balls $B_{i_j}$ such that $E \subset \bigcup 5 B_{i_j}$.

    Certainly $N(E,r)$ is $\delta$ discretzied, since $N(E,r)$ is a union of balls of radius $r_i + \delta \approx \delta$. Thus all that remains is to argue that for $\delta \leq u \leq s$ and $a_0 \in \RR^d$,
    %
    \[ |N(E,r) \cap B(a_0,u)| \lessapprox \delta^n (\delta / u)^s. \]
    %
    If $a \in N(E,r) \cap B(a_0,u)$, then there exists $i_j$ such that $d(a,x_{i_j}) \leq r_{i_j} + r$ and $d(a,a_0) \leq u$. Thus $d(a_0, x_{i_j}) \leq r_{i_j} + r + u$, so $B_{i_j}$ is contained in the ball of radius $2r_{i_j} + r + u \lessapprox u + 3 \delta = v$ centered at $a_0$. But the sum of all such balls has total volume bounded by
    %
    \[ |X_\delta \cap B(a,v)| \lessapprox \delta^n (\delta / v)^s \]


    %
    \[ |N(E,r) \cap B(a_0,u)| \leq \sum_j (r_{i_j} + r + u)^d \]
    %
    where the sum is over all $j$ such that $B_{j_i}$ 

    But one can argue that $N(E,r) \cap B(x_0,r_1) \subset$


, then one can argue that $N(E,r) \cap B(x_0,r_1) \subset N(E \cap B(x_0,r + r_1), r)$.


    $x_0 \in \RR^d$, and $x \in N(E,\delta) \cap B(x_0,r)$, then there is $y \in N(E,\delta)$ such that $d(x,y) < \delta$, and so $d(x_0,y) < r + \delta$. Thus $N(E,\delta) \cap B(x_0,r) \subset N(E \cap B(x_0,r + \delta), \delta)$.
\end{proof}
\end{comment}

A result of Katz and Tao gives the following.

\begin{theorem}
    Suppose $0 < s < n$, and let $E$ be a compact subset of $\RR^n$. If $\hausdim(E) \leq s$, we can find a $(\delta,s)$ set $X_\delta$ for each hyperdyadic number $\delta$ such that $\{ X_\delta \}$ strongly covers $E$. Conversely, if $C > 0$ is sufficiently large, we can find a family $\{ X_\delta \}$, where $X_\delta$ is a $(\delta, s)$ set for each $\delta$, with implicit constants bounded uniformly in $\delta$, then $\hausdim(E) \leq s$.
\end{theorem}
\begin{proof}
    Suppose the latter constraint. Since $X_\delta$ is a $(\delta, s)$ set, it is $\delta$ discretized. It is therefore the union of a family of radius $\delta$ balls $\{ B_i \}$. Applying the Vitali covering lemma, we may find a disjoint subfamily of balls $S = \{ B_{j_i} \}$ such that $X_\delta \subset \bigcup 5 B_{j_i}$. Thus
    %
    \[ \#(S) \delta^n \lesssim |X_\delta| = |X_\delta \cap B(0,2)| \lesssim \delta^{n-s}, \]
    %
    so $\#(S) \lesssim \delta^{-s}$. But this means that $X_\delta$ is covered by $O(\delta^{-s})$ balls of radius $5 \delta$, so
    %
    \[ H^{s+\varepsilon}_{5\delta}(X_\delta) \lesssim \delta^{-s} (5\delta)^{s + \varepsilon} \lesssim \delta^\varepsilon. \]
    %
    Since $E$ is compact, and strongly covered by the sets $\{ X_\delta \}$, for any hyperdyadic $\delta_1 > 0$, there exists $\delta_2$ such that
    %
    \[ E \subset \bigcup_{\delta_2 \leq \delta \leq \delta_1} X_\delta. \]
    %
    But this means that
    %
    \[ H^{s+\varepsilon}_{5\delta_1}(E) \leq \sum_{\delta_2 \leq \delta \leq \delta_1} H^{s + \varepsilon}_{5 \delta_1}(X_\delta) \lesssim \sum_{\delta_2 \leq \delta \leq \delta_1} \]

    in particular, $\delta$ discretized, so is the union of a family of balls $\{ B_i \}$, where $B_i$ has radius $r_i \approx \delta$. Applying Vitali's covering lemma, we may find a disjoint subset $\{ B_{i_j} \}$ such that $X_\delta$ is covered by the family of balls $\{ 5 B_{i_j} \}$. If we let $X_\delta'$ denote the union of balls $\{ 5 B_{i_j} \}$, then $X_\delta'$ is still a $(\delta, s - C \varepsilon_0)$ set, since it is certainly $\delta$ discretized, and
    %
    \[ |X_\delta' \cap B(x,r)| \]
    % If x \in X_\delta' \cap B(x_0,r), then d(x_0,x) <= r and also there is a ball B_{i_j} with centre x_{i_j} such that d(x,x_{i_j}) <= 5 r_{i_j} << delta^{1-C_1 varepsilon}, so d(x_0, x_{i_j}) <= r + delta^{1 - C_1 varepsilon} lessapprox r, so X_delta' cap B(x,r) is a approx delta neighborhood of X_delta cap B(x_0,approx r)

%     is a distance at most r from the centre of a ball 5B_{i_j} centered

%    then there is a ball B_{i_j} such that x \in 5 B_{i_j}. But then the center of $B_{i_j} is a distance


    Thus
    %
    \[ |X_\delta| \gtrsim_d \sum r_{i_j}^d \]



    Suppose the latter constraint. Since $X_\delta$ is a $(\delta,s - C \varepsilon_0)$ set, for any $x \in \RR^d$,
    %
    \[ |E \cap B(x,1)| \lesssim_{x,\varepsilon_0} \delta^{n - s + (C - C_1) \varepsilon_0}. \]
    %
    Since $E$ is covered by $O_d(C_0^d)$ balls of radius one independantly, it follows that
    %
    \[ |E| \lesssim_{C_0,\varepsilon_0,d} \delta^{n - s + (C - C_1) \varepsilon_0} \]


    it satisfies the bound $|X_\delta| \lessapprox \delta^{n - s + C \varepsilon}$


    it is a union of balls $\{ B(x_i,r_i)$, where $r_i \approx \delta$. But then $N(X_\delta,\varepsilon/2)$


    Thus $r_i \lesssim_\varepsilon \delta^{-O(\varepsilon)} \delta$
\end{proof}

% TODO: ASSOUD DIMENSION?

\end{document}
