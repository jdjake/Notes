\documentclass[dvipsnames,letterpaper,12pt]{article}

\usepackage[margin = 1.0in]{geometry}
\usepackage{amsmath,amssymb,graphicx,mathabx,accents}
\usepackage{enumerate,mdwlist}

\usepackage{tikz}

%\setlist[enumerate]{label*={\normalfont(\Alph*)},ref=(\Alph*)}

\numberwithin{equation}{section}

\usepackage{amsthm}

\usepackage{hyperref}

\usepackage{verbatim}

\usepackage{nag}

\DeclareMathOperator{\minkdim}{\dim_{\mathbb{M}}}
\DeclareMathOperator{\hausdim}{\dim_{\mathbb{H}}}
\DeclareMathOperator{\lowminkdim}{\underline{\dim}_{\mathbb{M}}}
\DeclareMathOperator{\upminkdim}{\overline{\dim}_{\mathbb{M}}}
\DeclareMathOperator{\fordim}{\dim_{\mathbb{F}}}

\DeclareMathOperator{\lhdim}{\underline{\dim}_{\mathbb{M}}}
\DeclareMathOperator{\lmbdim}{\underline{\dim}_{\mathbb{MB}}}

\DeclareMathOperator{\RR}{\mathbb{R}}
\DeclareMathOperator{\ZZ}{\mathbb{Z}}
\DeclareMathOperator{\QQ}{\mathbb{Q}}
\DeclareMathOperator{\TT}{\mathbb{T}}
\DeclareMathOperator{\CC}{\mathbb{C}}

\DeclareMathOperator{\B}{\mathcal{B}}

\newtheorem{theorem}{Theorem}
%\newtheorem{lemma}{Lemma}
%\newtheorem{corollary}{Corollary}
\newtheorem{lemma}[theorem]{Lemma}
\newtheorem{corollary}[theorem]{Corollary}
%\newtheorem{prop}[theorem]{Proposition}
\newtheorem{remark}[theorem]{Remark}
\newtheorem{remarks}[theorem]{Remarks}
%\newtheorem*{concludingremarks}{Concluding Remarks}
\numberwithin{theorem}{section}

\DeclareMathOperator{\EE}{\mathbb{E}}
\DeclareMathOperator{\PP}{\mathbb{P}}

\DeclareMathOperator{\DQ}{\mathcal{Q}}
\DeclareMathOperator{\DR}{\mathcal{R}}

\newcommand{\psitwo}[1]{\| {#1} \|_{\psi_2(L)}}
\newcommand{\TV}[2]{\| {#1} \|_{\text{TV}({#2})}}








\title{Large Salem Sets Avoiding Polynomial Patterns}
\author{Jacob Denson\footnote{University of Madison Wisconsin, Madison, WI, jcdenson@wisc.edu}}

\begin{document}

\maketitle

\begin{abstract}
    TODO
\end{abstract}

Adapting the discrete strategy of (TODO) to the continuous setting, and together with the translation dimension boosting argument of Schmerkin, we prove the existence of a Salem set $E \subset [0,1]$ such that for $x_1,x_2,x_3,x_4 \in E$ with $x_1 \neq x_2$ and $x_3 \neq x_4$, $x_1 - x_2 \neq (x_3 - x_4)^2$.

We construct $E$ as follows. Fix an integer $k \geq 20$, and consider a family of subsets $R_n \subset \{ 0, \dots, k-1 \}$ for each $n \geq 0$. Define
%
\[ E = \left\{ \sum_{n = 1}^\infty a_n k^{-n} : a_n \in R_n\ \text{for all $n \geq 1$} \right\}. \]
%
We claim that $E$ avoids patterns if $\{ R_n \}$ are chosen suitably well. Let us begin by making an apriori assumption that for each $n$, $1 \not \in R_n - R_n$. Let us suppose that there exists $x_1,x_2,x_3,x_4 \in E$ such that $x_1 - x_2 = (x_3 - x_4)^2$. Write
%
\[ x_1 = \sum_{n = 1}^\infty a_n k^{-n}, \quad x_2 = \sum_{n = 1}^\infty b_n k^{-n}, \quad x_3 = \sum_{n = 1}^\infty c_n k^{-n},\quad\text{and}\quad x_4 = \sum_{n = 1}^\infty d_n k^{-n}. \]
%
Let $\delta_n = a_n - b_n$, and $\varepsilon_n = c_n - d_n$. Then
%
\[ \sum_{n = 1}^\infty \delta_n k^{-n} = \left( \sum_{n = 1}^\infty \varepsilon_n k^{-n} \right)^2. \]
%
Let $i$ be the first index such that $\delta_i \neq 0$, and $j$ the first index where $\varepsilon_j \neq 0$. Then
%
\[ k^{-i} < \sum_{n = 1}^\infty \delta_n k^{-n} < k^{1-i} \quad\text{and}\quad k^{-2j} < \left( \sum_{n = 1}^\infty \varepsilon_n k^{-n} \right)^2 < k^{2-2j}. \]
%
% i = 2j + 1 case:
%      We have ( A k^{-i} + B k^{-i-1} + O(k^{-i-2}) )
%           = ( C k^{-j} + D k^{-j-1} + O(k^{-j-2}) )^2
%      So Ak^{-i} + B k^{-i-1} + O(k^{-i-2})
%           = C^2 k^{-i+1} + 2CD k^{-i} + D^2 k^{-i-1} + O(k^{-i-1))
%      As long as |(A - 2CD) - k C^2| >= 3 we should be good.
%
% i = 2j case:
%       We have
%           Ak^{-i} + Bk^{-i-1} + O(k^{-i-2})
%           = C^2 k^{-i} + 2CD k^{-i-1} + O(k^{-i-2})
%       As long as |k(A - C^2) + (B - 2CD)| >= 3 we should be good.
Equality is thus only possible if $k^{-2j} < k^{1-i}$ (so $i < 2j + 1$) and $k^{-i} < k^{2-2j}$ (so $i > 2j - 2$). Thus $i = 2j - 1$ or $i = 2j$.

Assume first that $i$ is even, so that $2j = i$. Write $(U,V,W) = (\delta_i, \delta_{i+1}, \delta_{i+2})$ and $(A,B,C) = (\varepsilon_j, \varepsilon_{j+1}, \varepsilon_{j+2})$. Then
%
\[ \left| \sum_{n = 1}^\infty \delta_n k^{-n} - (U k^{-i} + V k^{-i-1} + W k^{-i-2}) \right| < k^{-i-2} \]
%
and
%
\[ \left| \left( \sum_{n = 1}^\infty \varepsilon_n k^{-n} \right)^2 - \Big( A k^{-j} + B k^{-j - 1} + C k^{-j-2} \Big)^2 \right| < \Big( 2 A + 2 B k^{-1} + (2 C + 1) k^{-2} \Big) k^{-i-2}, \]
% A^2 - B^2 = (A - B) (A + B)
and so
%
\[ \left| \Big( U + V k^{-1} + W k^{-2} \Big) - \Big( A + B k^{-1} + C k^{-2} \Big)^2 \right| < \Big( 1 + 2 A + 2B k^{-1} + (2 C + 1) k^{-2} \Big) k^{-2} \]
%
We have
%
\[ \left| \Big( U + V k^{-1} + W k^{-2} \Big) - \Big( A + B k^{-1} + C k^{-2} \Big)^2 \right| > (A-1)^2 - k, \]
% 1.0054        1.22660
and so we obtain that $(A - 1)^2 < 1.006k$, and thus $A < 1.23 k^{1/2}$, which means
%
\[ \left| \Big( U + V k^{-1} + W k^{-2} \Big) - \Big( A + B k^{-1} + C k^{-2} \Big)^2 \right| < 3.16k^{-3/2}. \]
% CAN BE MADE AS CLOSE TO 4 FOR LARGE k
Now
%
\begin{align*}
    &\Bigg| \Big[ \Big( U + V k^{-1} + W k^{-2} \Big) - \Big( A + B k^{-1} + C k^{-2} \Big)^2 \Big]\\
    &\quad\quad\quad - \Big[ (U-A^2 - 2AB/k) + (V/k - B^2/k^2 - 2AC/k^2) \Big] \Bigg|\\
    &\quad\quad\quad\quad\quad\quad\quad\quad< 3.05 k^{-1}
\end{align*}
% ARBITRARILY CLOSE TO 3
and so
%
\[ |(U-A^2 - 2AB/k) + (V/k - B^2/k^2 - 2AC/k^2)| < 3.76k^{-1} \]
% ARBITRARILY CLOSE TO 3, but not smaller 2.68 k^{-1/2}
This only occurs if $|U - A^2 - 2AB/k| \leq 2 + 3.31k^{-1/2}$.
% Condition on R_j, R_{j+1}, R_{j+2}, R_{2j}, R_{2j+1}.

How about if $i$ is odd? A similar reduction as above shows that
%
\begin{align*}
    &|(U + Vk^{-1} + Wk^{-2}) - k^{-1} (A + Bk^{-1} + Ck^{-2})^2|\\
    &\quad\quad\quad< (1 + 2Ak^{-1} + 2Bk^{-2} + (2C + 1)k^{-3}) k^{-2}\\
    &\quad\quad\quad < 3.1k^{-2}.
\end{align*}
% CAN BE MADE AS CLOSE TO 1 AS POSSIBLE
Thus
%
\[ |(U - A^2 k^{-1}) + (Vk^{-1} - 2ABk^{-2} + Wk^{-2} -2AC k^{-3} - B^2 k^{-3})| < 5.1 k^{-2}, \]
%
which implies $|U - A^2 k^{-1}| \leq 1 + 1.1 k^{-1/2}$. Thus we have reduced the problem to choosing $\{ R_j, R_{2j-1} \}$ appropriately.

\newpage

We find such choices computationally. First, let's suppose $\{ R_j \}$ is constant for all $j$, equal to some common set $R \subset \{ 0, \dots, k - 1 \}$. Our goal is thus to choose $R$ such that the difference set $R - R$ does not contain $X,Y,Z$, with $X,Y \neq 0$, such that
%
\[ |X - Y^2 - 2YZ/k| \leq 2 + 2.1k^{-1/2} \quad\text{or}\quad |X - Y^2 k^{-1}| \leq 1 + 1.1 k^{-1/2}\quad\text{or}\quad |X| \leq 1. \]
%
For any choice of $R$, the resulting set have covering number $|R|^n$ at a length scale $k^{-n}$, and one can show the resulting set is Ahlfors-regular, with dimension $\log_k|R|$.


\newpage


%
\[ |(a_i - b_i) k^{-i} - (c_j - d_j)^2 k^{-2j}| \leq k^{-i} + (2k-1) k^{-2j}. \]
%
If $2j \geq i$, then
%
\[ |(a_i - b_i) k^{2j - i} - (c_j - d_j)^2| \leq k^{2j-i} + (2k-1). \]
%
If $i \geq 2j$, then we have
%
\[ |(a_i - b_i) - (c_j - d_j)^2 k^{i-2j}| \leq 1 + (2k-1) k^{i-2j}. \]
%
If $i > 2j$, then $|(a_i - b_i) - (c_j - d_j)^2 k^{i-2j}|$


If $2j > i + 2$, then we conclude that
%
\[ |(a_i - b_i) k^{-i} - (c_j - d_j)^2 k^{-2j}| \leq (1 + 2/k - 1/k^2) k^{-i}. \]
%
Since $a_i - b_i$ and $(c_j - d_j)^2$ are both even

TODO: Argue that we reach a contradiction unless $2j = i$, and $(a_i - b_i) = (c_j - d_j)^2$.

Provided we choose $R_i$ so that $R_i - R_i$ is disjoint from $(R_j - R_j)^2$, except at the origin, we reach a contradiction, which allows us to conclude that the resulting set $E$ is squarefree. We have $N(k^{-n}) \sim \prod_{j \leq n} |R_j|$, and so the Minkowski dimension of $E$ is equal to
%
\[ \frac{1}{\log k} \lim_{n \to \infty} \frac{1}{n} \sum_{j \leq n} \log(|R_j|). \]
% Choose k/2 each time: Dimension 1 - log(2)
We can argue that $E$ has the same Hausdorff dimension similarily, i.e. by taking a limiting probality measure and proving a Frostman condition.


\end{document}
