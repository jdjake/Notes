\documentclass[dvipsnames,letterpaper,12pt]{article}

\usepackage[margin = 1.0in]{geometry}
\usepackage{amsmath,amssymb,graphicx,mathabx,accents}
\usepackage{enumerate,mdwlist}

\usepackage{tikz}

%\setlist[enumerate]{label*={\normalfont(\Alph*)},ref=(\Alph*)}

\numberwithin{equation}{section}

\usepackage{amsthm}

\usepackage{hyperref}

\usepackage{verbatim}

\usepackage{nag}

\DeclareMathOperator{\minkdim}{\dim_{\mathbb{M}}}
\DeclareMathOperator{\hausdim}{\dim_{\mathbb{H}}}
\DeclareMathOperator{\lowminkdim}{\underline{\dim}_{\mathbb{M}}}
\DeclareMathOperator{\upminkdim}{\overline{\dim}_{\mathbb{M}}}
\DeclareMathOperator{\fordim}{\dim_{\mathbb{F}}}

\DeclareMathOperator{\lhdim}{\underline{\dim}_{\mathbb{M}}}
\DeclareMathOperator{\lmbdim}{\underline{\dim}_{\mathbb{MB}}}

\DeclareMathOperator{\RR}{\mathbb{R}}
\DeclareMathOperator{\ZZ}{\mathbb{Z}}
\DeclareMathOperator{\QQ}{\mathbb{Q}}
\DeclareMathOperator{\TT}{\mathbb{T}}
\DeclareMathOperator{\CC}{\mathbb{C}}

\DeclareMathOperator{\B}{\mathcal{B}}

\newtheorem{theorem}{Theorem}
%\newtheorem{lemma}{Lemma}
%\newtheorem{corollary}{Corollary}
\newtheorem{lemma}[theorem]{Lemma}
\newtheorem{corollary}[theorem]{Corollary}
%\newtheorem{prop}[theorem]{Proposition}
\newtheorem{remark}[theorem]{Remark}
\newtheorem{remarks}[theorem]{Remarks}
%\newtheorem*{concludingremarks}{Concluding Remarks}
\numberwithin{theorem}{section}

\DeclareMathOperator{\EE}{\mathbb{E}}
\DeclareMathOperator{\PP}{\mathbb{P}}

\DeclareMathOperator{\DQ}{\mathcal{Q}}
\DeclareMathOperator{\DR}{\mathcal{R}}

\newcommand{\psitwo}[1]{\| {#1} \|_{\psi_2(L)}}
\newcommand{\TV}[2]{\| {#1} \|_{\text{TV}({#2})}}








\title{Large Salem Sets Avoiding Polynomial Patterns}
\author{Jacob Denson\footnote{University of Madison Wisconsin, Madison, WI, jcdenson@wisc.edu}}

\begin{document}

\maketitle

\begin{abstract}
    TODO
\end{abstract}

Adapting the discrete strategy of (TODO) to the continuous setting, and together with the translation dimension boosting argument of Schmerkin, we prove that there exists a Salem set $E \subset [0,1]$ such that for any distinct $x_1,x_2,x_3,x_4 \in E$, $x_1 - x_2 \neq (x_3 - x_4)^2$.

We construct $E$ as follows. Fix a squarefree integer $k$, and consider a family of subsets $R_n \subset \{ 0, \dots, k-1 \}$ for each $n \geq 0$. Define
%
\[ E = \left\{ \sum_{n = 1}^\infty a_n k^{-n} : a_{2n} \in R_n\ \text{for all $n \geq 1$} \right\}. \]
%
We claim that $E$ avoids patterns if $\{ R_n \}$ are chosen suitably well. Let us suppose that there exists $x_1,x_2,x_3,x_4 \in E$ such that $x_1 - x_2 = (x_3 - x_4)^2$. Write
%
\[ x_1 = \sum_{n = 1}^\infty a_n k^{-n}, \quad x_2 = \sum_{n = 1}^\infty b_n k^{-n}, \quad x_3 = \sum_{n = 1}^\infty c_n k^{-n},\quad\text{and}\quad x_4 = \sum_{n = 1}^\infty d_n k^{-n}. \]
%
Then
%
\[ \sum_{n = 1}^\infty (a_n - b_n) k^{-n} = \left( \sum_{n = 1}^\infty (c_n - d_n) k^{-n} \right)^2. \]
%
Let $i$ be the first index such that $a_i \neq b_i$, and let $j$ be the first index where $c_i \neq d_i$. Then
%
\[ \left| \sum (a_n - b_n) k^{-n} - (a_i - b_i) k^{-i} \right| \leq k^{-i}, \]
%\[ \left| \sum (a_n - b_n) k^{-n} - (a_i - b_i) k^{-i} \right| \leq \frac{k-3}{k-1} \cdot k^{-i}, \]
%
and
%
\[ \left| \left( \sum (c_n - d_n) k^{-n} \right)^2 - (c_j - d_j)^2 k^{-2j} \right| \leq (2k-1) k^{-2j}. \]
%\[ \left| \left( \sum (c_n - d_n) k^{-n} \right)^2 - (c_j - d_j)^2 k^{-2j} \right| \leq \frac{(k-3)^2 (2k-1)}{(k-1)^2} k^{-2j}. \]
%
Thus
%
\[ |(a_i - b_i) k^{-i} - (c_j - d_j)^2 k^{-2j}| \leq k^{-i} + (2k-1) k^{-2j}. \]
%
If $2j > i$, then $|(a_i - b_j) k^{2j-i} - (c_j - d_j)^2| \leq k^{2j-i} + (2k-1)$.


\end{document}
