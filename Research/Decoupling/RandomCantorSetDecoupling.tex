\documentclass[dvipsnames,letterpaper,12pt]{article}

\usepackage[margin = 1.0in]{geometry}
\usepackage{amsmath,amssymb,graphicx,mathabx,accents}
\usepackage{enumerate,mdwlist}

\usepackage{tikz}

%\setlist[enumerate]{label*={\normalfont(\Alph*)},ref=(\Alph*)}

\numberwithin{equation}{section}

\usepackage{amsthm}

\usepackage{hyperref}

\usepackage{verbatim}

\usepackage{nag}

\DeclareMathOperator{\minkdim}{\dim_{\mathbb{M}}}
\DeclareMathOperator{\hausdim}{\dim_{\mathbb{H}}}
\DeclareMathOperator{\lowminkdim}{\underline{\dim}_{\mathbb{M}}}
\DeclareMathOperator{\upminkdim}{\overline{\dim}_{\mathbb{M}}}
\DeclareMathOperator{\fordim}{\dim_{\mathbb{F}}}

\DeclareMathOperator{\lhdim}{\underline{\dim}_{\mathbb{M}}}
\DeclareMathOperator{\lmbdim}{\underline{\dim}_{\mathbb{MB}}}

\DeclareMathOperator{\RR}{\mathbb{R}}
\DeclareMathOperator{\ZZ}{\mathbb{Z}}
\DeclareMathOperator{\QQ}{\mathbb{Q}}
\DeclareMathOperator{\TT}{\mathbb{T}}
\DeclareMathOperator{\CC}{\mathbb{C}}

\DeclareMathOperator{\B}{\mathcal{B}}

\newtheorem{theorem}{Theorem}
%\newtheorem{lemma}{Lemma}
%\newtheorem{corollary}{Corollary}
\newtheorem{lemma}[theorem]{Lemma}
\newtheorem{corollary}[theorem]{Corollary}
%\newtheorem{prop}[theorem]{Proposition}
\newtheorem{remark}[theorem]{Remark}
\newtheorem{remarks}[theorem]{Remarks}
\newtheorem*{remarksaboutresults}{Remarks About The Results Stated}
%\newtheorem*{concludingremarks}{Concluding Remarks}
\numberwithin{theorem}{section}

\DeclareMathOperator{\EE}{\mathbb{E}}
\DeclareMathOperator{\PP}{\mathbb{P}}

\DeclareMathOperator{\DQ}{\mathcal{Q}}
\DeclareMathOperator{\DR}{\mathcal{R}}

\newcommand{\psitwo}[1]{\| {#1} \|_{\psi_2(L)}}
\newcommand{\TV}[2]{\| {#1} \|_{\text{TV}({#2})}}








\title{Random Cantor Set Decoupling}
\author{Jacob Denson\footnote{University of Madison Wisconsin, Madison, WI, jcdenson@wisc.edu}}

\begin{document}

\maketitle

Our goal is to get decoupling bounds for random Cantor sets of various different forms. Let's start by discussing estimates that already exist. First, what range should we expect to get? Fix a sequence of constants $\{ q_i \}$. We then define a Cantor set $C = \lim C_i$ of dimension $s$, where $\{ C_i \}$ is a decreasing family of sets, such that for each $i$, $C_i$ is a union of $N_i$ length $\delta_i = ( \prod_{j \leq i} q_j )^{-1}$ intervals $\{ I_j \}$. We assume that the dimension
%
\[ s = \lim_{i \to \infty} \frac{\log N_i}{\log(1/\delta_i)} \]
%
exists. This will be the lower Minkowski dimension of the set in general, and the Hausdorff dimension for most nicely chosen parameters, i.e. this is the case if $q_i = q$ is independent of $i$, and $N_i = q^{is}$. Let $D_p(C_i)$ be the optimal constant such that for any Schwartz function $f$,
%
\[ \| f \|_{L^p(\RR)} \leq D_p(C_i) \left( \sum_I \| P_I f \|_{L^p(\RR)}^2 \right)^{1/2}. \]
%
where $I$ ranges over all intervals of length $\delta_i$ defining $C_i$, and $P_I$ denotes the Fourier multiplier with symbol $\xi \mapsto \mathbf{I}(\xi \in J)$. In particular, we are concerned with finding the optimal constant $\kappa_p(C)$ such that
%
\[ D_p(C_i) \lesssim_\varepsilon \delta(i)^{- (\kappa(C) + \varepsilon)} \]
%
for all $\varepsilon > 0$.

What is the range of $\kappa(C)$ we can expect? We have the trivial estimates $D_2(C_i) = 1$ (by orthogonality) and $D_\infty(C_i) \leq \delta(i)^{-s/2}$ (by the triangle inequality), which can be interpolated, yielding that
%
\[ D_p(C_i) \leq \delta(i)^{-s(1/2-1/p)}. \]
%
Thus $\kappa(C) \leq s(1/2 - 1/p)$. On the other hand, consider any family $C$ of length $\delta = 1/N$ intervals $I_1,\dots,I_M$ with endpoints $\xi_1, \dots, \xi_M \in [0,1]$, we fix $\phi \in C_c^\infty(\RR)$ with Fourier support on $[0,1]$ and with $\text{Re}(\phi(x)) \geq 1/2$ on $[-1/10,1/10]$. If we consider the function
%
\[ f(x) = \phi(x/N) \left( \sum e^{2 \pi i \xi_i x / N} \right) \]
%
then we find that $P_{I_j} f = e^{2 \pi i \xi_i x / N} \phi(x/N)$, and so
%
\[ D_p(C) \gtrsim M^{-1/2} N^{-1/p} \| f \|_{L^p(\RR)}. \]
%
Because $|\xi_i| \in [0,1]$, $\text{Re}(e^{2 \pi i \xi_i x}) \geq 1/2$ for $|x| \leq 1/10$ for all $i$, and so we obtain that $\text{Re}(f) \geq M/2$ for $|x| \leq 1/10$, thus leading to a bound
%
\[ \| f \|_{L^p(\RR)} \gtrsim M. \]
%
But this means
%
\[ D_p(C) \gtrsim M^{1/2} N^{-1/p}. \]
%
If $M = N^s$, we get $D_p(C) \gtrsim N^{s/2 - 1/p} = \delta^{-s/2 + 1/p}$. Thus for Cantor sets $C$ of dimension $s$ constructed as above, we have $\kappa(C) \geq s/2 - 1/p$. This leads to a bound of the form
%
\[ s/2 - 1/p \leq \kappa_p(C) \leq s/2 - s/p \]
%
with length $-s/p + 1/p = (1 - s) / p$. This makes sense, since we should expect to get \emph{nothing} except for trivial decoupling when $s = 1$ (complete flatness), or when $p = \infty$ (no orthogonality). TODO: The upper bound is tight by the paper below. How tight is this lower bound?

In (Bourgain, 1989), for any $p > 2$, and $N > 0$, arbitrarily large sets $S \subset \{ 0, \dots, N-1 \}^d$ are constructed with $\#(S) \gtrsim N^{2/p}$ such that for any constants $\{ c_a \}$,
%
\[ \| \sum_{\xi \in S} c_a e^{2 \pi i \xi \cdot x} \|_{L^p(\TT^d)} \lesssim_p \left( \sum_{\xi \in S} |c_\xi|^2 \right)^{1/2}. \]
%
Such sets are called $\Lambda(p)$ sets. In (Laba, Wang, 2016), these sets are used to construct Cantor sets $C = \lim C_i$ in $\TT^d$ of dimension $s$ for any $0 < s < d$, for any $p > 2$, satisfying the inequality
%
\[ \left( \sum_I \| f \|_{L^p(w_I)}^p \right)^{1/p} \lesssim C(p)^i \left( \sum_J \| P_J f \|_{L^p(w_J)}^2 \right)^{1/2} \]
%
where $I$ ranges over a tiling of $O(\delta_i^{-1})$ sidelength $1$ cubes covering $U = [0,\delta_i^{-1}]$ by cubes with sidelength $1$, and $J$ ranges over the intervals in $C_i$. We thus get that
%
\[ \| f \|_{L^p(w_U)} \lesssim C(p)^i \left( \sum_J \| P_J f \|_{L^p(w_J)}^2 \right)^{1/2}. \]
%
If $N(i) = q^i$, this leads to a bound $D_p(C_i) \lesssim_p \delta_i^{-c_{p,q}}$ for some $c_{p,q} > 0$, i.e. $\kappa(C) \lesssim_{p,q} 1$. But in that paper they take $\{ q_i \}$ monotone increasing, so this bound does not tell us anything about $\kappa(C)$ in this situation.
%In (Laba, Wang, 2016), they assume that $\{ q_i \}$ is monotone increasing, but that $q_{i+1} \lesssim_\varepsilon q_1 \dots q_i$ for all $\varepsilon > 0$ (so the dimension is well behaved). The fact the sequence is monotone increasing implies that $i / \log(1/\delta_i) \to 0$ as $i \to \infty$, and thus the decoupling constant $D_p(C_i) = C^i$ is $O_\varepsilon(\delta_i^{-\varepsilon})$ for every $\varepsilon > 0$, i.e. for this example $\kappa(C) = 0$.

In (Chang, Pont, Greenfeld, Jamneshan, Li, Madrid, 2021), decoupling for a Cantor set on the line is studied. They study Cantor sets $C^{q,d} = \lim C_i^{q,d}$ parameterized by an integer $q > 1$, an integer $0 < k < q$, and digits $0 \leq d_1 < \dots < d_k < q$. In this case, $C^{q,d}_i$ is a union of $N(i) = k^i$ length $1/q^i$ intervals, and at each step of the construction, each length $\delta_i = 1/q^i$ interval $I$ is divided into $q$ length $1/q^{i+1}$ intervals $J_0(I),\dots,J_{q-1}(I)$, and we define
%
\[ C^{q,d}_{i+1} = \bigcup_I \bigcup_{l = 1}^k J_l(I). \]
%
Let us denote by $D(C_i) = D_p(q,d,i)$ the decoupling constant at the scale $i$, i.e. the best constant such that for any Schwartz function $f$,
%
\[ \| f \|_{L^p(\RR)} \leq D_p(q,d,i) \left( \sum_I \| P_I f \|_{L^p(\RR)} \right)^{1/2}, \]
%
where $P_I$ is the projection operator onto the interval $I$ in frequency space:
%
\begin{itemize}
    \item For the classical Cantor set, it is obtained that
    %
    \[ D_4(3,\{ 0, 2 \}, i) \lesssim (2^i)^{0.25 \log_2(1.5)} = \delta_i^{-(1/4) \log_q(3/2)}. \]

    \item For $q > 2$,
    %
    \[ D_4(q,\{ 0, 1 \}, i) \lesssim (2^i)^{0.25 \log_2(1.5)} = \delta_i^{-(1/4) \log_q(3/2)}. \]
    % delta_i = 1/q^i = q^{-i} = (2^i)^{- log_2(q)}

    \item For $q > 4$,
    %
    \[ D_4(q, \{ 0, 1, 2 \}, i) \lesssim (3^i)^{(1/4) \log_3(5/17)} = \delta_i^{-(1/4) \log_q(5/17)}. \]

    \item For $q > 6$,
    %
    \[ D_4(q, \{ 0, 1, 3 \}, i) \lesssim (3^i)^{(1/4) \log_3(5/3)} = \delta_i^{-(1/4) \log_q(5/3)}. \]

    \item If $d$ is the vector of all squares of integers at most $q$, where $q \geq e^{e^{O(1/\varepsilon)}}$, then
    % q^{i/2}
    \[ D_4(q,d,i) \lesssim_\varepsilon N(i)^\varepsilon \approx \delta_i^{-\varepsilon / 2}. \]
\end{itemize}
%
How bad can the decoupling constant get? They prove (in Proposition 3.6 of their paper) that for any Hausdorff dimension $s$, and any $n > 0$, there exists a Cantor set constructed as above with dimension $s$, such that the smallest constant $\kappa_{2n}$ such that
%
\[ D_{2n}(C_i) \lesssim_\varepsilon \delta(i)^{- \kappa_{2n} + \varepsilon} \]
%
for all $\varepsilon > 0$ is at least $s(1/2 - 1/2n)$. This is the \emph{worst possible exponent}, 

\section{Toy Problem: Gaussians Supported on Fractal Intervals}

Let's consider a toy problem, which is easier than the Cantor set by virtue of the fact that it has less arithmetic structure. Fix an exponent $p$, fix a large integer $N$, a quantity $0 < s < 1$, and then set $M$ to be the closest integer to $N^s$. Choose $M$ points $\xi_1,\dots,\xi_M$ on $\TT$, uniformly at random. For $1 \leq k \leq M$, let
%
\[ f_k(x) = e^{2 \pi i \xi_k \cdot x} \phi(x/N), \]
%
where $\phi \in \mathcal{S}(\RR)$ is a function whose Fourier transform is supported on the interval $[-1,1]$. Then $f_k$, roughly speaking, has phase space support on the set
%
\[ \{ (x,\xi) : |x| \leq N\ \text{and}\ |\xi - \xi_k| \leq 1/N \}. \]
%
If $s < 1/2$, the intervals $I_j$ are disjoint from one another with high probability.

\begin{lemma}
    If $s < 1/2$, then for any $\varepsilon > 0$, if $N$ is sufficiently large, the intervals $I_1,\dots,I_M$ will be disjoint from one another with probability at least $1 - O(N^{2s - 1})$. We can also (TODO: Prove this) get this property with $90\%$ probability for $s = 1/2$ if $M = N^{1/2} / 100$.
\end{lemma}
\begin{proof}
    Let $P_l$ denote the probability that $I_1,\dots,I_l$ are $1/N$ separated from one another, i.e.
    %
    \[ d(I_j, I_k) \geq 1/N \]
    %
    for $j \neq k$. Then $P_1 = 1$, and we can obtain an inductive lower bound for the other $l$. Namely, if $I_1,\dots,I_l$ are $1/N$ separated from one another, then $I_1,\dots,I_{l+1}$ will be $1/N$ separated from one another if $\xi_{l+1}$ lies away from a $3/N$ neighborhood of $\xi_1,\dots,\xi_l$, which is a set of measure at least $1 - (3/N) l$. Thus we find that
    %
    \[ P_{l+1} \geq P_l (1 - 3l/N). \]
    %
    Thus we find that
    %
    \[ P_l \geq \prod_{j = 1}^{l-1} (1 - 3l/N) = (-3/N)^{l-1} \frac{\Gamma(l - N/3)}{\Gamma(1 - N/3)}. \]
    %
    Using asymptotics for the ratio of a Gamma function and their relation to Stirling series (Tracomi and Erdelyi, 1951, or Laforgia and Natalini, 2012), we find that
    %
    \[ P_l \geq 1 - \frac{3}{2} \frac{l(l-1)}{N} + O(1/N^2), \]
    %
    The asymptotic expansions in these papers don't necessarily hold uniformly as $l \to \infty$, but in the range we're consider, i.e. for $1 \leq l \leq N^s$, this probably isn't a problem (TODO). In particular, for $l = M$, we find that
    %
    \[ P_M \geq 1 - 1.5 M^2 / N + O(1/N^2) \geq 1 - 1.5 N^{2s - 1} + O(1/N^2) = 1 - O(N^{2s-1}). \qedhere \]
\end{proof}

It follows from Parseval's identity that in this situation, the functions $\{ f_i \}$ are orthogonal to one another. For $1/2 \leq s < 1$, we cannot guarantee this orthogonality since intervals will intersect, but we can expect \emph{almost orthogonality}.

\begin{lemma}
    For $0 < s < n$, and any $a \gtrsim 1$, with probability exceeding $1 - O_s(1/N^{a-1})$, all of the sets
    %
    \[ A_i = \{ j : |\xi_i - \xi_j| \leq 10/N \} \]
    %
    have cardinality $O_s(a)$.
\end{lemma}
\begin{proof}
    Let $I$ be an interval of length $L$. Then the random variable
    %
    \[ Z_I = \# \{ k: \xi_k \in I \} \]
    %
    is a $\text{Bin}(M,L)$ random variable. In particular, a Chernoff bound implies that for $t \geq ML$,
    %
    \[ \PP( Z_I \geq t ) \leq e^{-ML} \left( \frac{e \mu}{t} \right)^t. \]
    %
    Now let $I_1,\dots,I_N \subset \TT$ be the family of all sidelength $20/N$ intervals whose endpoints lie on integer multiples of $1/N$. The bound above implies that for any $1 \leq j \leq N$, and any $t \geq 3/N^{1-s}$,
    %
    \[ \PP( Z_{I_j} \geq t ) \leq e^{-3/N^{1-s}} \left( \frac{20e}{N^{1-s}} \right)^t \leq \left( \frac{20e}{N^{1-s} t} \right)^t. \]
    %
    In particular, for any $a \geq 20e$,
    %
    \begin{align*}
        \PP \left(Z_{I_j} \geq \frac{a}{1 - s} \right) &\leq \left( \frac{20e(1-s)}{a N^{1-s}} \right)^{\frac{a}{1-s}} \leq 1/N^a.
    \end{align*}
    %
    Taking a union bound, we conclude that
    %
    \[ \PP \left( \max_j Z_{I_j} \geq \frac{a}{1-s} \right) \leq 1/N^{a-1}. \]
    %
    But the fact that no interval $Z_{I_j}$ contains more than $O_s(1)$ points of $\{ \xi_1,\dots, \xi_M \}$ implies what was needed to be proved.
\end{proof}

It follows that with high probability, each of the functions $f_i$ is orthogonal to all but $O(1)$ of the functions $f_j$. We therefore find that
%
\[ \sup_i \sum_j |\langle f_i, f_j \rangle|^{1/2} \lesssim N^{1/2}. \]
%
Thus the Cotlar-Stein lemma implies that for any choice of constants $a_1,\dots,a_M$,
%
\[ \left\| \sum_i a_i f_i \right\|_{L^2(\RR^d)} \lesssim N^{1/2} |a| = \left( \sum_i \| a_i f_i \|_{L^2(\RR^d)}^2 \right)^{1/2}. \]
%
Our goal is to extend a result like this to the $L^p$ norm, i.e. to guarantee with high probability that
%
\[ \left\| \sum_{i = 1}^M a_i f_i \right\|_{L^p(\RR^d)} \lesssim \left( \sum_{i = 1}^M \| a_i f_i \|_{L^p(\RR^d)}^2 \right)^{1/2}. \]
%
Normalizing, it will suffice to prove that with high probability, for \emph{any} constants $a_1,\dots,a_M$ with $\sum |a_i|^2 = 1$,
%
\[ \left\| \sum_{k = 1}^M a_k f_k \right\|_{L^p(\RR^d)} \lesssim N^{1/p}. \]
%
We will obtain such a result using a \emph{Chaining argument}. 

\begin{comment}

\begin{lemma}
    With probability exceeding $1 - O( N^{2 - \log \log N} )$, all of the sets
    %
    \[ A_i = \{ j : |\xi_i - \xi_j| \leq 1/M \} \]
    %
    have cardinality $O( \log N )$.
\end{lemma}
\begin{proof}
    Let $I$ be an interval of length $L$. Then the random variable
    %
    \[ Z_I = \# \{ k: \xi_k \in I \} \]
    %
    is a $\text{Bin}(M,L)$ random variable. In particular, a Chernoff bound implies that for $t \geq ML$,
    %
    \[ \PP( Z_I \geq t ) \leq e^{-ML} \left( \frac{e \mu}{t} \right)^t. \]
    %
    Now let $I_1,\dots,I_{10M} \subset \TT$ be the family of all sidelength $1/M$ intervals whose endpoints lie on integer multiples of $1/M$. The bound above implies that for any $1 \leq j \leq N$, and any $t \geq 1$,
    %
    \[ \PP( Z_{I_j} \geq t ) \leq \left( \frac{e}{t} \right)^t. \]
    %
    Thus
    % (e/t)^t <= 1/N
    % t log(e/t) <= - log(N)
    % t log(t/e) >= log(N)
    % If t = log(N), this should give it
    \[ \PP( Z_{i_j} \geq \log N ) \leq \left( \frac{e}{\log N} \right)^{\log N} = N^{1 - \log \log N} \]
    %
    A union bound implies that
    %
    \[ \PP \left( \max_j Z_{i_j} \geq \log N \right) \lesssim N^{1 + s - \log \log N} \leq N^{2 - \log \log N}. \]
    %
    But if this is true, then the consequence of the lemma follows, because any length $1/M$ interval around any of the $\xi_i$ is contained in at most two of the intervals $Z_{i_j}$.
\end{proof}

This condition shows that with high probability, almost every pair of functions in the set $\{ f_k \}$ have Fourier transforms which are Gaussian separated far away from one another. We have
%
\begin{align*}
    |\langle f_i, f_j \rangle| &= \int \widehat{f}_i(\xi) \overline{\widehat{f}_j(\xi)}\; d\xi\\
    &= N^2 \int \phi(N(\xi - \xi_i)) \phi(N(\xi - \xi_j))\; d\xi\\
    &= N^2 \int e^{-2 \pi N^2 [|\xi - \xi_i|^2 + |\xi - \xi_j|^2}\; d\xi\\
    &= N \int e^{-2 \pi [ |\xi - N \xi_i|^2 + |\xi - N \xi_j|^2 ]}\; d\xi.
\end{align*}
%
Now
%
\begin{align*}
    \int e^{-2 \pi [ |\xi - N \xi_i|^2 + |\xi - N \xi_j|^2 ]}\; d\xi &\lesssim 1 + \int_{|\xi - N \xi_i|^2 + |\xi - N \xi_j|^2 \leq |\xi|^2/100} \{ \dots \}.
\end{align*}
%
The triangle inequality implies that for any such $\xi$ with $|\xi - N \xi_i|^2 + |\xi - N \xi_j|^2 \leq |\xi|/10$, we have
%
\[ N |\xi_i - \xi_j| \leq |\xi - N \xi_i| + |\xi - N \xi_j| \leq 2^{1/2} ( |\xi - N \xi_i|^2 + |\xi - N \xi_j|^2 )^{1/2} \leq |\xi| / 10. \]
%
Thus $|\xi| \geq 10 N |\xi_i - \xi_j|$. The region of integration here has length $O(N)$ since clearly it is contained in the union of the two sets
%
\[ \{ \xi : |\xi - N \xi_i| \leq |\xi| / 10 \} \cup \{ \xi : |\xi - N \xi_j | \leq |\xi| / 10 \}, \]
%
which are two length $O(N)$ intervals. But this means that
%
\[ \int e^{-2 \pi [ |\xi - N \xi_i|^2 + |\xi - N \xi_j|^2 ]}\; d\xi \lesssim 1 + N e^{- 100 N^2 |\xi_i - \xi_j|^2}, \]
%
and so
%
\[ |\langle f_i, f_j \rangle| \lesssim 1 + N e^{-100 N^2 |\xi_i - \xi_j|^2}. \]
%
The last lemma guarantees that with overwhelming probability, for any $i$, there are at most $\log N$ values $j$ such that $|\xi_i - \xi_j| \leq 1/M$, and here we obtain the trivial bound
%
\[ |\langle f_i, f_i \rangle| \lesssim N. \]
%
For any \emph{other} value $j$, we conclude that
%
\[ |\langle f_i, f_j \rangle| \lesssim 1 + N e^{-100 N^2 / M^2} = 1 + N e^{-100 N^{2(1 - s)}} \lesssim 1. \]
%
Thus we conclude that
%
\[ \sup_i \sum_j |\langle f_i, f_j \rangle|^{1/2} \lesssim \log N \cdot N^{1/2} + N^s \lesssim \max( N^{1/2} \log N, N^s). \]
%
Thus applying Cotlar-Stein allows us to conclude that for $s \leq 1/2$,
%
\[ \| \sum_i a_i f_i \|_{L^2(\RR^d)} \lesssim N^{1/2} (\log N)^{1/2} |a| = (\log N)^{1/2} \left( \sum |a_i|^2 \| f_i \|_{L^2(\RR^d)}^2 \right)^{1/2}, \]
%
and for $1/2 \leq s < 1$,
%
\[ \| \sum_i a_i f_i \|_{L^2(\RR^d)} \lesssim N^{s/2 + 1/4} |a| \lesssim N^{s/2 - 1/4} \left( \sum |a_i|^2 \| f_i \|_{L^2(\RR^d)}^2 \right)^{1/2}. \]

% If T_i a = f_i a_i
% Then int (f_i a_i) g = a_i int f_i g
% So T_i^* g = delta_i (int f_i g).

% T_i T_j^* = 0 unless i = j, if i = j, then T_i T_i^* g = <f_i, g> f_i
% So | T_i T_i^* g |_2 = |<f_i, g>| |f_i| <= |f_i|^2 |g|
% So sup_i sum_j sqrt(|T_iT_j^*|) = sup_i |f_i| = N^{1/2}

% T_i^* T_j a = delta_i a_j <f_i, f_j>
% Has operator norm at most <f_i,f_j>


% If T_i g = < f_i, g >
% Then T_i^*(a) = a f_i.
% Now T_i T_j^* has operator norm |<f_i, f_j>|
% So Cotlar-Stein wants us to bound sup_i sum_j sqrt{<f_i, f_j>}.


If $\phi(N(x - \xi_i))$
%
\[ \sup_i \sum_j |\langle f_i, f_j \rangle| \lesssim_s 1 \]

\end{comment}

For each point $a$ on the unit sphere, we write
%
\[ S(a,x) = \sum_{k = 1}^M a_k f_k(x) \]
%
and then setting
%
\[ Z(a) = \left\| \sum_{k = 1}^M a_k f_k \right\|_{L^p(\RR^d)}. \]
%
We now establish an upper bound on the average value of $Z(a)$.

\begin{theorem}
    We have
    %
    \[ \EE \left[ \sup_{|a| = 1} Z(a) \right] \lesssim (\log N)^{1/2} N^{s/2 + 1/p}. \]
\end{theorem}
\begin{proof}
    Hoeffding's inequality guarantees that for each $x \in \RR$,
    %
    \[ \| S(a,x) \|_{\psi_2} \lesssim |a| \phi(x/N). \]
    %
    A union bound guarantees that, for all $x$ in the integer lattice,
    %
    \[ \| \sup_{x \in \ZZ} S(a,x) \|_{\psi_2} \lesssim |a| (\log N)^{1/2} \phi(x/N)^{1/2}. \]
    %
    Applying the local constancy policy, this should show that
    %
    \[ \| Z(a) \|_{\psi_2} \lesssim |a| (\log N)^{1/2} N^{1/p}. \]
    %
    But now the triangle inequality implies that
    %
    \[ |Z(a) - Z(b)| = | \| S(a) \|_{L^p(\RR^d)} - \| S(b) \|_{L^p(\RR^d)} | \leq \| S(a-b) \|_{L^p(\RR^d)}. \]
    %
    Thus 
    %
    \[ \| Z(a) - Z(b) \|_{\psi_2} \leq \| Z(a-b) \|_{\psi_2} \lesssim |a - b| (\log N)^{1/2} N^{1/p}. \]
    %
    Thus Dudley's integral inequality implies that
    %
    \[ \EE [ \sup_{|a| = 1} Z(a) ] \lesssim (\log N)^{1/2} N^{1/p} \int_0^\infty (\log N(t))^{1/2}\; dt, \]
    %
    where $N(t)$ denotes the number of balls of radius $t$ required to cover the unit sphere in $\RR^M$. We have $N(t) \lesssim (1/t)^{M-1}$ for $t \lesssim 1$, and $N(t) = 1$ for $t \gtrsim 1$, which leads to
    %
    \[ \EE [ \sup_{|a| = 1} Z(a) ] \lesssim (\log N)^{1/2} N^{-1/p} M^{1/2} = (\log N)^{1/2} N^{s/2 + 1/p}. \]
    %
    Unfortunately, this only gives a bound on the decoupling constant of the form $(\log N)^{1/2} N^{s/2}$, which is not a great bound at all, i.e. since the triangle inequality trivially gives the decoupling constant $N^{s/2}$. What are we doing inefficiently with chaining, especially since \emph{generic chaining} should be optimal? Perhaps because the $L^\infty$ norm is sensitive to individual values of things, we should be able to improve bounds somehow when working with an $L^p$ norm - indeed, for any random choice of intervals, picking $a_i = 1/\sqrt{M}$ for all $i$ leads to $\sum a_i f_i$ being equal to $M^{s/2}$ at the origin so this will be tight for the $L^\infty$ norm, i.e. randomness doesn't help us at all?
\end{proof}

TODO: This is a local bound, and I think we can show this leads to a global bound, e.g. by Demeter's book. Unfortunately, we only have a good bound on the expected value of $\sup_{|a| = 1} Z(a)$, and not the tails, so iterating this using Markov's inequality to get a bound on decoupling on a random fractal doesn't yield great results. Indeed, suppose we iteratively construct a fractal at a scale $1/L^k$ consisting of $L^{ks}$ intervals. Then Markov's inequality guarantees that if $E$ is the random fractal constructed, then for sequences $\{ C_k \}$ with $\sum 1/C_k < \infty$, the Borel-Cantelli lemma implies that almost surely, if $\delta_k = 1/L^k$, the random set $E$ has a decoupling constant
%
\[ \text{Dec}(E(\delta_k),p) \leq C_k \log (1/\delta_k)^{1/2} (1/\delta_k)^{s/2 - 1/p} \]
%
for all $k$. For $s < 2/p$, we can select these constants well, leading to
%
\[ \text{Dec}(E(1/L^k),p) \lesssim 1. \]
%
In fact, this inequality gets \emph{better as a power in $\delta_k$} as $k \to \infty$. For $s = 2/p$, Markov's inequality only leads to bounds of the form
%
\[ \text{Dec}(E(1/L^k),p) \lesssim k (\log k)^2 \log(1/\delta_k)^{1/2} \lessapprox_{L,\varepsilon} \log(1/\delta_k)^{3/2}. \]
%
I'll have to (TODO) look into tail bounds on suprema of sub-Gaussian processes if we want to improve the implicit constants in $k$, i.e. if we want to replace $k$ with a power of $\log k$, so we get a decoupling constant $\widetilde{O}( \log(1/\delta_k)^{1/2})$.

\section{Toy Problem \# 2: Gaussians Supported on the Cantor Set}

Now we consider a different model of random Cantor sets which possesses more arithmetic structure, and thus makes the problem harder. TODO: For a normal $1/3$-Cantor set, I think the same $L^\infty$ analysis will work \emph{away from frequencies which are a power of 3}, but hopefully this is a small set so something trivial should work here.

Consider $0 < N < M$. Construct a random Cantor set $E = \lim E_k$, where $E_k \subset [0,1]$ is a union of length $1/N^k$ intervals, defined as follows. Consider a family of independent random variables
%
\[ \{ M_\alpha \} \]
%
where $\alpha$ ranges over all finite multi-indices over $\{ 1, \dots, M \}$, each a uniformly random integer selected from $\{ 0, \dots, N-1 \}$. We define $E_k$ as the union of the $M^k$ sidelength $1/N^k$ intervals $\{ I_{i_1,\dots,i_k} \}$, which has left endpoint
%
\[ \xi_{i_1, \dots, i_k} = \sum_{l = 1}^k \frac{M_{i_1, \dots, i_l}}{N^l} \]
%
Then almost surely (TODO), $C$ is a Salem set of dimension $s = \log N / \log M$. If $\phi$ is a standard Gaussian, then for $k > 0$, our goal is to understand the interactions of the functions
%
\[ f_{k,i}(x) = N^{-k/p} e^{2 \pi i \xi_i \cdot x} \phi(x/N^k), \]
%
where $i$ ranges over all $k$ multi-indices. Set
%
\[ S_k(a,x) = \sum_i a_i f_{k,i}. \]
%
for $x \in \RR^d$. In particular, we wish to understand the quantities
%
\[ Z_{k,p}(a) = \| S_k(a) \|_{L^p(\RR^d)}, \]
%
and more precisely, the quantity
%
\[ D_{k,p} = \sup_{|a| = 1} Z(a), \]
%
which represents the decoupling constant at the scale $k$.

Each $\xi_{i_1,\dots,i_k}$ is uniformly distributed on $[0,1]$, and so for any vector $a$,
%
\[ \EE[S_k(a,x)] = 0. \]
%
We would like to use Hoeffding's inequality to show $|S_k(a,x)| = |S_k(a,x) - \EE[S_k(a,x)]|$ is not too large with high probability. The problem is that $S_k(a,x)$ is not \emph{quite} a sum of independent random variables, though we can write the sum as
%
\[ S_k(a,x) = N^{-k/p} \left( \sum_{i_1} e^{2 \pi i (M_{i_1} / N) x} \sum_{i_2} e^{2 \pi i (M_{i_1,i_2} / N^2) x} \dots \sum_{i_k} e^{2 \pi i (M_{i_1,\dots,i_k} / N^k) x} a_i \right) \phi(x/N^k). \]
%
If we fix all of the values $\{ M_{i_1} \}, \dots, \{ M_{i_1,\dots,i_{k-1}} \}$, the sum
%
\[ S_{i_1,\dots,i_{k-1}} = \sum_{i_k} e^{2 \pi i (M_{i_1,\dots,i_k} / N^k) x} a_i \]
%
is a sum of independent random variables, and thus Hoeffding's inequality guarantees that
%
\[ \| S_{i_1,\dots,i_{k-1}} - \EE[S_{i_1,\dots,i_{k-1}}] \|_{\psi_2} \lesssim |a_{i_1,\dots,i_k}|. \]
%
Thus we can guarantee that with high probability, for all $i_1,\dots,i_{k-1}$
%
\[ |S_{i_1,\dots,i_{k-1}} - \EE[S_{i_1,\dots,i_{k-1}}]| \lesssim (\log M) |a_{i_1,\dots,i_k}|. \]
%
If we iterate this $k$ times, (TODO: Is this true?), we obtain that with high probability,
%
\[ |S_k(a,x)| \lesssim N^{-k/p} (\log M)^k |a| \phi(x/N^k). \]
%
This should imply that $\| Z_{k,p}(a) \|_{\psi_2} \lesssim (\log M)^k |a|$. Dudley's integral inequality then implies that
%
\[ \EE[D_{k,p}] \lesssim (\log M)^k M^{k/2}. \]
%
This is not a good Decoupling inequality, i.e. if $\delta = 1/N^k$, then this gives
%
\[ \EE[D_{k,p}] \lesssim \log(1/\delta)^{O(1)} \cdot (1/\delta)^{1/2s}. \]
%
However, since $D_{k,2} = 1$, can we interpolate this inequality to yield something that is better for $p$ closer to $2$? In other words, we should have that for $p \geq 2$,
%
\[ \EE[D_{k,p}] \leq \EE[D_{k,2}^{2/p}] \EE[D_{k,\infty}^{1-2/p}] \lesssim \log(1/\delta)^{O(1)} (1/\delta)^{(1/s)(1/2 - 1/p)}. \]
%
It still seems we need to use more in our problem though. This seems even worse than the trivial decoupling inequality, so I must be doing something wrong here. Is Dudley's inequality suboptimal in this scenario?

\begin{comment}
Now the $L^p$ norm of the function
%
\[ N^{-k/p} \left( \sum_{i_1} e^{2 \pi i (M_{i_1} / N) x} \sum_{i_2} e^{2 \pi i (M_{i_1,i_2} / N^2) x} \dots \EE \left( \sum_{i_k} e^{2 \pi i (M_{i_1,\dots,i_k} / N^k) x} a_i \right) \right) \phi(x/N^k) \]
%
is bounded by
%
\[ C \cdot N^{-k/p} D_{k-1} \left( \sum_{i_1,\dots,i_{k-1}} \left\| \EE \left( \sum_{i_k} e^{2 \pi i (M_{i_1,\dots,i_k} / N^k) x} a_i \phi(x/2N) \right) \right\|_{L^p(\RR^d)}^2 \right)^{1/2} \]
%
which assuming $\sum |a_i|^2 = 1$, can be as bad as
%
\[ C \cdot N^{1-1/p} D_{k-1} M^{1/2}. \]
%
s

Now
%
\[ \EE \left[ e^{2 \pi i (M_{i_1,\dots,i_k} / N^k) x} \right] = \frac{1}{N} \sum_{j = 0}^{N-1} e^{2 \pi i (x/N^k) j} = (1/N) \frac{e^{2 \pi i x/N^{k-1}} - 1}{e^{2 \pi i x / N^k} - 1}. \]
%
This quantity becomes small when $d(x, N^{k-1} \ZZ) \lesssim N^{k-1}$, and large (on the scale of $O(1)$) when $x$ nears an integer multiple of $N^k$. Applying Cauchy-Schwartz, we conclude that
%
\[ \EE \left[ \sum_{i_k} e^{2 \pi i (M_{i_1,\dots,i_k} / N^k) x} a_{i_1,\dots,i_k} \right] = \frac{e^{2 \pi i x/N^{k-1}} - 1}{e^{2 \pi i x / N^k} - 1} \cdot \EE \left[ \sum_{i_k} a_{i_1,\dots,i_k} \right]. \]
%
In the worst case then, this expected value can be $O(N M^{1/2})$, and outside a set of total length $O(N^{k-1})$, it is likely to be on the other of $O(M^{1/2})$. Applying Hoeffding's inequality, if
%
\[ S_{i_1,\dots,i_{k-1}} = \sum_{i_k} e^{2 \pi i (M_{i_1,\dots,i_k} / N^k) x} a_{i_1,\dots,i_k}, \]
%
\end{comment}

\begin{comment}
Thus applying a union bound over all $i_1,\dots,i_{k-1}$, $S_{i_1,\dots,i_{k-1}}$ deviates from it's expected value by at most $O(\log M \cdot |a_{i_1,\dots,i_{k-1}}|)$ with high probability. Thus summing over $i_1,\dots,i_{k-1}$, we get a total error of
%
\[ O( \log M \sum_{i_1,\dots,i_{k-1}} |a_{i_1,\dots,i_{k-1}}| ) \]

we get that for each $i_1,\dots,i_{k-1}$,
%
\[ \| \sum_{i_k} e^{2 \pi i (M_{i_1,\dots,i_k} / N^k) x} a_{i_1,\dots,i_k} - \EE[] \|_{\psi_2} \lesssim \left( \sum \right)^{1/2} \]
\end{comment}


Here, we get a Martingale at each scale, in some sense, so maybe instead of Hoeffding's inequality, some kind of Martingale inequality will work for obtaining good uniform bounds.



\section{Best Examples of Decoupling Constant}

What are some extremizers for the decoupling constant? If we consider the setup as above, i.e. at the scale $1/N^k$, and set $a_i = 1 / M^{k/2}$ for all $k$ multi-indexes $i$, then the resulting function $\sum a_i f_i$ has magnitude $\sim M^{k/2}$ on the interval of length $N^{k-1}$ centered at the origin, and thus the $L^p$ norm is at least
%
\[ M^{k/2} N^{(k-1)/p} = N^{-1/p} N^{k(1/p + s/2)}. \]
%
On the other hand,
%
\[ \left( \sum |a_i|^2 \| f_i \|_{L^p(\RR^d)}^2 \right)^{1/2} \sim N^{k/p}. \]
%
Thus we find that $D_k \gtrsim N^{ks/2 - 1/p}$. Thus the decoupling constant must get bad for large $k$. TODO
% The decoupling constant for s = 1 should be lower bounded by N^{k(1/2 - 1/p)} and this seems to do much better than that?

TODO: Zane Mentioned some weird extremizers for the Cantor sets that aren't like the Knaap example.


\section{Bourgain Beat Us To It}

In (Bourgain, 1989), it is proved that for any $2 < p < \infty$, and for any $\{ 0, \dots, N-1 \}$, there exists $S \subset \{ 0, \dots, N-1 \}$ with $\#(S) > N^{2/p}$ such that for any $\{ a_i : i \in S \}$,
%
\[ \| \sum_{i \in S} a_i e^{2 \pi i \xi_i \cdot x} \|_{L^p[0,1]} \lesssim_p \left( \sum_i |a_i|^2 \right)^{1/2}. \]
%
How is this proved? For $S \subset \{ 0, \dots, N-1 \}$ define
%
\[ D(S) = \sup_{|a| \leq 1} \| \sum_{i \in S} a_i e^{2 \pi i \xi_i \cdot x} \|_{L^p(\RR^d)}. \]
%
Let us begin with the case $2 < p \leq 3$. Applying, e.g. conveity, choose $0 < \gamma < 1$ with
%
\[ (1 - \gamma^2)^{p/2 - 1} + (\gamma^2)^{p/2} < 1. \]
%
Given any sequence $a \in \RR^M$, divide $\{ 1, \dots, M \}$ minus at most one index into two sets $I$ and $J$, where $\sum_{i \in I} a_i^2 < \gamma^2$ and $\sum_{j \in J} a_j^2 < 1 - \gamma^2$.

Choose $0 < \gamma < 1$ with $(1 - \gamma^2)^{p/2 - 1} + (\gamma^2)^{p/2} < 1$. Given any sequence $a \in \RR^M$, we divide all but at most one element of $\{ 1, \dots, M \}$ into two sets $I$ and $J$ such that $\min_{i \in I} |a_i| \geq \chi \geq \max_{j \in J} |a_j|$.

\begin{lemma}
    If $\xi_1,\dots,\xi_M$ are independent $\{ 0, 1 \}$-valued $\text{Ber}(\delta)$ random variables, $1 \leq K \leq M$, $q \geq 1$, and $\mathcal{E}$ is a subset of $[0,\infty)^M$. Then
    %
    \begin{align*}
        \EE \left( \left| \sup_{\substack{x \in \mathcal{E}\\ \#(A) \leq K}} \sum_{i \in A} \xi_i a_i \right|^q \right)^{1/q} &\lesssim \left( \delta K + \frac{q}{\log(1/\delta)} \right)^{1/2} \left( \sup_{a \in \mathcal{E}} |a| \right)\\
        &\quad\quad\quad + \log(1/\delta)^{-1/2} \int_0^B (\log N(\mathcal{E},t))^{1/2}\; dt.
    \end{align*}
    %
    Note that if we let $\delta \to 1$, then the bound in this theorem becomes useless, so some amount of switching on and off is necessary to prevent the trivial bounds that we obtain above. If $\mathcal{E}$ is the unit sphere, then we get a bound
    %
    \[ \left( \delta K + \frac{M + q}{\log(1/\delta)} \right)^{1/2}, \]
    %
    so the trivial bound of $M^{1/2}$ is eliminated if $\delta$ is chosen sufficiently small.
\end{lemma}
\begin{proof}
    Let $B = \sup_{x \in \mathcal{E}} |x|$. We first note that
    %
    \[ \EE \left( |\xi_1 + \dots + \xi_l|^p \right)^{1/p} \lesssim \delta l + \frac{p}{\log(2 + p/\delta l)}. \]
    %
    Now consider a family of sets $\{ \mathcal{E}_k : k \in \mathbf{Z} \}$ of vectors in the unit ball such that
    %
    \[ \log ( \#(\mathcal{E}_k) ) \lesssim \log ( N(\mathcal{E}, 2^{k-2}) ), \]
    %
    and such that for any $a \in \mathcal{E}$, we can find $a_k \in \mathcal{E}_k$ for each $k$ such that
    %
    \[ a = \sum_{2^k \leq B} 2^k a_k. \]
    %
    Thus
    %
    \[ \EE \left( \left| \sup_{\substack{x \in \mathcal{E}\\ \#(A) \leq K}} \sum_{i \in A} \xi_i a_i \right|^q \right)^{1/q} \leq \sum_{2^k \leq B} 2^k \cdot \EE \left( \left| \sup_{\substack{a \in \mathcal{E}_k\\\#(A) \leq m}} \sum_{i \in A} \xi_i a_i \right|^q \right)^{1/q}. \]
    %
    This reduces to proving the result assuming that $B = 1$ and summing in $k$, which we can simplify to showing that 
    %
    \[ \EE \left( \left| \sup_{\substack{x \in \mathcal{E}\\ \#(A) \leq K}} \sum_{i \in A} \xi_i a_i \right|^q \right)^{1/q} \lesssim \left( \delta M + \frac{q}{\log(1/\delta)} + \log \#(\mathcal{E}) \right)^{1/2}. \]
    %
    Indeed, the two bounds are equivalent if $\# \mathcal{E}$ is discretized, which is the case if we chose our sets $\mathcal{E}_k$ as above in an efficient manner. Now fix $\rho_1, \rho_2$. Then for any $A$ with $\#(A) \leq K$,
    %
    \begin{align*}
        \sum_{i \in A} \xi_i a_i &\leq \sum_{a_i \leq \rho_1} \xi_i a_i + \sum_{\rho_1 < a_i < \rho_2} \xi_i a_i + \sum_{\rho_2 \leq a_i} \xi_i a_i\\
        &\leq K \rho_1 + \sum_{\rho_1 < a_i < \rho_2} + TODO,
    \end{align*}
    %
    TODO FINISH THIS.
\end{proof}







\begin{comment}

In Zane's Paper, set with dimension s = log k / log q.

We get good decoupling for p <= 2/s, so it that scenario we get decoupling for p <= 2 log q /log k.
Here q is the base, and k is the maximum number of digits in the equation.


\end{comment}

\end{document}
