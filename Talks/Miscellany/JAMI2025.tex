\documentclass[12pt]{article}

\usepackage[text={16cm,24cm}]{geometry}

\usepackage{amssymb,amsmath,amsthm}
\usepackage{mathtools}
\usepackage{amsrefs}
\usepackage{comment}
\usepackage{amsthm}

\title{}
\author{}
\date{}

\begin{document}

\maketitle

\thispagestyle{empty}

Dear Organizers of the JAMI 2025 Conference,
\\

I am a harmonic analyst, working on problems very closely related to the work of Professor Sogge. In particular, my research focuses on the regularity of wave propogation,  eigenfunctions on Riemannian manifolds, and geometric measure theory. Work on my PhD problem was greatly elevated by a 2022 Summer School in Madison co-organized by Professor Sogge on eigenfunctions; I believe the conference you are organizing will benefit me just as much, giving me a huge boost to my post thesis work, in addition to giving me networking opportunities with the harmonic analysis and microlocal analysis communities. \\

I have recently made progress on endpoint bounds for spectral multipliers of eigenfunctions on manifolds, a problem very closely connected to local smoothing phenomena for wave equations, and thus to Kakeya-type problems. There has been notable progress on Kakeya type phenomena on manifolds. In particular, we now know that direct analogues of the Kakeya and Nikodym maximal function conjectures can only hold on constant curvature manifolds, and last month a preprint was submitted claiming that analysis of Kakeya on such manifolds can be reduced to the Euclidean case, and thus Kakeya on constant curvature manifolds can be reduced to the work of Wang and Zahl. Nonetheless, it remains unclear why constant curvature is necessary, and to what extent (and in terms of what geometric properties) we should expect bounds related to Kakeya to fail for such manifolds. I hope Hong Wang's lecture series will help me think more deeply about such problems, giving insight on the future work I hope to conduct related to improving bounds on manifolds which fail to have constant curvature.\\

Attending the conference also gives me the valuable chance to meet and network with mathematicians sharing my research interests. This is particularly important as I will be moving to Scotland in August for a postdoctoral position, making this one of my last opportunities to connect with professors in America before I relocate. While I am not completely aware of the work the speakers will present, I have great interest in Hart Smith's work on Fourier Integral Operators and Matthew Blair's work on spectral projection estimates. I am also starting to work on dispersive wave equations and scattering, and so look forward to talking with and hearing about the work of Oana Ivanovici, Makoto Nakamura, Fabrice Planchon, and Maciej Zworski. \\

\noindent Thank you for your consideration, \\

\noindent Jacob Denson

\end{document}