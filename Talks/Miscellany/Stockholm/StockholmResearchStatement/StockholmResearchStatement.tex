\documentclass[11pt]{article}

\usepackage[a4paper, margin=1.1in]{geometry}
\usepackage{mathptmx}

\usepackage{filecontents}
\begin{filecontents}{mybooknumeric.bbx}
\ProvidesFile{mybooknumeric.bbx}

\RequireBibliographyStyle{standard}
\RequireBibliographyStyle{numeric}
%
\DeclareBibliographyOption[boolean]{dashed}[true]{%
  \ifstrequal{#1}{true}
    {\ExecuteBibliographyOptions{pagetracker}%
     \renewbibmacro*{bbx:savehash}{\savefield{fullhash}{\bbx@lasthash}}}
    {\renewbibmacro*{bbx:savehash}{}}}
%
\newbibmacro*{bbx:savehash}{%
  \savefield{fullhash}{\bbx@lasthash}}

\renewbibmacro*{author}{%
  \ifboolexpr{
    test \ifuseauthor
    and
    not test {\ifnameundef{author}}
  }
  {\usebibmacro{bbx:dashcheck}
    {\bibnamedash}
    {\printnames{author}%
      \setunit{\addcomma\space}%
      \usebibmacro{bbx:savehash}}%
    \usebibmacro{authorstrg}}
  {\global\undef\bbx@lasthash}}

\newbibmacro*{bbx:dashcheck}[2]{%
  \ifboolexpr{
    test {\iffieldequals{fullhash}{\bbx@lasthash}}
    and
    not test \iffirstonpage
    % NOTE: the follow only matters if you have defined and set up the boolean `bbx@inset` (which is defined in authortitle.bbx).
    % and
    % (
    %   not bool {bbx@inset}
    %   or
    %   test {\iffieldequalstr{entrysetcount}{1}}
    % )
  }
  {#1}
  {#2}}

\end{filecontents}

\usepackage[
backend=biber,
bibstyle=mybooknumeric,
dashed=true,
citestyle=numeric,
firstinits=true
]{biblatex}

\renewbibmacro{in:}{}

\addbibresource{ResearchStatement.bib}
\renewcommand*{\bibnamedash}{\rule{3em}{0.4pt}\hspace*{.16667em}\addcomma\addspace}

\DeclareFieldFormat*{title}{#1}

%\usepackage[text={16cm,24cm}]{geometry}

\usepackage{amssymb,amsmath,amsthm}
\usepackage{mathtools}
\usepackage{comment}
 \usepackage{paralist}

     \setlength{\parskip}{0cm}
    \setlength{\parindent}{1em}

        \usepackage[compact]{titlesec}
    \titlespacing{\section}{0pt}{2ex}{2ex}
    \titlespacing{\subsection}{0pt}{1.5ex}{1.5ex}
    \titlespacing{\subsubsection}{0pt}{0.5ex}{0ex}

\theoremstyle{definition}
\newtheorem{theorem}{Theorem}

\DeclareMathOperator{\QQ}{\mathbb{Q}}
\DeclareMathOperator{\ZZ}{\mathbb{Z}}
\DeclareMathOperator{\RR}{\mathbb{R}}
\DeclareMathOperator{\NN}{\mathbb{N}}
\DeclareMathOperator{\HH}{\mathbb{H}}
\DeclareMathOperator{\BB}{\mathbb{B}}
\DeclareMathOperator{\CC}{\mathbb{C}}
\DeclareMathOperator{\AB}{\mathbb{A}}
\DeclareMathOperator{\PP}{\mathbb{P}}
\DeclareMathOperator{\MM}{\mathbb{M}}
\DeclareMathOperator{\VV}{\mathbb{V}}
\DeclareMathOperator{\TT}{\mathbb{T}}
\DeclareMathOperator{\LL}{\mathcal{L}}
\DeclareMathOperator{\DD}{\mathcal{D}}
\DeclareMathOperator{\SW}{\mathcal{S}}
\DeclareMathOperator{\EC}{\mathcal{E}}
\DeclareMathOperator{\AC}{\mathcal{A}}

\title{\vspace{-2em}Research Proposal}
\author{Jacob Denson}
\date{}

\begin{document}

\maketitle

I am an analyst, studying problems using harmonic analysis, and also some combinatorics and probability theory. My PhD research, advised by Andreas Seeger, focuses on the relation between radial Fourier multipliers and spectral multipliers on compact manifolds, through an understanding of the geometry and regularity of wave propagation. In addition, I have worked on problems investigating when `structure' occurs in fractals of large dimension. Both projects raise intriguing questions closely tied to the work of the analysis group at Stockholm University. A Postdoc position at the department would thus greatly enrich my work, and hopefully the analysis group in return.

\section*{Prior Research}

\subsection*{Relations Between Multipliers on the Sphere and $\RR^d$}

My thesis work studies connections between radial Fourier multipliers and multipliers of spherical harmonic expansions on the sphere, using the theory of Fourier integral operators. For any $a: [0,\infty) \to \CC$, we define a Fourier multiplier $T_a$ on $\RR^d$ and a multiplier operator $S_a$ on $S^d$ by setting
%
\[ T_a f(x) = \int_{\RR^d} a ( |\xi| ) \widehat{f}(\xi) e^{2 \pi i \xi \cdot x}\; d \xi \quad\text{and}\quad S_a f = \sum\nolimits_{k = 0}^\infty a(k) H_k f. \]
%
Here $H_k$ is the orthogonal projection operator onto the space of degree $k$ spherical harmonics on $S^d$. %Every operator on $\RR^d$ commuting with rotations and translations is a Fourier multiplier operator, and every operator on $S^d$ commuting with rotations is a multiplier operator of the form above, explaining their ubiquity in mathematics and related areas. %In harmonic analysis, it has proved incredibly profitable to study the boundedness of multiplier operators with respect to various $L^p$ norms. It seems to be one of the few tractable ways of quantifying interactions between different waves, thus underpinning all deeper understandings of the Fourier transform and expansions in spherical harmonics.

There is some evidence that the boundedness of the operator $T_a$ on $L^p(\RR^d)$ is connected to the uniform boundedness of the operators $\{ S_{a_R} : R > 0 \}$ on $L^p(S^d)$, where $a_R(\cdot) = a(\cdot / R)$ are dilates of $a$. Indeed, Mitjagin \cite{Mitjagin} proved $\| T_a \|_{L^p \to L^p} \lesssim \sup_R \| S_{a_R} \|_{L^p \to L^p}$ for all $p \in [1,\infty]$, a `transference principle' from $S^d$ to $\RR^d$. %The result is intuitive, since, very roughly speaking, one locally has $T_a = \lim_{R \to \infty} S_{a_R}$, because one can view the dilation of $a$ instead as a dilation of the metric on $S^d$, becoming flatter and flatter as $R \to \infty$. Mitjagin's result then follows by `taking operator norms on each side of the equation'.
On the other hand, the reverse inequality $\sup_R \| S_{a_R} \|_{L^p \to L^p} \lesssim \| T_a \|_{L^p \to L^p}$ remained unproved for all $p \neq 2$ for a half century, and there is evidence the inequality does not hold for all $p \in [1,\infty]$. \emph{In \cite{DensonCharacterization}, I proved this inequality for a limited range of $p$ and compactly supported $a$:}

\begin{theorem} \label{characterization}
    For $d \geq 4$ and $1/2 < |1/p - 1/2| < (d-1)^{-1}$, if $a(\rho) = 0$ for $\rho \not \in [1,2]$, then
    %
    \begin{equation}
        \sup\nolimits_R \| S_{a_R} \|_{L^p \to L^p} \lesssim \| T_a \|_{L^p \to L^p}.
    \end{equation}
    %
    Moreover,
    %
    \begin{equation} \label{secondinequality}
        \sup\nolimits_R \| S_{a_R} \|_{L^p \to L^p} \sim \left( \int_0^\infty ( 1 + |t| )^{(d-1)(1 - p/2)} | \widehat{a}(t) |^p\; dt \right)^{1/p}.
    \end{equation}
\end{theorem}

Equation \eqref{secondinequality} of Theorem \ref{characterization} completely characterizes $a$ whose corresponding operators $\{ S_{a_R} \}$ are uniformly bounded on $L^p$, \emph{and is the first such characterization for any $p \neq 2$.}

The proof method is based on an idea of H\"{o}rmander \cite{Hormander2}; Fourier inversion implies
%
\begin{equation} \label{SphereWave}
    S_{a_R} = \int_{-\infty}^\infty \widehat{a}_R(t) W_t\; dt
\end{equation}
%
where the operators $\{ W_t \}$, as $t$ varies, solve a half wave equation on $S^d$. Equation \eqref{SphereWave} connects the study of the multipliers $\{ S_{a_R} \}$ to the regularity of wave propogation on $S^d$, in particular to local smoothing bounds, which we then exploit. In a follow up paper, currently in preparation, I extend Theorem 1 to general $a$, combining atomic decompositions on $L^p(S^d)$ with square function estimates of Peetre \cite{Peetre}.

\subsection*{Fractal Sets Avoiding Configurations}

How large must a set $X \subset \RR^d$ be before it contains patterns, such as four points forming a parallelogram? Problems of this flavor have long been studied in combinatorics. Analogous problems for infinite subsets $X \subset \RR^d$ can also be considered, where the size of $X$ is replaced by a `fractal dimension', a statistic measuring how `spread out' $X$ is in space. The most common fractal dimension used in this setting is the Hausdorff dimension, but the Fourier dimension is also useful, providing greater structural information about the correlation of $X$ with planar waves.

A model problem, given $f: (\RR^d)^n \to \RR^m$, is to determine the smallest dimension of a set $X$ which guarantees that there exists distinct $x_1,\dots,x_n \in X$ so that $f(x_1,\dots,x_n) = 0$. My work so far has focused on obtaining lower bounds for this problem, constructing sets $X$ with large dimension such that for all distinct $x_1,\dots,x_n \in X$, $f(x_1,\dots,x_n) \neq 0$. We say such sets `avoid the zeroes of $f$'. %Constructions of sets with large Hausdorff dimension avoiding zeroes of $f$ have been considered when $f$ is a low degree polynomials \cite{Mathe} and when $f$ is a submersions \cite{FraserPramanik}.
Before my PhD, in joint work with Pramanik and Zahl \cite{DensonPramanikZahl}, I constructed large sets avoiding zeroes of $f$ assuming only dimension bounds on the zero set $f^{-1}(0)$:

\begin{theorem}
    Suppose $f^{-1}(0)$ has Minkowski diension $s$. Then there exists a set $X \subset [0,1]^d$ with Hausdorff dimension $(dn-s)/(n-1)$ avoiding the zeroes of $f$.
\end{theorem}

In my MSc thesis \cite{DensonThesis}, I constructed large sets avoiding zeroes of $f$, when $d = 1$, when $f$ factors through a low rank projection:

\begin{theorem}
    Suppose $f = g \circ T$, where $T: \RR^n \to \RR^l$ is a full-rank rational coefficient linear map and $g: \RR^l \to \RR^m$ is a submersion. Then there is a set $X \subset [0,1]$ with Hausdorff dimension $1/m$ avoiding the zeroes of $f$.
\end{theorem}

During my PhD, I tried to find analogues of these constructions in the much harder Fourier dimension setting, where the only known constructions assume $f$ is linear. In \cite{DensonFourier}, I found constructions when $f(x_1,\dots,x_n) = x_n - g(x_1,\dots,x_{n-1})$, for a function $g$ which is a submersion in each variable:

\begin{theorem}
    Fix $g: [0,1]^{d(n-1)} \to \RR^d$, and suppose that for each $k \in \{ 0, \dots, n-2 \}$, the matrix $D_k g = ( \partial g_i / \partial x_{dk+j} )_{i,j = 1}^d$ is invertible. Then there exists a set $X \subset [0,1]^d$ of Fourier dimension $d/(n-3/4)$ such that for all distinct $x_1,\dots,x_n \in X$, $x_1 \neq g(x_2,\dots,x_n)$. If, in addition, $g(x_2,\dots,x_n) = ax_2 + h(x_3,\dots,x_n)$ for some $a \in \QQ$, then there exists a set $X \subset [0,1]^d$ of Fourier dimension $d/(n-1)$ such that for all distinct $x_1,\dots,x_n \in X$, $x_1 \neq g(x_2,\dots,x_n)$.
\end{theorem}

\emph{This method remains the only method of constructing sets with large Fourier dimension avoiding nonlinear patterns, and remains the best method for avoiding general `linear' patterns when $d > 1$.} The proof involves the probabilistic method. Given the problem's nonlinearity, we employ results from the theory of concentration of measure to ensure Fourier decay does not deviate from the expected decay with high probability; we then control the expected decay via an inclusion-exclusion argument associated with a Whitney decomposition of a neighborhood of $f^{-1}(0)$.

\section*{A Plan For Research at Stockholm University}

I now propose four projects I plan to pursue during my postdoctoral work. In an attached letter, I connect these plans to the research goals of the analysis group at Stockholm University.

\subsection*{Spectral Multipliers Associated With Periodic Geodesic Flow}

A natural generalization of Theorem \ref{characterization} can be formulated for multipliers of the Laplace-Beltrami operator on a compact Riemannian manifold. H\"{o}rmander's generalizes to this situation, so multipliers for the Laplace-Beltrami operator can be studied for the half-wave equation with propogators $\{ e^{2 \pi i t \sqrt{-\Delta}} \}$, and solutions to this equation propogate along the geodesic flow of the manifold.

%In Section \ref{Section2}, I discussed that the results I were able to obtain for multiplier operators on $S^d$ generalized to multipliers of an arbitrary first order, elliptic, self-adjoint pseudodifferential operator $P$ on a compact manifold $M$, provided that $P$ satisfied two assumptions. Assumption A relates to the curvature of the principal symbol, and this assumption cannot really be weakened without significantly changing the character of the results, which heavily depend on this curvature. On the other hand, Assumption B arises as an artifact of the methods of our proof. We can likely obtain similar bounds while weakening this assumptions; for instance, Kim \cite{KimManifold} obtained bounds on the scale of Besov spaces only under Assumption A.

It is very difficult to generalize the results of \cite{DensonCharacterization} to all Riemannian manifolds $M$, since current technology cannot accurately estimate the large time behavior of wave equations on compact manifolds. The method of \cite{DensonCharacterization} suggest results would follow if we could prove
%
\begin{equation} \label{smoothingbound}
    \| \mathbf{I}_{[k,k+1]} e^{2 \pi i t \sqrt{-\Delta}} f \|_{L^q_x L^q_t} \lesssim k^\delta \| f \|_{L^q_{d(1/p - 1/2) - 1/p'}}
\end{equation}
%
for appropriate $\delta$. It is feasible to obtain \eqref{smoothingbound} from local smoothing inequalities if we assume that $e^{2 \pi i k \sqrt{-\Delta}}$ is close to the identity for all $k$. This happens precisely when the geodesic flow on $M$ is periodic, since the theory of propogation of singularities then tells us the operator $R = e^{2 \pi i \sqrt{-\Delta}}$ and its iterates are order zero pseudodifferential operators, with principal symbol controlled by a topological invariant of the flow known as the Maslov index. The operator has been studied a little by spectral theorists, and there it is known as the \emph{return operator}. If we are able to control the $L^p$ operator norm of iterates $R^k$ of the return operator, then local smoothing yields \eqref{smoothingbound}. More generally, one might investigate whether \eqref{smoothingbound} holds when the geodesic flow on a manifold has other nice dynamical properties, like forming an integrable system. I am also interested in studying bounds for multilinear variable-coefficient problems such as those related to $k$-linear restriction estimates to the cone, which may also have application to improve the range of Theorem \ref{characterization}.

\subsection*{Vector Field Methods}

In my PhD work, I mainly worked on obtaining $L^p$ control over linear wave equations. Recently, I have started exploring methods for studying the long time behaviour of nonlinear wave equations. The `vector field method' has proved a key tool in this setting for quite some time, capturing the dispersive nature of waves by using vector fields on the background space-time to captur approximate symmetries of an equation. The method has proven a powerful tool in the study of highly nonlinear hyperbolic equations. Recently, Ifrim and Tataru \cite{IfrimTataruWaterWaves} have developed a new method called `wave packet testing' complementing the vector field method, to obtain global control on various nonlinear dispersive equations, first applied to the non-linear Schr\"{o}dinger equation \cite{IfrimTataruNLS}, and later applied successfully for many other dispersive equations such as 2d water waves and 3d Maxwell-Dirac systems.

Notably, Ifrim and Tataru's methods have not been applied to the study of wave equations on black-hole spacetimes. For some time, obtaining sharp decay estimates of solutions to the linear wave-equation on high-dimensional Schwarzschild black hole spacetimes proved difficult, until resolved by Schlue \cite{Schlue}. But Schlue's method requires modification for spacetimes with dynamic backgrounds, which introduce a nonlinearity into the problem. I believe Ifram and Tataru's methods could apply the necessary control to obtain results.

\subsection*{Nonlinear Roth Theorems For Fractals}

My past work on fractals constructed large sets avoiding patterns. In my postdoctoral work, I want to work on proving that sets with large Hausdorff dimension must contain certain patterns. I'm particularly interested in recent methods establishing `non-linear Roth theorems' for fractal sets. Previous methods, first introduced by {\L}aba and Pramanik \cite{LabaPramanik} %, Chan {\L}aba and Pramanik \cite{ChanLabaPramanik}, Henriot {\L}aba and Pramanik \cite{HenriotLabaPramanik}, and Fraser, Guo, and Pramanik \cite{FraserGuoPramanik}
establish the existence of patterns in sets supporting measures with both Fourier and mass decay. But recently, Kuca, Orponen, and Sahlsten \cite{KucaOrponenSahlsten} showed the existence of solutions to $x_4 - x_3 = (x_2 - x_1)^2$ in subsets of $[0,1]$ with large Hausdorff dimension, avoiding any need of Fourier decay, using a new method of `spectral gaps'. The authors view this result as a quantitative improvement of a non-linear Roth theorem of Bourgain \cite{Bourgain2}. Durcik, Guo, and Roos \cite{DurcikGuoRoos} generalize \cite{Bourgain2}, proving the existence of solutions to $x_3 - x_1 = P(x_2 - x_1)$ in sets of suitably large density, where $P$ is a polynomial of degree $d \geq 2$, and I am interested in investigating whether the methods of \cite{KucaOrponenSahlsten} may yield a `quantitative improvement' on the result of \cite{DurcikGuoRoos}, proving the existence of such solutions to $x_3 - x_1 = P(x_2 - x_1)$ assuming only the Hausdorff dimension of the set is sufficiently large.

\subsection*{Genuine Decoupling On Random Fractals}

One major development in harmonic analysis in the past decade has been a greater understanding of the phenomenon of \emph{decoupling}, or \emph{Wolff-type estimates}. Given a family of almost disjoint subsets $\mathcal{E}_\delta$ of $\RR^d$ parameterized by $\delta > 0$, $L^p(l^2)$ decoupling discusses bounds of the form
%
\begin{equation} \label{decoupling}
    \Big\| \sum f_j \Big\|_{L^p(\RR^d)} \leq D_p(\delta) \Big( \sum\nolimits_j \| f_j \|_{L^p(\RR^d)}^2 \Big)^{1/2},
\end{equation}
%
where the $\widehat{f}_j$ are supported on distinct sets in $\mathcal{E}_\delta$, and $D_p(\delta)$ is the best constant so \eqref{decoupling} holds for all such $\{ f_j \}$. We say a \emph{genuine decoupling inequality} results when $D_p(\delta) \lesssim_\varepsilon \delta^{-\varepsilon}$ for all $\varepsilon > 0$.%  or the stronger logarithmic bound $D_p(\delta) \lesssim \log(\delta)$ and other variants of this form.

Much work has been carried out for decoupling inequalities associated with curves and surfaces. But decoupling on \emph{fractal sets} is still poorly understood. Let $\{ X_i \}$ be a nested family of sets, where $X_i$ is a union of disjoint cubes of some fixed sidelength $\delta_i$. This setup is a Cantor-type construction of the set $X = \bigcap X_i$. Now consider a sequence of integers $n(i)$, and define $\mathcal{E}_{\delta_i}$ be all sets of the form $Q \cap X_{i + n(i)}$, where $Q$ is a sidelength $\delta_i$ cube contained in $X_i$. We now ask for which sequences $\{ X_i \}$ and for which sequences $\{ n(i) \}$ do we obtain a genuine decoupling inequality for the families $\{ \mathcal{E}_{\delta_i} \}$.

Genuine decoupling bounds have not been established for any fractal set, but methods similar to those used in \cite{DensonFourier} may yield genuine decoupling for random Cantor-type constructions; The theory of concentration of measure has been applied by Bourgain \cite{Bourgain} and Talagrand \cite{Talagrand} to prove the existence of $\Lambda(p)$ sets. %, in particular, the method of majorizing measures and selection processes.
One might view $\Lambda(p)$ sets as a discrete variant of sets where genuine decoupling holds, so it is likely to believe these methods generalize to the continous setting. Using these methods, I hope to obtain an analogue of the proof of $l^2(L^p)$ decoupling for the paraboloid found in \cite{BourgainDemeterStudyGuide}, i.e. establishing an analogue of multilinear Kakeya for the sets $\mathcal{E}_\delta$, and then using an induction on scales to obtain genuine fractal decoupling.

\AtNextBibliography{\footnotesize}
\printbibliography

\end{document}