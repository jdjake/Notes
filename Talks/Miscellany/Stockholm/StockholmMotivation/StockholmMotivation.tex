%\title{Example letter using the newlfm LaTeX package}
%
% See http://texblog.org/2013/11/11/latexs-alternative-letter-class-newlfm/
% and http://www.ctan.org/tex-archive/macros/latex/contrib/newlfm
% for more information.
%
\documentclass[11pt,stdletter,orderfromtodate,sigleft]{newlfm}

\usepackage{etoolbox}

\usepackage{amssymb,amsmath,amsthm}
\usepackage{mathtools}
\usepackage{comment}
 \usepackage{paralist}

\DeclareMathOperator{\QQ}{\mathbb{Q}}
\DeclareMathOperator{\ZZ}{\mathbb{Z}}
\DeclareMathOperator{\RR}{\mathbb{R}}
\DeclareMathOperator{\NN}{\mathbb{N}}
\DeclareMathOperator{\HH}{\mathbb{H}}
\DeclareMathOperator{\BB}{\mathbb{B}}
\DeclareMathOperator{\CC}{\mathbb{C}}
\DeclareMathOperator{\AB}{\mathbb{A}}
\DeclareMathOperator{\PP}{\mathbb{P}}
\DeclareMathOperator{\MM}{\mathbb{M}}
\DeclareMathOperator{\VV}{\mathbb{V}}
\DeclareMathOperator{\TT}{\mathbb{T}}
\DeclareMathOperator{\LL}{\mathcal{L}}
\DeclareMathOperator{\DD}{\mathcal{D}}
\DeclareMathOperator{\SW}{\mathcal{S}}
\DeclareMathOperator{\EC}{\mathcal{E}}
\DeclareMathOperator{\AC}{\mathcal{A}}

%% Patch from https://tex.stackexchange.com/a/395529/226 to address newlfm bug
\makeatletter
\patchcmd{\@zfancyhead}{\fancy@reset}{\f@nch@reset}{}{}
\patchcmd{\@set@em@up}{\f@ncyolh}{\f@nch@olh}{}{}
\patchcmd{\@set@em@up}{\f@ncyolh}{\f@nch@olh}{}{}
\patchcmd{\@set@em@up}{\f@ncyorh}{\f@nch@orh}{}{}
\makeatother
 
\newlfmP{Headlinewd=0pt,Footlinewd=0pt}
 
\namefrom{}

\dateset{}
 
\greetto{Dear Members of the Sverker Lerheden Fellowship Selection Commitee,}

\closeline{}

\newlfmP{dateskipbefore=-4cm}
 \newlfmP{dateskipafter=1cm}

\begin{document}

\begin{newlfm}

I am a PhD student of Dr. Andreas Seeger at the University of Wisconsin-Madison, graduating this coming spring. My work has a strong affinity with the interests of Stockholm University's analysis group. In particular, this include Dr. Oliver Petersen's work in black hole stability, Dr. Olof Sisask's work in additive combinatorics, and Dr. Salvador Rodr\'{i}guez-L\'{o}pez on FIOs and spectral theory. In the remaining part of this letter, I describe the potential interactions between my research plans and the work of the group, justifying that the Sverker Lerheden fellowship would be a fantastic opportunity for me to help carry out my postdoctoral work.

Dr. Petersen specializes in the uniqueness and stability of black holes. Wave equations on black hole spacetimes are a useful model problem in this setting to test the strong cosmic censorship hypothesis. During my Postdoc, I plan to use vector field methods to study the decay of wave packets on Schwarzschild black hole spacetimes, which strongly relates to Dr. Peterson's work. There seems to be an interesting dichometry between the vector fields method, and recent work by Dr. Peterson. Namely, the vector field method controls long time solutions to the wave equation on a manifold given control over certain symmetries, which in this case are `Killing fields' on spacetime. Conversely, Dr. Peterson has shown that in certain spacetimes and certain compact Cauchy horizons, long time control of the wave equation implies a spacetime must have Killing fields, implying that the `sharpness' of the vector field method. I would be interested in discussing this dichotomy with Dr. Petersen if I were to obtain a position in Stockholm.

Stockholm University is also within walking distance of the KTH Institute, so the Sverker Lerheden fellowship would also give me the opportunity to interact with the geometric analysis department there, who also specialize in nonlinear wave equations in spacetime.

The study of patterns in fractals blends analytical and combinatorial methods. Some discrete arguments apply to fractal dimensions and vice versa, creating fruitful interactions between the subjects. For instance, my results on finding large fractal sets avoiding patterns was heavily inspired by methods of random graph theory. Dr. Sisask and I both share interests in Roth type theorems (discussed in my research plan) and the probabilistic method (used in the construction above, and a suggested approach to attack the problem of obtaining decoupling inequalities for fractals). Thus I believe interactions between our two closely related research areas should lead to some interesting collaboration.

My thesis work establishes endpoint estimates and transference principles for multiplier operators on the sphere, using methods of FIOs. Dr. Rodr\'{i}guez-L\'{o}pez works on establishing bounds for bilinear operators associated with FIOs; if I were to join the department, I would be highly interested in discussing whether methods I have used in the linear FIO setting can be applied in the multilinear setting. In particular, Dr. Rodr\'{i}guez has established a bilinear variant of the Seeger-Sogge-Stein theorem; I am interested in determining whether one can establish a bilinear variant of the Mockenhaupt-Seeger-Sogge theorem, which controls FIOs satisfying a cinematic curvature condition. This result would result in improved $L^p$ bounds for spectral multipliers of $S^d$, briefly mentioned in my research plan.

Thank you for your consideration,\\\\
Jacob Denson

\end{newlfm}
\end{document}