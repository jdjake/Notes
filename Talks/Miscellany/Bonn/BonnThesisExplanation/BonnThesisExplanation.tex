\documentclass[11pt]{article}

\usepackage[a4paper, margin=1in]{geometry}
\usepackage{mathptmx}

%\usepackage[text={16cm,24cm}]{geometry}

\usepackage{amssymb,amsmath,amsthm}
\usepackage{mathtools}
\usepackage{amsrefs}
\usepackage{comment}
\usepackage{amsthm}
 \usepackage{paralist}

     \setlength{\parskip}{0cm}
    \setlength{\parindent}{1em}

        \usepackage[compact]{titlesec}
    \titlespacing{\section}{0pt}{2ex}{2ex}
    \titlespacing{\subsection}{0pt}{1.5ex}{1.5ex}
    \titlespacing{\subsubsection}{0pt}{0.5ex}{0ex}

\newtheorem*{theorem}{Theorem}

\DeclareMathOperator{\QQ}{\mathbb{Q}}
\DeclareMathOperator{\ZZ}{\mathbb{Z}}
\DeclareMathOperator{\RR}{\mathbb{R}}
\DeclareMathOperator{\NN}{\mathbb{N}}
\DeclareMathOperator{\HH}{\mathbb{H}}
\DeclareMathOperator{\BB}{\mathbb{B}}
\DeclareMathOperator{\CC}{\mathbb{C}}
\DeclareMathOperator{\AB}{\mathbb{A}}
\DeclareMathOperator{\PP}{\mathbb{P}}
\DeclareMathOperator{\MM}{\mathbb{M}}
\DeclareMathOperator{\VV}{\mathbb{V}}
\DeclareMathOperator{\TT}{\mathbb{T}}
\DeclareMathOperator{\LL}{\mathcal{L}}
\DeclareMathOperator{\DD}{\mathcal{D}}
\DeclareMathOperator{\SW}{\mathcal{S}}
\DeclareMathOperator{\EC}{\mathcal{E}}
\DeclareMathOperator{\AC}{\mathcal{A}}

\title{\vspace{-2em}Thesis Summary}
\author{Jacob Denson}
\date{}

\begin{document}

\maketitle

My thesis project, advised by Andreas Seeger, studies connections between radial multipliers on $\RR^d$ and multipliers of spherical harmonic expansions on the sphere $S^d$, using methods of Fourier integral operators. For a function $a: [0,\infty) \to \CC$, we define a radial Fourier multiplier operator $T_a$ and a multiplier operator $S_a$ on $S^d$ by setting
%
\[ T_af(x) = \int_{\RR^d} a(|\xi|) \widehat{f}(\xi) e^{2 \pi i \xi \cdot x}\; dx \quad\text{and}\quad S_a f = \sum\nolimits_{k = 0}^\infty a(k) H_k f. \]
%
Here $H_k$ is the orthogonal projection operator onto the space of degree $k$ spherical harmonics on $S^d$.

There is some evidence that the boundedness of the operator $T_a$ on $L^p(\RR^d)$ and the \emph{uniform} boundedness of the operators $\{ S_{a_R} : R > 0 \}$ on $L^p(S^d)$ are connected, where $a_R(\cdot) = a( \cdot / R)$ are dilates of $a$. Indeed, Mitjagin \cite{Mitjagin} proved that $\| T_a \|_{L^p \to L^p} \lesssim \sup_R \| S_{a_R} \|_{L^p \to L^p}$ for all $1 \leq p \leq \infty$. The result is intuitive, since, very roughly speaking, one locally has $T_a = \lim_{R \to \infty} S_{a_R}$ because one can view the dilation of $a$ instead as a dilation of the metric on $S^d$, which becomes flatter and flatter as $R \to \infty$. Mitjagin's result follows by `taking operator norms on each side of the equation'. The reverse inequality $\sup_R \| S_{a_R} \|_{L^p \to L^p} \lesssim \| T_a \|_{L^p \to L^p}$ is less intuitive, much more difficult to establish, and there is some evidence the inequality does not hold in general for all $L^p$. Indeed, it was unknown whether the reverse inequality was true for all $p \neq 2$. In my thesis, I will establish this inequality for a range of $L^p$ spaces on $S^d$. More precisely, I prove the following:
%
\begin{itemize}
    \item I proved the inequality $\sup_R \| S_{a_R} \|_{L^p \to L^p} \lesssim \| T_a \|_{L^p \to L^p}$ for $1 < p < {\scriptstyle \frac{2(d-1)}{   (d+1)} }$ and ${\scriptstyle \frac{2(d-1)}{(d-3)} } < p < \infty$, thus proving the first known \emph{transference principle} from $\RR^d$ to $S^d$ for any $p \neq 2$, and more generally, the first transference principle from $\RR^d$ to analogous operators on any compact manifold for $p \neq 2$.

    \item Consider a decomposition $a(\rho) = \sum_{j \in \ZZ} a_j(\rho / 2^j)$, where $a_j(\rho) = 0$ for $\rho \not \in [1,2]$. Heo, Nazarov, and Seeger \cite{HeoNazarovSeeger} showed that for $1 < p < {\scriptstyle \frac{2(d-1)}{(d+1)} }$, $\| T_a \|_{L^p \to L^p} \sim \sup\nolimits_j C_p(a_j)$, where
    %
    \[ C_p(a) = \left( \int_0^\infty \big| \langle t \rangle^{(d-1)(1/p - 1/2)} \widehat{a}(t) \big|^p\; dt \right)^{1/p} \quad\text{and}\quad \langle t \rangle = (1 + |t|^2)^{1/2}. \]
    %
    I proved $\sup_R \| S_{a_R} \|_{L^p \to L^p} \sim \sup_j C_p(a_j)$ for $1 < p < {\scriptstyle \frac{2(d-1)}{(d+1)} }$, thus obtaining an analogue of the results of \cite{HeoNazarovSeeger} for multipliers on $S^d$. This is the first characterization of the uniform boundedness of the operators $S_{a_R}$ for any $p \neq 2$ and any $d \geq 2$.
\end{itemize}
%
The proofs of these results, for functions $a$ with \emph{compact support}, can be found in \cite{DensonCharacterization}, with a paper extending these results to the general case in preparation.  In the remainder of this summary I give a brief description of the methods by which we obtain these results.

\section*{Description of Methods}

Classical methods for studying multiplier operators on $S^d$ involve the analysis of special functions and orthogonal polynomials, e.g. in the work of Bonami and Clerc \cite{BonamiClerc}. However, it is tough to combine this approach with more modern harmonic analysis methods. In the 1960s, H\"{o}rmander made a breakthrough by introducing the theory of Fourier integral operators, where more modern techniques may be applied. Note that the pseudodifferential operator $P = \sqrt{ ( {\scriptstyle \frac{d-1}{2} } )^2 - \Delta } - \left( {\scriptstyle \frac{d-1}{2}} \right)$ on $S^d$ satisfies $Pf = kf$ for any spherical harmonic $f$ of degree $k$, since $\Delta f = k(k+d-1) f$. Thus we may write $S_{a_R} = a_R(P)$, using the language of functional calculus. H\"{o}rmander proposed using the Fourier inversion formula to write
%
\[ a_R(P) = \int_{\RR} \widehat{a}_R(t) e^{2 \pi i t P}\; dt = \int_{\RR} R \widehat{a}(Rt) e^{2 \pi i t P}\; dt. \]
%
The operators $\{ e^{2 \pi i t P} \}$, as $t$ varies, give solutions to the `half-wave equation' $\partial_t = i P$ on $M$. Thus the study of the boundedness of the operator $a(P)$ is connected to the regularity for averages of the wave equation on $M$, in particular to local smoothing inequalities. To obtain control over these averages, we exploit the fact that the operators $\{ e^{2 \pi i t P} \}$ have \emph{cinematic curvature}, and that the operators $\{ e^{2 \pi i t P} \}$ are 1-periodic because all eigenvalues of $P$ are integers.

For $|t| < 1/2$, the Lax-H\"{o}rmander parametrix approximates $e^{2 \pi i t P}$ by an oscillatory integral with a phase $\Phi$ related to an eikonal equation on $S^d$. This oscillatory integral reveals the underlying \emph{dynamics} of the wave equation; the operator $e^{2 \pi i t P}$ acts on wave packets localized in phase space $T^* S^d$ by transporting them along the geodesic flow on $T^* S^d$. Using this intuition, for functions $f_0$ and $f_1$ spatially supported near $x_0,x_1 \in S^d$, one should expect $\langle e^{2 \pi i t_0 P} f_0, e^{2 \pi i t_1 P} f_1 \rangle$ is negligible unless the radius $t_0$ annulus centered at $x_0$ is near tangent to the radius $t_1$ annulus centered at $x_1$. Obtaining sharp control over \emph{how negligible} is difficult given the non-explicit phase $\Phi$. However, I obtained such control by taking a geometric interpretation of the eikonal equation defining $\Phi$, and using the second variation formula for geodesics on $S^d$ to obtain new nondegeneracy estimates for critical points of the phase $\Phi$. Generalizations of these bounds for other pseudodifferential operators $P$ are also obtained in \cite{DensonCharacterization} by generalizing this method to geodesics on \emph{Finsler manifolds}.

Once the appropriate inner product estimates are obtained, our problem reduces to counting near tangencies of a family of annuli. We obtain suitable estimates for the number of tangencies when the annuli we are considering are suitably sparse. Combining these estimates with a `density decomposition' of an arbitrary family of annuli into subsets of different sparsity, using a stopping time argument akin to the Calder\'{o}n-Zygmund decomposition, we obtain the appropriate $L^p$ bounds.

% which heavily inspired the method of \cite{DensonCharacterization}:

%  \medskip
%  \hangindent0.5em
%  \hangafter=0
%    \noindent{\emph{Let $T$ be a radial multiplier. Then we can write $T f = k * f$, where $k$ is the Fourier transform of the symbol of $a$. We write $k = \sum k_\tau$ and $f = \sum f_\theta$, where the functions $\{ k_\tau \}$ are supported on disjoint annuli supported at the origin, and the functions $\{ f_\theta \}$ are supported on disjoint cubes. Then $T f = \sum_{\tau,\theta} k_\tau * f_\theta$. Using the Fourier transform and Bessel functions,  we can justify that the inner product $\langle k_\tau * f_\theta, k_{\tau'} * f_{\theta'} \rangle$ is negligible unless the annulus of radius $\tau$ centered at $\theta$ is near tangent to the annulus of radius $\tau'$ centered at $\theta'$ at some point. Combining this inner product estimate with a `sparse incidence argument' for such annuli, one can show that the $L^2$ norm of a sum $\sum_{(\tau,\theta) \in \mathcal{E}} k_\tau * f_\theta$ is well behaved if $\mathcal{E}$ is a suitably sparse set. Interpolation with a trivial $L^1$ estimate yields an $L^p$ estimate on the sum. Conversely, if the set $\mathcal{E}$ is clustered, then $\sum_{(\tau,\theta) \in \mathcal{E}} k_\tau * f_\theta$ will be concentrated on only a few annuli, and so we can also get good $L^p$ estimates simply using pointwise estimates. But then $\| Tf \|_{L^p(\RR^d)} = \| \sum k_\tau * f_\theta \|_{L^p(\RR^d)}$ can be estimated, depending on whether a sparse or clustered part of the sum dominates.}}
%\vspace{0.5em}



The approach above fails as $t \to \pm 1/2$, since the Lax-H\"{o}rmander parametrix becomes degenerate past the injectivity radius of the manifold $S^d$. This is a common problem in the study of multipliers on manifolds. Sogge \cite{SoggeRieszMeans} introduced the method of reducing bounds past the injectivity radius to the study of $L^p \to L^2$ discrete restriction bounds, but such methods are not sharp enough for our purpose. I found an alternate method to reduce the required bounds for $|t| \geq 1/2$ to $L^p_x L^p_t$ localized estimates
 % since various heuristics related to the `Ehrenfast time' imply that such operators should still be amenable to study through oscillatory integrals.
%However, we cannot use this method in this problem, since the method initially involves using the estimate $\| a(P) f \|_{L^p(M)} \lesssim \| a(P) f \|_{L^2(M)}$, which is too inefficient to fully characterize the required $L^p$ estimates.
% Sobolev embedding heuristics suggest this inequality incurs a lost of $1/p - 1/2$ derivatives, which is fine for obtaining bounds under the assumption that the functions $a_j$ uniformly have $d(1/p - 1/2)$ derivatives in $L^2$, as in \cite{KimManifold}, but not for $\sup_j C_p(a_j) < \infty$, when the functions $a_j$ are only guaranteed to have $(d-1)(1/p - 1/2)$ derivatives in $L^p$.
for the wave equation on $S^d$, and thus to the local smoothing estimates of Lee and Seeger \cite{LeeSeeger}.

Using the methods above, for a general function $a(\rho) = \sum_{j \in \ZZ} a_j(\rho / 2^j)$, one can individually bound the $L^p$ norm of the operators $a_j(P/ R 2^j)$. To combine scales, we decompose a general input in $L^p(S^d)$ into $L^\infty$ atoms, à la the decompositions of Chang and Fefferman \cite{ChangFefferman}. By refining the tangency estimates we obtain for annuli of large radius one can then control the interactions of $a_R(P)$ applied to different atoms by a square function introduced by Peetre \cite{Peetre}, and these square functions are bounded on $L^p(S^d)$, from which we conclude that the operators $a_R(P)$ are uniformly bounded on $L^p(S^d)$.

\bibliography{BonnThesisExplanation}
\bibliographystyle{plain}

\end{document}