\documentclass[11pt]{article}

\usepackage[a4paper, margin=1in]{geometry}
\usepackage{mathptmx}

%\usepackage[text={16cm,24cm}]{geometry}

\usepackage{amssymb,amsmath,amsthm}
\usepackage{mathtools}
\usepackage{amsrefs}
\usepackage{comment}
\usepackage{amsthm}
 \usepackage{paralist}

     \setlength{\parskip}{0cm}
    \setlength{\parindent}{1em}

        \usepackage[compact]{titlesec}
    \titlespacing{\section}{0pt}{2ex}{2ex}
    \titlespacing{\subsection}{0pt}{1.5ex}{1.5ex}
    \titlespacing{\subsubsection}{0pt}{0.5ex}{0ex}

\newtheorem*{theorem}{Theorem}

\DeclareMathOperator{\QQ}{\mathbb{Q}}
\DeclareMathOperator{\ZZ}{\mathbb{Z}}
\DeclareMathOperator{\RR}{\mathbb{R}}
\DeclareMathOperator{\NN}{\mathbb{N}}
\DeclareMathOperator{\HH}{\mathbb{H}}
\DeclareMathOperator{\BB}{\mathbb{B}}
\DeclareMathOperator{\CC}{\mathbb{C}}
\DeclareMathOperator{\AB}{\mathbb{A}}
\DeclareMathOperator{\PP}{\mathbb{P}}
\DeclareMathOperator{\MM}{\mathbb{M}}
\DeclareMathOperator{\VV}{\mathbb{V}}
\DeclareMathOperator{\TT}{\mathbb{T}}
\DeclareMathOperator{\LL}{\mathcal{L}}
\DeclareMathOperator{\DD}{\mathcal{D}}
\DeclareMathOperator{\SW}{\mathcal{S}}
\DeclareMathOperator{\EC}{\mathcal{E}}
\DeclareMathOperator{\AC}{\mathcal{A}}

\title{\vspace{-2em}Research Statement}
\author{Jacob Denson}
\date{}

\begin{document}

\maketitle

\section{Favourite Theorem and Why}

I think it's been 4 years since you last asked me, and the reasons might be different this time, but I do still feel like I have the same answer Stein-Tomas theorem is really the gold standard. It's a deep theorem, in the sense that it
%
\begin{itemize}
    \item Suggests further improvements and forms the basis of future thought on other problems (suggests the restriction conjecture, as well as other analogues related to curvature: During my postdoc I learned about the discrete variant on manifolds, and the version for fractal sets due to Mitsis and Bak-Seeger).

    \item Has deep consequences: Strichartz inequalities, Boundedness of Bochner-Riesz, Roth's Theorem in positive upper density subsets of the primes (Green).
\end{itemize}
%
If I had to pick a second favourite, something completely new I learned during my PhD, it would have to be Andreas's result with Heo and Nasarov on Radial Multipliers:
%
\begin{itemize}
    \item It's fundamental.
    \item It's underappreciated as a result, yet seems to lie parallel to several current conversations in the field. I think I prefer these types of problems to problems that many people are working on at once, problems that might not be worked on by other people unless I try and attack them.
    \item It also has analogues in other settings (compact manifolds), curvature arises as well.
    \item It's surprising: such a simple condition characterizes Lp boundedness, which has remained elusive for so long in the field.
\end{itemize}

\section{Why is my work relevant}

Researchers in spectral geometry are interested in transference principles and $L^p$ interactions of eigenfunctions. The result I proved is the first transference principle to go from $R^d$ to a manifold. Harmonic analysts are interested in the behaviour of multipliers, and my result provides a characterization of multipliers, which answers an old question. Recently, there's been lots of interest recently on variable-coefficient analogues of results on $R^d$.

\subsection{Proof of Mitsis}

Perform a dyadic decomposition $\widehat{\mu} = \sum k_j$. Then we get a bound for $k_j * f$ from $L^1 -> L^\infty$ bounds, using the Fourier decay of $\mu$. And we get an $L^2 -> L^2$ bound because we've localized in space to a scale $2^j$, and so the Fourier transform of $k_j$ is the convolution of $\mu$ with something localized to a scale $2^{-j}$, and the Frostman ball condition helps us to obtain a bound on the $L^\infty$ norm of $\widehat{k}_j$. Then we interpolate to get a bound that is summable in $j$ for $p$ appropriately close to $1$, and this gives $L^p \to L^{p'}$ boundedness.

\subsection{Discrete Restriction Theorem of Sogge}

Use Tauberian theorems to switch between indicator function and $\chi$. To analyze the operator
%
\[ \chi(\lambda \leq \sqrt{-\Delta} \leq \lambda + 1), \]
%
write the operator as the average of the Fourier transform of $\chi(\lambda \leq \cdot \leq \lambda + 1)$ averaged against the wave propogators. Since the Fourier transform of a smooth indicator function is rapidly decaying outside times $|t| \ll 1$ by uncertainty principle, and oscillating at a scale proportional to the endpoints of the interval. We need only analyze these times, and we switch to a parametrix for the wave operator to do so. But integrating in $t$ gives rapid decay unless the frequency variable is localized to $|\xi| \sim \lambda$. Once localized, we can rescale and use stationary phase in polar coordinates to reduce to an integral with a phase given by the distance function between points. And this phase satisfies the Carleson-Sjolin condition, which allows us to use oscillatory integral estimates to obtain the required estimates in the Stein-Tomas range. To prove Carleson-Sjolin, we freeze one of the output variables to write the operator as $T = \int T_t$, then prove $TT^*$ estimates on $T_t T_s^*$, and then use Hardy-Littlewood-Sobolev to obtain the required bounds.

\section{Research Projects}

\subsection{Radial Multipliers}

The main problem I want to work on are methods to extend best known results on radial multipliers. I didn't get so much of a chance to work on these problems while I was doing my PhD - I was mainly concentrating on the manifold analogues of the results, so I put several possible ideas in my back pocket that might lead to improved results to work on later:

\begin{itemize}
    \item Zahl (2022) used Lie-Sphere geometry to translate problems about sphere tangencies into problems about incidences of complex lines in the Heisenberg group (two lines are incident if and only if the spheres are tangent). Lines on the Heisenberg group satisfy the Wolff axioms, and so it's possible to get some results here for 'sparse'

    First we consider Lie sphere geometry - each sphere in $R^d$ is mapped to a point on a quadric surface $Q$ in $PR^{d+2}$, where $PR^{d+2}$ is equipped with a bilinear form of signature 2 negative eigenvalues, such that two points are orthogonal with respect to this bilinear form precisely when the spheres are tangent.

    Then in $R^3$, there's an additional trick, where we can map points on $Q$ to complex lines in the Heisenberg group $H = \{ (x,y,z) \in C^3: \text{Im}(z) = \text{Im}(xy') \}$, such that the two lines are incident precisely when the points in Q are orthogonal. 

    So now we can ask whether we can get good incidence bounds for sparse sums of lines, which is a much more widely studied problem in incidence geometry.

    \[ L((a,b,c,d,e,f),(a',b',c',d',e',f')) = 2(bb' + cc' + dd' - ff') - ae' - a'e. \]

    \[ Q = \{ x \in \PP \RR^5: L(x,x) = 0 \}. \]

    \[ [a:b:c:d:e:f] \mapsto \text{Sphere with center $(b/a,c/a,d/a)$ and oriented radius $f/a$} \]

    Then consider the map $\phi: PR^5 \to PC^5$ given by
    %
    \[ \phi[a:b:c:d:e:f] = [a: d+ f: -b+ic: -e: d - f: -b - ic]. \]
    %
    and we use the latter quantities as Pl\"{u}cker coordinates of lines on the Heisenberg group
    %
    \[ \mathbf{H}^3 = \{ (x,y,z) \in C^3: \text{Im}(z) = \text{Im}(x\overline{y}) \}. \]

    \item More promising is work on methods related to weighted extension. The sparsity condition in the multiplier conditions is '1 dimensional', in the sense that the number of annuli with radii in an interval of length $L$ and a ball of radius $L$ is proportional to $L$ to the power of $1$. If we consider the manifold approach, 
    %
    \[ \left| \int_{R^d} \int_1^\infty w(x,t) e^{2 \pi i t |D|}(t,x,y) f(y)\; dt\; dx\; dy \right| \lesssim u^s \| f \|_{L^2} \{ \text{Bounds on $w$} \}, \]
    %
    then this really is an extension operator. Pramanik-Zahl-Wang also consider more general estimates for curve tangencies related to cinematic curvature, which could help prove small time estimates for the radial-multiplier conjecture beyond the Tomas-Stein range.

    \item Another possible approach is to find a way to 'dualize' the estimates of Heo-Nasarov-Seeger, and prove results when $p > 2$ rather than when $p < 2$ (both approaches should be equivalent since a multiplier is bounded in $L^p$ if and only if it is bounded in $L^{p'}$). Results for $p < 2$ seems to suggest incidence type bounds, whereas results for $p > 2$ tend to involve more broad-narrow analysis / multilinear-to-linear arguments / decoupling type results. Beltran and Saari have a paper where they analyze variants of local smoothing by using broad-narrow analysis, as well as other results related to local smoothing / restriction theory due to Guth Wang Zhang, Ou and Wang, and Gao, Liu, Miao, and Xi. If broad-narrow analysis can yield local smoothing bounds, it's natural to investigate whether such analysis could also yield bounds for radial multipliers.
\end{itemize}

\section{Multipliers Assoicated With Periodic Geodesic Flow}

    This project relates back to my project on analogues for multipliers of elliptic operators with curvature on compact manifold, which is that in these endpoint estimates you have to worry a lot about the global geometry of the manifold you're dealing with (in particular, the large time behaviour of the geodesic flow on the manifold) that's really difficult to deal with except in certain simple situations.

    One of the main new ideas I obtained in my work is that one \emph{can} reduce the \emph{large time} behaviour of arbitrary $L^p$ bounded multipliers to the study of endpoint local smoothing type bounds on an interval of length $1$, with bounds that grow geometrically as the endpoints of the interval tend to infinity.

    For elliptic operators with integer eigenvalues, the wave operator is periodic, and so these bounds follow simply from endpoint local smoothing bounds for the wave equation (implied by small time results of my paper, but already proved by Lee and Seeger).

    The wave operator associated with the Laplace-Beltrami operator on the sphere is *not* periodic in time, and so these bounds are not so easy to obtain, but by propogation of singularities we can perturb the operator so that it does have integer eigenvalues, which gives the result in the original paper. Nonetheless, obtaining these local smoothing-type bounds would still imply new results for different types of rescaled multipliers.

    These is a path to obtaining such results on $S^d$, using the fact that for the Laplace-Beltrami operator, the wave propogator at time proportional to the circumference of the sphere is a pseudodifferential operator of order zero, and thus bounded in all $L^p$. If we were able to obtain uniform $L^p$ bounds for all integer times, then the endpoint local smoothing inequality of Lee and Seeger would give the required bounds, which would yield new results, and if that was possible, we might try more manifolds with periodic geodesic flow, or manifolds whose geodesic flow is integrable.

\section{$l^2$ decoupling related to randomness arguments}

    I think decoupling for fractals is a really interesting area because I think the 'correct' result is quite difficult to formulate: namely what the 'right' scale at which to decouple is. For decoupling on neighborhoods surfaces you split things into caps, which have the right square root cancellation properties because of the curvature of the manifold. But what is the scale at which we should decouple functions supported on neighborhoods of fractal sets on the line? It shouldn't be into intervals because that doesn't have enough 'transversality' / 'curvature' for you to obtain the right bounds, and for e.g. on a $3^{-j}$ neighborhood of a random middle thirds Cantor sets we cannot get subpolynomial decoupling unless we decouple at a scale bigger than $3^{-aj}$ for a fixed constant a depending on the dimension of the set we construct.

    Why do I think I might be able to help with the problem. I looked into high-dimensional probability a lot when working on Salem Sets, in particular, a set of techniques for bounding the tails of the supremum of a set of random variables $\sup_{t in T} X_t$. If the random variables $\{ X_t \}$ are independent, the supremum is likely very large, but if the random variables are correlated, then we can often do much better: how much better depends on the geometry of the metric space $d(s,t) = | X_t - X_s |_{L^2}$, certain entropy bounds somewhat related to the fractal dimension of the metric space (generic chaining / majorizing measures help quantify this).
    %
    \[ \gamma_\alpha(X) = \inf \sup_{t in T} \sum_n 2^{n/\alpha} \text{Diam}(A_n(t)) \]
    %
    where $A_n(t)$ denotes the unique element of the partition of T contianing t at each scale.

    These type of methods were used in Bourgain's $\Lambda(p)$ paper, and also in Talagrand's alternate proof. We can likely use similar methods in the theory of decoupling inequalities, by letting T range over all functions $f = \sum f_j$ with $\sum | f_j |_{L^p}^2 = 1$, and letting $X_f$ be the $L^p$ norm of $\sum f_j$. Then $\sup_{f in T} X_f$ is precisely the decoupling constant $D_p$ at the scale we are looking at in the problem.

    Alternatively, in order to prove that multilinear Kakeya / restriction / decoupling bounds hold with high probability, let $X_f$ denote the left hand side of the resulting inequalities, where f ranges over all choices of inputs for which the right hand side has size 1.
















\section{Proof of Heo-Nazarov-Seeger and Cladek}

\begin{itemize}
    \item We combine Fourier analytic properties with geometric properties.

    \item We don't expect to get any cancellation when integrating different annuli that are close to together in position and radii (no cancellation in L2), so we just use the triangle inequality to eliminate interactions between such radii, and thus reduce to a discretized version of the inequality, i.e. Lp bounds on weighted sums of annuli of various positions and radii, which is really a pth root cancellation bound for the Lp norm of the annuli.

    \item Then use real interpolation to reduce to a restricted bound, i.e. so that we're just literally dealing with sums of different annuli rather than weighted sums.

    \item Now exploit a decomposition between clustered annuli and sparse annuli, in a way analogous to the decomposition between bounded functions and mean-zero functions in the Calderon-Zygmund decomposition. Something similar occurs in Tao's work on Bochner-Riesz in the late 90s.

    \item We get L1 bounds for the singular integral applied to the mean-zero part of the decomposition, we get L2 bounds for the singular integral applied to the bounded part of the decomposition, and then interpolation and optimization gives you the Lp bound you're looking for.

    \item In our case, the sparse annuli have the good cancellation properties, so satisfy good L2 estimates. Clustered annuli are concentrated on a small set, so satisfy good L0/L1 estimates, and then we interpolate the two bounds to obtain the Lp estimates we need.

    \item What is the key incidence argument needed to obtain the L2 estimates? We split interactions into three cases: interactions of annuli with incomparable radii, interactions of annuli with 'slightly incomparable' radii, and interactions of annuli with comparable radii. All of the analysis is 'pinned', in the sense that we fix one annulus, compare interactions with other annuli, and then sum trivially over the pinned annulus. We don't consider the relationship at all with what points can interact with two different pinned annuli.

    \begin{itemize}
        \item If two incomparable annuli or slightly incomparable radii interact, then the center of the annuli with the smaller radius is contained in a slight thickening of the annuli with the larger radius, and since this slight thickening has small volume, by sparsity there can't be too many centers in that thickening.

        \item Conversely, if we're looking at annuli of comparable size, we first use Cauchy-Schwartz to forget about interactions of annuli that have different radii, and reduce our analysis to annuli with *exactly the same radii*.

        \item If we're looking at annuli of comparable size, we first use Cauchy-Schwartz to throw away the analysis of annuli with slightly different radii. And then for annuli with exactly the same radii, we can reduce our analysis to a simple convolution inequality which allows us to take Fourier transforms and use the Fourier decay of the surface measure on the sphere.
    \end{itemize}
    %
    \item Break down our input function into $L^\infty$ atoms supported on cubes that are at least as large as the inverse of the frequency scale they are supported on. The atoms are summable in $L^2$, but we need to get that to be summable in $L^p$, which we do by breaking down the behaviour of the multiplier into short and long range interactions. The short range interactions only interact with neighboring atoms, so we get orthogonality here. The long range interactions we need to do more, and provided we get a $2^{-l\varepsilon}$ improvement because as you start to use the density argument on vary large radii, one can forget about the sparsity of interactions and just use the trivial bound on the number of intersections.

     Atomic decomposition in $L^p$, we just need to obtain a slight amount of cancellation for long range interactions 
\end{itemize}
%
If we prove the $L^2$ square root cancellation result with a $u^s$ on the right hand side, then the interpolation argument above proves the radial multiplier conjecture for $|1/p - 1/2| > s$. So proving the result for all $s > 1/2d$ would completely resolve the conjecture.

In order to improve Heo-Nazarov-Seeger's bound and obtain a result when $d = 3$, Cladek has to deal with two problems: the analysis of annuli with radii at different dyadic scales is unproblematic when $d \geq 4$ in the full range of the radial multiplier problem, but *not* when $d = 3$ (up to a logarithm) and *certainly not* when $d = 2$ (where no results are known). So this part must be modified. And also the analysis of slightly incomparable and comparable radii must be made more efficient.

\begin{itemize}
    \item The main trick in both is to take greater advantage of the geometric information in the problem. In particular, to analyze the interactions of two pinned annuli with other annuli. To do this, Laura exploits the tensor structure of the bounds we need to prove, which implies an additional dichotomy to the problem: in any given region, there may be different distinct centers for annuli, or many distinct radii, but not both. The 'pinned quantities' that we sum over are precisely the things we have few of:

    \begin{itemize}
        \item If we have many radii, then we have few centers, and we can fix all possible pairs of centers, and then for any fixed radii of the first, the second is essentially fixed for any possible interaction.

        \item If the set has many centers, then we have to analyze very few rather than summing over each pinned annulus separately, Cladek instead switches to studying the number of 'rich centers' for any given radius t, i.e. bounding, in a set of sparse centers of annuli, the total number of points that lie a fixed distance t from many other centers. This is in turn bounded by proving that the intersection of three annuli of a fixed radius with distances close together cannot be large, and thus cannot contain many points in our sparse set of centers.
    \end{itemize}

    \item Cladek's Result Likely doesn't work in 2D at all, because we obtain very little gain by considering intersections of 3 or more annuli vs intersections of 2 annuli.
\end{itemize}

If Jonathan asks, can likely be done on manifold, especially with the inner product estimates I derived. Only question would be bounding the intersection of different geodesic spheres. I've identified Lemma $3.1$ of $Kolasa and Wolff 1999$ which gives an analogue of some of the bounds Cladek needs for intersections of spheres for more general cinematic curvature settings, but I haven't completed calculated this in general.

There are reasons to believe the result doesn't hold in all dimensions and on all manifolds

Guo, Wang, Zhang 2022: Whenever Bourgain's Condition Fails, $L^\infty \to L^q$ boundedness fails for $q > 2d/(d-1)$, extending a 3D result of Bourgain 1991.

    Thus restriction-type oscillatory integral estimates fail on manifolds with non sectional curvature.

        Oscillatory Integral operator with phase $|x - y|$, appropriately scaled, gives the restriction conjecture.

        Analogue on manifold is

                $T_R f(x) = \int e^{i R d(x,y)} a(x,y) f(y) dy$

                $|T_R| <<_eps R^{-d/p + eps} for all eps > 0$.

            Don't thiiiink anyone has gone through the proof that Bochner-Reisz implies restriction on a manifold - Wisewell has done restriction implies Kakeya. Tao's original proof follows from the observation that Bochner Riesz looks like restriction far away from the support of the original function. After throwing away remainder terms from applying the Lax-parametrix one can use oscillatory integral calculations from my paper to eliminate the angular variables and reduce to oscillatory integrals involving the distance function.

    Main innovations of my paper: analysis of non-degeneracy of the critical points of the phase of the parametrix for the wave operator in polar coordinates, and the large time analysis reduction to local smoothing estimates.

    I'm not sure if it necessary results in better estimates vs. a second dyadic decomposition, since I'm not quite able to completely understand what's going on under the hood of Kim's arguments. But it certainly simplifies the proof considerably. You can proceed directly with strong Lp estimates rather than exceptional set estimates. It plays more nicely with other techniques that will come to play in my next paper adapting the arguments of (Heo, Nazarov, Seeger, The Sequel to the original Heo,Nazarov,Seeger paper 2011), which allow for a more black box method for putting together dyadic scales for compact multipliers, and you can see the geometry of the problem way more clearly. I think it would be harder to adapt these arguments by a second dyadic decomposition.

\section{Advantage and Disadvantage of Stationary Set Principle}

Advantage is that it's very robust, and depends only on the 'complexity' of the phase. Reduces problem to investigating thickened level set estimates.

Doesn't Exploit Localization about critical points on the phase at all. For sublevel sets this currently is only known for convex phases of finite type.

Convolution inequality - Perhaps maximal functions for such convolutions / maximal average properties over balls.

Proof is very simple, exploits monotonicity of a phase that has bounded complexity. Then just use the step 1 of Van der Corput type bound (recall one derivative has monotonicity).

\section{Extra Projects}

I have several other projects that I've thought of that might be of - I also have some ideas on pursuing nodal domains 

\begin{itemize}
    \item Extremizers (e.g. in Bilbao)
    \item More Multilinear Estimates
    \item Nodal Sets for Sublaplacians (Letrouit in Orsay)
\end{itemize}

I'd be interested in exploring $p$-adic analogues of nodal sets, multilinear methods,

\subsection{Papers I want to read in 2024}

    Bennett, Gutierez, Nakamura, Oliveira 
        A phase-space approach to weighted Fourier extension inequalities

    Microlocal Analysts in France
        In particular, Letrouit Nodal Set Geometry For Subelliptic Operators

    Time Frequency Analysis

    Huge 6 Author Paper Chen, Gan, Guo, Hickman, Illiopoulou, Wright
        Sublevel sets of analytic functions

\subsection{Questions To Ask Committee}

\begin{itemize}
    \item Do you have particular questions you'd want me to work on?
    \item Does it matter what date I should defend by?
    \item Unlike when I was applying for my PhD Degree, Edinburgh is my number one choice for a postdoc position so I'd be willing to accept if offered the position.
    \item Would there be opportunities to teach seminars / classes. I assume there is no teaching load for the position based on the application but I thought.
    \item Is it easy to find housing? Is there postdoctoral housing available from the university?
\end{itemize}


Defence Date: May, Graduate in May
- Finish as a student as soon as I submit

- Inquire about teaching load.
- Would it matter when I defend my Thesis.
- Unlike my PhD, I can accept right now.

\end{document}

Such operators initially arose from the study of the classical partial differential equations in physics, and continue to have applications in areas as diverse as partial differential equations, mathematical physics, number theory, and ergodic theory. Every translation invariant operator on $\RR^d$ is a Fourier multiplier operator, and every rotation invariant operator on $S^d$ is a multiplier of the spherical harmonic expansion on $S^d$.

any such operator $T$, we can associate a function $m: \RR^n \to \CC$, known as the \emph{symbol} of $T$, such that for any function $f$, the Fourier transform of $Tf$ obeys the relation $\widehat{Tf} = m \widehat{f}$; thus translation invariant operators are also called \emph{Fourier multiplier operators}.

Understanding the boundedness of Fourier multiplier operators in an $L^p$ norm for $p \neq 2$ underpins any subtle understanding of the Fourier transform. Plane waves oscillating in different directions and with different frequencies are orthogonal to one another, and thus do not interact with one another significantly in terms of the $L^2$ norm, as justified by Bessel's inequality. But plane waves can interact with one another in the $L^p$ norm for $p \neq 2$, and so understanding $L^p$ bounds for Fourier multipliers indicate when this interaction is significant or insignificant. Similarily, spherical harmonics of different degrees on $S^d$ are orthogonal to one another, but studying the $L^p$ bounds of multipliers of the Laplacian on the sphere is crucial to understand when the interactions of different spherical harmonics are significant or not.

The general study of the characterizations of $L^p$ boundedness for the Fourier multipliers was initiated in the 1960s. Mathematicians quickly found simple necessary and sufficient conditions that ensure Fourier multipliers are bounded on $L^1(\RR^d)$, $L^2(\RR^d)$, and $L^\infty(\RR^d)$. But the problem of finding necessary and sufficient conditions for boundedness in $L^p(\RR^d)$ for any other exponent proved impenetrable. Indeed, many interesting problems about the boundedness of \emph{specific} Fourier multipliers, such as the Bochner-Riesz conjecture, remain largely unsolved today.

Thus it came as a surprise in the past decade when results emerged proving necessary and sufficient conditions for \emph{radial} Fourier multipliers to be bounded on $L^p(\RR^d)$. First came the result of BLAH, which gave a necessary and sufficient criteria for bounds of the form $\| Tf \|_{L^p(\RR^d)} \lesssim \| f \|_{L^p(\RR^d)}$ to hold uniformly over \emph{radial functions} $f$, for $|1/p - 1/2| > 1/2d$. An optimist might think this same condition causes the bound to hold uniformly over \emph{all functions} $f$ in the range above, a statement we call the \emph{radial multiplier conjecture}. We now know, by the results of BLAH and BLAH, that the radial multiplier conjecture holds when $d > 4$ and $|1/p - 1/2| > 1/(d-1)$, when $d = 4$ and $|1/p - 1/2| > 11/36$, and when $d = 3$ and $|1/p - 1/2| > 11/26$. But the radial multiplier conjecture has not yet been completely solved in any dimension $d$, and no bounds are known at all when $d = 2$.

The natural analogue of the study of radial multipliers on $\RR^d$ is the study of multipliers of a Laplace-Beltrami operator on a Riemannian manifold. The natural analogue of the study of quasiradial multipliers on $\RR^d$ is the study of multipliers of an operator associated with a \emph{Finsler geometry} on the manifold.