%\title{Example letter using the newlfm LaTeX package}
%
% See http://texblog.org/2013/11/11/latexs-alternative-letter-class-newlfm/
% and http://www.ctan.org/tex-archive/macros/latex/contrib/newlfm
% for more information.
%
\documentclass[11pt,stdletter,orderfromtodate,sigleft]{newlfm}

\usepackage{etoolbox}

\usepackage{amssymb,amsmath,amsthm}
\usepackage{mathtools}
\usepackage{comment}
 \usepackage{paralist}

\DeclareMathOperator{\QQ}{\mathbb{Q}}
\DeclareMathOperator{\ZZ}{\mathbb{Z}}
\DeclareMathOperator{\RR}{\mathbb{R}}
\DeclareMathOperator{\NN}{\mathbb{N}}
\DeclareMathOperator{\HH}{\mathbb{H}}
\DeclareMathOperator{\BB}{\mathbb{B}}
\DeclareMathOperator{\CC}{\mathbb{C}}
\DeclareMathOperator{\AB}{\mathbb{A}}
\DeclareMathOperator{\PP}{\mathbb{P}}
\DeclareMathOperator{\MM}{\mathbb{M}}
\DeclareMathOperator{\VV}{\mathbb{V}}
\DeclareMathOperator{\TT}{\mathbb{T}}
\DeclareMathOperator{\LL}{\mathcal{L}}
\DeclareMathOperator{\DD}{\mathcal{D}}
\DeclareMathOperator{\SW}{\mathcal{S}}
\DeclareMathOperator{\EC}{\mathcal{E}}
\DeclareMathOperator{\AC}{\mathcal{A}}

%% Patch from https://tex.stackexchange.com/a/395529/226 to address newlfm bug
\makeatletter
\patchcmd{\@zfancyhead}{\fancy@reset}{\f@nch@reset}{}{}
\patchcmd{\@set@em@up}{\f@ncyolh}{\f@nch@olh}{}{}
\patchcmd{\@set@em@up}{\f@ncyolh}{\f@nch@olh}{}{}
\patchcmd{\@set@em@up}{\f@ncyorh}{\f@nch@orh}{}{}
\makeatother
 
\newlfmP{Headlinewd=0pt,Footlinewd=0pt}
 
\namefrom{}

\dateset{}
 
\greetto{Dear Members of the Selection Committee for Postdoctoral Positions at Universidad de Los Andes,}

\closeline{}

\begin{document}

\begin{newlfm}

I am currently a PhD student in Mathematics at the University of Wisconsin-Madison, and expect to graduate in the coming spring. My Research Interests lie primarily in harmonic analysis, spectral geometry, geometric measure theory, and the study of nonlinear wave equations. My thesis problem focuses on studying spectral multipliers on compact manifolds using methods of Fourier integral operators, which I am working on under the supervision of Professor Andreas Seeger.

If awarded the H.C. Wang Professorship, I hope to work with Professor Camil Muscalu	. Professor Muscalu is an expert in multilinear harmonic analysis, including work related to the restriction conjecture, and I am greatly interested in seeing whether multilinear techniques can be used to extend current best known results about endpoint $L^p$ bounds for radial Fourier multipliers, a problem closely related to restriction problems. More information about the prospective research alluded to here can be found in my research statement.

Thank you for your consideration,\\\\
Jacob Denson

\end{newlfm}
\end{document}