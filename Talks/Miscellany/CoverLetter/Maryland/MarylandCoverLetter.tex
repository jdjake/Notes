%\title{Example letter using the newlfm LaTeX package}
%
% See http://texblog.org/2013/11/11/latexs-alternative-letter-class-newlfm/
% and http://www.ctan.org/tex-archive/macros/latex/contrib/newlfm
% for more information.
%
\documentclass[11pt,stdletter,orderfromtodate,sigleft]{newlfm}

\usepackage{etoolbox}

\usepackage{amssymb,amsmath,amsthm}
\usepackage{mathtools}
\usepackage{comment}
 \usepackage{paralist}

\DeclareMathOperator{\QQ}{\mathbb{Q}}
\DeclareMathOperator{\ZZ}{\mathbb{Z}}
\DeclareMathOperator{\RR}{\mathbb{R}}
\DeclareMathOperator{\NN}{\mathbb{N}}
\DeclareMathOperator{\HH}{\mathbb{H}}
\DeclareMathOperator{\BB}{\mathbb{B}}
\DeclareMathOperator{\CC}{\mathbb{C}}
\DeclareMathOperator{\AB}{\mathbb{A}}
\DeclareMathOperator{\PP}{\mathbb{P}}
\DeclareMathOperator{\MM}{\mathbb{M}}
\DeclareMathOperator{\VV}{\mathbb{V}}
\DeclareMathOperator{\TT}{\mathbb{T}}
\DeclareMathOperator{\LL}{\mathcal{L}}
\DeclareMathOperator{\DD}{\mathcal{D}}
\DeclareMathOperator{\SW}{\mathcal{S}}
\DeclareMathOperator{\EC}{\mathcal{E}}
\DeclareMathOperator{\AC}{\mathcal{A}}

%% Patch from https://tex.stackexchange.com/a/395529/226 to address newlfm bug
\makeatletter
\patchcmd{\@zfancyhead}{\fancy@reset}{\f@nch@reset}{}{}
\patchcmd{\@set@em@up}{\f@ncyolh}{\f@nch@olh}{}{}
\patchcmd{\@set@em@up}{\f@ncyolh}{\f@nch@olh}{}{}
\patchcmd{\@set@em@up}{\f@ncyorh}{\f@nch@orh}{}{}
\makeatother
 
\newlfmP{Headlinewd=0pt,Footlinewd=0pt}
 
\namefrom{}

\dateset{}
 
\greetto{Dear Members of the Michael Brin and Sergei Novikov Fellowship Selection Commitee,}

\closeline{}

\begin{document}

\begin{newlfm}

I am currently a PhD student in Mathematics at the University of Wisconsin-Madison, and expect to graduate in the coming spring. My research interests lie primarily in the fields of harmonic analysis and geometric measure theory, though I am also beginning to explore problems related to nonlinear wave equations. In your department, I would be interested in working with Professor Balan and Professor Benedetto. My thesis problem focuses on studying spectral multipliers on compact manifolds using methods of Fourier integral operators, which I am working on under the supervision of Professor Andreas Seeger.

Thank you for your consideration,\\\\
Jacob Denson

\end{newlfm}
\end{document}