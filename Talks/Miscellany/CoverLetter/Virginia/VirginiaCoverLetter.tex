%\title{Example letter using the newlfm LaTeX package}
%
% See http://texblog.org/2013/11/11/latexs-alternative-letter-class-newlfm/
% and http://www.ctan.org/tex-archive/macros/latex/contrib/newlfm
% for more information.
%
\documentclass[11pt,stdletter,orderfromtodate,sigleft]{newlfm}

\usepackage{etoolbox}

\usepackage{amssymb,amsmath,amsthm}
\usepackage{mathtools}
\usepackage{comment}
 \usepackage{paralist}

\DeclareMathOperator{\QQ}{\mathbb{Q}}
\DeclareMathOperator{\ZZ}{\mathbb{Z}}
\DeclareMathOperator{\RR}{\mathbb{R}}
\DeclareMathOperator{\NN}{\mathbb{N}}
\DeclareMathOperator{\HH}{\mathbb{H}}
\DeclareMathOperator{\BB}{\mathbb{B}}
\DeclareMathOperator{\CC}{\mathbb{C}}
\DeclareMathOperator{\AB}{\mathbb{A}}
\DeclareMathOperator{\PP}{\mathbb{P}}
\DeclareMathOperator{\MM}{\mathbb{M}}
\DeclareMathOperator{\VV}{\mathbb{V}}
\DeclareMathOperator{\TT}{\mathbb{T}}
\DeclareMathOperator{\LL}{\mathcal{L}}
\DeclareMathOperator{\DD}{\mathcal{D}}
\DeclareMathOperator{\SW}{\mathcal{S}}
\DeclareMathOperator{\EC}{\mathcal{E}}
\DeclareMathOperator{\AC}{\mathcal{A}}

%% Patch from https://tex.stackexchange.com/a/395529/226 to address newlfm bug
\makeatletter
\patchcmd{\@zfancyhead}{\fancy@reset}{\f@nch@reset}{}{}
\patchcmd{\@set@em@up}{\f@ncyolh}{\f@nch@olh}{}{}
\patchcmd{\@set@em@up}{\f@ncyolh}{\f@nch@olh}{}{}
\patchcmd{\@set@em@up}{\f@ncyorh}{\f@nch@orh}{}{}
\makeatother
 
\newlfmP{Headlinewd=0pt,Footlinewd=0pt}
 
\namefrom{}

\dateset{}
 
\greetto{Dear Members of the Selection Committee for the Whyburn Research Associate and Mary Ann Pitts Research Associate and Lecturer Postdoctoral Positions,}

\closeline{}

\begin{document}

\begin{newlfm}

I am currently a PhD student in Mathematics at the University of Wisconsin-Madison, advised by Dr. Andreas Seeger, and expect to graduate in the coming spring. My {\bf Research Interests} lie primarily in harmonic analysis, geometric measure theory, and probability theory, though I am also beginning to explore problems related to nonlinear wave equations. More details about passed and prospective research projects can be found in my Research Statement.

In my {\bf Future Research}, I hope to further my thesis work on spectral multipliers on compact manifolds, as well as extending related bounds on radial multipliers to the multilinear setting, and to continue to explore interactions between probability theory and Fourier analysis. Professor Yen Do shares my interests both in probability theory and harmonic analysis, and I have in mind several {\bf Potential Collaborations} in mind if I were to obtain a position at the University of Virginia, such as the study of decoupling inequalities on random fractal sets. Dr. Do's expertise in time frequency analysis should also prove very helping in studying multilinear radial multipliers. And I am also interested in exploring how new methods using o-minimality coming from real algebraic geometry can be used to obtain sharp decay for highly degenerate multivariate oscillatory integrals. More details about these research projects can be found in my Research Statement.

{\bf Teaching Experience} I have lectured for one course during my PhD degree, a class on Preparatory Algebra. In addition, I was a lecturer for the Summer Enhancement Program for two years, which is a summer course to help incoming graduate students prepare for their qualifying exam. I have also been a teaching assistant for a variety of mathematics courses during my MSc and PhD degrees, for single-variable and multi-variable calculus, linear algebra, discrete mathematics, partial differential equations, and probability theory, and an undergraduate teaching assistant for an introductory algorithms class for honors computing science students.

Thank you for your consideration,\\\\
Jacob Denson

\end{newlfm}
\end{document}