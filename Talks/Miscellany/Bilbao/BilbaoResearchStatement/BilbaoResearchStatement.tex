\documentclass[11pt]{article}

\usepackage[a4paper, margin=1in]{geometry}
\usepackage{mathptmx}

\usepackage{hyperref}

\usepackage{filecontents}
\begin{filecontents}{mybooknumeric.bbx}
\ProvidesFile{mybooknumeric.bbx}

\RequireBibliographyStyle{standard}
\RequireBibliographyStyle{numeric}
%
\DeclareBibliographyOption[boolean]{dashed}[true]{%
  \ifstrequal{#1}{true}
    {\ExecuteBibliographyOptions{pagetracker}%
     \renewbibmacro*{bbx:savehash}{\savefield{fullhash}{\bbx@lasthash}}}
    {\renewbibmacro*{bbx:savehash}{}}}
%
\newbibmacro*{bbx:savehash}{%
  \savefield{fullhash}{\bbx@lasthash}}

\renewbibmacro*{author}{%
  \ifboolexpr{
    test \ifuseauthor
    and
    not test {\ifnameundef{author}}
  }
  {\usebibmacro{bbx:dashcheck}
    {\bibnamedash}
    {\printnames{author}%
      \setunit{\addcomma\space}%
      \usebibmacro{bbx:savehash}}%
    \usebibmacro{authorstrg}}
  {\global\undef\bbx@lasthash}}

\newbibmacro*{bbx:dashcheck}[2]{%
  \ifboolexpr{
    test {\iffieldequals{fullhash}{\bbx@lasthash}}
    and
    not test \iffirstonpage
    % NOTE: the follow only matters if you have defined and set up the boolean `bbx@inset` (which is defined in authortitle.bbx).
    % and
    % (
    %   not bool {bbx@inset}
    %   or
    %   test {\iffieldequalstr{entrysetcount}{1}}
    % )
  }
  {#1}
  {#2}}

\end{filecontents}

\usepackage[
backend=biber,
bibstyle=mybooknumeric,
dashed=true,
citestyle=numeric,
firstinits=true
]{biblatex}

\renewbibmacro{in:}{}

\addbibresource{ResearchStatement.bib}
\renewcommand*{\bibnamedash}{\rule{3em}{0.4pt}\hspace*{.16667em}\addcomma\addspace}

\DeclareFieldFormat*{title}{#1}

%\usepackage[text={16cm,24cm}]{geometry}

\usepackage{amssymb,amsmath,amsthm}
\usepackage{mathtools}
\usepackage{comment}
 \usepackage{paralist}

     \setlength{\parskip}{0cm}
    \setlength{\parindent}{1em}

        \usepackage[compact]{titlesec}
    \titlespacing{\section}{0pt}{1ex}{1ex}
    \titlespacing{\subsection}{0pt}{0.8ex}{0.8ex}
    \titlespacing{\subsubsection}{0pt}{0.3ex}{0ex}

\theoremstyle{definition}
\newtheorem{theorem}{Theorem}

\DeclareMathOperator{\QQ}{\mathbb{Q}}
\DeclareMathOperator{\ZZ}{\mathbb{Z}}
\DeclareMathOperator{\RR}{\mathbb{R}}
\DeclareMathOperator{\NN}{\mathbb{N}}
\DeclareMathOperator{\HH}{\mathbb{H}}
\DeclareMathOperator{\BB}{\mathbb{B}}
\DeclareMathOperator{\CC}{\mathbb{C}}
\DeclareMathOperator{\AB}{\mathbb{A}}
\DeclareMathOperator{\PP}{\mathbb{P}}
\DeclareMathOperator{\MM}{\mathbb{M}}
\DeclareMathOperator{\VV}{\mathbb{V}}
\DeclareMathOperator{\TT}{\mathbb{T}}
\DeclareMathOperator{\LL}{\mathcal{L}}
\DeclareMathOperator{\DD}{\mathcal{D}}
\DeclareMathOperator{\SW}{\mathcal{S}}
\DeclareMathOperator{\EC}{\mathcal{E}}
\DeclareMathOperator{\AC}{\mathcal{A}}

\title{\vspace{-2.5em}Research Proposal\vspace{-0.6em}}
\author{Jacob Denson}
\date{}

\begin{document}

\maketitle

\setlength{\abovedisplayskip}{2.5pt}
\setlength{\belowdisplayskip}{2.5pt}

\vspace{-1.8em}

I am an analyst studying problems using harmonic analysis, combinatorics, and probability theory. In this research statement, I briefly describe my prior research, then discuss several future problems of interest which are closely tied to the work of the analysis group in Bilbao.

\section*{Prior Research}

\subsection*{Relations Between Multipliers on the Sphere and $\RR^d$}

My thesis work studies connections between radial Fourier multipliers and multipliers of spherical harmonic expansions on the sphere, using the theory of Fourier integral operators. For any $a: [0,\infty) \to \CC$, define a Fourier multiplier $T_a$ on $\RR^d$ and a multiplier operator $S_a$ on $S^d$ by setting
%
\[ T_a f(x) = \int_{\RR^d} a ( |\xi| ) \widehat{f}(\xi) e^{2 \pi i \xi \cdot x}\; d \xi \quad\text{and}\quad S_a f = \sum\nolimits_{k = 0}^\infty a(k) H_k f. \]
%
Here $H_k$ is the orthogonal projection operator onto the space of degree $k$ spherical harmonics on $S^d$. %Every operator on $\RR^d$ commuting with rotations and translations is a Fourier multiplier operator, and every operator on $S^d$ commuting with rotations is a multiplier operator of the form above, explaining their ubiquity in mathematics and related areas. %In harmonic analysis, it has proved incredibly profitable to study the boundedness of multiplier operators with respect to various $L^p$ norms. It seems to be one of the few tractable ways of quantifying interactions between different waves, thus underpinning all deeper understandings of the Fourier transform and expansions in spherical harmonics.
Mitjagin \cite{Mitjagin} connected these classes of operators, showing $\| T_a \|_{L^p \to L^p} \lesssim \sup_R \| S_{a_R} \|_{L^p \to L^p}$ for all $p \in [1,\infty]$, where $a_R(\cdot) = a(\cdot/R)$ are dilates of $a$, giving a `transference principle' from $S^d$ to $\RR^d$. But the reverse inequality $\sup_R \| S_{a_R} \|_{L^p \to L^p} \lesssim \| T_a \|_{L^p \to L^p}$ remained unproved for all $p \neq 2$ for a half century. \emph{In \cite{DensonCharacterization}, I proved this inequality for a limited range of $p$ and compactly supported $a$:}

%There is some evidence that the boundedness of $T_a$ on $L^p(\RR^d)$ is connected to the uniform boundedness of the operators $\{ S_{a_R} : R > 0 \}$ on $L^p(S^d)$, where $a_R(\cdot) = a(\cdot / R)$ are dilates of $a$. Indeed, Mitjagin \cite{Mitjagin} proved $\| T_a \|_{L^p \to L^p} \lesssim \sup_R \| S_{a_R} \|_{L^p \to L^p}$ for all $p \in [1,\infty]$, a `transference principle' from $S^d$ to $\RR^d$. %The result is intuitive, since, very roughly speaking, one locally has $T_a = \lim_{R \to \infty} S_{a_R}$, because one can view the dilation of $a$ instead as a dilation of the metric on $S^d$, becoming flatter and flatter as $R \to \infty$. Mitjagin's result then follows by `taking operator norms on each side of the equation'.
%On the other hand, the reverse inequality 

\begin{theorem} \label{characterization}
    For $d \geq 4$ and $1/2 < |1/p - 1/2| < (d-1)^{-1}$, if $a(\rho) = 0$ for $\rho \not \in [1,2]$, then
    %
    \begin{equation} \label{secondinequality}
        \sup\nolimits_R \| S_{a_R} \|_{L^p \to L^p} \sim \left( \int_0^\infty ( 1 + |t| )^{(d-1)(1 - p/2)} | \widehat{a}(t) |^p\; dt \right)^{1/p}.
    \end{equation}
    %
    This implies that $\sup\nolimits_R \| S_{a_R} \|_{L^p \to L^p} \lesssim \| T_a \|_{L^p \to L^p}$.
\end{theorem}

Equation \eqref{secondinequality} of Theorem \ref{characterization} completely characterizes those $a$ whose operators $\{ S_{a_R} \}$ are uniformly bounded on $L^p$, \emph{and is the first such characterization for any $p \neq 2$.} The result is obtained by writing $S_{a_R} = \int_{-\infty}^\infty \widehat{a}_R(t) W_t\; dt$, where the operators $\{ W_t \}$ are propagators for a half-wave equation on $S^d$. We can then exploit methods related to local smoothing for the half-wave equation. In a follow up paper in preparation, I extend Theorem 1 to non-compact $a$ using atomic decomposition methods on $L^p(S^d)$% and Littlewood-Paley type square function estimates of Peetre.% \cite{Peetre}.

\subsection*{Fractal Sets Avoiding Configurations}

How large must the fractal dimension of a set $X \subset \RR^d$ be before it contains structured subsets, such as four points forming a parallelogram? The fractal dimension studied is often the Hausdorff dimension, but Fourier dimension also arises as a subtle refinement, providing greater structural information about $X$ by controlling the Fourier decay of measures supported on $X$. A model problem, given $f: (\RR^d)^n \to \RR^m$, is to determine the dimension a set $X$ must have to guarantee there are distinct $x_1,\dots,x_n \in X$ so that $f(x_1,\dots,x_n) = 0$. My work so far has focused on obtaining lower bounds for this problem, constructing sets $X$ with large dimension such that for all distinct $x_1,\dots,x_n \in X$, $f(x_1,\dots,x_n) \neq 0$. We say such sets `avoid $f$'. %Constructions of sets with large Hausdorff dimension avoiding zeroes of $f$ have been considered when $f$ is a low degree polynomials \cite{Mathe} and when $f$ is a submersions \cite{FraserPramanik}.
Before my PhD, in work with Pramanik and Zahl I constructed sets with large Hausdorff dimension avoiding certain nonlinear functions $f$:

\begin{theorem} \cite{DensonPramanikZahl}
    Suppose $f^{-1}(0)$ has Minkowski dimension $s$. Then there exists a set $X \subset [0,1]^d$ with Hausdorff dimension $(dn-s)/(n-1)$ avoiding $f$.
\end{theorem}

\begin{theorem} \cite{DensonThesis}
    If $f = g \circ l$, where $l: \RR^n \to \RR^l$ is a full-rank rational coefficient linear map and $g: \RR^l \to \RR^m$ is a submersion, then there is $X \subset [0,1]$ with Hausdorff dimension $1/m$ avoiding $f$.
\end{theorem}

During my PhD, I found analogues of these constructions in the much harder Fourier dimension setting, where the only known constructions in the literature assume $f$ is linear. In \cite{DensonFourier}, I found constructions when $f(x_1,\dots,x_n) = x_n - g(x_1,\dots,x_{n-1})$, where $g$ is a submersion in each variable:

\begin{theorem}
    Fix $g: [0,1]^{d(n-1)} \to \RR^d$, and suppose that for each $k \in \{ 0, \dots, n-2 \}$, the matrix $D_k g = ( \partial g_i / \partial x_{dk+j} )_{i,j = 1}^d$ is invertible. Then there exists a set $X \subset [0,1]^d$ of Fourier dimension $d/(n-3/4)$ such that for all distinct $x_1,\dots,x_n \in X$, $x_1 \neq g(x_2,\dots,x_n)$. %If, in addition, $g(x_2,\dots,x_n) = ax_2 + h(x_3,\dots,x_n)$ for some $a \in \QQ$, then there exists a set $X \subset [0,1]^d$ of Fourier dimension $d/(n-1)$ such that for all distinct $x_1,\dots,x_n \in X$, $x_1 \neq g(x_2,\dots,x_n)$.
\end{theorem}

\emph{This method remains the only method of constructing sets with large Fourier dimension avoiding nonlinear patterns, and remains the best method for avoiding general `linear' patterns when $d > 1$.} The proof involves the probabilistic method. Given the problem's nonlinearity, we employ the theory of concentration of measure to ensure Fourier decay of measures on $X$ do not deviate from an expected decay with high probability, and then control expected decay via an inclusion-exclusion argument associated with a Whitney decomposition of a neighborhood of $f^{-1}(0)$.

\section*{A Plan For Research in Bilbao}

We now discuss four projects I plan to pursue during my postdoctoral work. In my letter of interest, I connect these projects to the research goals of the analysis group at BCAM.

\subsection*{Spectral Multipliers of Laplacians and sub-Laplacians}

A natural generalization of Theorem \ref{characterization} can be formulated for multipliers of the Laplace operator on any compact Riemannian manifold $M$, and the reduction to half-wave propagators $\{ W_t \}$ also applies, with waves propagating along the geodesic flow of $M$.
%In Section \ref{Section2}, I discussed that the results I were able to obtain for multiplier operators on $S^d$ generalized to multipliers of an arbitrary first order, elliptic, self-adjoint pseudodifferential operator $P$ on a compact manifold $M$, provided that $P$ satisfied two assumptions. Assumption A relates to the curvature of the principal symbol, and this assumption cannot really be weakened without significantly changing the character of the results, which heavily depend on this curvature. On the other hand, Assumption B arises as an artifact of the methods of our proof. We can likely obtain similar bounds while weakening this assumptions; for instance, Kim \cite{KimManifold} obtained bounds on the scale of Besov spaces only under Assumption A.
However, generalizing the results of \cite{DensonCharacterization} to all $M$ is very difficult, since in general the long time behavior of the flow cannot be estimated accurately enough. The method of \cite{DensonCharacterization} suggest results would follow if we could prove
%
\begin{equation} \label{smoothingbound}
    \| \mathbf{I}_{[k,k+1]}(t) W_t f \|_{L^q_x L^q_t} \lesssim k^\delta \| f \|_{L^q_{d(1/p - 1/2) - 1/p'}} \quad\quad\text{for some $\delta > 0$.}
\end{equation}
%
It seems feasible to obtain \eqref{smoothingbound} from local smoothing inequalities if we assume $W_k$ is close to the identity operator for $k \in \ZZ$, which happens precisely when the geodesic flow on $M$ is periodic. % since the theory of propogation of singularities tells us the operators $R = W_1$ and it's iterates differ from the identity by an order zero pseudodifferential operators whose principal symbol controlled by a topological invariant of the flow known as the Maslov index.
In this case the operator $R = W_1$ and it's iterates $R^k = W_k$ have been studied a little by spectral theorists, and there $R$ is known as the \emph{return operator}. If we are able to control the $L^p$ operator norm of iterates of $R$, local smoothing for $\{ W_k \}$ then yields \eqref{smoothingbound}. 

A more daring project would be to extend these methods to multipliers of non-elliptic operators, such as the sub-Laplacian on a Heisenberg or Carnot group. The problem is complicated here since the oscillatory integrals arising from the half-wave equation can become degenerate. Nonetheless, several recent results, including work in preparation due to Seeger, Stoval, M\"{u}ller, and Niedorf, have obtained $L^p$ boundedness of operators assuming their symbol $a$ lies in a Sobolev space $L^2_\alpha$ for appropriate $\alpha$. The condition in \eqref{secondinequality} is closely related to $L^p_\alpha$ bounds on $a$ for $p > 2$, which do not seem to have been studied in the Heisenberg setting, so it seems possible the methods of \cite{DensonCharacterization} may obtain new estimates.

Finally, I propose two exploratory projects related to the sub-Laplacian, related to two summaries I wrote during my PhD. The first project relates to work of Georgiev and Mukherjee \cite{GeorgievMukherjee}, who used stochastic diffusions related to an elliptic operator $P$ to analyze the in-radius of nodal domains of $P$ on bounded domains. If we could obtain control over stochastic diffusions associated with \emph{subelliptic operators}, then similar methods might help understand the nodal domains of eigenfunctions for \emph{sub-Laplacians} on bounded domains. The second project relates to work of Basu, Guo, Zhang, and Zorin-Kranich \cite{BasuGuoZhangZorinKranich}, who obtained methods for bounding highly degenerate oscillatory integrals with an algebraic phase using o-minimality methods from mathematical logic. Such methods remain highly underutilized in harmonic analysis. I believe such methods might be used to obtain control on the kinds of oscillatory integrals that arise for Fourier integral operators associated with highly degenerate subelliptic operators, which may have applications to the study of sub-Laplacians on highly degenerate algebraic Lie groups. My summary of \cite{GeorgievMukherjee} can be found on my website \href{jdjake.github.io/notes}{jdjake.github.io/notes}, and a study guide of \cite{BasuGuoZhangZorinKranich}, cowritten with Johnsrude, Sandberg, and de Oliveira Andrade, will be available on the ArXiv shortly.

\subsection*{Vector Field Methods}
\vspace{-0.2em}
My PhD work focused on the regularity of linear wave equations. Recently, I have begun studying the long time behavior of nonlinear wave equations. The `vector field method' has long proved a key tool in this setting, capturing the dispersive nature of waves using vector fields to capture approximate symmetries of a problem. Recently, Ifrim and Tataru \cite{IfrimTataruWaterWaves} have developed a method called `wave packet testing' complementing the vector field method, to obtain global control on various dispersive equations, first applied to the non-linear Schr\"{o}dinger equation %\cite{IfrimTataruNLS}, 
and later applied successfully for many other dispersive equations such as 2d water waves and 3d Maxwell-Dirac systems.

Notably, Ifrim and Tataru's methods have not been applied to the study of wave equations on black-hole spacetimes. For some time, obtaining sharp decay estimates to the wave-equation on high-dimensional Schwarzschild spacetimes proved difficult, until resolved by Schlue \cite{Schlue}. But Schlue's method does not apply to spacetimes with dynamic backgrounds, which introduce a nonlinearity to the problem. I plan to investigate whether Ifrim and Tataru's methods could give the necessary control to obtain results.

\subsection*{Fourier Dimension of High Codimension Hypersurfaces via Extremizer Methods}

The Fourier dimension of $X \subset \RR^d$ is the largest $s > 0$ for which there exists a probability measure $\mu$ supported on $X$ with a decay bound $|\widehat{\mu}(\xi)| \lesssim |\xi|^{-s/2}$. If $X \subset \RR^d$ is an appropriately non-degenerate surface with codimension $k$, then stationary-phase methods can be used to bound the Fourier decay of smooth measures supported on $X$. However, if $X$ has higher codimension, the best possible decay is unknown, even for non-degenerate curves in $\RR^3$; in that case smooth measures decay at a rate of $|\xi|^{-2/3}$, but it is difficult to rule out that non-smooth measures $\mu$ may decay at a rate faster than $|\xi|^{-2/3}$. I believe such measures may be ruled out using the method of extremizers for extension operators. Indeed, there are many such results in the area that characterize the \emph{smoothness} of extremizers. If we are able to obtain such results in this setting, we can rule out non-smooth fast decaying measures, and thus find the best possible Fourier decay of measures on surfaces. More generally, I am also interested in looking into other problems about extremizers, such as which multipliers optimize the inequality I obtained in \cite{DensonCharacterization}.

\subsection*{Genuine Decoupling On Random Fractals}

One major development in harmonic analysis in the past decade has been a greater understanding of the phenomenon of \emph{decoupling}, or \emph{Wolff-type estimates}. Given a family of almost disjoint subsets $\mathcal{E}_\delta$ of $\RR^d$ parameterized by $\delta > 0$, $L^p(l^2)$ decoupling discusses bounds of the form
%
\begin{equation} \label{decoupling}
    \Big\| \sum f_j \Big\|_{L^p(\RR^d)} \lesssim_\varepsilon \delta^{-\varepsilon} \Big( \sum\nolimits_j \| f_j \|_{L^p(\RR^d)}^2 \Big)^{1/2},
\end{equation}
%
where the $\widehat{f}_j$ are supported on distinct sets in $\mathcal{E}_\delta$, and the inequality holds for all $\varepsilon > 0$.
%  or the stronger logarithmic bound $D_p(\delta) \lesssim \log(\delta)$ and other variants of this form.
Genuine decoupling has been obtained for curves and surfaces, but no genuine decoupling has been established for any \emph{fractal set}. More precisely, consider a Cantor-type construction, a nested family of sets $\{ X_i \}$ where $X_i$ is a union of disjoint cubes of some fixed sidelength $\delta_i$. Given a sequence $n(i)$, let $\mathcal{E}_{\delta_i}$ be all sets of the form $Q \cap X_{i + n(i)}$, where $Q \subset X_i$ is a sidelength $\delta_i$ cube. For which such data do we obtain a decoupling inequality?
%We consider integers $n(i)$, and ask for which sequence the families $\mathcal{E}_{\delta_i}$ consisting of all sets of the form $Q \cap X_{i + n(i)}$, where $Q \subset X_i$ is a sidelength $\delta_i$ cube. satisfy a genuine decoupling inequality.%. We now ask for which sequences $\{ X_i \}$ and for which sequences $\{ n(i) \}$ do we obtain a genuine decoupling inequality for the families $\mathcal{E}_{\delta_i}$.
No such decoupling inequalities are known for any datum. But methods similar to those used in \cite{DensonFourier} may yield genuine decoupling for random Cantor-type constructions; indeed, similar methods have been applied by Bourgain \cite{Bourgain} and Talagrand \cite{Talagrand} to prove the existence of $\Lambda(p)$ sets %, in particular, the method of majorizing measures and selection processes.
which are a discrete variant of fractals with genuine decoupling. Using these methods, I plan to obtain fractal analogues of the decoupling methods of \cite{BourgainDemeterStudyGuide}, i.e. establishing an analogue of multilinear Kakeya for the sets $\mathcal{E}_\delta$, and then using an induction on scales argument.

\AtNextBibliography{\footnotesize}
\printbibliography

\end{document}