%\title{Example letter using the newlfm LaTeX package}
%
% See http://texblog.org/2013/11/11/latexs-alternative-letter-class-newlfm/
% and http://www.ctan.org/tex-archive/macros/latex/contrib/newlfm
% for more information.
%
\documentclass[12pt,stdletter,orderfromtodate,sigleft]{newlfm}
\let\geometry\relax
\usepackage[a4paper, margin=1in]{geometry}

\usepackage{etoolbox}

\usepackage{amssymb,amsmath,amsthm}
\usepackage{mathtools}
\usepackage{comment}
 \usepackage{paralist}

\DeclareMathOperator{\QQ}{\mathbb{Q}}
\DeclareMathOperator{\ZZ}{\mathbb{Z}}
\DeclareMathOperator{\RR}{\mathbb{R}}
\DeclareMathOperator{\NN}{\mathbb{N}}
\DeclareMathOperator{\HH}{\mathbb{H}}
\DeclareMathOperator{\BB}{\mathbb{B}}
\DeclareMathOperator{\CC}{\mathbb{C}}
\DeclareMathOperator{\AB}{\mathbb{A}}
\DeclareMathOperator{\PP}{\mathbb{P}}
\DeclareMathOperator{\MM}{\mathbb{M}}
\DeclareMathOperator{\VV}{\mathbb{V}}
\DeclareMathOperator{\TT}{\mathbb{T}}
\DeclareMathOperator{\LL}{\mathcal{L}}
\DeclareMathOperator{\DD}{\mathcal{D}}
\DeclareMathOperator{\SW}{\mathcal{S}}
\DeclareMathOperator{\EC}{\mathcal{E}}
\DeclareMathOperator{\AC}{\mathcal{A}}

%% Patch from https://tex.stackexchange.com/a/395529/226 to address newlfm bug
\makeatletter
\patchcmd{\@zfancyhead}{\fancy@reset}{\f@nch@reset}{}{}
\patchcmd{\@set@em@up}{\f@ncyolh}{\f@nch@olh}{}{}
\patchcmd{\@set@em@up}{\f@ncyolh}{\f@nch@olh}{}{}
\patchcmd{\@set@em@up}{\f@ncyorh}{\f@nch@orh}{}{}
\makeatother
 
\newlfmP{Headlinewd=0pt,Footlinewd=0pt}
 
\namefrom{}

\dateset{}
 
\greetto{\emph{Dear Members of the Selection Committee for the\\
Basque Center Postdoctoral Fellowship in Mathematical Analysis,}}

\closeline{}

\newlfmP{dateskipbefore=-4cm}
 \newlfmP{dateskipafter=1cm}

\begin{document}

\begin{newlfm}

\begin{spacing}{1.13}

I am a PhD student of Dr. Andreas Seeger at the University of Wisconsin-Madison, graduating in the coming spring. The mathematical problems I currently enjoy working on involve methods from harmonic analysis, but also some combinatorics and probability theory. My work has lead to several problems in the theory of sub-Laplacians, the theory of extremizers, and the study of nonlinear wave equations. Dr. Luz Roncal and Dr. Mateus Sousa have close ties to these areas, as do many of the other researchers at BCAM and the greater mathematics research community in Bilbao, and so the opportunity to conduct research at the Basque Center would be a fantastic place for me to cultivate my research interests, and foster fruitful collaborations with the analysis research group. In the remainder of this letter, I describe the potential interactions between my research and mathematicians in Bilbao, especially Dr. Roncal and Dr. Sousa.

My current PhD Thesis work focuses on endpoint estimates for radial Fourier multipliers, and related operators associated with eigenfunctions of the Laplace-Beltrami operator on compact manifolds, using tools from the theory of Fourier Integral Operators, in particular, the theory of cinematic curvature and local-smoothing phenomena for wave equations. My initial interest in this project actually began in Bilbao, when in 2019 I attended the 4th BCAM-UPV{\slash}EHU Summer School on Harmonic Analysis  and took part in a reading project organized by David Beltran on discrete restriction estimates on compact manifolds, which involves similar methods. In the future, I hope to extend the methods I developed to multipliers of other operators, in particular to multipliers of sub-elliptic operators, such as sub-Laplacians on Heisenberg or Carnot groups. Dr. Roncal has recently written several papers studying the resolvent and fractional powers of the Heisenberg sublaplacian. I am also interested in studying multipliers in other settings found in Dr. Roncals work, such as the study of radial multiplier operators on the infinite torus.

%My Thesis work, advised by Andreas Seeger, has focused on the study of endpoint estimates for radial Fourier multiplier operators on Euclidean space, and analogous operators on compact manifolds related to eigenfunctions of the Laplace-Beltrami operator, through an understanding of local-smoothing type phenomena for wave equations on these spaces.

My work in geometric measure theory, in particular, the study of the Fourier decay of fractal measures, has also resulted in an interest in the study of optimal inequalities, in particular the theory of smoothness of extremizers, as detailed in my research statement. I am also interested in studying sharp bounds for radial multipliers on $\RR^d$, i.e. determining whether there exists multipliers which are extremizers for the $L^p$ bounded radial multiplier operators, which would provide sharp variants for the radial multiplier bounds of Heo, Nazarov, and Seeger. Bilbao was also where I was introduced to the theory of extremizers, when Julien Sabin gave a series of talks at the 4th BCAM Summer School in Harmonic Analysis and PDEs. Much of Dr. Sousa's recent work has focused on the study of sharp inequalities and extremizers for Fourier restriction and so a position at the Basque Center would greatly help further these research projects.

It is also very opportune that the Basque Center is within walking distance of the University of the Basque Country, which would give me the opportunity to interact with the department there. In particular, I have recently developed an interest in studying the regularity of non-linear wave equations using vector field methods, in particular to study the decay of small solutions on dynamic Schwarzschild black hole spacetimes. I discuss this problem in more detail in my research statement. In addition to working with Dr. Roncal on these problems, I am interested to see what interactions the methods I am currently studying have with the work of Dr. Luis Vega at the University of the Basque Country.

Thank you for your consideration,\\\\
Jacob Denson

\end{spacing}

\end{newlfm}
\end{document}