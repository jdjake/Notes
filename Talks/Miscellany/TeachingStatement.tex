
\documentclass[11pt]{article}

\usepackage[a4paper, margin=1.05in]{geometry}

%\usepackage[text={16cm,24cm}]{geometry}

\usepackage{amssymb,amsmath,amsthm}
\usepackage{mathtools}
\usepackage{amsrefs}
\usepackage{comment}
\usepackage{amsthm}

\usepackage{hyperref}

\newtheorem*{theorem}{Theorem}

\DeclareMathOperator{\QQ}{\mathbb{Q}}
\DeclareMathOperator{\ZZ}{\mathbb{Z}}
\DeclareMathOperator{\RR}{\mathbb{R}}
\DeclareMathOperator{\NN}{\mathbb{N}}
\DeclareMathOperator{\HH}{\mathbb{H}}
\DeclareMathOperator{\BB}{\mathbb{B}}
\DeclareMathOperator{\CC}{\mathbb{C}}
\DeclareMathOperator{\AB}{\mathbb{A}}
\DeclareMathOperator{\PP}{\mathbb{P}}
\DeclareMathOperator{\MM}{\mathbb{M}}
\DeclareMathOperator{\VV}{\mathbb{V}}
\DeclareMathOperator{\TT}{\mathbb{T}}
\DeclareMathOperator{\LL}{\mathcal{L}}
\DeclareMathOperator{\DD}{\mathcal{D}}
\DeclareMathOperator{\SW}{\mathcal{S}}
\DeclareMathOperator{\EC}{\mathcal{E}}
\DeclareMathOperator{\AC}{\mathcal{A}}

     \setlength{\parskip}{0.3em}
    \setlength{\parindent}{0.2in}

\title{Teaching Statement}
\author{Jacob Denson}
\date{}

\begin{document}

\maketitle

When we first start teaching mathematics, we often have the view that we can spark a passion of mathematics in the hearts with everyone sitting in our classes, especially in classes for first and second year undergraduates. In the rare occasion this happens, that can be an amazing experience. But our ultimate goal as teachers is to meet the needs of all the students who attend our classes, students with a wide variety of interests and goals that have placed them in our classes. It's our job to motivate these students, to ensure they all leave feeling their goals for attending the class were met and that they trust us to achieve these goals. If we do our job well, this wider goal can be just as rewarding as sparking passion in a select few in our classes. When teaching, I keep two central principles in mind to increase the likelihood that students will be able to meet the goals in our classes:

{\bf Maintaining the Trust of our Students:} In order for students to succeed, they must trust us to teach and assign material that helps them achieve their goals and achieve the best results they can obtain in a class. Such trust is lost quite easily, and is difficult to earn back. Once students begin to lose trust in the work you're assigning, it becomes much more difficult to help guide students towards success.

One strategy I have used to immediately gain trust is to encourage student feedback as early and as often as is reasonable. On the first day of class, I take an anonymous class survey to determine what students are hoping to get out of the class, and what their anxieties might be about the class. Acknowledging the answers you receive and incorporating them into your teaching plans gives you a huge head-start to building trust. After this, asking for other feedback maintains this trust, in particular asking students to identify tough or unmotivated topics in class. One approach is the concept of `MUD cards', where students are allowed to anonymously answer what the most crucial part of the lecture was, and what the most confusing, or `muddy' aspect of the lecture is.

In addition, attitude towards teaching during lecture time can also go a long way to building trust, in particular responding to student questions in a non-judgemental way, posing questions and finding positive aspects of incorrect answers, so students can feel as comfortable as possible answering questions. Being patient and open minded is crucial.

{\bf Structure and Scaffolding}: I believe students learn best when they're able to identify clear principles that can be reapplied again and again. These principles are best shown by being clearly articulated to students when they are learning a topic. At the beginning of class, we should clearly identify the major principles of the mathematical work we will be discussing in the rest of the class. When doing examples in class, or responding to student questions, we should try to form explanations in terms of the principles we've articulated. Often lower level students rely on pattern recognition when determining the method by which a problem is solve, often resulting in incomprehensible solutions when the pattern recognition goes awry. By clearly articulated the principles we're using, and thus suggesting a more consistent way to thinking about mathematics, the hope is to drill in principled thinking to our teaching to encourage students away.

I tried to carry out this principle during my PhD when teaching the graduate student `Summer Enhancement Program', or SEP, which has the goal of preparing graduate students to pass their qualifying exams. To prepare for teaching the course, I looked through the solutions to the last 5-6 years SEP exams, and tried thinking of \emph{principles} that lead towards the solution of a problem, from `using the uniform boundedness theorem', to more technique based principles like `break a quantity up into dyadic parts'. After categorizing the different principles, I organized them into different days. On each day, before looking at particular problems that might occur on a qualifying exam, we identified some clear principles which often help us with the type of problems we might deal with on that day. I then encouraged students to actively `choose' which principle they apply for a particular problem. On the day after, in the first 10 minutes of class we ran through warm up questions from the previous day, but without the principles in front of them. I consider this a form of \emph{scaffolding}, i.e. giving structure up front to encourage student learning, and then slowly taking it away until they internalize structure.

The students enjoyed the approach of the course. The notes I prepared for this course in 2021 and 2022 continue to be used by educators in the 2023 and 2024 course, and new graduate students continue to thank me for writing these notes to help them prepare for the exam. For reference, these notes may be found on my website, entitled \emph{University of Wisconsin Analysis SEP Course Notes}, at the following link: \href{https://jdjake.github.io/notes.html}{https://jdjake.github.io/notes.html}.

Certainly, the principles that a graduate student must internalize to succeed are of a very different nature to the principles an undergraduate student may need to succeed -- for instance, one general principle might be ``checking whether answers make physical sense to guard against mental math errors'', or ``completing the square to isolate a variable in quadratic expression''. But the method of forming general principles sitll works in this session, and I used this approach to some degree of success when teaching a College Algebra class last year.

{\bf Future Work}: Effective teaching requires mastery of many skills, and thus requires an ongoing process of growth and adaption to new challenges in the classroom. I have only been teaching for a few years as a PhD, and I remain open minded about new approaches to keeping my class motivated. Recently, I took a class entitled \emph{Advancing Learning Through Evidence-Based STEM Teaching} from the Delta Program at the University of Wisconsin Campus which covers a variety of teaching methodology, such as Peer Instruction, Active Learning, and Cooperative learning. In my future work as an educator, I hope to try these approaches to in future classwork, in order to discover what works for the diverse kinds of students and classrooms we encounter as graduate teachers. I also look forward to learning other perspectives from more experienced teachers at other institutions. By fostering a dialogue about effective teaching practices, I hope to enhance not only my own teaching but also contribute to the broader educational community.

In conclusion, my goal as a educator is to find ways of building learning communities which allow students to feel confident to use the beauty and utility of mathematics to achieve their individual goals, and I plan to further my ability to build such communities in my post doctoral work.

\end{document}