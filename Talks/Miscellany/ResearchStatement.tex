\documentclass[12pt]{article}

\usepackage[a4paper, margin=1in]{geometry}

\usepackage{amssymb,amsmath,amsthm}

\DeclareMathOperator{\QQ}{\mathbb{Q}}
\DeclareMathOperator{\ZZ}{\mathbb{Z}}
\DeclareMathOperator{\RR}{\mathbb{R}}
\DeclareMathOperator{\NN}{\mathbb{N}}
\DeclareMathOperator{\HH}{\mathbb{H}}
\DeclareMathOperator{\BB}{\mathbb{B}}
\DeclareMathOperator{\CC}{\mathbb{C}}
\DeclareMathOperator{\AB}{\mathbb{A}}
\DeclareMathOperator{\PP}{\mathbb{P}}
\DeclareMathOperator{\MM}{\mathbb{M}}
\DeclareMathOperator{\VV}{\mathbb{V}}
\DeclareMathOperator{\TT}{\mathbb{T}}
\DeclareMathOperator{\LL}{\mathcal{L}}
\DeclareMathOperator{\DD}{\mathcal{D}}
\DeclareMathOperator{\SW}{\mathcal{S}}
\DeclareMathOperator{\EC}{\mathcal{E}}
\DeclareMathOperator{\AC}{\mathcal{A}}

\title{Research Statement}
\author{Jacob Denson}
\date{}

\begin{document}

\maketitle

{\bf My Research} focuses on the study of Fourier multiplier operators on Euclidean space, and their analogues on compact manifolds, through an understanding of the behaviour of wave propogation on such spaces. I also work on problems in geometric measure theory, trying to develop when `structure' occurs in fractals of large dimension. Both projects lead to interesting questions that I plan to pursue in my postgraduate work.

During my PhD, much of my work has concentrated on the relations between radial Fourier multiplier operators on $\RR^d$, and their relation to analogous operators on compact manifolds. In \cite{DensonCharacterization}, I showed that $L^p$ bounds for a radial Fourier multipliers imply corresponding bounds for a related operator on the sphere which is diagonalized by spherical harmonics for an open interval of exponents $p$ when $d \geq 4$. In that same paper, I obtained a complete characterization of operators diagonalized by spherical harmonics, whose 'dilates' are uniformly $L^p$ bounded, for the same range of $p$. This is the first characterization of $L^p$ boundedness for such multipliers for any $p \neq 2$, and no other characterizations has been proved for analogous families of operators on any other compact manifold.

My work in geometric measure theory focuses on whether having large 'Fourier dimension' guarantees the presence of certain geometric configurations in sets. I established a general probabilistic method of constructing sets with large Fourier dimension which avoid a variety of different 'nonlinear' geometric configurations. For instance, the method can construct subsets of $\RR^2$ of Fourier dimension $0.8$ which do not contain three points forming the vertices of an isosceles triangle. This method remains the only construction method in the literature for sets with large Fourier dimension avoiding nonlinear configurations, and also remains the best currently known construction method for constructing sets avoiding solutions to a general linear equation when $d > 1$.

{\bf In The Near Future}, I hope to generalize my bounds for multipliers of spherical harmonic expansions to the more general setting of multipliers for eigenfunction expansions of the Laplace-Beltrami operator on a Riemannian manifold $M$ with periodic geodesic flow. A local obstruction to this generalization requires obtaining control of a pseudodifferential operator on $M$ called the \emph{return operator}, and a global obstruction requires an endpoint refinement of the local smoothing inequality for the wave equation on $M$ to handle combining dyadic scales together. I am also interested in exploring what kinds of bounds for multipliers can be obtained via the wave equation on manifolds whose geodesic flow is \emph{integrable}. In the study of patterns, I hope to apply the tools I exploited to construct sets of large fractal dimension to the study of decoupling on random fractals. And I am interested in determining the interrelation of my, in particular studying distance sets using local smoothing bounds, and exploring analogues of Fourier dimension on Riemannian manifolds.

\pagebreak[3]






\section*{Fourier Multiplier Operators}

Much of my PhD has dealt with the study of bounds for radial Fourier multipliers on $\RR^d$, bounds for multipliers of eigenfunction expansions on compact manifolds, and the relation between such bounds.

Multipliers have been central to harmonic analysis from it's inception. Fourier showed that solutions to the classical equations of physics can be described by \emph{Fourier multiplier operators}. These operators $T$ are those defined from a function $m: \RR^d \to \CC$, the \emph{symbol} of $T$, by setting
%
\[ Tf(x) = \int_{\RR^d} m(\xi) \widehat{f}(\xi) e^{2 \pi i \xi \cdot x}\; dx. \]
%
We now know all translation-invariant operators on $\RR^d$ are Fourier multiplier operators, explaining their broad applicability, and the theory continues to have applications in areas as diverse as partial differential equations, mathematical physics, number theory, complex variables, and ergodic theory.

A theory of multiplier operators can be developed on the sphere $S^d$ with similar analogues to the theory of Fourier multipliers. The analogues of the planar waves $\{ e^{2 \pi i \xi \cdot x} : \xi \in \RR^d \}$ are the \emph{spherical harmonics}. Every function $f \in L^2(S^d)$ can be uniquely expanded as a sum $\sum_{k = 0}^\infty f_k$, where $f_k$ is a spherical harmonic of degree $k$, i.e. the restriction of a homogeneous, harmonic degree $k$ polynomial on $\RR^{d+1}$ to the unit sphere. A \emph{multiplier operator for the spherical harmonic expansion} is an operator $T$ on $S^d$ defined in terms of a function $m: \NN \to \CC$, the symbol of $T$, such that
%
\[ Tf(x) = \sum\nolimits_{k = 0}^\infty m(k) f_k. \]
%
Just as Fourier multipliers characterize all translation-invariant operators, all operators on the sphere which are \emph{rotation invariant} are multipliers for the spherical harmonic expansion, and so such operators arise in a diverse area of applications, including celestial mechanics and the analysis of angular momentum in quantum physics.

The general study of the $L^p$ boundedness of multipliers was initiated in the 1960s, brought on by the spur of applications the Calderon-Zygmund school and their contemporaries brought to the theory. Orthogonality immediately implies that a Fourier multiplier $T$ is bounded on $L^2(\RR^d)$ if and only if it's symbol $m$ lies in $L^\infty(\RR^d)$. Similarily, since spherical harmonics of different degrees are orthogonal to one another, a multiplier for spherical harmonic expansions is bounded on $L^2(S^d)$ if and only if $m \in L^\infty(\NN)$. The boundedness of such multipliers for $p \neq 2$ is a more subtle property, but underpins any deep understanding of the Fourier transform or the understanding of the interactions. Indeed, studying the boundedness of multipliers seems to be one of the few tractable ways of quantifying deconstructive interference between a family of planar waves, or spherical harmonics of different degrees. Necessary and sufficient conditions for a Fourier multiplier operator to be bounded on $L^1(\RR^d)$ or $L^\infty(\RR^d)$ were quickly realized.


Spherical harmonics of different degrees are orthogonal to one another, and this immediately implies $T$ is bounded on $L^2(S^d)$ if and only if $m \in L^\infty(\NN)$. But characterizations of $L^p$ boundedness for all $p \neq 2$ remain unknown.


% It was quickly realized that a Fourier multiplier operator is bounded on $L^1(\RR^d)$ or $L^\infty(\RR^d)$ if and only if the Fourier transform of the symbol $m$ lies in $L^1(\RR^d)$.
But the problem of finding necessary and sufficient conditions for boundedness in $L^p(\RR^d)$ for any other exponent proved impenetrable. Indeed, many interesting problems about the boundedness of \emph{specific} Fourier multipliers, such as the Bochner-Riesz conjecture, remain largely unsolved today.

It thus came as a recent surprise when necessary and sufficient conditions were found for bounding \emph{radial} Fourier multipliers (multipliers with a radial symbol) on $L^p(\RR^d)$. First came the result of \cite{GarrigosSeeger}, who found a necessary and sufficient condition in the range $|1/p - 1/2| > 1/2d$ for bounds of the form $\| Tf \|_{L^p(\RR^d)} \lesssim \| f \|_{L^p(\RR^d)}$ to hold uniformly over \emph{radial functions} $f$.
%If $T$ has symbol $m(|\xi|)$, then the condition says that $T$ is bounded if and only if the Fourier transforms of the functions $m_k(\xi) = m(2^k \xi) \chi(|\xi|)$ are uniformly bounded in $L^p(\RR^d)$, where $\chi \in C_c^\infty(0,\infty)$ and $\sum_{k \in \ZZ} \chi(2^k \xi) = 1$.
An optimist might think this same condition causes the bound to hold uniformly over \emph{all functions} $f$ in the range above, a statement we call the \emph{radial multiplier conjecture}. We now know, by the results of \cite{HeoNazarovSeeger} and \cite{Cladek}, that the radial multiplier conjecture holds when $d > 4$ and $|1/p - 1/2| > 1/(d-1)$, when $d = 4$ and $|1/p - 1/2| > 11/36$, and when $d = 3$ and $|1/p - 1/2| > 11/26$. But the radial multiplier conjecture is not completely resolved for any $d$, and no bounds are known at all when $d = 2$.

Several analogies e


Just as Fourier multipliers characterize all translation invariant operators, this class of operators characterizes all rotation invariant operators. 
s



 is an operator $T$ on $S^d$ for which there exists $m: \NN \to \CC$, the \emph{symbol} of $T$, such that $Tf = m(k) f$ for each $f \in \mathcal{H}_k(S^d)$.

Understanding the boundedness of Fourier multiplier operators in an $L^p$ norm for $p \neq 2$ underpins any subtle understanding of the Fourier transform. Plane waves oscillating in different directions and with different frequencies are orthogonal to one another, and thus do not interact with one another significantly in terms of the $L^2$ norm, as justified by Bessel's inequality. But plane waves can interact with one another in the $L^p$ norm for $p \neq 2$, and so understanding $L^p$ bounds for Fourier multipliers indicate when this interaction is significant or insignificant. Similarily, spherical harmonics of different degrees on $S^d$ are orthogonal to one another, but studying the $L^p$ bounds of multipliers of the Laplacian on the sphere is crucial to understand when the interactions of different spherical harmonics are significant or not.


$u(t,x) = e^{2 \pi i t \sqrt{-\Delta}} u_0$, where

First the \emph{Fourier multipliers} were studied, operators $T$ on $\RR^d$ defined by an expression of the form
%
\[ Tf(x) = \int_{\RR^d} m(\xi) \widehat{f}(\xi) e^{2 \pi i \xi \cdot x}\; d\xi, \]
%
where $\widehat{f}$ is the Fourier transform of the function $f$.

My main research project considers the study of the relation between the $L^p$ boundedness of two types of multipliers:
%
\begin{itemize}
	\item \emph{Fourier multipliers} are those operators $T$ on $\RR^d$ for which we can associate a function $m: \RR^d \to \CC$, known as the \emph{symbol} of $T$, such that for any function $f$, the Fourier transform of $Tf$ obeys the relation $\widehat{Tf} = m \widehat{f}$. Heuristically, this means that $T e^{2 \pi i \xi \cdot x} = m(\xi) e^{2 \pi i \xi \cdot x}$ for each $\xi \in \RR^d$.
\end{itemize}

  Every translation invariant operator on $\RR^d$ is a Fourier multiplier operator, and every rotation invariant operator on $S^d$ is a multiplier of the spherical harmonic expansion on $S^d$.



The natural analogue of the study of radial multipliers on $\RR^d$ is the study of multipliers of a Laplace-Beltrami operator on a Riemannian manifold. The natural analogue of the study of quasiradial multipliers on $\RR^d$ is the study of multipliers of an operator associated with a \emph{Finsler geometry} on the manifold.

\section*{Pattern Avoidance}

\section*{Future Lines of Research}



Both projects 

The work I have conducted naturally suggests several {\bf future problems}.
%
\begin{itemize}
	\item Analyzing the 'return time operator' to extend results on expansions of spherical harmonics to the study of the Laplace-Beltrami operator on $S^d$.

	\item Determining whether our methods extend to other manifolds whose geodesic flow is simpler to understand, such as integrable systems.

	\item Analyzing whether local smoothing bounds

	\item Constructing Random Salem Sets which satisfy a Decoupling Bound.

	\item Determining the relation between certain 'fractal weighted estimates' for the wave equation on $\RR^d$ and the 'density decomposition' of multiplier bounds.
\end{itemize}

\begin{thebibliography}{03}

\bibitem{GarrigosSeeger} Garrigos, Gustavo and Seeger, Andreas,
	\emph{Characterizations of {H}ankel Multipliers}.

\bibitem{HeoNazarovSeeger} Heo, Yaryong and Nazarov, Fedor and Seeger, Andreas,
	\emph{Radial {F}ourier Multipliers in High Dimensions}.

\bibitem{Cladek} Cladek, Laura,
	\emph{Radial {F}ourier Multipliers in $\RR^3$ and $\RR^4$}.

\bibitem{DensonCharacterization} Denson, Jacob,
	\emph{Multipliers of Spherical Harmonics}.


\end{thebibliography}

\end{document}