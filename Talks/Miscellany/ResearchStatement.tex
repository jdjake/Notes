\documentclass[12pt]{article}

\usepackage[a4paper, margin=1in]{geometry}

\usepackage{amssymb,amsmath,amsthm}

\DeclareMathOperator{\QQ}{\mathbb{Q}}
\DeclareMathOperator{\ZZ}{\mathbb{Z}}
\DeclareMathOperator{\RR}{\mathbb{R}}
\DeclareMathOperator{\NN}{\mathbb{N}}
\DeclareMathOperator{\HH}{\mathbb{H}}
\DeclareMathOperator{\BB}{\mathbb{B}}
\DeclareMathOperator{\CC}{\mathbb{C}}
\DeclareMathOperator{\AB}{\mathbb{A}}
\DeclareMathOperator{\PP}{\mathbb{P}}
\DeclareMathOperator{\MM}{\mathbb{M}}
\DeclareMathOperator{\VV}{\mathbb{V}}
\DeclareMathOperator{\TT}{\mathbb{T}}
\DeclareMathOperator{\LL}{\mathcal{L}}
\DeclareMathOperator{\DD}{\mathcal{D}}
\DeclareMathOperator{\SW}{\mathcal{S}}
\DeclareMathOperator{\EC}{\mathcal{E}}
\DeclareMathOperator{\AC}{\mathcal{A}}

\title{Research Statement}
\author{Jacob Denson}
\date{}

\begin{document}

\maketitle

{\bf My Research} focuses on problems in Harmonic analysis. In particular, I study Fourier multiplier operators on Euclidean space, and their analogues on compact manifolds, through an understanding of the geometry of wave propogation on these spaces. I also work on problems in harmonic analysis related to geometric measure theory, investigating when `structure' occurs in fractals of large dimension. Both projects lead to interesting questions that I plan to pursue in my postgraduate work.

During my PhD, much of my work on multipliers has concentrating on relating bounds on Fourier multiplier operators on $\RR^d$ to bounds for analogous operators on compact manifolds. I proved a `transference principle' \cite{DensonCharacterization} for zonal multipliers operators on the sphere $S^d$. For $d \geq 4$ and a range of $p$, this principle shows that $L^p$ bounds for a radial Fourier multiplier with symbol $m(|\xi|)$ imply bounds for a zonal multiplier operators on $S^d$ induced by the same symbol. In the process, I also completely characterized those symbols $m$ whose dilates give a uniformly bounded family of zonal multiplier operators on $L^p(S^d)$. This is the first such characterization on $S^d$ for any $p \neq 2$, and no other such characterization, or transference principle of this form has been proved for analogous operators on any other compact manifold.

My work in geometric merasure theory focuses on constructing sets of large fractal dimension avoiding certain point configurations. Together with Malabika Pramanik and Joshua Zahl, we obtained a method \cite{DensonPramanikZahl} for constructing sets avoiding configurations which have large Hausdorff dimension if the geometric configuration itself is 'low dimensional'. During my PhD, I continued this line of research by establishing several probabilistic extensions of the methods of the previous paper to address the more difficult problem of constructing sets of large \emph{Fourier dimension} avoiding configurations \cite{DensonFourier}. This method remains the only construction method for constructing sets of large Fourier dimension avoiding nonlinear configurations, and also remains the best currently known construction method for constructing sets avoiding general 'linear' configurations when $d > 1$.

%  Using techniques of high dimensional probability, I was able to fully recover the Hausdorff bound obtained in \cite{DensonPramanikZahl} in the Fourier dimension setting with the additional of a weak linearity assumption. An example application of the method is, for any smooth curve $c: [0,1] \to \RR^d$, a construction of a subset $X$ of $[0,1]$ of Fourier dimension $0.4$ such that $c(X)$ does not contain three vertices of an isosceles triangle. 

{\bf In The Near Future}, I hope to generalize the bounds obtained in \cite{DensonCharacterization} to the more general setting of multipliers for eigenfunction expansions of any Laplace-Beltrami operator on a Riemannian manifold $M$ with periodic geodesic flow. A local obstruction here requires obtaining control of a pseudodifferential operator on $M$ called the `return operator'. A global obstruction is an endpoint refinement of the local smoothing inequality for the wave equation on $M$. %to handle combining dyadic scales together.
I am also interested in exploring what kinds of bounds for multipliers can be obtained via wave equation type techniques on manifolds whose geodesic flow has well controlled dynamical properties, such as forming an integrable system. In the study of patterns, I hope to apply the square root cancellation techniques I exploited in the construction of sets of large Fourier dimension to construct random fractals with good decoupling constants. And I am interested in determining the interrelation of patterns with the study of multipliers on manifolds, in particular studying Falconer distance type problems on manifolds by using local smoothing bounds, and exploring analogues of Fourier dimension on Riemannian manifolds.

\pagebreak[3]






\section*{Fourier Multiplier Operators}

%Much of my PhD has dealt with the study of bounds for radial Fourier multipliers on $\RR^d$, bounds for multipliers of eigenfunction expansions on compact manifolds, and the relation between such bounds.

Multipliers have been central to harmonic analysis from its inception. Fourier showed that solutions to the classical equations of physics are described by \emph{Fourier multipliers}, operators $T$ defined from a function $m: \RR^d \to \CC$, the \emph{symbol}, by setting
%
\[ Tf(x) = \int_{\RR^d} m(|\xi|) \widehat{f}(\xi) e^{2 \pi i \xi \cdot x}\; dx. \]
%
Of particular interest are the \emph{radial} multipliers $T_a$, defined for a function $a: [0,\infty) \to \CC$ as the Fourier multiplier with symbol $m(\xi) = a(|\xi|)$. We now know all translation-invariant operators on $\RR^d$ are Fourier multiplier operators, explaining their broad applicability, and the theory continues to have applications in areas as diverse as partial differential equations, mathematical physics, number theory, complex variables, and ergodic theory.

A similar theory of multiplier operators can be developed on the sphere $S^d$ with similar analogues to the theory of Fourier multipliers. Roughly speaking, Fourier multipliers are operators on $\RR^d$ with $e^{2 \pi i \xi \cdot x}$ as eigenfunctions, and the analogoues on the sphere are operators with the \emph{spherical harmonics} as eigenfunctions, i.e. the homogeneous harmonic polynomials on $\RR^{d+1}$ restricted to $S^d$. Every function $f \in L^2(S^d)$ can be uniquely expanded as a sum $\sum_{k = 0}^\infty f_k$, where $f_k$ is a degree $k$ spherical harmonic. A \emph{zonal multiplier} is then an operator $Z_a$ on $S^d$ defined in terms of a function $a: \NN \to \CC$ by setting
%
\[ Z_m f = \sum\nolimits_{k = 0}^\infty a(k) f_k. \]
%
Just as Fourier multipliers characterize all translation-invariant operators, zonal multipliers characterize all operators on the sphere which are \emph{rotation invariant}, and thus arise in a diverse area of applications, including celestial mechanics, the analysis of angular momentum in quantum physics, and computer graphics.

In harmonic analysis, it has proved incredibly profitable to study the boundedness of Fourier multipliers with respect to the various $L^p$ norms. It seems to be one of the few tractable ways of quantifying how planar waves travelling in different directions and with different frequencies interact with one another, and thus underpins almost all subtle understandings of how the Fourier transform behaves. Similarily, understanding the $L^p$ boundedness of zonal harmonics tells us how spherical harmonics of different degrees interact with one another. In the past decade, several arguments \cite{HeoNazarovSeeger,Cladek,KimQuasiradial} have found sharp characterizations of $L^p$ boundedness for radial Fourier multipliers, and one of the main goals of my research on multipliers was to see whether these arguments had analogies in the theory of spherical harmonics.

Why might such arguments carry over? One heuristic reason is that both operators are connected to the \emph{Laplacian} on their respective spaces. If the Fourier transform of a distribution $f$ is supported on the sphere $\{ \xi : |\xi| = R \}$, then it follows from the Fourier inversion formula that $\Delta f = - 4 \pi^2 R^2 f$. This essentially follows because then $f$ is a superposition of the planar waves $e^{2 \pi i \xi \cdot x}$ with $|\xi| = R$, and $\Delta e^{2 \pi i \xi \cdot x} = - 4\pi^2 R^2 f$. But for such $f$, if $T_a$ is a radial Fourier multiplier, then $T_a f = a(R) f$. Thus a radial Fourier multiplier has the same eigenfunctions of the Laplacian. Similarily, if we consider the Laplace-Beltrami operator $\Delta$ on $S^d$ given by the Riemannian metric, and if $f$ is a spherical harmonic of degree $k$, then $\Delta f = - k(k+d-1) f$. Since $Z_a f = a(k) f$, we see that zonal multipliers have the same eigenfunctions as the Laplacian. Using the language of the functional calculus of operators, we can write $T_a = a( \sqrt{-\Delta} )$ and $Z_a = a( \sqrt{ \alpha^2 - \Delta} )$, where $\alpha = (d-1)/2$, and now the resemblance between the two families of operators becomes clear.

\pagebreak[3]

One disadvantage of the study of zonal harmonics when compared to the theory of Fourier multipliers is that unlike the planar waves $e^{2 \pi i \xi \cdot x}$, a general spherical harmonic of degree $k$ cannot be given a single explicit formula. Connected to this is that $\RR^d$ has natural dilation symmetries, whereas $S^d$ has nothing analogous. The dilation symmetries on $\RR^d$ tell us high frequency planar waves are just the dilates of low frequency planar waves; on the other hand, high degree spherical harmonics need not look anything like low degree spherical harmonics.

Suppose $a: [0,\infty) \to \CC$ is compactly supported on the interval $[1/2,2]$. Then the radial function $k(x) = \int a(|\xi|) e^{2 \pi i \xi \cdot x}\; d\xi$ is the a convolution kernel for the radial multiplier operator $T_a$, i.e. $T_a f = k * f$ for all inputs $f$. The bounds on radial multipliers obtained in BLAH are then of the form $\| T_a f \|_{L^p(\RR^d)} \lesssim \| k \|_{L^p(\RR^d)} \| f \|_{L^p(\RR^d)}$. Dilation symmetry then immediately implies BLAH DYADIC RESULT, and then some methods of atomic decompositions and Littlewood Paley theory can be combined to obtain a tight result. On a sphere, we \emph{can} define the zonal convolution kernel $k$ corresponding to a zonal multiplier $Z_a$, such that
%
\[ (Z_a f)(x) = \int_{S^d} k( y \cdot x ) f(y)\; dy. \]
%
If $a$ is supported on $[1/2,2]$, then weighted $L^p$ bounds on $k$ can be used to imply bounds on $Z_a$, but these bounds will not scale, and it is difficult to determine how the bounds scale under dilations since there is no relation between the zonal convolution kernel $k$ corresponding to $a$, and the convolution kernel corresponding to the dilations of $a$.

A fix is obtained by writing $Z_a f$ using the \emph{cosine transform} of $a$, i.e. in terms of
%
\[ \widehat{a}(t) = \int_0^\infty a(\rho) e^{2 \pi i \rho t}\; d\rho. \]
%
The cosine transform \emph{does} scale under dilations, and functional calculus and the Fourier inversion formula allows us to write $Z_a$ in terms of $\widehat{a}$, i.e. by setting
%
\begin{equation} \label{InversionFormula}
	Z_a f = a(P) f = \int_{-\infty}^\infty \widehat{a}(t) e^{2 \pi i t P}\; dt,
\end{equation}
%
where $P = \sqrt{ \alpha^2 - \Delta }$, and $e^{2 \pi i t P} f = e^{2 \pi i t k} f$ for a spherical harmonic $f$ of degree $k$. The operators $u(t) = e^{2 \pi i t P} f$ give solutions to the half-wave equation $\partial_t u = P u$. We can understand the geometric behaviour of the half-wave equation by using the theory of \emph{Fourier integral operators}.

. Indeed, if $f$ is a spherical harmonic of degree $k$, then $Z_a f = a(k) f$ and by the Fourier inversion formula $\int \widehat{a}(t) e^{2 \pi i t P} f\; dt = \int \widehat{a}(t) e^{2 \pi i t k} f\; dt = a(k) f(t)$. For any function $f$, the functions $u(t) = e^{2 \pi i t P} f$ give a solution to the wave equation $\partial_t^2 u(t) = P u(t)$ with $u(0) = f$ and $\partial_t u(0) = 0$.


The bounds on radial multipliers obtained in BLAH depend on bounds on the convolution kernel corresponding to the multiplier. 


The main goal of my research project on multipliers is to understand


 deconstructive interference between a family of planar waves, or spherical harmonics of different degrees. Necessary and sufficient conditions for a Fourier multiplier operator to be bounded on $L^1(\RR^d)$ or $L^\infty(\RR^d)$ were quickly realized.

Understanding the boundedness of Fourier multiplier operators in an $L^p$ norm for $p \neq 2$ underpins any subtle understanding of the Fourier transform. Plane waves oscillating in different directions and with different frequencies are orthogonal to one another, and thus do not interact with one another significantly in terms of the $L^2$ norm, as justified by Bessel's inequality. But plane waves can interact with one another in the $L^p$ norm for $p \neq 2$, and so understanding $L^p$ bounds for Fourier multipliers indicate when this interaction is significant or insignificant. Similarily, spherical harmonics of different degrees on $S^d$ are orthogonal to one another, but studying the $L^p$ bounds of multipliers of the Laplacian on the sphere is crucial to understand when the interactions of different spherical harmonics are significant or not.



The general study of the $L^p$ boundedness of general Fourier multipliers began in the 1960s, brought on by the spur of applications the Calderon-Zygmund school and their contemporaries brought to the theory. Necessary and sufficient conditions for a Fourier multiplier to be bounded on $L^2(\RR^d)$ follow by simple orthogonality, and conditions to be bounded on $L^1(\RR^d)$ and $L^\infty(\RR^d)$ also follow simply, because such spaces do not tend to allow much capacity for subtle cancellation. But the problem of finding necessary and sufficient conditions for boundedness in $L^p(\RR^d)$ for any other exponent proved impenetrable. Surprisingly, in the past decade necessary and sufficient conditions for $L^p$ boundedness have occured for \emph{radial} Fourier multipliers.

%Orthogonality immediately implies that a Fourier multiplier $T$ is bounded on $L^2(\RR^d)$ if and only if it's symbol $m$ lies in $L^\infty(\RR^d)$. Similarily, since spherical harmonics of different degrees are orthogonal to one another, a multiplier for spherical harmonic expansions is bounded on $L^2(S^d)$ if and only if $m \in L^\infty(\NN)$. The boundedness of such multipliers for $p \neq 2$ is a more subtle property, but underpins any deep understanding of the Fourier transform or the understanding of the interactions. Indeed, studying the boundedness of multipliers seems to be one of the few tractable ways of quantifying deconstructive interference between a family of planar waves, or spherical harmonics of different degrees. Necessary and sufficient conditions for a Fourier multiplier operator to be bounded on $L^1(\RR^d)$ or $L^\infty(\RR^d)$ were quickly realized.
%Spherical harmonics of different degrees are orthogonal to one another, and this immediately implies $T$ is bounded on $L^2(S^d)$ if and only if $m \in L^\infty(\NN)$. But characterizations of $L^p$ boundedness for all $p \neq 2$ remain unknown.


% It was quickly realized that a Fourier multiplier operator is bounded on $L^1(\RR^d)$ or $L^\infty(\RR^d)$ if and only if the Fourier transform of the symbol $m$ lies in $L^1(\RR^d)$.
 Indeed, many interesting problems about the boundedness of \emph{specific} Fourier multipliers, such as the Bochner-Riesz conjecture, remain largely unsolved today.

It thus came as a recent surprise when necessary and sufficient conditions were found for bounding \emph{radial} Fourier multipliers (multipliers with a radial symbol) on $L^p(\RR^d)$. First came the result of \cite{GarrigosSeeger}, who found a necessary and sufficient condition in the range $|1/p - 1/2| > 1/2d$ for bounds of the form $\| Tf \|_{L^p(\RR^d)} \lesssim \| f \|_{L^p(\RR^d)}$ to hold uniformly over \emph{radial functions} $f$.
%If $T$ has symbol $m(|\xi|)$, then the condition says that $T$ is bounded if and only if the Fourier transforms of the functions $m_k(\xi) = m(2^k \xi) \chi(|\xi|)$ are uniformly bounded in $L^p(\RR^d)$, where $\chi \in C_c^\infty(0,\infty)$ and $\sum_{k \in \ZZ} \chi(2^k \xi) = 1$.
An optimist might think this same condition causes the bound to hold uniformly over \emph{all functions} $f$ in the range above, a statement we call the \emph{radial multiplier conjecture}. We now know, by the results of \cite{HeoNazarovSeeger} and \cite{Cladek}, that the radial multiplier conjecture holds when $d > 4$ and $|1/p - 1/2| > 1/(d-1)$, when $d = 4$ and $|1/p - 1/2| > 11/36$, and when $d = 3$ and $|1/p - 1/2| > 11/26$. But the radial multiplier conjecture is not completely resolved for any $d$, and no bounds are known at all when $d = 2$.

Several analogies e


Just as Fourier multipliers characterize all translation invariant operators, this class of operators characterizes all rotation invariant operators. 
s



 is an operator $T$ on $S^d$ for which there exists $m: \NN \to \CC$, the \emph{symbol} of $T$, such that $Tf = m(k) f$ for each $f \in \mathcal{H}_k(S^d)$.


$u(t,x) = e^{2 \pi i t \sqrt{-\Delta}} u_0$, where

First the \emph{Fourier multipliers} were studied, operators $T$ on $\RR^d$ defined by an expression of the form
%
\[ Tf(x) = \int_{\RR^d} m(\xi) \widehat{f}(\xi) e^{2 \pi i \xi \cdot x}\; d\xi, \]
%
where $\widehat{f}$ is the Fourier transform of the function $f$.

My main research project considers the study of the relation between the $L^p$ boundedness of two types of multipliers:
%
\begin{itemize}
	\item \emph{Fourier multipliers} are those operators $T$ on $\RR^d$ for which we can associate a function $m: \RR^d \to \CC$, known as the \emph{symbol} of $T$, such that for any function $f$, the Fourier transform of $Tf$ obeys the relation $\widehat{Tf} = m \widehat{f}$. Heuristically, this means that $T e^{2 \pi i \xi \cdot x} = m(\xi) e^{2 \pi i \xi \cdot x}$ for each $\xi \in \RR^d$.
\end{itemize}

  Every translation invariant operator on $\RR^d$ is a Fourier multiplier operator, and every rotation invariant operator on $S^d$ is a multiplier of the spherical harmonic expansion on $S^d$.



The natural analogue of the study of radial multipliers on $\RR^d$ is the study of multipliers of a Laplace-Beltrami operator on a Riemannian manifold. The natural analogue of the study of quasiradial multipliers on $\RR^d$ is the study of multipliers of an operator associated with a \emph{Finsler geometry} on the manifold.

\section*{Pattern Avoidance}

How large must a set be before it must contain a certain point configuration? Problems of this flavor have long been studied in various areas of combinatorics. In the last 50 years, analysts have also begun studying analogous problems for infinite subsets $X \subset \RR^d$, where the size of $X$ is measured in terms of a suitable \emph{fractal dimension}, often \emph{Hausdorff dimension}, but also sometimes \emph{Fourier dimension}, the latter of which tending to imply more structure than the former.

Several definite conjectures on problems about the \emph{density} of certain point configurations in sets have been raised, such as the Falconer distance problem. %, which asks if an arbitrary subset $X$ of $[0,1]^d$ with \emph{Hausdorff dimension} exceeding $d/2$, then the set of all distances between pairs of points in $X$ must form a set of positive Lebesgue measure.
But there are relatively few definite conjectures about the dimension a set requires before it must contain \emph{at least one} family of points fitting a certain kind of configuration. For instance, we do not know for $d > 2$ how large the Hausdorff dimension a set $X \subset \RR^d$ must be before it contains all three vertices of an isosceles triangle, the threshold being somewhere between $d/2$ and $d-1$.

 It is not clear

, for instance, how large the Hausdorff dimension a set $X \subset \RR^d$ must have before it contains the vertices of at least one isosceles triangle, or, for a particular angle $\theta \in [0,\pi]$, how large $X$ must be before it contains three points $A$, $B$, and $C$ which when connected form an angle $\theta$; the only case here that is fully resolved is when $\theta \in \{ 0, \pi \}$, or when $\theta = \pi/2$ and $d$ is even: when $\theta = 0$ and $\theta = \pi$, the threshold is $d-1$, when $\theta = \pi/2$, the threshold is somewhere between $d/2$ and $\lceil d/2 \rceil$, for rational $\theta$ the threshold is somewhere between $d/4$ and $d-1$, and when $\theta$ is irrational the threshold is somewhere between $d/8$ and $d-1$.

Until recently, certain results \cite{LabaPramanik} seemed to indicate that subsets of $[0,1]$ of Fourier dimension one must necessarily contain an arithmetic progression of length three, but this has proved not to be the case \cite{Schmerkin}.

The ability to form definite conjectures depends on the ability to produce counterexamples for certain problems. In this case, counterexamples take the form of constructing sets with large fractal dimension that \emph{do not} contain certain point configurations. My research in geometric measure theory has so far focused on this type of problem.

During my MSc, my advisors and I found a construction that produces sets $X$ with large Hausdorff dimension that avoid a particular configuration, given that the particular configuration is 'small' \cite{DensonPramanikZahl}. More precisely, let us suppose we are looking at configurations of $k$ points in $\RR^d$. The set of all tuples of points that fit a given configuration can be identified with a subset $C$ of $(\RR^d)^k$.

Provided that the Minkowski dimension of $C$ is at most $\beta$, we constructed a set $X$ with Hausdorff dimension $(dk - \beta)/(k-1)$ such that if $x_1,\dots,x_k$ are distinct points in $X$, then $(x_1,\dots,x_k) \not \in C$. In particular, for a Lipschitz function $f: (\RR^d)^k \to \RR^d$, we construct a set $X$ with Hausdorff dimension $d/k$ such that for distinct $x_0,\dots,x_k \in X$, $x_0 \neq f(x_1,\dots,x_k)$, recovering the main result of \cite{FraserPramanik}.



During my PhD, I decided to investigate whether 

. My research in geometric measure theory so far has been on trying to produce such counterexamples. In BLAH, Pramanik and Fraser. In BLAH, I rephrased their argument in probabilistic terms 

 of a set must be before the set of all distances 

Several definite conjectures on problems of this kind have been established since the project begun, such as the Falconer distance problem or Kakeya conjecture, where the point configuration in mind are points lying at a certain distance from one another, or line segments pointing in other directions. For other

The natural fractal dimension used to measure the size of a set $X$ is often the Hausdorff dimension $\dim_{\mathbb{H}}(X)$ of $X$. But sometimes the \emph{Fourier dimension} $\dim_{\mathbb{F}}(X)$ proves useful, which measures the best possible decay that the Fourier transform of measures supported on $X$ can have; if $\alpha < \dim_{\mathbb{F}}(X)$, then there exists a nonzero measure $\mu$ on $X$ such that $|\widehat{\mu}(\xi)| \lesssim |\xi|^{-\alpha}$ for all $\xi \in \RR^d$. The Fourier dimension thus, morally speaking, measures how uncorrelated the set $X$ is with the Fourier characters $e_\xi(x) = e^{2 \pi i \xi \cdot x}$.

\section*{Future Lines of Research}

The work I have conducted naturally suggests several {\bf future problems}.
%
\begin{itemize}
	\item Analyzing the 'return time operator' to extend results on expansions of spherical harmonics to the study of the Laplace-Beltrami operator on $S^d$.

	\item Determining whether our methods extend to other manifolds whose geodesic flow is simpler to understand, such as integrable systems.

	\item Analyzing whether local smoothing bounds

	\item Constructing Random Salem Sets which satisfy a Decoupling Bound.

	\item Determining the relation between certain 'fractal weighted estimates' for the wave equation on $\RR^d$ and the 'density decomposition' of multiplier bounds.
\end{itemize}

In fact, this resemblance opens up a whole new world of families of operators. Given an arbitrary elliptic self-adjoint first order classical pseudo-differential operator $P$

This method is highly robust and depends very little that we are working on the sphere; pretty much the only property we end up using is that the wave equation $\partial_t u = P u$ has \emph{periodic solutions}.

\begin{thebibliography}{03}

\bibitem{Cladek} Cladek, Laura,
	\emph{Radial {F}ourier Multipliers in $\RR^3$ and $\RR^4$}.

\bibitem{DensonCharacterization} Denson, Jacob,
	\emph{Multipliers of Spherical Harmonics}.

\bibitem{DensonPramanikZahl} Denson, Jacob,
 	\emph{Large Sets Avoiding Rough Patterns}.

\bibitem{DensonFourier} Denson, Jacob,
	\emph{Large Salem Sets Avoiding Nonlinear Configurations}.

\bibitem{LabaPramanik} Laba, Izabella and Pramanik, Malabika,
	\emph{Arithmetic progressions in sets of fractional dimension}.

\bibitem{Schmerkin} Schmerkin, Pablo,
	\emph{{S}alem sets with no arithmetic progressions}.

\bibitem{FraserPramanik} Fraser, Robert and Pramanik, Malabika,
	\emph{Large Sets Avoiding Patterns}.

\bibitem{GarrigosSeeger} Garrigos, Gustavo and Seeger, Andreas,
	\emph{Characterizations of {H}ankel Multipliers}.

\bibitem{HeoNazarovSeeger} Heo, Yaryong and Nazarov, Fedor and Seeger, Andreas,
	\emph{Radial {F}ourier Multipliers in High Dimensions}.

\bibitem{KimQuasiradial} Kim, Jongchon,
	\emph{Endpoint Bounds for Quasiradial {F}ourier Multipliers}.

\bibitem{Cladek} Cladek, Laura,
	\emph{Radial {F}ourier Multipliers in $\mathbb{R}^3$}.

\end{thebibliography}

\end{document}

Such operators initially arose from the study of the classical partial differential equations in physics, and continue to have applications in areas as diverse as partial differential equations, mathematical physics, number theory, and ergodic theory. Every translation invariant operator on $\RR^d$ is a Fourier multiplier operator, and every rotation invariant operator on $S^d$ is a multiplier of the spherical harmonic expansion on $S^d$.

any such operator $T$, we can associate a function $m: \RR^n \to \CC$, known as the \emph{symbol} of $T$, such that for any function $f$, the Fourier transform of $Tf$ obeys the relation $\widehat{Tf} = m \widehat{f}$; thus translation invariant operators are also called \emph{Fourier multiplier operators}.

Understanding the boundedness of Fourier multiplier operators in an $L^p$ norm for $p \neq 2$ underpins any subtle understanding of the Fourier transform. Plane waves oscillating in different directions and with different frequencies are orthogonal to one another, and thus do not interact with one another significantly in terms of the $L^2$ norm, as justified by Bessel's inequality. But plane waves can interact with one another in the $L^p$ norm for $p \neq 2$, and so understanding $L^p$ bounds for Fourier multipliers indicate when this interaction is significant or insignificant. Similarily, spherical harmonics of different degrees on $S^d$ are orthogonal to one another, but studying the $L^p$ bounds of multipliers of the Laplacian on the sphere is crucial to understand when the interactions of different spherical harmonics are significant or not.

The general study of the characterizations of $L^p$ boundedness for the Fourier multipliers was initiated in the 1960s. Mathematicians quickly found simple necessary and sufficient conditions that ensure Fourier multipliers are bounded on $L^1(\RR^d)$, $L^2(\RR^d)$, and $L^\infty(\RR^d)$. But the problem of finding necessary and sufficient conditions for boundedness in $L^p(\RR^d)$ for any other exponent proved impenetrable. Indeed, many interesting problems about the boundedness of \emph{specific} Fourier multipliers, such as the Bochner-Riesz conjecture, remain largely unsolved today.

Thus it came as a surprise in the past decade when results emerged proving necessary and sufficient conditions for \emph{radial} Fourier multipliers to be bounded on $L^p(\RR^d)$. First came the result of BLAH, which gave a necessary and sufficient criteria for bounds of the form $\| Tf \|_{L^p(\RR^d)} \lesssim \| f \|_{L^p(\RR^d)}$ to hold uniformly over \emph{radial functions} $f$, for $|1/p - 1/2| > 1/2d$. An optimist might think this same condition causes the bound to hold uniformly over \emph{all functions} $f$ in the range above, a statement we call the \emph{radial multiplier conjecture}. We now know, by the results of BLAH and BLAH, that the radial multiplier conjecture holds when $d > 4$ and $|1/p - 1/2| > 1/(d-1)$, when $d = 4$ and $|1/p - 1/2| > 11/36$, and when $d = 3$ and $|1/p - 1/2| > 11/26$. But the radial multiplier conjecture has not yet been completely solved in any dimension $d$, and no bounds are known at all when $d = 2$.

The natural analogue of the study of radial multipliers on $\RR^d$ is the study of multipliers of a Laplace-Beltrami operator on a Riemannian manifold. The natural analogue of the study of quasiradial multipliers on $\RR^d$ is the study of multipliers of an operator associated with a \emph{Finsler geometry} on the manifold.