\documentclass[11pt]{article}

\usepackage[a4paper, margin=1in]{geometry}

\usepackage{amssymb,amsmath,amsthm}
\usepackage{mathtools}
\usepackage{amsrefs}
\usepackage{comment}
\usepackage{amsthm}

\newtheorem*{theorem}{Theorem}

\DeclareMathOperator{\QQ}{\mathbb{Q}}
\DeclareMathOperator{\ZZ}{\mathbb{Z}}
\DeclareMathOperator{\RR}{\mathbb{R}}
\DeclareMathOperator{\NN}{\mathbb{N}}
\DeclareMathOperator{\HH}{\mathbb{H}}
\DeclareMathOperator{\BB}{\mathbb{B}}
\DeclareMathOperator{\CC}{\mathbb{C}}
\DeclareMathOperator{\AB}{\mathbb{A}}
\DeclareMathOperator{\PP}{\mathbb{P}}
\DeclareMathOperator{\MM}{\mathbb{M}}
\DeclareMathOperator{\VV}{\mathbb{V}}
\DeclareMathOperator{\TT}{\mathbb{T}}
\DeclareMathOperator{\LL}{\mathcal{L}}
\DeclareMathOperator{\DD}{\mathcal{D}}
\DeclareMathOperator{\SW}{\mathcal{S}}
\DeclareMathOperator{\EC}{\mathcal{E}}
\DeclareMathOperator{\AC}{\mathcal{A}}

\title{\vspace{-3.5em}Research Statement}
\author{Jacob Denson}
\date{}

\begin{document}

\maketitle

\vspace{-1em}

I am an analyst who studies problems using techniques mainly from harmonic analysis, but also some methods of combinatorics and probability theory. My research over the past few years has focused on the study of radial Fourier multiplier operators on Euclidean space, and their analogues on compact manifolds, through an understanding of the geometry and regularity of wave propagation. In addition, I have explored problems in geometric measure theory, investigating when `structure' occurs in fractals of large dimension. Both areas of research have raised interesting questions which I plan to pursue in my postgraduate work.

During my PhD, my work on multipliers has concentrated on the relation between $L^p$ bounds for Fourier multiplier operators on $\RR^d$, and $L^p$ bounds for analogous operators on compact manifolds, such as the family of multiplier operators for spherical harmonic expansions on $S^d$. My main achievement, for $d \geq 4$, and a range of $L^p$ spaces, is a complete characterization of the functions whose dilations correspond to a uniformly family of multiplier operators on $L^p(S^d)$. This proof essentially implies an \emph{transference principle} between bounds for radial multiplier operators on $\RR^d$ and bounds for multiplier operators on $S^d$; the principle says that the $L^p$ boundedness of a radial Fourier multiplier operator implies the $L^p$ boundedness of the multiplier operator on $S^d$ given by the same multiplier. %; thus $L^p$ bounds `transfer' from $\RR^d$ to $S^d$.
The first part of this argument, which prove the result for compactly supported functions $m$, may be found in \cite{DensonCharacterization}, with the remaining part of this argument to be made available shortly. Both results are the first of their kind for multiplier operators on $S^d$ for $p \neq 2$; more broadly, no comparable results have been established for analogous multiplier operators on any other compact manifold. More detail about this project can be found in Section \ref{Section1} of this summary.

%
% S^d -> R^d transference principle, but there is no transference principle in the opposite direction. Therefore the problem on the sphere is strictly harder.
%
% Therefore effectively establishing a transference principle in the range under which the theorem is proven.
%
%

My work in geometric measure theory focuses on constructing sets of large fractal dimension avoiding certain point configurations. Before starting my PhD, I had worked with Malabika Pramanik and Joshua Zahl to construct sets with large Hausdorff dimension avoiding certain point configurations \cite{DensonPramanikZahl}. During my PhD, I continued this line of research by combining the methods of that paper with more robust probabilistic machinery to address the more difficult problem of constructing sets with large Fourier dimension avoiding configurations \cite{DensonFourier}. This method remains the only method of constructing sets of large Fourier dimension avoiding nonlinear configurations, and remains the best current method for constructing sets avoiding general `linear' point configurations when $d > 1$. This work is discussed further in Section \ref{Section2}.

%  Using techniques of high dimensional probability, I was able to fully recover the Hausdorff bound obtained in \cite{DensonPramanikZahl} in the Fourier dimension setting with the additional of a weak linearity assumption. An example application of the method is, for any smooth curve $c: [0,1] \to \RR^d$, a construction of a subset $X$ of $[0,1]$ of Fourier dimension $0.4$ such that $c(X)$ does not contain three vertices of an isosceles triangle. 

In Section 3, after an appropriate discussion of my PhD projects in Sections 1 and 2, I discuss my plans for future research and how they relate the techniques involved in my PhD work. These include generalizing bounds for spherical harmonic expansions to the study of multipliers of the Laplace-Beltrami operator on Riemannian manifolds with periodic flow, and obtaining $l^2(L^p)$ decoupling bounds for random fractal subsets of $\RR$. % TODO: CONCLUSION SENTENCE

\section{Multiplier Operators on Euclidean Space and on Manifolds} \label{Section1}

% on recent developments in the characterization of $L^p$ boundedness for radial Fourier multiplier operators and their relation to bounds for multipliers on the sphere, as well as a more detailed discussion of the bounds I was able to obtain for such operators. % This motivates the results I was able to obtain in this setting, which we discuss in more detail at the end of the section.

%Much of my PhD has dealt with the study of bounds for radial Fourier multipliers on $\RR^d$, bounds for multipliers of eigenfunction expansions on compact manifolds, and the relation between such bounds.

Multiplier operators have long been central objects in harmonic analysis. % since the field's inception.
In his pioneering work, Fourier showed solutions to the classical equations of physics are described by Fourier multiplier operators, operators $T$ defined by a function $m: \RR^d \to \CC$, a `multiplier', by setting

%
\[ Tf(x) = \int_{\RR^d} m(\xi) \widehat{f}(\xi) e^{2 \pi i \xi \cdot x}\; dx. \]
%
Of particular interest are the radial Fourier multiplier operators, defined by a radial function $m$. For a function $a: [0,\infty) \to \CC$, we denote the radial multiplier operator given by $m(\xi) = a(|\xi|)$ by $T_a$. Any operator on $\RR^d$ commuting with translations is a Fourier multiplier, and if in addition, the operator commutes with rotates, it is a radial Fourier multiplier operator. %explaining their broad applicability. % Fourier multiplier operator is a radial multiplier, explaining their broad applicability %,  explaining their broad applicability in areas as diverse as partial differential equations, number theory, complex variables, and ergodic theory.

In harmonic analysis, it has proved incredibly profitable to study the boundedness of Fourier multiplier operators with respect to various $L^p$ norms. It seems to be one of the few tractable ways of quantifying interactions between planar waves, thus underpinning all deeper understandings of the Fourier transform. The need for a general understanding of the $L^p$ boundedness of a general multiplier operator became of central interest in the 1950s, brought on by the spur of applications the Calderon-Zygmund school and their contemporaries brought to the theory. Some sufficient conditions to ensure boundedness were found, and some necessary conditions, but finding necessary \emph{and} sufficient conditions which guarantee $L^p$ boundedness of the corresponding operator proved to be an impenetrable, if not impossible problem. Effective characterizations of $L^p$ boundedness are only known in simple cases, where $p \in \{ 1, 2, \infty \}$.
%
%
% d(1/p - 1/2) derivatives in L^2
% (d-1)(1/p - 1/2) derivatives in L^p
%
% The multiplier m_t(rho) = chi(rho) e^{-it rho}
%
% Has L^p multiplier norm O( t^{(d-1)(1/p - 1/2)} )
%
% If we take s derivatives of m_t, the answer
% 	1sum of terms of the form chi_k(rho) t^k e^{it rho}
% 	where k <= s. Thus the Fourier transform
%
% 	is equal to k_t(t') = t^k chi_k^( t - t' )
% 	The L^p norm of this is O(t^k).
%
% Thus C_p(m_t) << t^{(d-1)(1/p - 1/2)}
%
% But the Fourier transform of the (d-1)(1/p - 1/2) + epsilon
% derivative is O( t^{(d-1)(1/p - 1/2)} )
%
% Thus epsilon loss techniques cannot analysis sum_k 2^{-k(d-1)(1/p - 1/2)} m_{2^k}(rho)
% 
%
%
%
%The general study of the $L^p$ boundedness of general Fourier multipliers began in the 1960s, brought on by the spur of applications the Calderon-Zygmund school and their contemporaries brought to the theory. 
%But aside from certain rather trivial cases where $p \in \{ 1, 2 , \infty \}$, no necessary and sufficient conditions on an operator's symbol to ensure boundedness on $L^p(\RR^d)$ have been found.
%

It thus came as a surprise when several recent arguments \cites{GarrigosSeeger,HeoNazarovSeeger,Cladek,KimQuasiradial} emerged giving necessary and sufficient conditions on a symbol $a$ for a radial Fourier multiplier operator $T_a$ to be bounded on $L^p(\RR^d)$.
%By duality, a Fourier multiplier is bounded on $L^p(\RR^d)$ if and only if it is bounded on $L^{p'}(\RR^d)$, so it is sufficient to consider the case $1 \leq p \leq 2$.
Consider a decomposition $a(\rho) = \sum a_k( \rho / 2^k)$, where $a_k(\rho) = 0$ for $\rho \not \in [1,2]$. For $1 \leq p \leq 2$, in order for $T_a$ to be bounded on $L^p(\RR^d)$ it must be true that the $L^p$ norms of the Fourier transforms of the functions $m_k(\xi) = a_k(|\xi|)$ are uniformly bounded in $k$. Garrigos and Seeger \cite{GarrigosSeeger} show this is equivalent to $\sup_j C_p(a_j) < \infty$, where
%
\[ C_p(a) = \left( \int_0^\infty \big| (1 + |t|)^{(d-1)(1/p - 1/2)} \widehat{a}(t) \big|^p\; dt \right)^{1/p}. \] %\quad\text{and}\quad \alpha(p) = (d-1)(1/p - 1/2). \]
%
%and where $\widehat{a}(t) = \int_0^\infty a(\rho) e^{2 \pi i \rho t}\; dt$ is the \emph{cosine transform} of the function $a$.
% 1/p - 1/2 = 29/36 - 1/2 = 11/36
%Duality implies the boundedness of $T_a$ on $L^p(\RR^d)$ is equivalent to it's boundedness on $L^{p'}(\RR^d)$ when $1/p + 1/p' = 1$, and so for $2 \leq p \leq \infty$ it is natural to define $C_p(a) = C_{p'}(a)$.
Using the Bochner-Riesz operators as endpoint examples, it is natural to conjecture the condition $\sup_j C_p(a_j) < \infty$ is not only necessary, but also \emph{sufficient} to guarantee $L^p$ boundedness for $1 < p < 2d/(d+1)$. For radial input functions this conjecture is true \cite{GarrigosSeeger}, though resolving this conjecture for general inputs is likely far beyond current research techniques, given that it implies the Bochner-Riesz conjecture, and thus also the restriction and Kakeya conjectures.

Heo, Nazarov, and Seeger \cite{HeoNazarovSeeger} have proved the conjecture for $d \geq 4$ and $1 < p < 2 {\scriptstyle \left( \frac{d-1}{d+1} \right)}$. Cladek \cite{Cladek} improved the range of the conjecture for compactly supported $a$. She proved the result when $d = 4$ and $1 < p < 36/29$, and when $d = 3$ and $1 < p < 13/12$. Also of note is the work of Kim \cite{KimQuasiradial}, who extended the bounds of \cite{HeoNazarovSeeger} to \emph{quasi-radial multiplier operators}, defined by a Fourier multiplier $m$ which is homogeneous, smooth, and non-vanishing, and whose level sets are hypersurfaces of non-vanishing Gauss curvature. Nonetheless, the full conjecture is not completely resolved for all $d \geq 2$.

We remark that various high powered techniques have recently been developed towards an understanding of the Bochner-Riesz conjecture, such as broad-narrow analysis, decoupling, and the polynomial method. However, these methods are difficult to apply in the conjectures formulated above, since they are \emph{endpoint results}. More precisely, in these methods, one allows for inequalities to have a multiplicative loss of factors of the form $R^\varepsilon$ or $\log R$, where $R$ is the frequency scale of the analysis. This is negligible to the analysis, since the Bochner-Riesz multipliers are conjectured to be bounded on $L^p$ for an \emph{open} interval of exponents, and so methods involving an interpolation between $L^p$ spaces allow us to remove these multiplicative factors. But an arbitrary multiplier bounded on $L^p(\RR^d)$ may not be bounded on $L^{p'}(\RR^d)$ for any $p' < p$, and so such methods are unavailable to us in the study of these conjectures, partially explaining the limited range in which the conjectures have been verified.

\subsection{Multipliers For Spherical Harmonic Expansions on $S^d$}

A theory of multiplier operators analogous to Fourier multiplier operators can be developed on the sphere $S^d$. Roughly speaking, Fourier multiplier operators are essentially operators on $\RR^d$ with $e^{2 \pi i \xi \cdot x}$ as eigenfunctions. Multipliers on $S^d$ are those operators with the \emph{spherical harmonics} as eigenfunctions, i.e. the restrictions to $S^d$ of homogeneous harmonic polynomials on $\RR^{d+1}$. Every function $f \in L^2(S^d)$ can be uniquely expanded as $\sum_{k = 0}^\infty H_k f$, where $H_k f$ is a degree $k$ spherical harmonics. A multiplier for spherical harmonic expansions on $S^d$ is then an operator defined in terms of a function $a: \NN \to \CC$ given by $S_a = \sum_{k = 0}^\infty a(k) H_k$.
%
%\[ S_a f = \sum\nolimits_{k = 0}^\infty a(k) H_k f. \]
%
%Just as any translation-invariant operator on $\RR^d$ is a Fourier multiplier,
For purposes of brevity, we will call such operators `multiplier operators on $S^d$' in the sequel. Every rotation invariant operator on $S^d$ is a multiplier. %, and thus such operators arise in diverse applications, including celestial mechanics, physics, and computer graphics.
A natural question is to characterize which functions $a$ give multiplier operators $S_a$ bounded on $L^p(S^d)$, but the fact that the operators are described by a sum, which is discrete, makes this problem tricky. A more tractable question is to determine when the operators $S_R = \sum a(k/R) H_k$ are \emph{uniformly} bounded on $L^p(S^d)$. I have completely characterized such functions, for a certain range of $L^p$ exponents and when $d \geq 4$.

%one exception is available if we weaken our assumptions, i.e. if we asume that $\sup\nolimits_j C_{p-\varepsilon}(a_j) < \infty$ or if the Besov norm $B^{2,p}_{d(1/p - 1/2) + \varepsilon}$ is finite, and progress has been made in this area. In this case the $\varepsilon$-variant of Besov norm conjecture has been completely resolved when $d = 2$ by Carbery \cite{Carbery}, though the higher dimensional cases remain unsolved.
% q = 2
% alpha > d|1/p - 1/2|


%But most of these techniques involving induction on scales, such as broad-narrow analysis or decoupling, or even basic applications of dyadic pigeonholing, cannot be used in characterizations of $L^p$ bounded multipliers; such methods must allow for a $R^\varepsilon$ or $\log R$ loss in the frequency scale $R$, which is permissible in the analysis of Bochner-Riesz multipliers given they lie in a range of Besov spaces, which allow for an epsilon of room at the dyadic scale when interpolating. But an arbitrary multiplier bounded on $L^p$ does not have such a property, and indeed may not even lie in any of the standard Besov spaces (for instance, consider the radial multipliers with symbol $a_t(\rho) = t^{-(d-1)(1/p-1/2)} \chi(\rho) e^{it \rho}$, which are uniformly bounded on $L^p(\RR^d)$).

Classical methods for studying multiplier operators on $S^d$ involve the analysis of special functions and orthogonal polynomials, e.g. in the work of Bonami and Clerc \cite{BonamiClerc}. But in the 1960s, H\"{o}rmander introduced the powerful theory of Fourier integral operators to the study of such operators, which allows one to apply more modern techniques of harmonic analysis the theory. This theory is more robust in other senses, applying to the study of multiplier operators associated with a first order self-adjoint pseudodifferential operator on a compact manifold, which we briefly outline. Given such an operator $P$ on a manifold $M$, if $\Lambda$ is the set of eigenvalues of $P$, then every function $f \in L^2(M)$ has an orthogonal decomposition $f = \sum_{\lambda \in \Lambda} f_\lambda$ where $Pf_\lambda = \lambda f_\lambda$. Given $a: \Lambda \to \CC$, we define
%
\[ a(P) f = \sum\nolimits_{\lambda \in \Lambda} a(\lambda) f_\lambda. \]
%
The operators $a(P)$ are thus multiplier operators for the eigenfunction expansion of $P$.

We study the multiplier operators $S^d$ by linking them to multiplier operators of a particular pseudodifferential operator $P$ on $S^d$. If $\Delta$ is the Laplace-Beltrami operator on $S^d$, then for any spherical harmonic $f$ of degree $k$, $\Delta f = k(k+d-1) f$. Thus if $P = \sqrt{ ( {\scriptstyle \frac{d-1}{2} } )^2 - \Delta }$, where $\alpha = (d-1)/2$, then $Pf = kf$, and so for any function $a: [0,\infty) \to \CC$, $S_a = a(P)$.

H\"{o}rmander's idea to studying the operator $a(P)$ was to write
%
\[ a(P) = \int \widehat{a}(t) e^{2 \pi i t P}\; dt, \]
%
a form of the Fourier inversion formula. The multiplier operators $e^{2 \pi i t P}$, as $t$ varies, give solutions to the \emph{half-wave equation} $\partial_t = i P$ on the manifold $M$, whose solutions allow one to describe solutions to the full wave equation $\partial_t^2 - P^2 = 0$. Thus the study of the boundedness of the operator $a(P)$ is connected to the regularity of the wave equation on $M$. %In particular, multipliers on $S^d$ are related to the wave equation $\partial_t^2 - \Delta = \alpha^2$ on $S^d$.
%We note that this method also connects the study of multipliers on $S^d$ to radial Fourier multipliers, since we can also write $T_a$ as $a(\sqrt{-\Delta})$, where $\Delta$ is the usual Laplacian on $\RR^d$.

Using this reduction, H\"{o}rmander was able to prove the $L^p$ boundedness of Bochner-Riesz operators \cite{HormanderRiesz}, later significantly improved by Sogge \cites{SoggeSpherical,SoggeRieszMeans} and Seeger and Sogge \cite{SeegerSoggeBochnerRiesz} for multipliers of an operator $P$ satisfying the following assumption:
%
\begin{center}
\fbox{%
    \parbox{\textwidth - 2em}{%
        {\bf Assumption A}: If $p_{\text{prin}}: T^* M \to [0,\infty)$ is the principal symbol of $P$, then for each $x \in M$ the `cosphere' $S_x^* = \{ \xi \in T^*_x M : p_{\text{prin}}(x,\xi) = 1 \}$ has non-vanishing Gaussian curvature.
    }%
}
\end{center}
%
Note that when $P = ( \left( \scriptstyle{ \frac{d-1}{2} } \right)^2 - \Delta )^{1/2}$ on $S^d$, the principal symbol is the Riemannian metric norm on $T^* M$, the cospheres are ellipses, and thus Assumption A is satisfied. These bounds were obtained by introducing the approach, which works within the Stein-Tomas range, of reducing the problem to $L^2(M) \to L^p(M)$ bounds for spectral projection operators on $M$. Recently, Kim \cite{KimManifold} adapted Sogge's approach to obtain certain necessary conditions ensuring $a(P)$ is bounded on $L^p(M)$ for operators $P$ satisfying Assumption A. But these bounds are far from a complete characterization of boundedness; they do not even imply the uniform boundedness of the wave multipliers $\chi(P) (1 + P)^{-(d-1)(1/p - 1/2)} e^{2 \pi i P}$. \emph{The main goal of my research project was to find such characterizations, which would be the first such result in the literature}.

% $(1 + |t|)^{-(d-1)(1/p - 1/2)} \chi( P ) e^{2 \pi i t P}$ on $L^p(M)$. My goal was
% Multiplier is chi(xi) e^{2 pi i t xi}
% Fourier transform is chi^( * - t )

% analyze multipliers of an operator $P$ satisfying the following assumption:

%Kim proved that under Assumption A, in the Stein-Tomas range $|1/p - 1/2| > (d+1)^{-1}$, if $\sup_j \| a_j \|_{B_p(\RR)} < \infty$, then the operator $a(P)$ is bounded on $L^p(M)$, thus obtaining an analogue of the result of Lee, Rogers and Seeger in this setting.

% However, there are no results in the literature which show that an operator $a(P)$ is bounded on $L^p(M)$ if $\sup_j C_p(a_j) < \infty$, for any $p$ and any manifold $M$. \emph{The main goal of my research project was to remedy this}.`'

\subsection{My Contributions To The Study of Multipliers} \label{ContributionMultipliers}

As mentioned above, the main goal of my PhD research into multipliers was to obtain analogues of the arguments of \cite{HeoNazarovSeeger,Cladek,KimQuasiradial} for multiplier operators on $S^d$, i.e. proving that for $P = ( ({\scriptstyle \frac{d-1}{2} })^2 - \Delta )^{1/2}$ on $S^d$, then the operators $\{ a(P/R) \}$ are uniformly bounded on $L^p(S)^d$ if and only if $\sup_j C_p(a_j) < \infty$. I obtained such analogues, and more generally applying to multipliers for a range of different operators $P$ that satisfy Assumption A and the following additional assumption:
%
\begin{center}
\fbox{%
    \parbox{\textwidth - 5em}{%
        {\bf Assumption B}: The eigenvalues of $P$ are contained in an arithmetic progression.
    }%
}
\end{center}
%
All eigenvalues of the operator $P$ above are positive integers, so this assumption is satisfied on $S^d$. The assumption also holds more generally for multipliers on the \emph{rank one symmetric spaces} $\RR \PP^d$, $\CC \PP^d$, $\HH \PP^d$, and $\mathbb{O} \PP^2$, i.e. operators diagonalized by analogous functions to the spherical harmonics on these spaces. It is very difficult to completely remove Assumption B, for reasons involving the inability to understand the large time behavior of the wave equation on compact manifolds. Nonetheless, in Section 3 I discuss potential methods for obtaining similar bounds under weaker assumptions. Under Assumption A and Assumption B, in \cite{DensonCharacterization} I proved a 'single scale' analogue of the bound of Heo, Nazarov and Seeger.

\begin{theorem} \cite{DensonCharacterization}
	Suppose $P$ is a first order, self-adjoint pseudodifferential operator of order one on a manifold $M$ satisfying Assumptions A and B. Then for a function $a$ supported on $[1,2]$, and for $1 < p < 2 ({\scriptstyle \frac{d-1}{d+1}})$, uniformly in $R > 0$, $\| a(P/R) f \|_{L^p(M)} \lesssim C_p(a) \| f \|_{L^p(M)}$.
\end{theorem}

In a paper to be submitted for publication shortly, I provide further arguments justifying that for an arbitrary function $a$, the operator $a(P)$ is bounded on $L^p(M)$ if $\sup_j C_p(a_j) < \infty$, thus obtaining a complete analogue of the argument of \cite{HeoNazarovSeeger} for multiplier operators on $S^d$.

An important corollary of this result is a \emph{transference principle} between Fourier multipliers and multiplier operators on $S^d$. Since the condition $\sup_j C_p(a_j)$ is necessary for $T_a$ to be bounded on $L^p(\RR^d)$, we conclude that for $|1/p - 1/2| > 1/d$, if $T_a$ is bounded on $L^p(\RR^d)$, then the multiplier $a(P)$ is bounded on $L^p(M)$. Aside from the study of Fourier multipliers on $\RR^d$, this is the first transference principle of this kind. There are no results in the literature for any $p \neq 2$, any other compact manifold $M$, and any operator $P$ which guarantee that $a(P)$ is bounded on $L^p(M)$ if $T_a$ is bounded on $L^p(\RR^d)$.

Another corollary is a characterization of the functions $a$ such that multipliers of the form $\{ a(P/R) : R > 0 \}$ are uniformly bound on $L^p(M)$. If $\sup_j C_p(a_j) < \infty$, then the results above imply that the operators $a(P/R)$ are uniformly bounded on $L^p(M)$, because the quantity $\sup_j C_p(a_j)$ changes by at most a constant when we dilate $a$ by a factor of $R$. The converse follows from a classic result of Mitjagin \cite{Mitjagin}.
% which implies that if a family of multipliers of the form $\{ a(P/2^j) : j > 0 \}$ are uniformly bounded on $L^p(S^d)$, and $P$ has principal symbol $p_{\text{prin}}$, then the Fourier multiplier $T$ with symbol $a \circ p_{\text{prin}}$ is bounded on $L^p(\RR^d)$. It follows that the operators $a(P/2^j)$ cannot be uniformly bounded on $L^p(M)$ unless $\sup_j C_p(a_j)$ to be finite. But since the quantity $\sup_j C_p(a_j)$ is \emph{scale invariant} (it is unchanged if we dilate $a$ by a factor of $2^j$), the bounds discussed above allow us to conclude that the operators $a(P/2^j)$ are uniformly bounded on $L^p(M)$ if and only if $\sup_j C_p(a_j) < \infty$ for $|1/p - 1/2| > (d-1)^{-1}$.
The uniform boundedness principle implies that a function $a$ satisfies $\lim_{R \to \infty} a(P/R) f = f$ for all $f \in L^p(M)$, where the limit is taken in $L^p(M)$, if and only if $a(0) = 1$ and $\sup_j C_p(a_j) < \infty$. As for the transference principle above, these results are the first of their kind for any $p \neq 2$ and any other compact manifold $M$.

As mentioned above, \cite{DensonCharacterization} only covers a `single frequency scale' analogue of the results of \cite{HeoNazarovSeeger}. The argument in \cite{HeoNazarovSeeger} for combining scales involves decomposing inputs using an $L^\infty$ atomic decomposition à la the decompositions of Chang and Fefferman \cite{ChangFefferman}, and then controlling the interactions between different frequency scales using certain inner product estimates. We have obtained analogues of these inner product estimates, discussed below, and the atomic decomposition method generalizes to an arbitrary compact manifold, and so we expect to submit a paper describing these methods, and obtaining the full method above, shortly.

% combining scales by using inner product estimates $\langle k_\tau * f_\theta, k_{\tau'} * f_{\theta'} \rangle$. Since we have obtained a generalization of these inner product estimates already in the manifold case in \cite{DensonCharacterization}, and Peetre square function and it's resulting atomic decomposition have direct analogues on compact manifolds.

I would like to finish this section by emphasizing two novel methods I introduced in \cite{DensonCharacterization}. The proof is an adaption of the argument of \cite{HeoNazarovSeeger} for bounding radial Fourier multiplier operator. This proof involves writing the operator as a convolution $Tf = k * f$. We consider a decomposition $k = \sum k_\tau$ and $f = \sum f_\theta$, where the functions $\{ k_\tau \}$ are supported on disjoint annuli supported at the origin, and the functions $\{ f_\theta \}$ are supported on disjoint cubes, and thus we must write $T f = \sum_{\tau,\theta} k_\tau * f_\theta$. Estimates are established for the inner products $\langle k_\tau * f_\theta, k_{\tau'} * f_{\theta'} \rangle$, which are negligible unless the annulus of radius $\tau$ centered at $\theta$ is near tangent to the annulus of radius $\tau'$ centered at $\theta'$ at some point. These inner product estimates are combined with a 'sparse incidence argument' which when interpolated, yields the required $L^p$ bounds. The main difficulty in adapting this approach is the difficulty in obtaining analogous inner product estimates, and handling the case where $\tau$ is large, and I developed two main techniques for dealing with this problem.

% which heavily inspired the method of \cite{DensonCharacterization}:

%  \medskip
%  \hangindent0.5em
%  \hangafter=0
%    \noindent{\emph{Let $T$ be a radial multiplier. Then we can write $T f = k * f$, where $k$ is the Fourier transform of the symbol of $a$. We write $k = \sum k_\tau$ and $f = \sum f_\theta$, where the functions $\{ k_\tau \}$ are supported on disjoint annuli supported at the origin, and the functions $\{ f_\theta \}$ are supported on disjoint cubes. Then $T f = \sum_{\tau,\theta} k_\tau * f_\theta$. Using the Fourier transform and Bessel functions,  we can justify that the inner product $\langle k_\tau * f_\theta, k_{\tau'} * f_{\theta'} \rangle$ is negligible unless the annulus of radius $\tau$ centered at $\theta$ is near tangent to the annulus of radius $\tau'$ centered at $\theta'$ at some point. Combining this inner product estimate with a `sparse incidence argument' for such annuli, one can show that the $L^2$ norm of a sum $\sum_{(\tau,\theta) \in \mathcal{E}} k_\tau * f_\theta$ is well behaved if $\mathcal{E}$ is a suitably sparse set. Interpolation with a trivial $L^1$ estimate yields an $L^p$ estimate on the sum. Conversely, if the set $\mathcal{E}$ is clustered, then $\sum_{(\tau,\theta) \in \mathcal{E}} k_\tau * f_\theta$ will be concentrated on only a few annuli, and so we can also get good $L^p$ estimates simply using pointwise estimates. But then $\| Tf \|_{L^p(\RR^d)} = \| \sum k_\tau * f_\theta \|_{L^p(\RR^d)}$ can be estimated, depending on whether a sparse or clustered part of the sum dominates.}}
%\vspace{0.5em}

A natural approach to obtaining analogous inner product estimates for the inner products $\langle k_\tau * f_\theta, k_{\tau'} * f_{\theta'} \rangle$ is to use the Lax-H\"{o}rmander parametrix for the wave equation, which reduces our inner product estimates \emph{for small $\tau$} to a bound for oscillatory integrals. But the phase of this integral that arises from this parametric non-explicit, given in terms of a solution to an eikonal equation on $M$. One novel approach I made in \cite{DensonCharacterization} was making the observation that if Assumption A holds, then $P$ gives an implicit geometric structure to the manifold $M$, turning it into a \emph{Finsler manifold}. %Finsler manifolds are like Riemannian manifolds, though instead of a smooth choice of inner products being fixed on the tangent spaces of the manifold, in Finsler geometry a smooth choice of strictly convex vector space norms are given on the tangent spaces of the manifold.
The phase of the oscillatory integral occurring from the Lax-H\"{o}rmander parametrix is then directly related to the length of certain geodesics on this Finsler manifold, and using the Finsler analogue of the second variation formula for geodesics, I was able to obtain the required inner product estimates that occur in \cite{HeoNazarovSeeger} for small $\tau$. Such estimates apply to multipliers of an arbitrary pseudodifferential operator $P$ satisfying Assumption A, and likely have applications in other problems.

The inner product estimates above are sufficcient to obtain bounds for small $\tau$, but for large $\tau$ this approach fails as the Lax-H\"{o}rmander parametrix breaks down past the \emph{injectivity radius} of the manifold $M$, preventing us from applying a direct analogue of the arguments of \cite{HeoNazarovSeeger}. Similar problems emerge in other approaches to the study of multipliers on manifolds. This was the impetus for Sogge's method of studying Bochner-Riesz multipliers in the Tomas Stein ranges, which reduces the problem to the study of $L^p \to L^2$ bounds for spectral projection operators on $M$, used in \cite{SoggeRieszMeans} and \cite{KimManifold}. % since various heuristics related to the `Ehrenfast time' imply that such operators should still be amenable to study through oscillatory integrals.
However, we cannot use this method in this problem, since the method initially involves using the estimate $\| a(P) f \|_{L^p(M)} \lesssim \| a(P) f \|_{L^2(M)}$, which is too inefficient to fully characterize the required $L^p$ estimates.
% Sobolev embedding heuristics suggest this inequality incurs a lost of $1/p - 1/2$ derivatives, which is fine for obtaining bounds under the assumption that the functions $a_j$ uniformly have $d(1/p - 1/2)$ derivatives in $L^2$, as in \cite{KimManifold}, but not for $\sup_j C_p(a_j) < \infty$, when the functions $a_j$ are only guaranteed to have $(d-1)(1/p - 1/2)$ derivatives in $L^p$.
I was able to work around this problem by reducing the required bounds to certain $L^p_x L^p_t$ estimates for the wave equation on the manifold. Such an argument behaves somewhat like Sogge's spectral projection argument, but does not involve a switch to $L^2$ norms, avoiding the problems of such an approach. The catch is that $L^p_x L^p_t$ estimates for the wave equation, related to the phenomenon of local smoothing on manifolds, are not as well understand as spectral projectors. This is why we must assume the rather strict Assumption B, which makes such estimates feasible. As discussed later in Section 3.1, in future work I hope investigate ways to weaken Assumption B.

%and thus smoothing bounds for the wave equation are equivalent to small time \emph{local smoothing} estimates for the wave equation, which gives us the required bounds in this settting.

\section{Configuration Avoidance} \label{Section2}

How large must a set $X \subset \RR^d$ be before it must contain a certain point configuration, such as three points forming a triangle congruent to a given triangle, or four points forming a parallelogram? Problems of this flavor have long been studied in combinatorics, such as when $X$ is restricted to a discrete set, such as the grid $\{ 1, \dots, N \}^d$. In the last 50 years, analysts have also begun studying analogous problems for infinite subsets $X \subset \RR^d$, where the size of $X$ is measured via a suitable \emph{fractal dimension}, one of various different numerical statistics which measure how `spread out' $X$ is in space. The most common fractal dimension in use is the \emph{Hausdorff dimension} of a set $X$, but we also consider the \emph{Fourier dimension} as a refinement of Hausdorff dimension which takes into account more subtle behavior of $X$ related to it's correlation with the planar waves $e^{2 \pi i \xi \cdot x}$ for $\xi \in \RR^d$. 

Unlike many other problems in harmonic analysis, we often do not have good expected \emph{lower} bounds for the dimension at which configurations must appear.
%for the threshold dimension $s_*$ corresponding to a given configuration, such that sets with dimension exceeding $s_*$ must contain a given configuration, and such that sets with dimension less than $s_*$ need not contain a given configuration.
%
%for the threshold at which we mu
%Several definite conjectures exist in this area, such as the \emph{Falconer distance set conjecture}, which predicts that if an arbitrary subset $X$ of $[0,1]^d$ with Hausdorff dimension exceeding $d/2$, then the set of all distances between pairs of points in $X$ must form a subset of $\RR$ with positive Lebesgue measure. But these are far and few between. For most kinds of configurations, the expected lower bound at which we can predict configurations is unknown.
%Several definite conjectures on problems about the \emph{density} of certain point configurations in sets have been raised, such as the Falconer distance problem. %, which asks if an arbitrary subset $X$ of $[0,1]^d$ with \emph{Hausdorff dimension} exceeding $d/2$, then the set of all distances between pairs of points in $X$ must form a set of positive Lebesgue measure.
%But there are relatively few definite conjectures about the dimension a set requires before it must contain \emph{at least one} family of points fitting a certain kind of configuration.
For instance, we do not know for $d > 2$ how large the Hausdorff dimension a set $X \subset \RR^d$ must be before it contains all three vertices of an isosceles triangle, the threshold being somewhere between $d/2$ and $d-1$. Similarly, for a fixed angle $\theta \in (0,\pi)$, we do not know how large the Hausdorff dimension of $X$ must be contains three distinct points $A$, $B$, and $C$ which when connected determine an angle $ABC$ equal to $\theta$. If $\cos^2 \theta$ is rational, the results of M\'{a}the \cite{Mathe} and Harangi, Keleti, Kiss, Maga, M\'{a}the, Mattila, and Strenner \cite{Harangi} imply the threshold is somewhere between $d/4$  and $d-1$. If $\cos^2 \theta$ is irrational, the threshold is somewhere between $d/8$ and $d-1$. 
% The only case here that is fully resolved is when $\theta \in \{ 0, \pi \}$, or when $\theta = \pi/2$ and $d$ is even: when $\theta = 0$ and $\theta = \pi$, the threshold is $d-1$, when $\theta = \pi/2$, the threshold is between $d/2$ and $\lceil d/2 \rceil$, for $\theta$ such that $\cos^2 \theta \in \QQ$ the threshold is somewhere between $d/4$ and $d-1$, and in all other cases the threshold is somewhere between $d/8$ and $d-1$. These results all follow from the results of \cite{Harangi,Mathe}.
We should not even necessarily expect currently known lower bounds to be the 'correct bounds' in these problems, as we do with other problems in harmonic analysis, such as the restriction conjecture and the Falconer distance problem; Until recently, certain results due to {\L}aba and Pramanik \cite{LabaPramanik} seemed to imply that subsets of $[0,1]$ of Fourier dimension one must necessarily contain an arithmetic progression of length three, but Shmerkin has shown this need not be the case \cite{Shmerkin}.

Given that we do not have good lower bounds with which to make definite conjectures, it is of interest to find general methods that we can use to produce counterexamples in these types of problems. That is, we wish to find methods with which to construct sets with large fractal dimension that \emph{do not} contain certain point configurations. My research in geometric measure theory has so far focused on finding these types of methods.

\subsection{A Review of Hausdorff Dimension and Configuration Avoidance}

%Let us now be a little more precise, and discuss work on constructing sets with large Hausdorff dimension avoiding configurations. The Hausdorff dimension of a set can be defined in terms of finite Borel measures on $X$, based on the work of Frostman. The Hausdorff dimension of $X$ is the least upper bound of the quantities $s$ for which there exists a finite measure $\mu$ supported on $X$ such that $\mu(B_r) \lesssim r^s$ for all $r > 0$ and all radius $r$ balls $B_r \subset \RR^d$,  Intuitively, if the condition $\mu(B_r) \lesssim r^s$ holds for a large $s$, then $\mu$ must have mass `spread out' over a larger set, and this is only possible while being supported on $X$ if $X$ itself is spread out. 
%If $\mu$ is a finite Borel measure, and there is $s > 0$ such that $|\widehat{\mu}(\xi)| \lesssim |\xi|^{-s/2}$ for all $\xi \in \RR^d$, then one can show $\mu(B_r) \lesssim r^{s - \varepsilon}$ for all $\varepsilon > 0$, and so the support of $\mu$ has Hausdorff dimension at least $s$. This motivates us to consider the \emph{Fourier dimension} $\dim_{\mathbb{F}}(X)$ of a set $X$, which is the least upper bound of the quantities $s$ for which these exists a finite Borel measure $\mu$ supported on $X$ with $|\widehat{\mu}(\xi)| \lesssim |\xi|^{-s/2}$

%and the Fourier dimension is the least upper bound of $s$ such that there is a measure $\mu$ supported on $X$ with $|\widehat{\mu}(\xi)| \lesssim |\xi|^{-s/2}$.
%Similarily, if $|\widehat{\mu}(\xi)| \lesssim |\xi|^{-s/2}$, then $\mu$ cannot be correlated with high frequency waves, which also requires the measure $\mu$ be spread out.

%Remarkably, the minimum dimension required to ensure a configuration exists can depend on the choice of dimension used. The Fourier dimension is always smaller than the Hausdorff dimension, but for many explicit sets $X$ this inequality is strict, with the Fourier dimension somehow measuring slightly more information about how correlated $X$ is with planar waves. This subtle difference can emerge in the study of configurations. For instance, for each $k > 0$, a construction of Keleti \cite{Keleti} shows there exists a set $X \subset [0,1]$ with $\dim_{\mathbb{H}}(X) = 1$ such that for all distinct $x_1,\dots,x_k \in X$, and any integers $a_0,\dots,a_k \in \ZZ$, one has $a_1x_1 + \dots + a_nx_n \neq a_0$. On the other hand, any set $X \subset [0,1]$ with $\dim_{\mathbb{F}}(X) > 2/k$ must necessarily contain distinct points $x_1,\dots,x_k \in X$ such that for some integers $a_0,\dots,a_k \in \ZZ$, one has $a_1x_1 + \dots + a_nx_n = a_0$.	 %sa result of Rudin \cite{Rudin}.



%
% 
% { x, y, z, w }
% x = a
% y = a+b
% z = a+c
% w = a+b+c
% y + z = w + x


%
%\begin{itemize}
%	\item The most natural dimension in this context is the \emph{Hausdorff dimension} $\dim_{\mathbb{H}}(X)$ of the set $X$, which can be defined as the least upper bound of the quantities $s$ for which there exists a finite Borel measure $\mu$ supported on $X$ such that $\mu(B_r) \lesssim r^s$ for all $r > 0$ and all radius $r$ balls $B_r \subset \RR^d$. Intuitively, if $s$ is large, $\mu$ has mass `spread out' over a larger set, and this is only possible while being supported on $X$ if $X$ itself is spread out.

%	\item If $\mu$ is a finite Borel measure, and there is $s > 0$ such that $|\widehat{\mu}(\xi)| \lesssim |\xi|^{-s/2}$ for all $\xi \in \RR^d$, then one can show $\mu(B_r) \lesssim r^{s - \varepsilon}$ for all $\varepsilon > 0$, and so the support of $\mu$ has Hausdorff dimension at least $s$. This motivates us to consider the \emph{Fourier dimension} $\dim_{\mathbb{F}}(X)$ of a set $X$, which is the least upper bound of the quantities $s$ for which these exists a finite Borel measure $\mu$ supported on $X$ with $|\widehat{\mu}(\xi)| \lesssim |\xi|^{-s/2}$. One has $\dim_{\mathbb{H}}(X) \geq \dim_{\mathbb{F}}(X)$, but for many explicit sets $X$ this inequality is strict.
%\end{itemize}

%The Fourier dimension thus, morally speaking, measures how uncorrelated the set $X$ is with the Fourier characters $e_\xi(x) = e^{2 \pi i \xi \cdot x}$.

Let us consider a model problem for pattern avoidance; given a fixed function $f: (\RR^d)^n \to \RR^m$, how large must the dimension of a set $X$ be to guarantee that there exists $x_1,\dots,x_n \in X$ such that $f(x_1,\dots,x_n) = 0$. We focus on finding lower bounds for this problem, constructing sets $X$ with large Hausdorff or Fourier dimension such that $X$ \emph{avoids the zeroes} of $f$, in the sense that for any distinct points $x_1,\dots,x_n \in X$, $f(x_1,\dots,x_n) \neq 0$. This model has been considered in various contexts:
%
\begin{itemize}
	\item[(A)] If $m = 1$, and $f$ is a polynomial of degree $n$ with rational coefficients, M\'{a}the \cite{Mathe} constructs a set with Hausdorff dimension $d/n$ avoiding the zeroes of $f$.

	\item[(B)] If $f$ is a $C^1$ submersion, Fraser and Pramanik \cite{FraserPramanik} constructs a set with Hausdorff dimension $m/(n-1)$ avoiding the zeroes of $f$. 

	\item[(C)] If the zero set $f^{-1}(0)$ has Minkowski dimension at most $s$, I, together with my Master's thesis advisors Malabika Pramanik and Joshua Zahl \cite{DensonPramanikZahl} constructed sets of Hausdorff dimension $(dn - s)/(n-1)$ avoiding the zeroes of $f$.

	\item[(D)] If $f$ can be factored as $f = g \circ T$, where $T: (\RR^d)^n \to \RR^l$ is a full-rank, rational coefficient linear transformation, and $g: \RR^l \to \RR^m$ is a $C^1$ submersion, then I \cite{DensonCharacterization} have constructed a set with Hausdorff dimension $m/l$ avoiding the zero sets of $f$.
\end{itemize}
%
Notice that the above four methods only construct sets with large \emph{Hausdorff dimension} avoiding patterns. They say nothing about constructing sets with large Fourier dimension, which in general is a much harder problem involving a delicate interplay between `randomness' and `structure'. Most `structured' sets have low Fourier dimension,% For instance, the standard middle thirds Cantor set has Fourier dimension zero, because the intervals at stage $n$ of the usual construction of the Cantor set lie on arithmetic progressions of length $1/3^n$, and are thus highly correlated with planar waves of frequency $3^n$.
and so most methods for constructing sets with large Fourier dimension require making certain `random choices' which on average do not correlate with any particular planar wave. Structure must be added to some degree to avoid containing a given configuration, but adding too much structure will likely add a high degree of correlation of your sets with certain planar waves, resulting in your set having Fourier dimension zero. Certain results have been obtained, however, for \emph{linear} functions $f$:
%
\begin{itemize}
	\item[(E)] If $f(x_1,\dots,x_n) = a_1x_1 + \dots + a_nx_n$ with $\sum a_j = 0$, Pramanik and Liang \cite{PramanikLiang} construct a set $X \subset [0,1]$ with Fourier dimension $\dim_{\mathbb{F}}(X) = 1$ avoiding the zeroes of $f$. This generalizes a construction of Shmerkin \cite{Shmerkin}, who proved the result in the special case where $f(x_1,x_2,x_3) = (x_3 - x_1) - 2 (x_2 - x_1)$ detects arithmetic progressions of length 3.

	\item[(F)] K\"{o}rner constructed subsets $X \subset [0,1]$ with Fourier dimension $(k-1)^{-1}$ such that for any integers $m_0,\dots,m_k$, and any distinct $x_1,\dots,x_k \in X$, $a_0 \neq a_1x_1 + \dots + a_nx_n$.
\end{itemize}
%
The focus on linear functions is natural, since the Fourier transform behaves in a predictable way with respect to linearity. On the other hand, the understanding of the Fourier transform with respect to other nonlinear phenomena is poorly understood. \emph{The main goal of my research project was to find constructions of sets with large Fourier dimension avoiding the zeroes of a nonlinear functions $f$}.

\subsection{My Contributions To The Study Of Configurations} \label{MyContributionFractals}

It seems very difficult, if not impossible to adapt methods (A) and (D) above to construct sets with positive Fourier dimension, since the constructions involve constructing $X$ at each spatial scale by choosing a good family of intervals, and then considering a large union of translates of the intervals along an arithmetic progression. This ensures a spread out family of intervals, and thus a set with large Hausdorff dimension. But it is not good for ensuring Fourier decay, since a function concentrated near an arithmetic progression must have a large Fourier coefficient at frequencies complementing the spacing of this progression. On the other hand, methods (B) and (C) involve mostly pigeonholing arguments, so they seem the most likely to be able to be adapted to the Fourier dimension setting. I was able to adapt some of the ideas of these methods to obtain such a result.

For simplicity, I focused on the case when $m = d$ and when the function $f$ was $C^1$ and full rank, as assumed in \cite{FraserPramanik}. Then by the implicit function theorem, after possibly rearranging indices, we can locally write $f(x_1,\dots,x_n) = x_1 - g(x_2,\dots,x_n)$ for a function $g: (\RR^d)^{n-1} \to \RR^d$. In \cite{DensonFourier}, under the assumption that $g$ was a submersion in each variable $x_2,\dots,x_n$, I was able to modify the construction of \cite{FraserPramanik} to construct sets with Fourier dimension $d/(n-3/4)$ avoiding the zeroes of $f$. Under the further assumption that we can write $g(x_2,\dots,x_n) = ax_2 + h(x_3,\dots,x_n)$ for $a \in \QQ$, I was able to construct sets with Fourier dimension $d/(n-1)$ avoiding the zeroes of $f$, recovering the Hausdorff dimension bound of \cite{FraserPramanik} in the Fourier dimension setting.

\begin{theorem}
	Suppose that $g: ([0,1]^d)^{n-1} \to \RR^d$ is a function such that for each $k \in \{ 1, \dots, n-1 \}$, the $d \times d$ matrix $D_k g = ( \partial g_i / \partial x_{k,j} )_{i,j = 1}^d$ is invertible. Then there exists a Salem set $X \subset [0,1]^d$ of dimension $d/(n-3/4)$ such that for all distinct $x_1,\dots,x_n \in X$, $x_1 \neq f(x_2,\dots,x_n)$. If, in addition, $g(x_2,\dots,x_n) = ax_2 + h(x_3,\dots,x_n)$ for some $a \in \QQ$, then there exists a Salem set $X \subset [0,1]^d$ of dimension $d/(n-1)$ such that for all distinct $x_1,\dots,x_n \in X$, $x_1 \neq f(x_2,\dots,x_n)$.
\end{theorem}

As with most of the other approaches discussed above, we construct a set $X$ avoiding zeroes via a `Cantor-type construction'. Fix a parameter $\alpha$. We iteratively define a nested family of sets $\{ X_k \}$, each a union of cubes of some fixed length $l_k$, and define $X = \bigcap_k X_k$. The set $X_{k+1}$ is obtained from $X_k$ by partitioning $X_k$ each sidelength $l_k$ cube into $N^d$ sidelength $l_{k+1}$ cubes, where $N \coloneqq l_k / l_{k+1}$, and letting $X_{k+1}$ be formed from the union of a subcollection of these cubes. The construction in \cite{DensonPramanikZahl} and \cite{DensonFourier} is very simple: To construct $X_{k+1}$ from $X_k$, we start by taking a set $S$ by taking $\sim N^\alpha$ points uniformly at random from the centers of the sidelength $l_{k+1}$ cubes in the partition of each sidelength $l_k$ cube in $X_k$. Some points from this set will form near zeroes of the function $f$; we let
%
\[ S_{\text{bad}} = \{ x \in S : |f(x,x_2,\dots,x_n)| \leq 10 l_{k+1}\ \text{for some $x_2,\dots,x_n \in S$} \}, \]
%
and define $X_{k+1}$ to be the union of all sidelength $l_{k+1}$ cubes centered at points in $S - S_{\text{bad}}$. The set $X$ will then avoid the zeroes of the function $f$. Provided that $\alpha \leq (nd - s)/(n-1)$, we have with high probability that $\#(S_{\text{bad}}) \ll \#(S)$, and so with high probability, at each stage of the construction $X_k$ is a union of $\sim l_k^{-\alpha}$ cubes of sidelength $\alpha$; it is therefore natural to expect the set $X$ almost surely has Hausdorff dimension $\alpha$, and indeed, in \cite{DensonPramanikZahl} this is shown to be the case.

Simply counting the number of cubes at each scale is not sufficient to obtain a Fourier dimension bound. In \cite{DensonFourier}, I made the observation that the core feature of constructions that yield Fourier dimension bounds is that they must involve a \emph{square root cancellation bound}. More precisely, if we denote the centers of the sidelength $l_k$ cubes forming $X_k$ by $\{ x_1,\dots,x_M \}$, then for all $1 \lesssim |\xi| \lesssim N$ then the resulting set $X$ will have Fourier dimension agreeing with it's Hausdorff dimension if the square root cancellation bound
%
\begin{equation} \label{squareRootCancellation}
	\left| \frac{1}{M} \sum_{j = 1}^M e^{2 \pi i \xi \cdot x_j} \right| \lessapprox M^{-1/2}
\end{equation}
%
holds at all scales. Indeed, consider the probability measure $\mu_k = M^{-1} \sum_{j = 1}^M \chi_j$ supported on $X_k$, wjere $\chi_j$ is a smooth bump function adapted to the cube centered at $x_j$. Then for $|\xi| \lesssim 1/l_k$, since $M \sim l_k^{-\alpha}$ with high probability, \eqref{squareRootCancellation} implies that $|\widehat{\mu}_k| \lessapprox M^{-1/2} \lesssim |\xi|^{-\alpha/2}$. On the other hand, the uncertainty principle implies that $\widehat{\mu}_k$ decays rapidly for $|\xi| \gtrsim 1/l_k$, and so $\widehat{\mu}_k$ has the appropriate Fourier decay required. Taking weak limits of the measures $\{ \mu_k \}$, we find that $|\widehat{\mu}(\xi)| \lesssim |\xi|^{-\alpha/2}$ has the right Fourier decay to justify that $X$ has Fourier dimension $\alpha$.

The necessity for square root cancellation bounds explains why random techniques often play a core role in the construction of sets with large Fourier dimension, since the phenomena of square root cancellation occurs in a plethora of random constructions, and probabilists have established many tools in the theory of \emph{concentration of measure} to determine when a sum of random variables has square root cancellation away from the mean \emph{with high probability}. If we are taking a sum of independent random variables, often Hoeffding's inequality gives sharp bounds ensuring square root cancellation. But in this case the random points $\{ x_j, \dots, x_M \}$ are \emph{not} chosen independently from one another. The initial set of points chosen to form the set $S$ in the construction above are taken uniformly at random, but the points in the set $S - S_{\text{bad}}$ are no longer independent from one another. There are certain standard tools to handle this problem, such as McDiarmid or Azuma's inequality, though in this setting they fail to ensure square root cancellation unless $\alpha$, which is not large enough for our purposes. In \cite{DensonFourier}, I found a novel way to interlace Hoeffding and McDiarmid's inequality together to ensure square root cancellation away from the mean occurs with high probability for $\alpha \leq 1/(n-1)$.

After ensuring square root cancellation of the mean, the final problem is to show that the mean of $M^{-1} \sum e^{2 \pi i \xi \cdot x_j}$ has square root cancellation, which proved to be the most inefficient aspect of the argument. This mean can be written as an oscillatory integral, though in $M$ variables, and so usual techniques in the theory of oscillatory integrals fail to handle this bound since they are usually \emph{dimension dependent}, and we need bounds uniform in $M$. Instead, I was able to use an inclusion exclusion argument, together with a Whitney decomposition of the thickened zero set of the function $f$ to obtain the required bounds. This is the least optimal part of the argument, yielding a Fourier dimension of $d/(n-3/4)$ rather than $d/(n-1)$; however, if $f$ satisfies a weak linearity a slight modification of the random construction ensures that the mean of $M^{-1} \sum e^{2 \pi i \xi \cdot x_j}$ is always zero, yielding the large Fourier dimension bound $d/(n-1)$ in this case. I am interested in determining whether techniques in the theory can yield the dimension $d/(n-1)$ bound in general, though I do not think this is a good research project to pursue immediately given the availability of techniques available at the time.

\section{Future Lines of Research} \label{Section3}

In the near future, I hope to generalize the bounds obtained in \cite{DensonCharacterization} to the more general setting of multipliers associated with the Laplace-Beltrami operator on Riemannian manifolds with periodic geodesic flow. A single scale obstruction to this generalization is obtaining control over iterates of a pseudodifferential operator on $M$ called the `return operator'.
% One obstruction to this generalization at a `single frequency scale' is obtaining control of iterates of a pseudodifferential operator on $M$ called the `return operator'. Another obstruction when 'combining frequency scales' is an endpoint refinement of the local smoothing inequality for the wave equation on $M$. %to handle combining dyadic scales together.
I am also interested of obtaining bounds on manifolds whose geodesic flow has well-controlled dynamical properties, such as forming an integrable system. Related to my work in geometric measure theory, I hope to apply the probabilistic methods I exploited in the construction of sets of large Fourier dimension to construct random fractals which exhibit good $l^2L^p$ decoupling properties. And I am interested in determining the interrelation of patterns with the study of multipliers on manifolds, in particular studying Falconer distance problems on Riemannian manifolds to local smoothing phenomena for the wave equation on manifolds. A more detailed justification for the potential of these projects can be found in Section 3 of this statement.

The remaining sections of this summary provides context and describes the results I have obtained during my PhD in further detail, finishing with a further elaboration of future work and it's feasibility given the tools I have gained from my previous work.

Section 3 will discuss my plans for future research from these two projects.

Given the context from the previous two sections, we finish this summary by describing in more detail several problems I believe may be accessible given the techniques I have used to solve previous problems.

\subsection{Multipliers Associated With Periodic Geodesic Flow}

In Section \ref{Section2}, I discussed that the results I were able to obtain for multiplier operators on $S^d$ generalized to multipliers of an arbitrary first order, elliptic, self-adjoint pseudodifferential operator $P$ on a compact manifold $M$, provided that $P$ satisfied two assumptions. Assumption A relates to the curvature of the principal symbol, and this assumption cannot really be weakened without significantly changing the character of the results, which heavily depend on this curvature. On the other hand, Assumption B arises as an artifact of the methods of our proof. We can likely obtain similar bounds while weakening this assumptions; for instance, Kim \cite{KimManifold} obtained bounds on the scale of Besov spaces only under Assumption A.

It is likely very difficult that we can completely removing Assumption B using current research methods while still recovering the results of \cite{DensonCharacterization}, a limitation of our current inability to understand the large time behavior of wave equations on compact manifolds. If we were able to follow the method of \cite{DensonThesis}, which reduced the large time argument to a smoothing inequality for the wave equation, then the results of that paper would follow for another operator $P$ if we could prove
%
\begin{equation} \label{smoothingbound}
	\left\| \left( \int_k^{k+1} |e^{2 \pi i t P} f|^{p'}\; dt \right)^{1/p'} \right\|_{L^{p'}(M)} \lesssim k^\delta \| f \|_{L^p_{d(1/p - 1/2) - 1/p'}(M)}
\end{equation}
%
for some $\delta < (d-1)(1/p - 1/2) - 1/p'$. If $P$ satisfies assumption $B$, then after rescaling, we may assume without loss of generality that all eigenvalues of $P$ are integers, so that $e^{2 \pi i k P} = I$ is the identity for all $k$, and then \eqref{smoothingbound} holds for all $|1/p - 1/2| > (d-1)^{-1}$ and with $\delta = 0$ by the local smoothing inequality of Lee and Seeger \cite{LeeSeeger}. 

Whether this bound is true in other contexts remains unknown. The next simplest case to consider would be when the operator $P$ has the property that $e^{2 \pi i k P}$ is \emph{close} to the identity for all $k$. This happens precisely when the \emph{Hamiltonian flow} on $T^* M$ given by the vector field $H = ( \partial p_{\text{prin}} / \partial \xi , - \partial p_{\text{prin}} / \partial x)$ is periodic, where $p_{\text{prin}}$ is the principal symbol of $P$. Indeed, results of Colin de Verdi\'{e}re \cite{ColinDeVerdiere} related to the theory of propagation of singularities of Fourier integral operators then tell us that the operator $R = e^{2 \pi i P}$ is a pseudodifferential operator of order zero, and it's principal symbol is related to an invariant of the flow known as the Maslov index. The operator has been studied a little by spectral theorists, and there it is known as the \emph{return operator}. If we are able to justify bounds of the form
%
\[ \| R^k f \|_{L^p_{d(1/p - 1/2) - 1/p'}} \lesssim k^\delta \| f \|_{L^p_{d(1/p - 1/2) - 1/p'}}, \]
%
or a frequency localized variation of this bound, then the local smoothing inequality of Lee and Seeger yields \eqref{smoothingbound}. Such bounds are of interest since they cover all the operators $P = \sqrt{-\Delta}$, where $\Delta$ is the Laplace-Beltrami operator on a Riemannian manifold with periodic geodesic flow. They are even of interest on the sphere, since our method only allows us to tell when multipliers of the form $a( P/R )$ are uniformly bounded on $L^p(S^d)$, where $P = \sqrt{ ( {\scriptstyle \frac{d-1}{2}} )^2 - \Delta }$ whereas these bounds would allow us to tell when the multipliers $a( \sqrt{-\Delta} / R )$ are uniformly bounded on $L^p(S^d)$.

\subsection{Genuine Decoupling On Random Fractals}

One major development in harmonic analysis in the past decade has been a greater understanding of the phenomenon of \emph{decoupling}, or \emph{Wolff-type estimates}. Given a family of almost disjoint subsets $\mathcal{E}_\delta$ of $\RR^d$ parameterized by $\delta > 0$, $L^p(l^2)$ decoupling discusses bounds of the form
%
\[ \Big\| \sum f_j \Big\|_{L^p(\RR^d)} \leq D_p(\delta) \Big( \sum\nolimits_j \| f_j \|_{L^p(\RR^d)}^2 \Big)^{1/2}, \]
%
where the Fourier transforms of the functions $f_j$ are supported on distinct subsets of $\mathcal{E}_\delta$, and $D_p(\delta)$ denotes the best constant under which this equation holds for all such $\{ f_j \}$. A \emph{genuine decoupling inequality} results when one can prove that $D_p(\delta) \lesssim_\varepsilon \delta^{-\varepsilon}$ for all $\varepsilon > 0$.%  or the stronger logarithmic bound $D_p(\delta) \lesssim \log(\delta)$ and other variants of this form.

Recently, much work has been carried out for $d \geq 2$, and when the sets $\mathcal{E}_\delta$ are $\delta$ caps associated with partitions of $\delta$-neighborhoods of curves and surfaces, and the decoupling inequalities are obtained by virtue of the curvature and torsion properties of the shapes they are associated with. But the analysis of decoupling on \emph{fractal sets} is still poorly understood. %, especially when $d = 1$ and no `curvature' in the usual sense is present.
Consider a sequence of integers $n(i)$, and a set $X$ obtained from a Cantor-like construction $\{ X_i \}$ as in Section \ref{MyContributionFractals}, where $X_i$ is a union of a family of cubes $\mathcal{Q}_i$ with some fixed sidelength $\delta \coloneqq \delta_i$. We let $\mathcal{E}_\delta = \{ Q \cap C_{i + n(i)} : Q \in \mathcal{Q}_i \}$, and ask for which Cantor type constructions $\{ X_i \}$ and for which sequences $\{ n(i) \}$ do we obtain a genuine decoupling inequality for the families $\{ \mathcal{E}_\delta \}$.

Some analysis has been done in this setting, but no genuine decoupling bounds have been established for any fractal set. Some decoupling bounds have been obtained for self-similar Cantor sets with good numerical properties \cite{ChangJaumeGreenfeldJamneshanLiMadrid}, but none of the bounds obtained give genuine decoupling inequalities in the above sense. Decoupling inequalities for random fractal sets have been obtained by {\L}aba and Wang \cite{LabaWang}; these are also not genuine decoupling inequalities, but the bounds they obtained were sufficient for their applications to the study of $L^p \to L^2$ fractal restriction bounds. In fact, in the range of $p$ they were considering, genuine decoupling is not possible; one can see by taking counter examples using the local constancy property and Khintchine type heuristics that if $X$ is chosen sufficiently randomly, and $\# \mathcal{Q}_i \gtrsim \delta_i^{-s}$ for each $i$, then genuine $L^p(l^2)$ decoupling is impossible for any choice of $\{ n(i) \}$ unless $2 \leq p \leq 2d/s$.

I believe the techniques related to my results in \cite{DensonFourier} can be applied to obtaining random decoupling inequalities. Methods from the theory of concentration of measure have been applied by Bourgain \cite{Bourgain} and Talagrand \cite{Talagrand} in order to prove the existence of $\Lambda(p)$ sets, in particular, the method of majorizing measures and selection processes. One might view $\Lambda(p)$ sets as a kind of discrete variant of sets upon which decoupling bounds hold, so it is likely to believe these methods generalize to the continous setting. Using these methods, I hope to obtain an analogue of the proof of $l^2(L^p)$ decoupling for the paraboloid found in \cite{BourgainDemeterStudyGuide}, i.e. establishing an analogue of multilinear Kakeya for the sets $\mathcal{E}_\delta$, and then apply an induction on scales to obtain a genuine fractal decoupling inequality.

%\section{Radial Multipliers}

%It is a classic heuristic that the \emph{smoothness} of the multiplier defining an operator determines it's $L^p$ boundedness. The bound $\sup_j C_p(a_j) < \infty$ can be viewed in some sense as such a condition, but is not equivalent to the usual norms that measure smoothness, the Sobolev, Besov, or Triebel-Lizorkin norm. However, it is true that $\| a \|_{A_p(\RR)} \lesssim C_p(a) \lesssim \| a \|_{B_p(\RR)}$, where
%
%\[ A_p(\RR) \coloneqq B_{p,\infty}^{(d-1)(1/p - 1/2)}(\RR)\quad \text{and}\quad B_p(\RR) \coloneqq B_{2,p}^{d(1/p - 1/2)}(\RR). \]
%
%Thus $C_p(a) < \infty$ is slightly stronger condition than control over $(d-1)(1/p - 1/2)$ derivatives of $a$ in $L^p(\RR^d)$, but weaker than control over $d(1/p - 1/2)$ derivatives in $L^2(\RR^d)$. One could conjecture that for $1 < p < 2d/(d+1)$, the operator $T_a$ is bounded on $L^p(\RR^d)$ if $\sup_j \| a_j \|_{B_p(\RR)} < \infty$. This conjecture only gives necessary, not sufficient, conditions for boundedness, but might be more accessible to current techniques. Indeed, this weaker conjecture has been verified by Lee, Rogers, and Seeger \cite{LeeRogersSeeger} to be true for all $d \geq 2$ in the larger Stein-Tomas range $1 < p < 2 {\scriptstyle \left( \frac{d+1}{d+3} \right)}$.

\begin{comment}

\subsection{Radial Multiplier Bounds And `Fractal Weighted Estimates'}

The main proof of the radial Fourier multiplier estimates in \cites{HeoNazarovSeeger,Cladek,KimQuasiradial} discussed in Section \ref{ContributionMultipliers} involves a `density decomposition arguments'. Roughly speaking, this involves proving bounds on quantities of the form
%
\[ \big\| \sum\nolimits_{(\tau,\theta) \in \mathcal{E}} k_\tau * f_\theta \big\|_{L^2(\RR^d)} \lesssim u^a \Big( \sum\nolimits_{(\tau,\theta) \in \mathcal{E}} \| k_\tau * f_\theta \|_{L^2(\RR^d)}^2 \Big)^{1/2}, \]
%
where $\mathcal{E}$ is a `sparse' discrete set such that $\#\{ \mathcal{E} \cap B_r \} \leq u r$ for all balls $B_r \subset \RR^{1+d}$ of sufficiently small radius $r$. One can then apply interpolation and a `density decomposition' argument to get suitable $L^p$ estimates for $|1/p - 1/2| > a$.

There are several techniques in the literature 

The discrete problem above is related to a `fractal weighted estimate' for the extension operator $E$ associated with the cone in $\RR^{1+d}$. Say a weight $w: \RR^{1+d} \to [0,\infty)$ is $\alpha$ dimensional if $w(B_r) \leq u r^\alpha$ for some $u > 0$ and all balls $B_r$. Then the discrete problem above is equivalent to showing that if $w$ is an $1$-dimensional weight, then 
%
\[ \left\| \int w(x,t) Ef(x,t)\; dt \right\|_{L^p(\RR^{d+1})} \]

Recently several results I have seen various results in the literature on various problems apply to a `dual' version of the bounds above. Namely, given the extension operator $E$ associated with a hypersurface $\Sigma$ of $\RR^{1+d}$, $L^2(\Sigma) \to L^2(\RR^{d+1}, w)$ bounds are obtained for the extension operator under the assumption that the weight function $w: \RR^{d+1} \to [0,\infty)$ is `$\alpha$-dimensional' for some $\alpha \in (0,d+1]$, which means $w(B_r) \lesssim r^\alpha$ for all balls $B_r$ and some $\alpha \in (0,d+1]$. We can view this problem as bounding the $L^2 \to L^2$ operator norm of the operator $Tf = w^{1/2} Ef$, whose dual is the operator $T^*g = R( w^{1/2} g)$, where $R$ is the restriction operator to $\Sigma$.

 \cite{DuZhang}.


 then the Schr\"{o}dinger propogators $\{ e^{it \Delta} \}$ satisfy $\| e^{it \Delta} f \|_{L^2(B_R,w)} \lessapprox R^{\frac{\alpha}{2(d+1)}} \| f \|_{L^2}$

 In the theory of restriction, Du, Li, Wang, and


In particular, BLAH has done work on fractal weighted estimates for the extension problem on the parabola, and BLAH has done work on fractal wiehgted estimates for the wave equation 

\end{comment}







\begin{comment}

\subsection{Relations Between Fourier Multipliers and Zonal Multipliers}

One connection which explains why analogues to the bounds for radial Fourier multipliers might be found in the study of zonal multipliers is that both classes of operators are related to the Laplace operator on their respective spaces. Namely, if $f$ is a distribution on $\RR^d$, and $\Delta f = - \lambda^2 f$, then $f$ has Fourier support on the sphere of radius $\lambda$ centered at the origin, and so $T_a f = a(\lambda) f$. Similarly, if $\Delta$ is the Laplace-Beltrami operator on $S^d$, and $\Delta f = - \lambda(\lambda+d-1)$, then $f$ is a spherical harmonic of degree $\lambda$, and so $Z_a f = a(\lambda) f$.
%If $f$ is a distribution on $\RR^d$ and the support of it's Fourier transform lies on the sphere of radius $k$ about the origin, then $\Delta f = - k^2 f$ and $T_a f = a(k) f$.
%Thus $T_a$ and $\Delta$ have the same eigenfunctions.
%If the Fourier transform of a tempered distribution $f$ is supported on the sphere $\{ \xi : |\xi| = R \}$, then it follows from the Fourier inversion formula that $\Delta f = - 4 \pi^2 R^2 f$.
%This essentially follows because then $f$ is a superposition of the planar waves $e^{2 \pi i \xi \cdot x}$ with $|\xi| = R$, and $\Delta e^{2 \pi i \xi \cdot x} = - 4\pi^2 R^2 f$.
%But for the same such $f$, if $T_a$ is a radial Fourier multiplier, then $T_a f = a(R) f$. Thus a radial Fourier multiplier has the same eigenfunctions of the Laplacian.
%Similarly, if $\Delta$ is the Laplace-Beltrami operator on $S^d$, and $f$ is a spherical harmonic of degree $k$, then $\Delta f = - k(k+d-1) f$, and $Z_a f = a(k) f$.
Using the notation of functional calculus, we can thus write $T_a = a \big( \sqrt{-\Delta} \big)$ and $Z_a = a \big( \sqrt{ \alpha^2 - \Delta} \big)$, where $\alpha = (d-1)/2$, and now the resemblance is clear. In the rest of this section, we let $P = \sqrt{\alpha^2 - \Delta}$, and let $a(P/R)$ denote the zonal multiplier with symbol $a(\cdot/R)$.

On the other hand, unlike the planar waves $e^{2 \pi i \xi \cdot x}$, it is difficult to understand what a general spherical harmonic might look. Dilation symmetries on $\RR^d$ tell us high frequency planar waves are just the dilates of low frequency planar waves; on the other hand, $S^d$ has no dilation symmetries, and high degree spherical harmonics need not look anything like low degree spherical harmonics. This could be alarming, because the transference principle we hope to prove implies a result related to dilation on $S^d$; namely, if the Fourier multiplier $T_a$ is bounded on $L^p(\RR^d)$, then the Fourier multipliers $T_{a,R}$ with symbols $a(\cdot/R)$ are all uniformly bounded on $L^p(\RR^d)$, and thus the transference principle we hope to prove shows that the zonal multipliers $a(P/R)$ are uniformly bounded on $L^p(S^d)$. Because of the lack of dilation symmetry on $S^d$, the behavior of the operators $a(P/R)$ might change as $R$ varies, which is discouraging.
 %Moreover, the eigenvalues of $\Delta$ on $S^d$ are discrete, whereas the eigenvalues of $\Delta$ on $\RR^d$ range over $(-\infty,0]$. This is discouraging, since the condition $\sup_j C_p(a_j)$ exploited controls the \emph{smoothness} of the function $a$, a property that should be irrelevant to zonal multipliers, given that only discrete values of the symbol are required in the definition of the operator. We should therefore not expect to find a complete characterization for zonal multipliers. % using the methods which bound radial Fourier multipliers.

%However, various heuristics tell us the operators $- \Delta / R^2$ on $\RR^d$ and $(\alpha^2 - \Delta) / R^2$ on $S^d$ behave asymptotically like one another as $R \to \infty$. Thus we might hope to find a completely characterization of the functions $a$ such that the operators $\{ Z_{a,R} \}$ are uniformly bounded on $L^p(S^d)$, where $Z_{a,R} = a( P/R )$ is the zonal multiplier operator with symbol $a_R(\rho) = a( \rho / R)$. In fact, a classic result of Mitjagin \cite{Mitjagin} proves that if the operators $\{ Z_{a,R} \}$ are uniformly bounded on $L^p(S^d)$, then $T_a$ is bounded on $L^p(\RR^d)$. Thus, if we could show that $\sup_j C_p(a_j) < \infty$ implies that the operators $\{ Z_{a,R} \}$ are uniformly bounded, then we would have obtained a \emph{transference principle} for such operators, that an operator $T_a$ is bounded on $L^p(\RR^d)$ if and only if the operators $\{ Z_{a,R} \}$ are uniformly bounded on $L^p(S^d)$. %and in particular that the operators $\{ Z_{a,R} \}$ are uniformly bounded if and only if $\sup_j C_p(a_j)$ is finite.

Fortunately, whatever differences the operators $a(P/R)$ have as $R$ varies are not as relevant to the study of $L^p$ boundedness as one might at first think. This is because zonal multipliers only fail to be bounded on $L^p(S^d)$ because of `high frequency behavior'; a zonal multiplier whose symbol is compactly supported is bounded on all of the $L^p$ spaces. And there are various heuristics that tell us that the Laplacian on $S^d$ begins to behave more and more similar to the Laplacian on $\RR^d$ when restricted to high frequency eigenfunctions. For instance, on $S^d$, the operator $P/R$ can be written as $\sqrt{ \alpha^2/R + \Delta_R }$, where $\Delta_R$ is the Laplacian associated with the metric $g_R = R^2 g$ on $S^d$. As $R \to \infty$, the metric $g_R$ gives $S^d$ less curvature and more volume, and so we might imagine that, as $R \to \infty$, the operator $P/R$ behaves more and more like $\sqrt{-\Delta}$ on $\RR^d$, and thus multipliers of $P/R$ behave more and more like radial Fourier multipliers, i.e. we might imagine that the equation $T_a = \lim_{R \to \infty} a(P/R)$ holds in a certain heuristic sense.

The last equation also leads us to believe that if the operators $a(P/R)$ are uniformly bounded on $L^p(S^d)$, then $T_a$ is bounded on $L^p(\RR^d)$. This is true for all $1 \leq p \leq \infty$, a classical transference result of Mitjagin \cite{Mitjagin}. On the other hand, the transference principle we prove is more unusual: the boundedness of a limit point need not necessitate uniform bounds on the limiting sequence. This might explain why the principle is more difficult to prove. Indeed, Mitjagin's argument
% Moreover, recent results on the failure of the Kakeya conjecture in three dimensional manifolds with non-constant sectional curvature \cite{DaiGongGuoZhang} provide weak evidence that the range of $L^p$ spaces to which inequality \eqref{upperboundmainresult} can be obtained may be more sensitive to the geometry of a manifold $M$ than the range of $L^p$ spaces under which \eqref{mitjagininequalityforlaplacian} holds. Obtaining bounds of the form \eqref{upperboundmainresult} therefore requires a more subtle analysis than required to obtain bounds of the form \eqref{mitjagininequalityforlaplacian}.
%In this setting, we were able to prove this transference principle on $S^d$ for all $d \geq 4$ and $|1/p - 1/2| > 1/(d-1)$,the same range of parameters under which the results of \cite{HeoNazarovSeeger} holds. A result like this is significant, since no analogous result has been proved on any other compact manifold.
Mitjagin's argument generalizes to show that for any compact manifold $M$, and any elliptic, self-adjoint pseudodifferential operator $P$, the uniform boundedness of the multiplier operators $a(P/R)$ on $L^p(M)$ implies the boundedness of the Fourier multiplier $T$ on $L^p(\RR^d)$ whose symbol is given by the principal symbol of $P$. Until our new transference principle, no analogous transference principle has been shown for any $P$ and any exponent $p \neq 2$, excluding trivial cases, nor any characterization of functions $a$ such that the operators $\{ a(P/R) \}$ are uniformly bounded on $L^p(M)$ for any $p \neq 2$.

\subsection{Bounding Band Limited Zonal Multipliers}

To discuss the techniques I developed which lead to the aforementioned characterization, let us begin by summarizing the rough methodology by which the arguments in \cite{HeoNazarovSeeger,Cladek,KimQuasiradial} are able to obtain these bounds. We  begin by describing the \emph{band limited} part of the argument:

\pagebreak[3]

  \medskip
  \hangindent0.5em
  \hangafter=0
    \noindent{\emph{Take a decomposition $T_a = \sum_{j \in \ZZ} T_j$, where $T_j$ is the multiplier operator with symbol $a_j( \cdot / 2^j)$. Our goal is the band limited bound $\| T_j f \|_{L^p(\RR^d)} \lesssim \| f \|_{L^p(\RR^d)}$, uniformly in $j$ % The advantage of this decomposition is that we may assume the inputs and outputs of the operator $T_j$ are band limited and thus we can apply various and heuristics related to the uncertainty principle.
    We then write $T_j f = k * f$, where $k$ is the Fourier transform of $a_j(\cdot / 2^j)$. We then write $k = \sum k_\tau$ and $f = \sum f_\theta$, where the functions $\{ k_\tau \}$ are supported on disjoint thickness $2^{-j}$ annuli, and the functions $\{ f_\theta \}$ are supported on disjoint side length $2^{-k}$ cubes, then $T_j f = \sum_{\tau,\theta} k_\tau * f_\theta$. Using the fact that the Fourier transform is unitary, that the function $f$ is band limited, and some Bessel function estimates, one can argue that the inner product $\langle k_\tau * f_\theta, k_{\tau'} * f_{\theta'} \rangle$ is negligible unless the annulus of radius $\tau$ centered at $\theta$ is near tangent to the annulus of radius $\tau'$ centered at $\theta'$.
    }}

  \medskip
  \hangindent0.5em
  \hangafter=0
  \noindent{\emph{Using the inner product estimate, together with an argument for counting incidences, one can show that the $L^2$ norm of a sum $\sum_{(\tau,\theta) \in \mathcal{E}} k_\tau * f_\theta$ is well behaved if $\mathcal{E}$ is suitably 'sparse', and interpolation with a trivial $L^1$ estimate yields an $L^p$ estimate on the sum. Conversely, if the set $\mathcal{E}$ is clustered, then $\sum_{(\tau,\theta) \in \mathcal{E}} k_\tau * f_\theta$ will be concentrated on only a few annuli, and so we can also get good $L^p$ estimates. But then we can estimate $\| Tf \|_{L^p(\RR^d)} = \| \sum k_\tau * f_\theta \|_{L^p(\RR^d)}$ by either approach, depending on whether a sparse part of the sum dominates, or whether a clustered part of the sum dominates.
}}

\vspace{0.3em}

\noindent My paper \cite{DensonCharacterization} proves an analogue of this argument for zonal multiplier operators on $S^d$ for $d \geq 4$ and $|1/p - 1/2| > 1/(d-1)$, in particular, obtaining the aforementioned transference principle in this range.

Giving that the argument above exploits convolution on $\RR^d$, one might expect that we might use `zonal convolution' in our argument, i.e. an analogue of convolution on $S^d$. It is likely one can use this approach to obtain $L^p$ bounds under assumptions on the integrability of the zonal convolution kernel. However, the lack of a dilation symmetry on $S^d$ means that the zonal convolution kernel for $a(P/R)$ is likely unrelated to the convolution kernel for $a(P)$ as $R$ varies, so based on our previous discussion it is likely difficult to obtain a transference principle using this technique. Instead, we follow an approach due to H\"{o}rmander BLAH, successfully used in several other problems on manifolds BLAH, and use the Fourier inversion formula to write
%
\[ Z_j f = \int_{-\infty}^\infty 2^j \widehat{a}_j( 2^j t ) e^{2 \pi i t P} f\; dt, \]
%
where $P = \sqrt{ \alpha^2 - \Delta }$ is as above, and $e^{2 \pi i t P}$ are the wave propagator operators which, as $t$ varies, give solutions to the wave equation $\partial_t^2 = \Delta - \alpha^2$ with zero velocity initial conditions, or equivalently, solutions to $\partial_t = P$, the `half-wave equation'. Given a general input $f$, we perform a decomposition analogous to the method above, writing $f = \sum f_\theta$ and $T_j = \sum T_\tau$, where the functions $\{ f_\theta \}$ are supported on disjoint sets of diameter $2^{-k}$, and $T_\tau = \int b_\tau(t) e^{2 \pi i t P}\; dt$, where $2^j \widehat{a}_j( 2^j t ) = \sum_\tau b_\tau(t)$ for a family of functions $\{ b_\tau \}$ is supported on disjoint side length $2^{-j}$ intervals. We can thus write $T_j f = \sum T_\tau f_\theta$.

The behavior of the wave equation is closely tied to the behavior of geodesics on $S^d$. In particular, for high frequency inputs we have a near explicit representation of the wave propagator operators for $|t| < 1/2$, in a coordinate system, by an oscillatory integral
%
\[ (e^{2 \pi i t P} f)(x) \approx \int_{\RR^d} a(t,x,y,\xi) e^{2 \pi i [ \phi(x,y,\xi) + t |\xi|_y ]} f(y)\; d\xi\; dy,  \]
%
where $a$ is a symbol of order $0$, $|\xi|_y = (\sum g^{jk}(y) \xi_j \xi_k)^{1/2}$ is obtained from the Riemannian metric of $S^d$, and $\phi$ solves the eikonal equation $| (\nabla_x \phi)(x,y,\xi) |_x = |\xi|_y$ subject to the constraint that $\phi(x,y,\xi) = 0$ for $(x - y) \cdot \xi = 0$.

Oscillatory integral representations of the operators $\{ e^{2 \pi i t P} \}$ \emph{can} be obtained for $e^{2 \pi i t P}$ simply by the fact they form a semigroup, and so for $n-1 \leq t < n$ we can write $e^{2 \pi i t P} = ( e^{2 \pi i (t / n) P} )^n$ as the composition of $n$ oscillatory integrals. The theory of phase reduction for Fourier integral operators can help us reduce this composition, but it is difficult to control this quantity quantitatively.

simply by the fact that they are obtained by repeated compositions of the propogators with $|t| < 1/2$, and qualitative understanding of their behavior can be understand from the general theory of \emph{Fourier integral operators}, but obtaining good control on $e^{2 \pi i t P}$ for large $t$ becomes difficult.

Using this oscillatory integral representation and the stationary phase formula, we can obtain a substitute for the Bessel estimates used in the original argument, justifying that $\langle T_\tau f_\theta, T_{\tau'} f_{\theta'} \rangle$ is negligible unless the geodesic annulus of radius $\tau$ and center $\theta$ is near tangent to the geodesic annulus of radius $\tau'$ and center $\theta'$, provided that $\tau$ is bounded away from $1$. Here we use the stationary phase formula to obtain good bounds on the oscillatory integrals that emerge. However, one subtlety is showing that, restricted to values in $|\xi| = 1$, the critical points of the function $\phi(x,y,\cdot)$ are appropriately non-degenerate. Using the Hamilton-Jacobi approach to the study of the eikonal equation, one can identify the quantity $\phi(x,y,\xi)$ with the signed distance from the hyperplane $\{ x' : (x' - y) \cdot \xi = 0 \}$ to the point $x$ with respect to the Riemannian metric. I came up with a geometric argument involving the second variation formula for geodesics, which justifies that the function $\phi$ has only two stationary points, and each is non-degenerate, with the Hessian at each point having magnitude proportional to $d_g(x,y)^{d-1}$.

After this, the argument for radial Fourier multipliers generalizes quite directly to the case of $S^d$, since the incidence properties required for annuli on $\RR^d$ are roughly analogous to the properties of geodesic annuli on $S^d$.


% this method likely leads to a condition on the different dyadic functions $a_j$ which depends on a $j$ in a rather difficult to understand way, reflecting the lack of a dilation symmetry on the manifold. For similar reasons, 

%. A general zonal multiplier $Z_a$ can be decomposed as $\sum_j Z_j$ simply by decomposing it's symbol $a(\lambda) = \sum a_j( \lambda / 2^j )$, and letting $Z_j$ be the zonal multiplier with symbol $a_j( \cdot / 2^k )$. There is an analogue of convolution on the sphere, called zonal convolution, and so we could write $Z_j$ in terms of this convolution, and decompose as above into pieces supported on different annuli. However, the zonal convolution kernel of an operator does not behave well under dilation of the symbol of the operator (a manifestation of the fact $S^d$ does not have a family of dilation symmetries) so the condition we would need to finish our argument would likely involve rather non explicit assumptions about the various functions $\{ a_j \}$. Instead, we directly use the Fourier transform of $a_j$, which scales under dilations.

%We are able to use this by using the Fourier inversion formula and the functional calculus to write
%
%\[ Z_j = \int_{-\infty}^\infty 2^k \widehat{a}_j( 2^k t ) e^{2 \pi i t P}\; dt, \]
%
%where $P = \sqrt{ \alpha^2 - \Delta }$ is as above, and $e^{2 \pi i t P}$ are the wave propagator operators which, as $t$ varies, give solutions to the wave equation $\partial_t^2 = \Delta - \alpha^2$ with zero velocity initial conditions, or equivalently, solutions to $\partial_t = P$, the `half-wave equation'. The behavior of this wave equation is closely tied to the behavior of geodesics on $S^d$. In particular, we have a close to explicit representation of the wave propagator operators for $t$ having magnitude smaller than $2\pi$ times the \emph{injectivity radius} on $S^d$, i.e. for $|t| < 1/2$. That is, we can consider the \emph{Lax parametrix} to write the wave propagators, in an arbitrary coordinate system for $S^d$, as an oscillatory symbol
%
%\[ (e^{2 \pi i t P} f)(x) = \int_{\RR^d} a(t,x,y,\xi) e^{2 \pi i [ \phi(x,y,\xi) + t |\xi|_y ]} f(y)\; d\xi\; dy,  \]
%
%where $a$ is a symbol of order $0$, $|\xi|_y = \sqrt{ g^{jk}(y) \xi_j \xi_k }$ is obtained from the Riemannian metric of $S^d$ in the coordinate system, and $\phi$ solves the eikonal equation $| (\nabla_x \phi)(x,y,\xi) |_x = |\xi|_y$ such that $\phi(x,y,\xi) = 0$ for $(x - y) \cdot \xi = 0$. Similar oscillatory integrals \emph{can} be obtained for $e^{2 \pi i t P}$ simply by the fact that they are obtained by repeated compositions of the propogators with $|t| < 1/2$, and qualitative understanding of their behavior can be understand from the general theory of \emph{Fourier integral operators}, but obtaining quantitative control on $e^{2 \pi i t P}$ for large $t$ becomes difficult.

%There is an analogue of convolution on spheres, and with any zonal multiplier there is a 'zonal convolution kernel'. However, this kernel is difficult to relate to the symbol of the multiplier. More importantly, the convolution kernel \emph{does not behave well under dilation symmetries of the symbol}, which makes it difficult to relate different scales of the problem together.

\subsection{Combining Dyadic Pieces With Atomic Decompositions}

\begin{itemize}
	\item Next, we consider a decomposition $f = \sum f_k$, where the Fourier transform of $f_k$ is supported on $|\xi| \sim 2^k$, then we can write $Tf = \sum T_k f_k$. Bounds on $T_k$ have been controlled by the previous argument, and we are now left with the job of `recombining scales'. To obtain this bound, we consider a decomposition of each of the functions $f_k$ into `$L^\infty$ atoms'. Morally speaking, we are able to write $f_k = \sum A_{k,\theta}$, where $\theta$ runs over a family of dyadic boxes, each with side length exceeding $2^{-k}$, which morally we should think of as disjoint, and where $A_{k,\theta}$ is a 'molecule', which decomposes as $A_{k,\theta} = \sum a_{k,\theta,j}$, where $a_{k,\theta,j}$ is an `$L^\infty$ atom' on $\theta$, in the sense that $\| a_{k,j,\theta} \|_{L^\infty(\RR^d)} \lesssim |\theta|^{-1/p} \| a_{k,j,\theta} \|_{L^p(\RR^d)}$ and satisfy a square function estimate,
	%
%	\[ \left( \sum\nolimits_j \bigg\| \Big( \sum\nolimits_{k,\theta} |a_{k,j,\theta}|^2 \Big)^{\frac{1}{2}} \bigg\|_{L^p(\RR^d)}^p \right)^{1/p} \lesssim \| f \|_{L^p(\RR^d)}. \]
	%
	which reduces proving the bound $\| Tf \|_{L^p(\RR^d)} \lesssim \| f \|_{L^p(\RR^d)}$ to proving a bound of the form
	%
	\[ \left\| \sum u_{k,j,\theta} \right\| \lesssim \left( \sum\nolimits_j \Big\| \Big( \sum\nolimits_{k,\theta} |a_{k,j,\theta}|^2 \Big)^{1/2} \Big\|_{L^p(\RR^d)}^p \right)^{1/p}, \]
	%
	where $u_{k,j,\theta} = T_k a_{k,j,\theta}$. Unlike the previous argument, we thus have to deal with the interaction between different frequency scales, but here we do have an additional square root cancellation to help obtain the bound.
\end{itemize}

The argument I obtained also follows this pattern, but we must introduce several new techniques when adapting the method the method to the study of spherical harmonics.

	The uniformity in $k$ actually follows immediately if we can prove the bound for $k = 0$ because of the dilation symmetry on $\RR^d$, and the fact that we have uniform control over the functions $\{ a_k \}$. In the analysis of the Fourier multiplier $T_0$, the support of the symbol implies 

The bound is then obtained by a geometric argument involving incidences of annuli. Once this is obtained, a square function bound implies control over $\sum T_k$.

First off, their assumptions are about uniform control over dyadic pieces of the symbol $a$. More precisely, if $a$ is dyadically decomposed as a sum $a(\rho) = \sum_{k \in \ZZ} a_k( \rho / 2^k)$, where $a_k$ has support on $[1,2]$, then the necessary and sufficient condition for boundedness is that the quantities $C_p(a_k)$ are uniformly bounded in $k$, where
%



First, a general symbol $a$ is dyadically decomposed as a sum $a(\rho) = \sum_{k \in \ZZ} b_k( \rho / 2^k)$, where each of the functions $b_k$ is supported on the interval $[1,2]$. If we set $a_k(\rho) = a_k( 2^k \rho)$, then $T_a = \sum T_{a_k}$.n

The proofs begin by establishing bounds of the form $\| T_{a_k} f \|_{L^p(\RR^d)} \lesssim \| f \|_{L^p(\RR^d)}$, uniformly in $k$. By applying a dilation symmetry, it suffices without loss of generality to look at $a_0$.




Now how 


Suppose $a: [0,\infty) \to \CC$ is compactly supported on the interval $[1/2,2]$. Then the radial function $k(x) = \int a(|\xi|) e^{2 \pi i \xi \cdot x}\; d\xi$ is the a convolution kernel for the radial multiplier operator $T_a$, i.e. $T_a f = k * f$ for all inputs $f$. The bounds on radial multipliers obtained in BLAH are then of the form $\| T_a f \|_{L^p(\RR^d)} \lesssim \| k \|_{L^p(\RR^d)} \| f \|_{L^p(\RR^d)}$. Dilation symmetry then immediately implies BLAH DYADIC RESULT, and then some methods of atomic decompositions and Littlewood Paley theory can be combined to obtain a tight result. On a sphere, we \emph{can} define the zonal convolution kernel $k$ corresponding to a zonal multiplier $Z_a$, such that
%
\[ (Z_a f)(x) = \int_{S^d} k( y \cdot x ) f(y)\; dy. \]
%
If $a$ is supported on $[1/2,2]$, then weighted $L^p$ bounds on $k$ can be used to imply bounds on $Z_a$, but these bounds will not scale, and it is difficult to determine how the bounds scale under dilations since there is no relation between the zonal convolution kernel $k$ corresponding to $a$, and the convolution kernel corresponding to the dilations of $a$.

A fix is obtained by writing $Z_a f$ using the \emph{cosine transform} of $a$, i.e. in terms of
%
\[ \widehat{a}(t) = \int_0^\infty a(\rho) e^{2 \pi i \rho t}\; d\rho. \]
%
The cosine transform \emph{does} scale under dilations, and functional calculus and the Fourier inversion formula allows us to write $Z_a$ in terms of $\widehat{a}$, i.e. by setting
%
\begin{equation} \label{InversionFormula}
	Z_a f = a(P) f = \int_{-\infty}^\infty \widehat{a}(t) e^{2 \pi i t P}\; dt,
\end{equation}
%
where $P = \sqrt{ \alpha^2 - \Delta }$, and $e^{2 \pi i t P} f = e^{2 \pi i t k} f$ for a spherical harmonic $f$ of degree $k$. The operators $u(t) = e^{2 \pi i t P} f$ give solutions to the half-wave equation $\partial_t u = P u$. We can understand the geometric behaviour of the half-wave equation by using the theory of \emph{Fourier integral operators}.

. Indeed, if $f$ is a spherical harmonic of degree $k$, then $Z_a f = a(k) f$ and by the Fourier inversion formula $\int \widehat{a}(t) e^{2 \pi i t P} f\; dt = \int \widehat{a}(t) e^{2 \pi i t k} f\; dt = a(k) f(t)$. For any function $f$, the functions $u(t) = e^{2 \pi i t P} f$ give a solution to the wave equation $\partial_t^2 u(t) = P u(t)$ with $u(0) = f$ and $\partial_t u(0) = 0$.


The bounds on radial multipliers obtained in BLAH depend on bounds on the convolution kernel corresponding to the multiplier. 


The main goal of my research project on multipliers is to understand


 deconstructive interference between a family of planar waves, or spherical harmonics of different degrees. Necessary and sufficient conditions for a Fourier multiplier operator to be bounded on $L^1(\RR^d)$ or $L^\infty(\RR^d)$ were quickly realized.

%Understanding the boundedness of Fourier multiplier operators in an $L^p$ norm for $p \neq 2$ underpins any subtle understanding of the Fourier transform. Plane waves oscillating in different directions and with different frequencies are orthogonal to one another, and thus do not interact with one another significantly in terms of the $L^2$ norm, as justified by Bessel's inequality. But plane waves can interact with one another in the $L^p$ norm for $p \neq 2$, and so understanding $L^p$ bounds for Fourier multipliers indicate when this interaction is significant or insignificant. Similarily, spherical harmonics of different degrees on $S^d$ are orthogonal to one another, but studying the $L^p$ bounds of multipliers of the Laplacian on the sphere is crucial to understand when the interactions of different spherical harmonics are significant or not.

%The general study of the $L^p$ boundedness of general Fourier multipliers began in the 1960s, brought on by the spur of applications the Calderon-Zygmund school and their contemporaries brought to the theory. Necessary and sufficient conditions for a Fourier multiplier to be bounded on $L^2(\RR^d)$ follow by simple orthogonality, and conditions to be bounded on $L^1(\RR^d)$ and $L^\infty(\RR^d)$ also follow simply, because such spaces do not tend to allow much capacity for subtle cancellation. But the problem of finding necessary and sufficient conditions for boundedness in $L^p(\RR^d)$ for any other exponent proved impenetrable. Surprisingly, in the past decade necessary and sufficient conditions for $L^p$ boundedness have occured for \emph{radial} Fourier multipliers.

%Orthogonality immediately implies that a Fourier multiplier $T$ is bounded on $L^2(\RR^d)$ if and only if it's symbol $m$ lies in $L^\infty(\RR^d)$. Similarily, since spherical harmonics of different degrees are orthogonal to one another, a multiplier for spherical harmonic expansions is bounded on $L^2(S^d)$ if and only if $m \in L^\infty(\NN)$. The boundedness of such multipliers for $p \neq 2$ is a more subtle property, but underpins any deep understanding of the Fourier transform or the understanding of the interactions. Indeed, studying the boundedness of multipliers seems to be one of the few tractable ways of quantifying deconstructive interference between a family of planar waves, or spherical harmonics of different degrees. Necessary and sufficient conditions for a Fourier multiplier operator to be bounded on $L^1(\RR^d)$ or $L^\infty(\RR^d)$ were quickly realized.
%Spherical harmonics of different degrees are orthogonal to one another, and this immediately implies $T$ is bounded on $L^2(S^d)$ if and only if $m \in L^\infty(\NN)$. But characterizations of $L^p$ boundedness for all $p \neq 2$ remain unknown.


% It was quickly realized that a Fourier multiplier operator is bounded on $L^1(\RR^d)$ or $L^\infty(\RR^d)$ if and only if the Fourier transform of the symbol $m$ lies in $L^1(\RR^d)$.
% Indeed, many interesting problems about the boundedness of \emph{specific} Fourier multipliers, such as the Bochner-Riesz conjecture, remain largely unsolved today.

%It thus came as a recent surprise when necessary and sufficient conditions were found for bounding \emph{radial} Fourier multipliers (multipliers with a radial symbol) on $L^p(\RR^d)$. First came the result of \cite{GarrigosSeeger}, who found a necessary and sufficient condition in the range $|1/p - 1/2| > 1/2d$ for bounds of the form $\| Tf \|_{L^p(\RR^d)} \lesssim \| f \|_{L^p(\RR^d)}$ to hold uniformly over \emph{radial functions} $f$.
%If $T$ has symbol $m(|\xi|)$, then the condition says that $T$ is bounded if and only if the Fourier transforms of the functions $m_k(\xi) = m(2^k \xi) \chi(|\xi|)$ are uniformly bounded in $L^p(\RR^d)$, where $\chi \in C_c^\infty(0,\infty)$ and $\sum_{k \in \ZZ} \chi(2^k \xi) = 1$.
%An optimist might think this same condition causes the bound to hold uniformly over \emph{all functions} $f$ in the range above, a statement we call the \emph{radial multiplier conjecture}. We now know, by the results of \cite{HeoNazarovSeeger} and \cite{Cladek}, that the radial multiplier conjecture holds when $d > 4$ and $|1/p - 1/2| > 1/(d-1)$, when $d = 4$ and $|1/p - 1/2| > 11/36$, and when $d = 3$ and $|1/p - 1/2| > 11/26$. But the radial multiplier conjecture is not completely resolved for any $d$, and no bounds are known at all when $d = 2$.

%My main research project considers the study of the relation between the $L^p$ boundedness of two types of multipliers:
%
%\begin{itemize}
%	\item \emph{Fourier multipliers} are those operators $T$ on $\RR^d$ for which we can associate a function $m: \RR^d \to \CC$, known as the \emph{symbol} of $T$, such that for any function $f$, the Fourier transform of $Tf$ obeys the relation $\widehat{Tf} = m \widehat{f}$. Heuristically, this means that $T e^{2 \pi i \xi \cdot x} = m(\xi) e^{2 \pi i \xi \cdot x}$ for each $\xi \in \RR^d$.
%\end{itemize}

%  Every translation invariant operator on $\RR^d$ is a Fourier multiplier operator, and every rotation invariant operator on $S^d$ is a multiplier of the spherical harmonic expansion on $S^d$.

\end{comment}

\pagebreak[4]

\bibliography{ResearchStatement}
\bibliographystyle{plain}

\end{document}

Such operators initially arose from the study of the classical partial differential equations in physics, and continue to have applications in areas as diverse as partial differential equations, mathematical physics, number theory, and ergodic theory. Every translation invariant operator on $\RR^d$ is a Fourier multiplier operator, and every rotation invariant operator on $S^d$ is a multiplier of the spherical harmonic expansion on $S^d$.

any such operator $T$, we can associate a function $m: \RR^n \to \CC$, known as the \emph{symbol} of $T$, such that for any function $f$, the Fourier transform of $Tf$ obeys the relation $\widehat{Tf} = m \widehat{f}$; thus translation invariant operators are also called \emph{Fourier multiplier operators}.

Understanding the boundedness of Fourier multiplier operators in an $L^p$ norm for $p \neq 2$ underpins any subtle understanding of the Fourier transform. Plane waves oscillating in different directions and with different frequencies are orthogonal to one another, and thus do not interact with one another significantly in terms of the $L^2$ norm, as justified by Bessel's inequality. But plane waves can interact with one another in the $L^p$ norm for $p \neq 2$, and so understanding $L^p$ bounds for Fourier multipliers indicate when this interaction is significant or insignificant. Similarily, spherical harmonics of different degrees on $S^d$ are orthogonal to one another, but studying the $L^p$ bounds of multipliers of the Laplacian on the sphere is crucial to understand when the interactions of different spherical harmonics are significant or not.

The general study of the characterizations of $L^p$ boundedness for the Fourier multipliers was initiated in the 1960s. Mathematicians quickly found simple necessary and sufficient conditions that ensure Fourier multipliers are bounded on $L^1(\RR^d)$, $L^2(\RR^d)$, and $L^\infty(\RR^d)$. But the problem of finding necessary and sufficient conditions for boundedness in $L^p(\RR^d)$ for any other exponent proved impenetrable. Indeed, many interesting problems about the boundedness of \emph{specific} Fourier multipliers, such as the Bochner-Riesz conjecture, remain largely unsolved today.

Thus it came as a surprise in the past decade when results emerged proving necessary and sufficient conditions for \emph{radial} Fourier multipliers to be bounded on $L^p(\RR^d)$. First came the result of BLAH, which gave a necessary and sufficient criteria for bounds of the form $\| Tf \|_{L^p(\RR^d)} \lesssim \| f \|_{L^p(\RR^d)}$ to hold uniformly over \emph{radial functions} $f$, for $|1/p - 1/2| > 1/2d$. An optimist might think this same condition causes the bound to hold uniformly over \emph{all functions} $f$ in the range above, a statement we call the \emph{radial multiplier conjecture}. We now know, by the results of BLAH and BLAH, that the radial multiplier conjecture holds when $d > 4$ and $|1/p - 1/2| > 1/(d-1)$, when $d = 4$ and $|1/p - 1/2| > 11/36$, and when $d = 3$ and $|1/p - 1/2| > 11/26$. But the radial multiplier conjecture has not yet been completely solved in any dimension $d$, and no bounds are known at all when $d = 2$.

The natural analogue of the study of radial multipliers on $\RR^d$ is the study of multipliers of a Laplace-Beltrami operator on a Riemannian manifold. The natural analogue of the study of quasiradial multipliers on $\RR^d$ is the study of multipliers of an operator associated with a \emph{Finsler geometry} on the manifold.