\documentclass[11pt]{article}

\usepackage[a4paper, margin=1in]{geometry}

\usepackage{amssymb,amsmath,amsthm}

\DeclareMathOperator{\QQ}{\mathbb{Q}}
\DeclareMathOperator{\ZZ}{\mathbb{Z}}
\DeclareMathOperator{\RR}{\mathbb{R}}
\DeclareMathOperator{\NN}{\mathbb{N}}
\DeclareMathOperator{\HH}{\mathbb{H}}
\DeclareMathOperator{\BB}{\mathbb{B}}
\DeclareMathOperator{\CC}{\mathbb{C}}
\DeclareMathOperator{\AB}{\mathbb{A}}
\DeclareMathOperator{\PP}{\mathbb{P}}
\DeclareMathOperator{\MM}{\mathbb{M}}
\DeclareMathOperator{\VV}{\mathbb{V}}
\DeclareMathOperator{\TT}{\mathbb{T}}
\DeclareMathOperator{\LL}{\mathcal{L}}
\DeclareMathOperator{\DD}{\mathcal{D}}
\DeclareMathOperator{\SW}{\mathcal{S}}
\DeclareMathOperator{\EC}{\mathcal{E}}
\DeclareMathOperator{\AC}{\mathcal{A}}

\title{Research Statement}
\author{Jacob Denson}
\date{}

\begin{document}

\maketitle

{\bf My Research} focuses on problems in Harmonic analysis. In particular, I study Fourier multiplier operators on Euclidean space, and their analogues on compact manifolds, through an understanding of the geometry of wave propagation on these spaces. I also work on problems in harmonic analysis related to geometric measure theory, investigating when `structure' occurs in fractals of large dimension. Both projects lead to interesting questions that I plan to pursue in my postgraduate work.

During my PhD, much of my work on multipliers has concentrating on relating bounds on Fourier multiplier operators on $\RR^d$ to bounds for analogous operators on compact manifolds. I proved a `transference principle' \cite{DensonCharacterization} for zonal multipliers operators on the sphere $S^d$. For $d \geq 4$ and a range of $p$, this principle shows that $L^p$ bounds for a radial Fourier multiplier with symbol $m(|\xi|)$ imply bounds for a zonal multiplier operators on $S^d$ induced by the same symbol. In the process, I also completely characterized those symbols $m$ whose dilates give a uniformly bounded family of zonal multiplier operators on $L^p(S^d)$. This is the first such characterization on $S^d$ for any $p \neq 2$, and no other such characterization, or transference principle of this form has been proved for analogous operators on any other compact manifold.

My work in geometric measure theory focuses on constructing sets of large fractal dimension avoiding certain point configurations. Together with Malabika Pramanik and Joshua Zahl,	I obtained a method \cite{DensonPramanikZahl} for constructing sets avoiding configurations which have large Hausdorff dimension if the geometric configuration itself is 'low dimensional'. During my PhD, I continued this line of research by establishing several probabilistic extensions of the methods of the previous paper to address the more difficult problem of constructing sets of large \emph{Fourier dimension} avoiding configurations \cite{DensonFourier}. This method remains the only construction method for constructing sets of large Fourier dimension avoiding nonlinear configurations, and also remains the best currently known construction method for constructing sets avoiding general 'linear' configurations when $d > 1$.

%  Using techniques of high dimensional probability, I was able to fully recover the Hausdorff bound obtained in \cite{DensonPramanikZahl} in the Fourier dimension setting with the additional of a weak linearity assumption. An example application of the method is, for any smooth curve $c: [0,1] \to \RR^d$, a construction of a subset $X$ of $[0,1]$ of Fourier dimension $0.4$ such that $c(X)$ does not contain three vertices of an isosceles triangle. 

{\bf In The Near Future}, I hope to generalize the bounds obtained in \cite{DensonCharacterization} to the more general setting of multipliers for eigenfunction expansions of any Laplace-Beltrami operator on a Riemannian manifold $M$ with periodic geodesic flow. A local obstruction here requires obtaining control of a pseudodifferential operator on $M$ called the `return operator'. A global obstruction is an endpoint refinement of the local smoothing inequality for the wave equation on $M$. %to handle combining dyadic scales together.
I am also interested in exploring what kinds of bounds for multipliers can be obtained via wave equation type techniques on manifolds whose geodesic flow has well controlled dynamical properties, such as forming an integrable system. In the study of patterns, I hope to apply the square root cancellation techniques I exploited in the construction of sets of large Fourier dimension to construct random fractals with good decoupling constants. And I am interested in determining the interrelation of patterns with the study of multipliers on manifolds, in particular studying Falconer distance type problems on manifolds by using local smoothing bounds, and exploring analogues of Fourier dimension on Riemannian manifolds.

In the remainder of this summary, I describe more formally the contributions I have made to the problems mentioned above, and finish with a further discussion of how the are feasible given the insights I have gained from work on previous problems.

\pagebreak[3]






\section{Multiplier Operators On $\RR^d$ And On Manifolds}

%Much of my PhD has dealt with the study of bounds for radial Fourier multipliers on $\RR^d$, bounds for multipliers of eigenfunction expansions on compact manifolds, and the relation between such bounds.

Multipliers have been a central object in harmonic analysis since the field's inception. In his pioneering work, Fourier showed that solutions to the classical equations of physics are described by \emph{Fourier multipliers}, operators $T$ defined from a function $m: \RR^d \to \CC$, the \emph{symbol}, by setting
%
\[ Tf(x) = \int_{\RR^d} m(|\xi|) \widehat{f}(\xi) e^{2 \pi i \xi \cdot x}\; dx. \]
%
Of particular interest are the \emph{radial} multipliers $T_a$, defined for a function $a: [0,\infty) \to \CC$ as the Fourier multiplier with symbol $m(\xi) = a(|\xi|)$. %We now know
Any translation-invariant operators on $\RR^d$ is a Fourier multiplier operators, explaining their broad applicability in areas as diverse as partial differential equations, number theory, complex variables, and ergodic theory.

A similar theory of multiplier operators can be developed on the sphere $S^d$. Roughly speaking, Fourier multipliers are operators on $\RR^d$ with $e^{2 \pi i \xi \cdot x}$ as eigenfunctions. Zonal multipliers on $S^d$ are those operators with the \emph{spherical harmonics} as eigenfunctions, i.e. the restrictions to $S^d$ of homogeneous harmonic polynomials on $\RR^{d+1}$. Every function $f \in L^2(S^d)$ can be uniquely expanded as $\sum_{k = 0}^\infty H_k f$, where $H_k f$ is a degree $k$ spherical harmonics. A \emph{zonal multiplier} is then an operator on $S^d$ defined in terms of a function $a: \NN \to \CC$ by setting
%
\[ Z_a f = \sum a(k) H_k f. \]
%
%Just as any translation-invariant operator on $\RR^d$ is a Fourier multiplier,
Every rotation invariant operator on $S^d$ is a zonal multiplier, and thus such operators arise in diverse applications, including celestial mechanics, physics, and computer graphics.

In harmonic analysis, it has proved incredibly profitable to study the boundedness of Fourier multipliers with respect to the various $L^p$ norms. It seems to be one of the few tractable ways of quantifying how different types of planar waves interact with one another, thus underpinning all deeper understandings of the Fourier transform. Similarly, understanding the $L^p$ boundedness of zonal multipliers gives insight into how spherical harmonics interact with one another.

%The general theory of such multipliers became of central interest in the 1960s, brought on by the spur of applications the Calderon-Zygmund school and their contemporaries brought to the theory.
However, completely characterizing those symbols which induce $L^p$ bounded multipliers has proved an impenetrable, if not potentially impossible problem, aside from trivial cases where $p \in \{ 1, 2, \infty \}$. Indeed, many interesting problems about the boundedness of \emph{specific} multipliers, such as the Bochner-Riesz conjecture, remain largely unsolved today. Various techniques have recently been developed towards an understanding of the Bochner-Riesz conjecture. But most of these techniques involving induction on scales, such as broad-narrow analysis or decoupling, or even basic applications of dyadic pigeonholing, cannot be used in characterizations of $L^p$ bounded multipliers; such methods must allow for a $R^\varepsilon$ or $\log R$ loss in the frequency scale $R$, which is permissible in the analysis of Bochner-Riesz multipliers given they lie in a range of Besov spaces, which allow for an epsilon of room at the dyadic scale when interpolating. But an arbitrary multiplier bounded on $L^p$ does not have such a property, and indeed may not even lie in any of the standard Besov spaces (for instance, consider the radial multipliers with symbol $a_t(\rho) = t^{-(d-1)(1/p-1/2)} \chi(\rho) e^{it \rho}$, which are uniformly bounded on $L^p(\RR^d)$).
%
%
% d(1/p - 1/2) derivatives in L^2
% (d-1)(1/p - 1/2) derivatives in L^p
%
% The multiplier m_t(rho) = chi(rho) e^{-it rho}
%
% Has L^p multiplier norm O( t^{(d-1)(1/p - 1/2)} )
%
% If we take s derivatives of m_t, the answer
% 	is a sum of terms of the form chi_k(rho) t^k e^{it rho}
% 	where k <= s. Thus the Fourier transform
%
% 	is equal to k_t(t') = t^k chi_k^( t - t' )
% 	The L^p norm of this is O(t^k).
%
% Thus C_p(m_t) << t^{(d-1)(1/p - 1/2)}
%
% But the Fourier transform of the (d-1)(1/p - 1/2) + epsilon
% derivative is O( t^{(d-1)(1/p - 1/2)} )
%
% Thus epsilon loss techniques cannot analysis sum_k 2^{-k(d-1)(1/p - 1/2)} m_{2^k}(rho)
% 
%
%
%

%The general study of the $L^p$ boundedness of general Fourier multipliers began in the 1960s, brought on by the spur of applications the Calderon-Zygmund school and their contemporaries brought to the theory. 

%But aside from certain rather trivial cases where $p \in \{ 1, 2 , \infty \}$, no necessary and sufficient conditions on an operator's symbol to ensure boundedness on $L^p(\RR^d)$ have been found.
It thus came as a surprise in the past decade when several arguments \cite{GarrigosSeeger,HeoNazarovSeeger,Cladek,KimQuasiradial} emerged giving necessary and sufficient conditions on a symbol $a$ for the corresponding \emph{radial} Fourier multipliers to be bounded on $L^p(\RR^d)$. By duality, we may assume without loss of generality that $1 \leq p \leq 2$.
%By duality, a Fourier multiplier is bounded on $L^p(\RR^d)$ if and only if it is bounded on $L^{p'}(\RR^d)$, so it is sufficient to consider the case $1 \leq p \leq 2$.
Consider a decomposition $a(\rho) = \sum a_k( \rho / 2^k)$, where $a_k(\rho) = 0$ for $\rho \not \in [1,2]$. For $T_a$ to be bounded on $L^p(\RR^d)$ for $p \in [1,2]$, testing by Schwartz functions reveals it is necessary that $\sup_j C_p(a_j) < \infty$, where
%
\[ C_p(a) = \left( \int_0^\infty \left[ t^{\alpha(p)} \widehat{a}(t) \right]^p\; dt \right)^{1/p} \quad\text{and}\quad \alpha(p) = (d-1)(1/p - 1/2). \]
%
%and where $\widehat{a}(t) = \int_0^\infty a(\rho) e^{2 \pi i \rho t}\; dt$ is the \emph{cosine transform} of the function $a$.
The papers prove that this condition is \emph{sufficient} for a certain range of exponents, and conjecture that this condition is also sufficient for $|1/p - 1/2| > 1/2d$, though resolving this conjecture is likely far beyond current research techniques given it implies the Bochner-Riesz conjecture, and thus also the restriction and Kakeya conjectures. A weaker condition is that the Besov norm $\| a \|_{B^{d(1/p-1/2)}_{2,p}}(\RR)$ is finite, a condition which brings us closer to the Bochner-Riesz range. But this also lies beyond the scope of current techniques.

{\bf The Main Goal} of my research project was to find analogues of these arguments in the study of zonal multipliers on $S^d$. Through this analogy, I proved the first transference principle from Fourier multiplier bounds to zonal multiplier bounds. Namely, for $d \geq 4$ and $|1/p - 1/2| > 1/(d-1)$, if a Fourier multiplier operator $T_a$ is bounded on $L^p(\RR^d)$, I proved that the zonal multiplier operator $Z_a$ is bounded on $L^p(S^d)$. My proof also completely characterizes those symbols whose dilates give a uniformly bounded family of zonal multiplier operators on $L^p(S^d)$. This result, in the special case when $a$ has compact support, has been submitted for publication in \cite{DensonCharacterization}. The method behind the general case will be submitted for publication shortly, with some of the ideas of which will be discussed in this section. In the remainder of this section, we discuss this result in more detail, emphasizing the new techniques introduced.

\pagebreak[3]

\subsection*{Relations Between Fourier Multipliers and Zonal Multipliers}

One connection which explains why analogues to the bounds for radial Fourier multipliers might be found in the study of zonal multipliers is that both classes of operators are related to the Laplace operator on their respective spaces. Namely, if $f$ is a distribution on $\RR^d$, and $\Delta f = - \lambda^2 f$, then $f$ has Fourier support on the sphere of radius $\lambda$ centered at the origin, and so $T_a f = a(\lambda) f$. Similarly, if $\Delta$ is the Laplace-Beltrami operator on $S^d$, and $\Delta f = - \lambda(\lambda+d-1)$, then $f$ is a spherical harmonic of degree $\lambda$, and so $Z_a f = a(\lambda) f$.
%If $f$ is a distribution on $\RR^d$ and the support of it's Fourier transform lies on the sphere of radius $k$ about the origin, then $\Delta f = - k^2 f$ and $T_a f = a(k) f$.
%Thus $T_a$ and $\Delta$ have the same eigenfunctions.
%If the Fourier transform of a tempered distribution $f$ is supported on the sphere $\{ \xi : |\xi| = R \}$, then it follows from the Fourier inversion formula that $\Delta f = - 4 \pi^2 R^2 f$.
%This essentially follows because then $f$ is a superposition of the planar waves $e^{2 \pi i \xi \cdot x}$ with $|\xi| = R$, and $\Delta e^{2 \pi i \xi \cdot x} = - 4\pi^2 R^2 f$.
%But for the same such $f$, if $T_a$ is a radial Fourier multiplier, then $T_a f = a(R) f$. Thus a radial Fourier multiplier has the same eigenfunctions of the Laplacian.
%Similarly, if $\Delta$ is the Laplace-Beltrami operator on $S^d$, and $f$ is a spherical harmonic of degree $k$, then $\Delta f = - k(k+d-1) f$, and $Z_a f = a(k) f$.
Using the notation of functional calculus, we can thus write $T_a = a \big( \sqrt{-\Delta} \big)$ and $Z_a = a \big( \sqrt{ \alpha^2 - \Delta} \big)$, where $\alpha = (d-1)/2$, and now the resemblance is clear. In the rest of this section, we let $P = \sqrt{\alpha^2 - \Delta}$, and let $a(P/R)$ denote the zonal multiplier with symbol $a(\cdot/R)$.

On the other hand, unlike the planar waves $e^{2 \pi i \xi \cdot x}$, it is difficult to understand what a general spherical harmonic might look. Dilation symmetries on $\RR^d$ tell us high frequency planar waves are just the dilates of low frequency planar waves; on the other hand, $S^d$ has no dilation symmetries, and high degree spherical harmonics need not look anything like low degree spherical harmonics. This could be alarming, because the transference principle we hope to prove implies a result related to dilation on $S^d$; namely, if the Fourier multiplier $T_a$ is bounded on $L^p(\RR^d)$, then the Fourier multipliers $T_{a,R}$ with symbols $a(\cdot/R)$ are all uniformly bounded on $L^p(\RR^d)$, and thus the transference principle we hope to prove shows that the zonal multipliers $a(P/R)$ are uniformly bounded on $L^p(S^d)$. Because of the lack of dilation symmetry on $S^d$, the behavior of the operators $a(P/R)$ might change as $R$ varies, which is discouraging.
 %Moreover, the eigenvalues of $\Delta$ on $S^d$ are discrete, whereas the eigenvalues of $\Delta$ on $\RR^d$ range over $(-\infty,0]$. This is discouraging, since the condition $\sup_j C_p(a_j)$ exploited controls the \emph{smoothness} of the function $a$, a property that should be irrelevant to zonal multipliers, given that only discrete values of the symbol are required in the definition of the operator. We should therefore not expect to find a complete characterization for zonal multipliers. % using the methods which bound radial Fourier multipliers.

%However, various heuristics tell us the operators $- \Delta / R^2$ on $\RR^d$ and $(\alpha^2 - \Delta) / R^2$ on $S^d$ behave asymptotically like one another as $R \to \infty$. Thus we might hope to find a completely characterization of the functions $a$ such that the operators $\{ Z_{a,R} \}$ are uniformly bounded on $L^p(S^d)$, where $Z_{a,R} = a( P/R )$ is the zonal multiplier operator with symbol $a_R(\rho) = a( \rho / R)$. In fact, a classic result of Mitjagin \cite{Mitjagin} proves that if the operators $\{ Z_{a,R} \}$ are uniformly bounded on $L^p(S^d)$, then $T_a$ is bounded on $L^p(\RR^d)$. Thus, if we could show that $\sup_j C_p(a_j) < \infty$ implies that the operators $\{ Z_{a,R} \}$ are uniformly bounded, then we would have obtained a \emph{transference principle} for such operators, that an operator $T_a$ is bounded on $L^p(\RR^d)$ if and only if the operators $\{ Z_{a,R} \}$ are uniformly bounded on $L^p(S^d)$. %and in particular that the operators $\{ Z_{a,R} \}$ are uniformly bounded if and only if $\sup_j C_p(a_j)$ is finite.

Fortunately, whatever differences the operators $a(P/R)$ have as $R$ varies are not as relevant to the study of $L^p$ boundedness as one might at first think. This is because zonal multipliers only fail to be bounded on $L^p(S^d)$ because of `high frequency behavior'; a zonal multiplier whose symbol is compactly supported is bounded on all of the $L^p$ spaces. And there are various heuristics that tell us that the Laplacian on $S^d$ begins to behave more and more similar to the Laplacian on $\RR^d$ when restricted to high frequency eigenfunctions. For instance, on $S^d$, the operator $P/R$ can be written as $\sqrt{ \alpha^2/R + \Delta_R }$, where $\Delta_R$ is the Laplacian associated with the metric $g_R = R^2 g$ on $S^d$. As $R \to \infty$, the metric $g_R$ gives $S^d$ less curvature and more volume, and so we might imagine that, as $R \to \infty$, the operator $P/R$ behaves more and more like $\sqrt{-\Delta}$ on $\RR^d$, and thus multipliers of $P/R$ behave more and more like radial Fourier multipliers, i.e. we might imagine that the equation $T_a = \lim_{R \to \infty} a(P/R)$ holds in a certain heuristic sense.

The last equation also leads us to believe that if the operators $a(P/R)$ are uniformly bounded on $L^p(S^d)$, then $T_a$ is bounded on $L^p(\RR^d)$. This is true for all $1 \leq p \leq \infty$, a classical transference result of Mitjagin \cite{Mitjagin}. On the other hand, the transference principle we prove is more unusual: the boundedness of a limit point need not necessitate uniform bounds on the limiting sequence. This might explain why the principle is more difficult to prove. Indeed, Mitjagin's argument
% Moreover, recent results on the failure of the Kakeya conjecture in three dimensional manifolds with non-constant sectional curvature \cite{DaiGongGuoZhang} provide weak evidence that the range of $L^p$ spaces to which inequality \eqref{upperboundmainresult} can be obtained may be more sensitive to the geometry of a manifold $M$ than the range of $L^p$ spaces under which \eqref{mitjagininequalityforlaplacian} holds. Obtaining bounds of the form \eqref{upperboundmainresult} therefore requires a more subtle analysis than required to obtain bounds of the form \eqref{mitjagininequalityforlaplacian}.
%In this setting, we were able to prove this transference principle on $S^d$ for all $d \geq 4$ and $|1/p - 1/2| > 1/(d-1)$,the same range of parameters under which the results of \cite{HeoNazarovSeeger} holds. A result like this is significant, since no analogous result has been proved on any other compact manifold.
Mitjagin's argument generalizes to show that for any compact manifold $M$, and any elliptic, self-adjoint pseudodifferential operator $P$, the uniform boundedness of the multiplier operators $a(P/R)$ on $L^p(M)$ implies the boundedness of the Fourier multiplier $T$ on $L^p(\RR^d)$ whose symbol is given by the principal symbol of $P$. Until our new transference principle, no analogous transference principle has been shown for any $P$ and any exponent $p \neq 2$, excluding trivial cases, nor any characterization of functions $a$ such that the operators $\{ a(P/R) \}$ are uniformly bounded on $L^p(M)$ for any $p \neq 2$.

\subsection*{Bounding Band Limited Zonal Multipliers}

To discuss the techniques I developed which lead to the aforementioned characterization, let us begin by summarizing the rough methodology by which the arguments in \cite{HeoNazarovSeeger,Cladek,KimQuasiradial} are able to obtain these bounds. We  begin by describing the \emph{band limited} part of the argument:

\pagebreak[3]

  \medskip
  \hangindent0.5em
  \hangafter=0
    \noindent{\emph{Take a decomposition $T_a = \sum_{j \in \ZZ} T_j$, where $T_j$ is the multiplier operator with symbol $a_j( \cdot / 2^j)$. Our goal is the band limited bound $\| T_j f \|_{L^p(\RR^d)} \lesssim \| f \|_{L^p(\RR^d)}$, uniformly in $j$ % The advantage of this decomposition is that we may assume the inputs and outputs of the operator $T_j$ are band limited and thus we can apply various and heuristics related to the uncertainty principle.
    We then write $T_j f = k * f$, where $k$ is the Fourier transform of $a_j(\cdot / 2^j)$. We then write $k = \sum k_\tau$ and $f = \sum f_\theta$, where the functions $\{ k_\tau \}$ are supported on disjoint thickness $2^{-j}$ annuli, and the functions $\{ f_\theta \}$ are supported on disjoint side length $2^{-k}$ cubes, then $T_j f = \sum_{\tau,\theta} k_\tau * f_\theta$. Using the fact that the Fourier transform is unitary, that the function $f$ is band limited, and some Bessel function estimates, one can argue that the inner product $\langle k_\tau * f_\theta, k_{\tau'} * f_{\theta'} \rangle$ is negligible unless the annulus of radius $\tau$ centered at $\theta$ is near tangent to the annulus of radius $\tau'$ centered at $\theta'$.
    }}

  \medskip
  \hangindent0.5em
  \hangafter=0
  \noindent{\emph{Using the inner product estimate, together with an argument for counting incidences, one can show that the $L^2$ norm of a sum $\sum_{(\tau,\theta) \in \mathcal{E}} k_\tau * f_\theta$ is well behaved if $\mathcal{E}$ is suitably 'sparse', and interpolation with a trivial $L^1$ estimate yields an $L^p$ estimate on the sum. Conversely, if the set $\mathcal{E}$ is clustered, then $\sum_{(\tau,\theta) \in \mathcal{E}} k_\tau * f_\theta$ will be concentrated on only a few annuli, and so we can also get good $L^p$ estimates. But then we can estimate $\| Tf \|_{L^p(\RR^d)} = \| \sum k_\tau * f_\theta \|_{L^p(\RR^d)}$ by either approach, depending on whether a sparse part of the sum dominates, or whether a clustered part of the sum dominates.
}}

\vspace{0.3em}

\noindent My paper \cite{DensonCharacterization} proves an analogue of this argument for zonal multiplier operators on $S^d$ for $d \geq 4$ and $|1/p - 1/2| > 1/(d-1)$, in particular, obtaining the aforementioned transference principle in this range.

Giving that the argument above exploits convolution on $\RR^d$, one might expect that we might use `zonal convolution' in our argument, i.e. an analogue of convolution on $S^d$. It is likely one can use this approach to obtain $L^p$ bounds under assumptions on the integrability of the zonal convolution kernel. However, the lack of a dilation symmetry on $S^d$ means that the zonal convolution kernel for $a(P/R)$ is likely unrelated to the convolution kernel for $a(P)$ as $R$ varies, so based on our previous discussion it is likely difficult to obtain a transference principle using this technique. Instead, we follow an approach due to H\"{o}rmander BLAH, successfully used in several other problems on manifolds BLAH, and use the Fourier inversion formula to write
%
\[ Z_j f = \int_{-\infty}^\infty 2^j \widehat{a}_j( 2^j t ) e^{2 \pi i t P} f\; dt, \]
%
where $P = \sqrt{ \alpha^2 - \Delta }$ is as above, and $e^{2 \pi i t P}$ are the wave propagator operators which, as $t$ varies, give solutions to the wave equation $\partial_t^2 = \Delta - \alpha^2$ with zero velocity initial conditions, or equivalently, solutions to $\partial_t = P$, the `half-wave equation'. Given a general input $f$, we perform a decomposition analogous to the method above, writing $f = \sum f_\theta$ and $T_j = \sum T_\tau$, where the functions $\{ f_\theta \}$ are supported on disjoint sets of diameter $2^{-k}$, and $T_\tau = \int b_\tau(t) e^{2 \pi i t P}\; dt$, where $2^j \widehat{a}_j( 2^j t ) = \sum_\tau b_\tau(t)$ for a family of functions $\{ b_\tau \}$ is supported on disjoint side length $2^{-j}$ intervals. We can thus write $T_j f = \sum T_\tau f_\theta$.

The behavior of the wave equation is closely tied to the behavior of geodesics on $S^d$. In particular, for high frequency inputs we have a near explicit representation of the wave propagator operators for $|t| < 1/2$, in a coordinate system, by an oscillatory integral
%
\[ (e^{2 \pi i t P} f)(x) \approx \int_{\RR^d} a(t,x,y,\xi) e^{2 \pi i [ \phi(x,y,\xi) + t |\xi|_y ]} f(y)\; d\xi\; dy,  \]
%
where $a$ is a symbol of order $0$, $|\xi|_y = (\sum g^{jk}(y) \xi_j \xi_k)^{1/2}$ is obtained from the Riemannian metric of $S^d$, and $\phi$ solves the eikonal equation $| (\nabla_x \phi)(x,y,\xi) |_x = |\xi|_y$ subject to the constraint that $\phi(x,y,\xi) = 0$ for $(x - y) \cdot \xi = 0$.

Oscillatory integral representations of the operators $\{ e^{2 \pi i t P} \}$ \emph{can} be obtained for $e^{2 \pi i t P}$ simply by the fact they form a semigroup, and so for $n-1 \leq t < n$ we can write $e^{2 \pi i t P} = ( e^{2 \pi i (t / n) P} )^n$ as the composition of $n$ oscillatory integrals. The theory of phase reduction for Fourier integral operators can help us reduce this composition, but it is difficult to control this quantity quantitatively.

simply by the fact that they are obtained by repeated compositions of the propogators with $|t| < 1/2$, and qualitative understanding of their behavior can be understand from the general theory of \emph{Fourier integral operators}, but obtaining good control on $e^{2 \pi i t P}$ for large $t$ becomes difficult.

Using this oscillatory integral representation and the stationary phase formula, we can obtain a substitute for the Bessel estimates used in the original argument, justifying that $\langle T_\tau f_\theta, T_{\tau'} f_{\theta'} \rangle$ is negligible unless the geodesic annulus of radius $\tau$ and center $\theta$ is near tangent to the geodesic annulus of radius $\tau'$ and center $\theta'$, provided that $\tau$ is bounded away from $1$. Here we use the stationary phase formula to obtain good bounds on the oscillatory integrals that emerge. However, one subtlety is showing that, restricted to values in $|\xi| = 1$, the critical points of the function $\phi(x,y,\cdot)$ are appropriately non-degenerate. Using the Hamilton-Jacobi approach to the study of the eikonal equation, one can identify the quantity $\phi(x,y,\xi)$ with the signed distance from the hyperplane $\{ x' : (x' - y) \cdot \xi = 0 \}$ to the point $x$ with respect to the Riemannian metric. I came up with a geometric argument involving the second variation formula for geodesics, which justifies that the function $\phi$ has only two stationary points, and each is non-degenerate, with the Hessian at each point having magnitude proportional to $d_g(x,y)^{d-1}$.

After this, the argument for radial Fourier multipliers generalizes quite directly to the case of $S^d$, since the incidence properties required for annuli on $\RR^d$ are roughly analogous to the properties of geodesic annuli on $S^d$.


% this method likely leads to a condition on the different dyadic functions $a_j$ which depends on a $j$ in a rather difficult to understand way, reflecting the lack of a dilation symmetry on the manifold. For similar reasons, 

%. A general zonal multiplier $Z_a$ can be decomposed as $\sum_j Z_j$ simply by decomposing it's symbol $a(\lambda) = \sum a_j( \lambda / 2^j )$, and letting $Z_j$ be the zonal multiplier with symbol $a_j( \cdot / 2^k )$. There is an analogue of convolution on the sphere, called zonal convolution, and so we could write $Z_j$ in terms of this convolution, and decompose as above into pieces supported on different annuli. However, the zonal convolution kernel of an operator does not behave well under dilation of the symbol of the operator (a manifestation of the fact $S^d$ does not have a family of dilation symmetries) so the condition we would need to finish our argument would likely involve rather non explicit assumptions about the various functions $\{ a_j \}$. Instead, we directly use the Fourier transform of $a_j$, which scales under dilations.

%We are able to use this by using the Fourier inversion formula and the functional calculus to write
%
%\[ Z_j = \int_{-\infty}^\infty 2^k \widehat{a}_j( 2^k t ) e^{2 \pi i t P}\; dt, \]
%
%where $P = \sqrt{ \alpha^2 - \Delta }$ is as above, and $e^{2 \pi i t P}$ are the wave propagator operators which, as $t$ varies, give solutions to the wave equation $\partial_t^2 = \Delta - \alpha^2$ with zero velocity initial conditions, or equivalently, solutions to $\partial_t = P$, the `half-wave equation'. The behavior of this wave equation is closely tied to the behavior of geodesics on $S^d$. In particular, we have a close to explicit representation of the wave propagator operators for $t$ having magnitude smaller than $2\pi$ times the \emph{injectivity radius} on $S^d$, i.e. for $|t| < 1/2$. That is, we can consider the \emph{Lax parametrix} to write the wave propagators, in an arbitrary coordinate system for $S^d$, as an oscillatory symbol
%
%\[ (e^{2 \pi i t P} f)(x) = \int_{\RR^d} a(t,x,y,\xi) e^{2 \pi i [ \phi(x,y,\xi) + t |\xi|_y ]} f(y)\; d\xi\; dy,  \]
%
%where $a$ is a symbol of order $0$, $|\xi|_y = \sqrt{ g^{jk}(y) \xi_j \xi_k }$ is obtained from the Riemannian metric of $S^d$ in the coordinate system, and $\phi$ solves the eikonal equation $| (\nabla_x \phi)(x,y,\xi) |_x = |\xi|_y$ such that $\phi(x,y,\xi) = 0$ for $(x - y) \cdot \xi = 0$. Similar oscillatory integrals \emph{can} be obtained for $e^{2 \pi i t P}$ simply by the fact that they are obtained by repeated compositions of the propogators with $|t| < 1/2$, and qualitative understanding of their behavior can be understand from the general theory of \emph{Fourier integral operators}, but obtaining quantitative control on $e^{2 \pi i t P}$ for large $t$ becomes difficult.

%There is an analogue of convolution on spheres, and with any zonal multiplier there is a 'zonal convolution kernel'. However, this kernel is difficult to relate to the symbol of the multiplier. More importantly, the convolution kernel \emph{does not behave well under dilation symmetries of the symbol}, which makes it difficult to relate different scales of the problem together.

\subsection*{Combining Dyadic Pieces With Atomic Decompositions}

\begin{itemize}
	\item Next, we consider a decomposition $f = \sum f_k$, where the Fourier transform of $f_k$ is supported on $|\xi| \sim 2^k$, then we can write $Tf = \sum T_k f_k$. Bounds on $T_k$ have been controlled by the previous argument, and we are now left with the job of `recombining scales'. To obtain this bound, we consider a decomposition of each of the functions $f_k$ into `$L^\infty$ atoms'. Morally speaking, we are able to write $f_k = \sum A_{k,\theta}$, where $\theta$ runs over a family of dyadic boxes, each with side length exceeding $2^{-k}$, which morally we should think of as disjoint, and where $A_{k,\theta}$ is a 'molecule', which decomposes as $A_{k,\theta} = \sum a_{k,\theta,j}$, where $a_{k,\theta,j}$ is an `$L^\infty$ atom' on $\theta$, in the sense that $\| a_{k,j,\theta} \|_{L^\infty(\RR^d)} \lesssim |\theta|^{-1/p} \| a_{k,j,\theta} \|_{L^p(\RR^d)}$ and satisfy a square function estimate,
	%
%	\[ \left( \sum\nolimits_j \bigg\| \Big( \sum\nolimits_{k,\theta} |a_{k,j,\theta}|^2 \Big)^{\frac{1}{2}} \bigg\|_{L^p(\RR^d)}^p \right)^{1/p} \lesssim \| f \|_{L^p(\RR^d)}. \]
	%
	which reduces proving the bound $\| Tf \|_{L^p(\RR^d)} \lesssim \| f \|_{L^p(\RR^d)}$ to proving a bound of the form
	%
	\[ \left\| \sum u_{k,j,\theta} \right\| \lesssim \left( \sum\nolimits_j \Big\| \Big( \sum\nolimits_{k,\theta} |a_{k,j,\theta}|^2 \Big)^{1/2} \Big\|_{L^p(\RR^d)}^p \right)^{1/p}, \]
	%
	where $u_{k,j,\theta} = T_k a_{k,j,\theta}$. Unlike the previous argument, we thus have to deal with the interaction between different frequency scales, but here we do have an additional square root cancellation to help obtain the bound.
\end{itemize}

The argument I obtained also follows this pattern, but we must introduce several new techniques when adapting the method the method to the study of spherical harmonics.

	The uniformity in $k$ actually follows immediately if we can prove the bound for $k = 0$ because of the dilation symmetry on $\RR^d$, and the fact that we have uniform control over the functions $\{ a_k \}$. In the analysis of the Fourier multiplier $T_0$, the support of the symbol implies 

The bound is then obtained by a geometric argument involving incidences of annuli. Once this is obtained, a square function bound implies control over $\sum T_k$.

First off, their assumptions are about uniform control over dyadic pieces of the symbol $a$. More precisely, if $a$ is dyadically decomposed as a sum $a(\rho) = \sum_{k \in \ZZ} a_k( \rho / 2^k)$, where $a_k$ has support on $[1,2]$, then the necessary and sufficient condition for boundedness is that the quantities $C_p(a_k)$ are uniformly bounded in $k$, where
%



First, a general symbol $a$ is dyadically decomposed as a sum $a(\rho) = \sum_{k \in \ZZ} b_k( \rho / 2^k)$, where each of the functions $b_k$ is supported on the interval $[1,2]$. If we set $a_k(\rho) = a_k( 2^k \rho)$, then $T_a = \sum T_{a_k}$.n

The proofs begin by establishing bounds of the form $\| T_{a_k} f \|_{L^p(\RR^d)} \lesssim \| f \|_{L^p(\RR^d)}$, uniformly in $k$. By applying a dilation symmetry, it suffices without loss of generality to look at $a_0$.




Now how 


Suppose $a: [0,\infty) \to \CC$ is compactly supported on the interval $[1/2,2]$. Then the radial function $k(x) = \int a(|\xi|) e^{2 \pi i \xi \cdot x}\; d\xi$ is the a convolution kernel for the radial multiplier operator $T_a$, i.e. $T_a f = k * f$ for all inputs $f$. The bounds on radial multipliers obtained in BLAH are then of the form $\| T_a f \|_{L^p(\RR^d)} \lesssim \| k \|_{L^p(\RR^d)} \| f \|_{L^p(\RR^d)}$. Dilation symmetry then immediately implies BLAH DYADIC RESULT, and then some methods of atomic decompositions and Littlewood Paley theory can be combined to obtain a tight result. On a sphere, we \emph{can} define the zonal convolution kernel $k$ corresponding to a zonal multiplier $Z_a$, such that
%
\[ (Z_a f)(x) = \int_{S^d} k( y \cdot x ) f(y)\; dy. \]
%
If $a$ is supported on $[1/2,2]$, then weighted $L^p$ bounds on $k$ can be used to imply bounds on $Z_a$, but these bounds will not scale, and it is difficult to determine how the bounds scale under dilations since there is no relation between the zonal convolution kernel $k$ corresponding to $a$, and the convolution kernel corresponding to the dilations of $a$.

A fix is obtained by writing $Z_a f$ using the \emph{cosine transform} of $a$, i.e. in terms of
%
\[ \widehat{a}(t) = \int_0^\infty a(\rho) e^{2 \pi i \rho t}\; d\rho. \]
%
The cosine transform \emph{does} scale under dilations, and functional calculus and the Fourier inversion formula allows us to write $Z_a$ in terms of $\widehat{a}$, i.e. by setting
%
\begin{equation} \label{InversionFormula}
	Z_a f = a(P) f = \int_{-\infty}^\infty \widehat{a}(t) e^{2 \pi i t P}\; dt,
\end{equation}
%
where $P = \sqrt{ \alpha^2 - \Delta }$, and $e^{2 \pi i t P} f = e^{2 \pi i t k} f$ for a spherical harmonic $f$ of degree $k$. The operators $u(t) = e^{2 \pi i t P} f$ give solutions to the half-wave equation $\partial_t u = P u$. We can understand the geometric behaviour of the half-wave equation by using the theory of \emph{Fourier integral operators}.

. Indeed, if $f$ is a spherical harmonic of degree $k$, then $Z_a f = a(k) f$ and by the Fourier inversion formula $\int \widehat{a}(t) e^{2 \pi i t P} f\; dt = \int \widehat{a}(t) e^{2 \pi i t k} f\; dt = a(k) f(t)$. For any function $f$, the functions $u(t) = e^{2 \pi i t P} f$ give a solution to the wave equation $\partial_t^2 u(t) = P u(t)$ with $u(0) = f$ and $\partial_t u(0) = 0$.


The bounds on radial multipliers obtained in BLAH depend on bounds on the convolution kernel corresponding to the multiplier. 


The main goal of my research project on multipliers is to understand


 deconstructive interference between a family of planar waves, or spherical harmonics of different degrees. Necessary and sufficient conditions for a Fourier multiplier operator to be bounded on $L^1(\RR^d)$ or $L^\infty(\RR^d)$ were quickly realized.

%Understanding the boundedness of Fourier multiplier operators in an $L^p$ norm for $p \neq 2$ underpins any subtle understanding of the Fourier transform. Plane waves oscillating in different directions and with different frequencies are orthogonal to one another, and thus do not interact with one another significantly in terms of the $L^2$ norm, as justified by Bessel's inequality. But plane waves can interact with one another in the $L^p$ norm for $p \neq 2$, and so understanding $L^p$ bounds for Fourier multipliers indicate when this interaction is significant or insignificant. Similarily, spherical harmonics of different degrees on $S^d$ are orthogonal to one another, but studying the $L^p$ bounds of multipliers of the Laplacian on the sphere is crucial to understand when the interactions of different spherical harmonics are significant or not.

%The general study of the $L^p$ boundedness of general Fourier multipliers began in the 1960s, brought on by the spur of applications the Calderon-Zygmund school and their contemporaries brought to the theory. Necessary and sufficient conditions for a Fourier multiplier to be bounded on $L^2(\RR^d)$ follow by simple orthogonality, and conditions to be bounded on $L^1(\RR^d)$ and $L^\infty(\RR^d)$ also follow simply, because such spaces do not tend to allow much capacity for subtle cancellation. But the problem of finding necessary and sufficient conditions for boundedness in $L^p(\RR^d)$ for any other exponent proved impenetrable. Surprisingly, in the past decade necessary and sufficient conditions for $L^p$ boundedness have occured for \emph{radial} Fourier multipliers.

%Orthogonality immediately implies that a Fourier multiplier $T$ is bounded on $L^2(\RR^d)$ if and only if it's symbol $m$ lies in $L^\infty(\RR^d)$. Similarily, since spherical harmonics of different degrees are orthogonal to one another, a multiplier for spherical harmonic expansions is bounded on $L^2(S^d)$ if and only if $m \in L^\infty(\NN)$. The boundedness of such multipliers for $p \neq 2$ is a more subtle property, but underpins any deep understanding of the Fourier transform or the understanding of the interactions. Indeed, studying the boundedness of multipliers seems to be one of the few tractable ways of quantifying deconstructive interference between a family of planar waves, or spherical harmonics of different degrees. Necessary and sufficient conditions for a Fourier multiplier operator to be bounded on $L^1(\RR^d)$ or $L^\infty(\RR^d)$ were quickly realized.
%Spherical harmonics of different degrees are orthogonal to one another, and this immediately implies $T$ is bounded on $L^2(S^d)$ if and only if $m \in L^\infty(\NN)$. But characterizations of $L^p$ boundedness for all $p \neq 2$ remain unknown.


% It was quickly realized that a Fourier multiplier operator is bounded on $L^1(\RR^d)$ or $L^\infty(\RR^d)$ if and only if the Fourier transform of the symbol $m$ lies in $L^1(\RR^d)$.
% Indeed, many interesting problems about the boundedness of \emph{specific} Fourier multipliers, such as the Bochner-Riesz conjecture, remain largely unsolved today.

%It thus came as a recent surprise when necessary and sufficient conditions were found for bounding \emph{radial} Fourier multipliers (multipliers with a radial symbol) on $L^p(\RR^d)$. First came the result of \cite{GarrigosSeeger}, who found a necessary and sufficient condition in the range $|1/p - 1/2| > 1/2d$ for bounds of the form $\| Tf \|_{L^p(\RR^d)} \lesssim \| f \|_{L^p(\RR^d)}$ to hold uniformly over \emph{radial functions} $f$.
%If $T$ has symbol $m(|\xi|)$, then the condition says that $T$ is bounded if and only if the Fourier transforms of the functions $m_k(\xi) = m(2^k \xi) \chi(|\xi|)$ are uniformly bounded in $L^p(\RR^d)$, where $\chi \in C_c^\infty(0,\infty)$ and $\sum_{k \in \ZZ} \chi(2^k \xi) = 1$.
%An optimist might think this same condition causes the bound to hold uniformly over \emph{all functions} $f$ in the range above, a statement we call the \emph{radial multiplier conjecture}. We now know, by the results of \cite{HeoNazarovSeeger} and \cite{Cladek}, that the radial multiplier conjecture holds when $d > 4$ and $|1/p - 1/2| > 1/(d-1)$, when $d = 4$ and $|1/p - 1/2| > 11/36$, and when $d = 3$ and $|1/p - 1/2| > 11/26$. But the radial multiplier conjecture is not completely resolved for any $d$, and no bounds are known at all when $d = 2$.

%My main research project considers the study of the relation between the $L^p$ boundedness of two types of multipliers:
%
%\begin{itemize}
%	\item \emph{Fourier multipliers} are those operators $T$ on $\RR^d$ for which we can associate a function $m: \RR^d \to \CC$, known as the \emph{symbol} of $T$, such that for any function $f$, the Fourier transform of $Tf$ obeys the relation $\widehat{Tf} = m \widehat{f}$. Heuristically, this means that $T e^{2 \pi i \xi \cdot x} = m(\xi) e^{2 \pi i \xi \cdot x}$ for each $\xi \in \RR^d$.
%\end{itemize}

%  Every translation invariant operator on $\RR^d$ is a Fourier multiplier operator, and every rotation invariant operator on $S^d$ is a multiplier of the spherical harmonic expansion on $S^d$.

\section{Pattern Avoidance}

How large must a set be before it must contain a certain point configuration? Problems of this flavor have long been studied in various areas of combinatorics. In the last 50 years, analysts have also begun studying analogous problems for infinite subsets $X \subset \RR^d$, where the size of $X$ is measured in terms of a suitable \emph{fractal dimension}, often \emph{Hausdorff dimension}, but also sometimes \emph{Fourier dimension}, the latter of which tending to imply more structure than the former.

Several definite conjectures on problems about the \emph{density} of certain point configurations in sets have been raised, such as the Falconer distance problem. %, which asks if an arbitrary subset $X$ of $[0,1]^d$ with \emph{Hausdorff dimension} exceeding $d/2$, then the set of all distances between pairs of points in $X$ must form a set of positive Lebesgue measure.
But there are relatively few definite conjectures about the dimension a set requires before it must contain \emph{at least one} family of points fitting a certain kind of configuration. For instance, we do not know for $d > 2$ how large the Hausdorff dimension a set $X \subset \RR^d$ must be before it contains all three vertices of an isosceles triangle, the threshold being somewhere between $d/2$ and $d-1$.

 It is not clear

, for instance, how large the Hausdorff dimension a set $X \subset \RR^d$ must have before it contains the vertices of at least one isosceles triangle, or, for a particular angle $\theta \in [0,\pi]$, how large $X$ must be before it contains three points $A$, $B$, and $C$ which when connected form an angle $\theta$; the only case here that is fully resolved is when $\theta \in \{ 0, \pi \}$, or when $\theta = \pi/2$ and $d$ is even: when $\theta = 0$ and $\theta = \pi$, the threshold is $d-1$, when $\theta = \pi/2$, the threshold is somewhere between $d/2$ and $\lceil d/2 \rceil$, for rational $\theta$ the threshold is somewhere between $d/4$ and $d-1$, and when $\theta$ is irrational the threshold is somewhere between $d/8$ and $d-1$.

Until recently, certain results \cite{LabaPramanik} seemed to indicate that subsets of $[0,1]$ of Fourier dimension one must necessarily contain an arithmetic progression of length three, but this has proved not to be the case \cite{Schmerkin}.

The ability to form definite conjectures depends on the ability to produce counterexamples for certain problems. In this case, counterexamples take the form of constructing sets with large fractal dimension that \emph{do not} contain certain point configurations. My research in geometric measure theory has so far focused on this type of problem.

During my MSc, my advisors and I found a construction that produces sets $X$ with large Hausdorff dimension that avoid a particular configuration, given that the particular configuration is 'small' \cite{DensonPramanikZahl}. More precisely, let us suppose we are looking at configurations of $k$ points in $\RR^d$. The set of all tuples of points that fit a given configuration can be identified with a subset $C$ of $(\RR^d)^k$.

Provided that the Minkowski dimension of $C$ is at most $\beta$, we constructed a set $X$ with Hausdorff dimension $(dk - \beta)/(k-1)$ such that if $x_1,\dots,x_k$ are distinct points in $X$, then $(x_1,\dots,x_k) \not \in C$. In particular, for a Lipschitz function $f: (\RR^d)^k \to \RR^d$, we construct a set $X$ with Hausdorff dimension $d/k$ such that for distinct $x_0,\dots,x_k \in X$, $x_0 \neq f(x_1,\dots,x_k)$, recovering the main result of \cite{FraserPramanik}.



During my PhD, I decided to investigate whether 

. My research in geometric measure theory so far has been on trying to produce such counterexamples. In BLAH, Pramanik and Fraser. In BLAH, I rephrased their argument in probabilistic terms 

 of a set must be before the set of all distances 

Several definite conjectures on problems of this kind have been established since the project begun, such as the Falconer distance problem or Kakeya conjecture, where the point configuration in mind are points lying at a certain distance from one another, or line segments pointing in other directions. For other

The natural fractal dimension used to measure the size of a set $X$ is often the Hausdorff dimension $\dim_{\mathbb{H}}(X)$ of $X$. But sometimes the \emph{Fourier dimension} $\dim_{\mathbb{F}}(X)$ proves useful, which measures the best possible decay that the Fourier transform of measures supported on $X$ can have; if $\alpha < \dim_{\mathbb{F}}(X)$, then there exists a nonzero measure $\mu$ on $X$ such that $|\widehat{\mu}(\xi)| \lesssim |\xi|^{-\alpha}$ for all $\xi \in \RR^d$. The Fourier dimension thus, morally speaking, measures how uncorrelated the set $X$ is with the Fourier characters $e_\xi(x) = e^{2 \pi i \xi \cdot x}$.

\section{Future Lines of Research}

The work I have conducted naturally suggests several {\bf future problems}.
%
\begin{itemize}
	\item Analyzing the 'return time operator' to extend results on expansions of spherical harmonics to the study of the Laplace-Beltrami operator on $S^d$.

	\item Determining whether our methods extend to other manifolds whose geodesic flow is simpler to understand, such as integrable systems.

	\item Analyzing whether local smoothing bounds

	\item Constructing Random Salem Sets which satisfy a Decoupling Bound.

	\item Determining the relation between certain 'fractal weighted estimates' for the wave equation on $\RR^d$ and the 'density decomposition' of multiplier bounds.
\end{itemize}

In fact, this resemblance opens up a whole new world of families of operators. Given an arbitrary elliptic self-adjoint first order classical pseudo-differential operator $P$

This method is highly robust and depends very little that we are working on the sphere; pretty much the only property we end up using is that the wave equation $\partial_t u = P u$ has \emph{periodic solutions}.

The natural analogue of the study of radial multipliers on $\RR^d$ is the study of multipliers of a Laplace-Beltrami operator on a Riemannian manifold. The natural analogue of the study of quasiradial multipliers on $\RR^d$ is the study of multipliers of an operator associated with a \emph{Finsler geometry} on the manifold.

\begin{thebibliography}{03}

\bibitem{Cladek} Cladek, Laura,
	\emph{Radial {F}ourier Multipliers in $\RR^3$ and $\RR^4$}.

\bibitem{DensonCharacterization} Denson, Jacob,
	\emph{Multipliers of Spherical Harmonics}.

\bibitem{DensonPramanikZahl} Denson, Jacob,
 	\emph{Large Sets Avoiding Rough Patterns}.

\bibitem{DensonFourier} Denson, Jacob,
	\emph{Large Salem Sets Avoiding Nonlinear Configurations}.

\bibitem{LabaPramanik} Laba, Izabella and Pramanik, Malabika,
	\emph{Arithmetic progressions in sets of fractional dimension}.

\bibitem{Schmerkin} Schmerkin, Pablo,
	\emph{{S}alem sets with no arithmetic progressions}.

\bibitem{FraserPramanik} Fraser, Robert and Pramanik, Malabika,
	\emph{Large Sets Avoiding Patterns}.

\bibitem{GarrigosSeeger} Garrigos, Gustavo and Seeger, Andreas,
	\emph{Characterizations of {H}ankel Multipliers}.

\bibitem{HeoNazarovSeeger} Heo, Yaryong and Nazarov, Fedor and Seeger, Andreas,
	\emph{Radial {F}ourier Multipliers in High Dimensions}.

\bibitem{KimQuasiradial} Kim, Jongchon,
	\emph{Endpoint Bounds for Quasiradial {F}ourier Multipliers}.

\bibitem{Cladek} Cladek, Laura,
	\emph{Radial {F}ourier Multipliers in $\mathbb{R}^3$}.

\bibitem{Mitjagin} Mitjagin, Boris S.,
	\emph{Divergenz von spektralentwicklungen in $L_p$-r\"{a}umen}.

\end{thebibliography}

\end{document}

Such operators initially arose from the study of the classical partial differential equations in physics, and continue to have applications in areas as diverse as partial differential equations, mathematical physics, number theory, and ergodic theory. Every translation invariant operator on $\RR^d$ is a Fourier multiplier operator, and every rotation invariant operator on $S^d$ is a multiplier of the spherical harmonic expansion on $S^d$.

any such operator $T$, we can associate a function $m: \RR^n \to \CC$, known as the \emph{symbol} of $T$, such that for any function $f$, the Fourier transform of $Tf$ obeys the relation $\widehat{Tf} = m \widehat{f}$; thus translation invariant operators are also called \emph{Fourier multiplier operators}.

Understanding the boundedness of Fourier multiplier operators in an $L^p$ norm for $p \neq 2$ underpins any subtle understanding of the Fourier transform. Plane waves oscillating in different directions and with different frequencies are orthogonal to one another, and thus do not interact with one another significantly in terms of the $L^2$ norm, as justified by Bessel's inequality. But plane waves can interact with one another in the $L^p$ norm for $p \neq 2$, and so understanding $L^p$ bounds for Fourier multipliers indicate when this interaction is significant or insignificant. Similarily, spherical harmonics of different degrees on $S^d$ are orthogonal to one another, but studying the $L^p$ bounds of multipliers of the Laplacian on the sphere is crucial to understand when the interactions of different spherical harmonics are significant or not.

The general study of the characterizations of $L^p$ boundedness for the Fourier multipliers was initiated in the 1960s. Mathematicians quickly found simple necessary and sufficient conditions that ensure Fourier multipliers are bounded on $L^1(\RR^d)$, $L^2(\RR^d)$, and $L^\infty(\RR^d)$. But the problem of finding necessary and sufficient conditions for boundedness in $L^p(\RR^d)$ for any other exponent proved impenetrable. Indeed, many interesting problems about the boundedness of \emph{specific} Fourier multipliers, such as the Bochner-Riesz conjecture, remain largely unsolved today.

Thus it came as a surprise in the past decade when results emerged proving necessary and sufficient conditions for \emph{radial} Fourier multipliers to be bounded on $L^p(\RR^d)$. First came the result of BLAH, which gave a necessary and sufficient criteria for bounds of the form $\| Tf \|_{L^p(\RR^d)} \lesssim \| f \|_{L^p(\RR^d)}$ to hold uniformly over \emph{radial functions} $f$, for $|1/p - 1/2| > 1/2d$. An optimist might think this same condition causes the bound to hold uniformly over \emph{all functions} $f$ in the range above, a statement we call the \emph{radial multiplier conjecture}. We now know, by the results of BLAH and BLAH, that the radial multiplier conjecture holds when $d > 4$ and $|1/p - 1/2| > 1/(d-1)$, when $d = 4$ and $|1/p - 1/2| > 11/36$, and when $d = 3$ and $|1/p - 1/2| > 11/26$. But the radial multiplier conjecture has not yet been completely solved in any dimension $d$, and no bounds are known at all when $d = 2$.

The natural analogue of the study of radial multipliers on $\RR^d$ is the study of multipliers of a Laplace-Beltrami operator on a Riemannian manifold. The natural analogue of the study of quasiradial multipliers on $\RR^d$ is the study of multipliers of an operator associated with a \emph{Finsler geometry} on the manifold.