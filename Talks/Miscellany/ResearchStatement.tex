\documentclass[12pt]{article}

\usepackage[a4paper, margin=1in]{geometry}

\usepackage{amssymb,amsmath,amsthm}

\DeclareMathOperator{\QQ}{\mathbb{Q}}
\DeclareMathOperator{\ZZ}{\mathbb{Z}}
\DeclareMathOperator{\RR}{\mathbb{R}}
\DeclareMathOperator{\NN}{\mathbb{N}}
\DeclareMathOperator{\HH}{\mathbb{H}}
\DeclareMathOperator{\BB}{\mathbb{B}}
\DeclareMathOperator{\CC}{\mathbb{C}}
\DeclareMathOperator{\AB}{\mathbb{A}}
\DeclareMathOperator{\PP}{\mathbb{P}}
\DeclareMathOperator{\MM}{\mathbb{M}}
\DeclareMathOperator{\VV}{\mathbb{V}}
\DeclareMathOperator{\TT}{\mathbb{T}}
\DeclareMathOperator{\LL}{\mathcal{L}}
\DeclareMathOperator{\DD}{\mathcal{D}}
\DeclareMathOperator{\SW}{\mathcal{S}}
\DeclareMathOperator{\EC}{\mathcal{E}}
\DeclareMathOperator{\AC}{\mathcal{A}}

\title{Research Statement}
\author{Jacob Denson}
\date{}

\begin{document}

\maketitle

{\bf My Research} focuses on problems in Harmonic analysis. In particular, I study Fourier multiplier operators on Euclidean space, and their analogues on compact manifolds, through an understanding of the behaviour of the wave equation on such spaces. I also work on problems in geometric measure theory, investigating when `structure' occurs in fractals of large dimension. Both projects lead to interesting questions that I plan to pursue in my postgraduate work.

During my PhD, much of my work on multipliers has concentrating on relating bounds on Fourier multiplier operators on $\RR^d$ to bounds for analogous operators on compact manifolds. In \cite{DensonCharacterization}, I proved a \emph{transference principle} for zonal multipliers operators on the sphere $S^d$. For certain exponents $p$ for $d \geq 4$, this principle shows that $L^p$ bounds for a radial Fourier multiplier with symbol $m(|\xi|)$ imply bounds for a zonal multiplier operators on $S^d$ induced by the same symbol. In the process, for the same range of exponents, I completely characterized those symbols $m$ whose dilates give a uniformly bounded family of zonal multiplier operators on $L^p(S^d)$. This is the first such characterization on $S^d$ for any $p \neq 2$, and no other such characterization, or transference principle of the form above, has been proved for analogous operators on any other compact manifold.

My MSc work mainly focused on geometric measure theory, where $I$, together with my Master's thesis advisors Malabika Pramanik and Josh Zahl, obtained a rather general method \cite{DensonPramanikZahl} for constructing sets of large \emph{Hausdorff dimension} which avoid `low dimensional geometric configurations'. During my PhD, I continued this line of research by establishing several probabilistic extensions of the methods of the previous paper \cite{DensonFourier} to address the more difficult problem of constructing sets of large \emph{Fourier dimension} avoiding configurations, where as I was able to fully recover the Hausdorff dimension obtained in \cite{DensonPramanikZahl} in the Fourier dimension setting by assuming a weak linearity assumption. An example application of the method is 

% |x - y| = |y - z|
% (x - y)^2 = (y - z)^2
% x^2 - 2xy + y^2 = y^2 - 2yz + z^2
% x^2 - 2xy + 2yz - z^2 = 0
% x^2 + 2y(z - x) - z^2 = 0
%
% It's *linear in y*,
% i.e. it's a system of linear equations in y with 
%
% F is a subset of R of dimension

My work in geometric measure theory focuses on whether having large 'Fourier dimension' guarantees the presence of certain geometric configurations in sets. I established a general probabilistic method of constructing sets with large Fourier dimension which avoid a variety of different 'nonlinear' geometric configurations. For instance, the method can construct subsets of $\RR^2$ of Fourier dimension $0.8$ which do not contain three points forming the vertices of an isosceles triangle. This method remains the only construction method in the literature for sets with large Fourier dimension avoiding nonlinear configurations, and also remains the best currently known construction method for constructing sets avoiding solutions to a general linear equation when $d > 1$.

{\bf In The Near Future}, I hope to generalize my bounds for multipliers of spherical harmonic expansions to the more general setting of multipliers for eigenfunction expansions of the Laplace-Beltrami operator on a Riemannian manifold $M$ with periodic geodesic flow. A local obstruction to this generalization requires obtaining control of a pseudodifferential operator on $M$ called the \emph{return operator}, and a global obstruction requires an endpoint refinement of the local smoothing inequality for the wave equation on $M$ to handle combining dyadic scales together. I am also interested in exploring what kinds of bounds for multipliers can be obtained via the wave equation on manifolds whose geodesic flow is \emph{integrable}. In the study of patterns, I hope to apply the tools I exploited to construct sets of large fractal dimension to the study of decoupling on random fractals. And I am interested in determining the interrelation of my, in particular studying distance sets using local smoothing bounds, and exploring analogues of Fourier dimension on Riemannian manifolds.

\pagebreak[3]






\section*{Fourier Multiplier Operators}

Much of my PhD has dealt with the study of bounds for radial Fourier multipliers on $\RR^d$, bounds for multipliers of eigenfunction expansions on compact manifolds, and the relation between such bounds.

Multipliers have been central to harmonic analysis from it's inception. Fourier showed that solutions to the classical equations of physics can be described by \emph{radial Fourier multipliers}, operators $T_m$ defined from a function $m: [0,\infty) \to \CC$, the \emph{symbol}, by setting
%
\[ T_mf(x) = \int_{\RR^d} m(|\xi|) \widehat{f}(\xi) e^{2 \pi i \xi \cdot x}\; dx. \]
%
We now know all translation-invariant operators on $\RR^d$ are Fourier multiplier operators, explaining their broad applicability, and the theory continues to have applications in areas as diverse as partial differential equations, mathematical physics, number theory, complex variables, and ergodic theory.

A similar theory of multiplier operators can be developed on the sphere $S^d$ with similar analogues to the theory of Fourier multipliers. The analogues of the planar waves $\{ e^{2 \pi i \xi \cdot x} : \xi \in \RR^d \}$ are the \emph{spherical harmonics}, i.e. homogeneous harmonic polynomials on $\RR^{d+1}$ restricted to the unit sphere. Every function $f \in L^2(S^d)$ can be uniquely expanded as a sum $\sum_{k = 0}^\infty f_k$, where $f_k$ is a spherical harmonic of degree $k$. A \emph{zonal multiplier} is an operator $Z_m$ on $S^d$ defined in terms of a function $m: \NN \to \CC$, the symbol of $T$, such that
%
\[ Z_m f = \sum\nolimits_{k = 0}^\infty m(k) f_k. \]
%
Just as Fourier multipliers characterize all translation-invariant operators, zonal multipliers characterize all operators on the sphere which are \emph{rotation invariant}, and so such operators arise in a diverse area of applications, including celestial mechanics and the analysis of angular momentum in quantum physics.

The general study of the $L^p$ boundedness of general Fourier multipliers began in the 1960s, brought on by the spur of applications the Calderon-Zygmund school and their contemporaries brought to the theory. Necessary and sufficient conditions for a Fourier multiplier to be bounded on $L^2(\RR^d)$ follow by simple orthogonality, and conditions to be bounded on $L^1(\RR^d)$ and $L^\infty(\RR^d)$ also follow simply, because such spaces do not tend to allow much capacity for subtle cancellation. But the problem of finding necessary and sufficient conditions for boundedness in $L^p(\RR^d)$ for any other exponent proved impenetrable. Surprisingly, in the past decade necessary and sufficient conditions for $L^p$ boundedness have occured for \emph{radial} Fourier multipliers.

%Orthogonality immediately implies that a Fourier multiplier $T$ is bounded on $L^2(\RR^d)$ if and only if it's symbol $m$ lies in $L^\infty(\RR^d)$. Similarily, since spherical harmonics of different degrees are orthogonal to one another, a multiplier for spherical harmonic expansions is bounded on $L^2(S^d)$ if and only if $m \in L^\infty(\NN)$. The boundedness of such multipliers for $p \neq 2$ is a more subtle property, but underpins any deep understanding of the Fourier transform or the understanding of the interactions. Indeed, studying the boundedness of multipliers seems to be one of the few tractable ways of quantifying deconstructive interference between a family of planar waves, or spherical harmonics of different degrees. Necessary and sufficient conditions for a Fourier multiplier operator to be bounded on $L^1(\RR^d)$ or $L^\infty(\RR^d)$ were quickly realized.
%Spherical harmonics of different degrees are orthogonal to one another, and this immediately implies $T$ is bounded on $L^2(S^d)$ if and only if $m \in L^\infty(\NN)$. But characterizations of $L^p$ boundedness for all $p \neq 2$ remain unknown.


% It was quickly realized that a Fourier multiplier operator is bounded on $L^1(\RR^d)$ or $L^\infty(\RR^d)$ if and only if the Fourier transform of the symbol $m$ lies in $L^1(\RR^d)$.
 Indeed, many interesting problems about the boundedness of \emph{specific} Fourier multipliers, such as the Bochner-Riesz conjecture, remain largely unsolved today.

It thus came as a recent surprise when necessary and sufficient conditions were found for bounding \emph{radial} Fourier multipliers (multipliers with a radial symbol) on $L^p(\RR^d)$. First came the result of \cite{GarrigosSeeger}, who found a necessary and sufficient condition in the range $|1/p - 1/2| > 1/2d$ for bounds of the form $\| Tf \|_{L^p(\RR^d)} \lesssim \| f \|_{L^p(\RR^d)}$ to hold uniformly over \emph{radial functions} $f$.
%If $T$ has symbol $m(|\xi|)$, then the condition says that $T$ is bounded if and only if the Fourier transforms of the functions $m_k(\xi) = m(2^k \xi) \chi(|\xi|)$ are uniformly bounded in $L^p(\RR^d)$, where $\chi \in C_c^\infty(0,\infty)$ and $\sum_{k \in \ZZ} \chi(2^k \xi) = 1$.
An optimist might think this same condition causes the bound to hold uniformly over \emph{all functions} $f$ in the range above, a statement we call the \emph{radial multiplier conjecture}. We now know, by the results of \cite{HeoNazarovSeeger} and \cite{Cladek}, that the radial multiplier conjecture holds when $d > 4$ and $|1/p - 1/2| > 1/(d-1)$, when $d = 4$ and $|1/p - 1/2| > 11/36$, and when $d = 3$ and $|1/p - 1/2| > 11/26$. But the radial multiplier conjecture is not completely resolved for any $d$, and no bounds are known at all when $d = 2$.

Several analogies e


Just as Fourier multipliers characterize all translation invariant operators, this class of operators characterizes all rotation invariant operators. 
s



 is an operator $T$ on $S^d$ for which there exists $m: \NN \to \CC$, the \emph{symbol} of $T$, such that $Tf = m(k) f$ for each $f \in \mathcal{H}_k(S^d)$.

Understanding the boundedness of Fourier multiplier operators in an $L^p$ norm for $p \neq 2$ underpins any subtle understanding of the Fourier transform. Plane waves oscillating in different directions and with different frequencies are orthogonal to one another, and thus do not interact with one another significantly in terms of the $L^2$ norm, as justified by Bessel's inequality. But plane waves can interact with one another in the $L^p$ norm for $p \neq 2$, and so understanding $L^p$ bounds for Fourier multipliers indicate when this interaction is significant or insignificant. Similarily, spherical harmonics of different degrees on $S^d$ are orthogonal to one another, but studying the $L^p$ bounds of multipliers of the Laplacian on the sphere is crucial to understand when the interactions of different spherical harmonics are significant or not.


$u(t,x) = e^{2 \pi i t \sqrt{-\Delta}} u_0$, where

First the \emph{Fourier multipliers} were studied, operators $T$ on $\RR^d$ defined by an expression of the form
%
\[ Tf(x) = \int_{\RR^d} m(\xi) \widehat{f}(\xi) e^{2 \pi i \xi \cdot x}\; d\xi, \]
%
where $\widehat{f}$ is the Fourier transform of the function $f$.

My main research project considers the study of the relation between the $L^p$ boundedness of two types of multipliers:
%
\begin{itemize}
	\item \emph{Fourier multipliers} are those operators $T$ on $\RR^d$ for which we can associate a function $m: \RR^d \to \CC$, known as the \emph{symbol} of $T$, such that for any function $f$, the Fourier transform of $Tf$ obeys the relation $\widehat{Tf} = m \widehat{f}$. Heuristically, this means that $T e^{2 \pi i \xi \cdot x} = m(\xi) e^{2 \pi i \xi \cdot x}$ for each $\xi \in \RR^d$.
\end{itemize}

  Every translation invariant operator on $\RR^d$ is a Fourier multiplier operator, and every rotation invariant operator on $S^d$ is a multiplier of the spherical harmonic expansion on $S^d$.



The natural analogue of the study of radial multipliers on $\RR^d$ is the study of multipliers of a Laplace-Beltrami operator on a Riemannian manifold. The natural analogue of the study of quasiradial multipliers on $\RR^d$ is the study of multipliers of an operator associated with a \emph{Finsler geometry} on the manifold.

\section*{Pattern Avoidance}

How large must a set be before it must contain a certain point configuration? Problems of this flavor have long been studied in various areas of combinatorics. In the last 50 years, analysts have also begun studying analogous problems for infinite subsets $X \subset \RR^d$, where the size of $X$ is measured in terms of a suitable \emph{fractal dimension}, often \emph{Hausdorff dimension}, but also sometimes \emph{Fourier dimension}, the latter of which tending to imply more structure than the former.

Several definite conjectures on problems about the \emph{density} of certain point configurations in sets have been raised, such as the Falconer distance problem. %, which asks if an arbitrary subset $X$ of $[0,1]^d$ with \emph{Hausdorff dimension} exceeding $d/2$, then the set of all distances between pairs of points in $X$ must form a set of positive Lebesgue measure.
But there are relatively few definite conjectures about the dimension a set requires before it must contain \emph{at least one} family of points fitting a certain kind of configuration. For instance, we do not know for $d > 2$ how large the Hausdorff dimension a set $X \subset \RR^d$ must be before it contains all three vertices of an isosceles triangle, the threshold being somewhere between $d/2$ and $d-1$.

 It is not clear

, for instance, how large the Hausdorff dimension a set $X \subset \RR^d$ must have before it contains the vertices of at least one isosceles triangle, or, for a particular angle $\theta \in [0,\pi]$, how large $X$ must be before it contains three points $A$, $B$, and $C$ which when connected form an angle $\theta$; the only case here that is fully resolved is when $\theta \in \{ 0, \pi \}$, or when $\theta = \pi/2$ and $d$ is even: when $\theta = 0$ and $\theta = \pi$, the threshold is $d-1$, when $\theta = \pi/2$, the threshold is somewhere between $d/2$ and $\lceil d/2 \rceil$, for rational $\theta$ the threshold is somewhere between $d/4$ and $d-1$, and when $\theta$ is irrational the threshold is somewhere between $d/8$ and $d-1$.

Until recently, certain results \cite{LabaPramanik} seemed to indicate that subsets of $[0,1]$ of Fourier dimension one must necessarily contain an arithmetic progression of length three, but this has proved not to be the case \cite{Schmerkin}.

The ability to form definite conjectures depends on the ability to produce counterexamples for certain problems. In this case, counterexamples take the form of constructing sets with large fractal dimension that \emph{do not} contain certain point configurations. My research in geometric measure theory has so far focused on this type of problem.

During my MSc, my advisors and I found a construction that produces sets $X$ with large Hausdorff dimension that avoid a particular configuration, given that the particular configuration is 'small' \cite{DensonPramanikZahl}. More precisely, let us suppose we are looking at configurations of $k$ points in $\RR^d$. The set of all tuples of points that fit a given configuration can be identified with a subset $C$ of $(\RR^d)^k$.

Provided that the Minkowski dimension of $C$ is at most $\beta$, we constructed a set $X$ with Hausdorff dimension $(dk - \beta)/(k-1)$ such that if $x_1,\dots,x_k$ are distinct points in $X$, then $(x_1,\dots,x_k) \not \in C$. In particular, for a Lipschitz function $f: (\RR^d)^k \to \RR^d$, we construct a set $X$ with Hausdorff dimension $d/k$ such that for distinct $x_0,\dots,x_k \in X$, $x_0 \neq f(x_1,\dots,x_k)$, recovering the main result of \cite{FraserPramanik}.



During my PhD, I decided to investigate whether 

. My research in geometric measure theory so far has been on trying to produce such counterexamples. In BLAH, Pramanik and Fraser. In BLAH, I rephrased their argument in probabilistic terms 

 of a set must be before the set of all distances 

Several definite conjectures on problems of this kind have been established since the project begun, such as the Falconer distance problem or Kakeya conjecture, where the point configuration in mind are points lying at a certain distance from one another, or line segments pointing in other directions. For other

The natural fractal dimension used to measure the size of a set $X$ is often the Hausdorff dimension $\dim_{\mathbb{H}}(X)$ of $X$. But sometimes the \emph{Fourier dimension} $\dim_{\mathbb{F}}(X)$ proves useful, which measures the best possible decay that the Fourier transform of measures supported on $X$ can have; if $\alpha < \dim_{\mathbb{F}}(X)$, then there exists a nonzero measure $\mu$ on $X$ such that $|\widehat{\mu}(\xi)| \lesssim |\xi|^{-\alpha}$ for all $\xi \in \RR^d$. The Fourier dimension thus, morally speaking, measures how uncorrelated the set $X$ is with the Fourier characters $e_\xi(x) = e^{2 \pi i \xi \cdot x}$.

\section*{Future Lines of Research}

The work I have conducted naturally suggests several {\bf future problems}.
%
\begin{itemize}
	\item Analyzing the 'return time operator' to extend results on expansions of spherical harmonics to the study of the Laplace-Beltrami operator on $S^d$.

	\item Determining whether our methods extend to other manifolds whose geodesic flow is simpler to understand, such as integrable systems.

	\item Analyzing whether local smoothing bounds

	\item Constructing Random Salem Sets which satisfy a Decoupling Bound.

	\item Determining the relation between certain 'fractal weighted estimates' for the wave equation on $\RR^d$ and the 'density decomposition' of multiplier bounds.
\end{itemize}

\begin{thebibliography}{03}

\bibitem{GarrigosSeeger} Garrigos, Gustavo and Seeger, Andreas,
	\emph{Characterizations of {H}ankel Multipliers}.

\bibitem{HeoNazarovSeeger} Heo, Yaryong and Nazarov, Fedor and Seeger, Andreas,
	\emph{Radial {F}ourier Multipliers in High Dimensions}.

\bibitem{Cladek} Cladek, Laura,
	\emph{Radial {F}ourier Multipliers in $\RR^3$ and $\RR^4$}.

\bibitem{DensonCharacterization} Denson, Jacob,
	\emph{Multipliers of Spherical Harmonics}.

\bibitem{DensonPramanikZahl} Denson, Jacob,
 	\emph{Large Sets Avoiding Rough Patterns}.

\bibitem{DensonFourier} Denson, Jacob,
	\emph{Large Salem Sets Avoiding Nonlinear Configurations}.

\bibitem{LabaPramanik} Laba, Izabella and Pramanik, Malabika,
	\emph{Arithmetic progressions in sets of fractional dimension}.

\bibitem{Schmerkin} Schmerkin, Pablo,
	\emph{{S}alem sets with no arithmetic progressions}.

\bibitem{FraserPramanik} Fraser, Robert and Pramanik, Malabika,
	\emph{Large Sets Avoiding Patterns}.

\end{thebibliography}

\end{document}

Such operators initially arose from the study of the classical partial differential equations in physics, and continue to have applications in areas as diverse as partial differential equations, mathematical physics, number theory, and ergodic theory. Every translation invariant operator on $\RR^d$ is a Fourier multiplier operator, and every rotation invariant operator on $S^d$ is a multiplier of the spherical harmonic expansion on $S^d$.

any such operator $T$, we can associate a function $m: \RR^n \to \CC$, known as the \emph{symbol} of $T$, such that for any function $f$, the Fourier transform of $Tf$ obeys the relation $\widehat{Tf} = m \widehat{f}$; thus translation invariant operators are also called \emph{Fourier multiplier operators}.

Understanding the boundedness of Fourier multiplier operators in an $L^p$ norm for $p \neq 2$ underpins any subtle understanding of the Fourier transform. Plane waves oscillating in different directions and with different frequencies are orthogonal to one another, and thus do not interact with one another significantly in terms of the $L^2$ norm, as justified by Bessel's inequality. But plane waves can interact with one another in the $L^p$ norm for $p \neq 2$, and so understanding $L^p$ bounds for Fourier multipliers indicate when this interaction is significant or insignificant. Similarily, spherical harmonics of different degrees on $S^d$ are orthogonal to one another, but studying the $L^p$ bounds of multipliers of the Laplacian on the sphere is crucial to understand when the interactions of different spherical harmonics are significant or not.

The general study of the characterizations of $L^p$ boundedness for the Fourier multipliers was initiated in the 1960s. Mathematicians quickly found simple necessary and sufficient conditions that ensure Fourier multipliers are bounded on $L^1(\RR^d)$, $L^2(\RR^d)$, and $L^\infty(\RR^d)$. But the problem of finding necessary and sufficient conditions for boundedness in $L^p(\RR^d)$ for any other exponent proved impenetrable. Indeed, many interesting problems about the boundedness of \emph{specific} Fourier multipliers, such as the Bochner-Riesz conjecture, remain largely unsolved today.

Thus it came as a surprise in the past decade when results emerged proving necessary and sufficient conditions for \emph{radial} Fourier multipliers to be bounded on $L^p(\RR^d)$. First came the result of BLAH, which gave a necessary and sufficient criteria for bounds of the form $\| Tf \|_{L^p(\RR^d)} \lesssim \| f \|_{L^p(\RR^d)}$ to hold uniformly over \emph{radial functions} $f$, for $|1/p - 1/2| > 1/2d$. An optimist might think this same condition causes the bound to hold uniformly over \emph{all functions} $f$ in the range above, a statement we call the \emph{radial multiplier conjecture}. We now know, by the results of BLAH and BLAH, that the radial multiplier conjecture holds when $d > 4$ and $|1/p - 1/2| > 1/(d-1)$, when $d = 4$ and $|1/p - 1/2| > 11/36$, and when $d = 3$ and $|1/p - 1/2| > 11/26$. But the radial multiplier conjecture has not yet been completely solved in any dimension $d$, and no bounds are known at all when $d = 2$.

The natural analogue of the study of radial multipliers on $\RR^d$ is the study of multipliers of a Laplace-Beltrami operator on a Riemannian manifold. The natural analogue of the study of quasiradial multipliers on $\RR^d$ is the study of multipliers of an operator associated with a \emph{Finsler geometry} on the manifold.