\documentclass[12pt]{article}

\usepackage[a4paper, margin=1in]{geometry}

\usepackage{amsmath}
\usepackage{amssymb}

\DeclareMathOperator{\QQ}{\mathbb{Q}}
\DeclareMathOperator{\ZZ}{\mathbb{Z}}
\DeclareMathOperator{\RR}{\mathbb{R}}
\DeclareMathOperator{\NN}{\mathbb{N}}
\DeclareMathOperator{\HH}{\mathbb{H}}
\DeclareMathOperator{\BB}{\mathbb{B}}
\DeclareMathOperator{\CC}{\mathbb{C}}
\DeclareMathOperator{\AB}{\mathbb{A}}
\DeclareMathOperator{\PP}{\mathbb{P}}
\DeclareMathOperator{\MM}{\mathbb{M}}
\DeclareMathOperator{\VV}{\mathbb{V}}
\DeclareMathOperator{\TT}{\mathbb{T}}
\DeclareMathOperator{\LL}{\mathcal{L}}
\DeclareMathOperator{\DD}{\mathcal{D}}
\DeclareMathOperator{\SW}{\mathcal{S}}
\DeclareMathOperator{\EC}{\mathcal{E}}
\DeclareMathOperator{\AC}{\mathcal{A}}

\title{Research Statement}
\author{Jacob Denson}
\date{}

\begin{document}

\maketitle

{\bf My current research} focuses on the study of Fourier multiplier operators, and their relation to multiplier operators on compact manifolds, as well as the study of fractal dimension and it's relation to geometric patterns in sets.

My study of patterns lead to a probabilistic construction of sets of large Fourier dimension avoiding the vertices of any isosceles triangle within any given smooth curve in $\RR^d$. This remains the only construction in the literature of a set with significant Fourier dimension which avoids a \emph{nonlinear} pattern.

My work on multipliers lead to a complete characterization of all operators on $S^d$ diagonalized by the spherical harmonics which are bounded on $L^p$, for a certain range of $p$. This is the first characterization of $L^p$ boundedness of such multipliers for any $p \neq 2$, and no other characterizations have been proved for multipliers of analogous operators on any other compact manifold.

Both projects 

The work I have conducted naturally suggests several {\bf future problems}.
%
\begin{itemize}
	\item Analyzing the 'return time operator' to extend results on expansions of spherical harmonics to the study of the Laplace-Beltrami operator on $S^d$.

	\item Determining whether our methods extend to other manifolds whose geodesic flow is simpler to understand, such as integrable systems.

	\item Analyzing whether local smoothing bounds

	\item Constructing Random Salem Sets which satisfy a Decoupling Bound.

	\item Determining the relation between certain 'fractal weighted estimates' for the wave equation on $\RR^d$ and the 'density decomposition' of multiplier bounds.
\end{itemize}
%
We provide a further elaboration of these directions later on in the summary.






\section*{Fourier Multiplier Operators}

Multiplier operators are central to the study of harmonic analysis. One research project of mine considers the study of the relation between the $L^p$ boundedness of two types of multipliers:
%
\begin{itemize}
	\item \emph{Fourier multipliers} are those operators $T$ on $\RR^d$ for which we can associate a function $m: \RR^d \to \CC$, known as the \emph{symbol} of $T$, such that for any function $f$, the Fourier transform of $Tf$ obeys the relation $\widehat{Tf} = m \widehat{f}$. Heuristically, this means that $T e^{2 \pi i \xi \cdot x} = m(\xi) e^{2 \pi i \xi \cdot x}$ for each $\xi \in \RR^d$.

	\item On the sphere $S^d$, we have an orthogonal decomposition of $L^2(S^d)$ into the spaces $\mathcal{H}_k(S^d)$ of spherical harmonics of degree $k$. A \emph{multiplier for spherical harmonics} is an operator $T$ on $S^d$ for which there exists $m: \NN \to \CC$, the \emph{symbol} of $T$, such that $Tf = m(k) f$ for each $f \in \mathcal{H}_k(S^d)$.
\end{itemize}

 Such operators initially arose from the study of the classical partial differential equations in physics, and continue to have applications in areas as diverse as partial differential equations, mathematical physics, number theory, and ergodic theory. Every translation invariant operator on $\RR^d$ is a Fourier multiplier operator, and every rotation invariant operator on $S^d$ is a multiplier of the spherical harmonic expansion on $S^d$.

any such operator $T$, we can associate a function $m: \RR^n \to \CC$, known as the \emph{symbol} of $T$, such that for any function $f$, the Fourier transform of $Tf$ obeys the relation $\widehat{Tf} = m \widehat{f}$; thus translation invariant operators are also called \emph{Fourier multiplier operators}.

Understanding the boundedness of Fourier multiplier operators in an $L^p$ norm for $p \neq 2$ underpins any subtle understanding of the Fourier transform. Plane waves oscillating in different directions and with different frequencies are orthogonal to one another, and thus do not interact with one another significantly in terms of the $L^2$ norm, as justified by Bessel's inequality. But plane waves can interact with one another in the $L^p$ norm for $p \neq 2$, and so understanding $L^p$ bounds for Fourier multipliers indicate when this interaction is significant or insignificant. Similarily, spherical harmonics of different degrees on $S^d$ are orthogonal to one another, but studying the $L^p$ bounds of multipliers of the Laplacian on the sphere is crucial to understand when the interactions of different spherical harmonics are significant or not.

The general study of the characterizations of $L^p$ boundedness for the Fourier multipliers was initiated in the 1960s. Mathematicians quickly found simple necessary and sufficient conditions that ensure Fourier multipliers are bounded on $L^1(\RR^d)$, $L^2(\RR^d)$, and $L^\infty(\RR^d)$. But the problem of finding necessary and sufficient conditions for boundedness in $L^p(\RR^d)$ for any other exponent proved impenetrable. Indeed, many interesting problems about the boundedness of \emph{specific} Fourier multipliers, such as the Bochner-Riesz conjecture, remain largely unsolved today.

Thus it came as a surprise in the past decade when results emerged proving necessary and sufficient conditions for \emph{radial} Fourier multipliers to be bounded on $L^p(\RR^d)$. First came the result of BLAH, which gave a necessary and sufficient criteria for bounds of the form $\| Tf \|_{L^p(\RR^d)} \lesssim \| f \|_{L^p(\RR^d)}$ to hold uniformly over \emph{radial functions} $f$, for $|1/p - 1/2| > 1/2d$. An optimist might think this same condition causes the bound to hold uniformly over \emph{all functions} $f$ in the range above, a statement we call the \emph{radial multiplier conjecture}. We now know, by the results of BLAH and BLAH, that the radial multiplier conjecture holds when $d > 4$ and $|1/p - 1/2| > 1/(d-1)$, when $d = 4$ and $|1/p - 1/2| > 11/36$, and when $d = 3$ and $|1/p - 1/2| > 11/26$. But the radial multiplier conjecture has not yet been completely solved in any dimension $d$, and no bounds are known at all when $d = 2$.

The natural analogue of the study of radial multipliers on $\RR^d$ is the study of multipliers of a Laplace-Beltrami operator on a Riemannian manifold. The natural analogue of the study of quasiradial multipliers on $\RR^d$ is the study of multipliers of an operator associated with a \emph{Finsler geometry} on the manifold.

\section*{Pattern Avoidance}

How large must a set be before it must contain a certain point configuration? Problems of this flavor have long been studied in combinatorics since the work of Ramsey. In the last 50 years, analysts have also begun studying analogous problems for infinite subsets $X \subset \RR^d$, where the size of $X$ is measured in terms of a suitable \emph{fractal dimension}. The natural fractal dimension used to measure the size of a set $X$ is often the Hausdorff dimension $\dim_{\mathbb{H}}(X)$ of $X$. But sometimes the \emph{Fourier dimension} $\dim_{\mathbb{F}}(X)$ proves useful, which measures the best possible decay that the Fourier transform of measures supported on $X$ can have; if $\alpha < \dim_{\mathbb{F}}(X)$, then there exists a nonzero measure $\mu$ on $X$ such that $|\widehat{\mu}(\xi)| \lesssim |\xi|^{-\alpha}$ for all $\xi \in \RR^d$. The Fourier dimension thus, morally speaking, measures how uncorrelated the set $X$ is with the Fourier characters $e_\xi(x) = e^{2 \pi i \xi \cdot x}$.

Several definite conjectures on problems about the \emph{density} of certain point configurations in sets have been raised, such as the Falconer distance problem, which asks if an arbitrary subset $X$ of $[0,1]^d$ with \emph{Hausdorff dimension} exceeding $d/2$, then the set of all distances between pairs of points in $X$ must form a set of positive Lebesgue measure. But there are relatively few definite conjectures about the dimension a set requires before it must contain \emph{at least one} family of points fitting a certain kind of configuration. It is not clear, for instance, how large the Hausdorff dimension a set $X \subset \RR^2$ must have before it contains the vertices of at least one isosceles triangle, or, for a particular angle $\theta \in [0,\pi]$, how large a set $X \subset \RR^2$ must be before it must contains three points $a$, $b$, and $c$ such that the angle $abc$ is equal to $\theta$. Until recently, it was thought that subsets of $[0,1]$ of Fourier dimension one must necessarily contain an arithmetic progression of length three, but this has proved not to be the case.

The ability to form conjectures depends heavily on the ability to produce counterexamples. My research in geometric measure theory so far has been on trying to produce such counterexamples. In BLAH, Pramanik and Fraser. In BLAH, I rephrased their argument in probabilistic terms 

 of a set must be before the set of all distances 

Several definite conjectures on problems of this kind have been established since the project begun, such as the Falconer distance problem or Kakeya conjecture, where the point configuration in mind are points lying at a certain distance from one another, or line segments pointing in other directions. For other

\section*{Future Lines of Research}

\end{document}