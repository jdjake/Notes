%%%%%%%%%%%%%%%%%%%%%%%%%%%%%%%%%%%%%%%%%
% Jacobs Landscape Poster
% LaTeX Template
% Version 1.1 (14/06/14)
%
% Created by:
% Computational Physics and Biophysics Group, Jacobs University
% https://teamwork.jacobs-university.de:8443/confluence/display/CoPandBiG/LaTeX+Poster
% 
% Further modified by:
% Nathaniel Johnston (nathaniel@njohnston.ca)
%
% This template has been downloaded from:
% http://www.LaTeXTemplates.com
%
% License:
% CC BY-NC-SA 3.0 (http://creativecommons.org/licenses/by-nc-sa/3.0/)
%
%%%%%%%%%%%%%%%%%%%%%%%%%%%%%%%%%%%%%%%%%

%----------------------------------------------------------------------------------------
%	PACKAGES AND OTHER DOCUMENT CONFIGURATIONS
%----------------------------------------------------------------------------------------

\documentclass[final]{beamer}

\usepackage[scale=1.24]{beamerposter} % Use the beamerposter package for laying out the poster

\usetheme{confposter} % Use the confposter theme supplied with this template

%\setbeamercolor{block title}{fg=ngreen,bg=white} % Colors of the block titles
\setbeamercolor{block title}{fg=dblue!80,bg=white} % Colors of the block titles
\setbeamercolor{block body}{fg=black,bg=white} % Colors of the body of blocks
\setbeamercolor{block alerted title}{fg=white,bg=dblue!70} % Colors of the highlighted block titles
\setbeamercolor{block alerted body}{fg=black,bg=dblue!10} % Colors of the body of highlighted blocks
% Many more colors are available for use in beamerthemeconfposter.sty

%-----------------------------------------------------------
% Define the column widths and overall poster size
% To set effective sepwid, onecolwid and twocolwid values, first choose how many columns you want and how much separation you want between columns
% In this template, the separation width chosen is 0.024 of the paper width and a 4-column layout
% onecolwid should therefore be (1-(# of columns+1)*sepwid)/# of columns e.g. (1-(4+1)*0.024)/4 = 0.22
% Set twocolwid to be (2*onecolwid)+sepwid = 0.464
% Set threecolwid to be (3*onecolwid)+2*sepwid = 0.708

\newlength{\sepwid}
\newlength{\onecolwid}
\newlength{\twocolwid}
\newlength{\threecolwid}
\setlength{\paperwidth}{48in} % A0 width: 46.8in
\setlength{\paperheight}{36in} % A0 height: 33.1in
\setlength{\sepwid}{0.024\paperwidth} % Separation width (white space) between columns
\setlength{\onecolwid}{0.22\paperwidth} % Width of one column
\setlength{\twocolwid}{0.464\paperwidth} % Width of two columns
\setlength{\threecolwid}{0.708\paperwidth} % Width of three columns
\setlength{\topmargin}{-0.5in} % Reduce the top margin size
%-----------------------------------------------------------

\usepackage{graphicx}  % Required for including images

\usepackage{booktabs} % Top and bottom rules for tables

%----------------------------------------------------------------------------------------
%	TITLE SECTION 
%----------------------------------------------------------------------------------------

\title{Fractals Avoiding Fractal Configurations} % Poster title

\author{Jacob Denson (\href{mailto:denson@math.ubc.ca}{denson@math.ubc.ca})} % Author(s)

\institute{Harmonic Analysis Group, University of British Columbia, Vancouver, Canada\\Advisors: Malabika Pramanik and Josh Zahl} % Institution(s)

%----------------------------------------------------------------------------------------

\begin{document}

%\addtobeamertemplate{headline}{} 
%{
%\begin{tikzpicture}[remember picture,overlay] 
%\node [shift={(-10 cm,-8.5cm)}] at (current page.north east) {\includegraphics[height=13cm]{ubclogo.pdf}}; 
%\end{tikzpicture} 
%}

\addtobeamertemplate{block end}{}{\vspace*{2ex}} % White space under blocks
\addtobeamertemplate{block alerted end}{}{\vspace*{2ex}} % White space under highlighted (alert) blocks

\setlength{\belowcaptionskip}{2ex} % White space under figures
\setlength\belowdisplayshortskip{2ex} % White space under equations

\begin{frame}[t] % The whole poster is enclosed in one beamer frame

\begin{columns}[t] % The whole poster consists of three major columns, the second of which is split into two columns twice - the [t] option aligns each column's content to the top

\begin{column}{\sepwid}\end{column} % Empty spacer column

\begin{column}{\onecolwid} % The first column


%----------------------------------------------------------------------------------------
%	INTRODUCTION
%----------------------------------------------------------------------------------------

\vspace{4cm}

\begin{block}{Method: Discretization of Scales}

\begin{itemize}
	\item Following Fraser, Pramanik, and Keleti, we construct solutions by repeatedly dissecting intervals, ala the construction of the Cantor set.

	\item If $X$ is the decreasing limit of sets $X_1, X_2, \dots$, which are unions of intervals, we can discretize the problem so that we only have to avoid a discrete version of the configuration at each dissection.
\end{itemize}

\end{block}

\begin{figure}
\begin{tikzpicture}[scale=4]
%\draw[help lines] (0,0) grid (6,3);

\draw[ultra thick] (0.5,2.5) -- (2.5,2.5);
\draw[ultra thick] (3,2.5) -- (5,2.5);

\draw[ultra thick] (0.7,2) -- (1.4,2);
\draw[ultra thick] (1.7,2) -- (2.4,2);
\draw[ultra thick] (3.4,2) -- (4.1,2);
\draw[ultra thick] (4.3,2) -- (5,2);

\draw[ultra thick] (0.9,1.5) -- (1,1.5);
\draw[ultra thick] (1.1,1.5) -- (1.2,1.5);
\draw[ultra thick] (1.9,1.5) -- (2.1,1.5);
\draw[ultra thick] (3.4,1.5) -- (3.5,1.5);
\draw[ultra thick] (3.6,1.5) -- (3.7,1.5);
\draw[ultra thick] (3.8,1.5) -- (3.9,1.5);
\draw[ultra thick] (4.5,1.5) -- (4.7,1.5);
\draw[ultra thick] (4.8,1.5) -- (4.9,1.5);

\node at (5.5,2.5) {$X_1$};
\node at (5.5,2) {$X_2$};
\node at (5.5,1.5) {$X_3$};
\node at (5.5,0.5) {$X$};

\draw[ultra thick, loosely dotted, line width=1mm] (0.95,1.35) -- (0.95,0.65);
\draw[ultra thick, loosely dotted, line width=1mm] (1.15,1.35) -- (1.15,0.65);

\draw[ultra thick, loosely dotted, line width=1mm] (1.92,1.35) -- (1.97,0.65);
\draw[ultra thick, loosely dotted, line width=1mm] (2.08,1.35) -- (2.03,0.65);

\draw[ultra thick, loosely dotted, line width=1mm] (3.45,1.35) -- (3.45,0.65);
\draw[ultra thick, loosely dotted, line width=1mm] (3.65,1.35) -- (3.65,0.65);
\draw[ultra thick, loosely dotted, line width=1mm] (3.85,1.35) -- (3.85,0.65);

\draw[ultra thick, loosely dotted, line width=1mm] (4.52,1.35) -- (4.58,0.65);
\draw[ultra thick, loosely dotted, line width=1mm] (4.68,1.35) -- (4.62,0.65);

\draw[ultra thick, loosely dotted, line width=1mm] (4.85,1.35) -- (4.85,0.65);

\draw[ultra thick, loosely dotted, line width=1mm] (5.5,1.25) -- (5.5,0.75);

\fill (0.95,0.5) circle [radius=0.5pt];
\fill (1.15,0.5) circle [radius=0.5pt];
\fill (2,0.5) circle [radius=1pt];
\fill (3.45,0.5) circle [radius=0.5pt];
\fill (3.65,0.5) circle [radius=0.5pt];
\fill (3.85,0.5) circle [radius=0.5pt];
\fill (4.6,0.5) circle [radius=1pt];
\fill (4.85,0.5) circle [radius=0.5pt];

\end{tikzpicture}
\caption{Interval Dissection at Discrete Scales}
\end{figure}

\vspace{-2cm}

\begin{block}{}
\begin{itemize}
	\item {\bf Discrete configuration problem}: If $x_1, \dots, x_d \in X_n$, and $f(x_1, \dots, x_d) = 0$, then some $x_i$ and $x_j$ lie in a common interval in $X_n$.

	\item Provided that the discrete configuration problem is satisfied and the length of the intervals forming $X_n$ tends to zero as $n \to \infty$, $X$ avoids all configurations.

	\item For technical reasons, we consider a slightly different discrete configuration problem where at each scale we consider a partition of $X_n$ into unions of intervals, and the problem then becomes that if $f(x_1, \dots, x_d) = 0$, then some $x_i$ and $x_j$ lie in a common part of the partition of $X_n$.

	\item For fractal avoidance, the discrete criterion is obtained if whenever $(x_1, \dots, x_d) \in X_n^d \cap Y$, then $|x_i - x_j| = o(1)$ for some indices $i$ and $j$.
\end{itemize}

%This statement requires citation \cite{Smith:2012qr}.

\end{block}

\end{column} % End of the first column

\begin{column}{\sepwid}\end{column} % Empty spacer column

\begin{column}{\twocolwid} % Begin a column which is two columns wide (column 2)

\begin{columns}[t,totalwidth=\twocolwid] % Split up the two columns wide column

\begin{column}{\onecolwid}\vspace{-.6in} % The first column within column 2 (column 2.1)

%----------------------------------------------------------------------------------------
%	MATERIALS
%----------------------------------------------------------------------------------------

%----------------------------------------------------------------------------------------

\end{column} % End of column 2.1

\begin{column}{\onecolwid}\vspace{-.6in} % The second column within column 2 (column 2.2)

%----------------------------------------------------------------------------------------
%	METHODS
%----------------------------------------------------------------------------------------

%----------------------------------------------------------------------------------------

\end{column} % End of column 2.2

\end{columns} % End of the split of column 2 - any content after this will now take up 2 columns width

%----------------------------------------------------------------------------------------
%	IMPORTANT RESULT
%----------------------------------------------------------------------------------------

\begin{alertblock}{Our Research Problem:\\How Large can Sets with a Fixed Irregularity Be?}
\begin{itemize}
\item The irregularities commonly manifest as avoiding the zero set of a function.
\item Largeness is quantified by the Hausdorff dimension of the irregular set.
\item Examples of such problems including finding a large set $X \subset \mathbf{R}^3$ such that the angles formed by any three distinct points in $X$ are distinct.
\item {\bf Configuration Avoidance}: Find $X$ such that for distinct $x_1, \dots, x_d \in X$, $f(x_1, \dots, x_d) \neq 0$. 
\item Our new method of finding $X$ more naturally considers a generalization of configuration avoidance.
\item {\bf Fractal Avoidance}: Given $Y \subset \mathbf{R}^d$, find $X$ such that $X^d \cap Y \subset \Delta$, where $\Delta = \{ x : x_i = x_j\ \text{for some $i,j$} \}$. Generalize configuration avoidance by setting $Y = f^{-1}(0)$.

\end{itemize}

\end{alertblock}

\begin{alertblock}{Main Result:\\
}

% TODO: Make Bigger?

\hspace{0.2cm}

\begin{center}
{\bf Theorem.}\\
If the zero set of a function $f: \mathbf{R}^{nd} \to \mathbf{R}$ is $\alpha$ dimensional, then we can find $X \subset \mathbf{R}^d$ with Hausdorff dimension $(nd - \alpha)/(n-1)$ such that $f(x_1, \dots, x_d) \neq 0$ for distinct $x_1, \dots, x_d \in X$.
\end{center}

\begin{itemize}
	\item Extends results of Pramanik and Fraser (2018) which give the result when $f$ is smooth and nonsingular.

	\item If $Y \subset \mathbf{R}$ has dimension $\alpha$, we can find a set $X \subset \mathbf{R}$ of dimension $1 - \alpha$ such that $X + X$, $X - X$, and $X \cdot X$ avoids elements of $Y$. We hope to extend this result to finding $X$ as a vector space over $\mathbf{Q}$.

	\item Given a $1$ dimensional set $Y$ such that $Y \cap L$ is zero dimensional for each straight line $L$, and a projection $\pi$ such that $\pi(Y)$ has non-empty interior, we can find a $1/2$ dimensional subset $X$ not containing the vertices of any isoceles triangles.
\end{itemize}

\end{alertblock} 

\begin{figure}
\begin{tikzpicture}[scale=4]
%\draw[help lines] (0,0) grid (14,5);

\draw[ultra thick] (0,0) -- (14,0) -- (14,5) -- (0,5) -- (0,0);

\draw[ultra thick] (0,3.3) -- (0.7,5);
\draw[ultra thick] (0,2.7) -- (9.642,0);
\draw[ultra thick] (0,2.5) -- (14,3.34);
\draw[ultra thick] (0,2.4) -- (14,1.56);
\draw[ultra thick] (0,1) -- (2.3,0);
\draw[ultra thick] (0,0) -- (14,4.76);
\draw[ultra thick] (0,0.8) -- (14,4.16);
\draw[ultra thick] (2.8,0) -- (2.8,5);
\draw[ultra thick] (1.7,5) -- (3.3,0);
\draw[ultra thick] (0,1.5) -- (4.3,5);
\draw[ultra thick] (0,3) -- (10,5);
\draw[ultra thick] (4,0) -- (5.66666,5);
\draw[ultra thick] (1,5) -- (2,0);

\draw[ultra thick] (0,2.9) -- (0.7,5);
\draw[ultra thick] (0,3.2) -- (14,0);
\draw[ultra thick] (0,0.4) -- (14,3.34);
\draw[ultra thick] (0,4.6) -- (14,1.56);
\draw[ultra thick] (0,1) -- (7.3,0);
\draw[ultra thick] (0,0) -- (14,4.76);
\draw[ultra thick] (0,0.1) -- (14,4.16);
\draw[ultra thick] (7.8,0) -- (7.8,5);
\draw[ultra thick] (0,5) -- (8.3,0);
\draw[ultra thick] (0,1.3) -- (9.3,5);
\draw[ultra thick] (0,3.8) -- (14,5);
\draw[ultra thick] (9,0) -- (10.66666,5);
\draw[ultra thick] (0,5) -- (7,0);

\draw[ultra thick] (14,3.3) -- (0.7,5);
\draw[ultra thick] (14,2.7) -- (9.642,0);
\draw[ultra thick] (14,2.5) -- (14,3.34);
\draw[ultra thick] (14,2.4) -- (14,1.56);
\draw[ultra thick] (14,1) -- (2.3,0);
\draw[ultra thick] (14,0) -- (14,4.76);
\draw[ultra thick] (14,0.8) -- (14,4.16);
\draw[ultra thick] (14,5) -- (3.3,0);
\draw[ultra thick] (14,1.5) -- (4.3,5);
\draw[ultra thick] (14,3) -- (10,5);
\draw[ultra thick] (14,0) -- (5.66666,5);
\draw[ultra thick] (14,5) -- (2,0);

\draw[ultra thick] (7,0) -- (6.6,5);
\draw[ultra thick] (9,0) -- (9.5,5);
\draw[ultra thick] (13,0) -- (12.4,5);
\draw[ultra thick] (4.6,0) -- (5,5);
\draw[ultra thick] (5.9,0) -- (6,5);
\end{tikzpicture}
\caption{Our method can now find a full dimensional $X$ avoiding $Y$ formed by uncountably many lines that aren't too `bushy'.}
\end{figure}

%\begin{alertblock}{Come to my talk on Saturday at 4:10PM to Find Out More!}
%\end{alertblock}

%----------------------------------------------------------------------------------------

\begin{columns}[t,totalwidth=\twocolwid] % Split up the two columns wide column again

\begin{column}{\onecolwid} % The first column within column 2 (column 2.1)

%----------------------------------------------------------------------------------------
%	MATHEMATICAL SECTION
%----------------------------------------------------------------------------------------


%----------------------------------------------------------------------------------------

\end{column} % End of column 2.2

\end{columns} % End of the split of column 2

\end{column} % End of the second column

\begin{column}{\sepwid}\end{column} % Empty spacer column

\begin{column}{\onecolwid} % The third column

%----------------------------------------------------------------------------------------
%	CONCLUSION
%----------------------------------------------------------------------------------------

\vspace{4cm}

\begin{block}{Method: Random Selection}

\begin{itemize}
	\item The dimension of $Y$ gives us very little structural information about $Y$, so it behaves like a random distribution of mass.

	\item To combat this, we choose random interval dissections to form $X$ at each discrete scale, pruning intersections with $Y$.

	\item If $Y$ concentrates at a particular location, the random choice of $X$ can stay away from this location. On the other hand, if $Y$ is spread out rather uniformly, we can spread out $X$ uniformly while still avoiding the elements of $Y$.
\end{itemize}
\end{block}

\begin{figure}
\begin{tikzpicture}[scale=4]
%\draw[help lines] (0,0) grid (6,4);

% Y1
\draw[ultra thick] (1,1) -- (1,2.6) -- (1.2,2.6) -- (1.2,2.8) -- (1.4,2.8) -- (1.4,3) -- (1.8,3) -- (1.8,2.8) -- (2,2.8) -- (2,2.4) -- (2.2,2.4) -- (2.2,1.6) -- (1.8,1.6) -- (1.8,1.2) -- (1.6,1.2) -- (1.6,1) -- (1,1);

% X1
% Blue: 88 122 160
\fill[fill=dblue!70] (0.5,3.5) -- (1,3.5) -- (1,3) -- (0.5,3) -- (0.5,3.5);
\fill[fill=dblue!70] (0.25, 0.75) -- (0.25, 1.25) -- (0.75,1.25) -- (0.75, 0.75) -- (0.25,0.75);
\fill[fill=dblue!70] (2,0.4) -- (2.5,0.4) -- (2.5,0.9) -- (2,0.9) -- (2,0.4);
\fill[fill=dblue!70] (2,3) -- (2,3.4) -- (2.4,3.4) -- (2.4,3) -- (2,3);

% Frame
\draw[ultra thick] (0,0) -- (3,0) -- (3,4) -- (0,4) -- (0,0);
\draw[ultra thick] (3,0) -- (6,0) -- (6,4) -- (3,4) -- (3,0);

% Y2
\draw[ultra thick] (3.2,3.2) -- (3.8,3.2) -- (3.8,3.4) -- (3.8,3.6) -- (3.6,3.6) -- (3.6,3.4) -- (3.2,3.4) -- (3.2,3.2);
\draw[ultra thick] (4,1.8) -- (4,2) -- (4.2,2) -- (4.2,2.2) -- (4.4,2.2) -- (4.4,1.8) -- (4,1.8);
\draw[ultra thick] (5,1) -- (5,1.2) -- (4.8,1.2) -- (4.8,1.6) -- (5.2,1.6) -- (5.2,1.4) -- (5.4,1.4) -- (5.4,1) -- (5,1);
\draw[ultra thick] (4.4,3) -- (4.4,3.2) -- (4.6,3.2) -- (4.6,3.4) -- (5,3.4) -- (5,3.2) -- (4.8,3.2) -- (4.8,2.8) -- (4.6,2.8) -- (4.6,3) -- (4.4,3);
\draw[ultra thick] (4.2,0.6) -- (4.2,0.8) -- (4.4,0.8) -- (4.4,1) -- (4.6,1) -- (4.6,0.6) -- (4.2,0.6);
\draw[ultra thick] (3.4,2.4) -- (3.6,2.4) -- (3.6,2.2) -- (3.8,2.2) -- (3.8,2) -- (3.4,2) -- (3.4,2.4);

% X2
\fill[fill=dblue!70] (3.8,0.8) -- (3.8,1) -- (4,1) -- (4,0.8) -- (3.8,0.8);
\fill[fill=dblue!70] (3.7, 2.6) -- (3.9,2.6) -- (3.9,2.8) -- (3.7,2.8) -- (3.7,2.6);
\fill[fill=dblue!70] (5.7, 1.6) -- (5.9,1.6) -- (5.9,1.8) -- (5.7,1.8) -- (5.7,1.6);
\fill[fill=dblue!70] (4.4,1.4) -- (4.4,1.6) -- (4.6,1.6) -- (4.6,1.4) -- (4.4,1.4);
\fill[fill=dblue!70] (3.5, 1.3) -- (3.7,1.3) -- (3.7,1.5) -- (3.5,1.5) -- (3.5,1.3);
\fill[fill=dblue!70] (5.2, 2.3) -- (5.4,2.3) -- (5.4,2.5) -- (5.2,2.5) -- (5.2,2.3);
\fill[fill=dblue!70] (4.6,2.4) -- (4.6,2.6) -- (4.8,2.6) -- (4.8,2.4) -- (4.6,2.4);
\fill[fill=dblue!70] (5.2,0.4) -- (5.4,0.4) -- (5.4,0.6) -- (5.2,0.6) -- (5.2,0.4);

\end{tikzpicture}
\caption{Random choices of $X$ avoid $Y$.}
\end{figure}

\vspace{-2cm}

\begin{block}{}
\begin{itemize}
	\item Our method currently works when the dimension of $Y$ is quantified as the box counting dimension. 

	\item We are currently working on using hyperdyadic scaling to extend the result where $Y$ is quantified by it's Hausdorff dimension.

	\item The construction parallels a random construction of an independant set in a hypergraph, similar to Turan's theorem.

	\item We are also currently looking at using other techniques on hypergraphs to improve the dimension of $X$ when $Y$ has more structure.
\end{itemize}

\end{block}

%----------------------------------------------------------------------------------------
%	ADDITIONAL INFORMATION
%----------------------------------------------------------------------------------------

%----------------------------------------------------------------------------------------
%	REFERENCES
%----------------------------------------------------------------------------------------

%\begin{block}{References}

%\nocite{*} % Insert publications even if they are not cited in the poster
%\small{\bibliographystyle{unsrt}
%\bibliography{sample}\vspace{0.75in}}

%\end{block}

%----------------------------------------------------------------------------------------
%	ACKNOWLEDGEMENTS
%----------------------------------------------------------------------------------------

\setbeamercolor{block title}{fg=red,bg=white} % Change the block title color

%----------------------------------------------------------------------------------------
%	CONTACT INFORMATION
%----------------------------------------------------------------------------------------

%\setbeamercolor{block alerted title}{fg=black,bg=norange} % Change the alert block title colors
%\setbeamercolor{block alerted body}{fg=black,bg=white} % Change the alert block body colors

%----------------------------------------------------------------------------------------

\end{column} % End of the third column

\end{columns} % End of all the columns in the poster

\end{frame} % End of the enclosing frame

\end{document}
