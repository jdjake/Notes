
% This is the template for the BIGS poster exhibition


\documentclass[12pt]{article}

% packages needed
\usepackage{subfig}

\usepackage[german]{babel}
\usepackage{latexsym,amssymb,amsmath, textcomp, amsthm}
\usepackage{xcolor}
\usepackage[pdftex]{graphicx}
\usepackage{amsmath,amsfonts}
\usepackage{multicol}
\usepackage{pifont}
\usepackage{geometry}

\usepackage{comment}
\usepackage{bbm}
\usepackage[utf8]{inputenc}
\usepackage{authblk}
\numberwithin{equation}{section}

\def\R{\mathbf{R}}
\def\N{\mathbf{N}}
\def\C{\mathbf{C}}
\def\Z{\mathbf{Z}}

\def\lesim{\lesssim}

\def\mlimits{\displaystyle\sum\limits}
\def\msup{\displaystyle\sup}
\def\mint{\displaystyle\int}

\def\lam{\lambda}

\def\begineq{\begin{equation}}
\def\endeq{\end{equation}}

\theoremstyle{plain}
\newtheorem*{thm}{Theorem}
\newtheorem{lem}{Lemma}
\newtheorem{cor}{Corollary}
\newtheorem{defi}{Definiton}


\theoremstyle{remark}
\newtheorem{rem}{Remark}


% graphics extensions
\DeclareGraphicsExtensions{.jpg,.pdf,.pdftex,.eps}

% parameters
\setlength{\pdfpageheight}{594mm}
\setlength{\pdfpagewidth}{420mm}
\setlength{\paperheight}{594mm}
\setlength{\paperwidth}{420mm}
\setlength{\voffset}{-.5in}
\setlength{\hoffset}{-1.0in}
\setlength{\evensidemargin}{5mm}
\setlength{\oddsidemargin}{5mm}
\setlength{\topmargin}{0mm}
\setlength{\headheight}{0mm}
\setlength{\headsep}{0mm}
\setlength{\textheight}{624mm}
\setlength{\textwidth}{410mm}
\setlength{\parindent}{0pt}
\setlength{\parskip}{2explus2ex}
\setlength{\fboxsep}{0.01\textwidth}
\setlength{\fboxrule}{0.0025\textwidth}
\newlength{\boxwidth}
\setlength{\boxwidth}{0.975\textwidth}
\setlength{\columnsep}{1cm}
\setlength{\columnseprule}{1pt}
\setlength{\multicolsep}{0cm}
%\setcounter{unbalance}{20}


% font for title of poster
\newcommand{\titlefont}[1]
{\protect{\fontencoding{T1}\fontfamily{pag}\fontseries{b}%
\fontshape{n}\fontsize{1cm}{1ex}
\selectfont{#1}}}


% command for page headings
\newcommand{\newpart}[1]
{\colorbox[rgb]{0.97,0.92,0.7}{\makebox[0.97\columnwidth]
{\rule[-1.2ex]{0pt}{3.7ex}\partfont{#1}}}\bigskip}


% font for headings of pages
\newcommand{\partfont}[1]{{\Large \textsf{\textbf{#1}}}}


% page style
\pagestyle{empty}




%-------------------------------------------------------------------------



%%%%%%%%%%%%%%%%%%%%%%%%%%%%%%%
%%%%%%%%%%%%%%%%%%%%%%%%%%%%%%%

\begin{document}


% BIGS-Logo, title and Uni-Logo
% Do not change anything in this poster heading
% except to fill in title, name and supervisor !
%
% BIGS-Logo -------------------------------------
\parbox{10cm}{
% \vskip1cm
\hspace{.5cm}
\begin{minipage}[b]{8.2cm}
\includegraphics[width=8cm]{UWlogo.jpg}
\end{minipage}
}
%
% Poster Title -------------------------------------
\parbox{22cm}{
% FILL IN THE TITLE OF YOUR POSTER.
%Example:
\titlefont{Salem Sets\\[0.5ex]
Avoiding Nonlinear Configurations}\\[1cm]
%
\vspace{0.5cm}
% FILL IN YOUR NAME, please do not change the format, in our experience this leads to confusion and errors.
% Example: \Huge{\bf\Mustermann, Ingrid\\
    \Huge{\bf  Jacob Denson \\[0.5ex]}
% FILL IN YOUR ADVISOR's NAME. Please do not change the format, in our experience this leads to confusion and errors.
% Example: \Huge{\bf\ Prof. Dr. John Doe\\
    \Huge{\bf  Advisor: Andreas Seeger}}
%
% Uni-Logo -------------------------------------------
\begin{minipage}[b]{8.6cm}
\titlefont{RTG \\
Analysis and PDEs\\
at Wisconsin}
\end{minipage}
% End of poster heading ---------------------------------------

%
\vspace{2cm}
%


% Do not change anything in the command lines for
% the frame box, the paragraph box or the
% multicolumns.
\fbox{
\parbox{0.9972\boxwidth}{
\setlength{\fboxsep}{0.005\textwidth}
\setlength{\fboxrule}{0.00125\textwidth}


\raggedcolumns
\begin{multicols}{2}
%------------------------------------------------------------------------------
% Replace the text "part 1" with something meaningful or erase the text.
% Example:\newpart{Introduction}
\newpart{Research Problem: Can Large Sets Avoid Patterns?}

% BEGIN OF PAGE 1.
% REMOVE THE FOLLOWING GARBAGE AND TYPE IN YOUR MATERIAL.

\large{More specifically: If a set $S \subset \mathbb{R}^d$ has large \emph{fractal dimension}, does it contain patterns? The main focus of this project is on the construction of counterexamples: for a given function $f$, can we construct large sets $S$ such that $S$ \emph{does not contain} distinct points $x_1,\dots,x_n$ satisfying $f(x_1,\dots,x_n) = 0$, i.e. such that $S$ avoids zeroes of $f$?
%
\begin{itemize}
    \item If $f(x_1,x_2,x_3) = (x_1 - x_2) - (x_2 - x_3)$, then sets avoiding zeroes of $f$ do not contain three term arithmetic progressions.

    \item If $f(x_1,x_2,x_3) = |x_1 - x_2|^2 - |x_2 - x_3|^2$, then sets in $\mathbb{R}^d$ avoiding zeroes of $f$ do not contain the vertices of any isosceles triangle.
\end{itemize}
%
Mainly, this project constructs large \emph{Salem sets} avoiding zeroes of \emph{nonlinear} functions.\\

There are several fractal dimensions, and they differ subtly in the properties they measure. The \emph{Hausdorff dimension} $\dim_{\mathbb{H}}(S)$ of a set $S \subset \mathbb{R}^d$ intuitively measures the possibility of distributing mass onto $S$ in a way that does not concentrate too strongly around points. The \emph{Fourier dimension} $\dim_{\mathbb{F}}(S)$ of a set $S \subset \mathbb{R}^d$ measures the possiblity, not only of avoiding mass concentration at points, but also of avoiding mass concentration near families of equally spaced points, i.e. concentration `at a particular frequency', as measured quantitatively through the Fourier transform: a set $S$ has $\dim_{\mathbb{F}}(S) > \alpha$ precisely when one can find a probability measure $\mu$ with $\text{supp}(\mu) \subset S$ such that $|\widehat{\mu}(\xi)| \leq |\xi|^{-\alpha/2}$.\\

If $S$ has large Hausdorff dimension, then one can distribute mass on $S$ not concentrated near points is also not concentrated near `most frequencies'. But in order to have large Fourier dimension, a distribution of mass must avoid concentrating near \emph{all frequencies}. It is always true that $\dim_{\mathbb{F}}(S) \leq \dim_{\mathbb{H}}(S)$ for any set $S \subset \mathbb{R}^d$, but the reverse is often \emph{not true} if the set is clustered `near particular frequencies'.
}

TODO: Picture of Cantor Set, Hyperplane, Curved Surface

\columnbreak
%------------------------------------------------------------------------------

% Replace the text "part 2" with something meaningful or erase the text.
\newpart{Salem Sets: Structure vs. Randomness}

\large{
A set is \emph{Salem} if it's Fourier dimension agrees with it's Hausdorff dimension. This is a common feature of \emph{random sets}, which tend to avoid clustering near equally spaced points with high probability. On the other hand, it is \emph{suprisingly difficult} to find Salem sets without employing randomness in some way, since adding \emph{structure} to a set can possibly introduce clustering near certain frequencies in very subtle ways, which makes it very difficult to compute Fourier dimensions. In particular, \emph{nonlinear structure} is especially difficult to understand, as indicated by the following open problems:
%
\begin{itemize}
    \item There are very few explicit (i.e. nonrandom) examples of Salem sets. Pretty much the only examples occur from the theory of Diophantine approximation (Cite Kauffman, Hambrook, Fraser, etc). For $d > 2$, it remains an open problem to construct Salem sets $S \subset \mathbb{R}^d$ of dimension $s$ for general values $s \in [0,d]$.

    \item We do not know the Fourier dimension of $\{ x + x^2 : x \in C \}$, where $C$ is the Cantor set, whereas we know the set has Hausdorff dimension $\log_3(2)$.
\end{itemize}
%
While there are many constructions of sets with large \emph{Hausdorff dimension} avoiding the zeroes of nonlinear functions $f$ (Cite:), most constructions of large Salem sets avoiding functions $f$ focus on the case when $f$ is linear, e.g. on the study of arithmetic progressions or other linear relations between points. Nonetheless, here we focus mostly on nonlinear functions $f$.\\

\vspace{-0.5cm}

\begin{thm}
    Suppose $f: (\mathbf{R}^d)^n \to \mathbf{R}^d$ is given by
    %
    \[ f(x^1,\dots,x^n) = x^1 - g(x^2,\dots,x^n), \]
    %
    where $g: (\mathbb{R}^d)^{n-1} \to \mathbb{R}^d$ is smooth, and $D_{x^k} g = (\partial g^i / \partial x^k_j)$ is an invertible matrix for all $2 \leq k \leq n$. Then we can construct a Salem set $S \subset \mathbb{R}^d$ with
    %
    \[ \dim_{\mathbf{F}}(S) = \frac{d}{n - 3/4} \]
    %
    avoiding solutions to $f$.
\end{thm}

Under these assumptions, (TODO) constructs sets $S$ with
%
\[ \dim_{\mathbb{H}}(S) \geq d/(n-1) \]
%
avoiding zeroes to $f$, and we conjecture the theorem above can be improved to this bound in the setting of Salem sets.
}



% BEGIN OF PAGE 2.
% REMOVE THE FOLLOWING GARBAGE AND TYPE IN YOUR MATERIAL.

\end{multicols}}}\\


% END OF PAGE 2.


%%%%%%%%%%%%%%%%%%%%%%%%%%%%%%%%%%%%%%%%%%%%%%%%%%%%%%%%%%%%%%%%%%%%%%%%%%
%%%%%%%%%%%%%%%%%%%%%%%%%%%%%%%%%%%%%%%%%%%%%%%%%%%%%%%%%%%%%%%%%%%%%%%%%%
\fbox{
\parbox{0.9972\boxwidth}{
\setlength{\fboxsep}{0.005\textwidth}
\setlength{\fboxrule}{0.00125\textwidth}

\raggedcolumns
\begin{multicols}{2}
%--------------------------------------------------------------------------
% Replace the text "part 3" with something meaningful or erase the text.
\newpart{Constructing Salem Sets}

% BEGIN OF PAGE 3.
% REMOVE THE FOLLOWING GARBAGE AND TYPE IN YOUR MATERIAL.
One can view the construction method as a random interval dissection method iterating on different scales, ala the construction of a Cantor set. The main importance is working with intervals is that we can \emph{discretize} the problem.

\vspace{20cm}

contents. 




% END OF PAGE 3.

\columnbreak


%----------------------------------------------------------------------------
% Replace the text "part 4" with something meaningful or erase the text.
\newpart{Section title}

% BEGIN OF PAGE 4.
% REMOVE THE FOLLOWING GARBAGE AND TYPE IN YOUR MATERIAL.

Contents. 

\vspace{14cm}
% END OF PAGE 4.


%%%%%%%%%%%%%%%%%%%%%%%%%%%%%%%%%%%%%%%%%%%%%
{\bf References}

% FILL IN THE REFERENCES YOU HAVE USING EITHER ITEMIZE (i.e. A BULLET) OR ENUMERATE (i.e. WITH NUMBERS)

Ekstrom Survey


%%%%%%%%%%%%%%%%%%%%%%%%%%%%%%%%%%%%%%%%%%%%%%
\hrule
\vspace{0.5cm}

% REPLACE THE TEXT BELOW WITH YOUR INSTITUTE AND YOUR EMAIL-ADDRESS. DO NOT ERASE THE YEAR.
% EXAMPLE:
% 2013 Mathematisches Institut
% ingrid.mustermann@imaginary-address.de
2022 University of Wisconsin Madison \\
Your email address

%%%%%%%%%%%%%%%%%%%%%%%%%%%%%%%%%%%%%%%%%%%%%%%

\end{multicols}}}


\end{document}
