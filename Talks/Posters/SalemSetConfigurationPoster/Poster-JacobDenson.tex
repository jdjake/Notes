        
% This is the template for the BIGS poster exhibition

\documentclass[12pt]{article}

% packages needed
\usepackage{subfig}

\usepackage[german]{babel}
\usepackage{latexsym,amssymb,amsmath, textcomp, amsthm}
\usepackage{xcolor}
\usepackage[pdftex]{graphicx}
\usepackage{amsmath,amsfonts}
\usepackage{multicol}
\usepackage{pifont}
\usepackage{geometry}
\usepackage{multicol}
\usepackage{tikz}
\usepackage{xcolor}
\usepackage{float}
\usepackage{wrapfig}
\usepackage{hyperref}

\usepackage{comment}
\usepackage{bbm}
\usepackage[utf8]{inputenc}
\usepackage{authblk}
\numberwithin{equation}{section}

\def\R{\mathbf{R}}
\def\N{\mathbf{N}}
\def\C{\mathbf{C}}
\def\Z{\mathbf{Z}}

\def\lesim{\lesssim}

\def\mlimits{\displaystyle\sum\limits}
\def\msup{\displaystyle\sup}
\def\mint{\displaystyle\int}

\def\lam{\lambda}

\def\begineq{\begin{equation}}
\def\endeq{\end{equation}}

\theoremstyle{plain}
\newtheorem{thm}{Theorem}
\newtheorem*{lem}{Lemma}
\newtheorem{cor}{Corollary}
\newtheorem{defi}{Definiton}


\theoremstyle{remark}
\newtheorem{rem}{Remark}


% graphics extensions
\DeclareGraphicsExtensions{.jpg,.pdf,.pdftex,.eps}

% parameters
\setlength{\pdfpageheight}{594mm}
\setlength{\pdfpagewidth}{420mm}
\setlength{\paperheight}{594mm}
\setlength{\paperwidth}{420mm}
\setlength{\voffset}{-.5in}
\setlength{\hoffset}{-1.0in}
\setlength{\evensidemargin}{5mm}
\setlength{\oddsidemargin}{5mm}
\setlength{\topmargin}{0mm}
\setlength{\headheight}{0mm}
\setlength{\headsep}{0mm}
\setlength{\textheight}{624mm}
\setlength{\textwidth}{410mm}
\setlength{\parindent}{0pt}
\setlength{\parskip}{2explus2ex}
\setlength{\fboxsep}{0.01\textwidth}
\setlength{\fboxrule}{0.0025\textwidth}
\newlength{\boxwidth}
\setlength{\boxwidth}{0.975\textwidth}
\setlength{\columnsep}{1cm}
\setlength{\columnseprule}{1pt}
\setlength{\multicolsep}{0cm}
%\setcounter{unbalance}{20}


% font for title of poster
\newcommand{\titlefont}[1]
{\protect{\fontencoding{T1}\fontfamily{pag}\fontseries{b}%
\fontshape{n}\fontsize{1.3cm}{1ex}
\selectfont{#1}}}


% command for page headings
\newcommand{\newpart}[1]
{\colorbox[rgb]{0.97,0.92,0.7}{\makebox[0.97\columnwidth]
{\rule[-1.2ex]{0pt}{3.7ex}\partfont{#1}}}\bigskip}


% font for headings of pages
\newcommand{\partfont}[1]{{\LARGE \textsf{\textbf{#1}}}}


% page style
\pagestyle{empty}




%-------------------------------------------------------------------------



%%%%%%%%%%%%%%%%%%%%%%%%%%%%%%%
%%%%%%%%%%%%%%%%%%%%%%%%%%%%%%%

\begin{document}


% BIGS-Logo, title and Uni-Logo
% Do not change anything in this poster heading
% except to fill in title, name and supervisor !
%
% BIGS-Logo -------------------------------------
\parbox{10cm}{
% \vskip1cm
\hspace{.5cm}
\begin{minipage}{9cm}
\includegraphics[width=8cm]{UWlogo.jpg}
\end{minipage}
}
%
% Poster Title -------------------------------------
\begin{minipage}{13cm}
\titlefont{Large Salem\\[0.5ex]
Sets Avoiding\\[0.5ex]
Nonlinear Patterns}
\end{minipage}
%
% Uni-Logo -------------------------------------------
\begin{minipage}{16cm}
\titlefont{Jacob Denson\\[0.5ex]
Advisor: Andreas Seeger}
\end{minipage}
% End of poster heading ---------------------------------------
%

\vspace{-1.2cm}

\quad\quad\quad\quad\quad\quad\quad\quad\quad\quad\quad\quad\quad\quad\quad\quad\quad\quad\quad\quad\quad\quad\quad\quad\quad\quad\quad\quad\quad\quad\quad\quad\quad\quad\quad\quad\quad\quad\quad\quad\quad\quad\quad\quad\quad\quad\quad\quad\quad\quad\quad\quad\quad\quad\quad\quad \emph{This material is based upon work supported by the}

\vspace{-0.4cm}

\quad\quad\quad\quad\quad\quad\quad\quad\quad\quad\quad\quad\quad\quad\quad\quad\quad\quad\quad\quad\quad\quad\quad\quad\quad\quad\quad\quad\quad\quad\quad\quad\quad\quad\quad\quad\quad\quad\quad\quad\quad\quad\quad\quad\quad\quad\quad\quad\quad\quad\quad\quad\quad\quad\quad\quad \emph{National Science Foundation under Grant DMS-1764295.}

\vspace{0.1cm}
%

% Do not change anything in the command lines for
% the frame box, the paragraph box or the
% multicolumns.
\fbox{
\parbox{0.9972\boxwidth}{
\setlength{\fboxsep}{0.005\textwidth}
\setlength{\fboxrule}{0.00125\textwidth}

\raggedcolumns
\begin{multicols}{2}
%------------------------------------------------------------------------------
% Replace the text "part 1" with something meaningful or erase the text.
% Example:\newpart{Introduction}
\newpart{Research Problem: Can Large Sets Avoid Patterns?}

% BEGIN OF PAGE 1.
% REMOVE THE FOLLOWING GARBAGE AND TYPE IN YOUR MATERIAL.

\large{More specifically: If a set $X \subset \mathbb{R}^d$ has large \emph{fractal dimension}, does it contain patterns? The main focus of this project is on the construction of counterexamples: for a given function $f$ with domain $(\mathbb{R}^d)^n$, can we construct large sets $X$ such that there are no distinct points $x_1,\dots,x_n \in X$ with $f(x_1,\dots,x_n) = 0$? We often study functions $f$ which vanish on the diagonal $\Delta = \{ (x,\dots,x) : x \in \mathbb{R}^d \}$, which makes it difficult to avoid zeroes if $X$ is `thick', i.e. has large fractal dimension.}

\begin{wrapfigure}{L}{0.45\linewidth}
\hspace{1cm} \begin{tikzpicture}[scale=0.09]
%\draw[help lines] (0,0) grid (6,3);

% 1 2 4 8 9 11 19 22 23 26 28 31 49 57 59 62 63 66 68 71 78 81 82

\draw[ultra thick] (1-0.3,0) -- (1+0.3,0);
\draw[ultra thick] (2-0.3,0) -- (2+0.3,0);
\draw[ultra thick] (4-0.3,0) -- (4+0.3,0);
\draw[ultra thick] (8-0.3,0) -- (8+0.3,0);
\draw[ultra thick] (9-0.3,0) -- (9+0.3,0);
\draw[ultra thick] (11-0.3,0) -- (11+0.3,0);
\draw[ultra thick] (19-0.3,0) -- (19+0.3,0);
\draw[ultra thick] (22-0.3,0) -- (22+0.3,0);
\draw[ultra thick] (23-0.3,0) -- (23+0.3,0);
\draw[ultra thick] (26-0.3,0) -- (26+0.3,0);
\draw[ultra thick] (28-0.3,0) -- (28+0.3,0);
\draw[ultra thick] (31-0.3,0) -- (31+0.3,0);
\draw[ultra thick] (49-0.3,0) -- (49+0.3,0);
\draw[ultra thick] (57-0.3,0) -- (57+0.3,0);
\draw[ultra thick] (59-0.3,0) -- (59+0.3,0);
\draw[ultra thick] (62-0.3,0) -- (62+0.3,0);
\draw[ultra thick] (63-0.3,0) -- (63+0.3,0);
\draw[ultra thick] (66-0.3,0) -- (66+0.3,0);
\draw[ultra thick] (68-0.3,0) -- (68+0.3,0);
\draw[ultra thick] (71-0.3,0) -- (71+0.3,0);
\draw[ultra thick] (78-0.3,0) -- (78+0.3,0);
\draw[ultra thick] (81-0.3,0) -- (81+0.3,0);
\draw[ultra thick] (82-0.3,0) -- (82+0.3,0);

\node[draw] at (41,-10) {A set avoiding 3-term APs};

\foreach \x in {-80, -78, -77, -74, -72, -68, -67, -61, -60, -57, -54, -52, -44, -37, -14, 2, 19, 22, 26, 31, 35, 42, 45, 49, 51, 55, 58, 63, 69, 75}
    {\draw[ultra thick] (0.5*\x+40,-45+\x*\x/200-9) -- (0.5*\x+40+0.4,-45+\x*\x/200 + \x/125 + 160/25 - 640/100-9);}
% z = 0.5x + 40
% y = -45 + x^2/200
%   = -45 + (2z - 80)^2/200
%   = -45 + z^2/50 - (320/200) z + 6400/20
% slope should be z/25 - (320/200) = x/50 + 40/25 - (320/200)

%\draw[ultra thick] (41-5,-50) -- (41+5,-50);
\node[draw,text width = 6cm] at (41,-68) {A subset of the parabola avoiding isosceles triangles};
\end{tikzpicture}
\end{wrapfigure}

\large{ \vspace{0.2cm} Example choices of $f$:
%
\begin{itemize}
    \item If $f(x_1,x_2,x_3) = (x_1 - x_2) - (x_2 - x_3)$, then sets avoiding zeroes of $f$ do not contain three term arithmetic progressions.

    \item If $f(x_1,x_2,x_3) = |x_1 - x_2|^2 - |x_2 - x_3|^2$, then sets in $\mathbb{R}^d$ avoiding zeroes of $f$ do not contain the vertices of any isosceles triangle.
\end{itemize}

Mainly, this project constructs large \emph{Salem sets} avoiding zeroes of \emph{nonlinear} functions. Here are some results taken from (D., 2021):

\begin{thm}
    \hspace{-0.1cm} Suppose $f: (\mathbb{R}^d)^n \to \mathbb{R}^i$ is a submersion. Then we can construct a Salem set $X \subset \mathbb{R}^d$ avoiding solutions to $f$ with $\dim(X) = i / (n - 1/2)$.
\end{thm}

\begin{thm}
    Let $g: (\mathbb{R}^d)^{n-1} \to \mathbb{R}^d$ be smooth, such that $D_{x^k} g = (\partial g_i / \partial x^k_j)$ is an invertible matrix for each $1 \leq k \leq n-1$. If
    %
    \[ f(x^1,\dots,x^n) = x^n - g(x^1,\dots,x^{n-1}), \]
    %
    then we can construct a Salem set $X \subset \mathbb{R}^d$ avoiding solutions to $f$ with $\dim(X) = d / (n - 3/4)$, larger than that guaranteed by Theorem 1.
\end{thm}

For instance, we can use these results to construct, for any smooth $\gamma: [0,1] \to \mathbb{R}^d$, a Salem set $X \subset [0,1]$ with dimension $4/9$ such that $\gamma(X)$ avoids vertices of isosceles triangles.}

\vspace{0.3cm}

\columnbreak
%------------------------------------------------------------------------------

% Replace the text "part 2" with something meaningful or erase the text.
\newpart{Salem Sets: Structure vs. Randomness}

\large{

There are several fractal dimensions, and they differ subtly in the properties they measure:
%
\begin{itemize}
    \item The \emph{Hausdorff dimension} $\dim_{\mathbb{H}}(X)$ of a set $X \subset \mathbb{R}^d$ measures the ability to distribute mass onto $X$ in a way that does not concentrate too strongly around individual points.

    \item The \emph{Fourier dimension} $\dim_{\mathbb{F}}(X)$ of a set $X \subset \mathbb{R}^d$ measures the ability to distribute mass avoiding concentration `at a particular frequency', as measured quantitatively through the Fourier transform. %a set $X$ has $\dim_{\mathbb{F}}(X) > \alpha$ precisely when one can find a probability measure $\mu$ with $\text{supp}(\mu) \subset X$ such that $|\widehat{\mu}(\xi)| \leq |\xi|^{-\alpha/2}$ for all $\xi \in \mathbb{R}^d$.
\end{itemize}

One always has $\dim_{\mathbb{F}}(X) \leq \dim_{\mathbb{H}}(X)$ for any set $X \subset \mathbb{R}^d$, but the reverse is often \emph{not true} if the set is clustered `near particular frequencies', like if $X$ is a flat surface (clustered near frequencies travelling tangent to the hyperplane), or a Cantor set (clustered near frequencies of the form $3^n$), both sets with Fourier dimension zero. On the other hand, a curved hypersurface in $\mathbb{R}^d$ has Fourier dimension equal to $d-1$, also it's Hausdorff dimension.

\begin{wrapfigure}{L}{0.45\linewidth}
\begin{tikzpicture}[scale=9]

\draw[ultra thick] (0,0) -- (1/3,0);
\draw[ultra thick] (0.66,0) -- (1,0);

\draw[ultra thick] (0,0.1) -- (1/9,0.1);
\draw[ultra thick] (2/9,0.1) -- (1/3,0.1);
\draw[ultra thick] (2/3,0.1) -- (7/9,0.1);
\draw[ultra thick] (8/9,0.1) -- (1,0.1);

\draw[ultra thick] (0,0.2) -- (1/27,0.2);
\draw[ultra thick] (2/27,0.2) -- (3/27,0.2);
\draw[ultra thick] (6/27,0.2) -- (7/27,0.2);
\draw[ultra thick] (8/27,0.2) -- (9/27,0.2);
\draw[ultra thick] (18/27,0.2) -- (19/27,0.2);
\draw[ultra thick] (20/27,0.2) -- (21/27,0.2);
\draw[ultra thick] (24/27,0.2) -- (25/27,0.2);
\draw[ultra thick] (26/27,0.2) -- (1,0.2);


\foreach \x in {0,2} {
    \draw[densely dotted, ultra thick, red] (\x/3,0) sin (\x/3 + 1/6,0.04);
    \draw[densely dotted, ultra thick, red] (\x/3 + 1/6,0.04) cos (\x/3 + 1/3,0);
}

\foreach \x in {1} {
    \draw[densely dotted, ultra thick, red] (\x/3,0) sin (\x/3 + 1/6,-0.04);
    \draw[densely dotted, ultra thick, red] (\x/3 + 1/6,-0.04) cos (\x/3 + 1/3,0);
}



\foreach \x in {0,2,4,6,8} {
    \draw[densely dotted, ultra thick, red] (\x/9,0.1) sin (\x/9 + 1/18,0.1+0.04);
    \draw[densely dotted, ultra thick, red] (\x/9 + 1/18,0.1+0.04) cos (\x/9 + 1/9,0.1);
}

\foreach \x in {1,3,5,7} {
    \draw[densely dotted, ultra thick, red] (\x/9,0.1) sin (\x/9 + 1/18,0.1-0.04);
    \draw[densely dotted, ultra thick, red] (\x/9 + 1/18,0.1-0.04) cos (\x/9 + 1/9,0.1);
}





\foreach \x in {0,2,4,6,8,10,12,14,16,18,20,22,24,26} {
    \draw[densely dotted, ultra thick, red] (\x/27,0.2) sin (\x/27 + 1/54,0.2+0.04);
    \draw[densely dotted, ultra thick, red] (\x/27 + 1/54,0.2+0.04) cos (\x/27 + 1/27,0.2);
}

\foreach \x in {1,3,5,7,9,11,13,15,17,19,21,23,25} {
    \draw[densely dotted, ultra thick, red] (\x/27,0.2) sin (\x/27 + 1/54,0.2-0.04);
    \draw[densely dotted, ultra thick, red] (\x/27 + 1/54,0.2-0.04) cos (\x/27 + 1/27,0.2);
}





%\node at (-1/3,0) {$\cdots$};

%\fill (0,0) circle [radius=0.5pt];
%\fill[fill = white] (0,0) circle [radius=0.3pt];

%\fill (1/3,0) circle [radius=0.5pt];
%\fill[fill = white] (1/3,0) circle [radius=0.3pt];

%\fill (2/3,0) circle [radius=0.5pt];
%\fill[fill = white] (2/3,0) circle [radius=0.3pt];

%\fill (1,0) circle [radius=0.5pt];
%\fill[fill = white] (1,0) circle [radius=0.3pt];

%\node at (4/3,0) {$\cdots$};





%\node at (-2/9,0.1) {$\cdots$};

%\foreach \x in {-1,...,10} {
%    \fill (\x/9,0.1) circle [radius=0.3pt];
%    \fill[fill = white] (\x/9,0.1) circle [radius=0.2pt];   
%}

%\node at (11/9,0.1) {$\cdots$};



%\node at (-2/27,0.2) {$\cdots$};

%\foreach \x in {0,...,27} {
%    \fill (\x/27,0.2) circle [radius=0.2pt];
%    \fill[fill = white] (\x/27,0.2) circle [radius=0.1pt];
%}

%\node at (29/27,0.2) {$\cdots$};



\node[draw,text width = 7cm] at (0.5,-0.2) {The Cantor Set correlates near frequencies of the form $3^n$};
\end{tikzpicture}
\end{wrapfigure}

%TODO: Picture of Cantor Set, Hyperplane, Curved Surface

\vspace{0.1cm}

We say a set $X$ is \emph{Salem} if $\dim_{\mathbb{F}}(X) = \dim_{\mathbb{H}}(X)$. \emph{Random sets} are often almost surely Salem, since pure randomness prevents clustering at frequencies with high probability. But it is \emph{suprisingly difficult} to control the Fourier dimension of sets when one introduces \emph{structure} to sets, which may introduce subtle clustering near particular frequencies. In particular, the relation between \emph{nonlinear structure} and Fourier dimension is especially difficult to understand. For instance, even determining the Fourier dimension of the set $\{ x + x^2 : x \in C \}$, where $C$ is the Cantor set, remains an open problem.

\vspace{0.1cm}

There are many constructions of sets with large \emph{Hausdorff dimension} avoiding zeroes of nonlinear functions $f$ (e.g. M\'{a}th\'{e}, 2017 or Fraser and Pramanik, 2018), but most constructions of large Salem sets avoiding functions $f$ focus on linear functions $f$ (e.g. Shmerkin, 2015 or Liang and Pramanik, 2020). Here we describe techniques to deal with the introduction of nonlinear structure to a random set via \emph{probabilistic concentration inequalities}, and \emph{oscillatory integrals}.}



% BEGIN OF PAGE 2.
% REMOVE THE FOLLOWING GARBAGE AND TYPE IN YOUR MATERIAL.

\end{multicols}}}\\


% END OF PAGE 2.


%%%%%%%%%%%%%%%%%%%%%%%%%%%%%%%%%%%%%%%%%%%%%%%%%%%%%%%%%%%%%%%%%%%%%%%%%%
%%%%%%%%%%%%%%%%%%%%%%%%%%%%%%%%%%%%%%%%%%%%%%%%%%%%%%%%%%%%%%%%%%%%%%%%%%
\fbox{
\parbox{0.9972\boxwidth}{
\setlength{\fboxsep}{0.005\textwidth}
\setlength{\fboxrule}{0.00125\textwidth}

\raggedcolumns
\begin{multicols}{2}
%--------------------------------------------------------------------------
% Replace the text "part 3" with something meaningful or erase the text.
\newpart{Constructing Salem Sets}

% BEGIN OF PAGE 3.
% REMOVE THE FOLLOWING GARBAGE AND TYPE IN YOUR MATERIAL.
\large{
We construct sets avoiding zeroes via a Cantor-type construction, i.e. iteratively defining sets $\{ X_k \}$ by dissecting cubes (intervals if $d = 1$) at each stage into smaller cubes, and keeping a union of smaller cubes chosen carefully so they have \emph{good Fourier analytic properties}, and \emph{avoid a discretized version of the pattern}. \vspace{0.1cm}}

%\begin{Figure}
\begin{wrapfigure}{L}{0.35\linewidth}
\begin{tikzpicture}[scale=1.35]
%\draw[help lines] (0,0) grid (6,3);

\draw[ultra thick] (0.5,2.5) -- (2.5,2.5);
\draw[ultra thick] (3,2.5) -- (5,2.5);

\draw[ultra thick] (0.7,2) -- (1.4,2);
\draw[ultra thick] (1.7,2) -- (2.4,2);
\draw[ultra thick] (3.4,2) -- (4.1,2);
\draw[ultra thick] (4.3,2) -- (5,2);

\draw[ultra thick] (0.9,1.5) -- (1,1.5);
\draw[ultra thick] (1.1,1.5) -- (1.2,1.5);
\draw[ultra thick] (1.9,1.5) -- (2.1,1.5);
\draw[ultra thick] (3.4,1.5) -- (3.5,1.5);
\draw[ultra thick] (3.6,1.5) -- (3.7,1.5);
\draw[ultra thick] (3.8,1.5) -- (3.9,1.5);
\draw[ultra thick] (4.5,1.5) -- (4.7,1.5);
\draw[ultra thick] (4.8,1.5) -- (4.9,1.5);

\node at (5.5,2.5) {$X_1$};
\node at (5.5,2) {$X_2$};
\node at (5.5,1.5) {$X_3$};
\node at (5.5,0.5) {$X$};

\draw[ultra thick, loosely dotted, line width=1mm] (0.95,1.35) -- (0.95,0.65);
\draw[ultra thick, loosely dotted, line width=1mm] (1.15,1.35) -- (1.15,0.65);

\draw[ultra thick, loosely dotted, line width=1mm] (1.92,1.35) -- (1.97,0.65);
\draw[ultra thick, loosely dotted, line width=1mm] (2.08,1.35) -- (2.03,0.65);

\draw[ultra thick, loosely dotted, line width=1mm] (3.45,1.35) -- (3.45,0.65);
\draw[ultra thick, loosely dotted, line width=1mm] (3.65,1.35) -- (3.65,0.65);
\draw[ultra thick, loosely dotted, line width=1mm] (3.85,1.35) -- (3.85,0.65);

\draw[ultra thick, loosely dotted, line width=1mm] (4.52,1.35) -- (4.58,0.65);
\draw[ultra thick, loosely dotted, line width=1mm] (4.68,1.35) -- (4.62,0.65);

\draw[ultra thick, loosely dotted, line width=1mm] (4.85,1.35) -- (4.85,0.65);

\draw[ultra thick, loosely dotted, line width=1mm] (5.5,1.25) -- (5.5,0.75);

\fill (0.95,0.5) circle [radius=0.5pt];
\fill (1.15,0.5) circle [radius=0.5pt];
\fill (2,0.5) circle [radius=1pt];
\fill (3.45,0.5) circle [radius=0.5pt];
\fill (3.65,0.5) circle [radius=0.5pt];
\fill (3.85,0.5) circle [radius=0.5pt];
\fill (4.6,0.5) circle [radius=1pt];
\fill (4.85,0.5) circle [radius=0.5pt];
\end{tikzpicture}
\end{wrapfigure}

\large{If, at each stage of the construction, we choose a large $N > 0$, subdivide each cube into smaller sidelength $1/N^{1/s}$ cubes, and take $N$ of these cubes from each of the original intervals, then iteration for a fixed $s$ should yield a set with Hausdorff dimension $s$. Since the subdivided set `lives at a scale $1/N^{1/s}$', the uncertainty principle tells us to care about frequencies $|\xi| \lesssim N^{1/s}$. And indeed, obtaining a Salem set reduces to verifying the following exponential sum square root cancellation bound can be obtained:

\begin{lem}
    For arbitrarily large $N > 0$, there exists an $N$ element subset $S$ of $[0,1]^d$ such that for any $\xi \in \mathbb{Z}^d$ with $|\xi| \lesssim N^{1/s}$
    % N = r^{-s}
    % r = N^{-1/s}
    \[ \left| \frac{1}{N} \sum_{x \in S} e^{2 \pi i \xi \cdot x} \right| \lessapprox N^{-1/2}, \]
    %
    and for distinct $y_1,\dots,y_n \in S$, $|f(y_1,\dots,y_n)| \gtrsim N^{-1/s}$ ($S$ contains no `near zeroes').
\end{lem}

Let us illustrate how the problem becomes harder as we \emph{increase $s$}, i.e. we try and construct larger Salem sets. To do this, pick $10 N$ points $\{ x_1, \dots, x_{10N} \}$ uniformly at random from $[0,1]^d$. There are roughly $O(N^n)$ tuples $(y_1,\dots,y_n)$, where each $y_i$ is taken from the points $x_i$. Each tuple has probability $O(N^{-i/s})$ of forming a near zero of $f$, since the zero set of $f$ is a $dn - i$ dimensional hypersurface in $(\mathbb{R}^n)^d$. Thus we expect there to be roughly $O(N^{n-i/s})$ tuples formed from the points $\{ x_i \}$ which give near zeroes.
%
\begin{itemize}
    \item If $s \leq i/n$, we expect no tuples will give near zeroes, so setting $S = \{ x_1, \dots, x_N \}$ will satisfy the contraints of the Lemma with high probability. Easy!

    \item If $s > i/n$, we expect there to be tuples giving near zeroes. So we let $S$ be the set of points from the set $\{ x_i \}$ which remain after \emph{pruning}, i.e. after removing any point $x_i$ which equals $y_n$ for some tuple $(y_1,\dots,y_n)$ forming a near zero of $f$. If $s \leq i/(n-1)$, then we will prune at most $O(N^{n-i/s}) \ll 10 N$ points, which means we can still guarantee $S$ contains $N$ points. For $s > i/(n-1)$, we cannot guarantee $S$ contains any points, so $i/(n-1)$ is the limiting dimension we can expect.
\end{itemize}
%
For $s \leq i/n$, the selection process above is completely random, and so the square root cancellation property is almost automatic. But the pruning we must perform for $s > i/n$ is \emph{structured}, i.e. it removes points clustered near zeroes of $f$, which may cause subtle problems with the Fourier dimension / square root cancellation.

}


% END OF PAGE 3.

\columnbreak

\newpart{Dealing With Pruning}

\large{
Random collections of points satisfy square root cancellation -- it is the pruning which makes the required Lemma difficult to prove. In other words, it suffices to prove the following `pruning inequality'
    %
    \[ \left| \frac{1}{N} \sum_{x_k\ \text{pruned}} e^{2 \pi i \xi \cdot x_k} \right| \lessapprox N^{-1/2}. \]

\vspace{0.1cm}

For $s \leq i/(n-1/2)$, we can guarantee $O(N^{n-i/s}) = O(N^{1/2})$ points have been pruned, so the pruning inequality follows trivially from the triangle inequality. For $s > i/(n-1/2)$ we work harder. Let us now make the assumption that $f(x) = x^n - g(x^1,\dots,x^{n-1})$ as in Theorem 2, and that $i = n$.

\vspace{0.1cm}

The left hand side of the pruning bound can be viewed as a very nonlinear function $F_\xi(x) = F_\xi(x_1,\dots,x_{10N})$ of the initial uniformly random points chosen. The theory of \emph{probabilistic concentration inequalities} gives various tools guaranteeing that $|F_\xi(x) - \mathbb{E}[F_\xi(x)]|$ is bounded with high probability provided the maximum `influence' of each variable $x_i$ on $F$ is not too large. Since we remove the points corresponding to the last coordinate of $(y_1,\dots,y_n)$, these points have `a little too much influence' relative to the other points, but this can be dealt with because these variables are `linear' in $f$ (because of the extra structure assumed in Theorem 2), so we obtain $|F_\xi(x) - \mathbb{E}[F_\xi(x)]| \lessapprox N^{-1/2}$ with high probability for $s \leq d/(n-1)$.

\vspace{0.1cm}

Finally, we can use some inclusion-exclusion bounds to reduce the study of $\mathbb{E}[F_\xi(x)]$ to an oscillatory integral and apply non-stationary phase. But the inclusion-exclusion bounds obtained only work for $s \leq d/(n-3/4)$ -- one must understand the `exclusion' in more detail past this range, which is why Theorem 2 only obtains a Salem set of dimension $d/(n-3/4)$ rather than dimension $d/(n-1)$.
}

\vspace{0.2cm}

%----------------------------------------------------------------------------
% Replace the text "part 4" with something meaningful or erase the text.
\newpart{What's Next}

\large{Here are some problems to improve the results described in this poster:

\begin{itemize}
    \item Can one improve the inclusion-exclusion bounds in the analysis of pruned sets to improve the dimension $d/(n-3/4)$ in Theorem 2 to $d/(n-1)$, the best possible bound we can expect purely via pruning random points.

    \item Is there a nontrivial concentration argument for general $f$ as in Theorem 1?

    \item Can we consider `fractal domain' avoidance problems: Given a Salem set $S$ and a nice function $f: S^n \to \mathbb{R}$, it is possible to construct a large Salem subset $X \subset S$ avoiding zeroes of $f$? In the simplest nontrivial example, $S$ could be a curved hypersurface.

    \item Can we use modern `square root cancellation methods', e.g. decoupling, to construct more structured Salem sets?
\end{itemize}
%
%{\bf
%Contact me via email at \textcolor{blue}{jcdenson@wisc.edu} so we can correspond / arrange a time to talk online!}

%%%%%%%%%%%%%%%%%%%%%%%%%%%%%%%%%%%%%%%%%%%%%%
\vspace{0.5cm}}

% REPLACE THE TEXT BELOW WITH YOUR INSTITUTE AND YOUR EMAIL-ADDRESS. DO NOT ERASE THE YEAR.
% EXAMPLE:
% 2013 Mathematisches Institut
% ingrid.mustermann@imaginary-address.de

%%%%%%%%%%%%%%%%%%%%%%%%%%%%%%%%%%%%%%%%%%%%%%%

\end{multicols}}}

\end{document}
