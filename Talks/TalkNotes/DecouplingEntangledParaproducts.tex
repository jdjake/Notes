%LaTeX template for Summer Schools
%
% !!!!!!!!!!!!   INSTRUCTIONS  !!!!!!!!!!!!!!

% 0) Submit your work in tex format (not pdf). It will be pasted into the proceedings file.
%
% 1) please name your file yourlastname.tex, e.g. thiele.tex.
%
% As several files will be concatenated, please 
% use some discipline as to the following:
%
% 2) Whenever you use \label{} to get automatic cross references
% through \ref{} (this is the preferred option for cross references)
% please add your initials to the label such as 
% \label{SLEct} for the Stochastic Loewner equation
% with initials of author Christoph Thiele
% Same with other citations such as in bibitem.
%
% 3) Please STRONGLY avoid using \def or \newcommand unless really necessary.
% We do have macros for black board bold. If you use your favorite
% macros while preparing your summary, please expand them
% (replace by the original definition) everywhere in your file.
% This will save me the work of doing the very same thing.
% Thank you! If you have to use \def, please also add your
% initials to the definition.
%
% 4) if you want to compile the header of the document, 
% uncomment remove the corresponding paragraph signs below
%
% 5) There is a sample lecture below. For the header 
% it is best to keep most of the
% commands and just change the name, title, text. etc
%
% 6) Please follow the conventions below in terms of capitalization of 
% headings etc:
% Only the beginning of a sentence and names are capitalized.


\documentclass[12pt]{article}
\usepackage{amssymb,amsmath,amsthm}
\usepackage{esint}
\usepackage{mathrsfs}
\usepackage{mathtools}
\usepackage{comment}
\usepackage{bbm,dsfont}

\newtheorem{theorem}{Theorem}
\newtheorem{lemma}[theorem]{Lemma}
\newtheorem{corollary}[theorem]{Corollary}
\newtheorem{conjecture}[theorem]{Conjecture}
\newtheorem{definition}[theorem]{Definition}
\newtheorem{example}[theorem]{Example}
\newtheorem{remark}[theorem]{Remark}
\newtheorem{proposition}[theorem]{Proposition}

\newcommand{\talktitle}[1]{\section{#1}}
\newcommand{\talkafter}[1]{\textbf{After #1} \addcontentsline{toc}{subsection}{after #1}}
\newcommand{\talkspeaker}[2]{\begin{center}
\textit{A summary written by #1}
\end{center}
\addcontentsline{toc}{subsection}{#1, #2}
}

\usepackage{graphicx}
\usepackage[utf8]{inputenc}
\usepackage[T1]{fontenc}
\usepackage[hidelinks]{hyperref}


\newcommand*{\Z}{\mathbb{Z}}
\newcommand*{\Q}{\mathbb{Q}}
\def\C{\mathbb{C}}
\newcommand*{\N}{\mathbb{N}}
\newcommand*{\R}{\mathbb{R}}

\newcommand*{\F}{\mathbf{F}}

% Absolute values and norms using mathtools.
% \\[lr][vV]ert produces correct spacing, as opposed to | and \|.
\DeclarePairedDelimiter\abs{\lvert}{\rvert}
\DeclarePairedDelimiter\norm{\lVert}{\rVert}

\title{Decoupling Entangled Paraproducts}

% \author{Summer School, Kopp}
% \thanks{supported by Hausdorff Center for Mathematics, Bonn}

% \date{October 2022}



\begin{document}

% \maketitle

% {
% \center{Organizers:}
% \center{
% Christoph Thiele, Universit\"at Bonn}
% \center{
% Olli Saari, Universit\"at Bonn}
% \center{$\ $}
% }
% \newpage
% \tableofcontents

% \newpage


\talktitle{Detangling Entangled Paraproducts}
\talkafter{P. Durcik \cite{dujdjf}}
\talkspeaker{Jacob Denson and Jacob Fiedler}{University of Wisconsin, Madison}

\setcounter{equation}{0}
\setcounter{theorem}{0}

\begin{abstract}
{ After Durcik, we introduce and summarize the proof of an $L^4$ estimate for a quadrilinear form of which the "twisted paraproduct" considered by Demeter and Thiele, and then Kova\u{c} is a special case.}
\end{abstract}

\subsection{Introduction}

Take four functions $F_1, F_2, F_3,$ and $F_4$ on $\R^2$, and `entangle them', forming the function
%
\begin{equation}\label{JDJFentangledproduct}
\F(x, x^\prime, y, y^\prime):=F_1(x, y)F_2(x^\prime, y)F_3(x, y^\prime)F_4(x^\prime, y^\prime)
\end{equation}
%
We will be interested in the following quadrilinear form: 

\begin{equation}\label{JDJFquadform}
\Lambda(F_1, F_2, F_3, F_4):=\int_{\R^2} \hat{\F}(\xi, -\xi, \eta, -\eta) m(\xi, \eta) d\xi d\eta
\end{equation}

\noindent where $m$ is a bounded and smooth (away from the origin) function on $\mathbb{R}^2$ such that $\vert \partial^\alpha m(\xi, \eta) \vert \lesssim (\vert\xi\vert + \vert\eta\vert)^{-\vert\alpha\vert}$ for all multi-indices up to a large finite order. The main theorem of Durcik's paper is the following $L^4$ bound:

\begin{theorem}\label{JDJFmaintheorem}
The quadrilinear form $\Lambda$ satisfies the estimate
\begin{equation}\label{JDJFmainbound}
\vert\Lambda(F_1, F_2, F_3, F_4)\vert \lesssim \norm{F_1}_{L^4(\mathbb{R}^2)}\norm{F_2}_{L^4(\mathbb{R}^2)}\norm{F_3}_{L^4(\mathbb{R}^2)}\norm{F_4}_{L^4(\mathbb{R}^2)}
\end{equation}
\end{theorem}

A special case of this quadrilinear form is the so-called ``twisted paraproduct'' introduced by Demeter and Thiele and defined as follows:

\begin{equation}\label{jdjfparaproduct}
T(F_1, F_2, F_3):= \Lambda(F_1, F_2, F_3, 1)
\end{equation}

\noindent Kova\u{c} \cite{kjdjf} proved the following bound on $T$:
\begin{equation}\label{jdjfparaproductbound}
\vert T(F_1, F_2, F_3) \vert \lesssim_{(p_j)} \norm{F_1}_{L^{p_1}(\mathbb{R}^2)}\norm{F_2}_{L^{p_2}(\mathbb{R}^2)}\norm{F_3}_{L^{p_3}(\mathbb{R}^2)}
\end{equation}

\noindent for $p_j$ such that $1/p_1 + 1/p_2 + 1/p_3 = 1$ and $2<p_j<\infty$.  
\subsection{Relevance to ergodic theory}

Improved bounds on some of the above transforms would have ramifications for open problems in ergodic theory. Let $(X, \Sigma, m)$ be a probability space and let $T, S: X \rightarrow X$ be two commuting measure-preserving point transformations on $X$. A key problem in ergodic theory is to prove the almost everywhere convergence of the average

\begin{equation}\label{jdjfergodic}
\dfrac{1}{N}\sum\limits_{n=1}^{N}f(T^n(x))g(S^{-n}(x))
\end{equation}

\noindent for all $f, g\in L^\infty(X)$. Using an estimate for a paraproduct, Demeter and Thiele established in \cite{dtjdjf} the convergence of a collection of related averages, including

\begin{equation}\label{jdjfergodic2}
\dfrac{1}{N^2}\sum\limits_{n=1}^{N}\sum\limits_{m=1}^{N}f(T^nS^m(x))g(T^{-n}S^m(x))
\end{equation}

The basic idea is that if one can bound the oscillation of a weighted version of the ergodic averages by $C_J \norm{f}_{L^{p_1}} \norm{g}_{L^{p_2}}$ (where $C_J$ is a term related to the oscillation) this is sufficient to conclude pointwise convergence on a full measure subset of $X$. In \cite{dejdjf}, Demeter details this argument in the course of reproving and extending a result of Bourgain on the convergence of (\ref{jdjfergodic}) when $S$ is a power of $T$. In this case, the desired inequality is
\begin{equation*}
\norm{\left(\sum\limits_{j=1}^{J-1}\sup_{ u_j\leq k<u_{j+1}}\vert W_k(f, g)(x) - W_{u_{j+1}}(f, g)(x)\vert^2\right)^{\frac{1}{2}}}_{L^{1, \infty}(X)}\lesssim J^{\frac{1}{4}}\norm{f}_{L^{2}} \norm{g}_{L^{2}}
\end{equation*}

\noindent where the bound is uniform in $J$ and all finite sequences $U_1, ..., U_J$, and where 

\begin{equation*}
W_k(f, g)(x)=\sum\limits_{n\in\mathbb{Z}}w_{n, k}f(T^n(x))g(T^{-n}(x)) 
\end{equation*}

Connecting bounds of this type to the types of estimates in this paper requires invoking a transfer principle. Equipped with the right inequality, one can consider functions on $\mathbb{R}^2$ which are constant on all the integer lattice squares $(n, n+1)\times(m, m+1)$, essentially functions on $\mathbb{Z}^2$. To complete the transfer to $X$, use the functions $F$ on $\mathbb{Z}^2$ which are of the form $F(n, m)=f(T^n S^mx)$ for some $x\in X$. For further reference, \cite{dlttjdjf} details the transfer of a bound on a maximal average to a bound on an ergodic average.

The salient point is that after the transfer, we have essentially the same upper bound. So, when Demeter and Thiele obtain an oscillation bound for a sum of integrals of the form 
\begin{equation*}
    \int_{\mathbb{R}^2} F_1(x + t, y + s) F_2(x -t, y+s) \Psi_k(t)\Phi_k(t) dtds
\end{equation*}

\noindent it implies the same bound for the oscillation of (\ref{jdjfergodic2}) (note the relationship between the exponents in the ergodic average and the arguments in the above integral). Their bounds in turn are good enough to ensure pointwise convergence almost everywhere, as indicated above. In an analogous manner, a better understanding of bounds for 

\begin{equation*}
    \int_{\mathbb{R}^2} F_1(x + t, y) F_2(x, y+t) dtds
\end{equation*}

\noindent would improve the understanding of the more challenging average (\ref{jdjfergodic}), and bounding this bilinear Hilbert transform is directly related to bounding the "triangular" Hilbert transform defined in (\ref{jdjfparaproduct}). 

\section{Proof Technique}

Recall the statement of Theorem \ref{JDJFmaintheorem}. Spending rescaling symmetries, we may assume that $\| F_1 \|_{L^4}, \dots, \| F_4 \|_{L^4} = 1$, and our goal is to prove that $|\Lambda| \lesssim 1$. The proof has a nice flavor to it, because the main tools are all very important to all harmonic analysts, but used in some novel clever ways:
%
\begin{enumerate}
    \item[(A)] Time-Frequency Analysis, i.e. simultaneous decompositions of functions to localize behaviour in space and frequency.

    \item[(B)] Exploiting cancellation using a `telescoping identity', which for intuition's sake behaves like a multilinear variant of an integration by parts.

    \item[(C)] Applying monotonicity to introduce properties to functions.
\end{enumerate}
%
For intuition's sake, (B) is a multilinear variant of an integration by parts.

Let's begin with technique (A). We may assume without loss of generality in our proof that $\text{supp}(m)$ is contained in a cone $\Gamma = \{ (\xi,\eta): |\xi| \leq 1.001 |\eta| \}$, since symmetry and the triangle inequality then give the general result. Next, we perform a time-frequency decomposition of the multiplier $m$, writing
%
\[ m(\xi,\eta) = \int_0^\infty \int_{-\infty}^\infty \int_{-\infty}^\infty \mu(u,v) \widehat{\varphi}_{t,u}(t\xi)^2 \widehat{\psi}_{t,v}(t\eta)^2\; dt/t\; du\; dv, \]
%
where $\varphi_{t,u}$ and $\psi_{t,v}$ are real-valued functions, with $\widehat{\varphi}_{t,u}$ concentrated on the set $\{ \xi : |\xi| \lesssim 1 \}$, and $\varphi_{t,u}$ concentrated on $\{ x : |x - u| \lesssim 1 \}$, and where $\widehat{\psi}_{t,v}$ is concentrated on the set $\{ \eta : |\eta| \sim 1 \}$, and $\psi_{t,v}$ is concentrated on $\{ y : |y - v| \lesssim 1 \}$. The squares in the exponent here are irrelevant to the existence of the decomposition, but will be necessary to get the convolution representation \eqref{FourSpatialSideJD} later on. The symbol properties of $m$ imply that the magnitude of $\mu(u,v)$ decays rapidly as $|u|, |v| \to \infty$. The result will therefore follow if we can obtain bounds on
%
\[ \int \Lambda_{t,u,v}\; dt/t \quad\text{uniformly in $u$ and $v$}, \]
%
where
%
\[ \Lambda_{t,u,v} = \int \widehat{\varphi}_{t,u}^2(\xi) \widehat{\psi}_{t,v}(\xi)^2 \widehat{\F}(\xi,-\xi,\eta,-\eta)\; d\xi\; d\eta. \]
%
The values of $u$ and $v$ are not too important to the discussion of the problem, so let us write $\Lambda_t$ for $\Lambda_{u,v}$, $\psi_t$ for $\psi_{t,u,v}$, and $\Lambda_t = \Lambda_{t,1,1}$. For intuition's sake, one can think of $\phi_t$ like a Gaussian supported in a neighborhood of $1$, and $\psi_t$ like a modulated Gaussian supported in a neighborhood of $1$.

% and since it is representative of the general analysis, let us focus on the components $m_{t,1,1}$ of the decomposition, which should behave like a Gaussian, i.e. $m_{t,1,1}(\xi,\eta)$ should behave like $e^{2 \pi i (\xi + \eta)} e^{- (t \xi)^2} ( e^{- ( t \eta - 1)^2} + e^{- ( t \eta + 1 )^2} )$. Let us write $m_t$ for $m_{t,1,1}$.

Inverting the Fourier transform, writing $[f]_t(x) = t^{-1} f(x/t)$ for the $L^1$ dilation of a function $f$, and $g^-(t) = g(-t)$ for the reflection, we can write $\Lambda_t$ as a `twisted convolution operator', i.e.
%
\begin{equation}
    \label{FourSpatialSideJD}
    \Lambda_t = \int \F(x,y,x',y') [\varphi_t]_t(\tilde{x} - x) [\varphi_t]_t^-(\tilde{x} - x') [\psi_t]_t(\tilde{y} - y) [\psi_t]_t^-(\tilde{y} - y').
\end{equation}
%
Let us write this quantity as $\Lambda_{[\varphi_t]_t, [\varphi_t]_t^-, [\psi_t]_t, [\psi_t]_t^-}(F_1,F_2,F_3,F_4)$, where
%
\[ \Lambda_{\phi_1,\phi_2,\phi_3,\phi_4} = \int \F(x,y,x',y') \phi_1(\tilde{x} - x) \phi_2(\tilde{x} - x') \phi_3(\tilde{y} - y) \phi_4(\tilde{y} - y').  \]
%
The terms involving $\psi$ are where the significant cancellation occurs in the integral, with $\phi$ providing little cancellation, and so we start by applying the triangle inequality, writing
%
\begin{equation} \label{JDJFLambdaCalculation}
\begin{aligned}
    |\Lambda_t| &\leq \int \left| \int F_1(x,y) F_2(x',y) [\psi_t]_t(\tilde{y} - y)\; dy \right|\\
    &\quad\quad\quad\quad \left| \int F_3(x',y') F_4(x,y') [\psi_t]_t(y' - \tilde{y})]\; dy' \right|\\
    &\quad\quad\quad\quad\quad |[\varphi_t]_t(x - \tilde{x})| \; |[\varphi_t]_t(\tilde{x} - x')|\; dx\; dx'\; d\tilde{x}\; d\tilde{y}.
\end{aligned}
\end{equation}
%
In the worst case, the two integrals in the absolute values of \eqref{JDJFLambdaCalculation} in could be equal to one another (e.g. if $F_1 = F_3$, $F_2 = F_4$ and $|v| \lesssim 1$), which means Cauchy-Schwartz in $\tilde{y}$ is likely to be efficient, and expanding out the squares that are obtained by Cauchy-Schwartz, we obtain.
%
\begin{align*}
    |\Lambda_t| &\leq \Lambda_{[\psi_t]_t, [\psi_t]_t, |[\varphi_t]_t, [\varphi_t]_t^-}(F_1,F_2,F_2,F_1)^{1/2}\\
    &\quad\quad \Lambda_{[\psi_t]_t^-, [\psi_t]_t^-, |[\varphi_t]_t|, |[\varphi_t]_t^-|}(F_3,F_4,F_4,F_3)^{1/2}.
\end{align*}
%
Notice that expanding out the square allows us to remove the absolute values we introduced via the triangle inequality. By symmetry, it suffices to bound $\Lambda_{[\psi_t]_t, [\psi_t]_t, |[\varphi_t]_t|, |[\varphi_t]_t^-|}(F_1,F_2,F_2,F_1)$. We will now apply Method (B), the telescoping identity given in Lemma 3 of \cite{dujdjf}, which exploits the oscillation of $[\psi_t]_t$.

\vspace{0.8em}

\noindent \textbf{Lemma 3 of \cite{dujdjf}.}
    If $-t\partial_t|\widehat{\rho}_i|^2 = |\widehat{\sigma}_i(t \tau)|^2$ for $i \in \{ 1, 2 \}$, and $c = |\widehat{\rho}_1(0)|^2 |\widehat{\rho}_2(0)|^2$, and for any choices of $F_1,F_2,F_3$, and $F_4$,
    %
    \[ \Lambda_{\sigma_1,\sigma_1,\rho_2,\rho_2} = c \int_{\R^2} F_1 F_2 F_3 F_4 - \Lambda_{\rho_1, \rho_1, \sigma_2, \sigma_2}. \]

The proof given in \cite{dujdjf} is very accessible, so for purposes of brevity we refer to reading that Lemma directly from the paper. This Lemma works like integration by parts, in the sense that we `antidifferentiate' $\sigma$, at the cost of 'differentiating' $\rho$, negating the integral, and introducing the `boundary term' $c \int F_1F_2F_3F_4$. Abusing notation, we will refer to pairs of functions satisfying the identity $- t \partial_t |\widehat{\rho}_i|^2 = |\widehat{\sigma}_i(t \tau)|^2$ as `derivatives' and `antiderivatives' respectively.

We wish to apply the Lemma to $\Lambda_{[\psi_t]_t, [\psi_t]_t, |[\varphi_t]_t|, |[\varphi_t]_t|^-}(F_1,F_2,F_1,F_2)$, except that $|[\varphi_t]_t|$ does not equal $|[\varphi_t]_t|^-$. But we can fix this by employing the monotonicity of $\Lambda_{a,b,c,d}(\phi_1,\phi_2,\phi_3,\phi_4)$ with respect to $\phi_3$ and $\phi_4$, noting that
%
\[ \Lambda_{[\psi_t]_t, [\psi_t]_t, |[\varphi_t]_t|, |[\varphi_t]_t|^-}(F_1,F_2,F_2,F_1) \lesssim \Lambda_{[\psi_t]_t, [\psi_t]_t, \Phi, \Phi}(F_1,F_2,F_2,F_1), \]
%
where, roughly speaking, $\Phi$ is a sum of two Gaussians, supported at $1$ and $-1$ respectively, chosen so that $|[\varphi_t]_t|, |[\varphi_t]_t^-| \lesssim \Phi$. If we let $\phi$ denote the `derivative' of $\Phi$, and $[\Psi_t]_t$ the `antiderivative' of $[\psi_t]_t$, then we obtain that
%
\[ \Lambda_{[\psi_t]_t, [\psi_t]_t, \Phi, \Phi} = \widehat{\phi}(0)^2 \int_{\R^2} F_1^2 F_2^2 - \Lambda_{[\Psi_t]_t, [\Psi_t]_t, \phi, \phi}(F_1,F_2,F_1,F_2). \]
%
We can choose $\psi_t$ such that $\Psi_t$ is uniformly compactly supported. By Cauchy-Schwartz, $\int F_1^2 F_2^2 \leq 1$, and so
%
\[ \Lambda_{[\psi_t]_t, [\psi_t]_t, \Phi, \Phi} \lesssim 1 + |\Lambda_{\Psi_t, \Psi_t, \phi, \phi}(F_1,F_2,F_2,F_1)|. \]
%
It thus suffices to estimate $\Lambda_{\Psi_t, \Psi_t, \phi, \phi}(F_1,F_2,F_2,F_1)$. But now we have used the oscillation in $\psi$, we can perform the mirror of our starting Cauchy-Schwartz, in the $x$ and $x'$ variables now instead of the $y$ and $y'$ terms, i.e. writing
%
\begin{align*}
    \Lambda_{\Psi_t, \Psi_t, \phi, \phi}(F_1,F_2,F_1,F_2) &\leq \Lambda_{\Psi_t,\Psi_t,|\phi|,|\phi|}(F_1,F_1,F_1,F_1)^{1/2} \Lambda_{\Psi_t, \Psi_t, |\phi|, |\phi|}(F_2,F_2,F_2,F_2)^{1/2}.
\end{align*}
%
By symmetry, we again focus on $\Lambda_{\Psi_t,\Psi_t,|\phi|,|\phi|}(F_1,F_1,F_1,F_1)$. We want to again apply integration by parts, so we apply monotonicity to replace $\phi$ with a symmetric Gaussian $\tilde{\Phi}$. But now integration by parts implies
%
\[ \Lambda_{\Psi_t,\Psi_t,\Phi, \Phi}(F_1,F_1,F_1,F_1) = - \Lambda_{\psi_t,\psi_t,\phi,\phi}(F_1,F_1,F_1,F_1) + \int_{\R^2} F_1^4. \]
%
But the two l

\begin{thebibliography}{03}

\bibitem[De]{dejdjf} Demeter, C., \emph{ Pointwise convergence of the ergodic bilinear Hilbert transform.} Illinois J. Math. 51 (4) 1123 - 1158, Winter 2007.

\bibitem[DLTT]{dlttjdjf}Demeter, C., Lacey, M.,Tao, T., Thiele, C., \emph{ Breaking the duality in the return times theorem} Duke
Math. J. 143 (2008), no. 2, 281-355.

\bibitem[DT]{dtjdjf} Demeter, C., Thiele, C.,\emph{ , On the two-dimensional bilinear Hilbert transform}, Amer. J. Math., 132
(2010), no. 1, 201-256.

\bibitem[Du]{dujdjf} Durcik, P.,
\emph{ An $L^4$ estimate for a singular entangled quadrilinear form.} 
Mathematical Research Letters, 22(5), 1317-
1332.

\bibitem[K]{kjdjf} V. Kova\u{c},\emph{ Boundedness of the twisted paraproduct.} Rev. Mat. Iberoam., 28 (2012), no. 4, 1143-1164.




\end{thebibliography}

\noindent \textsc{Jacob Denson and Jacob Fiedler, University of Wisconsin-Madison}\\
\textit{email:} \texttt{jcdenson@wisc.edu} and \texttt{jbfiedler2@wisc.edu} respectively.

%\newpage

\end{document}