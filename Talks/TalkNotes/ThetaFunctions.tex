\documentclass{article}

\usepackage{amsmath}
\usepackage{amssymb}
\usepackage{amsthm}
\usepackage{esint}
\usepackage{MnSymbol}

\theoremstyle{plain}
\newtheorem{theorem}{Theorem}
\newtheorem{lemma}[theorem]{Lemma}
\newtheorem{corollary}[theorem]{Corollary}
\newtheorem{prop}[theorem]{Proposition}

\theoremstyle{remark}
\newtheorem*{example}{Example}
\newtheorem*{remark}{Remark}

\theoremstyle{definition}
\newtheorem*{defi}{Definition}
\newenvironment{definition}
    {\begin{samepage}\begin{framed}\begin{defi}}
    {\end{defi}\end{framed}\end{samepage}}

\title{Theta Functions}
\author{Jacob Denson}
\date{\today}

\begin{document}

\maketitle

In this talk, we discuss a family of modular forms with some important applications to number theory, known as Theta functions. The most basic is the {\bf Jacobi Theta function}, a holomorphic function defined by the $q$ series
%
\[ \theta(\tau) = \sum_{n = -\infty}^\infty e^{\pi i n^2 \tau} \]
%
It arises both as a fundamental solution to the heat equation on the real line, and as a generating function representing certain number theory problems involving squares of integers. It is a special case of the more general {\bf Theta function}
%
\[ \Theta(z,\tau) = \sum_{n = -\infty}^\infty e^{\pi i n^2 \tau} e^{2 \pi i n z} \]
%
For each disk in $z$ and proper upper half plane, the series converges uniformly, and so we get a holomorphic function entire in $z$, and holomorphic for $\tau \in \mathbf{H}$. For convention as is done in the study of these classical functions like the theta series, even though we have used $q$ to denote $e^{2 \pi i \tau}$, in this talk we denote $q$ by $e^{\pi i \tau}$. Thus $\smash{\theta(\tau) = \sum q^{n^2}}$.

\begin{theorem}
    $\Theta(\cdot|\tau)$ is elliptic in $z$ with period $1$ and `quasiperiod' $\tau$.
\end{theorem}
\begin{proof}
    It is easy to see that $\Theta(z + 1|\tau) = \Theta(z|\tau)$, so the function is periodic. A manipulation gives
%
\begin{align*}
    \Theta(z+ \tau|\tau) &= \sum_{n = -\infty}^\infty e^{\pi i (n^2 + 2n) \tau} e^{2 \pi i n z}\\
    &= e^{-\pi i \tau} \sum_{n = -\infty}^\infty e^{\pi i (n+1)^2 \tau} e^{2 \pi i n z} = q^{-1} e^{-2 \pi i z} \Theta(z|\tau)
\end{align*}
%
So we have a kind of periodicity with an added term.
\end{proof}

\begin{theorem}
    $\Theta$ has modular symmetry in $\tau$.
\end{theorem}
\begin{proof}
    We have $\Theta(z|\tau + 2) = \Theta(z|\tau)$. To obtain an interesting modular transformation rule, we apply Poisson summation. Consider the function
%
\[ f(y) = e^{- \pi t(y + x)^2} \]
%
Then $(\delta_{-x} \circ M_{t^{1/2}})(e^{- \pi y^2}) = f$, and so since $e^{- \pi y^2}$ is it's own Fourier transform,
%
\[ \widehat{f}(\xi) = \frac{e^{- 2 \pi i \xi x}}{t^{1/2}} e^{- \pi \xi^2/t} \]
%
Applying Poisson's summation formula, we conclude that
%
\[ e^{- \pi t x^2} \sum_{n = -\infty}^\infty e^{- \pi t n^2} e^{- 2 \pi n t x} = \sum_{n = -\infty}^\infty e^{- \pi t(n + x)^2} = \frac{1}{t^{1/2}} \sum_{n = -\infty}^\infty e^{2 \pi i n x} e^{-\pi n^2/t} \]
%
We can rearrange this formula to give something interesting about the Theta function. The left hand side is $e^{- \pi t x^2} \Theta(xit|it)$ and the right hand side is $\Theta(x|i/t)/t^{1/2}$. Writing $z = x$ and $\tau = it$, then the equation can be rearranged to read
%
\[ \Theta(z|-1/\tau) = \sqrt{-i\tau} \cdot e^{\pi i \tau z^2} \cdot \Theta(z \tau | \tau) \]
%
Both sides of this equation are holomorphic, and we have shown the equation holds for a non isolated set of values $z$ and $\tau$. Because of this, the equation actually holds {\it everywhere}.
\end{proof}

\begin{corollary} $\theta(-1/\tau) = \Theta(0|-1/\tau) = \sqrt{-i\tau} \cdot \theta(\tau)$
\end{corollary}

Thus $\theta$ has a modular character with respect to the subgroup of $\Gamma$ generated by the transformations $T: \tau \mapsto \tau + 2$ and $S: z \mapsto -1/\tau$. We shall find that these generate a family of symmetries whose fundamental domain has two, non-equivalent cusps, one at $\infty$, and the other at $1$. We have already given the $q$ series expansion around $\infty$. Using $\Theta$, we can actually cheat out an expansion for $\theta$ near one.

\begin{theorem} $\theta(1 - 1/\tau) \sim 2 \sqrt{-i\tau} e^{i \pi \tau/4}$ as $\text{Im}(\tau) \to \infty$.
\end{theorem}
\begin{proof}
We calculate that
%
\[ \theta(1 + 2 \tau) = \sum_{n = -\infty}^\infty (-1)^{n^2} e^{i \pi n^2 \tau} = \Theta(1/2|\tau) \]
%
So
%
\begin{align*}
    \theta(1 - 1/\tau) &= \Theta(1/2|-1/\tau) = \sqrt{-i\tau} \cdot e^{\pi i \tau/4} \cdot \Theta(\tau/2 | \tau)\\
    &= \sqrt{-i\tau} \cdot \sum_{n = -\infty}^\infty e^{\pi i \tau(n^2 + n + 1/4)} = \sqrt{-i\tau} \cdot \sum_{n = -\infty}^\infty e^{\pi i \tau(n + 1/2)^2}
\end{align*}
%
and $n = 0$ and $n = -1$ give the dominant terms as $\text{Im}(\tau) \to \infty$.
\end{proof}

\section{The Modular Character}

The transformations $T$ and $S$ generate a subgroup of $\Gamma$, which correspond to a family of modular symmetries for the theta function.

\begin{theorem}
    $T$ and $S$ generate the group
%
\[ \Gamma_2(2) = \left\{ z \mapsto \frac{az + b}{cz + d}: a \equiv d\ (\text{mod}\ 2), b \equiv c\ (\text{mod}\ 2), a \not \equiv b\ (\text{mod}\ 2) \right\} \]
%
which has index three subgroup of $\Gamma$.
\end{theorem}
\begin{proof}
One can follow essentially the same proof for $\Gamma$ to show the first property. If we consider the reduction map $R: SL_2(\mathbf{Z}) \to SL_2(\mathbf{Z}_2)$, then
%
\[ \Gamma_2(2) = R^{-1} \left( \begin{pmatrix} 1 & 0 \\ 0 & 1 \end{pmatrix}, \begin{pmatrix} 0 & 1 \\ 1 & 0 \end{pmatrix} \right) \]
%
Since $GL_2(\mathbf{Z}_2) = SL_2(\mathbf{Z}_2)$ has $6$ elements, $\Gamma_2(2)$ is the inverse of an index 3 subgroup of $SL_2(\mathbf{Z}_2)$. In particular, this means $\Gamma_2(2)$ is also an index 3 subgroup of $\Gamma$.
\end{proof}

A geometric consequence is that we have two cusps, and three elliptic points, as well as a fundamental domain $D = \{ \tau \in \mathbf{H}: |\text{Re}(\tau)| \leq 1\ |\tau| \geq 1 \}$. In particular, the dimension of $M_k(\Gamma_2(2))$ is bounded above by $k/4 + 1$. We only need the fact that $M_1(\Gamma_2(2))$ is one dimensional, so every modular function with respect to $\Gamma_2(2)$ is constant.

Unfortunately, $\theta$ is not really a modular form over $\Gamma_2(2)$, even of half weight, because of the presence of the factor $\sqrt{i}$ in it's functional equation. One can define modular forms with a `nebentypus coefficient', and prove facts about the dimension of such spaces, which would suffice to classify the behaviour of $\theta^2$, but we prefer to do things in a much more elementary way. Given $f$ and $g$ with $f(-1/\tau) = (-i \tau)^k f(\tau)$ and $g(-1/\tau) = (-i \tau)^k g(\tau)$, we find that $f(-1/\tau)/g(-1/\tau) = f(\tau)/g(\tau)$. If we can prove $f/g$ is holomorphic at cusps, then $f/g \in M_1(\Gamma_2(2))$ must be a constant function, so there is $A$ such that $f(\tau) = Ag(\tau)$. The values at the cusps then dictate the value of $A$.

\section{Sums of Squares}

We let $r_2(N)$ denote the number of ordered pairs of integers $n,m \in \mathbf{Z}^2$ such that $n^2 + m^2 = N$. For instance, $r_2(5) = 8$ since
%
\[ 5 = 1^2 + 2^2 = (-1)^2 + 2^2 = 1^2 + (-2)^2 = (-1)^2 + (-2)^2 \]
%
and each of the four representations here corresponds to two different ordered sums. Since $\theta(z) = \sum q^{n^2}$, $\theta^2(z) = \sum q^{n^2 + m^2}$, so the $N$'th coefficient in the $q$ series expansion of $\theta^2$ is precisely $r_2(N)$.

\begin{theorem}
    For all $N$, we have $r_2(N) = 4[d_1(N) - d_3(N)]$, where $d_1(N)$ denotes the number of divisors of $N$ congruent to one modulo four, and $d_3(N)$ the divisors congruent to three.
\end{theorem}
\begin{proof}
    We reduce the proof to the simple identity
    %
    \[ \theta(\tau)^2 = 2 \sum_{n = -\infty}^\infty \frac{1}{q^n + q^{-n}} \]
    %
    Once this is proved, we have
    %
    \begin{align*}
        2 \sum_{n = -\infty}^\infty \frac{1}{q^n + q^{-n}} &= 1 + 4 \sum_{n = 1}^\infty \frac{q^{n}}{1 + q^{2n}}\\
        &= 1 + 4 \sum_{n = 1}^\infty \frac{q^n}{1 - q^{4n}} - \frac{q^{3n}}{1 - q^{4n}}
    \end{align*}
    %
    Plugging in the identity $(1 - q^{4n})^{-1} = \sum q^{4nm}$, then expanding out the right hand side gives
    %
    \[ \sum \frac{q^{2n}}{1 - q^{4n}} = \sum q^{n(4m + 1)} = \sum d_1(k) q^k\ \ \ \ \sum \frac{q^{3n}}{1 - q^{4n}} = \sum q^{n(4m + 3)} = \sum d_3(N) q^N \]
    %
    Thus provided we are able to prove the initial identity above, we will be done.

    Let us define
    %
    \[ f(\tau) = 2 \sum \frac{1}{q^n + q^{-n}} = \sum_{n = -\infty}^\infty \frac{1}{\cos(n \pi \tau)} \]
    %
    We will prove that $f$ has the same symmetry as $\theta^2$, and also has the same asymptotics near cusps. It is obvious that $f(\tau + 2) = f(\tau)$. We require two trigonometric identities:
    %
    \[ \sum_{n = -\infty}^\infty \frac{1}{\cosh(\pi n t)} = \frac{1}{t} \sum_{n = -\infty}^\infty \frac{1}{\cosh(\pi n/t)} \]
    %
    \[ \sum_{n = -\infty}^\infty \frac{(-1)^n}{\cosh(\pi n/t)} = t \sum_{n = -\infty}^\infty \frac{1}{\cos(\pi(n + 1/2)t)} \]
    %
    The first says exactly that $f(-1/\tau) = -i\tau f(\tau)$. The second ca be analytically continued to say
    %
    \[ f(1 - 1/\tau) = -i \tau \sum_{n = -\infty}^\infty \frac{1}{\cos(\pi(n + 1/2)\tau)} \]
    %
    This gives $f(1 - 1/\tau) \sim -4i \tau \cdot e^{\pi i \tau/2}$ as $\text{Im}(\tau) \to \infty$. And $f(i \infty) = 1$. Thus $f(\tau)/\theta^2(\tau)$ is holomorphic at cusp points, and since $f(i\infty)/\theta^2(i \infty) = 1$, we conclude $f(\tau) = \theta^2(\tau)$ for all $\tau$. This completes the proof.
\end{proof}

\begin{corollary}
    An integer $N$ is the sum of two perfect squares if and only if every prime $p$ congruent to three modulo four divides into $N$ an even number of times.
\end{corollary}
\begin{proof}
    Let $p$ be a prime with $p \equiv 1$ modulo 4. Then $r_2(p^n) = 4(n+1)$. This is because $p$ has precisely $n+1$ divisors equal to 1 modulo 4, and zero equal to 3 modulo 4. In particular, $r_2(p) = 8$, and thus there is a {\it unique} choice of perfect squares $n^2$ and $m^2$ such that $n^2 + m^2 = p$. On the other hand, if $p \equiv 3$ modulo 4, then $d_1(p^n) = \lfloor (n+1)/2 \rfloor$, since only odd powers are congruent to one modulo 4, and $d_3(p^n) = \lceil (n+1)/2 \rceil$. Now these two numbers are equal if and only if $n$ is odd, so $p^n$ can be written as a sum of squares if and only if $n$ is odd. Now if $N$ and $M$ are relatively prime, a divisor of $NM$ can be split uniquely into a divisor of $N$ and $M$ and so
    %
    \begin{align*}
        r_2(NM) &= 4[d_1(NM) - d_3(NM)]\\
        &= 4(d_1(N) d_1(M) + d_3(N) d_3(M)) - 4(d_1(N) d_3(M) + d_3(N) d_1(M))\\
        &= r_1(N)r_2(M)/4
    \end{align*}
    %
    Thus $r_2(NM) = 0$ if and only if $r_1(N) = 0$ or $r_2(M) = 0$.
\end{proof}

\section{Forbidden Eisenstein Series}

We couldn't define the weight two Eisenstein series properly since the series
%
\[ \sum_{n,m \in \mathbf{Z}} \frac{1}{(n\tau + m)^2} \]
%
does not converge absolutely. Nonetheless, we shall find it {\it is} useful to define a `forbidden' weight two Eisenstein series so we can attack the four squares problem. We consider the two functions
%
\[ F(\tau) = \sum_m \sum_n \frac{1}{(m\tau + n)^2}\ \ \ \ \ \tilde{F}(\tau) = \sum_n \sum_m \frac{1}{(m\tau + n)^2} \]
%
where the order of $m$ and $n$ is {\it integral} in both series is integral. The modular identity $F(-1/\tau) = \tau^2 F(\tau)$ does not hold, since we cannot rearrange sums, but the standard manipulations for Eisenstein series gives $F(-1/\tau) = \tau^2 \tilde{F}(\tau)$. Using the Dedekind eta function $\eta(\tau) = q^{1/12} \prod (1 - q^{2n})$, which has the modular symmetry $\eta(-1/\tau) = \sqrt{-i \tau} \eta(\tau)$, we can give a modular symmetry for $F$ itself.

\begin{theorem}
    $F(-1/\tau) = \tau^2 F(\tau) - 2\pi i \tau$.
\end{theorem}
\begin{proof}
    The same manipulations as for normal Eisenstein series give that $F(-1/\tau) = \tau^2 \tilde{F}(\tau)$, albeit this time we cannot interchange summation to give $\tilde{F}(\tau) = F(\tau)$. Now if we take the logarithmic derivative of $\eta$, we get
    %
    \[ \frac{\eta'(\tau)}{\eta(\tau)} = \frac{\pi i}{12} - 2 \pi i \sum \frac{1}{1 - q^{2n}} = \frac{\pi i}{12} - 2 \pi i \sum_{n = 1}^\infty \sigma_1(n) q^{2n} \]
    %
    But the standard expansion of Eisenstein series in terms of the divisor function give that $F(\tau) = \pi^2/3 - 8 \pi^2 \sum \sigma_1(n) q^{2n}$, so we have shown that $\eta'(\tau)/\eta(\tau) = (i/4\pi) F(\tau)$. Taking the logarithmic derivative of both sides of the functional equation gives
    %
    \[ \frac{\eta'(-1/\tau)}{\eta(-1/\tau)} = \frac{i}{4 \pi \tau^2} F(-1/\tau) = \frac{i \tilde{F}(\tau)}{4\pi} \]
    \[ \frac{(\sqrt{-i\tau} \eta(\tau))'}{\sqrt{-i \tau} \eta(\tau)} = \frac{1}{2 \tau} + \frac{i F(\tau)}{4 \pi} \]
    %
    Putting these two equations together gives the required result.
\end{proof}

\section{Four Squares}

We now prove Lagrange's theorem that every positive integer is the sum of four squares. Moreover, to do this we will define an exact formula for the number of ways $r_4(N)$ that we can do this for each $N$.

\begin{theorem}
    For all $N$, $r_4(N) = 8 \sigma_1^*(N)$, where $\sigma_1^*(N)$ denotes the number of divisors of $N$ {\it not} divisible by 4.
\end{theorem}
\begin{proof}
    Like with two squares, it shall suffice to prove an identity for $\theta^4(\tau)$. To do this, we must dive into the `forbidden' Eisenstein series of weight two. We consider
%
\[ E_2^*(\tau) = \sum_m \sum_n \frac{1}{(m\tau/2 + n)^2} - \sum_m \sum_n \frac{1}{(m \tau + n/2)^2} = F(\tau/2) - 4F(2\tau) \]
%
A manipulation of the Eisenstein series gives that
%
\[ E_2^*(\tau) = \left( \frac{\pi^2}{3} - 8 \pi^2 \sum_{k = 1}^\infty \sigma_1(k) q^k \right) - 4 \left( \frac{\pi^2}{3} - 8 \pi^2 \sum_{k = 1}^\infty \sigma_1(k) q^{4k} \right) \]
%
The fact that $\sigma_1^*(N) = \sigma_1(N)$ if $N$ is not divisible by four, and $\sigma_1^*(N) = \sigma_1(N) - 4 \sigma_1(N/4)$ if $N$ is divisible by four, shows
%
\[ E_2^*(\tau) = - \pi^2 - 8 \pi^2 \sum_{k = 1}^\infty \sigma_1^*(k) q^k \]
%
It therefore suffices to prove that $\theta^4(\tau) = -E_2^*(\tau)/\pi^2$. To do this, we need only verify that $E_2^*$ satisfies the same transformational properties as $\theta^4$, and has the same asymptotic properties near cusp points.

It is obvious that $E_2^*(\tau + 2) = E_2^*(\tau)$. We then calculate that
%
\begin{align*}
    E_2^*(-1/\tau) &= F(-1/2\tau) - 4F(-2/\tau)\\
    &= [4\tau^2 - 4 \pi i \tau] - 4[(\tau/2)^2 F(\tau/2) - \pi i \tau]\\
    &= 4 \tau^2 F(2 \tau) - 4(\tau^2/4) F(\tau/2)\\
    &= - \tau^2(F(\tau/2) - 4F(2\tau)) = - \tau^2 E_2^*(\tau)
\end{align*}
%
We have already calculated the $q$ series expansion for $E_2^*$. To calculate the $q$ series expansion around $1$, we calculate that
%
\begin{align*}
    F(1/2 - 1/2\tau) &= F \left( \frac{\tau - 1}{2 \tau} \right) = \left( \frac{2 \tau}{\tau - 1} \right)^2 F \left( \frac{2 \tau}{1 - \tau} \right) - \frac{4 \pi i \tau}{1 - \tau}
\end{align*}
%
\begin{align*}
    F \left( \frac{2\tau}{1 - \tau} \right) &= F \left(-2 + \frac{2}{1 - \tau} \right)\\
    &= F \left( \frac{2}{1 - \tau} \right) = \left( \frac{1 - \tau}{2} \right)^2 F \left( \frac{\tau - 1}{2} \right) - 2 \pi i \left( \frac{\tau - 1}{2} \right)
\end{align*}
%
Thus
%
\[ F(1/2 - 1/2\tau) = \tau^2 F \left( \frac{\tau - 1}{2} \right) - \frac{4 \pi i \tau}{1 - \tau} - 2 \pi i \frac{4 \tau^2}{(\tau - 1)^2} \left( \frac{\tau - 1}{2} \right) \]
%
But $F(2 - 2/\tau) = F(-2/\tau) = (\tau^2/4) F(\tau/2) - 2 \pi i \tau/2$, and so
%
\begin{align*}
    E_2^*(1 - 1/\tau) &= F(1/2 - 1/2\tau) - 4 F(2 - 2/\tau)\\
    &= \tau^2 \left( F\left( \frac{\tau - 1}{2} \right) - F(\tau/2) \right) - 2 \pi i \left( \frac{2 \tau}{1 - \tau} + \frac{2 \tau^2}{\tau - 1} \right) + 4 \pi i \tau\\
    &= \tau^2 \left( F \left( \frac{\tau - 1}{2} \right) - F(\tau/2) \right)
\end{align*}
%
Thus we conclude that $|E_2^*(1 - 1/\tau)| \lesssim |\tau^2 e^{\pi i \tau}|$ as $\text{Im}(\tau) \to \infty$, since we have $F(\tau) = \pi^2/3 + O(e^{2 \pi i \tau})$ as $\text{Im}(\tau) \to \infty$. But this means $E_2^*(\tau)/\theta^4(\tau)$ is bounded as we approach cusp points, and therefore holomorphic there. Since we have $E_2^*(i \infty)/\theta^4(\tau) = -\pi^2$, this completes the proof.
\end{proof}

Since it is obvious that $\sigma_1^*(N) \geq 1$ for all $N \geq 1$, this shows that it is possible to write {\it every} integer as a sum of four squares.

\begin{remark}
    We can continue this pattern. In the case of sums of eight squares, it is actually much simpler than in the two cases to come up with a formula for the number of representations, because we actually do have a proper modular form of weight 2.
\end{remark}

%\section{Theta Series in Multiple Variables}

%We now consider a multivariate generalization of theta series. Given any positive definite quadratic form $Q: \mathbf{Z}^m \to \mathbf{Z}$, we associate a series $\Theta_Q(\tau) = \sum q^{Q(x_1, \dots, x_m)}$, whose $q^N$ coefficient measures the number of representations of an integer as the number of vectors $x$ with $Q(x) = N$. We write $Q(x) = \sum A_{ij} x_ix_j/2$ where $A$ is a positive definite, integral symmetric $m \times m$ matrix, whose diagonal elements are even. Then $A^{-1}$ is a rational matrix, and we define the {\bf level} of $Q$ to be the smallest integer $N$ such that $NA^{-1}$ is an even integral matrix. The {\bf discriminant} is $\Delta_Q = (-1)^m \cdot \det A$. If $m$ is an even integer, we get a precise description of the modular behaviour of the theta series.

%\begin{example}
%    If $Q(x) = x^2$, then $\Theta_Q = \theta$. It is level one, with discriminant $-1$.
%\end{example}

%\begin{theorem}[Hecke, Schoenberg]
%    Let $Q: \mathbf{Z}^{2k} \to \mathbf{Z}$ be a quadratic form, then $\Theta_Q$ satisfies
%\end{theorem}
%\begin{proof}
    
%\end{proof}

%\section{Euler's Pentagonal Theorem}

%We will be discussing Theta functions, and their connections with various estimation problems in analytic number theory. Given an integer $N$, we define the {\bf partition function} $p(N)$ to be the number of unordered representations of $N$ as a sum of positive integers. For instance $p(1) = 1$, and $p(4) = 5$, since
%
%\[ 4 = 3 + 1 = 2 + 2 = 2 + 1 + 1 = 1 + 1 + 1 + 1 \]%
%
%We set $p(0) = 0$. The classical way to study sequences of numbers representing additive properties is to form a power series, so in this case we consider the generating function
%
%\[ \sum_{n = 0}^\infty p(n) q^n \]
%
%Just as the product formula for the zeta functions gives analytical information about the function, a product formula for this expansion will also give analytical information.
%\begin{theorem}
%    If $|q| < 1$, then
    %
%    \[ \sum_{n = 0}^\infty p(n) q^n = \prod_{k = 1}^\infty \frac{1}{1 - q^k} \]
%\end{theorem}
%\begin{proof}
%    We provide the gist of the proof. We can write
    %
%    \[ \frac{1}{1 - q^k} = \sum_{n = 1}^\infty q^{nk} \]
    %
%    If we take the infinite product of all these infinite sums, then the coefficient corresponding to each term in the power series expansion corresponding to $q^N$ to the number of ways to represent $N$ as $\sum_{n = 1}^\infty k_n \cdot n$ for some positive integers $k_n$, and this is easily seen to be equal to $p(N)$.
%\end{proof}

%By manipulating the product formula, we get certain variants to the partition function. The infinite product
%
%\[ \prod_{k = 1}^\infty \frac{1}{1 - q^{2k - 1}} \]
%
%gives a function $p_o(N)$, which counts the number of ways we can write $N$ as an unordered sum of {\it odd} integers. If we consider the infinite product
%
%\[ \prod_{k = 1}^\infty (1 + q^k) \]
%
%we get the function $p_u(N)$ which counts the number of ways to write $N$ as a sum of {\it unequal } integers.

%\begin{theorem} $p_u(N) = p_o(N)$ \end{theorem}
%\begin{proof}
%    We calculate
%
%\[ \prod_{k = 1}^\infty (1 - q^k) = \prod_{k = 1}^\infty (1 - q^{2n})(1 - q^{2n - 1}) = \prod (1 + q^k)(1 - q^k)(1 - q^{2k-1}) \]
%
%and dividing both sides by $(1 - q^k)(1 - q^{2k - 1})$ gives
%
%\[ \prod_{k = 1}^\infty (1 + q^k) = \prod_{k = 1}^\infty \frac{1}{1 - q^{2k-1}} \]
%
%from which we can take power series to get the result.
%\end{proof}



\end{document}