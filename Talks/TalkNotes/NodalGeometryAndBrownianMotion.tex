% LaTeX template for Summer Schools
%
% !!!!!!!!!!!!   INSTRUCTIONS  !!!!!!!!!!!!!!

% 0) Submit your work in tex format (not pdf). It will be pasted into the proceedings file.
%
% 1) please name your file yourlastname.tex, e.g. thiele.tex.
%
% As several files will be concatenated, please 
% use some discipline as to the following:
%
% 2) Whenever you use \label{} to get automatic cross references
% through \ref{} (this is the preferred option for cross references)
% please add your initials to the label such as 
% \label{SLEct} for the Stochastic Loewner equation
% with initials of author Christoph Thiele
% Same with other citations such as in bibitem.
%
% 3) Please STRONGLY avoid using \def or \newcommand unless really necessary.
% We do have macros for black board bold. If you use your favorite
% macros while preparing your summary, please expand them
% (replace by the original definition) everywhere in your file.
% This will save me the work of doing the very same thing.
% Thank you! If you have to use \def, please also add your
% initials to the definition.
%
% 4) if you want to compile the header of the document, 
% uncomment remove the corresponding paragraph signs below
%
% 5) There is a sample lecture below. For the header 
% it is best to keep most of the
% commands and just change the name, title, text. etc
%
% 6) Please follow the conventions below in terms of capitalization of 
% headings etc:
% Only the beginning of a sentence and names are capitalized.


\documentclass[12pt]{article}
\usepackage{amssymb,amsmath,amsthm}
\usepackage{esint}
\usepackage{mathrsfs}
\usepackage{mathtools}
\usepackage{comment}
\usepackage{bbm,dsfont}

\newtheorem{theorem}{Theorem}
\newtheorem{lemma}[theorem]{Lemma}
\newtheorem{corollary}[theorem]{Corollary}
\newtheorem{conjecture}[theorem]{Conjecture}
\newtheorem{definition}[theorem]{Definition}
\newtheorem{example}[theorem]{Example}
\newtheorem{remark}[theorem]{Remark}
\newtheorem{proposition}[theorem]{Proposition}

\newcommand{\talktitle}[1]{\section{#1}}
\newcommand{\talkafter}[1]{\textbf{After #1} \addcontentsline{toc}{subsection}{after #1}}
\newcommand{\talkspeaker}[2]{\begin{center}
\textit{A summary written by #1}
\end{center}
\addcontentsline{toc}{subsection}{#1, #2}
}

\usepackage{graphicx}
\usepackage[utf8]{inputenc}
\usepackage[T1]{fontenc}
\usepackage[hidelinks]{hyperref}

\newcommand*{\Z}{\mathbb{Z}}
\newcommand*{\Q}{\mathbb{Q}}
\def\C{\mathbb{C}}
\newcommand*{\N}{\mathbb{N}}
\newcommand*{\R}{\mathbb{R}}

% Absolute values and norms using mathtools.
% \\[lr][vV]ert produces correct spacing, as opposed to | and \|.
\DeclarePairedDelimiter\abs{\lvert}{\rvert}
\DeclarePairedDelimiter\norm{\lVert}{\rVert}

\title{Brascamp--Lieb inequalities}

% \author{Summer School, Kopp}
% \thanks{supported by Hausdorff Center for Mathematics, Bonn}

% \date{September 2021}



\begin{document}

% \maketitle

% {
% \center{Organizers:}
% \center{
% Christoph Thiele, Universit\"at Bonn}
% \center{
% Pavel Zorin-Kranich, Universit\"at Bonn}
% \center{$\ $}
% }
% \newpage
% \tableofcontents

% \newpage


\talktitle{Capacity of Rank Decreasing Operators}
\talkafter{Gurvits \cite{gurv2004}}
\talkspeaker{Jacob Denson}{University of Madison, Wisconsin}

\setcounter{equation}{0}
\setcounter{theorem}{0}

\begin{abstract}
{ We describe the theory of the capacity of rank-decreasing positive operators, and Gurvit's operator rescaling algorithm to compute the capacity of such operators, with a brief discussion of the connection between this theory and the computation of Brascamp-Lieb constants. }
\end{abstract}

Let $M$ be a compact Riemannian manifold, and let $e_\lambda \in C^\infty(M)$ be an eigenfunction for the Laplacian on $M$, i.e. such that $\Delta e_\lambda = -\lambda^2 e_\lambda$. Our goal is the study of the nodal set
%
\[ \Omega_\lambda = \{ x \in M: e_\lambda(x) = 0 \} \]
%
and in particular, the asymptotic geometry of this set as $\lambda \to \infty$.

Our first question is whether we can fit $\Omega_\lambda$ in a tubular neighborhood of a set $\Sigma \subset M$. Intuitively, one should not expect to fit $\Omega_\lambda$ in a $O(\lambda^{-1/2})$ neighborhood of a hypersurface $\Sigma$ if the surface is flat (TODO: Provide Intuition). We prove a version of this result.

\begin{theorem}
    Let $\Sigma = \Sigma_1 \cup \dots \cup \Sigma_m$, where $\Sigma_i$ are smooth surfaces in $M$ of dimension $k$, such that for any $x \in M$ with $d(x,\Sigma) \leq \lambda^{-1/2}$, there exists a unique point on $\Sigma$ closest to $x$. (TODO: Why is this connected with flatness). Then no nodal domain $\Omega_\lambda$ can be contained in a $O(\lambda^{-1/2})$ neighborhood of $\Sigma$.
\end{theorem}



\begin{thebibliography}{3}

\end{thebibliography}

\noindent \textsc{Jacob Denson, UW Madison}\\
\textit{email:} \texttt{jcdenson@wisc.edu}

\end{document}