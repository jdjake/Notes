\documentclass{article}

%% for editing
%\usepackage{changes}
%\usepackage[final]{changes} %% comment above line and uncomment this line to see final copy without markup
%\setremarkmarkup{~(#2)}
\usepackage{color}


\usepackage{amsmath}
\usepackage{amssymb}
\usepackage{amsthm}
\usepackage{multicol}
%\usepackage[margin=1in]{geometry}
\usepackage{graphicx}
\usepackage{tikz}
\usepackage{hyperref}
\usepackage{mathabx}
\usepackage{comment}

\usepackage{tensor}

\theoremstyle{plain}
\newtheorem{theorem}{Theorem}
\newtheorem{lemma}[theorem]{Lemma}
\newtheorem{corollary}[theorem]{Corollary}
\newtheorem{prop}[theorem]{Proposition}

\theoremstyle{remark}
\newtheorem*{remark}{Remark}
\newtheorem*{example}{Example}
\newtheorem*{proof*}{Proof}

\theoremstyle{definition}
\newtheorem*{defi}{Definition}
\newenvironment{definition}
    {\begin{samepage}\begin{framed}\begin{defi}}
    {\end{defi}\end{framed}\end{samepage}}


\DeclareMathOperator{\diam}{\text{diam}}

\DeclareMathOperator{\QQ}{\mathbb{Q}}
\DeclareMathOperator{\ZZ}{\mathbb{Z}}
\DeclareMathOperator{\RR}{\mathbb{R}}
\DeclareMathOperator{\HH}{\mathbb{H}}
\DeclareMathOperator{\BB}{\mathbb{B}}
\DeclareMathOperator{\CC}{\mathbb{C}}
\DeclareMathOperator{\AB}{\mathbb{A}}
\DeclareMathOperator{\PP}{\mathbb{P}}
\DeclareMathOperator{\MM}{\mathbb{M}}
\DeclareMathOperator{\VV}{\mathbb{V}}
\DeclareMathOperator{\TT}{\mathbb{T}}
\DeclareMathOperator{\LL}{\mathcal{L}}
\DeclareMathOperator{\DD}{\mathcal{D}}
\DeclareMathOperator{\SW}{\mathcal{S}}
\DeclareMathOperator{\EC}{\mathcal{E}}
\DeclareMathOperator{\AC}{\mathcal{A}}

\DeclareMathOperator{\EE}{\mathbb{E}}
\DeclareMathOperator{\NN}{\mathbb{N}}

\DeclareMathOperator{\II}{\mathbb{I}}

\DeclareMathOperator{\DQ}{\mathcal{Q}}


\DeclareMathOperator{\IA}{\mathfrak{a}}
\DeclareMathOperator{\IB}{\mathfrak{b}}
\DeclareMathOperator{\IC}{\mathfrak{c}}
\DeclareMathOperator{\IP}{\mathfrak{p}}
\DeclareMathOperator{\IQ}{\mathfrak{q}}
\DeclareMathOperator{\IM}{\mathfrak{m}}
\DeclareMathOperator{\IN}{\mathfrak{n}}
\DeclareMathOperator{\IK}{\mathfrak{k}}
\DeclareMathOperator{\ord}{\text{ord}}
\DeclareMathOperator{\Ker}{\textsf{Ker}}
\DeclareMathOperator{\Coker}{\textsf{Coker}}
\DeclareMathOperator{\emphcoker}{\emph{coker}}
\DeclareMathOperator{\pp}{\partial}
\DeclareMathOperator{\tr}{\text{tr}}
\DeclareMathOperator{\Ree}{\text{Re}}


\DeclareMathOperator{\BL}{\text{BL}}

\DeclareMathOperator{\dstrike}{//}

\DeclareMathOperator{\supp}{\text{supp}}

\DeclareMathOperator{\codim}{\text{codim}}

\DeclareMathOperator{\minkdim}{\dim_{\mathbb{M}}}
\DeclareMathOperator{\hausdim}{\dim_{\mathbb{H}}}
\DeclareMathOperator{\sobdim}{\dim_{\mathbb{S}}}
\DeclareMathOperator{\lowminkdim}{\underline{\dim}_{\mathbb{M}}}
\DeclareMathOperator{\upminkdim}{\overline{\dim}_{\mathbb{M}}}
\DeclareMathOperator{\lhdim}{\underline{\dim}_{\mathbb{M}}}
\DeclareMathOperator{\lmbdim}{\underline{\dim}_{\mathbb{MB}}}
\DeclareMathOperator{\packdim}{\text{dim}_{\mathbb{P}}}
\DeclareMathOperator{\fordim}{\dim_{\mathbb{F}}}

\DeclareMathOperator{\CT}{ {{\otimes}^\wedge} }

\DeclareMathOperator{\msupp}{\text{$\mu$-supp}}
\DeclareMathOperator{\singsupp}{\text{sing-supp}}
\DeclareMathOperator{\Char}{\text{Char}}

\DeclareMathOperator*{\argmax}{arg\,max}
\DeclareMathOperator*{\argmin}{arg\,min}

\DeclareMathOperator{\ssm}{\smallsetminus}

\newcommand{\myphi}[1]{ \tensor[^{\phi}]{{#1}}{}}




\title{Averaging over Curves}
\author{Jacob Denson}

\begin{document}

\maketitle

Consider a smooth family of curves $\gamma: \RR^2 \to \RR$, and consider the associated averaging operator
%
\[ Af(v,x) = \int f(x + \gamma_v(t)) \phi(v,t)\; dt, \]
%
where $\gamma_v''(t) = 0$, and $\phi$ is smooth with compact support. We can write this operator as
%
\[ Af(v,x) = (f * \mu_v)(x), \]
%
where $\mu_v$ is the Borel measure such that for any bounded, measurable function $g$,
%
\[ \int g(x) d\mu_v(x) = \int g(\gamma_v(t)) \phi(t)\; dt. \]
%
We can then write
%
\[ \widehat{\mu}_v(\xi) = \int e^{-2 \pi i \xi \cdot x} d\mu_v(x) = \int e^{-2 \pi i \xi \cdot \gamma_v(t)} \phi(t)\; dt. \]
%
This is an oscillatory integral, which is stationary at points $t$ where $\xi \cdot \gamma_v'(t) = 0$. Under the assumption that $\gamma_v''$ is non-vanishing, these stationary points are non-degenerate, and so provided we choose $\phi$ to have small support, for each $\xi \in \RR^d$, there is at most one value of $t$ such that $\xi \cdot \gamma_v'(t) = 0$. Let us write this value by $t_0(\xi)$. We can then find a smooth function $\psi_v: \RR^d \to \RR$ such that on the domain of $t_0$,
%
\[ \psi_v(\xi) = \xi \cdot \gamma_v(t_0(\xi)). \]
%
Then the theory of stationary phase guarantees that
%
\[ \widehat{\mu}_v(\xi) = e^{2 \pi i \psi_v(\xi)} b(\xi), \]
%
where $b$ is a symbol of order $-1/2$, with microsupport on the domain of $t_0$. Using the multiplication formula for the Fourier transform, we can thus write
%
\[ Af(v,x) = \int \widehat{\mu}_v(\xi) \widehat{f}(\xi) e^{2 \pi i \xi \cdot x}\; d\xi = \int b(\xi) e^{2 \pi i [\psi_v(\xi) + \xi \cdot (x - y)]} f(y)\; d\xi\; dy. \]
%
This is a Fourier integral operator with phase
%
\[ \phi(v,x,y,\xi) = \psi_v(\xi) + \xi \cdot (x - y). \]
%
Let's compute it's canonical relation.

We have $(\nabla_\xi \phi)(v,x,y,\xi) = \nabla_\xi \psi_v(\xi) + (x - y)$. Since $\psi_v(\xi) = - \xi \cdot \gamma_v(t_0(\xi))$, the chain rule implies that
%
\[ (\nabla_\xi \psi_v)(\xi) = - \gamma_v(t_0) - (\xi \cdot \gamma_v'(t_0)) (\nabla_\xi t_0) = - \gamma_v(t_0). \]
%
Thus the stationary points occur for values of $\xi$ such that $x - y = \gamma_v(t_0(\xi))$. We then have
%
\[ \nabla_x \phi(v,x,y,\xi) = \xi \quad\text{and}\quad \nabla_y \phi(v,x,y,\xi) = - \xi \]
%
and
%
\[ \nabla_v \phi(v,x,y,\xi) = \partial_v \psi_v(\xi). \]
%
Thus the canonical relation of the Fourier integral operator is
%
\[ \mathcal{C} = \Big\{ (v,x,y,\nu,\xi,\eta) : \nu = \partial_v \gamma_v(t_0(\xi))\ \text{and}\ x = y + \gamma_v(t_0(\xi))\ \text{and}\ \xi = \eta \Big\}. \]
%
The projection of $\mathcal{C}$ onto the $(y,\eta)$ variables give a submersion, and the projection of $\mathcal{C}$ onto $(v,x)$ also form a submersion. For each fixed $z = (v,x)$, let
%
\[ \Gamma_z = \Big\{ (\nu,\xi) : \nu = \partial_v \gamma_v(t_0(\xi)) \Big\} \]
%
be the projection of $\mathcal{C}$ onto the $(\nu,\xi)$ variables at $(v,x)$. The cinematic curvature condition amounts to saying that $\Gamma_z$ is a hypersurface of dimension $2$ in $\RR^3$, with one non-vanishing principal curvature. If we write $\Phi(v,\xi) = \partial_v \gamma_v(t_0(\xi))$, then this amounts to saying that the matrix $D_\xi \Phi$ has rank one. By the chain rule, this will hold if $\nabla_\xi t_0$ is non-zero at $\xi$, and $\partial^2_{v,t} \gamma_v$ is non-zero at $t_0(\xi)$. But  $\nabla_\xi t_0 \neq 0$ using the fact that $\gamma_v'' \neq 0$, so the cinematic curvature condition holds under the assumption that $\gamma_v'' \neq 0$, and $\partial^2_{v,t} \gamma \neq 0$. Thus the cinematic curvature condition is satisfied under the assumption that each of the curve you are averaging over has non-vanishing curvature, and if $\partial^2_{v,t} \gamma \neq 0$, i.e. the tangent vectors of $\gamma$ change as we vary $v$.

\end{document}











- Let X be be a compact Riemannian manifold
	- Then -Delta is essentially self-adjoint and non-negative, so spectral calculus tells us that sqrt(-Delta)
	- 

