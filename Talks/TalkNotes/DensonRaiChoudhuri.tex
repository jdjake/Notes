\documentclass[12pt]{amsart}
\usepackage{amssymb,amsmath,amsthm}
\usepackage{esint}
\usepackage{mathrsfs}
\usepackage{mathtools}
\usepackage{bbm,dsfont}

\newtheorem{thm}{Theorem}
\newtheorem{lemma}[thm]{Lemma}
\newtheorem{corollary}[thm]{Corollary}
\newtheorem{conjecture}[thm]{Conjecture}
\newtheorem{definition}[thm]{Definition}
\newtheorem{example}[thm]{Example}
\newtheorem{remark}[thm]{Remark}
\newtheorem{proposition}[thm]{Proposition}

\addtolength{\hoffset}{-0.5cm}
\addtolength{\textwidth}{1cm}
 

\newcommand{\talktitle}[1]{\section{#1}}
\newcommand{\talkafter}[1]{\begin{center}{After #1} \end{center}
 \addcontentsline{toc}{subsection}{after #1}}
\newcommand{\talkspeaker}[2]{\begin{center}
{A summary  by #1}
\end{center}
\addcontentsline{toc}{subsection}{#1, #2}
}

\usepackage{graphicx}
\usepackage[utf8]{inputenc}
\usepackage[T1]{fontenc}
\usepackage[hidelinks]{hyperref}

\newcommand*{\Z}{\mathbb{Z}}
\newcommand*{\Q}{\mathbb{Q}}
\def\C{\mathbb{C}}
\newcommand*{\N}{\mathbb{N}}
\newcommand*{\R}{\mathbb{R}}

%Leave empty
\title{}
 


\begin{document}


 
% {
% \center{Organizers:}
% \center{
% ..}
% \center{
% ..}
% \center{$\ $}
% }
% \newpage
% \tableofcontents

% \newpage


%%% REPLACE WITH YOUR DATA

\talktitle{On Trilinear Oscillatory Integral Inequalities and Related Topics}
\talkafter{M. Christ \cite{ChristTopicPaperjdmrc}}
\talkspeaker{Jacob Denson and Mukul Rai Choudhuri}{University of Wisconsin, Madison and University of British Columbia}

\setcounter{equation}{0}
\setcounter{thm}{0}


\begin{abstract}
	We discuss decay estimates for trilinear oscillatory integrals on $L^2(\R) \times L^2(\R) \times L^2(\R)$ which give power decay in the frequency parameter with exponent having magnitude larger than $1/2$, under non-degeneracy conditions on the phase. The main consequence of these estimates, for the purpose of this summer school, are smoothing bounds for a family of nonlinear Brascamp-Lieb functionals in $\R^3$, which include results for the functions induced by $(x,y) \mapsto (x,x + y,x + y^2)$.
\end{abstract} 

 \maketitle

\subsection{Introduction} Consider an oscillatory integral of the form
%
\begin{equation} \label{ClassicalOscillatoryIntegraljdmrc}
    T^\phi_\lambda(f) = \int_{\R^J} e^{i \lambda \phi(x)} f(x)\; dx,
\end{equation}
%
for $\phi: \R^J \to \R$ and $f: \R^J \to \C$. If $f$ is appropriately smooth and $\phi$ is appropriately `non-stationary' %\footnote{The `non-stationarity' of a function $\phi$ is normally measured by non-vanishing of certain derivatives of $\phi$ or, as we discuss later, in terms of the size of level sets of $\phi$.}
then, as $\lambda \to \infty$, the integrand in \eqref{ClassicalOscillatoryIntegraljdmrc} begins to oscillate faster and faster, and so we expect \eqref{ClassicalOscillatoryIntegraljdmrc} to decay as $\lambda \to \infty$.

% and so we expect greater and greater cancellation to occur in the integral. In particular, if $\text{supp}(f)$ is contained in a fixed compact set $K \subset \R^J$, the principle of stationary phase can then guarantee a bound $| T^\phi_\lambda(f) | \lesssim_K \lambda^{-\gamma} \max\nolimits_{|\alpha| \leq n} \| \partial^\alpha f \|_{L^\infty(\R^J)}$ with power decay in $\lambda$, for appropriate exponents $\gamma$ and $n$.
If we remove the smoothness assumptions on $f$, then it is impossible to obtain any decay in \eqref{ClassicalOscillatoryIntegraljdmrc} as $\lambda \to \infty$. Setting
% for a fixed compact set $K$, taking
$f_\lambda(x) = \boldsymbol{1}_{[0,1]^J}(x) e^{-i \lambda \phi(x)}$, we find $\| f_\lambda \|_{L^p(\R^J)} = 1$ and $|T^\phi_\lambda(f_\lambda)| = 1$. Thus the only uniform bound of the form
%
\begin{equation} \label{NoSmoothnessDecayEquationjdmrc}
    | T^\phi_\lambda(f)| \lesssim \lambda^{-\gamma} \| f \|_{L^p(\R^J)}
\end{equation}
%
is the trivial bound with $\gamma = 0$. 

On the other hand, it is possible for non-trivial cancellation to occur by making structural assumptions on the function $f$. Here we assume that $f = f_1 \otimes \dots \otimes f_J$ for some functions $f_1,\dots,f_J: \R \to \C$, i.e. so that $f(x) = f_1(x_1) \cdots f_J(x_J)$. Equation \eqref{NoSmoothnessDecayEquationjdmrc} can then be written as a multi-linear inequality of the form
%
\begin{equation} \label{MultilinearInequalityjdmrc}
    | T^\phi_\lambda(f_1 \otimes \cdots \otimes f_J) | \lesssim \lambda^{-\gamma} \| f_1 \|_{L^2(\R)} \cdots \| f_J \|_{L^2(\R)}.
\end{equation}
%
The restrictions on $f$ allow for cancellation in \eqref{MultilinearInequalityjdmrc}, since the functions $f_\lambda$ are no longer counterexamples to \eqref{MultilinearInequalityjdmrc} with $\gamma > 0$ unless $\phi$ can be written as
%
\begin{equation} \label{phaseTensorDecompositionjdmrc}
    \phi(x) = \phi_1(x_1) + \dots + \phi_J(x_J)
\end{equation}
%
for some functions $\phi_1,\dots,\phi_J: \R \to \R$. One might hope that the best possible decay in \eqref{MultilinearInequalityjdmrc} is closely related to how `far' a phase function $\phi$ is from being decomposed as in \eqref{phaseTensorDecompositionjdmrc}. Current research is far from an optimal quantitative understanding of this relation. But the main result of \cite{ChristTopicPaperjdmrc} obtains new bound of the form \eqref{MultilinearInequalityjdmrc} in the trilinear case, for $p = 2$ and with $\gamma > 1/2$ by making assumptions naturally related to preventing a decomposition of the form \eqref{phaseTensorDecompositionjdmrc}, along with non-linear analogues of \eqref{phaseTensorDecompositionjdmrc}, called \emph{rank one degeneracies}.

\subsection{Preventing degeneracies}

One can prevent a decomposition of the form \eqref{phaseTensorDecompositionjdmrc} by assuming $\partial_1 \partial_2 \phi$ is non-vanishing on $[0,1]^3$. It is then a result of H\"{o}rmander \cite{Hormanderjdmrc} that the operator
%
\begin{equation}
    S_zf(x) = \int_{[0,1]} e^{i \lambda \phi(x,y,z)} f(y)\; dy
\end{equation}
%
satisfies, uniformly for $z \in [0,1]$,
%
\begin{equation} \label{HormanderBoundjdmrc}
    \| S_z f \|_{L^2[0,1] \to L^2[0,1]} \lesssim \lambda^{-1/2}.
\end{equation}
%
Using \eqref{HormanderBoundjdmrc}, Cauchy-Schwartz and H\"{o}lder's inequality then implies
%
\begin{equation} \label{HormanderL2L2L1Boundjdmrc}
    |T^\phi_\lambda(f)| \lesssim \lambda^{-1/2} \| f_1 \|_{L^2[0,1]} \| f_2 \|_{L^2[0,1]} \| f_3 \|_{L^1[0,1]}.
\end{equation}
%
Thus \eqref{MultilinearInequalityjdmrc} holds with $\gamma = 1/2$. It will be a good warmup for later techniques to briefly provide a proof of \eqref{HormanderBoundjdmrc} via a micro-local decomposition.

For any $\lambda > 0$, and $f \in L^2[0,1]$, we consider some decomposition of $f$ into wave packets, of the form $f = \sum f_{y_0,\eta_0}$,
%
where we sum over $(y_0,\eta_0)$ lying in the Cartesian product of $\lambda^{-1/2} \Z \cap [0,1]$ with $\lambda^{1/2} \Z$, and $(f_{y_0,\eta_0}, \widehat{f}_{y_0,\eta_0})$ is essentially supported on the rectangle centered at $(y_0,\eta_0)$ with dimensions $\lambda^{-1/2} \times \lambda^{1/2}$ (see Section \ref{Decompositionjdmrc} for a more detailed description of this ind of decomposition).
%
% Int e( L phi(x,y,z) ) f(y) dy
% e( L [phi(x,y0,z) - Dphi(x,y0,z) y0 ] ) Int e( L Dphi(x,y0,z) y ] ) f(y)
% e( L [ phi(x,y0,z) - Dphi(x,y0,z) y0 ] ) f^( -L Dphi(x,y0,z) y )
%
%\[ |\langle f_{y_0,\eta_0}, f_{y_0', \eta_0'} \rangle| \lesssim_N \Big( 1 + \lambda^{1/2} |y_0 - y_0'| + \lambda^{-1/2} |\eta_0 - \eta_0'| \Big)^{-N} \| f \|_{L^2[0,1]}^2. \]
%
Our assumption on $\phi$ implies that, unless the size of $T_z f_{y_0,\eta_0}$ is negligible to \eqref{HormanderBoundjdmrc}, the function $T_z f_{y_0,\eta_0}$ is also a wave packet $g_{x_0,\xi_0}$ at the same spatial and frequency scale as $f_{y_0,\eta_0}$, with $\| g_{x_0,\xi_0} \|_{L^2}^2 \lesssim \lambda^{-1} \| f_{y_0,\eta_0} \|_{L^2}^2$, and where
%
\begin{equation} \label{FrequencyTransformationjdmrc}
    \lambda (\partial_2 \phi)(x_0,y_0,z) = - \eta_0 \quad\text{and}\quad  \xi_0 = \lambda ( \phi(x_0,y_0,z) + \partial_2 \phi_2(x_0,y_0,z) y_0 ).
\end{equation}
%
One can verify from \eqref{FrequencyTransformationjdmrc} and the mixed derivative assumption on $\phi$ that distinct wave packets are mapped by $S_z$ to distinct wave packets. Thus the functions $g_{x_0,\xi_0}$ are almost orthogonal to one another, which justifies the calculation
%implies the existence of an injective map $(x_z,\xi_z): \Phi_\lambda \to \Phi_\lambda$ such that the function $S_z f_{y_0,\eta_0}$ rapidly decays away from a sidelength $\lambda^{-1/2}$ interval centered at $x_z(y_0,\eta_0)$, it's Fourier transform rapidly decays away from a sidelength $\lambda^{1/2}$ interval centered at $\xi_z(y_0,\eta_0)$, and $\| S_z f_{y_0,\eta_0} \|_{L^2[0,1]}^2 \lesssim \lambda^{-1} \| f_{y_0,\eta_0} \|_{L^2[0,1]}^2$. Thus the functions $\{ S_z f_{y_0,\eta_0} \}$ are also wave packets, which are essentially disjoint from one another in phase space. They are therefore almost orthogonal, implying that
%
\[ \| S_z f \|_{L^2}^2 \lesssim \sum\nolimits_{x_0,\xi_0} \| g_{x_0,\xi_0} \|_{L^2}^2 \lesssim \lambda^{-1} \sum\nolimits_{y_0,\eta_0} \| f_{y_0,\eta_0} \|_{L^2[0,1]}^2 \lesssim \lambda^{-1} \| f \|_{L^2[0,1]}^2. \]
%
Thus we have proved \eqref{HormanderBoundjdmrc}. In the proofs we wish to discuss here, we perform an analogous wave packet decomposition, but break the decomposition into two different pieces: a `pseudorandom' part, and a `sparse' part, in a decomposition detailed in Section \ref{Decompositionjdmrc}.

% Interacts with O( L^{1/2} ) other terms
%
% Int e( L phi(x,y,z) ) f(y) dy
% = e( L [phi(x,y_0,z) - phi_2(x,y_0,z) y_0] )  Int e( L phi_2(x,y_0,z) y + O(1) )
% = e( L [phi(x,y_0,z) - phi_2(x,y_0,z) y_0 ] ) f^( - L phi_2(x,y_0,z) )
%
%           So Should be supported on |-L phi_2(x,y_0,z) - eta_0| << L^{1/2}
%           Since phi_{12} nonvanishing, this is a O(L^{-1/2}) neighborhood
%           of solutions to phi_2(x,y_0,z) = -eta_0/L
%
%           But only frequencies that are << L are relevant to the output
%
% Then the oscillation is a O(L) neighborhood of e( L phi(x,y_0,z) + eta_0 )
%
% L phi_2(x(y_0,eta_0)) = - eta_0
%
% e( L [phi(x,y_0,z) + L phi_2(x,y_0,z) y_0 ) f^( - L phi_2(x,y_0,z) )
% L [ phi(x_0,y_0,z) + partial_2 phi_2(x_0,y_0,z) y_0 ]
%
% So rapidly decays away from |-L phi_2(x,y_0,z) - eta_0| << L
% Thus rapidly decays away from a O(1) neighborhood of solutions to L phi_2(x,y_0,z) = eta_0
% And in this neighborhood the phase is e( L [ phi(x,y_0,z) + eta_0 y_0 ] ), thus having Fourier support in a O(L) neighborhood of this phase
%
% Then the integral is non-negligible on a ~ L^{-1} neighborhood of x(y_0,eta_0)
%
% Frequency is e( L phi(x(y_0,eta_0), y_0, z) - eta_0 )

The main result of \cite{ChristTopicPaperjdmrc} is that one can improve H\"{o}rmander's estimate under a slightly stronger assumption on $\phi$. Say $\phi$ is \emph{rank one degenerate} if there exists $\psi_1,\dots,\psi_J$, and a hypersurface $H$ such that the function
%
\begin{equation}
    x \mapsto \phi(x) + \psi_1(x_1) + \dots + \psi_J(x_J)
\end{equation}
%
has gradient vanishing on $H$. An example of a rank one degenerate function is $\phi(x) = x_1x_2 + x_2x_3 + x_1x_3$, in which case we can set $H = \{ x \in [0,1]^3 : x_1 + x_2 + x_3 = 0 \}$ and $\psi_j(x_j) = x_j^2$ for all $j \in \{ 1, 2, 3 \}$.

Being rank one degenerate is a direct obstruction to \eqref{MultilinearInequalityjdmrc} with $\gamma > 1/2$. If we set $f_\lambda(x) = e^{i \lambda (\psi_1(x_1) + \dots + \psi_J(x_J))} \chi(x)$ for some smooth function $\chi = \chi_1 \otimes \dots \otimes \chi_J$ supported on a small neighborhood $x_0 \in H$, then by fibering the neighborhood of $x_0$ into curves transverse to $H$ and applying stationary phase on each curve, we can conclude that $|T^\phi_\lambda f_\lambda| \gtrsim \lambda^{-1/2}$. Theorem 4.1 of \cite{ChristTopicPaperjdmrc}  says that this is essentially the only obstruction.

\begin{thm} \label{MainOscillatoryTheoremjdmrc} \textbf{(Theorem 4.1 of \cite{ChristTopicPaperjdmrc})}
    If $\phi: [0,1]^3 \to \R$ is real analytic, not rank one degenerate, and $\partial_j \partial_k \phi \neq 0$ on $[0,1]^3$ for all $j \neq k$. Then \eqref{MultilinearInequalityjdmrc} holds for some $\gamma > 1/2$, i.e.
    %
    \begin{equation}
        |T^\phi_\lambda(f)| \lesssim \lambda^{-\gamma} \| f_1 \|_{L^2(\R)} \| f_2 \|_{L^2(\R)} \| f_3 \|_{L^2(\R)}.
    \end{equation}
\end{thm}

% The proof of Theorem 4.1, like \eqref{HormanderBound}, involves a microlocal decomposition like that discussed in the proof of \eqref{HormanderBound}, but with an \emph{additional} decomposition. We fix $\sigma > 0$, to be optimized later. For $f \in L^2[0,1]$, we let $\Phi_{\lambda,\psi} = \{ (y_0,\eta_0) \in \Phi_\lambda: \| f_{y_0,\eta_0} \|_{L^2(\R)} \leq \lambda^{-\delta} \| f \|_{L^2(\R)} \}$ and then define $f = f_\psi + f_\tau$ where $f_\psi = \sum\nolimits_{(y_0,\eta_0) \in \Phi_\lambda} f_{y_0,\eta_0}$ and $f_\tau = \sum\nolimits_{(y_0,\eta_0) \in \Phi_\lambda^c} f_{y_0,\eta_0}$. The function $f_\psi$ is the `pseudo-random' part of $f$, and $f_\tau$ is `thin', in the sense that, because $\sum \| f_{y_0,\eta_0} \|_{L^2[0,1]}^2 \sim \| f \|_{L^2[0,1]}$, we must have $\#(\Phi_\lambda^c) \lesssim \lambda^{2\sigma}$.

% In the remainder of this summary, we discuss how this implies certain smoothing inequalities for trilinear forms of singular Brascamp-Lieb type.

One potentially surprising application of Theorem 4.1, of direct relevance to other papers in this summer school, are bounds for trilinear operators in which no oscillation is immediately apparent. We consider this application in the next subsection, and then follow that section by taking the remainder of the paper to sketching the main details of the proof of Theorem 4.1.

\subsection{Smoothing inequalities}

For three functions $\varphi_1,\varphi_2,\varphi_3: \R^2 \to \R$, three functions $f_1,f_2,f_3: \R \to \C$, and a smooth cutoff $\eta: \R^2 \to \R$, define the trilinear function
%
\begin{equation}
    A^\varphi_\eta(f_1,f_2,f_3) = \int_{\R^2} \eta (f_1 \circ \varphi_1) (f_2 \circ \varphi_2) (f_3 \circ \varphi_3).
\end{equation}
%
If for some $x_0 \in \R^2$ and all $j \neq k$, $\nabla \varphi_i(x_0)$ and $\nabla \varphi_j(x_0)$ are linearly independent, H\"{o}lder and interpolation imply that for $\eta$ supported near $x_0$,
%
\begin{equation}
    \left\| A^\varphi_\eta(f_1,f_2,f_3) \right\|_{L^1(\R^2)} \lesssim \| f_1 \|_{L^{3/2}(\R)} \| f_2 \|_{L^{3/2}(\R)} \| f_3 \|_{L^{3/2}(\R)}.
\end{equation}
%
It is natural question whether we can ensure $A^\varphi_\eta(f_1,f_2,f_3)$ is integrable under weaker smoothness assumptions on the functions $\{ f_j \}$, i.e. assuming the $\{ f_j \}$ are distributions lying in some negative index Sobolev space. In general, this is only possible by assuming the functions $\{ \varphi_j \}$ have `curvature' in an appropriate sense. Christ gives a geometrically invariant definition of the curvature he uses, in terms of an induced `web', but we stick to an equivalent, more rudimentary definition for simplicity: we will say the $\{ \varphi_j \}$ `have curvature' if $\varphi_3 = F(\varphi_1, \varphi_2)$ for a function $F$ for which $\partial_1 \partial_2 F$ is non-vanishing in a neighborhood of $x_0$. The main example for our purposes occurs when $\varphi_1(x,y) = x$, $\varphi_2(x,y) = x + y$, and $\varphi_3(x,y) = x + y^2$, in which case we can set $F(a,b) = a + (b - a)^2$.

\begin{thm} \label{TheoremFourPointTwojdmrc} \textbf{(Theorem 4.2 of \cite{ChristTopicPaperjdmrc})}
    Suppose $\{ \varphi_j: \R^2 \to \R \}$ are analytic functions, and $x_0 \in \R^2$. If the vectors $\nabla \varphi_j(x_0)$ and $\nabla \varphi_k(x_0)$ are linearly independent at $x_0$ for all $j \neq k$, and in addition, the triple $\{ \varphi_j \}$ `has curvature' in the sense above, then for $\eta$ supported near $x_0$, and for any $p > 3/2$, there exists $s > 0$ such that
    %
    \begin{equation}
        \left\| A^\varphi_\eta(f_1,f_2,f_3) \right\|_{L^1(\R^2)} \lesssim \| f_1 \|_{W^{-s,p}(\R)} \| f_2 \|_{W^{-s,p}(\R)} \| f_3 \|_{W^{-s,p}(\R)}.
    \end{equation}
\end{thm}

We very briefly describe how Theorem 4.2 follows from Theorem 4.1. By interpolating, it suffices to prove
%
\begin{equation} \label{BoundToProve123jdmrc}
        \left\| A^\varphi_\eta(f_1,f_2,f_3) \right\|_{L^1} \lesssim \| f_1 \|_{L^2} \| f_2 \|_{L^2} \| f_3 \|_{H^{-s}}
\end{equation}
%
for some $s > 0$. A coordinate change allows us to assume that $\varphi_1(x,y) = x$ and $\varphi_2(y) = y$, and then our assumption is that for $\varphi = \varphi_3$, the mixed partial $\partial_1 \partial_2 \varphi$ is non-vanishing on the support of the $\eta$ we have picked. We then apply Fourier inversion to write
%
\begin{equation}
    f_3(\varphi(x,y)) = \int e^{i \tau \varphi(x,y)} \widehat{f}_3(\tau)\; d\tau.
\end{equation}
%
A dyadic decomposition allows us to write, modulo a smoothing term,
%
\begin{equation}
    f_3(\varphi(x,y)) = \sum\nolimits_\lambda \lambda^{1/2} \int a_\lambda( \tau ) e^{i \lambda \tau \varphi(x,y)}\; d\tau,
\end{equation}
%
where $\lambda \in \{ 2^j : j > 0 \}$, and where $a_\lambda(\tau) = \boldsymbol{1}_{[1/2,1]}(\tau) \lambda^{-1/2} \widehat{f}_3( \tau / \lambda) $ satisfies $\| a_\lambda \|_{L^2(\R)} \lesssim \lambda^{s/2} \| f_3 \|_{H^s(\R)}$. The function $(x,y,\tau) \mapsto \tau \varphi(x,y)$ is rank one non-degenerate, and so Theorem 4.1 implies that for some $\gamma > 1/2$,
%
\begin{equation}
    \left| \int_{\R^3} e^{i \lambda \tau \varphi(x,y)} f_1(x) f_2(y) a_\lambda(\tau) \right| \lesssim \lambda^{-\gamma} \| f_1 \|_{L^2} \| f_2 \|_{L^2} \| a_\lambda \|_{L^2}.
\end{equation}
%
Picking $0 < s < 2(\gamma - 1/2)$, and then summing in $\lambda$ gives \eqref{BoundToProve123jdmrc}.

\subsection{Preliminary reductions}

We now discuss the details of the proof of Theorem 4.1. Let us begin by making some preliminary reductinos. First off, for a fixed $\lambda$, we may assume that the functions $f_1$, $f_2$, and $f_3$ functions are band-limited, in the sense that their Fourier transforms are supported on the interval $[-\lambda, \lambda]$. This is because if we replace $f_1$ in $T^\phi_\lambda(f_1,f_2,f_3)$ by a pure high frequency function $e^{ix\xi}$, with $|\xi| \gg \lambda$, and then integrate by parts $N$ times in the $x_1$ variable using the identity
%
\begin{equation*}
    e^{i\xi x_1+i\lambda \phi(x)}= \big( [i\xi+ \partial_1 \phi (x)]^{-1} \partial_1 \big)^N e^{i\xi x_1+i\lambda \phi(x)},
\end{equation*}
%
then we find that $T_\lambda^\phi(e^{ix_1 \xi},f_2,f_3)\leq C_N|\xi|^{-N}||f_2||_1 ||f_3||_1$. So contributions to $T^\phi_\lambda(f_1,f_2,f_3)$ from high frequency components of $f_1$ are small, and thus for $f_2$ and $f_3$ as well by symmetry.

A second reduction we can make is the following. By interpolation with \eqref{HormanderL2L2L1Boundjdmrc}, and some tricks involving the mixed partials assumption, it suffices to prove that
%
\begin{equation}
    |T^\phi_\lambda(f_1,f_2,f_3)| \lesssim \lambda^{-\gamma} \| f_1 \|_{L^\infty} \| f_2 \|_{L^\infty} \| f_3 \|_{L^\infty}
\end{equation}
%
for some $\gamma > 1/2$. Without loss of generality, we may assume $||f_j||_\infty\leq 1$ for $j \in \{1,2,3 \}$, and we aim to prove $|T_\lambda^\phi(f_1,f_2,f_3)|\lesssim \lambda^{-\gamma}$ for some $\gamma>1/2$.

\subsection{Decomposition} \label{Decompositionjdmrc} We will now decompose each $f_j$ in phase space into wave packets that are essentially supported in $\lambda^{-1/2} \times \lambda^{1/2}$ rectangles in $[0,1]_x\times \R_\xi$, similar to those occuring in the proof of \eqref{HormanderBoundjdmrc}. To do this, first partition $[0,1]$ into $\sim \lambda^{1/2}$ intervals $I_m$ of length $|I_m|=\lambda^{-1/2}$. Let $\eta_m$ be $C^\infty$ functions with each $\eta_m$ supported on the interval $I_m^*$ of length $2\lambda^{-1/2}$ concentric with $I_m$, with $\sum_m \eta_m^2\equiv 1$ on $[0,1]$. Next, expand $f_j\eta_m$ into Fourier series
\begin{equation}\label{decomp1jdmrc}
    f_j(x)\eta_m(x)=\boldsymbol{1}_{I_m^*}(x)\sum\nolimits_{n\in \Z} c_{j,m,n} e^{i\pi\lambda^{1/2}nx},
\end{equation}
Define $g_{j,m}$ to be the sum of all terms with $|c_{j,m,n}|>\lambda^{-\sigma}||f_j||_\infty$, multiplied by $\eta_m$. Here $\sigma$ denotes a small positive constant to be chosen later. Define $h_{j,m}=f_j\eta_m-g_{j,m}$. Then we have the following properties. For each $I_m$, we get the decomposition
\begin{equation}
    f_j\eta_m^2=g_{j,m}+h_{j,m}
\end{equation}
with $g_{j,m},h_{j,m}$ identically zero outside of $I_m^*$, and
\begin{equation}
    \begin{cases}
    g_{j,m}(x)=\eta_m(x)\sum_{k=1}^N a_{j,m,k}e^{i\pi \lambda^{1/2}n_{k,j,m} x}\\
    |a_{j,m,k}|=O(||f_j||_\infty),\\
    N=\lceil\lambda^{2\sigma}\rceil.
\end{cases}
\end{equation}
%
We also have
%
\begin{equation}
    \begin{cases}
    h_{j,m}(x)=\eta_m(x)\sum_{n\in \Z} b_{j,m,n}e^{i\pi \lambda^{1/2}n x},\\
    \big(\sum_n |b_{j,m,n}|^2\big)^{1/2}=O(||f_j||_\infty),\\
    |b_{j,m,k}|=O(\lambda^{-\sigma}||f_j||_\infty).\\
\end{cases}
\end{equation}
Since the functions $f_j$ are band limited (by our reduction), we can neglect the tail terms in the above expansion of $h_{j,m}$. More precisely, we can write 
$f_j=g_j+h_j+F_j$ where $g_j=\sum_m g_{j,m}$, where $h_j$ is the sum over $m$ of all the terms in $h_{j,m}=\sum_{n\in \Z} \eta_m(x)b_{j,m,n}e^{i\pi\lambda^{1/2}nx}$ with frequency $n\leq \lambda^{1/2+\rho}$, and where $F_j$ is whatever is left over. Here $\rho<<\sigma$ is a parameter to be chosen later. Using the band limited assumption, we get that $||F_j||_2=O(\lambda^{-N})$ for any $N<\infty$ and hence these terms can be neglected. So henceforth we assume $f_j=g_j+h_j$.

\subsection{Reduction to sublevel set estimate}
We can expand $T_\lambda^\phi(\boldsymbol{f})$ using \eqref{decomp1jdmrc} to get
\begin{equation}
    T_\lambda^\phi(\boldsymbol{f})=\sum\nolimits_{\boldsymbol{m}}\sum\nolimits_{\boldsymbol{n}}\prod\nolimits_{j=1}^3 c_{j,m_j,n_j} \int e^{i\Phi_{\boldsymbol{n}}(\boldsymbol{x})}\eta_{\boldsymbol{m}}(\boldsymbol{x}) \, d\boldsymbol{x}
\end{equation}
where $\Phi_{\boldsymbol{n}(x)}=\lambda^{1/2}\boldsymbol{n}\cdot \boldsymbol{x}+\lambda \phi(\boldsymbol{x})$. We call a tuple $(\boldsymbol{n},\boldsymbol{m})$ \emph{stationary} if $|\nabla \Phi_{\boldsymbol{n}}(\boldsymbol{x}_{\boldsymbol{m}})|\leq \lambda^{1/2+\rho}$, where $\boldsymbol{x}_{\boldsymbol{m}}$ is the center of the cube $I_{m_1}\times I_{m_2}\times I_{m_3}$. We call other tuples non-stationary. The main contribution will be from the stationary tuples, so we can neglect the non-stationary tuples here.

Write $T^\phi_\lambda( \boldsymbol{f}) = T^\phi_\lambda(f_1,f_2,h_3) + T^\phi_\lambda(f_1,h_2,g_3) + T^\phi_\lambda(h_1,g_2,g_3) + T^\phi_\lambda(g_1,g_2,g_3)$. Let's begin by bounding $T_\lambda^\phi(f_1,f_2,h_3)$. A similar analysis, left out of this summary, will bound $T^\phi_\lambda(f_1,h_2,g_3)$ and $T^\phi_\lambda(h_1,g_2,g_3)$. Note $b_{j,m,n}\leq \lambda^{-\sigma}$, and the measure of any cube $I_{m_1}\times I_{m_2}\times I_{m_3}$ is $\lambda^{-3/2}$. Thus
%
\begin{align*}
    |T_\lambda^\phi(f_1,f_2,h_3)| &\leq O(\lambda^{-N}) + C \lambda^{-3/2-\sigma}\sum\nolimits_{\boldsymbol{m}}\sum\nolimits_{\boldsymbol{n}} |c_{1,m_1,n_1}c_{2,m_2,n_2}|\\
    &\leq O(\lambda^{-N}) + C \lambda^{-3/2-\sigma+C\rho}\sum\nolimits_{m_1,m_2}\sum\nolimits_{n_1,n_2} |c_{1,m_1,n_1}c_{2,m_2,n_2}|\\
    &\lesssim O(\lambda^{-N}) + C \lambda^{-3/2-\sigma+C\rho}\sum\nolimits_{m_1,m_2}O(1)\\
    &\lesssim O(\lambda^{-N}) + C \lambda^{-1/2-\sigma+C\rho}.
\end{align*}
%
It remains to bound $T^\phi_\lambda(g_1,g_2,g_3)$. Since each $g_{j,m}$ is a sum of at most $N\leq \lambda^{2\sigma}$ terms, we regroup and define $G_{j,k}=\sum_m \eta_m a_{j,m,k} e^{i\pi \lambda^{1/2}n_{k,j,m}x}$. Thus we get the decomposition $g_j=\sum_{k=1}^N G_{j,k}$. Therefore
\begin{equation}
    |T_\lambda^\phi (g_1,g_2,g_3)|\leq \sum\nolimits_{\boldsymbol{k}\in\{1,\ldots,N\}^3 } |T_\lambda^\phi (G_{1,k_1},G_{2,k_2},G_{3,k_3})|.
\end{equation}
If we can prove that $|T_\lambda^\phi (G_{1,k_1},G_{2,k_2},G_{3,k_3})|\lesssim \lambda^{-1/2-\epsilon}$ for some $\epsilon > 0$, then we conclude that $|T_\lambda^\phi (g_1,g_2,g_3)|\lesssim \lambda^{-1/2+6\sigma-\epsilon}$. But this means that $|T^\phi_\lambda(f_1,f_2,f_3)| \lesssim \lambda^{-1/2+6\sigma-\epsilon} + \lambda^{-1/2-\sigma+\rho}$, and the proof is completed by choosing $\sigma$ and $\rho$ appropriately.

Using the expansion $G_{j,k}=\sum_m \eta_m a_{j,m,k} e^{i\pi \lambda^{1/2}n_{k,j,m}x}$, we get
\begin{equation}
    |T_\lambda^\phi (G_{1,k_1},G_{2,k_2},G_{3,k_3})|\leq O(\lambda^{-N}) +C\lambda^{-3/2}\sum\nolimits_{\text{stationary }\boldsymbol{m}} O(1).
\end{equation}
Let $E$ denote the union of the stationary cubes. All that remains is to show that $|E|\lesssim \lambda^{-1/2-\epsilon}$. For this the following sublevel set lemma is key. We shall not be able to include a proof here, but the interested reader may find the proof in Section 13 of \cite{ChristTopicPaperjdmrc}.

\begin{lemma} \label{LevelSetLemmajdmrc}
    Suppose $\phi$ satisfies the assumptions of Theorem 4.1, i.e. so that for every distinct pair of indices $j\neq k\in\{1,2,3\}$, the mixed partial derivative $\partial_j \partial_k \phi$ is non-vanishing on the support of $\Tilde{\eta}$, and $\phi$ is not rank-one degenerate. Then there exists $\delta>0$ and $C<\infty$ such that for any $\epsilon_0\in (0,1]$ and any Lebesgue measurable real-valued functions $l_1,l_2,l_3$, the sublevel set 
    $
    \mathcal{E}=\big\{\boldsymbol{x}:|\partial_j \phi(\boldsymbol{x})-l_j(x_j)|\leq \epsilon_0 \text{  for each } j\in \{1,2,3\}\big\}
    $
    satisfies $|\mathcal{E}|\leq C \epsilon_0^{1+\delta}$.
\end{lemma}

This is a version of Lemma 13.1 of \cite{ChristTopicPaperjdmrc}, but with the implicit rank-one degeneracy condition stated explicitly. The version of the lemma with rank one degeneracy explicitly stated was found in the final slide of the presentation by Zhou \cite{Zhoujdmrc}. If we apply Lemma \ref{LevelSetLemmajdmrc} with $\epsilon_0=\lambda^{-1/2+\rho}$ and $l_j$ the constant function $-\lambda^{-1/2} n_{k,j,m}$, then $E=\mathcal{E}$ and so $|E|\lesssim (\lambda^{-1/2+\rho})^{1+\delta}=\lambda^{-1/2-\delta/2+\rho+\rho\delta}$. Choosing $\rho$ sufficiently small proves the required bound, completing the sketch of the proof of Theorem \ref{MainOscillatoryTheoremjdmrc}.




% since the functions $x \mapsto f_j(x_j)$, despite having no smoothness assumptions in the direction of the $x_j$ axis, are constant (and thus very smooth) along hyperplanes orthogonal to the $x_j$ axis, and thus $f$ cannot be . However, it still remains unclear \emph{exactly} what the relationship is in \eqref{MultilinearInequality} between the function $\phi$ and the optimal decay rate $\alpha$.

\begin{thebibliography}{03}

\bibitem{ChristTopicPaperjdmrc}
Christ, M., \emph{On Trilinear Oscillatory Integral Inequalities and Related Topics}.
Preprint (2022), \href{https://arxiv.org/pdf/2007.12753}{https://arxiv.org/pdf/2007.12753}.

\bibitem{Hormanderjdmrc}
H\"{o}rmander, L., \emph{Oscillatory Integrals and Multipliers on $FL^p$}. Ark. Mat. 11 (1973), 1--11.

\bibitem{Zhoujdmrc}
Zhou, Z., \emph{On Trilinear Oscillatory Integral Inequalities and Related Topics.} HAPPY (Harmonic Analysis People's Presentations on YouTube), https://www.youtube.com/watch?v=gn9vsPYuvkk.

\end{thebibliography}

\vspace{1em}
\noindent \textsc{Jacob Denson, University of Wisconsin, Madison}. \\
 \texttt{jcdenson@math.wisc.edu}\\
 
%%% IF YOU ARE WORKING IN PAIRS,  uncomment the following two lines and replace with the data of the other author
 
\noindent \textsc{Mukul Rai Choudhuri, University of British Columbia}. \\
 \texttt{mukul@math.ubc.ca}


%\newpage

\end{document}