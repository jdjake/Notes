\documentclass[12pt]{amsart}
\usepackage{amssymb,amsmath,amsthm}
\usepackage{esint}
\usepackage{mathrsfs}
\usepackage{mathtools}
\usepackage{bbm,dsfont}

\newtheorem{thm}{Theorem}
\newtheorem{lemma}[thm]{Lemma}
\newtheorem{corollary}[thm]{Corollary}
\newtheorem{conjecture}[thm]{Conjecture}
\newtheorem{definition}[thm]{Definition}
\newtheorem{example}[thm]{Example}
\newtheorem{remark}[thm]{Remark}
\newtheorem{proposition}[thm]{Proposition}

\addtolength{\hoffset}{-0.5cm}
\addtolength{\textwidth}{1cm}
 

\newcommand{\talktitle}[1]{\section{#1}}
\newcommand{\talkafter}[1]{\begin{center}{After #1} \end{center}
 \addcontentsline{toc}{subsection}{after #1}}
\newcommand{\talkspeaker}[2]{\begin{center}
{A summary  by #1}
\end{center}
\addcontentsline{toc}{subsection}{#1, #2}
}

\usepackage{graphicx}
\usepackage[utf8]{inputenc}
\usepackage[T1]{fontenc}
\usepackage[hidelinks]{hyperref}

\newcommand*{\Z}{\mathbb{Z}}
\newcommand*{\Q}{\mathbb{Q}}
\def\C{\mathbb{C}}
\newcommand*{\N}{\mathbb{N}}
\newcommand*{\R}{\mathbb{R}}

%Leave empty
\title{}
 


\begin{document}

\talktitle{On Trilinear Oscillatory Integral Inequalities and Related Topics}
\talkafter{M. Christ \cite{ChristTopicPaper}}
\talkspeaker{Jacob Denson and Mukul Rai Choudhuri}{University of Wisconsin, Madison and University of British Columbia}

\setcounter{equation}{0}
\setcounter{thm}{0}


\begin{abstract}
	We discuss decay estimates for trilinear oscillatory integrals on $L^2(\R) \times L^2(\R) \times L^2(\R)$ which give power decay in the frequency parameter with exponent having magnitude larger than $1/2$, under non-degeneracy conditions on the phase. The main consequence of these estimates, for the purpose of this summer school, are smoothing bounds for singular Brascamp-Lieb type multilinear forms, which in particular, include bounds on the $L^1(\R^2)$ norm of $f(x) g(x + y) h(x + y^2)$, in terms of negative index Sobolev estimates on the functions $f$, $g$, and $h$.
\end{abstract} 

 \maketitle



\subsection{Introduction} Consider an oscillatory integral of the form
%
\begin{equation} \label{ClassicalOscillatoryIntegral}
	T^\phi_\lambda(f) = \int_{\R^J} e^{i \lambda \phi(x)} f(x)\; dx,
\end{equation}
%
for a function $\phi: \R^J \to \R$ and $f: \R^J \to \C$. If $f: \R^J \to \C$ is appropriately smooth and $\phi$ is appropriately `non-stationary' %\footnote{The `non-stationarity' of a function $\phi$ is normally measured by non-vanishing of certain derivatives of $\phi$ or, as we discuss later, in terms of the size of level sets of $\phi$.}
then, as $\lambda \to \infty$, the integrand in \eqref{ClassicalOscillatoryIntegral} begins to oscillate faster and faster, and so we expect greater and greater cancellation to occur in the integral. In particular, if $\text{supp}(f)$ is contained in a fixed compact set $K \subset \R^J$, the principle of stationary phase can then guarantee a bound $| T^\phi_\lambda(f) | \lesssim_K \lambda^{-\gamma} \max\nolimits_{|\alpha| \leq n} \| \partial^\alpha f \|_{L^\infty(\R^J)}$ with power decay in $\lambda$, for appropriate exponents $\gamma$ and $n$.

If we remove the smoothness assumptions on $f$, then for an arbitrary input it is impossible to obtain any bound on \eqref{ClassicalOscillatoryIntegral} with a power decay in $\lambda$. If we set
% for a fixed compact set $K$, taking
$f_\lambda(x) = \chi_K(x) e^{-i \lambda \phi(x)}$, then $\| f_\lambda \|_{L^p(\R^J)} \lesssim_K 1$ and $|T^\phi_\lambda(f_\lambda)| = |K|$. Thus the only bound of the form
%
\begin{equation} \label{NoSmoothnessDecayEquation}
	| T^\phi_\lambda(f)| \lesssim_K \lambda^{-\gamma} \| f \|_{L^p(\R^J)}
\end{equation}
%
which can hold uniformly over all $\lambda > 0$ and all $f \in L^p(\R^J)$ supported on a compact set $K$ is the trivial bound with $\gamma = 0$. The main result about oscillatory inegrals we wish to discuss from \cite{ChristTopicPaper} establishes non-trivial decay estimates of the form \eqref{NoSmoothnessDecayEquation} for $p = 2$, $J = 3$, and $K = [0,1]^3$, under a \emph{structural assumption} on the functions $f$. Namely, we assume that $f$ can be written as $f_1 \otimes \dots \otimes f_J$ for some functions $f_1,\dots,f_J: \R \to \C$, i.e. so that $f(x) = f_1(x_1) \cdots f_J(x_J)$. Equation \eqref{NoSmoothnessDecayEquation} can then be written as a multilinear inequality of the form
%
\begin{equation} \label{MultilinearInequality}
	| T^\phi_\lambda(f_1 \otimes \cdots \otimes f_J) | \lesssim \lambda^{-\gamma} \| f_1 \|_{L^2(\R)} \cdots \| f_J \|_{L^2(\R)}.
\end{equation}
%
Unlike the generality of \eqref{NoSmoothnessDecayEquation}, the multilinearity in \eqref{MultilinearInequality} allows for cancellation, since the functions $f_\lambda$ are no longer counterexamples to \eqref{MultilinearInequality} with $\gamma > 0$ unless the phase $\phi$ can be written as
%
\begin{equation} \label{phaseTensorDecomposition}
	\phi(x) = \phi_1(x_1) + \dots + \phi_J(x_J)
\end{equation}
%
for some functions $\phi_1,\dots,\phi_J: \R \to \R$. One might hope that the best possible decay in \eqref{MultilinearInequality} is closely related to how `far' a phase function $\phi$ is from being decomposed as in \eqref{phaseTensorDecomposition}. Current research is far from an optimal quantitative understanding of this relation, but \cite{ChristTopicPaper} obtains improved decay estimates by making assumptions naturally related to preventing a decomposition of the form \eqref{phaseTensorDecomposition}, along with non-linear analogues, called \emph{rank one degeneracies}.

\subsection{Degeneracies}

One can prevent a decomposition of the form \eqref{phaseTensorDecomposition} by assuming a mixed derivative, such as $\partial_1 \partial_2 \phi$, is non-vanishing on $[0,1]^3$. It is then a result of H\"{o}rmander \cite{Hormander} that the operator
%
\begin{equation}
	S_zf(x) = \int_{[0,1]} e^{i \lambda \phi(x,y,z)} f(y)\; dy
\end{equation}
%
satisfies, uniformly for $z \in [0,1]$,
%
\begin{equation} \label{HormanderBound}
	\| S_z f \|_{L^2[0,1] \to L^2[0,1]} \lesssim \lambda^{-1/2}.
\end{equation}
%
Using \eqref{HormanderBound}, Cauchy-Schwartz and H\"{o}lder's inequality then implies
%
\begin{equation}
	|T^\phi_\lambda(f)| \lesssim \lambda^{-1/2} \| f_1 \|_{L^2[0,1]} \| f_2 \|_{L^2[0,1]} \| f_3 \|_{L^1[0,1]} \leq \lambda^{-1/2} \| f \|_{L^2[0,1]}.
\end{equation}
%
Thus \eqref{MultilinearInequality} holds with $\gamma = 1/2$. It will be enlightening to later discussion to briefly explain the proof of \eqref{HormanderBound} via a microlocal decomposition. For any $\lambda > 0$, and $f \in L^2[0,1]$, we can consider a decomposition of $f$ into wave packets, of the form
%
\begin{equation}
	f(y) = \sum\nolimits_{(y_0,\eta_0) \in \Phi_\lambda} f_{y_0,\eta_0}(y),
\end{equation}
%
where $\Phi_\lambda$ is the Cartesian product of $\lambda^{-1/2} \Z \cap [0,1]$ with $\lambda^{1/2} \Z$, where the function $f_{y_0,\eta}$ is supported on a sidelength $\lambda^{-1/2}$ interval centered at $y_0$, and where it's Fourier transform rapidly decays away from a sidelength $\lambda$ interval centered at $\eta_0$.
%
%\[ |\langle f_{y_0,\eta_0}, f_{y_0', \eta_0'} \rangle| \lesssim_N \Big( 1 + \lambda^{1/2} |y_0 - y_0'| + \lambda^{-1/2} |\eta_0 - \eta_0'| \Big)^{-N} \| f \|_{L^2[0,1]}^2. \]
%
Our assumption on $\phi$, roughly speaking, implies the existence of an injective map $(x_z,\xi_z): \Phi_\lambda \to \Phi_\lambda$ such that the function $S_z f_{y_0,\eta_0}$ rapidly decays away from a sidelength $\lambda^{-1/2}$ interval centered at $x_z(y_0,\eta_0)$, it's Fourier transform rapidly decays away from a sidelength $\lambda^{1/2}$ interval centered at $\xi_z(y_0,\eta_0)$, and $\| S_z f_{y_0,\eta_0} \|_{L^2[0,1]}^2 \lesssim \lambda^{-1} \| f_{y_0,\eta_0} \|_{L^2[0,1]}^2$. Thus the functions $\{ S_z f_{y_0,\eta_0} \}$ are also wave packets, which are essentially disjoint from one another in phase space. They are therefore almost orthogonal, implying that
%
\[
 \| S_z f \|_{L^2[0,1]}^2 \lesssim \sum_{y_0,\eta_0} \| S_z f_{y_0,\eta_0} \|_{L^2[0,1]}^2\\
	\lesssim \lambda^{-1} \sum_{y_0,\eta_0} \| f_{y_0,\eta_0} \|_{L^2[0,1]}^2 \lesssim \lambda^{-1} \| f \|_{L^2[0,1]}^2.
\]
%
Thus we have proved \eqref{HormanderBound}.

% Interacts with O( L^{1/2} ) other terms
%
% Int e( L phi(x,y,z) ) f(y) dy
% = e( L [phi(x,y_0,z) - phi_2(x,y_0,z) y_0] )  Int e( L phi_2(x,y_0,z) y + O(1) )
% = e( L [phi(x,y_0,z) - phi_2(x,y_0,z) y_0 ] ) f^( - L phi_2(x,y_0,z) )
%
%			So Should be supported on |-L phi_2(x,y_0,z) - eta_0| << L^{1/2}
% 			Since phi_{12} nonvanishing, this is a O(L^{-1/2}) neighborhood
% 			of solutions to phi_2(x,y_0,z) = -eta_0/L
%
% 			But only frequencies that are << L are relevant to the output
%
% Then the oscillation is a O(L) neighborhood of e( L phi(x,y_0,z) + eta_0 )
%
% L phi_2(x(y_0,eta_0)) = - eta_0
%
% e( L [phi(x,y_0,z) + L phi_2(x,y_0,z) y_0 ) f^( - L phi_2(x,y_0,z) )
%
% So rapidly decays away from |-L phi_2(x,y_0,z) - eta_0| << L
% Thus rapidly decays away from a O(1) neighborhood of solutions to L phi_2(x,y_0,z) = eta_0
% And in this neighborhood the phase is e( L [ phi(x,y_0,z) + eta_0 y_0 ] ), thus having Fourier support in a O(L) neighborhood of this phase
%
% Then the integral is non-negligible on a ~ L^{-1} neighborhood of x(y_0,eta_0)
%
% Frequency is e( L phi(x(y_0,eta_0), y_0, z) - eta_0 )

The main result of \cite{ChristTopicPaper} related to oscillatory integrals is that one can improve H\"{o}rmander's estimate under a slightly stronger assumption on $\phi$. Say $\phi$ is \emph{rank one degenerate} if there exists $\psi_1,\dots,\psi_J$, and a hypersurface $H$ such that the function $x \mapsto \phi(x) + \psi_1(x_1) + \dots + \psi_J(x_J)$ has gradient vanishing on $H$. Being rank one degenerate is a direct obstruction to proving \eqref{MultilinearInequality} with $\gamma > 1/2$, since if we set $f_\lambda(x) = e^{i \lambda (\psi_1(x_1) + \dots + \psi_J(x_J))} \chi(x)$ for a bump function $\chi = \chi_1 \otimes \dots \otimes \chi_J$ supported on a small neighborhood $x_0 \in H$, then by fibering the neighborhood of $x_0$ into lines transverse to $H$ and applying stationary phase on each line, we can conclude that $|T^\phi_\lambda f_\lambda| \gtrsim \lambda^{-1/2}$. Theorem 4.1 of \cite{ChristTopicPaper} says that this is essentially the only obstruction.

\begin{thm}[Theorem 4.1 of \cite{ChristTopicPaper}]
	Let $\phi: [0,1]^3 \to \R$ be real analytic, not rank one degenerate, and for each $i \neq j$ suppose $\partial_i \partial_j \phi$ is non-vanishing on $[0,1]^3$. Then \eqref{MultilinearInequality} holds for some $\gamma > 1/2$.
\end{thm}

The proof of Theorem 4.1 also involves a microlocal decomposition like that discussed in the proof of \eqref{HormanderBound}, but with an \emph{additional} decomposition into a `pseudorandom' part. We fix a small exponent $\sigma > 0$, to be optimized later. Then given $f \in L^2[0,1]$, we partition $\Phi_\lambda$ into $\Phi_{\lambda,0} + \Phi_{\lambda,1}$, where $(y_0,\eta_0) \in \Phi_{\lambda,0}$ if and only if $\| f_{y_0,\eta_0} \|_{L^2[0,1]} \gtrsim \lambda^{-\sigma} \| f \|_{L^2[0,1]}$. Almost orthogonality justifies that $\| f \|_{L^2[0,1]} \sim \sum \| f_{y_0,\eta_0} \|_{L^2[0,1]}^2$, which means that $\#(\Phi_{\lambda,0}) \leq \lambda^{2\sigma}$. The wave packets corresponding to the elements of $\Phi_{\lambda,0}$ are then the sparse part, and the wave packets in $\Phi_{\lambda,1}$ are the pseudorandom part.

In the remainder of this summary, we discuss how this implies certain smoothing inequalities for trilinear forms of singular Brascamp-Lieb type.

\subsection{Smoothing Inequalities}

One potentially surprising application of Theorem 4.1 is to results in which no oscillation is immediately apparent. Consider analytic functions $\varphi_j: \R^2 \to \R$ for $1 \leq j \leq 3$ with nowhere vanishing gradient. If the vectors $\nabla \varphi_j(x_0)$ are linearly independent at $x_0$, then for a bump function $\eta$ supported near $x_0$, it is simple to obtain an inequality of the form
%
\[ \left\| \eta \prod\nolimits_j f_j(\varphi_j) \right\|_{L^1(\R^2)} \lesssim \prod\nolimits_j \| f_j \|_{L^{3/2}(\R)}. \]
%
Moreover, one cannot replace $3/2$ with any smaller exponent. Nonetheless, other interesting inequalities might be obtained by 

% since the functions $x \mapsto f_j(x_j)$, despite having no smoothness assumptions in the direction of the $x_j$ axis, are constant (and thus very smooth) along hyperplanes orthogonal to the $x_j$ axis, and thus $f$ cannot be . However, it still remains unclear \emph{exactly} what the relationship is in \eqref{MultilinearInequality} between the function $\phi$ and the optimal decay rate $\alpha$.

\begin{thebibliography}{03}

\bibitem{ChristTopicPaper}
Christ, M., \emph{On Trilinear Oscillatory Integral Inequalities and Related Topics}.
Preprint 2020 (Revised 2022), \href{https://arxiv.org/pdf/2007.12753}{https://arxiv.org/pdf/2007.12753}.

\bibitem{Hormander}
H\"{o}rmander, L., \emph{Oscillatory Integrals and Multipliers on $FL^p$}. Ark. Mat. 11 (1973), 1--11.

%\bibitem{MV95jr}
%Melnikov, M. S., and Verdera, J.,  \emph{A geometric proof of the $L^2$ boundedness of the Cauchy integral on Lipschitz graphs}.  Inter. Math. Res. Not. 7 (1995), 325--331.




\end{thebibliography}

\vspace{1em}
\noindent \textsc{Jacob Denson, University of Wisconsin, Madison}. \\
 \texttt{jcdenson@math.wisc.edu}\\
 
%%% IF YOU ARE WORKING IN PAIRS,  uncomment the following two lines and replace with the data of the other author
 
\noindent \textsc{Mukul Rai Choudhuri, University of British Columbia}. \\
 \texttt{mukul@math.ubc.ca}


%\newpage

\end{document}