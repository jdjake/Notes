\documentclass{article}

\usepackage{amsmath}
\usepackage{amssymb}
\usepackage{amsthm}
\usepackage{esint}
\usepackage{comment}

\newcommand{\End}{\operatorname{End}}
\newcommand{\Mor}{\operatorname{Mor}}
\newcommand{\Id}{\operatorname{id}}
\newcommand{\visspace}{\text{\textvisiblespace}}
\newcommand{\Gal}{\text{Gal}}

\newcommand{\xor}{\oplus}
\newcommand{\ft}{\wedge}
\newcommand{\ift}{\vee}

\newcommand{\prob}{\mathbf{P}}
\newcommand{\expect}{\mathbf{E}}
\DeclareMathOperator{\Var}{\mathbf{V}}
\newcommand{\Ber}{\text{Ber}}
\newcommand{\Bin}{\text{Bin}}

\newcommand{\loc}[1]{#1_{\text{loc}}}

%\newcommand{\widecheck}[1]{{#1}^{\ft}}

\DeclareMathOperator{\diam}{\text{diam}}

\DeclareMathOperator{\QQ}{\mathbf{Q}}
\DeclareMathOperator{\ZZ}{\mathbf{Z}}
\DeclareMathOperator{\RR}{\mathbf{R}}
\DeclareMathOperator{\HH}{\mathbf{H}}
\DeclareMathOperator{\CC}{\mathbf{C}}
\DeclareMathOperator{\AB}{\mathbf{A}}
\DeclareMathOperator{\PP}{\mathbf{P}}
\DeclareMathOperator{\MM}{\mathbf{M}}
\DeclareMathOperator{\VV}{\mathbf{V}}
\DeclareMathOperator{\TT}{\mathbf{T}}
\DeclareMathOperator{\LL}{\mathcal{L}}
\DeclareMathOperator{\DD}{\mathcal{D}}
\DeclareMathOperator{\SW}{\mathcal{S}}
\DeclareMathOperator{\EC}{\mathcal{E}}

\DeclareMathOperator{\EE}{\mathbf{E}}
\DeclareMathOperator{\NN}{\mathbf{N}}
\DeclareMathOperator{\DQ}{\mathcal{Q}}
\DeclareMathOperator{\IA}{\mathfrak{a}}
\DeclareMathOperator{\IB}{\mathfrak{b}}
\DeclareMathOperator{\IC}{\mathfrak{c}}
\DeclareMathOperator{\IP}{\mathfrak{p}}
\DeclareMathOperator{\IQ}{\mathfrak{q}}
\DeclareMathOperator{\IM}{\mathfrak{m}}
\DeclareMathOperator{\IN}{\mathfrak{n}}
\DeclareMathOperator{\IK}{\mathfrak{k}}
\DeclareMathOperator{\ord}{\text{ord}}
\DeclareMathOperator{\Ker}{\textsf{Ker}}
\DeclareMathOperator{\Coker}{\textsf{Coker}}
\DeclareMathOperator{\emphcoker}{\emph{coker}}
\DeclareMathOperator{\pp}{\partial}
\DeclareMathOperator{\tr}{\text{tr}}

\DeclareMathOperator{\supp}{\text{supp}}

\DeclareMathOperator{\codim}{\text{codim}}

\DeclareMathOperator{\minkdim}{\dim_{\mathbf{M}}}
\DeclareMathOperator{\hausdim}{\dim_{\mathbf{H}}}
\DeclareMathOperator{\lowminkdim}{\underline{\dim}_{\mathbf{M}}}
\DeclareMathOperator{\upminkdim}{\overline{\dim}_{\mathbf{M}}}
\DeclareMathOperator{\lhdim}{\underline{\dim}_{\mathbf{M}}}
\DeclareMathOperator{\lmbdim}{\underline{\dim}_{\mathbf{MB}}}
\DeclareMathOperator{\packdim}{\text{dim}_{\mathbf{P}}}
\DeclareMathOperator{\fordim}{\dim_{\mathbf{F}}}

\DeclareMathOperator{\CT}{ {\hat{\otimes}} }

\DeclareMathOperator{\msupp}{\text{$\mu$-supp}}
\DeclareMathOperator{\singsupp}{\text{sing-supp}}
\DeclareMathOperator{\Char}{\text{Char}}

\DeclareMathOperator*{\argmax}{arg\,max}
\DeclareMathOperator*{\argmin}{arg\,min}

\DeclareMathOperator{\ssm}{\smallsetminus}

\theoremstyle{plain}
\newtheorem{theorem}{Theorem}
\newtheorem{lemma}[theorem]{Lemma}
\newtheorem{corollary}[theorem]{Corollary}
\newtheorem{prop}[theorem]{Proposition}

\theoremstyle{remark}
\newtheorem*{example}{Example}
\newtheorem*{remark}{Remark}

\theoremstyle{definition}
\newtheorem*{defi}{Definition}
\newenvironment{definition}
    {\begin{samepage}\begin{framed}\begin{defi}}
    {\end{defi}\end{framed}\end{samepage}}

\title{Logarithmic Improvements in $L^p$ Bounds for Eigenfunctions at the Critical Exponent}
\author{Jacob Denson}
\date{August 2nd, 2022}

\begin{document}

\maketitle

Given a compact Riemannian manifold $M$, we can consider the Laplace-Beltrami operator $\Delta$ on $M$. This operator induces an eigenspace decomposition
%
\[ L^2(M) = \bigoplus_{\lambda \in \Lambda_M} E_\lambda, \]
%
where $\Lambda_M \subset [0,\infty)$ is a discrete set of eigenvalues, each $E_\lambda$ is a finite dimensional subspace of $C^\infty(M)$, and for $u \in E_\lambda$,
%
\[ \Delta u = - \lambda^2 u. \]
%
A natural question is to study the asymptotic behaviour of elements of $E_\lambda$ as $\lambda \to \infty$. In particular, in this talk, we focus on determining the asymptotic behaviour of the $L^p$ norms of functions in $E_\lambda$, i.e. the study of the quantities
%
\[ A(\lambda,p) = \sup_{f \in E_\lambda} \frac{\| f \|_{L^p(M)}}{\| f \|_{L^2(M)}}. \]
%
Understanding the quantities $A(\lambda,p)$ is a natural analogue of the Tomas-Stein theorem, since that theorem is equivalent to establishing that for any function $f \in L^2(\RR^d)$ with $\Delta f = - \lambda^2 f$,
%
\[ \frac{\| f \|_{L^p(M)}}{\| f \|_{L^2(M)}} \lesssim \lambda^{\delta(p,d)}, \]
%
where
%
\[ \delta(p,d) = \begin{cases} d( \frac{1}{2} - \frac{1}{p} ) - 1/2 &: p_c \leq p \leq \infty \\ \frac{d-1}{2} \left( \frac{1}{2} - \frac{1}{p} \right) &: 2 \leq p \leq p_c \end{cases}, \]
%
and where $p_c = 2(d+1)/(d-1)$ is the \emph{critical exponent} for the Tomas-Stein theorem. In particular, at the critical exponent we have $\delta(p_c,d) = 1/p_c$:

In 1988, Sogge showed that $A(\lambda,p) \lesssim \lambda^{\delta(p,d)}$ on a general compact Riemannian manifold, actually showing the stronger conclusion that the operator norm of the \emph{spectral band projection operators}
%
\[ \chi_{[\lambda,\lambda + 1]} f = \sum_{\lambda \leq \lambda' \leq \lambda + 1} f_{\lambda'} \]
%
from $L^2(M)$ to $L^p(M)$ have magnitude $O(\lambda^{\delta(p,d)})$. This is more powerful than an $L^p$ bound on an individual eigenfunction because the Weyl law says that the dimension of $\bigoplus_{\lambda \leq \lambda' \leq \lambda + O(1)} E_{\lambda'}$ is $\Theta(\lambda^{d-1})$, and thus the spaces $\{ E_\lambda \}$, which are orthogonal in $L^2(M)$, `remain orthogonal' in $L^p(M)$ to some capacity, i.e.
%
\[ \left\| \sum_{\lambda \leq \lambda' \leq \lambda + 1} a_{\lambda'} e_{\lambda'} \right\|_{L^p(M)} \lesssim \lambda^{\delta(p,d)} \left( \sum_{\lambda \leq \lambda' \leq \lambda + 1} |a_{\lambda'}|^2 \right)^{1/2}. \]
%
The \emph{exponent} here is tight for any manifold $M$, and any exponent $p$, i.e.
%
\[ \| \chi_{[\lambda,\lambda+1]} \|_{L^2(M) \to L^p(M)} \gtrsim \lambda^{\delta(p,d)}. \]
%
However, that does \emph{not} mean that the bound $A(\lambda,p) \lesssim \lambda^{\delta(p,d)}$ is in general tight, since an extremizer for the operator norm above could be composed of a superposition of eigenfunctions corresponding to several different clustered `resonant' eigenvalues.

The bound \emph{is} sharp for the sphere $S^d$. The eigenfunctions here are the spherical harmonics, and $\Lambda_M = \{ \lambda_n : n \geq 0 \}$, where
%
\[ \lambda_n = \sqrt{n(d + n - 1)}. \]
%
For each $n$, we consider two important elements of $E_{\lambda_n}$, the \emph{high weight harmonic}
%
\[ H_n(x) = (x_1 + ix_2)^d \] 
%
and the \emph{zonal harmonic} $Z_n(x) = L_n(x_n)$, where $L_n$ is the Legendre polynomial of degree $n$. The high weight harmonic is concentrated near the equator
%
\[ \{ x_3 = \dots = x_d = 0 \}, \]
%
which means the harmonic should have large $L^p$ norm when $p$ is small, because most points on a sphere are close to the equator. The zonal harmonic is concentrated near the poles of the sphere, which should give large $L^p$ norm for large $p$. And indeed, we find that for $p_c \leq p \leq \infty$,
%
\[ \| Z_n \|_{L^p(S^n)} \gtrsim \lambda^{\delta(n,p)}, \]
%
and for $2 \leq p_c \leq \infty$,
%
\[ \| H_n \|_{L^p(S^n)} \gtrsim \lambda^{\delta(n,p)}. \]
%
Thus the bound $A(\lambda,p) \lesssim \lambda^{\delta(p,d)}$ \emph{is sharp} on a sphere.

%and
%
%\[ \dim(E_{\lambda_n}) = \frac{2n + d - 1}{n} {{n + d - 2}\choose{n-1}} = \frac{\pi^{(d+1)/2}}{d \cdot \Gamma \left( \frac{d+3}{2} \right)} n^{d-1} + o(1). \]
%
%Let us denote this quantity by $N_{n,d}$. $(x_1 + i x')^d$
% ( \sqrt{d-1} x_1 + i x' )^d
%For each $n$, there exists $N_{n,d}$ unit vectors $\xi_1,\dots,\xi_{N_{n,d}}$ such that a general element of $E_{\lambda_n}$ can be written as
%
%\[ x \mapsto \sum_{k = 1}^{N_{n,d}} a_k L_n(\xi_k \cdot x), \]
%
%where $L_n$ is the $n$th Legendre polynomial. Each of the functions $L_n(\xi_k \cdot x)$ oscillate in different directions on the sphere, and are sharply peaked on the poles, i.e. they are roughly equal to $n^{(d-1)/2}$ on a ball of radius $1/n$ about $\xi_k$. This implies these polynomials are quite orthogonal to one another, and so to maximize the $L^p$ norm when $p$ is large, it makes sense to consider a single element of this sum, i.e. a \emph{zonal spherical harmonic}. TODO: Is this analogous to the Knapp example in restriction theory?, and saturates the bound as $n \to \infty$ for $p_c \leq p \leq \infty$. For $2 \leq p \leq p_c$, we consider a linear combination of all $N_{n,d}$ different elements of the sum to yield a function that is distributed more widely on the sphere, obtaining the \emph{high weight harmonic} $x \mapsto n^{(d-1)/4}$, and this yields a family saturating the bound for $2 \leq p \leq p_c$.

On the other hand, the bound is \emph{not} tight on the torus $\TT^d = \RR^d / \ZZ^d$. The set $\Lambda_{\TT^d}$ here is precisely the set of all integers $n$ for which $n^2$ can be written as the sum of $d$ perfect squares, and for each such $n$, the eigenspace $E_n$ is the linear space of the exponentials $e_m(x) = e^{i m x}$, where $m \in \ZZ^d$ and $n = |m|$. The dimension of this space is thus equal to the number of representations of $n^2$ as a sum of squares of $d$ inegers. The number of solutions is, heuristically (by Vinogradov type results), about $O(n^{d-2})$. The triangle inequality gives that
%
\begin{align*}
    \| \sum a_m e_m \|_{L^\infty(\TT^d)} &\lesssim \sum |a_m| \| e_m \|_{L^\infty(\TT^d)}\\
    &\lesssim n^{\frac{d-2}{2}} \left( |a_m|^2 \right)^{1/2}\\
    &\lesssim n^{\frac{d-2}{2}} \| \sum a_m e_m \|_{L^2(\TT^d)}.
\end{align*}
%
Thus
%
\[ A(\lambda,\infty) \lesssim \lambda^{\delta(\lambda,\infty) - \frac{1}{2}}, \]
%
i.e. we get a $1/2$ improvement in the exponent than on the sphere. Interpolation gives an improvement in the exponent of the form
%
\[ A(\lambda,p) \lesssim \lambda^{\delta(p,d) - \varepsilon(p,d)} \]
%
where $\varepsilon(p,d) > 0$ for all $2 < p < \infty$. Thus on $\TT^d$, the general $L^p$ bounds for eigenfunctions are \emph{not tight}.

In fact, we should expect that in general, this bound is \emph{not} tight. In Zhongkai's talk based on (Sogge, Zelditch, 2009), and in Ben and Jaume's talk based on (Sogge, Zelditch, 2016), we saw that a generic Riemannian manifold $M$ satisfies
%
\[ A(\lambda,p) = o( \lambda^{\delta(p,d)} ), \]
%
namely, we should only expect the bound $A(\lambda,p) \lesssim \lambda^{\delta(p,d)}$ to be tight if $M$ has many recurrent geodesics.

On negatively curved compact manifolds, recurrent geodesics are quite rare. For example, on the spaces of the form $\HH / \Gamma$ discussed in the talk by James and Rajula, based on (Marshall, 2016), there are only countably many periodic geodesics. And on a negatively curved, simply connected compact manifold, there do not exist \emph{any} geodesics that return to their original position. Indeed, if we take any three points on a hypothetical recurrent geodesic, these three points form a geodesic triangle with total angle greater than $2\pi$, which is impossible by the Gauss-Bonnet theorem. Thus we expect to make improvements to the bound above on a general compact, negatively curved manifold.

In Hongki's talk, based on (Sogge, 2017), we saw that on a negatively curved compact manifold, one could obtain estimates of the form
%
\[ A(\lambda,p) \lesssim (\log \log \lambda)^{-\varepsilon(p,d)} \lambda^{\delta(p,d)} \quad\text{for all $2 < p < \infty$}. \]
%
In Chamsol's talk (Blair, Sogge, 2018), we saw that some Kakeya-Nikodym estimates yielded the logarithmic improvements
%
\[ A(\lambda,p) \lesssim  (\log \lambda)^{-\sigma(p,d)} \lambda^{\delta(p,d)} \quad\text{for $2 < p < p_c$}. \]
%
The goal of this talk is to discuss the methods of (Blair, Sogge, 2019), which allow us to obtain such a bound for $p = p_c$, and thus to obtain that $\sigma(p,d) > 0$ for all $2 < p < \infty$. We will do this by obtaining bounds of the form
%
\[ \| \chi_{[\lambda,\lambda + 1/T]} \|_{L^2(M) \to L^{p_c}(M)} \lesssim (\log \lambda)^{- \sigma(p_c,d)} \lambda^{1/p_c} \]
%
where $T \sim \log \lambda$.

To gain intuition about the intricacies of the critical exponent, let us compute some examples that hint that this result is tight, by considering an analogous problem in the Euclidean setting, namely, bounding the operators
%
\[ \chi_{[\lambda,\lambda+1/T]} f = \int_{\lambda \leq |\xi| \leq \lambda + 1} \widehat{f}(\xi) e^{2 \pi i \xi \cdot x}. \]
%
from $L^p(\RR^d)$ to $L^2(\RR^d)$. Here simply applying the standard Tomas-Stein restriction theorem, and rescaling the variables shows that we should expect to have
%
\[ \| \chi_{[\lambda,\lambda+1/T]} \|_{L^p(\RR^n) \to L^2(\RR^n)} \sim T^{-1} \lambda^{\delta(p,d)}. \]
%
If $T \sim \log \lambda$, the operator norm is $O((\log \lambda)^{-1} \lambda^{ \delta(p,d)})$, which hints that this is the value of $T$ we should aim for if we expect to get logarithmic improvement in $L^p$ bounds for eigenfunctions.

\begin{comment}
Let us consider some functions that show the operator norm bound above is tight. These should be familiar from the proof that the Stein-Tomas theorem has the right exponents. For $2 < p \leq p_c$, if we apply stationary phase to a Schwartz function $\phi$ with frequency support on $\lambda \leq |\xi| \leq \lambda + 1/T$, and roughly constant on this annulus, then we conclude that $\| \chi_{[\lambda,\lambda + 1/T]} \phi \|_{L^p(\RR^d)} \gtrsim (1/T) \lambda^{\delta(p,d)} \| \phi \|_{L^2(\RR^d)}$. For $p_c \leq p \leq \infty$, we consider a Knapp-type example, i.e. an $L^2$ normalized Schwartz function $\phi$ with frequency support on a subset of frequencies $\lambda \leq |\xi| \leq \lambda + 1/T$ which point within an angle $O(\lambda^{-1/2})$ of a specific direction, then $\| \chi_{[\lambda,\lambda + 1/T]} \phi \|_{L^p(\RR^d)} \gtrsim (1/T) \lambda^{\delta(p,d)}$. The important thing to notice is that both results give extremizers at the critical exponent, and `interpolating' between them leads to a whole spectrum of extremizers, i.e. any Schwartz function $\phi_\theta$ with frequency support on a subset of frequencies $\lambda \leq |\xi| \leq \lambda + 1/T$ that point with an angle $\theta$ of a specific direction, where $\theta \in [\lambda^{-1/2},1]$. This hints that on a Riemannian manifold, at the critical exponent, we should expect a whole spectrum of scenarios which are very close to making the inequality fail, rather than just by analyzing mass concentration on $O(\lambda^{-1/2})$ tubular neighborhoods, which worked in the setting $p_c < p < \infty$ in (Sogge, 2017). However, we will find modifying this technique slightly leads to an analysis of mass concentration on $O(\lambda^{\varepsilon - 1/2})$ tubes for any fixed $\varepsilon \in (0,1/2)$, and this will lead to the improved result found here.
\end{comment}

%Let us consider some basic examples:
%
%\begin{itemize}
%    \item The eigenfunctions of the Laplace Beltrami operator on $S^d$ are the spherical harmonics. Each eigenvalue of the Laplace Beltrami operator is of the form $\lambda_n = \sqrt{n(d + n - 1)} = n + o_d(1)$, and the vector space of eigenfunctions to the operator with eigenvalue $\lambda_n$ has dimension
    %
%    \[ N_{n,d} = \frac{2n + d - 1}{n} {{n + d - 2}\choose{n-1}} = \frac{\pi^{(d+1)/2}}{d \cdot \Gamma \left( \frac{d+3}{2} \right)} n^{d-1} + o(1). \]
    %
%    There exists vectors $\xi_1,\dots,\xi_{N_{n,d}}$ such that any eigefunction with eigenvalue $\lambda_n$ can be written as
    %
%    \[ x \mapsto \sum_{k = 1}^{N_{n,d}} a_k L_n( \xi_k \cdot x ) \]
    %
%    where $L_n$ is the Legendre polynomial of degree $n$, and $\{ a_k \}$ are constants.

    % Number of degree n harmonic polynomials: (2n + d - 1)/n  {(n + d - 2) choose (n-1)}

%    \item The torii $\RR^d / \Lambda$, where $\Lambda$ is some lattice in $\RR^d$.
%\end{itemize}

%Basic examples include the spherical harmonics
%In Zhongkai's talk on the paper of Sogge, Toth, and Zelditch (2009), we saw that 


\section{Some Familiar Tools}

Set $P = \sqrt{-\Delta}$. As with many talks in this summer school, To begin our analysis, we introduce a very useful tool to understand a multiplier operator of the form
%
\[ m f = m \left( P \right) f = \sum_{\lambda \in \Lambda_M} m(\lambda) f_\lambda, \]
%
This tool is just the Fourier inversion formula in disguise, i.e. we can write
%
\[ m \left( P \right) f = \int_{-\infty}^\infty \widehat{m}(t) e^{2 \pi i t P} f\; dt. \]
%
The function $u = e^{2 \pi i t P} f$ solves the \emph{half wave equation} $\partial_t u = 2 \pi i P u$ with initial conditions $f$, and so one can understand the behaviour of the multiplier $m(P)$ by studying the behaviour of the half wave equation at times where the mass of $\widehat{m}$ is concentrated.

The uncertainty principle hints that to understand $\chi_{[\lambda,\lambda + 1/T]}(P)$, we must understand the behaviour of the wave equation on times $|t| \lesssim T$. We can make this intuition even more precise by replacing the multiplier $\chi_{[\lambda,\lambda + 1/T]}(P)$ with the operator $\rho_{\lambda,T}(\lambda') = \rho(T(\lambda - \lambda'))$, where $\rho \in \mathcal{S}(\RR)$ is a fixed, even function, with $\rho(0) = 1$ and where $\widehat{\rho}$ is compactly supported on $\{ 1/4 \leq |t| \leq 1/2 \}$. Then $|m_\lambda| \lesssim |\rho_\lambda|$, which implies that
%
\[ \| m_\lambda \|_{L^2(M) \to L^{p_c}(M)} \lesssim \| \rho_\lambda \|_{L^2(M) \to L^{p_c}(M)}. \]
%
Thus it suffices to bound the multiplier operators $\rho_{\lambda,T}$, whose Fourier transform is supported on $T/4 \leq |t| \leq T/2$, and it is for these times that we must understand the solution to the wave equation.

The best control of the wave equation occurs for small times, when we have a \emph{parametrix} for the half-wave equation. More precisely, if $(a,U)$ is a coordinate chart, and $K \subset U$ is compact, then there exists $T_0$ such that for $|t| \leq T_0$, we have a \emph{parametrix} for the wave equation for initial data on $K$. Namely, we can write $e^{2 \pi i t P}$ as an oscillatory integral, namely for $f \in L^2(M)$ with $\text{supp}(f) \subset K$,
%
\[ e^{2 \pi i t P} f \approx \int \int v(t,x,y,\xi) e^{2 \pi i \Phi(t,x,y,\xi)} f(y)\; dy\; d\xi \]
%
where:
%
\begin{itemize}
    \item $v$ is a symbol of order zero such that $\text{supp}_x(v)$ is a compact subset of $U$.
    \item If $\nabla_\xi \Phi(t,x,y,\xi) = 0$, then
    %
    \[ \nabla_y \Phi(t,x,y,\xi) = \xi \quad\text{and}\quad \nabla_x \Phi(t,x,y,\xi) = \exp_y( t \xi). \]
    %
    Thus stationary phase hints that the mass of the wave equation travels microlocally along the geodesics of the manifold $M$.
\end{itemize}
%
This approximation only holds for small times. The fact that $\{ e^{2 \pi i t P} \}$ is a semigroup means we can compose these operators to give integral representations for slightly larger times, but the 'fuzz' in the approximation grows worse and worse as we continue to compose these operators. There is a heuristic that suggests that his fuzz starts to dominate after the \emph{Ehrenfast time} for a general geodesic flow; given $f$ with $\text{supp}_\xi(f) \subset \{ |\xi| \sim \lambda \}$, the fuzz starts to dominate past times $\log \lambda$. Thus we expect that pushing past the time, and thus obtaining tighter estiamtes for projections operators over small times, requires us to either apply different techniques, or find a way to push past the Ehrenfast time.

\begin{comment}
. That is, there exists a pair of operators $Q(t)$ and $R(t)$ for $|t| \leq T_0$ such that for $f \in L^2(M)$ with $\text{supp}(f) \subset K$,
%
\[ e^{2 \pi i t \sqrt{-\Delta}} f = Q(t) f + R(t) f, \]
%
where $Q(t)$ is a Fourier integral, and $R(t)$ is a smoothing operator. More precisely:
%
\begin{itemize}
    \item $R(t)$ is an operator whose kernel is in $C^\infty([-T_0,T_0], M \times M)$.

    \item We can write $Q(t) = \sum_i Q_i(t)$, where $i$ ranges over a finite family of coordinate systems $\{ (a_i, U_i) \}$ which cover $M$, the kernel of $Q_i(t)$ is contained in a common precompact subset of $U_i$ for all $|t| \leq T_0$, and in the coordinate system $(a_i,U_i)$ we can write
    %
    \[ Q_i(t) f = \int \int q_i(t,x,y,\xi) e^{2 \pi i \Phi_i(t,x,y,\xi)} f(y)\; dy\; d\xi, \]
    %
    where $q_i \in C^\infty([-T_0,T_0], S^0)$, and
    %
    \[ \Phi_i(t,x,y,\xi) = (x - y) \cdot \xi + t |\xi|_g + O(|x - y|^2 |\xi|). \]
\end{itemize}
%
This gives us sharp control over the solution for $|t| \leq T_0$, and is the primary tool in (Sogge, 1988) to bound the operators $\chi_{[\lambda,\lambda + 1]}$. Similarily, if $c$ is a very small, fixed number, and we define the \emph{local projectors} $\sigma_\lambda = \rho( T_0 (\lambda - P))$, then we should expect to be able to make the most effective generalization of techniques from the Euclidean wave equation to the study of the operator $\sigma_\lambda$. But since we want to study operators with $T$ depending on $\lambda$, and thus for $\lambda \gtrsim 1$ this parametrix will not even directly apply.

Nonetheless, there is some hope we can apply this result in some way. This is because for any $\lambda \in \RR$ we have $(1 - \rho(\lambda)) \lesssim |\lambda|$, and since $\rho(T \lambda)$ is supported on $|\lambda| \leq 1/T$, we conclude that
%
\[ (1 - \rho(T_0 \lambda)) \cdot \rho(T \lambda) \lesssim_N T_0 |\lambda| (1 + T |\lambda|)^{-N} \lesssim_N T^{-1} (1 + T |\lambda|)^{-N}. \]
%
Thus we find that the $L^2(M) \to L^{p_c}(M)$ operator norm of the difference between $\rho_\lambda$ and $\sigma_\lambda \circ \rho_\lambda$ is upper bounded (up to a constant depending on $N$) by the $L^2(M) \to L^{p_c}(M)$ operator norm of
%
\[ T^{-1} \sum_{l = 1}^\infty (1 + T |\lambda - l|)^{-N} \chi_{[l,l+1]} \]
%
which is upper bounded by
%
\[ T^{-1} \sum_{l = 1}^\infty (1 + T |\lambda - l|)^{-N} \lambda^{1/p_c} \lesssim T^{-1} \lambda^{1/p_c}. \]
%
Thus it suffices to analyze the operators $\sigma_\lambda \circ \rho_\lambda$, which now seems more tractable via the parametrix method.
\end{comment}




\section{Microlocal Decomposition}

On $\RR^d$, for each $\theta \in (0,1)$, it is natural to decompose $\RR^n$ into $N = O(1/\theta)$ disjoint conic sectors $\Gamma_1,\dots,\Gamma_N$ with aperture $O(\theta)$, and consider decompositions of the form $f = f_1 + \dots + f_N$, where $\widehat{f_i}$ is supported on $\Gamma_i$ for each $i$. Our goal is to consider an analogous decomposition for functions on a Riemannian manifold, in a way that respects the geodesics, at least locally in space.

Let $\chi: \RR \times T^*M \to T^*M$ denote the geodesic flow map. Let us consider some local coordinate system $(s,x) \in \RR \times \RR^{d-1}$ for $M$, defined near the origin, and suppose that in this coordinate system $g^{ij}(0) = \delta^{ij}$. The map $\chi$, restricted to $\RR \times (T^* M|_U) \to T^* M$, is a diffeomorphism in a neighborhood of $(0,0,e_1)$. We can therefore find an open set $V \subset \RR^d$ of the origin, and $W \subset S^{d-1}_\xi$, such that we can Consider a locally defined inverse
%
\[ \chi( \iota(x,\omega), (0,\Phi(x,\omega)), \Psi(x,\omega) ) = (x,\omega). \]
%
We now use these functions to construct analogues of the conic localization above on a general Riemannian manifold.

n space.

Let $\chi: \RR \times T^*M \to T^*M$ denote the geodesic flow map. Let us consider some local coordinate system $(s,x) \in \RR \times \RR^{d-1}$ for $M$. If $x_0 \in U \subset \RR^{d-1}$, and $W \subset S^{d-1}_\xi$ is a small enough open neighborhood around $(1,0)$, then there is an open set $V \subset \RR^d$ containing $\{ 0 \} \times U$, and there is a map
%
\[ \Psi: V \times W \to U, \]
%
such that $\Psi(x,\omega)$ is the unique point on $U$ such that the geodesic passing through $x$ in the direction $\omega$ passes through $\{ 0 \} \times U$ in the direction 


Consider a small conic neighborhood $\Gamma$ of some vector $\xi \in S^{d-1}$. Consider a maximal $O(\lambda^{-1/8})$ separated family of vectors $\nu$ on $\Gamma$, and consider a partition of unity $\{ \beta_\nu \}$ on $\Gamma$ such that the support of the functions $\beta_\nu$ is contained on a diameter $O(\lambda^{-1/8})$ cap centered at $\nu$. Define a symbol $q_{\lambda,\nu}$ supported on a small neighborhood of $V$, such that on $V$,
%
\[ q_{\lambda,\nu}(x,\xi) \approx \beta_\nu(\Psi(x, \xi / |\xi|_g ))   \]
%
Thus $q_{\lambda,\nu}$ is supported on a $O(\lambda^{-1/8})$ neighborhood of the geodesic passing through $\nu$, at least on the set $V$, the collection $\{ q_{\lambda,\nu} \}$ forms a partition of unity on $V$. Most importantly, the functions $\{ q_{\lambda,\nu} \}$ are invariant under the geodesic flow. We also have bounds of the form
%
\[ |\nabla_x^n \nabla_\xi^m q_{\lambda,\nu}(x,\xi)| \lesssim_{n,m} \lambda^{(1/8)(n + m)}. \]
%
Given these symbols, we define the semiclassical pseudodifferential operators $Q_\nu$, which have kernel
%
\[ \lambda^d \int q_{\lambda,\nu}(x,\xi) e^{2 \pi i \lambda \xi \cdot (x - y)}\; d\xi. \]
%
In the remaining part of this lecture, we will see how to individually bound terms that occur in our understanding of the equation once decomposed using the operators $\{ Q_\nu \}$. It is also very important that these decompositions can be recombined in an optimal way. On Thursday, Hong will show that these operators satisfy almost orthogonality conditions that will enable one to recombine these terms, and thus prove the required result.



\section{Bounding Individual Terms}

Our goal is now to understand the operators $\{ \rho_\lambda \}$ as $\lambda \to \infty$. Using the decompositions we just introduced, we write
%
\[ \rho_\lambda = \sum_\nu (q_{\lambda,\nu} \circ \rho_\lambda). \]
%
For now, we understand each of these terms individually, without putting them back together. Thus we wish to understand an element of the form $q_\lambda \circ \rho_\lambda$, and in particular, we focus on obtaining estimates of the form
%
\[ \| q_\lambda \circ \rho_\lambda \|_{L^{p_c,\infty}(M)} \lesssim \frac{\lambda^{1/p_c}}{(\log \lambda)^\varepsilon}. \]
%
Thus fix $\alpha > 0$, and consider a set $A \subset M$ such that for $x \in A$,
%
\[ |(q_\lambda \circ \rho_\lambda) f (x)| \geq \alpha. \]
%
Our goal is to show that
%
\[ \alpha |A|^{1/p_c} \lesssim \frac{\lambda^{1/p_c}}{(\log \lambda)^{\varepsilon / p_c}}. \]
%
Set
%
\[ r = \lambda \alpha^{\frac{-4}{d-1}} (\log \lambda)^{- \frac{2}{d-1}} \]
%
Hongki's talk actually already addresses the case whe $\alpha \geq \lambda^{\frac{d-1}{4} + \frac{1}{8}}$, so it suffices to consider $\alpha < \lambda^{\frac{d-1}{4} + \frac{1}{8}}$. In this case, $r > \lambda^{-\frac{1}{2(d-1)}} (\log \lambda)^{- \frac{2}{d-1}} \gg \lambda^{-3/4}$.

Without loss of generality, throwing away all but a fixed percentage, say $10\%$ of $A$, we may assume without loss of generality that we can write $A = A_1 \cup \dots \cup A_N$, where each of the sets $\{ A_i \}$ has diameter at most $r$, and $d(A_i,A_j) \gtrsim r$ for $i \neq j$.

Write $a_i = \chi_{A_i} \cdot \text{Sgn}(Q_\lambda)$. Then
%
\[ \alpha |A| \leq \left| \int_M (Q_\lambda \circ \rho_\lambda) f \cdot \text{Sgn}(Q_\lambda) \chi_A \right| \lesssim \left( \int_M \left( \sum_{i = 1}^N (\rho_\lambda \circ Q_\lambda^*) a_i \right)^2 \right)^{1/2}. \]
%
Then
%
\begin{align*}
    \alpha^2 |A|^2 &\leq \sum_{i = 1}^N \int |(\rho_\lambda \circ Q_\lambda^*) a_i |^2 + \sum_{i \neq j} \int \left(  Q_\lambda \circ \rho_\lambda^2 \circ Q_\lambda^* \right) a_i \cdot a_j\\
    &= I + II.
\end{align*}
%
Let us address each of these terms individually.

The decay estimates on the symbols defining the pseudodifferential operators $\{ Q_\lambda^* \}$ imply that the kernels start to rapidly decay only outside of a $O(\lambda^{-7/8})$ neighbourhood of the diagonal. In Hongki's talk (Sogge, 2017), it was / will be shown that for any ball $B$ of radius $r$,
%
\[ \| \rho_\lambda \|_{L^2(B) \to L^2(M)} \lesssim r^{1/2} \]
%
This is used to bound equivalent expressions like that in $I$, except that $Q_\lambda$ does not appear in the equation. But the same method there yields that
%
\[ I \lesssim r \sum_{i = 1}^N \int |Q_{\lambda}^* a_i|^2 \lesssim r |A| = \lambda \alpha^{- \frac{4}{d-1}} (\log \lambda)^{- \frac{2}{d - 1}} |A|. \]
%
because the kernel of $Q_\lambda^*$ begins to rapidly decay outside a $O(\lambda^{-7/8})$ neighborhood of the diagonal, and by assumption, $r \gg \lambda^{-3/4}$.

The analysis of $II$ is where the negative curvature comes into play, and we use the fact that $Q_\lambda$ is invariant under the geodesic flow. If $K(w,z)$ is the integral kernel of $Q_\lambda \circ \rho_\lambda^2 \circ Q_\lambda^*$, then our goal is to obtain a bound of the form
%
\[ |K(w,z)| \lesssim \frac{1}{T} \left( \frac{\lambda}{\lambda^{-1} + d_g(w,z)} \right)^{\frac{d-1}{2}} + c(\lambda) \lambda^{\frac{d-1}{2}}, \]
%
where $c(\lambda) \to 0$ at least as fast as some small power of $(\log \lambda)^{-1}$, but not faster than $(\log \lambda)^{-1}$ itself. A trivial bound thus implies that
%
\begin{align*}
    II &\lesssim \left( \sup_{i \neq j} \sup_{w \in A_i, z \in A_j} |K(w,z)| \right) \sum_{i \neq j} \| a_i \|_{L^1} \| a_j \|_{L^1}\\
    &\lesssim \left( \sup_{i \neq j} \sup_{w \in A_i, z \in A_j} |K(w,z)| \right) |A|^2\\
    &\lesssim ( C_0 \alpha^2 + \lambda^{\frac{d-1}{2}} c(\lambda) ) |A|^2,
\end{align*}
%
where we can make $C_0$ arbitrarily small by changing the separation parameter of the sets $\{ A_i \}$. Combining these results gives that
%
\[ \alpha^2 |A|^2 \lesssim \lambda \alpha^{- \frac{4}{d-1}} (\log \lambda)^{- \frac{2}{d - 1}} |A| + ( C_0 \alpha^2 + \lambda^{\frac{d-1}{2}} c(\lambda) ) |A|^2, \]
%
and thus that for $\lambda^{\frac{d-1}{4}} c(\lambda)^{1/2} \lesssim \alpha \leq \lambda^{\frac{d-1}{4} + \frac{1}{8}}$,
%
\[ \alpha |A|^{1/p_c} \lesssim \lambda^{1/p_c} (\log \lambda)^{- \frac{1}{d+1}}. \]
%
This completes the proof for the interesting values of $\alpha$.

Now let's argue why we have estimates of the form
%
\[ |K(w,z)| \lesssim \frac{1}{T} \left( \frac{\lambda}{\lambda^{-1} + d_g(w,z)} \right)^{\frac{d-1}{2}} + c(\lambda) \lambda^{\frac{d-1}{2}}. \]
%
Taking Fourier transforms,
%
\begin{align*}
    Q_\lambda \circ \rho_\lambda^2 \circ Q_\lambda^* &= \frac{1}{T} \int \widehat{\rho^2}(t/T) e^{-2 \pi i \lambda t} (Q_\lambda \circ e^{2 \pi i t P} \circ Q_\lambda^*)\; dt\\
    &\approx \frac{1}{T} \int_{-T}^T \widehat{\rho^2}(t/T) e^{-2 \pi i \lambda t} (Q_\lambda \circ \cos(2 \pi t P) \circ Q_\lambda^*)\; dt
\end{align*}
%
Thus we can write $K(w,z) = A(w,z) + B(w,z)$, where
%
\[ A(w,z) = \frac{1}{T} \int_{-T}^T (1 - \beta)(t) \left\{ \widehat{\rho} \right\}^2(t/T) e^{2 \pi i \lambda t} (Q_\lambda \circ \cos(t P) \circ Q_\lambda^*)(w,z)\; dt, \]
%
\[ B(w,z) = \frac{1}{T} \int_{-T}^T \beta)(t) \left\{ \widehat{\rho} \right\}^2(t/T) e^{2 \pi i \lambda t} (Q_\lambda \circ \cos(t P) \circ Q_\lambda^*)(w,z)\; dt, \]
%
where $\beta(t) = 1$ for small $t$, and is compactly supported.

Now we have to employ the \emph{Hadamard parametrix} for the wave equation, since $u = \cos(tP) f$ gives a solution to the wave operator. This gives that the kernel of $\cos(tP)$ is approximately equal to a constant multiple of (for times less than the injectivity radius of the manifold $M$)
%
\[ |t| \cdot \zeta(x,y) (d_g(x,y)^2 - t^2)^{- \frac{n-3}{2} - 1}, \]
%
where
%
\[ \frac{dV_g}{dx} = \zeta(x,y)^{-2}. \]
%
In order to use this kernel for \emph{all times}, we replace $M$ with a universal covering space $\tilde{M}$, and fix a universal domain $D \subset \tilde{M}$, so that we have a map $z \mapsto \tilde{z}$ from $M$ to $D$. The operators $Q_\lambda$ and $Q_\lambda^*$ all extend to $\tilde{M}$ be employing periodicity, $\cos(tP)$ lifts to $\cos(t \tilde{P})$, and we have
%
\[ \cos(tP)(w,z) = \sum_\alpha \cos(t \tilde{P})(\tilde{w}, \alpha(\tilde{z})). \]
%
By finite speed of propogation, it suffices to understand the deck transformations $\alpha$ with $d_g(\tilde{w}, \alpha) \lesssim T$. Now consider the operator
%
\[ \tilde{A}_0(\tilde{x},\tilde{y}) = \frac{1}{T} \int_{-T}^T (1 - \beta)(t) \left( \widehat{\rho} \right)^2(t/T) e^{2 \pi i \lambda t} \cos(t \tilde{P})(\tilde{x},\tilde{y})\; dt, \]
%
Then
%
\[ A(w,z) = \sum_\alpha U_\alpha(\tilde{w}, \tilde{z}), \]
%
where
%
\[ U_a(\tilde{w}, \tilde{z}) = \int_{\alpha(D)} \int_D Q_\lambda(\tilde{w}, \tilde{z}) \tilde{A}_0(\tilde{x}, \tilde{y}) Q_\lambda^*(\tilde{y}, \alpha^{-1}(\tilde{z}))\; d\tilde{x} d \tilde{y} \]
%
Applying the Hadamard parametrix in $\RR^d$, and then stationary phase gives that
%
\[ \tilde{A}_0(\tilde{x},\tilde{y}) \approx \frac{\lambda^{\frac{d-1}{2}}}{T d_g(\tilde{x}, \tilde{y})^{\frac{d-1}{2}}} \sum_{\pm} e^{\pm 2\pi i \lambda d_g(x,y)} \zeta(\tilde{x}, \tilde{y}) a_{\lambda, \pm}(d_g(\tilde{x}, \tilde{y})). \]
%
for some symbol $a_{\lambda, \pm}$ of order zero which vanishes for $d_g(\tilde{x}, \tilde{y}) \gtrsim T$ and for $d_g(\tilde{x}, \tilde{y}) \lesssim 1$, and
%
%
\emph{Since $M$ has nonpositive curvatures}, the G\"{u}nther comparison theorem says that if the sectional curvatures of $M$ are strictly negative, and bounded from above by $- \kappa^2$, then
%
\[ \zeta(\tilde{x}, \tilde{y}) \lesssim \exp \left( - \frac{\kappa (d-1)}{2} d_g(\tilde{x}, \tilde{y}) \right) \]
%
This is because the volumes of balls \emph{increases exponentially} in a negatively curved space.

We find that
%
\[ U_a(\tilde{w},\tilde{z}) = \sum_{\pm} \frac{\lambda^{\frac{5d-1}{2}}}{T} \int e^{2 \pi i \lambda \phi_{\pm}(\tilde{w}, \tilde{x}, \tilde{y}, \tilde{z}, \eta, \zeta)} q_\lambda(\tilde{w}, \tilde{x}, \eta) \zeta(\tilde{x}, \tilde{y}) a_{\pm, \lambda}(d_g(\tilde{x}, \tilde{y})) q_\lambda^*(\tilde{y}, \alpha^{-1}(\tilde{z}), \eta) \]
%
where
%
\[ \phi_{\pm} = (\tilde{w} - \tilde{x}) \cdot \eta \pm d_g(\tilde{x}, \tilde{y}) + (\tilde{y} - \alpha^{-1}(\tilde{z})) \cdot \xi. \]
%
Applying stationary phase again shows that for $d_g(\alpha,w) \lesssim T$,
%
\[ |U_\alpha(\tilde{w},\tilde{z})| \lesssim \frac{\lambda^{\frac{d-1}{2}}}{T} (\zeta(\tilde{w}, \alpha(\tilde{z})) + \lambda^{-2}) (1 + d_g(\tilde{w}, \alpha(\tilde{z})))^{- \frac{d-1}{2}}. \]
%
But we haven't used the fact that the $\{ q_\lambda \}$ are concentrated on tubes yet. To do this, we use the fact that the terms are negligible unless things are recurrent, and most of the trms aren't recurrent because of negative curature again.

%\[ \tilde{A}_0(\tilde{x}, \tilde{y}) \]

%Then $\tilde{M}$ is diffeomorphic to $\RR^d$ by taking geodesic coordinates starting at any point. We may assume without loss of generality that the geodesic from $\tilde{w}$ to $\tilde{z}$ is of the form $\tilde{\gamma}(t) = (t,0,\dots,0)$. We have
%
%\[ \cos(tP)(w,z) = \sum_{\alpha} \cos(t \tilde{P})(\tilde{w}, \alpha(\tilde{z})), \]
%
%where $\alpha$ ranges over the deck transformations of the covering space. Because the wave equation has finite speed of propogation, we may assume that $\alpha$ lies within a ball of radius $O(T)$ of $\tilde{w}$. If we lift $Q_\lambda^*$ to $\tilde{M}$, then

\end{document}