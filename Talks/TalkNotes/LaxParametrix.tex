\documentclass{article}

%% for editing
%\usepackage{changes}
%\usepackage[final]{changes} %% comment above line and uncomment this line to see final copy without markup
%\setremarkmarkup{~(#2)}
\usepackage{color}


\usepackage{amsmath}
\usepackage{amssymb}
\usepackage{amsthm}
\usepackage{multicol}
%\usepackage[margin=1in]{geometry}
\usepackage{graphicx}
\usepackage{tikz}
\usepackage{hyperref}
\usepackage{mathabx}
\usepackage{comment}

\usepackage{tensor}

\theoremstyle{plain}
\newtheorem{theorem}{Theorem}
\newtheorem{lemma}[theorem]{Lemma}
\newtheorem{corollary}[theorem]{Corollary}
\newtheorem{prop}[theorem]{Proposition}

\theoremstyle{remark}
\newtheorem*{remark}{Remark}
\newtheorem*{example}{Example}
\newtheorem*{proof*}{Proof}

\theoremstyle{definition}
\newtheorem*{defi}{Definition}
\newenvironment{definition}
    {\begin{samepage}\begin{framed}\begin{defi}}
    {\end{defi}\end{framed}\end{samepage}}


\DeclareMathOperator{\diam}{\text{diam}}

\DeclareMathOperator{\QQ}{\mathbb{Q}}
\DeclareMathOperator{\ZZ}{\mathbb{Z}}
\DeclareMathOperator{\RR}{\mathbb{R}}
\DeclareMathOperator{\HH}{\mathbb{H}}
\DeclareMathOperator{\BB}{\mathbb{B}}
\DeclareMathOperator{\CC}{\mathbb{C}}
\DeclareMathOperator{\AB}{\mathbb{A}}
\DeclareMathOperator{\PP}{\mathbb{P}}
\DeclareMathOperator{\MM}{\mathbb{M}}
\DeclareMathOperator{\VV}{\mathbb{V}}
\DeclareMathOperator{\TT}{\mathbb{T}}
\DeclareMathOperator{\LL}{\mathcal{L}}
\DeclareMathOperator{\DD}{\mathcal{D}}
\DeclareMathOperator{\SW}{\mathcal{S}}
\DeclareMathOperator{\EC}{\mathcal{E}}
\DeclareMathOperator{\AC}{\mathcal{A}}

\DeclareMathOperator{\EE}{\mathbb{E}}
\DeclareMathOperator{\NN}{\mathbb{N}}

\DeclareMathOperator{\II}{\mathbb{I}}

\DeclareMathOperator{\DQ}{\mathcal{Q}}


\DeclareMathOperator{\IA}{\mathfrak{a}}
\DeclareMathOperator{\IB}{\mathfrak{b}}
\DeclareMathOperator{\IC}{\mathfrak{c}}
\DeclareMathOperator{\IP}{\mathfrak{p}}
\DeclareMathOperator{\IQ}{\mathfrak{q}}
\DeclareMathOperator{\IM}{\mathfrak{m}}
\DeclareMathOperator{\IN}{\mathfrak{n}}
\DeclareMathOperator{\IK}{\mathfrak{k}}
\DeclareMathOperator{\ord}{\text{ord}}
\DeclareMathOperator{\Ker}{\textsf{Ker}}
\DeclareMathOperator{\Coker}{\textsf{Coker}}
\DeclareMathOperator{\emphcoker}{\emph{coker}}
\DeclareMathOperator{\pp}{\partial}
\DeclareMathOperator{\tr}{\text{tr}}
\DeclareMathOperator{\Ree}{\text{Re}}


\DeclareMathOperator{\BL}{\text{BL}}

\DeclareMathOperator{\dstrike}{//}

\DeclareMathOperator{\supp}{\text{supp}}

\DeclareMathOperator{\codim}{\text{codim}}

\DeclareMathOperator{\minkdim}{\dim_{\mathbb{M}}}
\DeclareMathOperator{\hausdim}{\dim_{\mathbb{H}}}
\DeclareMathOperator{\sobdim}{\dim_{\mathbb{S}}}
\DeclareMathOperator{\lowminkdim}{\underline{\dim}_{\mathbb{M}}}
\DeclareMathOperator{\upminkdim}{\overline{\dim}_{\mathbb{M}}}
\DeclareMathOperator{\lhdim}{\underline{\dim}_{\mathbb{M}}}
\DeclareMathOperator{\lmbdim}{\underline{\dim}_{\mathbb{MB}}}
\DeclareMathOperator{\packdim}{\text{dim}_{\mathbb{P}}}
\DeclareMathOperator{\fordim}{\dim_{\mathbb{F}}}

\DeclareMathOperator{\CT}{ {{\otimes}^\wedge} }

\DeclareMathOperator{\msupp}{\text{$\mu$-supp}}
\DeclareMathOperator{\singsupp}{\text{sing-supp}}
\DeclareMathOperator{\Char}{\text{Char}}

\DeclareMathOperator*{\argmax}{arg\,max}
\DeclareMathOperator*{\argmin}{arg\,min}

\DeclareMathOperator{\ssm}{\smallsetminus}

\newcommand{\myphi}[1]{ \tensor[^{\phi}]{{#1}}{}}




\title{The Lax Parametrix for the Half Wave Equation}
\author{Jacob Denson}

\begin{document}

\maketitle

In this talk, we consider a motivating example that gave rise to much of the general theory of Fourier integral operators: the study of variable-coefficient wave equations. This show a precise example of how Fourier integral operators can be used as a tool to generalize the tools of harmonic analysis we normally use to analyze constant coefficient differential operators like the wave equation, and apply them to variable coefficient analogues.

% Prelude
\section{Euclidean Half-Wave Propogators}

Let's start with a quick review. Consider a solution $u(x,t)$ to the wave equation
%
\[ (\partial_t^2 - \Delta) u = 0 \]
%
on $\RR^d$. We take Fourier transforms on both sides; if $\widehat{u}(\xi,t)$ denotes the Fourier transform of $u$ in the $x$-variable, then we conclude that
%
\[ (\partial_t^2 + 4 \pi^2 |\xi|^2) \cdot \widehat{u}(\xi,t) = 0. \]
%
This is an ordinary differential equation in the $t$ variable for each fixed $\xi$, which we can solve, given that $u(x,0) = f(x)$, and $\partial_t u(x,0) = g(x)$, to write
%
\[ \widehat{u}(\xi,t) = \widehat{f}(\xi) \cos(2 \pi t |\xi|) + \widehat{g}(\xi) \frac{\sin(2 \pi t |\xi|)}{2 \pi |\xi|}. \]
%
It is often natural to write this as
%
\[ \left( \frac{\widehat{f}(\xi) + \widehat{g}(\xi)}{2} \right) e^{2 \pi i t |\xi|} + \left( \frac{\widehat{f}(\xi) - i (2 \pi |\xi|)^{-1} \widehat{g}(\xi)}{2} \right) e^{-2 \pi i t |\xi|}. \]
%
If we write $u = v + w$, where
%
\[ \widehat{v}(\xi,t) = \left( \frac{\widehat{f}(\xi) + \widehat{g}(\xi)}{2} \right) e^{2 \pi i t |\xi|} \quad\text{and}\quad \widehat{w}(\xi,t) = \left( \frac{\widehat{f}(\xi) - i |\xi|^{-1} \widehat{g}(\xi)}{2} \right) e^{-2 \pi i t |\xi|}, \]
%
then we have decomposed $u$ into the sum of solutions to the \emph{half-wave equations}
%
\[ \Big( \partial_t - i \sqrt{-\Delta} \Big) v = 0 \quad\text{and}\quad \Big(\partial_t + i \sqrt{-\Delta} \Big) w = 0. \]
%
The two operators $\partial_t - i \sqrt{-\Delta}$ and $\partial_t + i \sqrt{-\Delta}$ are identical, up to a time-reversal symmetry, so we focus on solutions to the equation $\partial_t - i \sqrt{-\Delta} = 0$. Solutions to the half-wave equation behave similarly to solutions to the wave equation, with one notable exception: the wave equation has a finite speed of propogation, whereas the half-wave equation does not.

The fact that the half-wave equation $(\partial_t - i \sqrt{-\Delta}) u = 0$ is a first-order operator makes the Cauchy problem somewhat simpler to study, since we need less initial data than in the wave equation. We can therefore define operators $e^{it \sqrt{-\Delta}}$ such that
%
\[ v(x,t) = (e^{i t \sqrt{-\Delta}} v_0)(x). \]
%
We now briefly look at these operators, and the solution operator
%
\[ Sf(x,t) = (e^{i t \sqrt{-\Delta}} f)(x) \]
%
to the half-wave equation, from the perspective of FIO theory.

\begin{comment}

Fourier transform of e^{-|x|} is

	c_d (1 + |xi|^2)^{- (d+1) / 2}

Fourier transform of e^{- lambda |x|} for lambda > 0 is

	int e^{- lambda |x|} e^{-i xi * x} dx
	= c_d lambda^{-d} (1 + |xi|^2 / lambda^2)^{-(d+1)/2}

Can we now apply analytic continuation to conclude that
the Fourier transform of e^{-it |x|} is equal to

	c_d (it)^{-d} ( 1 + |xi|^2 / (it + 0)^2 )^{-(d+1)/2}

where we take the branch of the square root that is the analytic extension
away of the normal square root away, defined away from the imaginary axis.

Then for |xi| > t this quantity is equal to

	c_d i t^{-d} ( |xi|^2 / t^2 - 1 )^{-(d+1)/2}

e^{-i pi / 2}

sqrt( a e^{it} ) = sqrt(a) sqrt(e^{it/2})

\end{comment}

Let's start with the operators $\{ e^{it \sqrt{-\Delta}} \}$. We can write
%
\begin{align*}
	e^{i t \sqrt{-\Delta}} f(x) &= \int e^{2 \pi i (t |\xi| + \xi \cdot x)} \widehat{f}(\xi)\; d\xi\\
	&= \int e^{2 \pi i (t |\xi| + \xi \cdot (x - y))} f(y)\; dy\; d\xi\\
	&= \int a(x,y,\xi) e^{2 \pi i \phi_t(x,y,\xi)} f(y)\; dy\; d\xi,
\end{align*}
%
where $a(x,y,\xi) = 1$, and $\phi_t(x,y,\xi) = t |\xi| + \xi \cdot (x - y)$. The function $a$ is a symbol of order zero, and the function $\phi_t$ is a non-degenerate phase function with canonical relation defined by the equations
%
\[ \left\{ x = y + t \frac{\xi}{|\xi|}\ \text{and}\ \xi = \eta \right\}, \]
%
which, for a fixed $y$, we can think of as consisting of the sphere of radius $t$ at $y$, and all \emph{outward} pointing cotangent vectors on this sphere. Thus the operators $e^{it \sqrt{-\Delta}}$ are Fourier integrals of order $0$.

The solution operator $W$ is very similar, since we have
%
\begin{align*}
	Sf(x,t) &= e^{it \sqrt{-\Delta}} f(x)\\
	&= \int e^{2 \pi i (t |\xi| + \xi \cdot (x - y))} f(y)\; dy\; d\xi\\
	&= \int a(x,t,y,\xi) e^{2 \pi i \phi(x,t,y,\xi)} f(y)\; dy\; d\xi.
\end{align*}
%
The function $a(x,t,y,\xi) = 1$ here is still a symbol of order zero, and the phase $\phi(x,t,y,\xi) = t |\xi| + \xi \cdot (x - y)$ is non-degenerate, so $S$ is a Fourier integral operator from $\RR^d$ to $\RR^d \times \RR$, of order $-1/4$, with canonical relation defined by the three equations
%
\[ \left\{ x = y + t \frac{\xi}{|\xi|}\ \text{and}\ \xi = \eta\ \text{and}\ \tau = |\xi| \right\}. \]
%
In this talk, we will construct approximate solutions (parametrices) for variable-coefficient analogues of the half-wave equation using Fourier integral operators.

\section{Variable-Coefficient Half-Wave Equations}

Our goal is to consider variable coefficient analogues of the half-wave operator $\partial_t - 2\pi i \sqrt{-\Delta}$. A natural variable-coefficient generalization of such an operator would be
%
\[ L = \partial_t - i P, \]
%
where $P$ is a first order \emph{pseudodifferential operator} with principal symbol $p$, i.e. a Fourier integral operator given by the expression
%
\[ Pf(x) = \int_{\RR^d} P(x,\xi) e^{2 \pi i \xi \cdot (x - y)} f(y)\; dy. \]
%
As harmonic analysts, we're mainly considered with the regularity of solutions to PDEs, i.e. the mapping properties of the operators in $L^p$ norms or Sobolev spaces, rather than the existence and uniqueness of solutions. Referring to the literature on hyperbolic equations, we make the assumptions that $P$ is a formally positive operator given by an elliptic symbol, which is classical\footnote{A symbol of order $\mu$ is classical if we have an asymptotic expansion of symbols of the form $P \sim \sum p_{\mu-j}$, where $p_k$ is a smooth, homogeneous function of order $k$.} of order one. Under these assumptions, solutions to the half-wave equation exist and unique, provided that initial data for the problem is supported on a small region of space, and that we only care about solutions for small time.

For simplicity, we'll mainly focus on a situation where the existence and uniqueness of solutions are relatively easy to obtain. We work with the equation on a compact manifold $M$, and with a pseudodifferential operator $P$ which is formally positive, and given by an elliptic symbol. The existence and uniqueness of solutions to the half-wave equation for smooth input data can then be established, so that we can define propogators $\{ e^{2 \pi i t P} \}$ and a solution operator $S$ which takes in smooth input data, and gives solutions to the equations.

\begin{comment}
For such an operator, we can find a countable, orthogonal basis $\{ e_\lambda : \lambda \geq 0 \}$ of $L^2(M)$, such that $Pe_\lambda = \lambda e_\lambda$, and such that for each $\lambda$, $e_\lambda$ is a smooth function, with $L^2_s$ norm $O_s(\lambda^s)$. The number of elements of this basis with eigenvalue at most $\lambda$ is $O(1 + \lambda^d)$. Given this basis, we can define a family of bounded operators $\{ e^{2 \pi itP} \}$ on $L^2(M)$ by setting $e^{2 \pi it P} e_\lambda = e^{2 \pi it \lambda} e_\lambda$, and then consider a solution operator
%
\[ (Wf)(x,t) = (e^{2 \pi it \lambda} f)(x). \]
%
Using the $L^2_s$ norm properties of the eigenfunctions $\{ e_\lambda \}$, and the fact that for any $f \in C^\infty(M)$,
%
\[ \langle f, e_\lambda \rangle \lesssim_N \lambda^{-N} \quad\text{for all $N > 0$,} \]
%
it is simple to check that $W$ maps $C^\infty(M)$ into $C^\infty_{\text{loc}}(M \times \RR)$, and that for $f \in C^\infty(M)$, the smooth function $u = Wf$ solves the equation $Lu = 0$ with initial conditions $f$. One can also prove the uniqueness of solution solutions, e.g. using energy estimates, but this takes us a little far afield of what we want to talk about in these notes.
\end{comment}

Instead of proving the existence and uniqueness of solutions to the half-wave equation, our goal is isntead to construct a \emph{parametrix} for the solution operator $S$ to the Cauchy problem $Lu = 0$. That is, we wish to find an operator $A$ such that $R = A - S$ is a \emph{smoothing operator}, i.e. a Schwartz operator whose kernel is a smooth function. The mapping properties of the operator $S$ with respect to Sobolev norms then immediately reduces to the mapping properties of the operator $A$, because the operator $R$ has trivial mapping properties. For example, if $u$ is a compactly supported distribution, then $Ru$ is a smooth function, and moreover, $R$ maps $H^{s_1}_c$ to $H^{s_2}_{x, \text{loc}} H^{s_3}_{t, \text{loc}}$ for any $s_1$, $s_2$, and $s_3$.

The reason parametrices arise is that in many variable coefficient problems, it is often possible to find operators $A$ which can be expressed in a simple manner, whereas none may exist for $S$. This in particular arises from the perspective of harmonic analysis, since it is often the case that the behaviour of solution operators can only be given analytical expressions asymptotically, i.e. via expressions that are never completely accurate, but become more accurate as we take input data which is oscillating more and more rapidly. Taking the difference between the actual solution, and this asymptotic expression, we obtain an expression which becomes minute as we plug in input data that is oscillating rapidly. Slowly oscillating functions are \emph{smooth}, and so the operator formed from taking the difference between the actual solution, and the asymptotic formula, will then be smoothing.

The main example of equations that fit these assumptions are obtained by considering some Riemannian metric $g$, and consider the resulting half-wave equation
%
\[ \partial_t - 2 \pi i \sqrt{-\Delta_g} \]
%
where
%
\[ \sqrt{-\Delta_g} f = |g|^{-1/2} \sum_i \partial_i \{ |g|^{1/2} g^{ij} \partial_j f \} \]
%
is the Laplace-Beltrami operator. Similar equations are obtained in quantum mechanics, Schr\"{o}dinger-type equations of the form $\partial_t = i P(x,D)$, describing the behaviour of a classical mechanical system described by the \emph{Hamiltonian} $P(x,\xi)$, i.e. the system
%
\[ \frac{dx}{dt} = \frac{\partial P}{\partial \xi} \quad\text{and}\quad \frac{d\xi}{dt} = - \frac{\partial P}{\partial x} \]
%
at a quantum scale. However, the principal part of $P$ will generally be homogeneous of degree two in the $\xi$-variable, since kinetic energy is often a quadratic form in the momentum variables. One can use the methods described here to construct parametrices. But the resulting operators, given by oscillatory integrals, will not have phases that are homogeneous of degree one, a necessary part of the system of Fourier integral operator theory, and so the methods here do not directly apply. Nonetheless, the physical intuition behind the Schr\"{o}dinger equation will be helpful for the construction of the parametrices we construct here. Indeed, we will see that the `semiclassical' behaviour of the Schr\"{o}dinger equation at large scales is analogous to the analytical expressions we will obtain for the half-wave equation, associated with a suitable Hamiltonian equation. Indeed, the methods we used here, first applied to the half wave equation in the 1960s, really have their root in the methods of quantum physicists of the 1920s, with their WKB method for approximating solutions to the Schr\"{o}dinger equation applying approximations like we will give here. One could even argue that the roots of these methods emerged even earlier, in the analytical methods of geometric optics discovered by Fresnel and Airy in the 1800s.

% The parameterices constructed using the techniques here will therefore have a phase which is homogeneous of degree two in the $\xi$ variable, which doesn't quite fit the system considered here, because the Fourier Integral operators we study have phases that are homogeneous of degree one in the phase variable.

% In physics, the use of such parametrices in the study of the Schr\"{o}dinger equation is called the \emph{WKB method}.

\section{High-Frequency Asymptotic Solutions}

Fix three quantities $0 < r < R$, and $\varepsilon > 0$, to be specified later. Our goal is to find a general family of `high-frequency asymptotic solutions' to the half-wave equation on the ball $B_R(x_0)$ with radius $R$ and center $y$ for $|t| \leq \varepsilon$, given initial data supported on the smaller ball $B_r(y)$.

%Since we're trying to construct such a family locally, despite working on a compact manifold, we can switch to studying the operator in coordinates, so we'll abuse notation, and assume we're working with a pseudodifferential operator defined on some open subset $\Omega$ of $\RR^d$, and that we only care about values of $y$ lying in some compact subset $K$ of $\Omega$, where $\Omega$ contains the closure of the $R$-neighborhood of the set $K$.

Let us describe what we mean by this. Fix an expression of the form
%
\[ u_\lambda(x,t) = e^{2 \pi i \lambda \phi(x,t)} a(x,t,\lambda), \]
%
where $a$ is a classical symbol of order zero in the $\lambda$ variable, compactly supported on $B_R(x_0)$, and for $|t| \leq \varepsilon$, and $\phi$ is a smooth, real-valued function, such that $\nabla_x \phi(x,t) \neq 0$ on the support of $a$. This latter condition is necessary to interpret $u_\lambda$ as a function `oscillating at a magnitude $\lambda$'. Indeed, if the condition is true, stationary phase shows that the Fourier transform of $u_\lambda$ rapidly decays outside the annulus of frequencies $|\xi| \sim \lambda$.

%
%\begin{itemize}
%	\item $a$ is a `classical' symbol of order zero in the $\lambda$-variable, i.e. we can write
	%
%	\[ a \sim \sum_{j = 0}^\infty a_{-j}, \]
	%
%	where $a_{-j}$ is a smooth function, homogeneous of order $-j$ in the $\lambda$ variable. Moreover, $a$ is compactly supported in the $x$-variable and the $t$-variable.

%	\item The function $\phi$ is smooth, real-valued, and we assume $\nabla_x \phi(x,t) \neq 0$ on the support of $a$.
%\end{itemize}
%
%The latter assumption is natural because we think of $u_\lambda$ as being localized in phase space about the family of points
%
%\[ \Big\{ (x, t, \lambda \nabla_x \phi(x,t), \lambda \partial_t \phi(x,t)) \Big\} \]
%
%and so we need $\nabla_x \phi(x,t)$ to be nonvanishing, in order for us to think of $u_\lambda$ as being localized near spatial frequencies that have magnitude approximately $\lambda$, and thus localized to large frequencies when $\lambda$ is large.

In a lemma shortly following this discussion, we will show that for any choice of $a$ and $\phi$ as above, there exists a symbol $b$ of order $1$ such that
%
\[ L u_\lambda = e^{2 \pi i \lambda \phi(x,t)} b(x,t,\lambda). \]
%
For \emph{some} choices of $a$ and $\phi$, it might be true that the higher order parts of $b$ are eliminated, i.e. so that $b$ is also of order much smaller than $1$. If $a$ and $\phi$ are chosen in a \emph{very} particular way, it might be true that all finite order parts of $b$ are eliminated, so that $b$ is a symbol of order $-\infty$ in a neighborhood of $x_0$ and for small times. In such a situation, we say $\{ u_\lambda \}$ is a `\emph{high-frequency asymptotic solution}' to the wave equation. We then conclude that
%
\[ |\partial_x^\alpha \partial_t^\beta \{ L u_\lambda \}| \lesssim_{\alpha,\beta,N} \lambda^{-N} \quad\text{for all $N > 0$}, \]
% \quad\text{for all $N > 0$, $|t| \lesssim 1$, and $|x - x_0| \lesssim 1$}
which justifies that $u_\lambda$ behaves like a solution to the half-wave equation when $\lambda$ is large. We will prove the following `Cauchy-type' initial value problem for high-frequency asymptotic solutions to the equation.

\begin{theorem}
	Suppose $\varphi(x,y,\theta)$ is a smooth, real-valued function solving the \emph{eikonal equation}
	%
	\[ p(y,\theta) = p(x, \nabla_x \varphi(x,y,\theta)). \]
	%
	for $|x - y| \leq R$, subject to the constraints that
	%
	\begin{itemize}
		\item $(\nabla_x \varphi)(y,y,\theta) = \theta$.

		\item $\varphi$ vanishes on the hyerplane passing through $y$, orthogonal to $\theta$.

		\item For $|x - y| \leq R$, $(\nabla_\theta \varphi)(x,y,\theta) = 0$ if and only if $x = y$.
	\end{itemize}
	%
 	Set $\phi(x,t,y,\theta) = \varphi(x,y,\theta) + t \cdot p(y,\theta)$. Then there exists $\varepsilon > 0$ and $r > 0$ such that any classical symbol $a(x,0,\lambda)$ of order zero, supported on $|x - y| \leq r$, extends to a unique classical symbol $a(x,t,\lambda)$ of order zero, supported on $|x - y| \leq R$ and defined for $|t| \leq \varepsilon$, such that the associated family of functions
	%
	\[ u_\lambda(x,t) = e^{2 \pi i \lambda \phi(x,t,y,\theta)} a(x,t,\lambda) \]
	%
	are high-frequency asymptotic solutions to the half-wave equation for $|x - y| \leq R$ and $|t| \leq \varepsilon$.
\end{theorem}

\begin{remark}
	The constraints on the initial data $a$ imply we should think of $u_\lambda(x,0)$ as being localized in space near $y$, and oscillating at a frequency roughly $\lambda \theta$.
\end{remark}

In order to prove this result, we need to obtain some formulas that tell us what the symbol $b$ looks like whose existence was postulated above, in terms of the phase $\phi$, the operator $P$, and the symbol $a$. In order to prove the theorem above, we'll construct $a$ recursively by slowly fixing the contributions of the higher order parts of $a$. One then studies the lower order terms separately, so it is wise to make a study of the functions
%
\[ u_\lambda(x,t) = e^{2 \pi i \lambda \phi(x,t)} a(x,t,\lambda),\]
%
where $a$ is a classical symbol of some arbitrary order $\mu$, rather than just a symbol of order zero.
 The proof of the following Lemma is a somewhat technical application of stationary phase, and can be relegated to a second reading of these notes.

\begin{lemma}
	Write $P \sim p + p_0 + p_-$, where $p$ is the principal symbol, homogeneous of order one, $p_0$ is homogeneous of order zero, and $p_-$ is a symbol of order $-1$. Consider
	% Let $P$ be a pseudodifferential operator of order $1$ given by a classical symbol of order one, i.e.
	%
	%\[ P \sim p + p_0 + p_-, \]
	%
	%where $p$ is the principal symbol, homogeneous of order one, $p_0$ is homogeneous of order zero, and $p_-$ is a symbol of order $-1$. Consider
	%
	\[ u_\lambda(x,t) = e^{2 \pi i \lambda \phi(x,t)} a(x,t,\lambda), \]
	%
	where $a$ and $\phi$ are as above. Then $e^{-2 \pi i \lambda \phi(x)} L u_\lambda$ is a classical symbol of order $\mu+1$, with principal symbol
	%
	%Then
	%
	%\[ e^{-2 \pi i \lambda \phi(x)} P \{ u_\lambda \}(x) \]
	%
	%is a classical symbol of order $\mu+1$, with principal symbol
	%
	%\[ \lambda \cdot p(x,\nabla_x \phi) \cdot a_\mu, \]
	%
	%and with order $\mu$ part given by
	%
	%\begin{align*}
	%	&(\nabla_\theta p)(x, \nabla_x \phi) \cdot (D_x a_\mu) + s(x) \cdot a_\mu + \lambda \cdot p(x,\nabla_x \phi) \cdot a_{\mu-1},
	%\end{align*}
	%
	\[ 2\pi i \lambda \Big( \partial_t \phi - p \left( x, \nabla_x \phi \right) \Big) a_\mu, \]
	%
	and with order $\mu$ part given by
	%
	\begin{align*}
		&2 \pi i \lambda \Big( \partial_t \phi - p \left( x, \nabla_x \phi \right) \Big) a_{\mu - 1}\\
		&\quad + \partial_t a_\mu -  (\nabla_\xi p)(x, \nabla_x \phi) \cdot (\nabla_x a_\mu) - i s \cdot a_\mu,
	\end{align*}
	%
	for a smooth, real-valued function $s$ depending only on $\phi$, $p_1$, and $p_0$.
\end{lemma}

\begin{remark}
	The result of this lemma shows the reason we must choose $\varphi$ to satisfy the eikonal equation, i.e. so that the principal symbol of $e^{-2 \pi i \lambda \phi(x)} L u_\lambda$ vanishes.
\end{remark}

\begin{proof}
	Let us temporarily write terms without the $t$ variable, since $P$ is a pseudodifferential operator only in the $x$ variables, and so the $t$ variable won't come into effect in the argument. We write
	%
	\[ P \{ u_\lambda \} (x) = \int a(y,\lambda) P(x, \xi) e^{2 \pi i [ \xi \cdot (x - y) + \lambda \phi(y) ]}\; dy\; d\xi. \]
	%
	This integral has a unique, non-degenerate stationary point when $y = x$, and when $\xi = \lambda \nabla_x \phi(x)$. Fix $C > 0$, and suppose
	%
	\[ (1/C) \leq |\nabla_x \phi(x)| \leq C \]
	%
	for all $x$ on the support of $a$. Consider a smooth function $\chi$ such that
	%
	\[ \chi(v) = 1 \quad\text{for}\ 1/2C \leq |v| \leq 2C, \]
	%
	and vanishing outside a neighborhood of this set. Write
	%
	\[ \phi(y) - \phi(x) = \nabla_x \phi(x) \cdot (y - x) + \phi_R(x,y). \]
	%
	Write
	%
	\[ P_\lambda(x,\xi) = \chi(\xi / \lambda) P(x,\xi). \]
	%
	The theory of non-stationary phase, i.e. integrating by parts sufficiently many times, can be used to show that
	%
	\begin{align*}
		& e^{- 2 \pi i \lambda \phi(x) } (P - P_\lambda) \{ u_\lambda \} (x,\lambda)\\
		& \quad\quad\quad = \lambda^d \int a(y,\lambda) (P - P_\lambda)(x,\lambda \xi) e^{2 \pi i \lambda [ \xi \cdot (x - y) + \phi_R(x,y) ]}\; dy\; d\xi
	\end{align*}
	%
	is a symbol of order $-\infty$. Thus it suffices to analyze the quantities
	%
	\begin{align*}
		e^{-2 \pi i \lambda \phi(x)} P_\lambda u_\lambda(x) &= \int P_\lambda(x, \xi) e^{2 \pi i [ (\xi - \lambda \nabla_x \phi(x)) \cdot (x - y) + \lambda \phi_R(x,y) ]} a(y, \lambda)\; dy\; d\xi\\
		&= \int P_\lambda(x, \lambda \nabla_x \phi(x) + \xi) e^{2 \pi i [ \xi \cdot (x - y) + \lambda \phi_R(x,y) ]} a(y,\lambda)\; dy\; d\xi.
	\end{align*}
	%
	Using a Taylor expansion, we can write
	%
	\[ P_\lambda(x, \lambda \nabla_x \phi(x) + \xi) = \sum_{|\alpha| < N} (\partial_\xi^\alpha P)(x, \lambda \nabla_x \phi(x)) \cdot \xi^\alpha + R_N(x,\xi,\lambda), \]
	%
	where $R_N$ vanishes of order $N$ as $\xi \to 0$, and, using the remainder formula for the Taylor expansion, and the support properties of $P_\lambda$, for all multi-indices $\alpha$ we have
	%
	\[ |(\partial_\xi^\alpha R_N)(x,\xi,\lambda)| \lesssim_\alpha \lambda^{1 - |\alpha|}, \]
	%
	where the implicit constant is uniform in $\xi$ and $\lambda$, and locally uniform in $x$.
	%
	% We can write R_N as a
	% finite sum of terms of the form
	%
	% xi^a int_0^1 (1 - t)^{N-1} (partial_xi^a P_lambda)(x, lambda nabla_x phi(x) + t xi)
	%
	% where |a| = N.
	%
	% Denote the integral by I_a(x,lambda,xi). If v = lambda nabla_x phi(x), then
	%
	% (d_xi^b I_a) <<
	% 			lambda^{c - b}
	% 				int_0^1 (1 - t)^{N-1} t^b (partial_xi^{a + c} P_lambda)(x, v + t xi) dt
	%
	% where c <= b.
	%
	% For |xi| >> lambda, the integral vanishes for |t| >= 1/|xi|,
	% and so this integral is bounded by
	%
	% lambda^{c - b} |xi|^{-b-1} lambda^{1 - a - c}
	% 	= lambda^{1 - a - b} |xi|^{-b-1}
	%
	% For |xi| << lambda, the integral is
	%
	% lambda^{c - b} lambda^{1 - a - c}
	% 	= lambda^{1 - a - b}
	%
	% In general, we conclude that
	But then the stationary phase formula tells us that
	%
	\[ \int R_N(x,\xi,\lambda) e^{2 \pi i [ \xi \cdot (x - y) + \lambda \phi_R(x,y) ]} a(y,\lambda)\; dy\; d\xi \]
	%
	is a symbol of order $\mu + 1 - \lceil N/2 \rceil$ in the $\lambda$ variable. Conversely, if we let
	%
	\[ a_R(x,y,\lambda) = e^{2 \pi i \lambda \phi_R(x,y)} a(y,\lambda), \]
	%
	we calculate that
	%
	\begin{align*}
		&\int (\partial_\xi^\alpha P)(x, \lambda \nabla_x \phi(x)) \cdot \xi^\alpha e^{2 \pi i [\xi \cdot (x - y) + \lambda \phi_R(x,y)]} a(y,\lambda)\; dy\; d\xi\\
		&\quad\quad = (\partial_\xi^\alpha P)(x, \lambda \nabla_x \phi(x)) (D_y^\alpha a_R)(x,x,\lambda).
	\end{align*}
	%
	Thus we conclude that, modulo order $\mu + 1 - \lceil N/2 \rceil$ symbols in the $\lambda$ variable,
	%
	\[ e^{-2 \pi i \lambda \phi(x)} Pu_\lambda(x) = \sum_{|\alpha| < N} (\partial_\xi^\alpha P)(x, \lambda \nabla_x \phi(x)) (D^\alpha_y a_R)(x,x,\lambda). \]
	%
	The formula is very simple for $N = 1$, which allows us to work modulo symbols of order $\mu$, but we must work with $N = 3$, which is slightly more complicated, because we wish to work modulo symbols of order $\mu - 1$. For $|\alpha| \leq 1$, we have
	%
	\[ (D^\alpha_y a_R)(x,x,\lambda) = (D^\alpha_x a)(x,\lambda). \]
	% phi_R(x,y) = phi(y) - phi(x) - (nabla_x phi)(x) * (y - x)
	% (D^alpha_x phi)(x)
	% phi(y) - phi(x) = nabla_x \phi(x) \cdot (y - x) + \phi_R(x,y)
	% 
	For $|\alpha| = 2$, we have
	% phi_R(x,y)
	\begin{align*}
		(D^\alpha_y a_R)(x,x,\lambda) &= \Big( (2 \pi i \lambda) (D^\alpha_x \phi)(x) \Big) a(x,\lambda) + (D^\alpha_x a)(x,\lambda)\\
		&= \lambda (\partial^\alpha_x \phi)(x) a(x,\lambda) + (D^\alpha_x a)(x,\lambda).
	\end{align*}
	%
	Thus summing up all the terms, modulo symbols of order $\mu - 1$, we find
	%
	\begin{align*}
		e^{-2 \pi i \lambda \phi(x)} Pu_\lambda(x) &= \Bigg( \sum_{|\alpha| \leq 1} (\partial_\xi^\alpha P)(x,\lambda \nabla_x \phi(x)) (D^\alpha_x a)(x,\lambda) \Bigg)\\
		&\quad + \sum_{|\alpha| = 2} \lambda (\partial^\alpha_x \phi)(x) (\partial_\xi^\alpha P)(x, \lambda \nabla_x \phi(x)) a(x,\lambda).
	\end{align*}
	%
	The order $\mu + 1$ part of this sum is
	%
	\[ p(x, \lambda \nabla_x \phi(x)) \cdot a_\mu(x,\lambda). \]
	%
	The order $\mu$ part is
	%
	\begin{align*}
		& \Big( p_0(x, \nabla_x \phi(x)) + \sum_{|\alpha| = 2} (\partial^\alpha_x \phi)(x) (\partial_\xi^\alpha p)(x, \nabla_x \phi(x)) \Big) \cdot a_\mu(x,\lambda)\\
		&\quad + (\nabla_\xi p)(x,\lambda \nabla_x \phi(x)) \cdot (D_x a_\mu)(x,\lambda) \\
		&\quad + p(x,\lambda \nabla_x \phi(x)) \cdot a_{\mu - 1}(x,\lambda).
	\end{align*}
	%
	Putting these formulas together with the terms corresponding to $\partial_t \{ L u_\lambda \}$, calculated simply using the product rule, and then splitting apart the homogeneous parts of the sum of various orders, we obtain the required result.
\end{proof}

We are now ready to prove Theorem 1.

\begin{proof}[Proof of Theorem 1]

Notice that because the phase $\phi$ is chosen according to the assumptions of Theorem 1, i.e. in terms of a solution to the eikonal equation, regardless of the symbol $a$, the order 1 part of $b$ vanishes, because
%
\[ \partial_t \phi = p(y, \theta) \]
%
and
%
\[ p( x, \nabla_x \phi ) = p(y,\theta). \]
%
Let us see what conditions are required in order to conclude that $b$ is a symbol of order $-1$ for $|x - y| \leq R$ and $|t| \leq \varepsilon$. Looking at the terms guaranteed by Lemma 1, we see that we must have
%
\[ \partial_t a_0 - (\nabla_\xi p)(x, \nabla_x \phi) \cdot (\nabla_x a_0) - i s \cdot a_0 = 0. \]
%
We note that the contribution to the order $\mu$ part of the symbol depending on the symbol $a_{-1}$ implied in Lemma 1 disappears because $\phi$ was chosen via the Eikonal equation. If we consider the \emph{real} vector field
%
\[ X(x,t) = \partial_t - (\nabla_\xi p)(x,\nabla_x \phi) \cdot \nabla_x, \]
%
defined on $\RR^d \times \RR$, then the equation above becomes
%
\[ X \{ a_0 \} = i s a_0. \]
%
This is a \emph{transport equation}, which can be solved using the methods of characteristics. In particular, there exists $v > 0$ such that the speed of propogation of the transport equation is bounded by $v$ for $|x - y| \leq R$. This quantity depends only on properties of $X$, and thus the principal symbol $p$. Thus for any $\varepsilon > 0$, if we choose $R > r + v \varepsilon$, then the solution to the transport equation exists, is uniquely determined, and is supported on
%
\[ \Sigma(y,r,v) = \{ (x,t) : |x - y| \leq r + v |t| \}. \]
%
for $|t| \leq \varepsilon$, provided that the initial values of $a_\mu$ are fixed, and compactly supported on $|x - y| \leq r$.

Let us now assume that $a_0$ is chosen so that this is the case. We now apply a recursive procedure, to show that, by induction, for all $k \geq 0$, the symbol $b$ is of order $-k-1$ for a unique choice of homogeneous functions $a_0, \dots, a_{-k}$, given their initial conditions for $t = 0$ and that they are supported on $|x - y| \leq r$, and each of these functions is supported on $\Sigma(y,r,v)$. Let us assume the result holds for $k = 0$, and let's define
%
\[ u_{\lambda,k} = e^{2 \pi i \lambda \phi(x,t)} \left\{ \sum_{j = 0}^{k-1} a_{-j} \right\}, \]
%
By our inductive hypothesis, the function
%
\[ e^{-2 \pi i \lambda \phi(x,t)} L \{ u_{\lambda,k} \} \]
%
is a symbol of order $-k$. Let us denote the principal part of degree $-k$ by $c_k$. Lemma 1 now applies to the quantity
%
\[ e^{-2 \pi i \lambda \phi(x,t)} L \Big\{ \sum_{j = k}^\infty a_{-j} \Big\} \]
%
with $\mu = -k$. We conclude that the quantity is a symbol of order $1 - k$. The choice of $\phi$ implies the principal symbol of order $1-k$ actually vanishes for $|x - y| \leq R$. The order $-k$ part is given by
%
\[ \partial_t a_{-k} - (\nabla_\xi p)(x, \nabla_x \phi) \cdot (\nabla_x a_{-k}) - i s a_{-k} \]
%
But this means the order $-k$ part of $b$ vanishes for $|t| \leq \varepsilon$ and $|x - y| \leq R$ provided that
%
\[ \partial_t a_{-k} - (\nabla_\xi p)(x, \nabla_x \phi) \cdot (\nabla_x a_{-k}) - i s a_{-k} + c_k = 0. \]
%
We can write this as
%
\[ X \{ a_{-k} \} - i s a_{-k} + c_k = 0. \]
%
This is the same transport equation as we dealt with in the case $k = 0$, except it is now a non-homogeneous problem. This causes us no worry because $c_k$ is, by the inductive hypothesis, supported on $\Sigma(y,r,v)$, and so the finite speed of propogation guarantees that $a_{-k}$ is uniquely determined for $|t| \leq \varepsilon$ provided it's initial conditions are supported on $|x - y| \leq r$, and that this unique solution is supported on $\Sigma(y,r,v)$. We have thus completed the inductive argument. But this is sufficient to justify the result of the Theorem. \qedhere

\end{proof}

\begin{remark}
	If we restrict $y$ and $\theta$ to lie in a compact subset $K$, then the local uniformity of the quantities in Lemma 1 implies we can choose $r$ and $\varepsilon$ uniformly in $y$ and $\theta$.
\end{remark}

\section{Construction of the Parametrix}

By finding asymptotic solutions to the half-wave equation in the generality above, we've essentially constructed the parametrix in secret we've wanted -- the idea is to take a general input, break it up into the superposition of wave packets that are localized in space and frequency, and then apply the asymptotic solution constructed above for each of these wave packets, which behaves better as these wave packets are localized to higher and higher frequencies.

It's best to break down our solution into a \emph{continuous} superposition of wave packets rather than the usual discrete decomposition that comes up in decoupling theory. Let's review a simple approach, due to Gabor, which won't quite work for our purposes, but gives us intuition for what we want. Consider the \emph{Gabor transform}
%
\[ Gf(x_0,\xi_0) = \int f(x) \eta(x - x_0) e^{- 2 \pi i x \cdot \xi_0}\; dx, \]
%
where $\eta$ is some fixed, non-negative, function $\eta$ supported on $|x| \leq \delta$ in a neighborhood of the origin, equal to one in an even smaller neighborhood, and with
%
\[ \int \eta(x)^2\; dx = 1. \]
%
This is a unitary transformation, with adjoint given by
%
\[ (G^*h)(x) = \iint h(x_0,\xi_0) \eta(x - x_0) e^{2 \pi i x \cdot \xi_0}\; dx_0\; d\xi_0, \]
%
so we obtain a `localized Fourier inversion formula'
%
\[ f(x) = \int Gf(x_0,\xi_0) \eta(x - x_0) e^{2 \pi i \lambda x \cdot \xi_0}\; dy\; d\xi\; d\lambda, \]
%
which expresses $f$ as a superposition of the wave packets
%
\[ x \mapsto \eta(x - x_0) e^{2 \pi i x \cdot \xi_0}. \]
%
This approach doesn't quite work for our purposes, since our choice of asymptotic solutions to the half-wave equation leads us to try and decompose our function into wave packets of the form
%
\[ x \mapsto s(x,y,\theta) e^{2 \pi i \varphi(x,y,\theta)}. \]
%
for \emph{some} symbol $s$, where $\varphi$ satisfies the eikonal equation as in Theorem 1. The fact that $\varphi(x,y,\theta) \approx \theta \cdot (x - y)$, however, implies a similar approach will work to the Gabor transform.

The trick here is to consider an inversion formula of the form
%
\[ f(x) = \iiint s(x,y,\theta) e^{2 \pi i \varphi(x,y,\theta)} f(y)\; dy\; d\theta, \]
%
which, if held, would imply we could decompose an arbitrary function $f$ into wave packets. We claim that we can use the \emph{equivalence of phase theorem} to find a symbol $a$ of order zero such that this inversion formula holds. Indeed, the left hand side is the identity operator applied to $f$, which is a Fourier integral operator of order zero associated with the \emph{diagonal} canonical relation
%
\[ \Delta = \Big\{ x = y\ \text{and}\ \xi = \theta \Big\}. \]
%
For any symbol $s$, the right hand side is \emph{also} a Fourier integral operator of order zero associated with the diagonal canonical relation, because we assume that $\nabla_{\theta} \varphi = 0$ only holds when $x = y$, and that when $x = y$, $\nabla_x \varphi = \theta$ and $\nabla_{y} \varphi = - \theta$. The equivalence of phase theorem thus guarantees that a unique symbol $s$ exists with the property above.

Now that we've obtained an expression of $f$ as a superposition of wave packets of the required form, the construction of the parametrix is rather easy. If we set
%
\[ a(x,y,\theta,\lambda) = s( x, y, \lambda \theta ), \]
%
then $a$ is a symbol of order zero in the $\lambda$ variable, and so we can find asymptotic solutions
%
\[ a(x,t,y,\theta,\lambda) \]
%
to the wave equation as $\lambda \to \infty$. We now introduce our parametrix, the operator
%
\begin{align*}
	Af(x,t) &= \int a \left( x, t, y, \frac{\theta}{|\theta|} , |\theta| \right) e^{2 \pi i \phi(x,t,y,\theta)} f(y)\; dy\; d\theta.
\end{align*}
%
The inversion formula we constructed in the previous paragraph implies that $Af = f$ for $t = 0$. And the fact that $a$ gives high-frequency asymptotic solutions to the half-wave equation for each fixed $y$ and $\theta$ implies that the kernel of $L \circ A$ can be written as
%
\[ \int b \left( x, t, y, \frac{\theta}{|\theta|} , |\theta| \right) e^{2 \pi i \phi(x,t,y,\theta)}\; d\theta, \]
%
where $b$ is a symbol of order $-\infty$ in $|\theta|$. But this is sufficient to conclude that $L \circ A$ is a smoothing operator.

These facts immediately justify that $A$ is a parametrix for the solution operator $S$ to the half-wave equation. Indeed, if we let $L \circ A$ have smooth kernel $K$, then Duhamel's principle implies that
%
\[ A(x,t,y) = S(x,t,y) + \int_0^t (e^{isP} K)(x,s,y)\; ds. \]
%
It is simple to see that the integral on the right hand side is a smooth function of $x$, $y$, and $t$ for $|t| \leq \varepsilon$. Thus we see that $A - S$ is smoothing, and so $A$ is a parametrix. Fantastic!

\section{Hamilton-Jacobi Theory}

We have constructed a parametrix of the form
%
\[ Af(x,t) = \int a(x,y,t,\theta) e^{2 \pi i \phi(x,y,t,\theta)} f(y)\; dy. \]
%
where $a$ is a symbol of order zero, and $\phi$ is a non-degenerate phase function. Thus $A$ is a Fourier integral operator of order $-1/4$. But what is it's canonical relation? To determine this, we must look at the eikonal equation
%
\[ p(y,\eta) = p(x, \nabla_x \varphi(x,y,\eta)), \]
%
and in particular, the Hamilton-Jacobi theory used to guarantee the existence of solutions to this equation for $|x - y| \leq R$. Since our pseudodifferential operator is formally positive elliptic, $p$ is a positive function.

We introduce the Hamiltonian vector field
%
\[ H = \sum_j \frac{\partial p}{\partial \xi_j} \frac{\partial}{\partial x_j} - \frac{\partial p}{\partial x_j} \frac{\partial}{\partial \xi_j} \]
%
on $T^* \Omega$. We will call the integral curves of this Hamiltonian vector field the \emph{bicharacteristics} of $p$.

For a given $y$ and $\theta$, with $\theta \neq 0$, the ellipticity of $p$ implies that the hypersurface
%
\[ \Sigma(y,\theta) = \{ (x,\xi) : p(x,\xi) = p(y,\theta) \} \]
%
has dimension $2d - 1$. The eikonal equation we must solve states precisely that
%
\[ (x, \nabla_x \varphi(x)) \in \Sigma(y,\theta) \]
%
for all $|x - y| \leq R$. This is an underdetermined equation, i.e. for each $x$, there is a $d-1$ dimensional family of choices of $\nabla_x \varphi(x)$ which satisfy the equation.

To prove the existence and uniqueness of $\varphi$, we now rely on some Lagrangian geometry.  For any smooth function, $Hf = \{ p, f \}$, where $\{ \cdot, \cdot \}$ is the Poisson bracket on $T^* \Omega$. In particular, we conclude that $Hp = \{ p, p \} = 0$, and thus $p$ is constant on it's bicharacteristics. But this means that $\Sigma(y,\theta)$ is the union of all the bicharacteristics that pass through it. We call such a manifold \emph{coisotropic}.

Any coisotropic manifold has a unique foliation into $d$ dimensional, Lagrangian submanifolds of $T^* \Omega$; take the leaf $\Sigma$ of the foliation passing through $(y,\theta)$ itself. Since $\Sigma$ is Lagrangian, it must contain all bicharacteristics that pass through $\Sigma$, and in fact, it is the union of these bicharacteristics. But the Hamiltonian vector field $H$ never directly lies in the $\xi$ direction, which implies that $\Sigma$ is actually given by a local section of $T^* \Omega$. The theory of Lagrangian sections thus implies the existence of $R > 0$, and a unique real-valued function $\varphi$ defined for $|x - y| \leq R$, such that $\nabla \varphi(y) = \theta$, and $(x,\nabla \varphi(x)) \in \Sigma$ for $|x - y| \leq R$. We immediately see from this that $\varphi$ solves the eikonal equation. And $\varphi$ is the unique such solution, because any such solution must parameterize some Lagrangian submanifold passing through $(y,\theta)$ and contained in $\Sigma(y,\theta)$, and $\Sigma$ is the unique such submanifold.

This geometry actually gives us more information about $\varphi$ then is directly postulated from the form of the eikonal equation. Namely, the value of the quantity $\nabla_x \varphi(x,y,\theta)$ is equal to the unique point on the Lagrangian submanifold of $\Sigma(y,\theta)$ passing through $(y,\theta)$, which lies above $x$. This gives enough information to characterize the canonical relation of the parametrix $A$.

Let $\{ \Phi_t \}$ denote the phase flow corresponding to the Hamiltonian vector field $H$. We claim that the canonical relation of the parametrix $A$ is equal to
%
\[ \mathcal{C} = \Big\{ (x,\xi) = \Phi_t(y, \eta)\ \text{and}\ \tau = p(y,\eta) = p(x,\xi) \Big\}. \]
%
Indeed, the operator of $A$ is given by a phase of the form
%
\[ \phi(x,t,y,\theta) = \varphi(x,y,\theta) + t p(y,\theta), \]
%
which is non-degenerate. The associated canonical relation must therefore be a $2d + 1$ dimensional Lagrangian submanifold of $T^*(\RR^d_x \times \RR_t \times \RR^d_\xi)$. Provided we can show any values $(x,\xi,t,\tau,y,\theta)$ that lies in $\mathcal{C}$ in the set above \emph{also} lies in the canonical relation defining $A$, the fact that both manifolds have equal dimension would be sufficient to prove equality.

In the last paragraph, we argued that if $(x,\xi,t,\tau,y,\eta) \in \mathcal{C}$, then we must necessarily have
%
\[ \xi = \nabla_x \varphi(x,y,\theta) = \nabla_x \phi(x,t,y,\theta). \]
%
We also check directly that
%
\[ \tau = p(y,\theta) = (\partial_t \phi)(x,t,y,\theta). \]
%
We must also check that
%
\[ (\nabla_\xi \phi)(x,t,y,\theta) = 0. \]
%
If $t = 0$, then $(x,\xi) = (y,\theta)$, and we see that $(\nabla_\xi \phi)(x,0,y,\theta) = 0$. since $x = y$ is the only case in which this gradient vanishes. Now set
%
\[ F(t) = (\nabla_\xi \phi)( x(t), t, y, \theta ) = (\nabla_\xi \varphi)(x(t), y, \theta) + t (\nabla_\xi p)(y,\theta), \]
%
where $(x(t), \xi(t)) = \Phi_t(y,\theta)$. Then
%
\begin{align*}
	\frac{dF_j}{dt} &= \sum_k \frac{\partial^2 \varphi}{\partial x_k \partial \theta_j}(x(t), y, \theta) \frac{dx_k(t)}{dt} + \frac{\partial p}{\partial \xi_j}(y,\theta)\\
	&= \sum_k \frac{\partial^2 \varphi}{\partial x_k \partial \theta_j}(x(t), y, \theta) \frac{\partial p}{\partial \xi_k}(x(t), \xi(t)) + \frac{\partial p}{\partial \xi_j}(y,\theta).
\end{align*}
%
Since
%
\[ p(x, \nabla_x \varphi(x,y,\theta)) = p(y, \theta), \]
%
taking derivatives on both sides in $\theta_j$ gives that
%
\[ \sum_k \frac{\partial \varphi}{\partial x_k \partial \theta_j}(x,y,\theta) \frac{\partial p}{\partial \xi_k}(x, \nabla_x \varphi(x,y,\theta)) = \frac{\partial p}{\partial \theta_j}(y,\theta). \]
%
Taking $x = x(t)$, we have $(\nabla_x \varphi)(x(t), y,\theta) = \xi(t)$, which implies that (TODO: FIND SIGN ERROR SOMEWHERE) that $dF_j / dt = 0$, which implies $F_j(t) = 0$ for all $t$ given that $F_j(0) = 0$.

The only thing left to check is that
%
\[ (\nabla_{y} \phi)(x,t,y,\theta) = \theta. \]
%
TODO.

As a particular example of this, consider the example of the Laplace-Beltrami operator
%
\[ P = \sqrt{-\Delta_g} \]
%
introduced in Section 2. The principal symbol of this equation is given by $p(x,\xi) = |\xi|_g$, where $|\cdot|_g$ is the Riemannian metric applied to the covector $\xi$. If we plug this principal symbol into the Hamilton-Jacobi theory above, we see that the bicharacteristics of the Hamiltonian vector field $H$ are precisely the integral curves of the \emph{geodesic flow} in the cotangent space. Thus we conclude that the wavefront set of the parametrix for the half-wave operator is \emph{precisely} the `geodesic light cone'.

Let's return to the primordial sources of the parametrix for the construction of the half-wave equation. In the study of quantum physics, scientists were lead to the study of the equation
%
\[ \partial_t = i P(x,D), \]
%
where $P$ described the motion of a classical system. It is a heuristic in quantum physics that \emph{high energy} wave packets behave like their classical counterpart (for what is classical physics but the behaviour of objects at a scale much larger than the quantum realm, such that any significant motion carries with it an absurdly high amount of energy from the quantum perspective). The description of $\varphi$ given above 


 there is a unique one dimensional bicharacteristic $\Sigma$ passing through $(y,\xi)$ in $T^* \RR^d$.





\begin{comment}

\section{Leftovers}




The formula we have given above is now sufficient to obtain a family of asymptotic solutions to the wave equation which works for a large family of initial conditions localized in phase space. 
%
%Let us suppose that
%
%\[ \phi(x,0) = (x - x_0) \cdot \xi_0 \]
%
%for some fixed $x_0$ and $\xi_0$, and that
%
%\[ a(x,0,\lambda) = \chi(x - x_0) \]
%
%for some smooth, compactly supported function $\chi$.
We have
%
\[ L \{ u_\lambda \} = e^{2 \pi i \lambda \phi(x,t)} b(x,t,\lambda). \]
%
If we assume that $a_\mu(x_0,0,\lambda) \neq 0$, then in order for $b$ to be a symbol of order $\mu$, the phase $\phi$ must solve the equation
%
\[ \partial_t \phi = p \left( x, \frac{\nabla_x \phi}{2 \pi} \right). \]
%
To guess how to solve this equation, let's separate variables, and suppose $\phi$ is of the form
%
\[ \phi(x,t) = \varphi(x) + a t, \]
%
for some constant $a$, and some function $\varphi$, then the equation above becomes the \emph{eikonal equation}
%
\[ a = p \left( x, \frac{\nabla_x \varphi}{2 \pi} \right). \]
%
Let us prescribe the initial conditions that $\nabla_x \varphi(x_0) = \xi_0$, that $\varphi$ vanishes on the hyperplane passing through $x_0$, and orthogonal to $\xi_0$. Then we see that
%
\[ a = p \left( x_0, \frac{\xi_0}{2 \pi} \right), \]
%
and the equation becomes
%
\[ p \left( x_0, \xi_0 \right) = p \left( x, \nabla_x \varphi \right). \]
%
TODO: Verify derivative conditions. The theory of Hamilton-Jacobi equations shows that, given the initial conditions above, there is a \emph{unique} function $\varphi$ solving this equation in a small neighborhood of the point $x_0$. We'll take $\phi$ to be this unique function in what follows, i.e. we set
%
\[ \phi(x,t) = \varphi(x) + \frac{p( x_0, \xi_0)}{2 \pi} \cdot t \]
%
where $\varphi$ is specified as above. This is sufficient to conclude that $b$ is a symbol of order $\mu$ in a neighborhood of $x_0$ and for small times, regardless of the choice of symbol $a$.

Stone's Theorem tells us that we should only expect a well-defined theory of operators of the form $e^{itP}$ if the operator $P$ is a self-adjoint operator. Conveniently, this condition also implies that the principal symbol $p$ is real-valued, which implies that the zeroes of the symbol of the operator $L$ in the variable $\tau$ are purely imaginary, so we can apply the theory of hyperbolic equations to such operators. In order to have a non-singular theory of characteristics for the hyperbolic equation, it is necessary to assume that $p$ is a non-vanishing symbol, i.e., that $P$ is an elliptic pseudodifferential operator. So these are our assumptions, we study an equation of the form $L$, where $P$ is a self-adjoint, elliptic pseudodifferential operator of order one. Here are some important examples to keep in mind:
%
\begin{itemize}
	\item If $M$ is a Riemannian manifold, then we can define the Laplace-Beltrami operator
	%
	\[ \Delta_g f = |g|^{-1/2} \sum_i \partial_i ( |g|^{1/2} g^{ij} \partial_ j f ). \]
	%
	The operator $\sqrt{-\Delta_g}$ is a self-adjoint, elliptic pseudodifferential operator of order one with principal symbol $|\xi|_g$.

	\item s 
\end{itemize}

\end{comment}

\begin{thebibliography}{9}

\end{thebibliography}

\end{document}











- Let X be be a compact Riemannian manifold
	- Then -Delta is essentially self-adjoint and non-negative, so spectral calculus tells us that sqrt(-Delta)
	- 

