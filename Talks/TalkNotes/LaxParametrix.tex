\documentclass{article}

%% for editing
%\usepackage{changes}
%\usepackage[final]{changes} %% comment above line and uncomment this line to see final copy without markup
%\setremarkmarkup{~(#2)}
\usepackage{color}


\usepackage{amsmath}
\usepackage{amssymb}
\usepackage{amsthm}
\usepackage{multicol}
%\usepackage[margin=1in]{geometry}
\usepackage{graphicx}
\usepackage{tikz}
\usepackage{hyperref}
\usepackage{mathabx}
\usepackage{comment}

\usepackage{tensor}

\theoremstyle{plain}
\newtheorem{theorem}{Theorem}
\newtheorem{lemma}[theorem]{Lemma}
\newtheorem{corollary}[theorem]{Corollary}
\newtheorem{prop}[theorem]{Proposition}

\theoremstyle{remark}
\newtheorem*{remark}{Remark}
\newtheorem*{example}{Example}
\newtheorem*{proof*}{Proof}

\theoremstyle{definition}
\newtheorem*{defi}{Definition}
\newenvironment{definition}
    {\begin{samepage}\begin{framed}\begin{defi}}
    {\end{defi}\end{framed}\end{samepage}}


\DeclareMathOperator{\diam}{\text{diam}}

\DeclareMathOperator{\QQ}{\mathbb{Q}}
\DeclareMathOperator{\ZZ}{\mathbb{Z}}
\DeclareMathOperator{\RR}{\mathbb{R}}
\DeclareMathOperator{\HH}{\mathbb{H}}
\DeclareMathOperator{\BB}{\mathbb{B}}
\DeclareMathOperator{\CC}{\mathbb{C}}
\DeclareMathOperator{\AB}{\mathbb{A}}
\DeclareMathOperator{\PP}{\mathbb{P}}
\DeclareMathOperator{\MM}{\mathbb{M}}
\DeclareMathOperator{\VV}{\mathbb{V}}
\DeclareMathOperator{\TT}{\mathbb{T}}
\DeclareMathOperator{\LL}{\mathcal{L}}
\DeclareMathOperator{\DD}{\mathcal{D}}
\DeclareMathOperator{\SW}{\mathcal{S}}
\DeclareMathOperator{\EC}{\mathcal{E}}
\DeclareMathOperator{\AC}{\mathcal{A}}

\DeclareMathOperator{\EE}{\mathbb{E}}
\DeclareMathOperator{\NN}{\mathbb{N}}

\DeclareMathOperator{\II}{\mathbb{I}}

\DeclareMathOperator{\DQ}{\mathcal{Q}}


\DeclareMathOperator{\IA}{\mathfrak{a}}
\DeclareMathOperator{\IB}{\mathfrak{b}}
\DeclareMathOperator{\IC}{\mathfrak{c}}
\DeclareMathOperator{\IP}{\mathfrak{p}}
\DeclareMathOperator{\IQ}{\mathfrak{q}}
\DeclareMathOperator{\IM}{\mathfrak{m}}
\DeclareMathOperator{\IN}{\mathfrak{n}}
\DeclareMathOperator{\IK}{\mathfrak{k}}
\DeclareMathOperator{\ord}{\text{ord}}
\DeclareMathOperator{\Ker}{\textsf{Ker}}
\DeclareMathOperator{\Coker}{\textsf{Coker}}
\DeclareMathOperator{\emphcoker}{\emph{coker}}
\DeclareMathOperator{\pp}{\partial}
\DeclareMathOperator{\tr}{\text{tr}}
\DeclareMathOperator{\Ree}{\text{Re}}


\DeclareMathOperator{\BL}{\text{BL}}

\DeclareMathOperator{\dstrike}{//}

\DeclareMathOperator{\supp}{\text{supp}}

\DeclareMathOperator{\codim}{\text{codim}}

\DeclareMathOperator{\minkdim}{\dim_{\mathbb{M}}}
\DeclareMathOperator{\hausdim}{\dim_{\mathbb{H}}}
\DeclareMathOperator{\sobdim}{\dim_{\mathbb{S}}}
\DeclareMathOperator{\lowminkdim}{\underline{\dim}_{\mathbb{M}}}
\DeclareMathOperator{\upminkdim}{\overline{\dim}_{\mathbb{M}}}
\DeclareMathOperator{\lhdim}{\underline{\dim}_{\mathbb{M}}}
\DeclareMathOperator{\lmbdim}{\underline{\dim}_{\mathbb{MB}}}
\DeclareMathOperator{\packdim}{\text{dim}_{\mathbb{P}}}
\DeclareMathOperator{\fordim}{\dim_{\mathbb{F}}}

\DeclareMathOperator{\CT}{ {{\otimes}^\wedge} }

\DeclareMathOperator{\msupp}{\text{$\mu$-supp}}
\DeclareMathOperator{\singsupp}{\text{sing-supp}}
\DeclareMathOperator{\Char}{\text{Char}}

\DeclareMathOperator*{\argmax}{arg\,max}
\DeclareMathOperator*{\argmin}{arg\,min}

\DeclareMathOperator{\ssm}{\smallsetminus}

\newcommand{\myphi}[1]{ \tensor[^{\phi}]{{#1}}{}}




\title{The Lax Parametrix for the Half Wave Equation}
\author{Jacob Denson}

\begin{document}

\maketitle

In this talk, we consider a motivating example that gave rise to much of the general theory of Fourier integral operators: the study of variable-coefficient wave equations. This show a precise example of how Fourier integral operators can be used as a tool to generalize the tools of harmonic analysis we normally use to analyze constant coefficient differential operators like the wave equation, and apply them to variable coefficient analogues.

% Prelude
\section{Euclidean Half-Wave Propogators}

Let's start with a quick review. Consider a solution $u(x,t)$ to the wave equation
%
\[ (\partial_t^2 - \Delta) u = 0 \]
%
on $\RR^d$. We take Fourier transforms on both sides; if $\widehat{u}(\xi,t)$ denotes the Fourier transform of $u$ in the $x$-variable, then we conclude that
%
\[ (\partial_t^2 + 4 \pi^2 |\xi|^2) \cdot \widehat{u}(\xi,t) = 0. \]
%
This is an ordinary differential equation in the $t$ variable for each fixed $\xi$, which we can solve, given that $u(x,0) = f(x)$, and $\partial_t u(x,0) = g(x)$, to write
%
\[ \widehat{u}(\xi,t) = \widehat{f}(\xi) \cos(2 \pi t |\xi|) + \widehat{g}(\xi) \frac{\sin(2 \pi t |\xi|)}{2 \pi |\xi|}. \]
%
It's often easier to see what's going on if we work with complex exponentials, so we rewrite this as
%
\[ \left( \frac{\widehat{f}(\xi) + (2 \pi |\xi|)^{-1} \widehat{g}(\xi)}{2} \right) e^{2 \pi i t |\xi|} + \left( \frac{\widehat{f}(\xi) - (2 \pi |\xi|)^{-1} \widehat{g}(\xi)}{2i} \right) e^{-2 \pi i t |\xi|}. \]
%
If we write $u = v + w$, where
%
\[ \widehat{v}(\xi,t) = \left( \frac{\widehat{f}(\xi) + \widehat{g}(\xi)}{2} \right) e^{2 \pi i t |\xi|} \quad\text{and}\quad \widehat{w}(\xi,t) = \left( \frac{\widehat{f}(\xi) - i (2 \pi |\xi|)^{-1} \widehat{g}(\xi)}{2} \right) e^{-2 \pi i t |\xi|}, \]
%
thus we have decomposed the solution $u$ to the wave equation into a sum of solutions to the \emph{half-wave equations}
%
\[ \Big( \partial_t - i \sqrt{-\Delta} \Big) v = 0 \quad\text{and}\quad \Big(\partial_t + i \sqrt{-\Delta} \Big) w = 0. \]
%
The two operators $\partial_t - i \sqrt{-\Delta}$ and $\partial_t + i \sqrt{-\Delta}$ are identical, up to a time-reversal symmetry, so we focus on solutions to the equation $\partial_t - i \sqrt{-\Delta} = 0$. Solutions to the half-wave equation behave similarly to solutions to the wave equation, with one notable exception: the wave equation has a finite speed of propogation, whereas the half-wave equation does not.

The fact that the half-wave equation $(\partial_t - i \sqrt{-\Delta}) u = 0$ is a first-order operator makes the Cauchy problem somewhat simpler to study, since we need less initial data than in the wave equation. In particular, we can define operators $e^{it \sqrt{-\Delta}}$ such that
%
\[ v(x,t) = (e^{i t \sqrt{-\Delta}} v_0)(x) \]
%
gives solutions too the wave equation. We now briefly look at these operators, and the solution operator
%
\[ Sf(x,t) = (e^{i t \sqrt{-\Delta}} f)(x) \]
%
to the half-wave equation, from the perspective of FIO theory.

\begin{comment}

Fourier transform of e^{-|x|} is

	c_d (1 + |xi|^2)^{- (d+1) / 2}

Fourier transform of e^{- lambda |x|} for lambda > 0 is

	int e^{- lambda |x|} e^{-i xi * x} dx
	= c_d lambda^{-d} (1 + |xi|^2 / lambda^2)^{-(d+1)/2}

Can we now apply analytic continuation to conclude that
the Fourier transform of e^{-it |x|} is equal to

	c_d (it)^{-d} ( 1 + |xi|^2 / (it + 0)^2 )^{-(d+1)/2}

where we take the branch of the square root that is the analytic extension
away of the normal square root away, defined away from the imaginary axis.

Then for |xi| > t this quantity is equal to

	c_d i t^{-d} ( |xi|^2 / t^2 - 1 )^{-(d+1)/2}

e^{-i pi / 2}

sqrt( a e^{it} ) = sqrt(a) sqrt(e^{it/2})

\end{comment}

\begin{comment}

Let's start with the operators $\{ e^{it \sqrt{-\Delta}} \}$. We can write
%
\begin{align*}
	e^{i t \sqrt{-\Delta}} f(x) &= \int e^{2 \pi i (t |\xi| + \xi \cdot x)} \widehat{f}(\xi)\; d\xi\\
	&= \int e^{2 \pi i (t |\xi| + \xi \cdot (x - y))} f(y)\; dy\; d\xi\\
	&= \int a(x,y,\xi) e^{2 \pi i \phi_t(x,y,\xi)} f(y)\; dy\; d\xi,
\end{align*}
%
where $a(x,y,\xi) = 1$, and $\phi_t(x,y,\xi) = t |\xi| + \xi \cdot (x - y)$. The function $a$ is a symbol of order zero, and the function $\phi_t$ is a non-degenerate phase function with canonical relation defined by the equations
%
\[ \left\{ x = y + t \frac{\xi}{|\xi|}\ \text{and}\ \xi = \eta \right\}, \]
%
which, for a fixed $y$, we can think of as consisting of the sphere of radius $t$ at $y$, and all \emph{outward} pointing cotangent vectors on this sphere. Thus the operators $e^{it \sqrt{-\Delta}}$ are Fourier integrals of order $0$.

\end{comment}

We can write
%
\begin{align*}
	Sf(x,t) &= e^{2 \pi it \sqrt{-\Delta}} f(x)\\
	&= \int e^{2 \pi i (t |\xi| + \xi \cdot (x - y))} f(y)\; dy\; d\xi\\
	&= \int a(x,t,y,\xi) e^{2 \pi i \phi(x,t,y,\xi)} f(y)\; dy\; d\xi,
\end{align*}
%
where
%
\[ a(x,t,y,\xi) = 1 \]
%
is a symbol of order zero, and the phase
%
\[ \phi(x,t,y,\xi) = t |\xi| + \xi \cdot (x - y) \]
%
is non-degenerate. Thus $S$ is a Fourier integral operator from $\RR^d$ to $\RR^d \times \RR$, of order $-1/4$, with canonical relation defined by the three equations
%
\[ \left\{ x = y + t \frac{\xi}{|\xi|}\ \text{and}\ \xi = \eta\ \text{and}\ \tau = |\xi| \right\}. \]
%
In this talk, we will construct approximate solutions (parametrices) for variable-coefficient analogues of the half-wave and wave equations using Fourier integral operators.

\section{Variable-Coefficient Half-Wave Equations}

A natural variable-coefficient extension of the wave equation $\partial_t^2 - \Delta = 0$ is an equation of the form $\partial_t^2 - Q = 0$, where $Q$ is a pseudo-differential operator of order two, which is \emph{formally positive}\footnote{A Schwartz operator $Q$ is formally positive if for any $f \in C_c^\infty(\RR^d)$, $\langle Qf, f \rangle \geq 0$.}, \emph{elliptic}\footnote{A ${\Psi}DO$ $Q(x,D)$ is elliptic if it's principal symbol $q(x,\xi)$ satisfies $q(x,\xi) > 0$ for $\xi \neq 0$.}, and classical \footnote{A symbol $Q(x,\xi)$ of order $\mu$ is classical if we have an asymptotic expansion $Q \sim \sum q_{\mu-j}$, where $q_k$ is a smooth, homogeneous function of order $k$.}. For those not too comfortable with pseudodifferential operators, any formally positive, elliptic second order differential operator $Q$ fits these assumptions. As a non-trivial example, consider a Riemannian metric $g$ on $\RR^d$, introduce the Laplace-Beltrami operator
%
\[ \Delta_g f = |g|^{-1/2} \sum_j \frac{\partial}{\partial x_j} \left\{ |g|^{1/2} \sum_k g^{jk} \partial_k f \right\}, \]
%
and set $Q = - \Delta_g$. If we discard the first order terms of this operator, we find that the second order terms of $\Delta_g$ is
%
\[ \sum_{j,k} g^{jk} \partial_j \partial_k. \]
%
Thus the principal symbol of $Q$ is
%
\[ q(x,\xi) = - \sum_{j,k} g^{jk} ( 2 \pi i \xi_j ) (2 \pi i \xi_k) = 4 \pi^2 |\xi|_g^2, \]
%
which is immediately verified to be elliptic.

We will not analyze these wave equations directly. If we define $P = Q^{1/2}$, well defined by the calculus of pseudodifferential operators, then $P$ is a formally positive, classical pseudodifferential operator of order one. Using similar tricks to the wave equation, we can split solutions to the wave equation $\partial_t^2 - Q$ into the sum of two solutions to the \emph{half-wave equations} $\partial_t - i P = 0$ and $\partial_t + i P = 0$. By time symmetry, we can focus solely on studying the solutions to the operator
%
\[ L = \partial_t - i P, \]
%
where $P$ is a first order \emph{pseudodifferential operator}. The pseudodifferential operator $P$ has principal symbol $p(x,\xi) = q(x,\xi)^{1/2}$; in particular, the operator $P = \sqrt{-\Delta_g}$ has principal symbol $p(x,\xi) = 2 \pi |\xi|_g$.

We are not interested in the study of the existence or uniqueness of solutions to this PDE, but the problem of \emph{regularity}, e.g. the mapping properties of the solution and propogator operators in $L^p$ norms or Sobolev spaces. We thus refer to the literature on hyperbolic equations, which states that, given the assumptions on the pseudodifferential operator $P$ above, for any compact set $K \subset \RR^d$, there exists $\varepsilon > 0$ such that smooth solutions
%
\[ u: \RR^d \times [-\varepsilon,+\varepsilon] \to \RR \]
%
to the half-wave equation exist, and (via energy type arguments) are the unique such solutions in $L^\infty_t L^2_x$ to solve the wave equation with some initial condition given by a smooth, compactly supported functions on $K$. We are interested in finding integral expressions which give approximate expressions for $u$, which are sufficient to study more quantitative regularity problems associated with solutions to the half-wave equation.

\begin{comment}
For such an operator, we can find a countable, orthogonal basis $\{ e_\lambda : \lambda \geq 0 \}$ of $L^2(M)$, such that $Pe_\lambda = \lambda e_\lambda$, and such that for each $\lambda$, $e_\lambda$ is a smooth function, with $L^2_s$ norm $O_s(\lambda^s)$. The number of elements of this basis with eigenvalue at most $\lambda$ is $O(1 + \lambda^d)$. Given this basis, we can define a family of bounded operators $\{ e^{2 \pi itP} \}$ on $L^2(M)$ by setting $e^{2 \pi it P} e_\lambda = e^{2 \pi it \lambda} e_\lambda$, and then consider a solution operator
%
\[ (Wf)(x,t) = (e^{2 \pi it \lambda} f)(x). \]
%
Using the $L^2_s$ norm properties of the eigenfunctions $\{ e_\lambda \}$, and the fact that for any $f \in C^\infty(M)$,
%
\[ \langle f, e_\lambda \rangle \lesssim_N \lambda^{-N} \quad\text{for all $N > 0$,} \]
%
it is simple to check that $W$ maps $C^\infty(M)$ into $C^\infty_{\text{loc}}(M \times \RR)$, and that for $f \in C^\infty(M)$, the smooth function $u = Wf$ solves the equation $Lu = 0$ with initial conditions $f$. One can also prove the uniqueness of solution solutions, e.g. using energy estimates, but this takes us a little far afield of what we want to talk about in these notes.
\end{comment}

By an integral expression `approximating' solutions to the half-wave equation, we mean finding a \emph{parametrix} $A$ for the solution operator $S$ to the Cauchy problem $Lu = 0$, such that $A$ has a good integral expression as a Fourier integral operator. A parametrix in this context is an operator $A$ which differs from the operator $S$ by a \emph{smoothing operator} $R = A - S$, i.e. a Schwartz operator whose kernel is a smooth function. The mapping properties of the operator $S$ with respect to Sobolev norms then immediately reduces to the mapping properties of the operator $A$, because the operator $R$ has trivial mapping properties. For example, if $u$ is a compactly supported distribution, then $Ru$ is a smooth function, and moreover, $R$ maps any compactly supported function in one Sobolev space continuously into a function locally lying in \emph{any} other Sobolev space, e.g. mapping $H^s_{x,c}$ continuously into $H^{s_1}_{x,\text{loc}} H^{s_2}_{t,\text{loc}}$ for any parameters $s$, $s_1$, and $s_2$.

The reason parametrices arise is that in many variable coefficient problems, is that it is often possible to find operators approximating $S$ given by simpler expressions, whereas no simple expression might exist giving a precise formula for $S$. This in particular arises from the perspective of harmonic analysis, since it is often the case that we can find good approximations to solutions to partial differential equations for \emph{high frequency initial data}, whereas these approximations are no longer so accurate for \emph{low frequency initial data}. From the perspective of parametrices, this is not a problem since low frequency data is automatically smooth, and thus does not need to be approximated as well as high frequency data.

% the behaviour of solution operators can only be given analytical expressions asymptotically, i.e. via expressions that are never completely accurate, but become more accurate as we take input data which is oscillating more and more rapidly. Taking the difference between the actual solution, and this asymptotic expression, we obtain an expression which becomes minute as we plug in input data that is oscillating rapidly. Slowly oscillating functions are \emph{smooth}, and so the operator formed from taking the difference between the actual solution, and the asymptotic formula, will then be smoothing.

Our proof technique will show that solutions to the half-wave equation induced by initial data given by a high-frequency wave packet, localized in space near a point $x_0$, and oscillating at a frequency $\xi_0$, where $|\xi_0|$ is large, will at each time $t$, look like a wave packet localized near a point $x(t)$ and oscillating a frequency $\xi(t)$, where $x$ and $\xi$ are functions solving the \emph{Hamiltonian system}
%
\[ \frac{dx}{dt} = - (\nabla_\xi p)(x, \xi) \quad \frac{d \xi}{dt} = (\nabla_x p)(x,\xi). \]
%
In particular, we will find that the \emph{singularities} of the kernel to the solution operator to the half-wave equation are supported on the integral curves of this system. This formalizes a number of heuristics used by physicists well before they were studied by mathematicians, namely, that visible light (a high-frequency wave relative to the scale of the human eye) travels according to Fermat's principle of least time, and that high energy particles in quantum mechanics whose state is given by a wave packet behave like classical particles, moving through space according to the principles of classical mechanics. The methods we use here to construct asymptotics for the half-wave equation are thus descendents of the \emph{WKB methods} of quantum physicists of the 1920s, and going even further back, descendents of the analytical methods of geometric optics discovered by Fresnel and Airy in the 1800s.

% Indeed, the methods we used here, first applied to the half wave equation in the 1960s, really have their root in the methods of quantum physicists of the 1920s, with their WKB method for approximating solutions to the Schr\"{o}dinger equation. One could even argue that the roots of these methods emerged even earlier, in the analytical methods of geometric optics discovered by Fresnel and Airy in the 1800s.

\begin{comment}

The main example of a half-wave equation to which we can apply our method are obtained by considering some non-flat Riemannian metric $g$, and considering the resulting equation
%
\[ \partial_t - 2 \pi i \sqrt{-\Delta_g} \]
%
where
%
\[ \Delta_g f = |g|^{-1/2} \sum_i \partial_i \{ |g|^{1/2} g^{ij} \partial_j f \} \]
%
is the Laplace-Beltrami operator. Similar equations are obtained in quantum mechanics, Schr\"{o}dinger-type equations of the form $\partial_t = i P(x,D)$, describing the behaviour of a classical mechanical system described by the \emph{Hamiltonian} $P(x,\xi)$, i.e. the system
%
\[ \frac{dx}{dt} = \frac{\partial P}{\partial \xi} \quad\text{and}\quad \frac{d\xi}{dt} = - \frac{\partial P}{\partial x} \]
%
at a quantum scale. However, the principal part of $P$ will generally be homogeneous of degree two in the $\xi$-variable, since kinetic energy is often a quadratic form in the momentum variables. One can use the methods described here to construct approximate solutions to these equations. But the resulting operators, though given by oscillatory integrals, will not have phases that are homogeneous of degree one, a necessary part of the theory Fourier integrals we've discussed this semester, and so the methods here do not directly apply. Nonetheless, the physical intuition behind the Schr\"{o}dinger equation will be helpful for the construction of the parametrices we construct here. Indeed, we will see that the `semiclassical' behaviour of the Schr\"{o}dinger equation at large scales is analogous to the analytical expressions we will obtain for our solutions, associated with a suitable Hamiltonian equation. Indeed, the methods we used here, first applied to the half wave equation in the 1960s, really have their root in the methods of quantum physicists of the 1920s, with their WKB method for approximating solutions to the Schr\"{o}dinger equation. One could even argue that the roots of these methods emerged even earlier, in the analytical methods of geometric optics discovered by Fresnel and Airy in the 1800s.

\end{comment}

% The parameterices constructed using the techniques here will therefore have a phase which is homogeneous of degree two in the $\xi$ variable, which doesn't quite fit the system considered here, because the Fourier Integral operators we study have phases that are homogeneous of degree one in the phase variable.

% In physics, the use of such parametrices in the study of the Schr\"{o}dinger equation is called the \emph{WKB method}.

\section{High-Frequency Asymptotic Solutions}

Fix $x_0 \in K$, as well as three quantities $0 < r < R$, and $\varepsilon > 0$, to be specified later. Our goal is to find a general family of `high-frequency asymptotic solutions' to the half-wave equation, supported on the ball $B_R(x_0) = \{ x: |x - x_0| \leq R \}$, for $|t| \leq \varepsilon$, given some initial conditions supported on the smaller ball $B_r(x_0)$.

%Since we're trying to construct such a family locally, despite working on a compact manifold, we can switch to studying the operator in coordinates, so we'll abuse notation, and assume we're working with a pseudodifferential operator defined on some open subset $\Omega$ of $\RR^d$, and that we only care about values of $y$ lying in some compact subset $K$ of $\Omega$, where $\Omega$ contains the closure of the $R$-neighborhood of the set $K$.

Let us describe what we mean by `high-frequency asymptotic solutions'. Fix an expression of the form
%
\[ u_\lambda(x,t) = e^{2 \pi i \lambda \phi(x,t)} a(x,t,\lambda), \]
%
where $a$ is a classical symbol of order zero in the $\lambda$ variable, defined for $|t| \leq \varepsilon$, and with $\text{supp}_x(u_\lambda) \subset B_R(x_0)$, and where $\phi$ is a smooth, real-valued function, such that $\nabla_x \phi(x,t) \neq 0$ on the support of $a$. This latter condition is necessary to interpret $u_\lambda$ as a function `oscillating at a magnitude $\lambda$'. Indeed, if the condition is true, the principle of nonstationary phase shows that the Fourier transform of $u_\lambda$ rapidly decays outside the annulus of frequencies $|\xi| \sim \lambda$. As $\lambda \to \infty$, the solution $u_\lambda$ thus begins to oscillate more and more rapidly.

%
%\begin{itemize}
%	\item $a$ is a `classical' symbol of order zero in the $\lambda$-variable, i.e. we can write
	%
%	\[ a \sim \sum_{j = 0}^\infty a_{-j}, \]
	%
%	where $a_{-j}$ is a smooth function, homogeneous of order $-j$ in the $\lambda$ variable. Moreover, $a$ is compactly supported in the $x$-variable and the $t$-variable.

%	\item The function $\phi$ is smooth, real-valued, and we assume $\nabla_x \phi(x,t) \neq 0$ on the support of $a$.
%\end{itemize}
%
%The latter assumption is natural because we think of $u_\lambda$ as being localized in phase space about the family of points
%
%\[ \Big\{ (x, t, \lambda \nabla_x \phi(x,t), \lambda \partial_t \phi(x,t)) \Big\} \]
%
%and so we need $\nabla_x \phi(x,t)$ to be nonvanishing, in order for us to think of $u_\lambda$ as being localized near spatial frequencies that have magnitude approximately $\lambda$, and thus localized to large frequencies when $\lambda$ is large.

In a lemma shortly following this discussion, we will show that for any choice of $a$ and $\phi$ as above, there exists a classical symbol $b$ of order $1$ such that
%
\[ L u_\lambda(x,t) = e^{2 \pi i \lambda \phi(x,t)} b(x,t,\lambda). \]
%
For \emph{some} choices of $a$ and $\phi$, it might be true that the higher order parts of $b$ are eliminated, i.e. so that $b$ is of order much smaller than $1$. If $a$ and $\phi$ are chosen in a \emph{very} particular way, it might be true that all finite order parts of $b$ are eliminated, so that $b$ is a symbol of order $-\infty$. In such a situation, we say $\{ u_\lambda \}$ is a `\emph{high-frequency asymptotic solution}' to the wave equation. If this is the case, then
%
\[ |\partial_x^\alpha \partial_t^\beta \{ L u_\lambda \}| \lesssim_{\alpha,\beta,N} \lambda^{-N} \quad\text{for all $N > 0$}, \]
% \quad\text{for all $N > 0$, $|t| \lesssim 1$, and $|x - x_0| \lesssim 1$}
which justifies that $u_\lambda$ behaves like a solution to the half-wave equation as $\lambda \to \infty$. We will prove the following `Cauchy-type' initial value problem for high-frequency asymptotic solutions to the half-wave equation, given that our phase satisfies an \emph{eikonal equation}.

\begin{theorem}
	Fix $(x_0,\xi_0)$, and suppose $\varphi$ is a smooth-real valued function on $B_R(x_0)$, solving the eikonal equation
	%
	\[ p(x, \nabla_x \varphi(x) ) = p(x_0,\xi_0), \]
	%
	where $p$ is the principal symbol of $P$, such that $(\nabla_x \varphi)(x_0) = \xi_0$. Set
	%
	\[ \phi(x,t) = \varphi(x) + t \cdot p(x_0,\xi_0). \]
	%
	Then there exists $\varepsilon > 0$, $r > 0$, and $R > 0$ such that any classical symbol $a(x,0,\lambda)$ of order zero supported on $|x - x_0| \leq r$, extends to a unique classical symbol $a(x,t,\lambda)$ of order zero, supported on $|x - x_0| \leq R$ and defined for $|t| \leq \varepsilon$, such that the associated family of functions
	%
	\[ u_\lambda(x,t) = e^{2 \pi i \lambda \phi(x,t)} a(x,t,\lambda) \]
	%
	are high-frequency asymptotic solutions to the half-wave equation.
\end{theorem}

In order to prove this result, we need to obtain some formulas that tell us what the symbol $b$ looks like, whose existence was postulated above, in terms of the phase $\phi$, the operator $P$, and the symbol $a$. In order to prove the theorem above, we'll construct $a$ recursively by slowly fixing the contributions of the higher order parts of $a$. One then studies the lower order terms separately, so it is wise to make a study of the functions
%
\[ u_\lambda(x,t) = e^{2 \pi i \lambda \phi(x,t)} a(x,t,\lambda),\]
%
where $a$ is a classical symbol of some arbitrary order $\mu$, rather than just a symbol of order zero. This is done in the following Lemma, whose proof can be found on the online version of these notes.
% is a somewhat technical application of stationary phase, and can be relegated to a second reading of these notes.

\begin{lemma}
	Let $p$ be the principal symbol of $P$. Consider
	% Let $P$ be a pseudodifferential operator of order $1$ given by a classical symbol of order one, i.e.
	%
	%\[ P \sim p + p_0 + p_-, \]
	%
	%where $p$ is the principal symbol, homogeneous of order one, $p_0$ is homogeneous of order zero, and $p_-$ is a symbol of order $-1$. Consider
	%
	\[ u_\lambda(x,t) = e^{2 \pi i \lambda \phi(x,t)} a(x,t,\lambda), \]
	%
	where $a$ and $\phi$ are as above, i.e. $a$ is a symbol of order $\mu$. Then
	%
	\[ b(x,t,\lambda) = e^{-2 \pi i \lambda \phi(x)} (L u_\lambda)(x,t) \]
	%
	is a classical symbol of order $\mu+1$, with principal symbol
	%
	%Then
	%
	%\[ e^{-2 \pi i \lambda \phi(x)} P \{ u_\lambda \}(x) \]
	%
	%is a classical symbol of order $\mu+1$, with principal symbol
	%
	%\[ \lambda \cdot p(x,\nabla_x \phi) \cdot a_\mu, \]
	%
	%and with order $\mu$ part given by
	%
	%\begin{align*}
	%	&(\nabla_\theta p)(x, \nabla_x \phi) \cdot (D_x a_\mu) + s(x) \cdot a_\mu + \lambda \cdot p(x,\nabla_x \phi) \cdot a_{\mu-1},
	%\end{align*}
	%
	\[ 2\pi i \lambda \Big( \partial_t \phi - p \left( x, \nabla_x \phi \right) \Big) a_\mu, \]
	%
	and with order $\mu$ part given by
	%
	\begin{align*}
		&2 \pi i \lambda \Big( \partial_t \phi - p \left( x, \nabla_x \phi \right) \Big) a_{\mu - 1}\\
		&\quad + \partial_t a_\mu -  (\nabla_\xi p)(x, \nabla_x \phi) \cdot (\nabla_x a_\mu) - i s \cdot a_\mu,
	\end{align*}
	%
	for a smooth, real-valued function $s$ depending only on $\phi$ and $P$.
\end{lemma}

\begin{remark}
	The result of this lemma shows why we must choose $\varphi$ to satisfy the eikonal equation in order for Theorem 1 to hold, i.e. because otherwise the order $1$ part of $e^{-2 \pi i \lambda \phi(x)} L u_\lambda$ can never vanish. Under the assumption that $\varphi$ satisfies the eikonal equation, we thus conclude in the Theorem above that for any symbol $a$ of order $\mu$, the function $b$ is a symbol of order $\mu$, with principal symbol
	%
	\[ \partial_t a_\mu -  (\nabla_\xi p)(x, \nabla_x \phi) \cdot (\nabla_x a_\mu) - i s \cdot a_\mu, \]
	%
	because the eikonal equation causes two of the three lines in the symbol expansion above to vanish.
\end{remark}

\begin{comment}

\begin{proof}
	Write $P(x,\xi) \sim p(x,\xi) + p_0(x,\xi) + p_-(x,\xi)$, where $p$ is the principal symbol, homogeneous of order one, $p_0$ is homogeneous of order zero, and $p_-$ is a symbol of order $-1$. Let us temporarily write terms without the $t$ variable, since $P$ is a pseudodifferential operator only in the $x$ variables, and so the $t$ variable won't come into effect in the argument. We write
	%
	\[ P \{ u_\lambda \} (x) = \int P(x, \xi) a(y,\lambda) e^{2 \pi i [ \xi \cdot (x - y) + \lambda \phi(y) ]}\; dy\; d\xi. \]
	%
	This integral has a unique, non-degenerate stationary point when $y = x$, and when $\xi = \lambda \nabla_x \phi(x)$. Fix $C > 0$, and suppose
	%
	\[ (1/C) \leq |\nabla_x \phi(x)| \leq C \]
	%
	for all $x$ on the support of $a$. Consider a smooth function $\chi$ such that
	%
	\[ \chi(v) = 1 \quad\text{for}\ 1/2C \leq |v| \leq 2C, \]
	%
	and vanishing outside a neighborhood of this set. Write
	%
	\[ \phi(y) - \phi(x) = \nabla_x \phi(x) \cdot (y - x) + \phi_R(x,y). \]
	%
	Write
	%
	\[ P_\lambda(x,\xi) = \chi(\xi / \lambda) P(x,\xi). \]
	%
	The theory of non-stationary phase, i.e. integrating by parts sufficiently many times, can be used to show that
	%
	\begin{align*}
		& e^{- 2 \pi i \lambda \phi(x) } (P - P_\lambda) \{ u_\lambda \} (x,\lambda)\\
		& \quad\quad\quad = \lambda^d \int a(y,\lambda) (P - P_\lambda)(x,\lambda \xi) e^{2 \pi i \lambda [ \xi \cdot (x - y) + \phi_R(x,y) ]}\; dy\; d\xi
	\end{align*}
	%
	is a symbol of order $-\infty$ in the $\lambda$ variable. Thus it suffices to analyze the quantities
	%
	\begin{align*}
		e^{-2 \pi i \lambda \phi(x)} P_\lambda u_\lambda(x) &= \int P_\lambda(x, \xi) e^{2 \pi i [ (\xi - \lambda \nabla_x \phi(x)) \cdot (x - y) + \lambda \phi_R(x,y) ]} a(y, \lambda)\; dy\; d\xi\\
		&= \int P_\lambda(x, \lambda \nabla_x \phi(x) + \xi) e^{2 \pi i [ \xi \cdot (x - y) + \lambda \phi_R(x,y) ]} a(y,\lambda)\; dy\; d\xi.
	\end{align*}
	%
	Using a Taylor expansion, we can write
	%
	\[ P_\lambda(x, \lambda \nabla_x \phi(x) + \xi) = \sum_{|\alpha| < N} (\partial_\xi^\alpha P)(x, \lambda \nabla_x \phi(x)) \cdot \xi^\alpha + R_N(x,\xi,\lambda), \]
	%
	where $R_N$ vanishes of order $N$ as $\xi \to 0$. Using the remainder formula for the Taylor expansion, and the support properties of $P_\lambda$, for all multi-indices $\alpha$ we have
	%
	\[ |(\partial_\xi^\alpha R_N)(x,\xi,\lambda)| \lesssim_\alpha \lambda^{1 - |\alpha|}, \]
	%
	where the implicit constant is uniform in $\xi$ and $\lambda$, and locally uniform in $x$.
	%
	% We can write R_N as a
	% finite sum of terms of the form
	%
	% xi^a int_0^1 (1 - t)^{N-1} (partial_xi^a P_lambda)(x, lambda nabla_x phi(x) + t xi)
	%
	% where |a| = N.
	%
	% Denote the integral by I_a(x,lambda,xi). If v = lambda nabla_x phi(x), then
	%
	% (d_xi^b I_a) <<
	% 			lambda^{c - b}
	% 				int_0^1 (1 - t)^{N-1} t^b (partial_xi^{a + c} P_lambda)(x, v + t xi) dt
	%
	% where c <= b.
	%
	% For |xi| >> lambda, the integral vanishes for |t| >= 1/|xi|,
	% and so this integral is bounded by
	%
	% lambda^{c - b} |xi|^{-b-1} lambda^{1 - a - c}
	% 	= lambda^{1 - a - b} |xi|^{-b-1}
	%
	% For |xi| << lambda, the integral is
	%
	% lambda^{c - b} lambda^{1 - a - c}
	% 	= lambda^{1 - a - b}
	%
	% In general, we conclude that
	But then the stationary phase formula tells us that
	%
	\[ \int R_N(x,\xi,\lambda) e^{2 \pi i [ \xi \cdot (x - y) + \lambda \phi_R(x,y) ]} a(y,\lambda)\; dy\; d\xi \]
	%
	is a symbol of order $\mu + 1 - \lceil N/2 \rceil$ in the $\lambda$ variable. Conversely, if we let
	%
	\[ a_R(x,y,\lambda) = e^{2 \pi i \lambda \phi_R(x,y)} a(y,\lambda), \]
	%
	we calculate that
	%
	\begin{align*}
		&\int (\partial_\xi^\alpha P)(x, \lambda \nabla_x \phi(x)) \cdot \xi^\alpha e^{2 \pi i [\xi \cdot (x - y) + \lambda \phi_R(x,y)]} a(y,\lambda)\; dy\; d\xi\\
		&\quad\quad = (\partial_\xi^\alpha P)(x, \lambda \nabla_x \phi(x)) (D_y^\alpha a_R)(x,x,\lambda).
	\end{align*}
	%
	Thus we conclude that, modulo order $\mu + 1 - \lceil N/2 \rceil$ symbols in the $\lambda$ variable,
	%
	\[ e^{-2 \pi i \lambda \phi(x)} Pu_\lambda(x) = \sum_{|\alpha| < N} (\partial_\xi^\alpha P)(x, \lambda \nabla_x \phi(x)) (D^\alpha_y a_R)(x,x,\lambda). \]
	%
	The formula is very simple for $N = 1$, which allows us to work modulo symbols of order $\mu$, but we must work with $N = 3$, which is slightly more complicated, because we wish to work modulo symbols of order $\mu - 1$. For $|\alpha| \leq 1$, we have
	%
	\[ (D^\alpha_y a_R)(x,x,\lambda) = (D^\alpha_x a)(x,\lambda). \]
	% phi_R(x,y) = phi(y) - phi(x) - (nabla_x phi)(x) * (y - x)
	% (D^alpha_x phi)(x)
	% phi(y) - phi(x) = nabla_x \phi(x) \cdot (y - x) + \phi_R(x,y)
	% 
	For $|\alpha| = 2$, we have
	% phi_R(x,y)
	\begin{align*}
		(D^\alpha_y a_R)(x,x,\lambda) &= \Big( (2 \pi i \lambda) (D^\alpha_x \phi)(x) \Big) a(x,\lambda) + (D^\alpha_x a)(x,\lambda)\\
		&= \lambda (\partial^\alpha_x \phi)(x) a(x,\lambda) + (D^\alpha_x a)(x,\lambda).
	\end{align*}
	%
	Thus summing up all the terms, modulo symbols of order $\mu - 1$, we find
	%
	\begin{align*}
		e^{-2 \pi i \lambda \phi(x)} Pu_\lambda(x) &= \Bigg( \sum_{|\alpha| \leq 1} (\partial_\xi^\alpha P)(x,\lambda \nabla_x \phi(x)) (D^\alpha_x a)(x,\lambda) \Bigg)\\
		&\quad + \sum_{|\alpha| = 2} \lambda (\partial^\alpha_x \phi)(x) (\partial_\xi^\alpha P)(x, \lambda \nabla_x \phi(x)) a(x,\lambda).
	\end{align*}
	%
	The order $\mu + 1$ part of this sum is
	%
	\[ p(x, \lambda \nabla_x \phi(x)) \cdot a_\mu(x,\lambda). \]
	%
	The order $\mu$ part is
	%
	\begin{align*}
		& \Big( p_0(x, \nabla_x \phi(x)) + \sum_{|\alpha| = 2} (\partial^\alpha_x \phi)(x) (\partial_\xi^\alpha p)(x, \nabla_x \phi(x)) \Big) \cdot a_\mu(x,\lambda)\\
		&\quad + (\nabla_\xi p)(x,\lambda \nabla_x \phi(x)) \cdot (D_x a_\mu)(x,\lambda) \\
		&\quad + p(x,\lambda \nabla_x \phi(x)) \cdot a_{\mu - 1}(x,\lambda).
	\end{align*}
	%
	Putting these formulas together with the terms corresponding to $\partial_t \{ L u_\lambda \}$, calculated simply using the product rule, and then splitting apart the homogeneous parts of the sum of various orders, we obtain the required result.
\end{proof}

\end{comment}

We are now ready to prove Theorem 1.

\begin{proof}[Proof of Theorem 1]

Write
%
\[ a(x,t,\lambda) \sim \sum_{k = 0}^\infty \lambda^{-k} a_k(x,t), \]
%
where $a_k$ is a smooth function. Fix $v > 0$ to be determined later. We prove the following result by induction, which yields the required claim:
%
\begin{center}
	\emph{For each $n \geq 0$, there exists a unique choice of $a_0, \dots, a_n$ such that $b$ is a symbol of order less than $-n$, and moreover, for $0 \leq i \leq n$, $a_i$ is supported on the cone $\Sigma(x_0,r,v) = \{ (x,t) : |x - x_0| \leq r + v |t| \}$}.
\end{center}
%
Let us begin with the case $n = 0$. Plugging $a$ into the result of Lemma 2, we see that $b$ can only have order less than zero provided
%
\[ \partial_t a_0 - (\nabla_\xi p)(x, \nabla_x \phi) \cdot (\nabla_x a_0) - i s \cdot a_0 = 0. \]
%
If we consider the \emph{real} vector field
%
\[ X(x,t) = \partial_t - (\nabla_\xi p)(x,\nabla_x \phi) \cdot \nabla_x, \]
%
defined on $\RR^d \times \RR$, then the equation above becomes
%
\[ X \{ a_0 \} = i s a_0. \]
%
This is a \emph{transport equation}, which can be solved using the methods of characteristics. In particular, suppose $v$ is chosen larger than the speed of propogation for the transport equation, restricted to $B_R(x_0)$. Then, provided we choose $R > r + v \varepsilon$, then a smooth solution to the transport equation exists with initial conditions $a_0(\cdot,0)$, and is the unique such solution supported on $\Sigma(x_0,r,v)$ for $|t| \leq \varepsilon$. The function $a_0$ must agree with this solution in order for $b$ to be a symbol of order $-1$, so we conclude that $a_0$ is uniquely determined, verifying the base case of the inductive statement.

Lets now address the case $n > 0$, which is not too different. By the inductive hypothesis, in order for $b$ to be a symbol of order at most $-n$, the symbols $a_0, \dots, a_{n-1}$ are uniquely determined given their values at $t = 0$, and supported on $\Sigma(x_0,r,v)$. Define $b_n(x,t,\lambda)$ to be the order $-n$ part of
%
\[ L \left\{ e^{2 \pi i \lambda \phi(x,t)} \sum_{k < n} \lambda^{-k} a_k(x,t) \right\}. \]
%
Applying Lemma 2 again, we find that the function
%
\[ L \left\{ e^{2 \pi i \lambda \phi(x,t)} \sum_{k \geq n} \lambda^{-k} a_k(x,t) \right\} \]
%
is a symbol of order $-n$, with principal part
%
\[ X \{ a_k \} - i s a_k. \]
%
Thus $b$ is a symbol of order less than $-n$ if and only if the \emph{non-homogeneous} transport equation
%
\[ X \{ a_n \} = i s a_n + b_n \]
%
holds. Since $\{ a_0, \dots, a_k \}$ all have support on $\Sigma(x_0,r,v)$, then (because $P$ is pseudolocal), the function $b_n$ also has support on $\Sigma(x_0,r,v)$. But then the theory of transport equations implies that if $R > r + v \varepsilon$, then a unique smooth solution to this equation exists which is smooth, compactly supported on $\Sigma(x_0,r,v)$, and agrees with the values that $a_n$ must take at time $t = 0$. But this means that we have verified the inductive step of the argument, which completes the proof of the theorem.
\end{proof}

\begin{remark}
	In the theory of PDEs of Hamilton-Jacobi type, one constructs solutions $\varphi$ to the eikonal equation above by the \emph{method of characteristics}. For any hypersurface $\Sigma$ in $\RR^d$, and for any smooth choice of vectors $v(x) \in T_x \RR^d - T_x \Sigma$, for all $x \in \Sigma$, there exists a unique choice of $\varphi$ vanishing on $\Sigma$, and defined in a neighborhood of $x_0$, such that $(\nabla_x \varphi)(x) = v(x)$ for all $x \in \Sigma$. This choice has the property that if $\{ \Phi_t \}$ is the phase flow along the integral curves of the Hamiltonian equations
	%
	\[ \frac{dx}{dt} = - (\nabla_\xi p)(x,\xi) \quad \frac{d\xi}{dt} = (\nabla_x p)(x,\xi) \]
	%
	and if $(x,\xi) = \Phi_t(x_1,v(x_1))$ for some $x_1 \in \Sigma$, then
	%
	\[ (\nabla_x \varphi)(x) = \xi. \]
	%
	In Theorem 1 above, we choose $(\nabla_x \varphi)(x_0) = \xi_0$, and thus choose $u_\lambda$ to have initial conditions supported in a small neighborhood of $x_0$, and frequency localized near $\xi_0$. This means the vector field
	%
	\[ X = \partial_t - (\nabla_x \phi) \cdot \nabla_x \]
	%
	will \emph{approximately} act by moving $a$ along the integral curve $t \mapsto \Phi_t(x_0,\xi_0)$, as expected by the physical theories we discussed in the last section.
\end{remark}

\begin{comment}
provided that the initial values of $a_\mu$ are fixed, and compactly supported on $|x - x_0| \leq r$.

We have shown the existence of and uniqueness of $a_0$ in order for $b$ to be a symbol of order $-1$. We will obtain the existence and uniqueness of the remaining parts of the symbol $a$ be a recursive procedure, i.e. showing by induction that for all $n \geq 0$, there exists a unique choice of $a_{-n}$ given initial conditions such that $b$ is a symbol of order $-n-1$. To prove this, we consider the additional inductive hypothesis that $a_0, \dots, a_{1-n}$ are supported on $\Sigma(x_0,r,v)$. The hypotheses hold for $n = 0$. Assuming the hypothesis for $n = k-1$. Define
%
\[ u_{\lambda,k} = e^{2 \pi i \lambda \phi(x,t)} \left\{ \sum_{j = 0}^{k-1} a_{-j} \right\}, \]
%
By our inductive hypothesis, the function
%
\[ e^{-2 \pi i \lambda \phi(x,t)} L \{ u_{\lambda,k} \} \]
%
is a symbol of order $-k$. Let us denote the principal part of this symbol by $c_k$. Lemma 1 now applies to the quantity
%
\[ e^{-2 \pi i \lambda \phi(x,t)} L\; \Big\{ \sum_{j = k}^\infty a_{-j} \Big\} \]
%
with $\mu = -k$. We conclude that the quantity is a symbol of order $1 - k$. The choice of $\phi$ implies the principal symbol of order $1-k$ actually vanishes for $|x - x_0| \leq R$. The order $-k$ part is given by
%
\[ \partial_t a_{-k} - (\nabla_\xi p)(x, \nabla_x \phi) \cdot (\nabla_x a_{-k}) - i s a_{-k} \]
%
But this means the order $-k$ part of $b$ vanishes provided that
%
\[ \partial_t a_{-k} - (\nabla_\xi p)(x, \nabla_x \phi) \cdot (\nabla_x a_{-k}) - i s a_{-k} + c_k = 0. \]
%
We can write this as
%
\[ X \{ a_{-k} \} - i s a_{-k} + c_k = 0. \]
%
This is the same transport equation as we dealt with in the case $n = 0$, except the equation is now non-homogeneous. This causes us no worry because $c_k$ is, by the inductive hypothesis, supported on $\Sigma(x_0,r,v)$, and so the finite speed of propogation guarantees that $a_{-k}$ is uniquely determined for $|t| \leq \varepsilon$ provided it's initial conditions are supported on $|x -  x_0| \leq r$, and moreover, that this unique solution is supported on $\Sigma(x_0,r,v)$. We have thus verifies the inductive hypothesis for $n = k$, completing the inductive argument. But this is sufficient to justify the result of the Theorem. \qedhere
\end{comment}

\section{Construction of the Parametrix}

By finding asymptotic solutions to the half-wave equation in the generality above, we've essentially gotten the idea of constructing the parametrix to the half-wave equation -- the idea now is to take a general input, break it up into the superposition of wave packets that are localized in space and frequency, and then apply the asymptotic solution constructed above for each of these wave packets, which behaves better and better for wave packets oscillating at a larger and larger magnitude.

It's best to break down our solution into a \emph{continuous} superposition of wave packets rather than the usual discrete decomposition that comes up in decoupling theory. Let's review a simple approach, due to Gabor, which won't quite work for our purposes, but gives us intuition for how the continuous superposition comes about. Consider the \emph{Gabor transform}
%
\[ Gf(x_0,\xi_0) = \int f(x) \eta(x - x_0) e^{- 2 \pi i x \cdot \xi_0}\; dx, \]
%
where $\eta$ is some fixed, non-negative, function $\eta$ supported on the set $\{ |x| \leq r \}$, and with
%
\[ \int \eta(x)^2\; dx = 1. \]
%
This is a unitary transformation, with adjoint given by
%
\[ (G^*h)(x) = \iint h(x_0,\xi_0) \eta(x - x_0) e^{2 \pi i x \cdot \xi_0}\; dx_0\; d\xi_0, \]
%
so we obtain a `localized Fourier inversion formula'
%
\[ f(x) = \int Gf(x_0,\xi_0) \eta(x - x_0) e^{2 \pi i x \cdot \xi_0}\; dy\; d\xi, \]
%
which expresses $f$ as a superposition of the wave packets
%
\[ x \mapsto \eta(x - x_0) e^{2 \pi i x \cdot \xi_0}. \]
%
This approach doesn't quite work for our purposes, since our choice of asymptotic solutions to the half-wave equation leads us to try and decompose a given initial condition into wave packets of the form
%
\[ x \mapsto s(x,x_0,\xi_0) e^{2 \pi i \varphi(x,x_0, \xi_0)}. \]
%
for some function $s$, where $\varphi$ satisfies the eikonal equation $p(x,\nabla_x \varphi(x,x_0,\xi_0)) = p(x_0,\xi_0)$ as in the last few sections\footnote{Though the Gabor transform \emph{would} work if we considering the usual half-wave equation $\partial_t - i \sqrt{-\Delta}$, in which case the eikonal equation becomes $p(x,\nabla_x \varphi) = |\xi_0|$, which has solution $\varphi(x,x_0,\xi_0) = (x - x_0) \cdot \xi_0$}. But the fact that we can always choose a solution to the eikonal equation satisfying $\varphi(x,x_0,\xi_0) \approx (x - x_0) \cdot \xi_0$ intuitively shows that this family of wave packets is enough to represent wave packets of each frequency and position, and that a similar approach should work as for the Gabor transform.

The trick here is to consider an inversion formula of the form
%
\[ f(x) = \int s(x,x_0,\xi_0) e^{2 \pi i \varphi(x,x_0,\xi_0)} f(x_0)\; dx_0\; d\xi_0, \]
%
which, if held, would imply we could decompose an arbitrary function $f$ into wave packets. We claim that we can use the \emph{equivalence of phase theorem} to find a symbol $a$ of order zero such that this inversion formula holds. Indeed, suppose that for each $x_0$ and $\xi_0$ we can choose $\varphi(\cdot,x_0,\xi_0)$ to solve the eikonal equation
%
\[ p(x,\nabla_x \varphi) = p(x_0,\xi_0), \]
%
so that $\varphi(x,x_0,\xi_0) \approx (x - x_0) \cdot \xi_0$, in the sense that on $B_R(x_0)$,
%
\[ (\nabla_{\xi_0} \varphi)(x,x_0,\xi_0) = 0 \quad\text{if and only if}\quad x = x_0, \]
%
\[ (\nabla_x \varphi)(x_0,x_0,\xi_0) = \xi_0 \quad\text{and}\quad (\nabla_{x_0} \varphi)(x_0,x_0,\xi_0) = - \xi_0. \]
%
These three assumptions imply precisely that the phase on the right hand side is non-degenerate, and is associated with the Lagrangian manifold
%
\[ \Delta_{T^* \RR^d} = \Big\{ (x,x_0;\xi,\xi_0): x = x_0\ \text{and}\ \xi = \xi_0 \Big\}. \]
%
This is \emph{also} the Lagrangian manifold associated with the phase function $(x,x_0,\xi_0) \mapsto (x - x_0) \cdot \xi_0$ which is the phase defining the family of pseudodifferential operators. Since we can write the identity
%
\[ f(x) = \int e^{2 \pi i (x - x_0) \cdot \xi_0} f(x_0)\; dx_0\; d\xi_0 \]
%
as a pseudodifferential operator, the equivalence of phase function theorem thus implies that there exists a symbol $s$ of order zero such that
%
\[ f(x) = \int e^{2 \pi i (x - x_0) \cdot \xi_0} f(x_0)\; dx_0\; d\xi_0 = \int s(x,x_0,\xi_0) e^{2 \pi i \varphi(x,x_0,\xi_0)} f(x_0)\; dx_0\; d\xi_0. \]
%
The equivalence of phase function theorem does not guarantee that
%
\[ \text{supp}_{(x,x_0)}(s) \subset \{ (x,x_0) : x \in B_r(x_0) \}. \]
%
In fact, if one looks more carefully at the proof of the equivalence of phase theorem, we see that the principal symbol of $s$ is the constant function $(x,x_0,\xi_0) \mapsto 1$. But we can always replace $s$ by the function $(x,x_0,\xi_0) \mapsto \eta(x - x_0) s(x,x_0,\xi_0)$, where $\eta$ is equal to one in a neighborhood of the origin, and is supported on $|x| \leq r$, because then we are only modifying our oscillatory integral away from it's canonical relation. The cost of doing this, however, is that the equation
%
\[ f(x) = \int s(x,x_0,\xi_0) e^{2 \pi i \varphi(x,x_0,\xi_0)} f(x_0)\; dx_0\; d\xi_0 \]
%
will now only hold \emph{modulo a smoothing operator}. Since we only hope to construct a parametrix for the half-wave equation, working modulo a smoothin operator causes us no issues.

The rest of the construction is rather easy. If we set
%
\[ a(x,0,x_0,\xi_0,\lambda) = s( x, x_0, \lambda \xi_0 ), \]
%
then $a$ is a symbol of order zero in the $\lambda$ variable. Theorem 1 allows us to find asymptotic solutions
%
\[ a(x,t,x_0,\xi_0,\lambda) \]
%
to the wave equation as $\lambda \to \infty$. Our parametrix $A$ is now defined by setting
%
\begin{align*}
	Af(x,t) &= \int a \left( x, t, x_0, \frac{\xi_0}{|\xi_0|} , |\xi_0| \right) e^{2 \pi i \phi(x,t,x_0,\xi_0)} f(x_0)\; dx_0\; d\xi.
\end{align*}
%
The inversion formula we constructed in the previous paragraph implies that $A_0 - I$ is a smoothing operator, where $A_0 f(x) = Af(x,0)$. And the fact that $a$ gives high-frequency asymptotic solutions to the half-wave equation for each fixed $x_0$ and $\xi_0$ implies that $L \circ A$ is equal to
%
\[ \int b \left( x, t, x_0, \frac{\xi_0}{|\xi_0|} , |\xi_0| \right) e^{2 \pi i \phi(x,t,x_0,\xi_0)}\; d\xi_0, \]
%
where $b$ is a symbol of order $-\infty$ in $|\xi_0|$. But this is sufficient to conclude that $L \circ A$ is a smoothing operator.

These facts immediately justify that $A$ is a parametrix for the solution operator $S$ to the half-wave equation. Indeed, if we let $L \circ A$ have smooth kernel $K$, and let $A_0 - I$ have smooth kernel $K'$. Then Duhamel's principle implies that the unique solution to $L \circ A = K$ with initial conditions $A_0 = I + K'$, namely, the kernel $A$, can be written as
%
\[ A(x,t,y) = S(x,t,y) + S \circ K' + \int_0^t (e^{isP} K)(x,s,y)\; ds. \]
%
But we can simply conclude from this that
%
\[ A - S = S \circ K' + \int_0^t e^{isP} K \]
%
is a smoothing operator. Thus we have constructed a parametrix $A$ for the half-wave equation. Fantastic!

\section{Hamilton-Jacobi Theory}

We have constructed a parametrix of the form
%
\[ Af(x,t) = \int a(x,t,x_0,\xi_0) e^{2 \pi i \phi(x,t,x_0,\xi_0)} f(x_0)\; d\xi_0. \]
%
where $a$ is a symbol of order zero, and $\phi$ is a non-degenerate phase function. Thus $A$ is a Fourier integral operator of order $-1/4$. But what is it's canonical relation? To answer this, we must return back to one of our assumptions at the beginning of these notes, namely, that we can choose a function $\varphi$ which satifies the eikonal equation
%
\[ p(x_0,\xi_0) = p(x, \nabla_x \varphi(x,x_0,\xi_0)), \]
%
subject to the constraint that $\varphi(x_0,x_0,\xi_0) = 0$, that $\nabla_x \varphi(x_0,x_0,\xi_0) = \xi_0$, and that for $x$ on the hyperplane
%
\[ \Sigma(x_0,\xi_0) = \{ x : (x - x_0) \cdot \xi_0 \}, \]
%
$\nabla_x \varphi(x,x_0,\xi_0)$ is normal to $\Sigma(x_0,\xi_0)$, i.e. so that $\varphi$ vanishes on $\Sigma(x_0,\xi_0)$. The Hamilton-Jacobi theory used to prove the existence and uniqueness of solutions to this equation will give us more information about the behaviour of the phase $\varphi$, which will in turn allow us to find the canonical relation of $A$. Since $A$ is a parametrix for the solution operator $S$, the wavefront set of $S$ is equal to the wavefront set of $A$, so this will tell us where the singularities of the solutions to the half-wave equation concentrate. We will not give a detailed description of the construction, which involves a hefty amount of Lagrangian geometry. All we need know for our purposes is that $\varphi$ is unique given our assumptions, and that if $\{ \Phi_t \}$ is the flow on $T^* \RR^d$ given by the Hamiltonian equations
%
\[ \frac{dx}{dt} = - (\nabla_x p)(x,\xi) \quad \frac{d\xi}{dt} = (\nabla_\xi p)(x,\xi), \]
%
then
%
\[ (\nabla_x \varphi)(x',x_0,\xi_0) = \xi' \quad\text{whenever $\nabla_x \varphi(x,x_0,\xi_0) = \xi$}. \]
%
Applying Green's theorem and Euler's homogeneous function theorem, we thus conclude that if $(x',\xi') = \Phi_t(x,\xi)$, if $(\nabla_x \varphi)(x,x_0,\xi_0) = \xi$, and if we set $(x(s), \xi(s)) = \Phi_s(x,\xi)$, then
%
\begin{align*}
	\varphi(x',x_0,\xi_0) - \varphi(x,x_0,\xi_0) &= \int_0^t (\nabla_x \varphi)(x(s),x_0,\xi_0) \cdot \partial_t x(s)\; ds\\
	&= - \int_0^t \xi(s) \cdot (\nabla_\xi p)(x(s), \xi(s))\; ds\\
	&= - \int_0^t p(x(s), \xi(s))\; ds\\
	&= - t p(x,\xi).
\end{align*}
%
The last equality follows from the fact that $p$ is \emph{constant} on the integral curves to the Hamiltonian vector field.

\begin{comment}

So let's describe the story of Hamilton-Jacobi theory used to construct solutions to the eikonal equation. We begin with the introduction of the Hamiltonian vector field
%
\[ H = \sum_j \frac{\partial p}{\partial \xi_j} \frac{\partial}{\partial x_j} - \frac{\partial p}{\partial x_j} \frac{\partial}{\partial \xi_j} \]
%
on $T^* \RR^d$. We will call the integral curves of this Hamiltonian vector field the \emph{bicharacteristics} of $p$.

For a given $x_0$ and $\xi_0$, let $\lambda = p(x_0,\xi_0)$. Then $\lambda \neq 0$, and the ellipticity of $p$ implies that the set
%
\[ \Sigma_\lambda = \{ (x,\xi) : p(x,\xi) = \lambda \} \]
%
is a hypersurface of dimension $2d - 1$. The eikonal equation we must solve states precisely that
%
\[ (x, \nabla_x \varphi(x)) \in \Sigma_\lambda \]
%
for $|x - x_0| \leq R$, with $\nabla_x \varphi(x_0) = \xi_0$. This is an underdetermined equation, i.e. for each $x$, there is a $d-1$ dimensional family of choices of $\nabla_x \varphi(x)$ such that this equation is true, i.e. the elements of the surface $\Sigma_\lambda \cap T^*_x \RR^d$. However, if we perscribe that $\varphi(x) = 0$ whenever $(x - x_0) \cdot \xi_0 = 0$, i.e. $x$ lies on the hyperplane $x_0 + V$, where
%
\[ V = V_{\xi_0} = \{ x: x \cdot \xi_0 = 0 \}. \]
%
By specifying $\varphi$ on $x_0 + V$, we should fix the issue that the problem is underdetermined, and we would should therefore expect to get a unique solution to the equation. 

If we can prove such a solution is unique, we can then define a function $\varphi(x,x_0,\xi_0)$ consisting of all the solutions to the equation. By uniqueness, $\varphi$ is homogeneous in the $\xi_0$ variable, i.e.
%
\[ \varphi(x,x_0,\lambda \xi_0) = \lambda \varphi(x,x_0,\xi_0), \]
%
since the right hand side solves the eikonal equation by the homogeneity of $p$. This implies by Euler's homogeneity relation that
%
\[ \varphi(x,x_0,\xi_0) = \xi_0 \cdot (\nabla_{\xi_0} \varphi)(x,x_0,\xi_0). \]
%
This implies that $\nabla_{\xi_0} \varphi(x,x_0,\xi_0)$ can only vanish if $\varphi(x,x_0,\xi_0)$, i.e. if $x \in x_0 + V_{\xi_0}$. Since $\varphi(x_0, x_0, \xi) = 0$ for all $\xi$, we do have $\nabla_{\xi_0} \varphi(x_0,x_0,\xi_0) = 0$, but this is the only such value, since if $0 < |x - x_0| \leq R$,
%
\[ |\varphi(x,x_0,\xi_0 + \delta ( x - x_0 ) )| \gtrsim_{\xi_0} \delta |x - x_0|, \]
%
and so $|\nabla_{\xi_0} \varphi(x,x_0,\xi_0)| \gtrsim_{\xi_0} |x - x_0|$.

To prove that $\varphi$ is unique, we have to do some geometry. Recall the Poisson bracket
%
\[ \{ f, g \} = \sum \frac{\partial f}{\partial \xi_i} \frac{\partial g}{\partial x_i} - \frac{\partial g}{\partial \xi_i} \frac{\partial f}{\partial x_i} = \omega \left( df, dg \right), \]
%
where $\omega$ is the Lagrangian 2-form defined on $T^*(\RR^d_x \times \RR^d_\xi)$. We note that the Hamiltonian vector field $H$ defined above was defined precisely so that for any smooth function $f$,
%
\[ Hf = \{ p, f \}, \]
%
where $Hf$ denotes differentiation in the direction of the Hamiltonian vector field. The fact that $\omega$ is antisymmetric implies that $Hp = \{ p, p \} = 0$, which means precisely that \emph{$p$ is constant on its bicharacteristics}. This implies that $\Sigma_\lambda$ is a \emph{coisotropic} submanifold of $T^* \RR^d$, i.e. $(T_{(x,\xi)} \Sigma_\lambda)^\perp \subset T_{(x,\xi)} \Sigma_\lambda$ for each $(x,\xi) \in \Sigma_\lambda$, where the annihilator $\perp$ is defined with respect to the Lagrangian form.

The assumption that
%
\[ \varphi(x) = 0 \quad\text{for all $x \in x_0 + V$} \]
%
implies that for all $x \in x_0 + V$, $\nabla_x \varphi(x)$ is a multiple of $\xi_0$. But if $\varphi$ satisfies the eikonal equation, we conclude that we must actually have
%
\[ \nabla_x \varphi(x) = \frac{p(x_0,\xi_0)}{p(x,\xi_0)} \xi_0 \quad\text{for all $x \in x_0 + V$}. \]
%
Consider the $d-1$ dimensional surface
%
\[ \Pi = \left\{ \left( x, \frac{p(x_0,\xi_0)}{p(x,\xi_0)} \xi_0 \right) : x \in x_0 + V \right\} \]
%
We note that $\Pi$ is an \emph{isotropic} manifold, because the tangent space to $\Pi$ at each point is a subspace of $V \oplus \RR \xi_0$, a Lagrangian subspace of $\RR^d_x \times \RR^d_\xi$.

Now at $(x_0,\xi_0)$, the Hamiltonian vector field moves in the $x$-plane in the direction of the vector $\nabla_\xi p$. By Euler's homogeneous function theorem,
%
\[ \xi_0 \cdot (\nabla_\xi p)(x_0,\xi_0) = p(x_0,\xi_0) \neq 0, \]
%
which tells us the Hamiltonian vector field moves \emph{transverse} to $\Pi$ in a neighborhood of $x_0$. But that means that the union of all the bicharacteristics that pass through $\Pi$ is a $d$ dimensional manifold $\Lambda$. It is actually a \emph{Lagrangian manifold} contained in $\Sigma_\lambda$. Moreover, we see that $\Lambda$ is a \emph{Lagrangian section}, i.e. the projection map $\Lambda \to \RR^d_x$ is a submersion. The theory of Lagrangian sections implies the existence of $R > 0$, and a unique real-valued function $\varphi$ defined for $|x - x_0| \leq R$, such that $\varphi(x_0) = 0$, $\nabla \varphi(x_0) = \xi_0$, and such that $(x,\nabla \varphi(x)) \in \Lambda$ for $|x - x_0| \leq R$. And the fact that $\nabla_x \varphi(x)$ is a multiple of $\xi_0$ as we move along $x_0 + V$ implies from this that $\varphi$ vanishes on $x_0 + V$. Thus we've proved the existence of a solution to the eikonal equation.

How about uniqueness? If $\varphi$ is \emph{any} solution satisfying the required initial conditions, then the section $\tilde{\Lambda} = \{ (x, \nabla_x \varphi(x)) \}$ is a Lagrangian submanifold of $T^* \RR^d$, contained in $\Sigma$, and containing $\Pi$. But $\Lambda$ is the \emph{unique} such Lagrangian submanifold of $T^* \RR^d$, since any coisotropic manifold has a unique foliation into Lagrangian submanifolds. Thus $\Lambda = \tilde{\Lambda}$, and the uniqueness of the parameterization of Lagrangian sections implies that $\varphi$ is the phase function constructed in the last paragraph.

Notice how this geometry gives us a useful characterization of $\varphi$. To calculate $\nabla_x \varphi(x)$, it suffices to find the unique point on $\Lambda$ that lies above $x$, and the projection onto the $\xi$-variable will give the gradient. We will now use this property to characterize the canonical relation of the parametrix $A$.

\end{comment}

\begin{theorem}
	Let $\{ \Phi_t \}$ denote the phase flow corresponding to the Hamiltonian vector field $H$. Then the canonical relation of the parametrix $A$ is equal to
	%
	\[ \mathcal{C} = \Big\{ (x,\xi,t,\tau,x_0,\xi_0) : (x,\xi) = \Phi_{-t}(x_0,\xi_0)\ \text{and}\ \tau = p(x_0,\xi_0) \Big\}. \]
\end{theorem}

\begin{proof}
	The set $\mathcal{C}$ is a $2d$ dimensional submanifold of $T^*(\RR^d_x \times \RR^d_{x_0})$. Since the canonical relation of $A$ is also $2d$ dimensional, it suffices to show that $\mathcal{C}$ is contained in the canonical relation $A$. So fix $(x,\xi,t,\tau,x_0,\xi_0) \in \mathcal{C}$. Recall that
	%
	\[ \phi(x,t,x_0,\xi_0) = \varphi(x,x_0,\xi_0) + t p(x_0,\xi_0). \]
	%
	Thus we immediately see that
	%
	\begin{equation} \label{equationtime}
		\nabla_t \phi(x,t,x_0,\xi_0) = p(x_0,\xi_0).
	\end{equation}
	%
	Because $(x,\xi) = \Phi_{-t}(x_0,\xi_0)$, and $(\nabla_x \varphi)(x_0,x_0,\xi_0) = \xi_0$, we know that
	%
	\begin{equation} \label{equationtwotime}
		(\nabla_x \phi)(x,t,x_0,\xi_0) = (\nabla_x \varphi)(x,x_0,\xi_0) = \xi.
	\end{equation}
	%
	We note that for each tangent vector $v$ to $\Sigma(x_0,\xi_0)$, we must have
	%
	\[ (\nabla_x \varphi)(x_0, x_0 + h v, \xi_0) = c(h) \xi_0 \quad\text{for some function $c$}. \]
	%
	The gradient must be a multiple of $\xi_0$ by the initial conditions of $\varphi$ because $x_0 \in \Sigma(x_0 + hv, \xi_0)$. Plugging the value into the eikonal equation,
	%
	\begin{align*}
		c(h) p(x_0,\xi_0) &= p( x_0, \nabla_x \varphi(x_0, x_0 + hv, \xi_0) )\\
		&= p(x_0 + hv, \xi_0),
	\end{align*}
	%
	from which we conclude that $c(0) = 1$, and $c'(0) = p(x_0,\xi_0)^{-1} (\nabla_x p)(x_0,\xi_0) \cdot v$. The homogeneity of $p$ implies that for all $s$,
	%
	\[ \Phi_s( x_0, c\; \xi_0 ) = c\; \Phi_s(x_0, \xi_0). \]
	%
	Thus $\Phi_{-t}(x_0, c\; \xi_0) = (x, c \xi)$, and so we conclude that
	%
	\begin{align*}
		\varphi(x, x_0 + hv, \xi_0) &= \varphi(x_0, x_0 + hv, \xi_0) + t c(h) p( x_0, \xi_0 )\\
		&= t c(h) p(x_0, \xi_0).
	\end{align*}
	%
	Taking $h \to 0$, we conclude that
	%
	\[ \lim_{h \to 0} \frac{\varphi(x, x_0 + hv, \xi_0)}{h} = t c'(0) p(x_0,\xi_0) = t (\nabla_x p)(x_0,\xi_0) \cdot v. \]
	%
	On the other hand, by the implicit function theorem, by the implicit function theorem, for small $h$, there exists unique values $t_*(h)$, $x_*(h)$, and $\xi_*(h)$ such that $x_*(h) \in \Sigma(x_0 + h \xi_0, \xi_0)$, and
	%
	\[ (x_0 , \xi_*(h) ) = \Phi_{t_*(h)} ( x_*(h), \xi_0 ). \]
	%
	We have $t_*(0) = 0$, $\xi_*(0) = \xi_0$, and $x_*(0) = x_0$. Expanding in $h$, we have
	%
	\begin{align*}
		&(x_0, \xi_*(h))\\
		&\quad = ( x_*(h), \xi_0  ) + t_*(h) \Big( (-\nabla_x p)(x_*(h), \xi_0), (\nabla_\xi p)(x_*(h), \xi_0) \Big) + O(t_*(h)^2)\\
		&\quad = ( x_*(h), \xi_0  ) + t_*(h) \Big( (-\nabla_x p)(x_*(h), \xi_0), (\nabla_\xi p)(x_*(h), \xi_0) \Big) + O(h^2).
	\end{align*}
	%
	Thus
	%
	\begin{align*}
		x_0 &= x_*(h) - t_*(h) (\nabla_x p)(x_*(h), \xi_0) + O(h^2)\\
		&= x_0 + h x_*'(0) - t_*'(0) (\nabla_x p)(x_0, \xi_0) + O(h^2)
	\end{align*}
	%
	and
	%
	\begin{align*}
		\xi_*(h) &= \xi_0 + t_*(h) (\nabla_\xi p)(x_*(h), \xi_0) + O(h^2)\\
		&= \xi_0 + h t_*'(0) (\nabla_\xi p)(x_0, \xi_0) + O(h^2).
	\end{align*}
	% xi_0 + h xi_*'(0)
	Thus we conclude that
	%
	\[ x_*'(0) = t_*'(0) (\nabla_x p)(x_0,\xi_0) \quad\text{and}\quad \xi_*'(0) = t_*'(0) (\nabla_\xi p)(x_0,\xi_0). \]
	%
	We have $(x_*(h) - (x_0 + h \xi_0)) \cdot \xi_0 = 0$, which implies that
	%
	\[ x_*(0)' \cdot \xi_0 = |\xi_0|^2. \]
	%
	Thus
	%
	\[ t_*'(0) = \frac{|\xi_0|^2}{(\nabla_x p)(x_0,\xi_0) \cdot \xi_0} \]
	%
	and so
	%
	\[ \xi_*'(0) = |\xi_0|^2 \frac{(\nabla_\xi p)(x_0,\xi_0)}{(\nabla_x p)(x_0, \xi_0) \cdot \xi_0}. \]

	Applying the Green's theorem calculation above, we find that for any tangent vector $v$ to $\Sigma(x_0,\xi_0)$,
	%
	\begin{align*}
		\varphi(x,x_0 + v, \xi_0) = \varphi(x_0,x_0 + v, \xi_0) + t p(x_0,\xi_0).
	\end{align*}
	%
	If $v$ is tangent to $\Sigma(x_0,\xi_0)$, $\varphi(x_0,x_0 + v, \xi_0) = 0$ since $x_0 \in \Sigma(x_0 + v, \xi_0)$, and thus we find that $\varphi(x,x_0 + v,\xi_0) = t p(x_0,\xi_0)$ is independent of $v$. This means that $(\nabla_{x_0} \varphi)(x,x_0,\xi_0)$ must be a scalar multiple of $\xi_0$. To work out which scalar multiple this is, we must calculate
	%
	\[ \nabla_{x_0} \varphi(x,x_0,\xi_0) \cdot \xi_0 = \lim_{h \to 0} \frac{\varphi(x_0, x_0 + h \xi_0, \xi_0)}{h}. \]
	%
	

	Now if $(x(t),\xi(t)) = \Phi_t(x_0,\xi_0)$, then
	%
	\begin{align*}
		(x(t) - (x_0 + h \xi_0)) \cdot \xi_0 &= t[x'(0) \cdot \xi_0] - h |\xi_0|^2 + O(t^2)\\
		&= - t (\nabla_\xi p)(x_0,\xi_0) \cdot \xi_0 - h |\xi_0|^2 + O(t^2)\\
		&= - t p(x_0,\xi_0) - h |\xi_0|^2 + O(t^2). 
	\end{align*}
	%
	By the implicit function theorem, for suitably small $h$, there exists a unique value $t^*(h)$ making the above quantity zero, and moreover,
	%
	\[ t^*(h) = \frac{- |\xi_0|^2 h}{p(x_0,\xi_0)} + O(h^2). \]
	%
	If we let $x^*(h) = x(t^*(h))$, then this means that $\varphi(x^*(h), x_0 + h \xi_0, \xi_0) = 0$, so
	%
	\begin{align*}
		0 &= \varphi(x^*(h), x_0 + h \xi_0, \xi_0)\\
		&= \varphi(x_0, x_0 + h \xi_0, \xi_0) - t^*(h) p(x_0,\xi_0).
	\end{align*}
	%
	Rearranging, and taking limits as $h \to 0$, we conclude that
	%
	\begin{align*}
		(\nabla_{x_0} \varphi)(x,x_0,\xi_0) \cdot \xi_0 &= \lim_{h \to 0} \frac{\varphi(x_0, x_0 + h \xi_0, \xi_0)}{h}\\
		&= \lim_{h \to 0} \frac{t^*(h)}{h} p(x_0,\xi_0)\\
		&= - |\xi_0|^2.
	\end{align*}
	%
	Thus we conclude that
	%
	\begin{equation} \label{equationthreetime}
		(\nabla_{x_0} \varphi)(x,x_0,\xi_0) = - \xi_0.
	\end{equation}
	%
	Finally, we come to show that $\nabla_{\xi_0} \phi(x,t,x_0,\xi_0) = 0$. Set $(x(t), \xi(t)) = \Phi_t(x_0,\xi_0)$. Then $x(0) = x_0$,and
	%
	\[ \nabla_{\xi_0} \phi(x_0,t,x_0,\xi_0) = \nabla_{\xi_0} \varphi(x_0,x_0,\xi_0) = 0. \]
	%
	Let
	%
	\[ F(t) = \nabla_{\xi_0} \phi(x(t), t, x_0, \xi_0). \]
	%
	Then $F(0) = 0$, and the chain rule implies that
	%
	\[ F_j'(t) = \sum_k \left[ \frac{\partial^2 \varphi}{\partial x_k \partial \xi_0^j}(x(t), t, x_0, \xi_0) \frac{dx_k(t)}{dt} \right] + \frac{\partial p}{\partial \xi_j}(x_0,\xi_0). \]
	%
	But
	%
	\[ \frac{dx_k(t)}{dt} = - \frac{\partial p}{\partial \xi_k}( x(t), \xi(t) ). \]
	%
	Since
	%
	\[ p(x, \nabla_x \varphi(x(t) ,x_0,\xi_0)) = p(x_0, \xi_0), \]
	%
	taking derivatives on both sides in $\xi_0$ implies that for each $j$,
	%
	\[ \sum_k \frac{\partial p}{\partial \xi_k}(x(t), x_0, \xi_0) \frac{\partial^2 \varphi}{\partial \xi_0^j \partial x_k}(x(t), x_0, \xi_0) = \frac{\partial p}{\partial \xi_j}(x_0, \xi_0). \]
	%
	Substituting this into the equation for $F_j'$, together with the value of $dx(t) / dt$, we conclude that $F_j'(t) = 0$. But this implies that $F(t) = 0$ for all $t$, and so in particular, for $(x,\xi,t,\tau,x_0,\xi_0) \in \mathcal{C}$,
	%
	\begin{equation} \label{equationstationary}
		\nabla_{\xi_0} \phi(x,t,x_0,\xi_0) = 0.
	\end{equation}
	%
	Combining \eqref{equationtime}, \eqref{equationtwotime}, \eqref{equationthreetime}, and \eqref{equationstationary} implies that $\mathcal{C}$ is contained in the canonical relation, as was required to be shown.
\end{proof}

Let us consider a particular example, i.e. the Laplace-Beltrami operator
%
\[ P = \sqrt{-\Delta_g} \]
%
introduced in Section 2. The principal symbol of this equation is given by $p(x,\xi) = |\xi|_g$, the length of the covector $\xi$ with respect to the Riemannian metrix $g$. If we plug this principal symbol into the Hamilton-Jacobi theory above, we see that the bicharacteristics of the Hamiltonian vector field $H$ are precisely the integral curves of the \emph{geodesic flow} in $T^* \RR^d$. Thus we conclude that the wavefront set of the parametrix for the half-wave operator is \emph{precisely} the `geodesic light cone'
%
\[ \Big\{ (x,\xi,t,\tau,x_0,\xi_0) : (x,\xi) = \exp_{x_0}(- t \xi_0)\ \text{and}\ \tau = |\xi_0|_g \Big\}, \]
%
where $\exp_{x_0}: T^*_{x_0} \RR^d \to \RR^d$ denotes the geodesic map for cotangent inputs.

\section{Global Time Parametrix}

For simplicity, let us now work on a compact manifold $M$. All the techniques we have worked on so far are local. Thus given a first-order, classical, formally positive pseudodifferential operator $P$ on $M$, we can apply a compactness and partition of unity argument in coordinates to find a parametrix $A$ for the operator $\partial_t - iP$ on $M$ which is now \emph{global in space}, and defined for times $|t| \lesssim \varepsilon$. We will now show that given this parametrix, for any $N > 0$, we can define a parametrix on the interval $\{ |t| \leq N \}$.

The trick here is to use the semigroup property of the wave equation, together with the \emph{composition calculus} for Fourier integral operators we discussed last week. For $|t| \leq \varepsilon$, let $A_t$ be the operator such that
%
\[ Af(x,t) = A_tf(x). \]
%
Then $A_t$ is a parametrix for the half-wave propogator $e^{2 \pi i t P}$, and moreover, each operator is a Fourier integral operator of order zero, with canonical relation
%
\[ \mathcal{C}_t = \Big\{ (x,\xi,x_0,\xi_0) : (x,\xi) = \Phi_{-t}(x_0,\xi_0) \Big\}. \]
%
Since $M$ is a compact manifold, the Hamiltionian flows $\{ \Phi_t \}$ are globally defined in time, and so we can actually define $\mathcal{C}_t$ for each $t \in \RR$. Each such set will be a Lagrangian submanifold of $T^* M \times T^* M$, and moreover, will satisfy the relation
%
\[ \mathcal{C}_{t_1} \circ \dots \circ \mathcal{C}_{t_N} = \mathcal{C}_{t_1 + \dots + t_N}. \]
%
One can immediately check that these Lagrangian manifolds are compatible, i.e. the required transversality conditions hold, so that  we are able to apply the composition calculus for Fourier integral operators.

Since $M$ is compact, solutions to the wave equation are actually globally defined in time. The family of propogators $\{ e^{2 \pi i t P} \}$ satisfy the semigroup property
%
\[ e^{2 \pi i t_1 P} \circ \dots \circ e^{2 \pi i t_N P} = e^{2 \pi i t P}, \]
%
if $t = t_1 + \dots + t_N$. Now under this assumption, the composition calculus for Fourier integral operators thus allows us to conclude that
%
\[ A_{t_1} \circ \dots \circ A_{t_N} \]
%
is a Fourier integral operator of order zero, with canonical relation $\mathcal{C}_t$. Moreover, one can do some algebraic calculations to show that
%
\[ A_{t_1} \circ \dots \circ A_{t_N} - e^{2 \pi i t P} \]
%
is a smoothing operator.

We can use this trick to extend our parametrix. Given our parametrix $A$, defined on $\{ |t| \leq \varepsilon \}$, let's define a parametrix $A'$ which works for times $\{ |t| \leq 2 \varepsilon \}$, by setting
%
\[ A' f(x,t) = \begin{cases} A_t f(x) &: |t| \leq \varepsilon \\ (A_{t - \varepsilon} \circ A_1)f(x) &: \varepsilon \leq t \leq 2\varepsilon \\ (A_{t + \varepsilon} \circ A_{-1})f(x) &: -2\varepsilon \leq t \leq - \varepsilon. \end{cases} \]
%
It is simple to check that $A' - S$ is a smoothing operator, by using the semigroup property of the half-wave equation. So we've extended the parametrix a distance $\varepsilon$ more than we started with!

But now you should see how the trick to get an arbitrarily large interval works. Given that we want to construct a parametrix for times $|t| \leq N$, we just need to iterate the argument we have just given $O(N / \varepsilon)$ times. The resulting operator will be a composition of many compatible Fourier integral operators, and will thus also be a Fourier integral operator of order zero. But the symbol of such an operator becomes very difficult to understand as $N \to \infty$; it is obtained by an $O(N / \varepsilon)$-fold iterated oscillatory integral, and these things can get hairy as $N \to \infty$. We should expect this, because, if $M$ is a strange manifold, or $P$ is a strange pseudodifferential operator, for large times the Hamiltonian equation can loop around in very strange ways that we might expect might make the operator fairly pathological to deal with from the Hamiltonian approximation methods we have given in these notes. But for a fixed $N$, one can at least conclude from the calculus that the symbol is of order zero.





%Let's return to the sources of inspiration for the parametrix construction. In the study of quantum physics, scientists were lead to the study of the equation
%
%\[ \partial_t = i P(x,D), \]
%
%where $P$ described the motion of a classical system. It is a heuristic in quantum physics that \emph{high energy} wave packets behave like their classical counterpart (for what is classical physics but the behaviour of objects at a scale much larger than the quantum realm, such that any significant motion carries with it an absurdly high amount of energy from the quantum perspective). The canonical relation we specified above gives a mathematical precise formulation of this type of heuristic; for high frequency data, the solution to the half-wave equation propogates along the Hamiltonian vector fields corresponding to the principal symbol of the symbol $P$.




\begin{comment}

\section{Leftovers}




The formula we have given above is now sufficient to obtain a family of asymptotic solutions to the wave equation which works for a large family of initial conditions localized in phase space. 
%
%Let us suppose that
%
%\[ \phi(x,0) = (x - x_0) \cdot \xi_0 \]
%
%for some fixed $x_0$ and $\xi_0$, and that
%
%\[ a(x,0,\lambda) = \chi(x - x_0) \]
%
%for some smooth, compactly supported function $\chi$.
We have
%
\[ L \{ u_\lambda \} = e^{2 \pi i \lambda \phi(x,t)} b(x,t,\lambda). \]
%
If we assume that $a_\mu(x_0,0,\lambda) \neq 0$, then in order for $b$ to be a symbol of order $\mu$, the phase $\phi$ must solve the equation
%
\[ \partial_t \phi = p \left( x, \frac{\nabla_x \phi}{2 \pi} \right). \]
%
To guess how to solve this equation, let's separate variables, and suppose $\phi$ is of the form
%
\[ \phi(x,t) = \varphi(x) + a t, \]
%
for some constant $a$, and some function $\varphi$, then the equation above becomes the \emph{eikonal equation}
%
\[ a = p \left( x, \frac{\nabla_x \varphi}{2 \pi} \right). \]
%
Let us prescribe the initial conditions that $\nabla_x \varphi(x_0) = \xi_0$, that $\varphi$ vanishes on the hyperplane passing through $x_0$, and orthogonal to $\xi_0$. Then we see that
%
\[ a = p \left( x_0, \frac{\xi_0}{2 \pi} \right), \]
%
and the equation becomes
%
\[ p \left( x_0, \xi_0 \right) = p \left( x, \nabla_x \varphi \right). \]
%
TODO: Verify derivative conditions. The theory of Hamilton-Jacobi equations shows that, given the initial conditions above, there is a \emph{unique} function $\varphi$ solving this equation in a small neighborhood of the point $x_0$. We'll take $\phi$ to be this unique function in what follows, i.e. we set
%
\[ \phi(x,t) = \varphi(x) + \frac{p( x_0, \xi_0)}{2 \pi} \cdot t \]
%
where $\varphi$ is specified as above. This is sufficient to conclude that $b$ is a symbol of order $\mu$ in a neighborhood of $x_0$ and for small times, regardless of the choice of symbol $a$.

Stone's Theorem tells us that we should only expect a well-defined theory of operators of the form $e^{itP}$ if the operator $P$ is a self-adjoint operator. Conveniently, this condition also implies that the principal symbol $p$ is real-valued, which implies that the zeroes of the symbol of the operator $L$ in the variable $\tau$ are purely imaginary, so we can apply the theory of hyperbolic equations to such operators. In order to have a non-singular theory of characteristics for the hyperbolic equation, it is necessary to assume that $p$ is a non-vanishing symbol, i.e., that $P$ is an elliptic pseudodifferential operator. So these are our assumptions, we study an equation of the form $L$, where $P$ is a self-adjoint, elliptic pseudodifferential operator of order one. Here are some important examples to keep in mind:
%
\begin{itemize}
	\item If $M$ is a Riemannian manifold, then we can define the Laplace-Beltrami operator
	%
	\[ \Delta_g f = |g|^{-1/2} \sum_i \partial_i ( |g|^{1/2} g^{ij} \partial_ j f ). \]
	%
	The operator $\sqrt{-\Delta_g}$ is a self-adjoint, elliptic pseudodifferential operator of order one with principal symbol $|\xi|_g$.

	\item s 
\end{itemize}

\end{comment}

\begin{thebibliography}{9}

\bibitem{FeynmanLectures} R. Feynman (1963). \emph{The Feynman Lectures in Physics Vol. 1 (Ch. 26-32).}

\bibitem{HormanderSpectrum} L. H\"{o}rmander (1968). \emph{The Spectral Function of an Elliptic Operator}.

\bibitem{HormanderFirst} L. H\"{o}rmander (1971). \emph{Fourier Integral Operators I}.

\bibitem{HormanderSecond} J.J. Duistermaat and L. H\"{o}rmander (1972). \emph{Fourier Integral operators II}.

\bibitem{DuistermaatHuygen} J.J. Duistermaat (1992). \emph{Huygen's Principle for Linear Partial Differential Equations}.

\bibitem{Zworski} M. Zworski (2012). \emph{Semiclassical Analysis}.

\bibitem{SoggeHangzhou} C.D. Sogge (2014). \emph{Hangzhou Lectures on Eigenfunctions of the Laplacian}.

\bibitem{SoggeClassical} C.D. Sogge (2017) \emph{Fourier Integrals in Classical Analysis}, 2nd ed.

\end{thebibliography}

\end{document}