% Compile with xelatex
% UTF-8 encoding
\documentclass{article}
\usepackage{xeCJK}
\setCJKmainfont{SimSun}

\begin{document}

The Chinese language has many dialects, which are mutually uncomprehendable. We will be learning modern standard Chinese, also known as Mandarin Chinese (though Mandarin Chinese also has many subdialects). It is generally recognized as the `lingua franca' of China.

The first words everyone should learn in chinese are
%
\begin{center}
    n\'{i} (你) -- you\ \ \ \ \ h\v{a}o (好) -- good
\end{center}
%
Put together, we get the basic greeting of Chinese: n\v{i} h\^{a}o, literally meaning ``you good''. Notice the structure of these words. Each syllable in Chinese corresponds to a single character, and is formed from three parts. An initial part, a final part, and a tone. In this case, we have the initial parts `n' and `h', which are pronounced just like there english counterparts. The finals `i' is pronounced like the english `ee', like the i in pizza. The `ao' is pronunced like you've just stubbed your toe -- ow! The carat \'{a} indicates that we should pronunce the word raising our voices from low to high. \v{a} indcates that we should keep our voice low and bounce our voice.

\section{The Simple Finals}

There are six simple finals, and other finals which can be obtained from combining this simple finals. In order, they are a,o,e,i,u,\"{u}.

\begin{itemize}
    \item a as is pronounced as in English, like `fa la la'. The tongue remains in a natural relaxed position in the mouth.
    \item o is prounounced as in English, like `log'. Round you lips.
    \item e has no counterpart in English. Start by beginning to pronounce o, then unround your lips and spread them apart as if you were smiling, somewhat like the the `ir' in `bird'.
    \item i is proununced like the enlish `ee' in `sheep'. However, the tongue is raised slightly higher.
    \item u is pronounced like the `oo' in `moon'.
    \item \"{u} is proununced like the German letter. First position your mouth to proununce i, then then move your mouth to the `oo' position.
\end{itemize}

\section{Tones}

There are four tone indicators, which indicate syllables to be annunciated, along with a fifth neutral tone. They are indicated by a symbol, called a tone mark, above a letter in the syllable. If there is more than one letter, the first letter in the order a,o,e,i,u,\"{u} gets the tone mark. In order to learn thest ones, partition the lowest tone you can speak comfortable to the highest tone uniformly in the range 1 to 5.

\begin{itemize}
    \item The first tone is \-{a} is the easiest. You just say the word as a constant 5 pitch. The sound that results normally sounds sing songy.
    \item The second tone is \'{a}, which we've already encountered in the word n\'{i}. You start at at 1, then raise to a 5. Try not to remain at any tone while you pronunce your syllable -- it should sound as if you're asking a question.
    \item The third tone is \v{a}, which officially starts at 2, goes down to 1, and then back up to 4. It feels more bouncy the the fourth tone, which is a straight drop.
    \item The fourth tone is \`{a}, which is the opposite of 2nd tone. You start at the top, then quickly drop your voice to the lowest tone, without remaining at any particular pitch. It should sound like you are being direct, like answering a question.
    \item The neutral tone is proununced briefly and softly, like an unannunciated syllable in English.
\end{itemize}

\section{Simple Finals}

There are twenty-one initial consonans in Chinese. They are divided into six groups, based on the finals they are matched with when they are pronunced as characters (just like we say a as ah in English).

\begin{enumerate}
    \item b,p,m,f.
    \begin{itemize}
        \item The b is proununced as in English, but the vocal cords do not vibrate, so it sounds more similar to a p.
        \item The p is the aspirated equivalent of b, pronunced exactly as in english. If you put a tissue to your face, then the tissue should move when you pronununce p, but not b.
        \item m and f are pronounced as in English.
    \end{itemize}

    \item d,t,n,l
    \begin{itemize}
        \item These letters are pronounced very similar to their english counterparts, but the tongue is slightly more back than in English. In english, our tongue touches our teeth. In Chinese, we want our tongue to touch our pallete instead.
    \end{itemize}

    \item g,k,h


    \item j,q,x
    \item z,c,s
    \item zh,ch,sh,r
\end{enumerate}

You should be able to proununce n\'{i} h\v{a}o with ease. Here's some more phrases
%
\begin{center}
    xi\`{e} xie (谢谢) -- `thank' `thank'.

    b\'{u} y\`{o}ng xi\`{e} (不 用 谢) -- `no' `need' `thank'.

    b\`{a} ba (爸爸) -- father

    m\`{a} ma (妈妈) -- mother

    d\`{i} di (弟弟) -- younger brother

    g\-{e} ge (哥哥) -- older brother
\end{center}

\section{The Chinese Writing System}

Unlike English, each syllable in Chinese has a certain meaning, and a unique character. Characters are formed, traditionally at least, in six ways.

\begin{itemize}
    \item Pictographs: These characters are drawn directly to describe what they denote (though they may have morphed and been stylized beyond recognition sometimes). Examples include:
    \begin{itemize}
        \item r\'{e}n (人) means person.
        \item z\v{i} () means baby.
        \item n\"{u} () means woman.
        \item r\`{i} () means sun.
    \end{itemize}
\end{itemize}

\section{Chinese Numerals}

A great set of words to begin learning in Chinese are the numerals.

\begin{itemize}
    \item[(1)] y\-{i} (一) - The character is just a single slash.
    \item[(2)] \`{e}r (二) - Two slashes.
    \item[(3)] s\-{a}n (三) - Three slashes.
    \item[(4)] s\`{i} (四) - A picture of a nose.
    \item[(5)] w\v{u} (五) - 5 slashes?
    \item[(6)] li\`{u} (六) - 
    \item[(7)] q\-{i} (七) - An upsidedown seven?
    \item[(8)] b\-{a} (八)
    \item[(9)] ji\v{u} (九)
    \item[(10)] sh\'{i} (十)
\end{itemize}

\section{Common Classroom Expressions}



\end{document}