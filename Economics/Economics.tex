\documentclass[12pt]{report}

\usepackage{amsmath}
\usepackage{amssymb}
\usepackage{amsthm}
\usepackage{amsopn}
\usepackage{kpfonts}
\usepackage{graphicx}
\usepackage{kbordermatrix}
\usepackage{tikz}
\usetikzlibrary{arrows, petri, topaths}%
\usepackage{tkz-berge}
\usepackage{multicol}

\usepackage{framed}
\usepackage{mathtools}
\usepackage{float}
\usepackage{subfig}
% \usepackage{cmbright}

\theoremstyle{plain}
\newtheorem{theorem}{Theorem}[chapter]
\newtheorem{lemma}[theorem]{Lemma}
\newtheorem{corollary}[theorem]{Corollary}
\newtheorem{prop}[theorem]{Proposition}
\newtheorem{exercise}{Exercise}[chapter]

\newtheorem*{example}{Example}
\newtheorem*{proof*}{Proof}

\theoremstyle{definition}
\newtheorem*{defi}{Definition}
\newenvironment{definition}
    {\begin{samepage}\begin{framed}\begin{defi}}
    {\end{defi}\end{framed}\end{samepage}}





\usepackage{hyperref} 
\hypersetup{
    colorlinks = true,
    linkcolor = black,
}

\makeatletter
\renewcommand*\env@matrix[1][*\c@MaxMatrixCols c]{%
  \hskip -\arraycolsep
  \let\@ifnextchar\new@ifnextchar
  \array{#1}}
\makeatother

\renewcommand*\contentsname{\hfill Table Of Contents \hfill}

\newcommand{\optionalsection}[1]{\section[* #1]{(Important) #1}}
\newcommand{\deriv}[3]{\left. \frac{\partial #1}{\partial #2} \right|_{#3}}

\title{Economics}
\author{Jacob Denson}

\begin{document}

\pagenumbering{gobble}
\maketitle
\tableofcontents
\pagenumbering{arabic}



\chapter{Gross Domestic Product}

The \emph{gross domestic product} (GDP) is a measure of the economic activity within a particular region, as measured against the market value of products produced within that region. It has been a major measure since the 1960s. Prior to this, the major measurement was \emph{gross national product} (GNP), which measures the market value of products produced by people living in a certain region, but via products that may have been produced outside of that region.

The interest shifted to GDP because policy makers felt they could manipulate GDP more directly than GNP. Such measures of economic activity became important in the late 17th century in Britain, with the main importance being measuring the amount of taxes a king could raise to fight wars with France. After the great depression, economists such as Simon Kuznets and Colin Clark began using what we would recognize as the modern measure of GDP was developed to give governments a measure of how badly the economy in their country was doing.

\end{document}