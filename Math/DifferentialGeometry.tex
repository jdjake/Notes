\documentclass[12pt]{report}

\usepackage{amsmath}
\usepackage{amssymb}
\usepackage{amsthm}
\usepackage{amsopn}
\usepackage{kpfonts}
\usepackage{graphicx}
\usepackage{kbordermatrix}
\usepackage{tikz}
\usetikzlibrary{arrows, petri, topaths}%
\usepackage{tkz-berge}
\usepackage{multicol}

\usepackage{framed}
\usepackage{mathtools}
\usepackage{float}
\usepackage{subfig}
% \usepackage{cmbright}

\theoremstyle{plain}
\newtheorem{theorem}{Theorem}[chapter]
\newtheorem{lemma}[theorem]{Lemma}
\newtheorem{corollary}[theorem]{Corollary}
\newtheorem{prop}[theorem]{Proposition}
\newtheorem{exercise}{Exercise}[chapter]

\newtheorem*{example}{Example}
\newtheorem*{proof*}{Proof}

\theoremstyle{definition}
\newtheorem*{defi}{Definition}
\newenvironment{definition}
    {\begin{samepage}\begin{framed}\begin{defi}}
    {\end{defi}\end{framed}\end{samepage}}





\usepackage{hyperref} 
\hypersetup{
    colorlinks = true,
    linkcolor = black,
}

\makeatletter
\renewcommand*\env@matrix[1][*\c@MaxMatrixCols c]{%
  \hskip -\arraycolsep
  \let\@ifnextchar\new@ifnextchar
  \array{#1}}
\makeatother

\renewcommand*\contentsname{\hfill Table Of Contents \hfill}

\newcommand{\optionalsection}[1]{\section[* #1]{(Important) #1}}
\newcommand{\deriv}[3]{\left. \frac{\partial #1}{\partial #2} \right|_{#3}}

\title{Differential Geometry}
\author{Jacob Denson}

\begin{document}

\pagenumbering{gobble}
\maketitle
\tableofcontents
\pagenumbering{arabic}

\chapter{Topological Considerations}

\section{Manifolds}

Topology attempts to describe the properties of space invariant under actions which stretch and squash continuously. Differential geometry extends this description to spatial properties constant when space is stretched and squashed, but not `bent' in some form. Four centuries of calculus have established differentiability in the nice cartesian spaces $\mathbf{R}^n$. A basic environment to extend the notions of `bentness' should then be to consider spaces which are locally similar to $\mathbf{R}^n$. These are the topological manifolds.

\begin{definition}
    A {\bf manifold} is a Hausdorff space which is {\it locally euclidean}. In detail, every point $p$ on the manifold is such that
    %
    \begin{enumerate}
        \item[(1)] There exists a neighbourhood $U$ of $p$, and a non-negative integer $n$ such that $U$ is homeomorphic to $\mathbf{R}^n$. We take $\mathbf{R}^0 = \{0\}$.
    \end{enumerate}
\end{definition}

Non-Hausdorff manifolds are far and few between, and occur naturally only when constructing unnatural paradoxes, so we enforce the Hausdorff property on our first bat with manifold theory.

\begin{example}
    $\mathbf{R}^n$ is a manifold. Any ball in Euclidean space is also a manifold; in these examples, we may simply take the entire space as the neighbourhood $U$ in (1), since a ball in $\mathbf{R}^n$ is homeomorphic to $\mathbf{R}^n$.
\end{example}

The above example is easily extended to show that any topological space homeomorphic to a manifold is also a manifold! This is a bad omen, for we wish to discuss properties invariant under differentiability, which are not invariant under homeomorphism. To do this, we will need additional structure, which we will address in the next chapter.

\begin{example}[The Circle]
    Consider the circle $S^1 = \{ x \in \mathbf{R}^2 : \|x\| = 1 \}$. For any proper subset $U$ of $S^1$, define an {\bf angle function} $\theta:U \to \mathbf{R}$ to be a continuous function such that $e^{i\theta(x)} = x$ for all $x \in U$. This restriction immediately implies $\theta$ is a homeomorphism onto its image, with inverse function $\theta^{-1}(t) = e^{it}$. Angle functions exist on any proper subset $U$ of $S^1$, and therefore cover $S^1$, which is therefore a 1-manifold.
\end{example}

The circle is different to $\mathbf{R}^n$ in the sense that we cannot put coordinates over the whole space at once, we must analyze the circle piece by piece to determine the structure on the whole space. This is the main trick to manifold theory -- we might not be able to look at the space nicely everywhere, but we can at least analyze the space locally in a pleasant manner.

\begin{example}[$n$-Spheres \& Stereoscopic Projection]
    There is a less obvious map on the circle which permits an easy generalization to higher dimensional space. We will project the open subset $S^n - \{(1,0,\dots,0)\}$ onto the hyperplane $\{-1\} \times \mathbf{R}^{n-1}$ by taking the intersection of this line with the line that passes through the point $x = (1,0, \dots, 0)$. If $y$ is a point on the sphere, then the set of points that lie on the line between $x$ and $y$ is defined by
    %
    \[ \{ x + [\lambda y + (1 - \lambda) x] : \lambda \in \mathbf{R} \} = \{ [(2 - \lambda)x + \lambda y_1, \lambda y_2, \dots, \lambda y_n] : \lambda \in \mathbf{R} \} \]
    %
    A simple calculation will verify that the intersection of this line with the hyperplane considered is designated by the mapping $f: S^1 - \{ (1,0,\dots,0) \} \to \mathbf{R}^{n-1}$,
    %
    \[ f(y_1, \dots, y_n) = \frac{2}{1 - y_1}(y_2, \dots, y_n) \]
    %
    In the same way, by computing intersections of points on the hyperplane with the sphere, we obtain a much more nasty formula for the inverse function $f^{-1}$,
    %
    \begin{align*}
        f^{-1}(y_2, \dots, y_n) &= \left(1 - \frac{8}{4 + \sum_{k = 2}^n y_k^2}, \frac{4y_2}{4 + \sum_{k = 2}^n y_k^2}, \dots, \frac{4y_n}{4 + \sum_{k = 2}^n y_k^2} \right)\\
        &= \frac{1}{4 + \sum_{k = 2}^n y_k^2} \left( \sum_{k = 2}^n y_k^2 - 4, 4y_2, \dots, 4y_n \right)
    \end{align*}
    %
    An easy computation will verify that this function truly is the inverse, and a brief glance at the formula shows it is continuous. If we project from the point $(-1,0,\dots,0)$ to $\{ 1 \} \times \mathbf{R}^{n-1}$, then the homeomorphism $g:[S^1 - \{ (-1, 0, \dots, 0) \}] \to \mathbf{R}^{n-1}$ obtained is
    %
    \[ g(x_1, \dots, x_n) = \frac{1}{1 + x_1}(x_2, \dots, x_n) \]
    %
    \[ g^{-1}(y_2, \dots, y_n) = \frac{1}{4 + \sum_{k = 2}^n y_k^2} \left( 4 - \left(\sum_{k = 2}^n y_k^2 \right), 4y_2, \dots, 4y_n \right) \]
    %
    And we have covered $S^n$ with homeomorphisms -- the space is a manifold.
\end{example}

Any open subset of $\mathbf{R}^n$ is a manifold -- around any point, we may take $U$ to be an open ball contained in the subspace. In fact, any open subset of a manifold, with the subspace topology, also satisfies the properties of being a manifold, and we call this space an {\bf open submanifold}.

\begin{example}
    Consider the set $M_n(\mathbf{R})$ of $n \times n$ matrices with entries in the real numbers. We can identify $M_n(\mathbf{R})$ with the space $\mathbf{R}^{n \times n}$, and therefore $M_n(\mathbf{R})$ is a topological space -- specifically, a topological manifold of dimension $n^2$. The determinant map $\det:M_n(\mathbf{R}) \to \mathbf{R}$ can be viewed as a polynomial in the entries of a matrix, so the function is continuous. The set $GL(n)$ of all invertible $n$ by $n$ matrices is equivalent to the set of matrices with non-zero determinant: $\det^{-1}(\mathbf{R} - \{0\})$, and hence $GL(n)$ is an open subset of $M_n(\mathbf{R})$. From our discussion above, we conclude that $GL(n)$ is precisely an open submanifold of $M_n(\mathbf{R})$.
\end{example}

Quite a few proofs about manifolds rely on a certain trick. First, we conjure forth a homeomorphism to $\mathbf{R}^n$. Then we transport nice properties of $\mathbf{R}^n$ across the homeomorphism to the manifold itself, thereby obtaining the properties in general. In fact, the general philosophy of manifold theory is that most properties of $\mathbf{R}^n$ will carry across to arbitrary spaces that look locally like the space itself -- we can perform linear algebra on spaces that are not really linear!

\begin{theorem}
    Every manifold is locally compact.
\end{theorem}
\begin{proof}
    Let $x$ be an arbitrary point in a manifold. Then there is an open neighbourhood $U$ of $x$ homeomorphic to $\mathbf{R}^n$ by a map $f:U \to \mathbf{R}^n$. Take any ball $B$ around $f(x)$, whose closure $\overline{B}$ is compact. Since compactness is topologically invariant, $f^{-1}(\overline{B})$ is a compact neighbourhood of $x$.
\end{proof}

The next problem requires more foresight on the reader, though the basic technique used is exactly the same.

\begin{theorem}
    Every manifold is locally path-connected. A connected manifold is path-connected.
\end{theorem}
\begin{proof}
    Local path-connectedness follows easily from the fact that each space $\mathbf{R}^n$ is connected. Consider a connected manifold $M$, and let $x$ be an arbitrary point. Let $P$ be the set of all points in $M$ path connected to $x$. Local path-connectedness shows $P$ is open. Suppose $y$ is a limit point of $P$. Take some neighbourhood $U$ of $y$ homeomorphic to $\mathbf{R}^n$. Then $U$ contains a point $p \in P$, which is path connected to $x$. $f(y)$ can be path connected to $f(p)$, since both points lie in Euclidean space. Then, taking the inverse image of the path, we obtain a path between  $y$ and $p$, which can be extended to connect $y$ to $x$. Since $P$ is non-empty (it obviously contains $x$), $P = M$, so all points in the manifold are connected to $x$.
\end{proof}

\begin{corollary}
    Every manifold is locally connected.
\end{corollary}

Corollary (1.3) tells us that any manifold can be split up into the disjoint sum of its connected components. It is therefore interesting to prove theorems about connected manifolds, since any manifold can be built up as a bundle of manifolds of this variety.

\begin{example}
    $GL(n)$ is a disconnected manifold, since otherwise $\det(GL(n))$ would be connected. By Corollary (1.3) we should expect to be able to identify the connected components. In order to identify components of the manifold, we shall construct paths that represent matrix operations you should be very familiar with. Let $v_1, \dots, v_n$ be arbitrary row vectors in $\mathbf{R}^n$.

    Consider adding a row to another row in a matrix,
    %
    \[ (v_1, \dots, v_p, \dots, v_q, \dots, v_n)^t \mapsto (v_1, \dots, v_p + v_q, \dots, v_q, \dots, v_n)^t \]
    %
    Every pair of matrices of this form are path connected in $GL(n)$ to its image above by the map
    %
    \[ t \mapsto (v_1, \dots, v_p + t v_q, \dots, v_q, \dots, v_n) \]
    %
    Subtracting rows is similarly a path-connected operation. Next, consider multiplying a row by a scalar $\gamma > 0$,
    %
    \[ (v_1, \dots, v_p, \dots, v_n) \mapsto (v_1, \dots, \gamma v_p, \dots, v_n) \]
    %
    A path that connects the two matrices is defined by
    %
    \[ t \mapsto \begin{pmatrix} v_1 & \dots & [1 + t(\gamma - 1)]v_p & \dots & v_n \end{pmatrix}^t \]
    %
    This path only remains in $GL(n)$ if $\gamma > 0$. We should not expect to find a path when $\gamma < 0$, since multiplying by a negative number reverses the sign of the determinant, and we know from the continuity of the determinant that the sign of the determinant separates into at least two connected components. The same reason shows we can't necessarily swap two rows. Fortunately, we don't need these operations -- we may use the path-connected elementary matrices to reduce any matrix to a canonical form. A modification of the Gauss Jordan elimination algorithm (left to the reader as a simple exercise) shows all matrices can be path-reduced to a matrix of the form
    %
    \[ \begin{pmatrix} 1 & 0 & \dots & 0 & 0 \\ 0 & 1 & & 0 & 0 \\ 0 & 0 & \ddots & 0 & 0 \\ 0 & 0 &  & 1 & 0 \\ 0 & 0 & \dots & 0 & \pm 1 \end{pmatrix} \]
    %
    One matrix has determinant greater than zero, the other has determinant less than zero. Thus $GL(n)$ consists of two homeomorphic path-connected components: the matrices with determinant greater than zero, and the component with determinant less than zero.
\end{example}

\section{Euclidean Neighbourhoods are Open}

In these notes, we consider a neighbourhood as in the French school, as any subset containing an open set, regardless of whether it is open or not. Nonetheless, let $M$ be a manifold, and take a point $p$ with neighbourhood $U$ homeomorphic to $\mathbf{R}^n$, lets say, by some continuous function $f: U \to \mathbf{R}^n$. Then $U$ contains an open set $V$, and $f(V)$ is open in $\mathbf{R}^n$, so that $f(V)$ contains an open ball $W$ around $f(x)$. But then $W$ is homeomorphic to $\mathbf{R}^n$, and $f^{-1}(W)$ is a neighbourhood of $x$ open in $V$ (and therefore open in $M$) homeomorphic to $\mathbf{R}^n$. This complicated discussion stipulates that we may always choose open neighbourhoods in the definition in a manifold. Remarkably, it turns out that all neighbourhoods homeomorphic to $\mathbf{R}^n$ {\it must} be open; to prove this, we require an advanced theorem.

% Draw construction above

\begin{theorem}[Invariance of Domain]
    If $f:U \to \mathbf{R}^n$ is a continuous, injective function, where $U$ is an open subset of $\mathbf{R}^n$, then $f(U)$ is open, so that $f$ is a homeomorphism.
\end{theorem}

A domain is a connected open set, and this theorem shows that the property of being a domain is invariant under continuous, injective maps from $\mathbf{R}^n$ to itself. In multivariate calculus, the inverse function theorem shows this for differentiable mappings with non-trivial Jacobian matrices across its domain; invariance of domain stipulates that the theorem in fact holds for any such continuous map $f$ on an open domain. The theorem can be proven after a brief excursion in some basic algebraic topology (homology theory, to be {\it exact}). In an appendix to this chapter, we shall prove the theorem based on the weaker assumption of the Jordan curve theorem, which should be more intuitive theorem to assume.

\begin{lemma}
    If a point has two neighbourhoods homeomorphic to Euclidean space, the dimensions of those spaces must be the same.
\end{lemma}
\begin{proof}
    Let $U$ and $V$ be two non-disjoint neighbourhoods of a point homeomorphic to $\mathbf{R}^n$ and $\mathbf{R}^m$ by $f:U \to \mathbf{R}^n$ and $g:V \to \mathbf{R}^m$. Then $U \cap V$ is also open, and therefore we have an injective map $f \circ g^{-1}|_{g(U \cap V)}$ from the set $g(U \cap V)$ open in $\mathbf{R}^n$ to $f(U \cap V)$, open in $\mathbf{R}^m$. If $n < m$, then $\mathbf{R}^n$ is homeomorphic to the hyperplane in $\mathbf{R}^m$ defined by
    %
    \[ P = \{ (x_1, \dots, x_m): x_1 = x_2 = \dots = x_n = 0 \} \]
    %
    Obviously, no subset of $P$ is open. Nonetheless, $f(U \cap V)$, as a subset of $\mathbf{R}^n$, is homeomorphic to a subset of $P$ by an injective map $h: f(U \cap V) \to P$. The map $h \circ f \circ g^{-1}$ is an injective and continuous map from an open subset of $\mathbf{R}^m$ to the rest of $\mathbf{R}^m$. By an appeal to invariance of domain, we conclude the image of the domain is open in $\mathbf{R}^m$. This is in complete contradiction to the fact that the range is a subset of $P$, and shows that the dimensions of the two spaces must be equal.
\end{proof}

\begin{corollary}
    Euclidean spaces are not Homeomorphic: $\mathbf{R}^n \not \cong \mathbf{R}^m$ for $n \neq m$.
\end{corollary}

Because of Lemma 1.5, any point $x$ on a manifold has a unique dimension of euclidean space to which a neighbourhood is mapped. We shall speak of this unique number as the {\bf dimension of the manifold at x}. When a manifold is connected, one can show simply that the dimension across the entire space is invariant, and we may call this the {\bf dimension of the manifold}. An $n$-dimensional manifold $M$ is often denoted $M^n$.

A subtlety that often goes unspoken of is that not a disconnected manifolds might not have a well defined dimension in its entirety. If $M$ and $N$ are two different manifolds, then their disjoint union is also a manifold, even if $M$ contains points with dimension 5, and $N$ points with dimension 2364. Unavoidably, any construction of this sort results in a disconnected manifold, and we may break this construction up into its individual connected components, each of which will have a unique dimension.

\begin{theorem}
    Any subset of a manifold locally homeomorphic to Euclidean space is open in the original topology.
\end{theorem}
\begin{proof}
    Let $M$ be a manifold, and $U$ a subset homeomorphic to $\mathbf{R}^n$ by a function $f$. Let $x \in U$ be arbitrary. We have already established that there is an open neighbourhood $V$ of $x$ that is homeomorphic into $\mathbf{R}^n$ by a function $g$ (the dimension of the space must be the same by the last corollary). Since $V$ is open in $M$, $U \cap V$ is open in $U$, so $f(U \cap V)$ is open in $\mathbf{R}^n$. We obtain a one-to-one continuous function from $f(U \cap V)$ to $g(U \cap V)$ by the function $g \circ f^{-1}$. It follows by invariance of domain that $g(U \cap V)$ is open in $\mathbf{R}^n$, so $U \cap V$ is open in $V$, and, because $V$ is open in $M$, $U \cap V$ is open in $M$. In a complicated manner, we have shown that around every point in $U$ there is an open neighbourhood contained in $U$, so $U$ itself must be open.
\end{proof}

Really, this theorem is just a generalized invariance of domain for arbitrary manifolds -- since the concept of a manifold is so intertwined with Euclidean space, it is no surprise we need the theorem for $\mathbf{R}^n$ before we can prove the theorem here.

\section{Equivalence of `Nice' Properties}

Many important results in differentiable geometry require spaces with more stringent properties than those that are merely Hausdorff. At times, we will want to restrict ourselves to topological manifolds with these properties. Fortunately, most of these properties are equivalent.

\begin{theorem}
    For any manifold, the following properties are equivalent:
    %
    \begin{enumerate}
        \item Every component of the manifold is $\sigma$-Compact.
        \item Every component of the manifold is second countable.
        \item The manifold is metrizable.
        \item The manifold is paracompact (so every compact manifold is metrizable).
    \end{enumerate}
\end{theorem}

\begin{lemma}[$1) \to (2$]
    Every $\sigma$-compact, locally second countable space is globally second countable.
\end{lemma}
\begin{proof}
    Let $X$ be a space with the properties above, equal to the union $\bigcup_{i = 1}^\infty A_i$, where each $A_i$ is compact. For each $x$, there is an open neighbourhood $U_x$ with a countable base $\mathcal{C}_x$. If, for some $A_i$, we consider the set of $U_x$ for $x \in A_i$, we obtain a cover, which therefore must have a finite subcover $U_{x_1}, U_{x_2}, \dots, U_{x_n}$. Taking $\bigcup_{i = 1}^n \mathcal{C}_{x_i}$, we obtain a countable base $\mathcal{C}_i$ for all points in a neighbourhood of $A_i$. Then, taking the union $\bigcup_{i = 1}^\infty \mathcal{C}_i$, we obtain a countable base for $X$.
\end{proof}

\begin{lemma}[$2) \to (3$]
    If a manifold is second countable, then it is metrizable.
\end{lemma}
\begin{proof}
    This is a disguised form the Urysohn metrization theorem, proved in a standard course in general topology, which we unfortunately do not have time to discuss. If you do not have the background, you will have to have faith that this lemma holds. All we need show here is that a second countable manifold is regular, and this follows because every locally compact Hausdorff space is Tychonoff.
\end{proof}

\begin{lemma}[$3) \to (1$]
    Every connected, locally compact metrizable space is $\sigma$-compact.
\end{lemma}
\begin{proof}
    Consider any connected, locally compact metric space $(X,d)$. For each $x$ in $X$, let
    %
    \[ r(x) = \frac{\sup \{ r \in \mathbf{R} : \overline{B}_r(x)\ \text{is compact} \}}{2} \]
    %
    Since $X$ is locally compact, this function is well defined for all $x$. If $r(x) = \infty$ for any $x$, then $\{ \overline{B}_n(x) : n \in \mathbf{Z} \}$ is a countable cover of the space with compact sets. Otherwise, $r(x)$ is finite for every $x$. Suppose that $d(x,y) + r' < r(x)$. By the triangle inequality, this tells us that $\overline{B}_{r'}(y)$ is a closed subset of $\overline{B}_{r(x)}(x)$, which is hence compact. This shows that, when $d(x,y) < r(x)$,
    %
    \begin{equation} r(y) \geq r(x) - d(x,y) \end{equation}
    %
    Put more succinctly, (1.1) tells us that the function $r:X \to \mathbf{R}$ is continuous:
    %
    \begin{equation} |r(x) - r(y)| < d(x,y) \end{equation}
    %
    This has an important corollary. Consider a compact set $A$, and let
    %
    \begin{equation} A' = \bigcup_{x \in A} \overline{B}_{r(x)}(x) \end{equation}
    %
    We claim that $A'$ is also compact. Consider some sequence $x_1, x_2, \dots$. To show $A'$ is compact, we need only show that some subsequence converges. Consider some sequence $a_1, a_2, \dots$ such that $x_k \in B_{r(a_k)}(a_k)$. Since this sequence is part of the compact set $A$, some subsequence converges to a point $a$. For simplicity, assume the sequence itself converges. By equation (1.2), when $d(a_i, a) < \varepsilon$
    %
    we have $r(a_i) < r(a) + \varepsilon$. Thus eventually, when $d(a_i,a) < r(a)/3$,
    %
    \begin{equation} d(a,x_i) \leq d(a,a_i) + d(a_i,x_i) < r(a)/3 + r(a_i) < r(a)/3 + r(a)/3 = 2r(a)/3 \end{equation}
    %
    Since we chose $r(a)$ to be half the supremum of compact sets, the sequence $x_k$ will eventually end up in the compact ball $B_{3r(a)/4}(a)$, and hence will converge, which shows exactly that $A'$ is compact.

    If $A$ is a compact set, we will let $A'$ be the compact set constructed above. Let $A_0$ consist of an arbitrary point $x_0$ is $X$, and inductively, define $A_{k+1} = A_k'$, and $A = \bigcup_{i = 0}^\infty A_k$. Then $A$ is the union of countably many compact sets. What's more, $A$ is a non-empty clopen set, and hence is equal to $X$. The fact that $A$ is open is trivial, since if $x$ is in $A_k$ for some $k$, then the ball $B_{r(x)}(x)$ is contained in $A_{k+1}$. If $x$ is a limit point of $A$, then there is some sequence $x_1, x_2, \dots$ in $A$ which converges to $x$, so $r(x_i) \to r(x)$. If $|r(x_i) - r(x)| < \varepsilon$, and also $d(x_i,x) < r(x) - \varepsilon$, then $x$ is contained in $B_{r(x_i)}(x_i)$, and hence if $x_i$ is in $A_k$, then $x$ is in $A_{k+1}$. Thus $X = A = \bigcup A_k$ and our space is $\sigma$-compact.
\end{proof}

\begin{lemma}[$4) \to (1$]
    A connected, locally compact, paracompact space is $\sigma$ compact.
\end{lemma}
\begin{proof}
    Consider a cover $\mathcal{C}$ of a space $X$ of the above variety, where each open set in the cover has compact closure. By the assumption of paracompactness, we may assume the cover is locally finite. Now let $x \in X$ be an arbitrary point. Then $x$ intersects finitely many elements of $\mathcal{C}$, which we may label $U_1, U_2, \dots, U_{n_1}$. Then $U_1 = \overline{U_{1,1}} \cup \overline{U_{1,2}} \cup \dots \cup \overline{U_{{1,n_1}}}$ intersects only finitely many more of $\mathcal{C}$, since the set is compact, and we add finitely more open sets $U_{2,1}, \dots, U_{2,n_2}$, obtaining a set $U_2 = U_1 \cup \overline{U_{2,1}} \cup \dots \cup \overline{U_{2,n_2}}$ Doing this inductively, we obtain a sequence of compact neighbourhoods. We claim the union $\bigcup U_i$ is all of $X$. Openness follows since if $y \in U_k$, then $y$ is in an open set intersected in $U_{k+1}$. If $y$ is a limit point of $\bigcup U_i$, then $y$ is in some $C \in \mathcal{C}$, and $C$ intersects some $U_k$ (otherwise $y$ cannot be in the limit point). Then $y \in C \subset U_{k+1}$ is contained in the next iteration, so $U$ is closed. We conclude $X$ is $\sigma$ compact.
\end{proof}

\begin{lemma}[$1) \to (4$]
    A $\sigma$ compact, locally compact Hausdorff space is paracompact.
\end{lemma}
\begin{proof}
    Let $X$ be a $\sigma$ compact space, and consider a cover of compact sets $C_1, C_2, \dots$. Since $C_1$ is compact, it is contained in an open neighbourhood $U_1$ with compact closure (take a cover of open sets with compact closure, then take a finite subcover). Similarily, $C_2 \cup \overline{U_1}$ is contained in an open neighbourhood $U_2$ with compact closure. In total, we obtain a chain of open sets $U_1 \subset U_2 \subset \dots$, each with compact closure, and which cover the entire space.

    Now let $\mathcal{U}$ be an arbitrary open cover of $X$. Each $V_k = U_{k+2} - \overline{U_k}$ is open, and its closure $\overline{V_k}$ is a closed subset of compact space, hence compact. Since $\mathcal{U}$ covers $\overline{V_k}$, it has a finite subcover $U_1, \dots, U_n$, and we let
    %
    \[ \mathcal{V}_1 = (U_1 \cap V_1), (U_2 \cap V_1), \dots, (U_n \cap V_1) \]
    %
    be a collection of refined open sets which cover $V_1$. Do the same for each $V_k$, obtaining $\mathcal{V}_2, \mathcal{V}_3, \dots$, and consider $\mathcal{V} = \bigcup \mathcal{V}_i$. Surely this is a cover of $X$, and each point is contained only in $\mathcal{V}_k$ and $\mathcal{V}_{k+1}$ for some $k$, and so this cover is locally finite.
\end{proof}

Not all manifolds have the properties above. Nonetheless, we will often specialize our discussion to manifolds with these special properties to obtain deeper theorems, most particularly those which follow from the existence of partitions of unity.

\section{Manifold Constructions}

The set of manifolds form a category. It would be useful therefore to find common constructions which work in general categories. We have already encountered the coproduct in the form of the disjoint union of two spaces. Here are some more constructions.

\begin{example}[Manifold Products and the Torus]
    If $M$ and $N$ are manifolds, then the product $M \times N$ is also a manifold -- we simply take products of homeomorphisms on the space. Since $S^1$ is a 1-manifold, we obtain a 2-manifold $T = S^1 \times S^1$, the Torus. More generally, the $n$-torus $S^1 \times S^1 \times \dots \times S^1$ is an $n$-manifold.
\end{example}

Most quotient spaces of manifolds will not be manifolds. Fortunately, a few quotient spaces do turn out to be manifolds, and they are very important.

\begin{example}[The M\"{o}bius Strip]
    Consider the quotient space obtained from the product space $[0,1] \times (-1,1)$ by identifying $(0,x)$ with $(1,-x)$, for each $x \in (0,1)$. Any point in the interior of a square has a neighbourhood homeomorphic to $\mathbf{R}^2$. At $[(0,x)] = [(1,x)]$, we may choose two semicircles on each side of the product space, and in the projection, we will have a neighbourhood homeomorphic to $\mathbf{R}^2$. The twisted surface we have created is known as the M\"{o}bius strip. It only has one edge, even though it exists in three dimensional space, and if you have a paper copy at hand, surprise yourself by cutting it down the middle!
\end{example}

\begin{example}[Projective Space]
    Consider the sphere $S^2$, and consider the quotient space obtained by identifying opposite sides of the sphere: glue each point $x$ to $-x$. Any neighbourhood small enough in $S^2$ projects homeomorphically onto the space created, so the space is a 2-manifold, denoted $\mathbf{P}^2$ and known as projective space. In general $\mathbf{P}^n$ is created by identifying opposite points on $\mathbf{S}^n$. The projection map is open, since if $U$ is open, then $-U$ is open (negation is a homeomorphism), so $\pi^{-1}(\pi(U)) = U \cup -U$ is open. To obtain explicit homeomorphisms on $\mathbf{P}^n$, define a map from $S^n$ to $\mathbf{R}^{n-1}$ by
    %
    \[ f(x_1, \dots, x_n) = \frac{1}{x_i}(x_1, x_2, \dots, x_{i-1}, x_{i + 1}, \dots, x_n) \]
    %
    This map is continuous everywhere but at $x_i = 0$. For all points $p$, $f(p) = f(-p)$, so we may define the map on $\mathbf{P}^n$ instead, and this map will be continuous since the projection map is open. It even has a continuous inverse, defined by
    %
    \[ f^{-1}(x_1, \dots, x_{n-1}) = \left[ \frac{1}{\sqrt{\sum x_i^2 + 1}} \left(x_1, x_2, \dots, 1, \dots, x_n \right) \right] \]
    %
    Since our maps cover the space, $\mathbf{P}^n$ is a manifold.
\end{example}

Even though $\mathbf{P}^2$ is locally trivial, the space is very strange globally, and cannot be embedded in $\mathbf{R}^3$. Nonetheless, we see it every day -- the topology of how the eye projects space down to our two dimensional vision is modelled very accurately by the spherical construction of projective space. We don't see the really weird part of $\mathbf{P}^2$, since our eye cannot see a full sphere of vision.

\begin{example}[Gluing Surfaces]
    Let $M$ and $N$ be connected $n$-manifolds. We shall define the connected sum $M \# N$ of the two manifolds. There are two sets $B_1$ and $B_2$ in $M$ and $N$ respectively, both homeomorphic to the closed unit ball in $\mathbf{R}^n$. Then there is a homeomorphism $h:\partial B_1 \to \partial B_2$ which connects the boundaries of the two manifolds. We may define the connected sum as
    %
    \[ M \# N = (M - B_1^\circ) \cup_h (N - B_2^\circ) \]
    %
    The topological structure formed can be shown to be unique up to homeomorphism, but this is non-trivial to prove. The $n$-holed torus $T \# T \# \dots \# T \# T$ is an example of such a structure.
\end{example}

Now that we have discussed the topological parts of geometry, we can move on to describing the differentiable structure that underlies differential geometry.

\begin{exercise} Consider two copies of the real line, $\mathbf{R}_a = \mathbf{R} \times \{a\}$ and $\mathbf{R}_b = \mathbf{R} \times \{b\}$. Denote $(x,a)$ by $x_a$ and $(x, b)$ by $x_b$. Take the disjoint union of $\mathbf{R}_0$ and $\mathbf{R}_1$, and attach $x_a$ to $x_b$ for $x \neq 0$. Show that we obtain a non-Hausdorff space in the quotient, yet all points in the space satisfy property (1).
\end{exercise}

\newpage

\section{* A Non Metrizable Manifold}

Recall that a {\bf well-ordered set} is a set $X$ together with a linear ordering such that every subset has a least element. A subset $Y$ of a well-ordered segment is an {\bf initial segment} if $y \in Y$ and $x < y$ imply $x \in Y$.

\begin{definition}
    An {\bf order morphism} between two well-ordered sets $X$ and $Y$ is a map $f:X \to Y$ such that if $x < y$, $f(x) < f(y)$. A bijective order morphism is called an {\bf order isomorphism}, and all order morphisms are order isomorphisms onto their codomains. An {\bf ordinal} is an equivalence class of order isomorphic well ordered sets.
\end{definition}

It is helpful to visualize ordinals as the well-ordered set they represent, since we need no further properties of well ordered sets other than the ordering they possess. We will often (to our convenience) confuse the two. One key feature of ordinals is that they allow us to measure the size of infinite sets. It should come as no surprise then, that ordinals will allow us to construct a manifold too large to be metrizable.

The most well known ordinals are the natural numbers. 0 can be considered the equivalence class containing the empty set. 1 can be considered the equivalence class of well ordered sets consisting of a single element (which obviously must be order isomorphic). In general, the number $n$ can be considered the equivalence class of well ordered sets consisting of $n$ elements (which, less obviously, must be order isomorphic). It doesn't stop here though, for we can consider the equivalence class containing $\mathbf{N}$ of all natural numbers, which is also a well ordered set. By custom, this ordinal is denoted $\omega$. Then we may consider $\omega + 1$, the equivalence class of the well ordered set obtained by taking $\mathbf{N}$ and popping a greatest element on the end, and so on and so forth. There's many more ordinals in this magnificant menagerie, and they form a beautiful transfinite chain:

\[ 0, 1, 2, 3, \dots, \omega, \omega + 1, \dots, \omega 2, \omega 2 + 1, \dots, \omega 3, \dots, \omega^2, \dots \omega^\omega, \dots  \]

\begin{lemma}
    If $X$ and $Y$ are well ordered sets, and if for $A \subset B \subset X$ there are two order morphisms $f:A \to Y$ and $g:B \to Y$ whose ranges are initial segments of $Y$, then $g|_A = f$.
\end{lemma}
\begin{proof}
    Consider the set of all elements in $B$ that do not agree on $f$ and $g$. If this set is non-empty, there must be a least such element $b$, so either $f(b) < g(b)$, or $g(b) < f(a)$. In the first case, there must be $b'$ such that $g(b') = f(b)$ (since $g$ maps onto an initial segment). We also must have $b' < b$, and so $f(b') = g(b') = f(b)$. All order isomorphisms are injective, so we reach a contradiction. The latter case is similar, and shows by contradiction that there can be no elements that disagree on the domains of the functions.
\end{proof}

\begin{corollary}
    There is at most one map $f:X \to Y$ which maps onto an initial segment of $Y$.
\end{corollary}

\begin{lemma}
    If $X$ and $Y$ are well ordered sets, there either exists a unique order morphism from $X$ to an initial segment of $Y$, or a unique order morphism from $Y$ to an initial segment of $X$. What's more, this map is unique.
\end{lemma}
\begin{proof}
    Consider the set $A$ of all initial segments of $X$ which have order morphisms $f_A$ (which are necessarily unique) onto initial segments of $Y$. If we have a linear chain $\{A_k\}$ of such sets, we may by the last corollary take the union $\bigcup f_A$ of order morphisms to form an order morphism on $\bigcup A_k$. By Zorn's lemma, we must have a maximal initial segment $A$. If $A = X$, we are done. If $A \neq X$, and $f_A(A) = Y$, then we may invert the domain of $f_A$ to obtain an order morphism from $Y$ to $A$, and initial segment of $X$. These are all of the possibilities, since if $f_A(A) \neq Y$, we may consider the least element $y$ in $f_A(A)^c$ and $x$ in $A^c$, and extend the map $f_A$ by defining $f_A(x) = y$, contradicting the fact that $A$ is maximal.
\end{proof}

We say $X \leq Y$ if there is an order morphism from $X$ to an initial segment of $Y$. Because of the above theorem, we can visualize any ordinal as an initial segment of an ordinal of a larger size. In fact, with the above ordering, any ordinal is the equivalence class of the set of ordinals less than itself. From this, we can also see than any set of ordinals is well ordered, and that any set of ordinals is contained within an ordinal.

\begin{lemma}
    If $A$ is an initial segment which is a proper subset of a well ordered set $B$, there is no order isomorphism from $B$ to $A$.
\end{lemma}
\begin{proof}
    Let $f:A \to B$ be an order isomorphism from $A$ to $B$. Consider the smallest element $a \in A$ such that $f(a) \neq a$. There must be one such $a$, since $f$ is surjective, and there are some $b \in B$ which are not in $A$. We cannot have $f(a) < a$, since $f$ is injective, and this would imply $f(f(a)) \neq f(a)$, and $f(a)$ an element of $A$ since $A$ is an initial segment. We also cannot have $f(a) > a$, since there is $a' \in A$ such that $f(a') = a$, and since $f(a') < f(a)$, we have $a' < a$. By contradiction, there cannot be an order isomorphism $f$.
\end{proof}

If two well-ordered sets are order isomorpic, they have the same cardinality, and therefore it makes sense to discuss the cardinality of an ordinal. The well ordering theorem stipulates that any set can be well ordered. Therefore, taking the equivalence class of a well-ordering of $\mathbf{R}$, we obtain an uncountable ordinal. All countable ordinals can be considered initial segments of $X$, and we may therefore consider the set $\Omega$ of all countable ordinals.

\begin{theorem}
    $\Omega$ is uncountable.
\end{theorem}
\begin{proof}
    Suppose $\Omega$ is countable, Then $\Omega$ itself represents a countable ordinal $\alpha \in \Omega$. But $\alpha$ is order isomorphic to the set of ordinals less than $\alpha$, and so $\Omega$ is order isomorphic to a proper initial segment of itself, contradicting the above lemma.
\end{proof}

After this development, we can now release our non-metrizable manifolds.

\begin{example}[The Long Line]
    Take the set $\Omega$ of all countable ordinals. Then $\Omega$ is itself an ordinal, and we may consider the space $L = \Omega \times [0,1)$ together with the dictionary order. The order topology established forms a space, the long ray. Now take two copies of the long ray, and attach them at the smallest elements. This create a one-manifold -- the long line. Obviously, the space isn't metrizable -- it contains an uncountable discrete subset, so none of the other nice properties that we considered above hold.
\end{example}

\begin{example}[Long 2-Manifolds]
    The two-manifold $L \times S^1$ is called the long cylinder, and is also non-metrizable, and the long plane $L \times L$ is the same. A 2-manifold that is long only in one direction is the long strip $L \times \mathbf{R}$.
\end{example}

We'll encounter more unmetrizable manifolds in later chapters.

\newpage

\section{* A Proof of Invariance of Domain}

For this section, we will prove invariance of domain, relying on two unproved (but obviously true) theorems. It sure takes a lot to build up to this theorem, but the result is worth every penny.

\begin{theorem}[The Generalized Jordan Curve Theorem]
    Every subspace $X$ of $\mathbf{R}^n$ homeomorphic to $S^{n-1}$ splits $\mathbf{R}^n - X$ into two components, and $X$ is the boundary of each.
\end{theorem}

\begin{theorem}
    If a subspace $Y$ of $\mathbf{R}^n$ is homeomorphic to the unit disc $\mathbf{D}^n$, then $\mathbf{R}^n - Y$ is connected.
\end{theorem}

We'll put on the finishing touches to Invariance of Domain now. Hopefully this will give you intuition to why the theorem is true.

\begin{lemma}
    One of the components of $\mathbf{R}^n - X$ is bounded, and the other is unbounded. We call the bounded component the {\bf inside} of $X$, and the unbounded component the {\bf outside}.
\end{lemma}
\begin{proof}
    Since $X$ is homeomorphic to $S^n$, it is a compact set, and therefore contained in some ball $B$. $\mathbf{R}^n - B$ is connected, so therefore one component of $\mathbf{R}^n - X$ is contained within $B$. Since $B$ is bounded, this component is bounded. If both components are bounded, we conclude that the union of the two components plus $X$ is bounded, a contradiction. Therefore the other component is unbounded.
\end{proof}

\begin{lemma}
    If $U \subset \mathbf{R}^n$ is open, $A \subset U$ is homeomorphic to $S^n$, $f:U \to \mathbf{R}^n$ is one-to-one and continuous, and $A \cup (\text{inside of}\ A)$ is homeomorphic to $\mathbf{D}^n$, then $f(\text{inside of}\ A) = \text{inside of}\ f(A)$.
\end{lemma}
\begin{proof}
    Since $f$ is continuous, $f(\text{inside of}\ A)$ is connected, and is therefore contained either entirely within the outside of $f(A)$ or the inside of $f(A)$. The same is true of $f(\text{outside of}\ A)$. The difference is that, due to compactness, $f(A \cup (\text{inside of}\ A))$ is homeomorphic to $A \cup (\text{inside of}\ A)$, and in connection, homeomorphic to $\mathbf{D}^n$. Therefore $\mathbf{R}^n - f(A \cup \text{inside of}\ A)$ is connected. It follows that $f(\text{inside of}\ A)$ is a component of $\mathbf{R}^n - f(A)$, so it is equal to either the inside of outside of space. Since $f(\text{inside of}\ A)$ is contained within a bounded ball, we conclude that it is equal to the inside.
\end{proof}

\begin{theorem}[Invariance of Domain]
    If $f:U \to \mathbf{R}^n$ is an injective continuous function, where $U$ is an open subset of $\mathbf{R}^n$, then $f(U)$ is open, and therefore $f$ is homeomorphic onto its image.
\end{theorem}
\begin{proof}
    Let $V$ be an arbitrary open subset of $U$. We must show $f(V)$ is also open. Let $x \in V$ be arbitrary, and consider a closed ball $\overline{B}$ containing $x$, and contained in $V$. The boundary of $\overline{B}$ is homeomorphic to $S^{n-1}$, and the interior $B$ is equal to the inside of $\overline{B}$. By the lemma (2.4) above, we conclude that
    %
    \[ f(B) = \text{inside of}\ f(\partial B) \]
    %
    Since $\partial B$ is closed in $\mathbf{R}^n$, the inside is open, hence $f(B)$ is open. By an extension of this argument, we have shown the the image of any open set is open, so invariance of domain is proved.
\end{proof}

The unproved theorems we rely on here require quite advanced techniques in algebraic topology. Hopefully, the results seem intuitive enough that the theorem now should be `correct' in your mind, so we'll leave the algebra for a different set of notes, and seek other interests.








\chapter{Differentiable Structures}

\section{Differentiability}

As a topological space, we known when a map between manifolds is continuous, but when is a map differentiable? What we seek is a definition that is abstract enough to work on any manifold, yet possesses the same invariant properties of differentiable functions on $\mathbf{R}^n$. Let us be given a map $f:M \to N$ between manifolds. Given a correspondence $y = f(x)$, a reasonable inquiry would be to consider local homeomorphisms (hereafter called charts) $g:U \to \mathbf{R}^n$, where $U$ is a neighbourhood of $x$, and $h:V \to \mathbf{R}^m$, where $V$ is a neighbourhood of $y$. We obtain a map $h \circ f \circ g^{-1}$ from open sets in Euclidean space, and we may therefore consider $f$ differentiable at $x$ if $h \circ f \circ g^{-1}$ is differentiable at $g(x)$.

Unfortunately, this idea is doomed to fail, since we can hardly expect that this statement holds for all such sets of charts when it holds for one pair. For instance, if $k:W \to \mathbf{R}^m$ is any chart whose domain contains $f(x)$, then on $f^{-1}(W \cap V)  \cap U$, we locally have the equality
%
\[ k \circ f \circ g^{-1} = (k \circ h^{-1}) \circ h \circ f \circ g^{-1} \]
%
This function is differentiable if and only if $k \circ h^{-1}$ is also differentiable. Clearly we can choose some non-differentiable homeomorphism $l$ of $\mathbf{R}^n$, and then $k = l \circ h$ will result in a non-differentiable function. Again, we encounter the fact that a topological structure does not possess the specialization required to obtain the properties of differentiable functions.

If we are to stick with this definition, we either need to define differentiability in terms of the charts $g$ and $h$ used, or identify additional structure to manifolds. The latter option is clearly more elegant. Our method will be to identify charts which are `correct', and ignore unnatural constructions like $k$.

\begin{definition}
    Two homeomorphisms $x:U \to \mathbf{R}^n$ and $y:V \to \mathbf{R}^m$ are {\bf $\mathbf{C^\infty}$ related}, if either $U$ and $V$ are disjoint, or
    %
    \[ y \circ x^{-1} : x(U \cap V) \to y(U \cap V) \]
    %
    \[ x \circ y^{-1} : y(U \cap V) \to x(U \cap V) \]
    %
    are $C^\infty$ functions (diffeomorphisms, to be particular).
\end{definition}

Charts can be visualized as a way of assigning some particular choice of coordinates to a subset of a manifold, hence we use characters such as $x$ and $y$ to confuse the term with classical coordinates. One can see a chart as laying a blanket down onto a manifold. Two charts are $C^\infty$ related if, when we lay them down, they contain no creases!

The fact that manifolds do not have a particular preference for coordinates is both a help and a hindrance. On one side, it forces us to come up with elegant, coordinate free approaches to geometry. On the other end, these coordinate free approaches can also be incredibly unnatural!

\begin{definition}
    A {\bf smooth} or {\bf $\mathbf{C^\infty}$ atlas} for a manifold is a family of $C^\infty$ charts whose domains cover the entire manifold. A maximal atlas is called a {\bf smooth structure} on a manifold, and a manifold together with a smooth structure is called a {\bf smooth} or {\bf differentiable manifold}. In the literature, each map $y \circ x^{-1}$ is known as a {\bf transition map}, and an atlas for a manifold has $C^\infty$ transition maps.
\end{definition}

From now on, when we mention a chart on a differentiable manifold, we implicitly assume the chart is the member of the smooth structure of the manifold.

\begin{definition}
    Let $f:M \to N$ be a map between two manifolds.
    %
    \begin{enumerate}
        \item $f$ is {\bf differentiable} at a point $p \in M$ if it is continuous at $p$, and if for some chart $x:U \to \mathbf{R}^n$ whose domain contains $p$, and for some chart $y:V \to \mathbf{R}^m$ whose domain contains $f(p)$, the map $y \circ f \circ x^{-1}:x(f^{-1}(V) \cap U) \to \mathbf{R}^m$ is differentiable at $x(p)$.
        \item $f$ itself is {\bf differentiable} if it is differentiable at every point on its domain, or correspondingly, if $y \circ f \circ x^{-1}$ is differentiable for any two charts $x$ and $y$.
        \item More stringently, $f$ is $\mathbf{C^k}$ if each map $y \circ f \circ x^{-1}$ is $C^k$.
        \item $f$ is a {\bf diffeomorphism} if it is bijective, differentiable, and if $f^{-1}$ is differentiable.
        \item A {\bf $\mathbf{C^k}$ diffeomorphism} requires $f$ and $f^{-1}$ to be $C^k$ functions.
    \end{enumerate}
\end{definition}

It is uncomfortable to construct a maximal atlas explicitly on a manifold. Fortunately, we do not need to specify every single valid chart in our manifold.

\begin{lemma}
    Every atlas extends to a smooth structure.
\end{lemma}
\begin{proof}
Let $\mathcal{A}$ be an atlas for a manifold $M$, and consider the set $\mathcal{A}'$, which is the union of all atlases containing $\mathcal{A}$. We shall show that $\mathcal{A}'$ is also an atlas, and therefore necessarily the unique maximal one. Let $x:U \to \mathbf{R}^n$ and $y:V \to \mathbf{R}^n$ be two charts in $\mathcal{A}'$ with non-disjoint domain, containing a point $p$. Let $z:W \to \mathbf{R}^n$ be a chart in $\mathcal{A}$ containing $p$. Then, on $U \cap V \cap W$, an open set containing $p$, we have
%
\[ x \circ y^{-1} = (x \circ z^{-1}) \circ (z \circ y^{-1}) \]
%
and by assumption, each component map is $C^\infty$ on this domain, so $x \circ y^{-1}$ is smooth in a neighbourhood of $p$. The proof for $y \circ x^{-1}$ is exactly the same. Since the point $p$ was arbitrary, we conclude that $x$ and $y$ are $C^\infty$ related across their domains. $\mathcal{A}'$ is therefore an atlas, and necessarily a maximal such one.
\end{proof}

We have basically argued this corollary:

\begin{corollary}
    If $x$ is a chart defined on a differentiable manifold $M$, and is $C^\infty$ related to each map in a generating atlas $\mathcal{A}$, then $x$ is in the maximal atlas generated by $\mathcal{A}$.
\end{corollary}

\begin{example}
    Let $M$ be a manifold, and $U$ an open submanifold. Define a differentiable structure on $U$ consisting of all charts defined on $M$ whose domain is a subset of $U$. This is a maximal atlas, and satisfies the following:
    %
    \begin{enumerate}
        \item If $f: M \to N$ is differentiable, then $f|_U: U \to M$ is differentiable.
        \item The inclusion map $i:U \to M$ is differentiable.
    \end{enumerate}
\end{example}

The next few theorems are justified by the fact that $C^\infty$ related charts play nicely with one another.

\begin{lemma}
    If a map $f$ is differentiable at a point $p$ in charts $x$ and $y$, it is differentiable at $p$ for any other charts containing $p$ and $q$.
\end{lemma}
\begin{proof}
    Suppose $y \circ f \circ x^{-1}$ is differentiable at a point $x(p)$, and consider any other charts $y'$ and $x'$. Then
    %
    \[ y' \circ f \circ x'^{-1} = (y' \circ y^{-1}) \circ (y \circ f \circ x^{-1}) \circ (x \circ x'^{-1}) \]
    %
    On a smaller open neighbourhood than was considered. Nonetheless, since differentiability is a local concept, we need only prove the theorem for this map on a reduced domain. This follows since the component maps are differentiable.
\end{proof}

\begin{example}
    Consider the manifold $\mathbf{R}^n$, and define a generating atlas on $\mathbf{R}^n$ containing only the identity map $id_{\mathbf{R}^n}$. This defines a smooth structure on $\mathbf{R}^n$, which satisfies the following familiar properties:
    %
    \begin{enumerate}
        \item $x$ is a chart on $\mathbf{R}^n$ if and only if $x$ and $x^{-1}$ are $C^\infty$.
        \item A map $f:\mathbf{R}^n \to \mathbf{R}^m$ is differentiable in the sense of a manifold if and only if it is differentiable in the usual sense.
        \item A map $f:M \to \mathbf{R}^n$ is differentiable if and only if each coordinate $f_i:M \to \mathbf{R}$ is differentiable.
        \item A chart $x:U \to \mathbf{R}^n$ on an arbitrary manifold is a diffeomorphism from $U$ to $x(U)$.
    \end{enumerate}
    %
    Our definition has naturally extended calculus to arbitrary manifolds.
\end{example}

\begin{theorem}
    A map $f:M \to N$ between manifolds is a $C^\infty$ diffeomorphism iff the map from charts on $N$ to charts on $M$ defined by $x \mapsto x \circ f$ is a bijection of the atlas' of the two manifolds.
\end{theorem}
\begin{proof}
    Suppose $f$ is a diffeomorphism, and let $x$ be a chart on $N$. To verify that $x \circ f$ is a chart on $N$, we need only check it is $C^\infty$ related to any other chart in the atlas of $M$. But this is just the definition of a diffeomorphism. Conversely, if $x \circ f$ is a chart on $M$, for some map $x$, then the previous argument shows, since $f^{-1}$ is also a diffeomorphism, that $x = x \circ f \circ f^{-1}$ is a chart on $N$. The converse is trivial.
\end{proof}

\begin{example}
    Let $\mathbf{R}_1$ be the canonical smooth manifold on $\mathbf{R}$. Let $\mathbf{R}_2$ be the smooth structure on $\mathbf{R}$ generated by the map $x$, such that $x(t) = t^3$. Then $\mathbf{R}_1$ and $\mathbf{R}_2$ are diffeomorphic. Let $x:\mathbf{R}_2 \to \mathbf{R}_1$ be our diffeomorphism. It is surely bijective. Let $y$ be a chart on $\mathbf{R}_2$. We must verify that $y = z \circ x$, where $z$ is a chart on $\mathbf{R}_1$. We may show this by verifying that $y \circ x^{-1} = z$, and $x \circ y^{-1} = z^{-1}$ is $C^\infty$ on $\mathbf{R}_1$. But this was exactly why $y$ was a chart on $\mathbf{R}_2$ in the first place, hence the map is a diffeomorphism.
\end{example}

\section{Partial Derivatives and Differentiable Rank}

In calculus, when a function is differentiable, we obtained a derivative, a measure of a function's local change. On manifolds, determining an analogous object is difficult due to the coordinate invariant definition required. For now, we shall stick to structures corresponding to some particular set of coordinates. Consider a differentiable map $f$ from a manifold $M$ to the real numbers. We have no conventional coordinates to consider partial derivatives on, but consider some chart $x:U \to \mathbf{R}^n$ on $M$. We obtain a differentiable map $f \circ x^{-1}$. We define, for a point $p \in U$,
%
\[ \left. \frac{\partial f}{\partial x_k} \right|_p = D_k(f \circ x^{-1})(x(p)) \]
%
We are tracing the coordinates lines placed on $U$ by the map $x^{-1}$; literally, if $c$ is the curve defined by $c(t) = (f \circ x^{-1})(x(p) + tx_k)$, then
%
\[ c'(0) = \left.\frac{\partial f}{\partial x_k}\right|_p \]
%
Our new definition of the partial derivative satisfies the familiar chain rule.

\begin{theorem}
    If $x$ and $y$ are coordinate systems at a point $p$, and $f:M \to \mathbf{R}$ is differentiable, then
    %
    \[ \left. \frac{\partial f}{\partial x_i} \right|_p = \sum \left. \frac{\partial y_j}{\partial x_i} \right|_p \left. \frac{\partial f}{\partial y_j} \right|_p \]
\end{theorem}
\begin{proof}
    We just apply the chain rule in Euclidean space.
    %
    \begin{align*}
        \left.\frac{\partial f}{\partial x_i}\right|_p = D_i(f \circ x^{-1})(x(p)) &= D_i((f \circ y^{-1}) \circ (y \circ x^{-1}))(x(p))\\
        &= \sum D_j(f \circ y^{-1})(y(p)) D_i(y_j \circ x^{-1})(x(p))\\
        &= \sum \left.\frac{\partial f}{\partial y_j}\right|_p \left.\frac{\partial y_j}{\partial x_i}\right|_p
    \end{align*}
\end{proof}

Now we could define the full derivative of a function $f:M \to N$ at a point $p$ to be $D(y \circ f \circ x^{-1})(x(p))$, for coordinate systems $x$ and $y$. The trouble is that in this way we can only talk of properties of the derivative that are invariant under the coordinate systems chosen, since the linear operator $D$ is not invariant of the coordinate system. Later on, we will be able to come up with a universal differential operator that contains all coordinate representations of $D$.

\begin{theorem}
    The rank of the matrix $D(y \circ f \circ x^{-1})(x(p))$ is the same regardless of which coordinate systems $x$ and $y$ are chosen.
\end{theorem}
\begin{proof}
    Let $y'$ and $x'$ be two more coordinate systems around $p$ and $f(p)$.
    %
    \begin{align*}
        D(y' \circ f \circ x'^{-1})(x'(p)) &= D(y' \circ y^{-1} \circ y \circ f \circ x^{-1} \circ x \circ x'^{-1})(x'(p))\\
        &= D(y' \circ y^{-1})(y(f(p))) \circ D(y \circ f \circ x^{-1})(x(p)) 
        \\ &\ \ \ \ \circ D(x \circ x'^{-1})(x'(p))
    \end{align*}
    %
    The first and last derivatives in the last composition are invertible linear operators, so the rank is invariant.
\end{proof}

The rank of this matrix tells us the freedom of movement of the mapping around the point $p$. Using the rank will allow us to extend the inverse function theorem to arbitrary manifolds.

\begin{theorem}
    If $f:M^n \to N^m$ is rank $k$ at a point $p$, there is a coordinate system $x$ at $p$ and $y$ at $f(p)$ such that for $1 \leq i \leq k$,
    %
    \[ y_i \circ f \circ x^{-1}(a_1, \dots, a_n) = a_i \]
\end{theorem}
\begin{proof}
    Let $x$ and $y$ be arbitrary coordinate systems around $p$ and $f(p)$. By a permutation of the coordinates, we may arrange that the matrix
    %
    \[ \left( \frac{\partial y_i \circ f}{\partial x_j}(p) \right)_{i,j = 1}^k \]
    %
    is invertible. Define a coordinate system $z$ around $p$ by $z_k = y_k \circ f$, for $1 \leq k \leq n$, and $z_k = x_k$ otherwise. The matrix
    %
    \[ D(z \circ x^{-1})(p) = \begin{pmatrix} \left( \frac{\partial y_i \circ f}{\partial x_j}(p) \right) & X \\ 0 & I \end{pmatrix} \]
    %
    is invertible, hence, by the inverse function theorem, $x \circ z^{-1}$ is a diffeomorphism in a neighbourhood of $z(p)$. It follows that $z$ is a coordinate system at $p$, and
    %
    \[ y \circ f \circ z^{-1}(a_1, \dots, a_n) = (a_1, \dots, a_k, *, \dots, *) \]
\end{proof}

\begin{corollary}
    If $f$ is rank $k$ in a neighbourhood of a point $p$, then we may choose coordinate systems $x$ and $y$ such that
    %
    \[ y \circ f \circ x^{-1}(a_1, \dots, a_n) = (a_1, \dots, a_k, 0, \dots, 0) \]
\end{corollary}
\begin{proof}
    Choose $x$ and $y$ from the theorem above. Then
    %
    \[ D(y \circ f \circ x^{-1})(p) = \begin{pmatrix} I & 0 \\ X & \left( \frac{\partial y_i \circ f}{\partial x_j} \right) \end{pmatrix} \]
    %
    Since $f$ is rank $k$, the matrix in the bottom right corner must vanish in a neighbourhood of $p$. Since $p$ was arbitrary, the function is smooth across its entire domain. Therefore, $y_i \circ f \circ x^{-1}$, for $i > k$ can be viewed only as a function of the first $k$ coordinates. Define $z_i = y_i$, for $i < k$, and
    %
    \[ z_i = y_i - (y_i \circ f)(y_1 \dots y_k) \]
    %
    We have an invertible change of coordinate matrix,
    %
    \[ D(z \circ y^{-1}) = \begin{pmatrix} I & 0 \\ X & I \end{pmatrix} \]
    %
    So $z$ is a coordinate system, and
    %
    \[ z \circ f \circ x^{-1}(a_1, \dots, a_n) = (a_1, \dots, a_k, 0, \dots, 0) \]
    %
    we have constructed a coordinate system as needed.
\end{proof}

\begin{corollary}
    If $f: M^n \to N^m$ is rank $m$ at $p$, then for any coordinate system $y$, there exists a coordinate system $x$ such that
    %
    \[ y \circ f \circ x^{-1} (a_1, \dots, a_n) = (a_1, \dots, a_m) \]
\end{corollary}
\begin{proof}
    In the proof of the theorem, we need not rearrange coordinates of $y$ in the case that the matrix is rank $m$.
\end{proof}

\begin{corollary}
    If $f: M^n \to N^m$ is rank $n$ at $p$, then for any coordinate system $x$, there exists a coordinate system $y$ such that
    %
    \[ y \circ f \circ x^{-1} (a_1, \dots, a_n) = (a_1, \dots, a_n, 0, \dots, 0) \]
\end{corollary}
\begin{proof}
    If $f$ is rank $n$ at a point, it is clearly rank $n$ on a neighbourhood by the continuity of the determinant. Choose coordinate systems $u$ and $v$ such that $u \circ f \circ v^{-1}(a_1, \dots, a_n) = (a_1, \dots, a_n, 0, \dots, 0)$. Define a map $\lambda$ on $\mathbf{R}^m$ by $\lambda(a_1, \dots, a_m) = (x \circ v^{-1}(a_1, \dots, a_n), a_{n+1}, \dots, a_m)$. Then $\lambda$ is a diffeomorphism, hence $\lambda \circ y$ is a coordinate system, and
    %
    \begin{align*}
        (\lambda \circ y) \circ f \circ x^{-1} (a_1, \dots, a_n) &= \lambda \circ (y \circ f \circ v^{-1}) \circ (v \circ x^{-1}) (a_1, \dots, a_n)\\
        &= \lambda (v \circ x^{-1} (a_1 \dots a_n), 0 \dots 0)\\
        &= (a_1 \dots a_n, 0 \dots 0)
    \end{align*}
\end{proof}

A function $f:M^n \to N^m$ which has rank $n$ across its domain is called a {\bf $\mathbf{C^k}$ immersion} of $M$. The map need not be one-to-one, but by applying the maps above we may definitely conclude $f$ is locally one-to-one. At any point $f(p) \in N$, we may choose a coordinate system $(y,U)$ around $f(p)$ such that, for some open set $V$, $((y_1, \dots, y_n),V)$ is a coordinate system around $p$, and
%
\[ f(V) = \{ q \in U : y_{n+1}(q) = y_{n+2}(q) = \dots = y_m(q) = 0 \} \]
%
This follows from theorems in the previous section. Immersions are mostly well behaved, apart from the odd inconsistency. Let $g:P \to N$ be a differentiable function with $g(P) \subset f(M)$. If $f$ is globally one-to-one, we may define $g:P \to M$. A suitable question to ask is whether this function is differentiable. In most cases, the answer is yes.

\begin{example}
    Immerse the non-negative real numbers in $S^2$ via the map $f$ defined by
    %
    \[ f(x) = e^{2\pi(1 - e^{-x})i} \]
    %
    whose image is all of $S^2$. Define $g:(-1,1) \to S^2$ by $g(x) = e^{-ix}$. Then $g$ is $C^\infty$ onto $S^2$, yet is not even continuous when considered a map into the real numbers via the immersion defined above.
\end{example}

Continuity is all that can go wrong in this situation.

\begin{theorem}
    Let $f:M^n \to N^m$ be an injective immersion of $M$ in $N$, and suppose $g: P \to N$ is differentiable, and $g(P) \subset N$. If $g$ is continuous considered as a map into $M$, then $g$ is differentiable considered as a map into $M$.
\end{theorem}
\begin{proof}
    Consider an arbitrary point $p \in P$. There is a coordinate system $(y_1, \dots, y_m,U)$ around $g(p)$ such that $(y_1, \dots, y_n)$ is a coordinate system around $f^{-1}(g(p))$. Since $g$ is continuous into $M$, there is a coordinate system $x$ around $p$ which maps into $U \cap f(M)$. $y \circ g \circ x^{-1}$ is differentiable, so each $y_i \circ g \circ x^{-1}$, which we have constructed a coordinate system at $f^{-1}(g(p))$ at, is differentiable. Thus $g$ is differentiable mapping into $M$.
\end{proof}

None of this can happen if $M$ is just a subspace of $N$. In this case, we say that $M$ is embedded in $N$, and call $M$ a {\bf submanifold}. If $M$ is closed in $N$, $M$ is called a {\bf closed submanifold}.

\section{The $\mathbf{C^\infty}$ Category}

We now show that the topological category of manifolds naturally restricts to the category of differentiable manifolds.

\begin{example}[Differentiable Product]
    If $M$ and $N$ are differentiable manifolds, we may consider an atlas on $M \times N$ with the differentiable structure generated by all maps $x \times y$, where $x$ is a chart on $M$ and $y$ is a chart on $N$. From this definition, we have the property that
    %
    \begin{enumerate}
        \item[(1)] $f:M \to N$ and $g:M \to N'$ are differentiable if and only if $(f,g) : M \to (N \times N')$ is differentiable.
    \end{enumerate}
    %
    This is the unique differentiable structure on $M \times N$ which makes the projection maps onto $M$ and $N$ $C^\infty$. To see this, suppose that $M \times_2 N$ forms a different differential structure on the cartesian product of the two sets, such that the projection $\pi_1: M \times_2 N \to M$ and $\pi_2: M \times_2 N \to N$ are differentiable for some arbitrary differential structure on $M \times N$. It then follows that the identify map $(\pi_1, \pi_2) : M \times_2 N \to M \times N$ is differentiable. But this means exactly that the $C^\infty$ maps on the two structures are exactly the same.
\end{example}

\begin{example}[Differentiable Quotients and $\mathbf{P}^n$]
    If $N$ is a quotient space of a differentiable manifold $M$ whose projection $\pi:M \to N$ is open and locally injective, then we may ascribe a differentiable structure to it. We take all charts $x:U \to \mathbf{R}^n$ on $M$ such that $U$ is homeomorphic to $\pi(U)$ by $\pi$. We may then push the chart onto $N$, and all the charts placed down on $N$ will be $C^\infty$ related. As a covering, this can be extended to a maximal atlas. In fact, this is the unique structure on $N$ which causes $\pi$ to be differentiable. It allows us to consider $\mathbf{P}^n$ a differentiable manifold, taking the differentiable structure on $S^n$.
    %
    \begin{enumerate}
        \item[(1)] A function $f:N \to K$ is differentiable if and only if the function $f \circ \pi: M \to K$ is differentiable, and the two functions have the same rank at corresponding points.
    \end{enumerate}
\end{example}

\begin{example}[Differentiable Structure of the Connect Sum]
    I WILL FINISH THIS LATER.

    Consider the connected sum $M \# N$ of two connected differentiable manifolds. ...

    It is another difficult theorem to verify that any choices will be diffeomorphic.
\end{example}




\section{Defining $C^\infty$ functions: Partitions of Unity}

The use of $C^\infty$ functions relies on the fact that manifolds possess them in plenty. The following theorem gives us our first plethora. First, we detail some $C^\infty$ functions on $\mathbf{R}^n$.

\begin{enumerate}
    \item The map $f:\mathbf{R} \to \mathbf{R}$, defined by
    %
    \[
    g(t) =
    \begin{cases}
        e^{-t} & : t > 0\\
        0 & : \text{elsewhere}
    \end{cases}
    \]
    %
    is $C^\infty$. We have $0 < f(t) < 1$ on $(0,\infty)$, and $f^{(n)}(0) = 0$ for all $n$.
    \item The $C^\infty$ map $g(t) = f(t-1)f(t+1)$ is positive on $(-1,1)$, and zero everywhere else. Similarily, for any $\varepsilon$, there is a map $g_\varepsilon$ which is positive on $(-\varepsilon, \varepsilon)$ and zero elsewhere.
    \item The map 
    %
    \[ l(t) = \begin{cases}
        \left(\int_{-\varepsilon}^t g_\varepsilon \right)/\left(\int_{-\varepsilon}^\varepsilon g_\varepsilon \right) & : t \in (0, \infty) \\
        0 & : \text{elsewise}
    \end{cases} \]
    %
    is $C^\infty$, is zero for negative $t$, increasing on $(0, \varepsilon)$, and one on $[\varepsilon, \infty)$.
    \item There is a differentiable map $h:\mathbf{R}^n \to \mathbf{R}$ defined by $h(x_1, \dots, x_n) = g(x_1) g(x_2) \dots g(x_n)$ which is positive on $(-1, 1)^n$, and zero elsewhere.
\end{enumerate}

With these nice functions in hand, we may form them on arbitrary manifolds.

\begin{theorem}
    If $M$ is a differentiable manifold, and $C$ is a compact set contained in an open set $U$, then there is a differentiable function $f:M \to \mathbf{R}$ such that $f(x) = 1$ for $x$ in $C$, and whose support $\overline{\{ x \in M : f(x) \neq 0 \}}$ is contained entirely within $U$.
\end{theorem}
\begin{proof}
    For each point $p$ in $C$, consider a chart $(x,V)$ around $p$, with $\overline{V} \subset U$, and $x(V)$ containing the open unit square $(-1,1)^n$ in $\mathbf{R}^n$. We may clearly select a finite subset of these charts $(x_k,V_k)$ such that the $x_k^{-1}((-1,1)^n)$ cover $C$. We may define a map $f_k:V_k \to \mathbf{R}$ equal to $h \circ x_k$, where $h$ is defined above. It clearly remains $C^\infty$ if we extend it to be zero outside of $V_k$. Then $\sum f_k$ is positive on $C$, and whose closure is contained within $\bigcup \overline{V_k} \subset U$. Since $C$ is compact, and the function is continuous, $\sum f_k$ is bounded below by $\varepsilon$ on $C$. Taking $f = l \circ (\sum f_k)$, where $l$ is defined above, we obtain the map needed.
\end{proof}

To enable us to define $C^\infty$ functions whose support lie beyond this range, we need to consider a technique to extend $C^\infty$ functions defined locally to manifolds across the entire domain.

\begin{definition}
    A partition of unity on a manifold $M$ is a family of $C^\infty$ functions $\{ \phi_i : i \in I \}$, and such that the following two properties hold:
    %
    \begin{enumerate}
        \item The supports of the functions forms a locally finite set.
        \item For each point $p \in M$, the finite sum $\sum_{i \in I} \phi_i(p)$ is equal to 1.
    \end{enumerate}
    %
    If $\{ U_i \}$ is an open cover of $M$, then a partition of unity is subordinate to this cover if it also satisfies (3):
    %
    \begin{enumerate}
        \item[3.] The closure of each function is contained in some element of the cover.
    \end{enumerate}
\end{definition}

It is finally our chance to use the topological `niceness' established in the previous chapter.

\begin{lemma}[The Shrinking Lemma]
    If $M$ is a paracompact manifold, and $\{ U_i \}$ is an open cover, then there exists a refined cover $\{ V_i \}$ such that for each $i \in I$ there exists $i'$ such that $\overline{V_i} \subset U_{i'}$.
\end{lemma}
\begin{proof}
    Without loss of generality, we may assume $\{ U_i \}$ is locally finite, and $M$ is connected. Then $M$ is also $\sigma$-compact, $M = \bigcup C_i$. Since $C_i$ is compact, and each $p \in C_i$ locally intersects only finitely many $U_i$, then $C_i$ intersects only finitely many $U_i$. Therefore $\bigcup C_i$ intersects only countably many $U_i$, and thus our locally finite cover is countable.

    Consider an ordering $\{ U_1, U_2, \dots \}$ of $\{ U_i \}$. Let $C_1$ be the closed set $U_1 - (U_2 \cup U_3 \cup \dots)$. Let $V_1$ be an open set with $C_1 \subset V_1 \subset \overline{V_1} \subset U_1$. Inductively, let $C_k$ be the closed set $U_k - (V_1 \cup \dots \cup V_{k-1} \cup U_{k+1} \cup \dots)$, and define $V_k$ to be an open set with $C_k \subset V_k \subset \overline{V_k} \subset U_k$. Then $\{ V_1, V_2 \dots \}$ is the desired refinement.
\end{proof}

\begin{theorem}
    Any cover on a paracompact manifold induces a subordinate partition of unity.
\end{theorem}
\begin{proof}
    Let $\{ U_i \}$ be an open cover of a paracompact manifold $M$. Without loss of generality, we may consider $\{ U_i \}$ locally finite. Suppose that each $U_i$ has compact closure. Choose $\{ V_i \}$ satisfying the shrinking lemma. Apply Theorem (2.13) to $\overline{V_i}$ to obtain functions $\psi_i$ that are 1 on $\overline{V_i}$ and zero outside of $U_i$. These functions are locally finite, and thus we may define $\phi_i$ by
    %
    \[ \phi_i(p) = \frac{\psi_i(p)}{\sum_j \psi_j(p)} \]
    %
    Then $\phi_i$ is the partition of unity we desire.

    This theorem holds for any $\{ U_i \}$ provided Theorem (2.13) holds on any closed set, rather than just a compact one. Let $C$ be a closed subset of a manifold, contained in an open subset $U$, and for each $p \in C$, choose an open set $U_p$ with compact closure contained in $U$. For each $p \in C^c$, choose an open subset $V_p$ contained in $C^c$ with compact closure. Then our previous compact case applies to this cover, and we obtain a subordinate partition of unity $\{ \zeta_i \}$. Define $f$ on $M$ by
    %
    \[ f(p) = \sum_{\overline{\zeta_i} \subset U_p} \zeta_i(p) \]
    %
    Then the support of $f$ is contained within $U$, and $f = 1$ on $C$.
\end{proof}

Partitions of unity allow us to extend local results on a manifold to global results. The utility of these partitions is half the reason that some mathematicians mandate that manifolds must be paracompact -- otherwise many nice results are lost.

\begin{theorem}
    In a $\sigma$-compact manifold, there exists a smooth function $f:M \to \mathbf{R}$ such that $f^{-1}((-\infty, t])$ is compact for each $t$.
\end{theorem}
\begin{proof}
    Let $M$ be a $\sigma$-compact manifold with $M = \bigcup B_i$, Where $\overline{B_i}$ is compact and $B_i$ is diffeomorphic to a ball. Consider a partition of unity $\{\psi_i\}$ subordinate to $\{B_i\}$, and consider the function
    %
    \[ f(x) = \sum k \psi_k(x) \]
    %
    Then $f$ is smooth, since locally it is the finite sum of smooth functions. If $x \not \in B_1, \dots, B_n$, then
    %
    \[ f(x) = \sum_{k = 1}^\infty k \psi_k(x) = \sum_{k = n+1}^\infty k \psi_k(x) \geq (n+1) \sum_{k = n+1}^\infty \psi_k(x) = (n+1) \]
    %
    Therefore if $f(x) < n$, $x$ is in some $B_i$. Thus $f^{-1}((-\infty, n])$ is a closed subset of $\overline{B_1} \cup \dots \cup \overline{B_n}$, and is therefore compact.
\end{proof}





\chapter{Tensor Calculus}

Tensor calculus is often left out of a linear algebra course in interests of time. Nonetheless, it is an important field of linear algebra, and even more so in differential geometry. We shall use it extensively.

In Linear algebra, one analyzes the structure of a vector space $V$ by analyzing linear transformations from the space to another space $W$. In particular, it is of interest to analyze the linear maps from $V$ to the field $\mathbf{F}$ over which the vector space is defined. We call the set of all such linear maps the dual space of $V$, and denote it $V^*$.

Now suppose $V$ is a {\it finite dimensional} vector space. Since $\mathbf{F}$ is also finite dimensional (almost every represented with the obvious basis $\{ 1 \}$), given any basis $\beta = (v_1, \dots, v_n)$ for $V$, we can express any map $\Lambda \in V^*$ as a $1 \times n$ matrix, denoted $[\Lambda]_{\beta}$. It follows that $V^*$ is $n$-dimensional.

Fixing $\beta$ as a basis, we may consider, for each $v_i \in \beta$, the dual element $v_i^* \in V^*$, defined by the fact that $v_i^*(v_j) = \delta_{i,j}$, or from the matrix representation
%
\[ [v_i^*]_\beta = (0, \dots, 1, \dots, 0) \]
%
This becomes a basis for $V^*$, known as the {\bf dual basis} so that $V^*$ is naturally isomorphic to $V$, given a basis representing $\beta$. Nonetheless, choosing a basis will become our bane, since in spaces that do not have a clear basis there is no clear isomorphism between $V$ and $V^*$. In fact, $V$ and $V^*$ are often not isomorphic in the infinite dimensional setting. The basis transformation only provides an embedding in this case.

Now consider another vector space $W$, with a basis $\alpha = (w_1, \dots, w_n)$. Consider any linear map $T:V \to W$. This in fact induces a map $T^*:W^* \to V^*$, defined by $(T^*\Lambda)(v) = \Lambda(Tv)$. This satisfies the identity $(T \circ G)^* = G^* \circ T^*$, since
%
\[ (T \circ G)^*\Lambda(v) = \Lambda(T \circ G)(v) = \Lambda(T(G(v)))) = (T^*\Lambda)(G(v)) = G^*(T^*(v)) \]
%
Suppose that $[T]_\beta^\alpha = (a_{ij})$, so that $Tv_i = \sum \alpha_{ij} w_j$. This implies that
%
\[ (T^*w_i^*)(v_j) = w_i^*(\sum a_{jk} w_k) = a_{ji} \]
%
Thus $T^*w_i^* = \sum a_{ji} v_i^*$, so that, in the induces basis $\alpha^*$ and $\beta^*$, $[T^*]_{\alpha^*}^{\beta^*} = ([T]_\beta^\alpha)^t$. It is for this reason that $T^*$ is known as the {\bf transpose} of $T$.

Since $V^*$ is a vector space as well, we may consider the {\bf double dual} $V^{**}$. There is a much nicer relationship between $V$ and its double dual. For each $v \in V$, we define $v^{**}(\Lambda) = \Lambda(v)$. This is a natural embedding of $V$ in $V^{**}$, which is an isomorphism in the case $V^{**}$ is finite dimensional.


\chapter{The Tangent Bundle}

Geometrically, $\mathbf{R}^n$ is just a system of points in space. Identification of these points with coordinates gives us numerical facts about the space we live in. Algebraically, $\mathbf{R}^n$ is a system of arrows about an origin point, which can be added, subtracted, and scaled. But why do these arrows have to start at the origin? A differentiable curve $c:(a,b) \to \mathbf{R}^n$ has a tangent vector $c'(x)$, canonically pictured as eminating from the point $c(x)$. The idea of a vector space eminating from each point on a manifold will pave the way to all future endeavors. We call such a space a vector bundle.

\begin{definition}
    A {\bf vector bundle} is a pair of spaces $E$ and $B$, called the {\bf total} and {\bf base} space respectively, with a continuous function $\pi$ mapping $E$ onto $B$, such that
    %
    \begin{enumerate}
        \item[(1)] For each point $p \in B$, the {\bf fibre space} $\pi^{-1}(p)$ forms a real vector space. We denote this vector space by $B_p$.
        \item[(2)] The vector bundle is locally trivial. For every point $p \in B$, there is a neighbourhood $U$ of $p$, a space $\mathbf{R}^k$, and a homeomorphism $\phi$ mapping $\pi^{-1}(U)$ onto $U \times \mathbf{R}^k$, linear on each fibre.
    \end{enumerate}
    %
    Every connected component of $B$ has a unique dimension $k$ for which (2) holds. A vector bundle is {\bf k-dimensional} if the dimension $k$ is invariant across all of $B$. If $E$ and $B$ are manifolds, and the mappings $\phi$ and $E$ differentiable, we call the vector bundle {\bf differentiable}. Collectively, we shall denote a vector bundle by $(E,B,\pi)$.
\end{definition}

\begin{example}
    Consider any topological space $U$, and let $\varepsilon^n(U) = U \times \mathbf{R}^n$. Denote $(p,v) \in \varepsilon^n(U)$ by $v_p$, and define $\pi$ by $v_p \mapsto p$. If we define $v_p + w_p = (v + w)_p$, and $c(v_p) = (cv)_p$, we form a vector bundle on $U$, known as the {\bf trivial bundle}. We may picture $\varepsilon^1(\mathbf{R})$ as the plane, and looking at $\mathbf{R}_p$ will tell you why we call such a space a fibre.
\end{example}

\begin{definition}
    If $(E,A,\pi)$ and $(F,B,\phi)$ are two vector bundles, a {\bf bundle map} between the two is a pair of continuous functions $(f,f_\sharp)$ where $f:A \to B$ and $f_\sharp:E \to F$, where $f_\sharp$ maps $A_p$ linearly into $B_{f(p)}$, for any $p \in A$. An isomorphism in the category of bundle maps is known as an equivalence.
\end{definition}

Here, bundle maps are introduced to extend the notion of a derivative.

\begin{example}
    The derivative of a map at a point is a linear approximation. We can collect these maps together to form a bundle map: if $f:\mathbf{R}^n \to \mathbf{R}^m$ is differentiable, then we may construct the map $f_*:\varepsilon^n(\mathbf{R}^n) \to \varepsilon^m(\mathbf{R}^m)$, defined by $f_*(v_p) = [Df(p)(v)]_{f(p)}$.
\end{example}

Our current aim is to define, for each manifold $M$, a bundle $TM$ extending off the space, such that if $f:M \to N$ is differentiable, then it induces a map $f_*:TM \to TN$, which possesses the properties of differentiation familiar to Euclidean space.

\begin{thebibliography}{10}
    \bibitem{intro} Michael Spivak,
    \emph{A Concise Introduction to Differential Geometry: Vol. One}

    \bibitem{halm} Paul Halmos,
    \emph{Naive Set Theory}

    \bibitem{wiki} Wikipedia,
    \emph{Lie Groups}
\end{thebibliography}

\end{document}


\section{And Now Briefly, Some Boundaries}

\begin{definition}
    A {\bf manifold with boundary} is a space also containing points that are {\it locally bounded}. Points in such a space must satisfy either (1) or (2):
    %
    \begin{enumerate}
        \item[(2)] $p$ has a neighbourhood homeomorphic to a `halfspace':
        %
        \[ \mathbf{H}^n = \{ (x_1, \dots, x_n) \in \mathbf{R}^n: x_1 \geq 0 \} \]
    \end{enumerate}
    %
    The {\bf boundary} of a manifold with boundary consists of those points satisfying (2), but never (1). If $M$ is a manifold with boundary, then we denote its boundary by $\partial M$.
\end{definition}

As we've introduced manifolds with boundary, we might as well mention a useful theorem about them before we get onto the deeper topics of this chapter.

\begin{theorem}
    If $M^n$ is a manifold with boundary, then $\partial M$, considered as a subspace of $M$, is a manifold (without boundary) of dimension $n-1$.
\end{theorem}
\begin{proof}
    Let $p$ be a point in $\partial M$, and let $U$ be a neighbourhood homeomorphic to $\mathbf{H}^n$ by a map $f:U \to \mathbf{H}^n$. Consider the points in $U$ that map to the boundary plane
    %
    \[ P = \{ q \in U : f(q) = (0,x_2, \dots, x_n) \} \]
    %
    We contend that $P = U \cap \partial M$. Surely this is a subset, since if a point $x \in U$ does not lie on this line, we can select a subneighbourhood which is open in $\mathbf{R}^n$. If a point $x$ lies on this subplane, we cannot find a neighbourhood homeomorphic to $\mathbf{R}^n$, since any neighbourhood of this point in $\mathbf{H}^n$ is not open in $\mathbf{R}^n$, and invariance of domain applies. It is simple to show $U \cap \partial M$, as we have now described it, contains a subneighbourhood homeomorphic to $\mathbf{R}^{n-1}$, and thus we have shown that all points of $\partial M$ have neighbourhoods homeomorphic to $\mathbf{R}^{n-1}$.
\end{proof}

\begin{example}
    $\mathbf{H}^n$ is the easiest example of a manifold with boundary. It's boundary consists of $\{ 0 \} \times \mathbf{R}^{n-1}$, which is an $n - 1$ manifold. Another manifold with boundary is the unit disc $D^n = \{ x \in \mathbf{R}^n : \|x\| \leq 1 \}$. We have already shown that the discs boundary, $\partial D^n = S^{n-1}$, is an $n - 1$ manifold. In general, if $M$ is an open submanifold of $\mathbf{R}^n$, then its topological boundary $\overline{M} = M \cup \partial M$ is a manifold with boundary.
\end{example}

We will see later that considering a manifold boundary as a disjoint structure is very useful. Stoke's theorem provides a direct application.










For the next theorem, notice that partitions of unity exist on a paracompact manifold with boundary, since we may first take a partition of unity on the interior of the manifold, then combine it with a partition of unity on charts of the boundary, locally extended to be in the manifold.

\section{Differentiable Manifolds with Boundary}

Recall that we may extend differentiability on subsets of Euclidean space in the following way. A map $f:B \to C$ is differentiable, where $B \subset \mathbf{R}^n$ and $C \subset \mathbf{R}^m$, if $f$ can be extended to an open subset containing $B$, such that the extended map is differentiable.

\begin{theorem}
    If $f:\mathbf{H}^n \to \mathbf{R}$ is locally differentiable (every point has a neighbourhood on which $f$ can be locally extended to be differentiable).
\end{theorem}
\begin{proof}
    Let $\mathcal{O}$ be a open cover of $\mathbf{H}^n$ in $\mathbf{R}^n$ such that for each $U \in \mathcal{O}$, there is a differentiable function $g_U:U \to \mathbf{R}$ such that $g|_{\mathbf{H}^n} = f$. Consider a partition of unity $\{ \Phi_k \}$ subordinate to $\mathcal{O}$. For each $\Phi_k$, pick some $U_k \in \mathcal{O}$ such that the support of $\Phi_k$ is contained within $U_k$, and define a function $g = \sum g_k \Phi_k$, defined on $\bigcup \mathcal{O}$, a open extension of $\mathbf{H}^n$. Each pair $g_k$ and $\Phi_k$ are differentiable, so the multiplication of the two is differentiable. Since these functions are locally finite, we also have $g$ differentiable across its domain. If $p \in \mathbf{H}^n$, then $g_k(p) = f(p)$. Thus
    %
    \[ g(p) = \sum g_k(p) \Phi_k(p) = \sum f(p) \Phi_k(p) = f(p) \]
    %
    since the $\Phi_k$ sum up to one. Thus $g|_{\mathbf{H}^n} = f$, and we have extended $f$ to be differentiable.
\end{proof}

This should aid us in thinking about differentiability on arbitrary manifold with boundary.

\begin{definition}
    Let $(x,U)$ and $(y,V)$ be charts on two manifolds $M$ and $N$ respectively. A map $f:M \to N$ between manifolds with boundary is {\bf differentiable} if $y \circ f \circ x^{-1}:x(U) \to y(V)$ can be extended to be a differentiable function, regardless of which charts $x$ and $y$ are differentiable.
\end{definition}

\begin{example}[(Differentiable structures on the boundary of a manifold.)]]
    With this definition, the canonical differentiable structure on $\partial M$ is the unique atlas such that the inclusion map $i:\partial M \to M$ is differentiable. We can generate it from the atlas consisting of the restriction of charts on $M$ to the boundary, projected into an $(n-1)$ dimensional subspace of $\mathbf{R}^n$
\end{example}