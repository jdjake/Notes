\documentclass[12pt]{report}

\usepackage{amsmath}
\usepackage{amssymb}
\usepackage{amsthm}
\usepackage{amsopn}
\usepackage{kpfonts}
\usepackage{graphicx}
\usepackage{kbordermatrix}
\usepackage{tikz}
\usetikzlibrary{arrows, petri, topaths}%
\usepackage{tkz-berge}
\usepackage{multicol}

\usepackage{framed}
\usepackage{mathtools}
\usepackage{float}
\usepackage{subfig}
% \usepackage{cmbright}

\theoremstyle{plain}
\newtheorem{theorem}{Theorem}[chapter]
\newtheorem{lemma}[theorem]{Lemma}
\newtheorem{corollary}[theorem]{Corollary}
\newtheorem{prop}[theorem]{Proposition}
\newtheorem{exercise}{Exercise}[chapter]

\newtheorem*{example}{Example}
\newtheorem*{proof*}{Proof}

\theoremstyle{definition}
\newtheorem*{defi}{Definition}
\newenvironment{definition}
    {\begin{samepage}\begin{framed}\begin{defi}}
    {\end{defi}\end{framed}\end{samepage}}





\usepackage{hyperref} 
\hypersetup{
    colorlinks = true,
    linkcolor = black,
}

\makeatletter
\renewcommand*\env@matrix[1][*\c@MaxMatrixCols c]{%
  \hskip -\arraycolsep
  \let\@ifnextchar\new@ifnextchar
  \array{#1}}
\makeatother

\renewcommand*\contentsname{\hfill Table Of Contents \hfill}

\newcommand{\optionalsection}[1]{\section[* #1]{(Important) #1}}
\newcommand{\deriv}[3]{\left. \frac{\partial #1}{\partial #2} \right|_{#3}}

\title{Partial Differential Equations}
\author{Jacob Denson}

\begin{document}

\pagenumbering{gobble}

\maketitle

\tableofcontents

\pagenumbering{arabic}

\chapter{Mean Value Theorem}

The mean value theorem says that if $f$ is differentiable on $(a,b)$ and continuous on $[a,b]$, then there exists $c \in (a,b)$ such that
%
\[ f(b) - f(a) = (b-a) f'(c) \]
%
So the slope between the points is actually the slope of a derivative of $f$ between $a$ and $b$. More generally, if $f$ and $g$ are differentiable on $(a,b)$ and continuous on $[a,b]$, then there exists $c \in (a,b)$ such that
%
\[ [f(b) - f(a)] g'(c) = [g(b) - g(a)] f'(c) \]
%
A more intuitive way to interpret this generalization is to rewrite the formula as
%
\[ \frac{g'(c)}{f'(c)} = \frac{\frac{g(b) - g(a)}{b - a}}{\frac{f(b) - f(a)}{b - a}} \]
%
So the ratio in slopes between the two functions is actually a ratio in the derivatives. The advantage of the mean value theorem is that in enables us to use properties of the derivative to infer properties about the slopes, which are often harder to bound individually.

\begin{theorem}
	If $f$ is a continuously differentiable function, then $f(x+y) - f(x)$ converges to $(\nabla f)(x) \cdot y$ locally uniformly.
\end{theorem}
\begin{proof}
	Without loss of generality assume we are working over a compact region $K$. Then $\nabla f$ is uniformly continuous on $K$, so for each $\varepsilon$, there is $\delta$ such that if $|y| < \delta$, then $|\nabla f(x+y) - \nabla f(x)| \leq \varepsilon$. The mean value theorem implies that there is a point $z$ on the line between $x$ and $x + y$ such that
	%
	\[ f(x+y) - f(x) = (D_y f)(z) = (\nabla f)(z) \cdot y \]
	%
	Thus
	%
	\begin{align*}
		\left| f(x+y) - f(x) - (\nabla f)(x) \cdot y \right| &= |[(\nabla f)(z) - (\nabla f)(x)] \cdot y|\\
		&\leq |(\nabla f)(z) - (\nabla f)(x)||y|\\
		&\leq \varepsilon \delta
	\end{align*}
	%
	Since this bound depends only on $y$, the convergence is locally uniform.
\end{proof}

\end{document}