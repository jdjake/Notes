\documentclass[12pt]{report}

\usepackage{amsmath}
\usepackage{amssymb}
\usepackage{amsthm}
\usepackage{amsopn}
\usepackage{kpfonts}
\usepackage{graphicx}
\usepackage{kbordermatrix}
\usepackage{tikz}
\usetikzlibrary{arrows, petri, topaths}%
\usepackage{tkz-berge}
\usepackage{multicol}

\usepackage{framed}
\usepackage{mathtools}
\usepackage{float}
\usepackage{subfig}
% \usepackage{cmbright}

\theoremstyle{plain}
\newtheorem{theorem}{Theorem}[chapter]
\newtheorem{lemma}[theorem]{Lemma}
\newtheorem{corollary}[theorem]{Corollary}
\newtheorem{prop}[theorem]{Proposition}
\newtheorem{exercise}{Exercise}[chapter]

\newtheorem*{example}{Example}
\newtheorem*{proof*}{Proof}

\theoremstyle{definition}
\newtheorem*{defi}{Definition}
\newenvironment{definition}
    {\begin{samepage}\begin{framed}\begin{defi}}
    {\end{defi}\end{framed}\end{samepage}}





\usepackage{hyperref} 
\hypersetup{
    colorlinks = true,
    linkcolor = black,
}

\makeatletter
\renewcommand*\env@matrix[1][*\c@MaxMatrixCols c]{%
  \hskip -\arraycolsep
  \let\@ifnextchar\new@ifnextchar
  \array{#1}}
\makeatother

\renewcommand*\contentsname{\hfill Table Of Contents \hfill}

\newcommand{\optionalsection}[1]{\section[* #1]{(Important) #1}}
\newcommand{\deriv}[3]{\left. \frac{\partial #1}{\partial #2} \right|_{#3}}

\title{Putnum and Beyond Solution Manual}
\author{Jacob Denson}

\begin{document}

\pagenumbering{gobble}
\maketitle
\tableofcontents
\pagenumbering{arabic}

\chapter{Methods of Proof}

\section{Argument by Contradiction}

\begin{exercise} Prove that $\sqrt{2} + \sqrt{3} + \sqrt{5}$ is irrational. \end{exercise}
\begin{proof}
    Suppose $w = \sqrt{2} + \sqrt{3} + \sqrt{5}$ is rational. Then
    %
    \[ w^2 = 10 + 2(\sqrt{6} + \sqrt{10} + \sqrt{15}) \]
    %
    is rational, since $10$ and $2$ are rational numbers, we may subtract and divide them, respectively, to conclude that $v = \sqrt{6} + \sqrt{10} + \sqrt{15}$ is rational. Taking the square again, we obtain another rational number
    %
    \[ v^2 = 21 + 2(\sqrt{60} + \sqrt{150} + \sqrt{90}) = 21 + 2(2\sqrt{15} + 5 \sqrt{6} + 3 \sqrt{10}) \]
    %
    From which we conclude $u = 2\sqrt{15} + 5 \sqrt{6} + 3\sqrt{10}$ is rational. But then
    %
    \[ u - 2v = 3 \sqrt{6} + \sqrt{10} \]
    %
    is rational, and
    %
    \[ (u - 2v)^2 = 64 + 12 \sqrt{15} \]
    %
    From which we conclude $\sqrt{15}$ is rational. But 15 is not a perfect square, so its square root must be an irrational number.
\end{proof}

\begin{exercise}
    Show that no set of nine consecutive integers can be partitioned into two sets with the product of the elements of the first set equal to the product of the elements of the second set.
\end{exercise}
\begin{proof}
    Suppose the integers $n, n+1, \dots, n+8$ can be partitioned into two sets $A$ and $B$, for which
    %
    \[ \prod_{a \in A} a = \prod_{b \in B} b \]
    %
    Let $p_1, p_2, \dots, p_n$ be all prime factors of the nine consecutive numbers. Let
    %
    \[ \prod_{a \in A} a = \prod_{i = 1}^n p_i^{j_i} \]
    %
    Then
    %
    \[ \prod_{i = 0}^8 (n + k) = \prod_{a \in A} a \prod_{b \in B} b = (\prod_{a \in A} a)^2 = \prod_{i = 1}^n p_i^{2j_i}  \]
    %
    By the fundamental theorem of arithmetic, the prime decomposition of the product of the range of numbers sums up all of their prime factors, and thus for each prime, the sum of the number of times the prime divides one of the integers, over all integers in the range, must be even, for each prime in $p_1$ to $p_n$. This already makes it impossible for
    %
    \[ 1,2, \dots, 9 \]
    %
    To be partitioned into two sets, since there are $7$ twos in the sum of the prime decompositions, an odd number.

    Divisors occur in periods. If $p | k$, then $p | k + p$, and $p | k - p$, but $p$ cannot divide any other integers in $(k-p,k+p)$. This implies that, if it is possible to partition consecutive integers, then no prime greater than 9 can divide a number in the range of integers, for if this were so, by the remark above, it must be the only integer in the range divisible by the prime, and therefore there is no way to partition integers into two sets for which both sets contain a number divisible by that prime. hence we may assume our integers are only divisible by the primes $2,3,5$, and $7$.

    In general, the divisors of any integer also occur in periods. Therefore, only at most 3 integers can be divisor by 4, and at most 1 can be divisible by 9, 25, and 49. Therefore there are at least 3 integers, which can only divisible by the primes $2,3,5,7$ at most once. Therefore these 3 integers are among the integers
    %
    \[ 2,3,5,6,7,10,14,15,21,30,35,42,70,105,210 \]
    %
    But these number must be at most 9 integers apart, so that they must be among the integers in the range $1$ to $22$. We may now proceed on a case by case basis. We have already proved the impossibility of our problem for the range $1$ to $9$. The range $2$ to $10$ has similar problems, since only one integer is divisible by $7$. Any range which contains a number divisible by 11 cannot be possible, so this rules out all the ranges from $3$ to $11$ up to $11$ to $19$. But then the the range $12$ to $20$ up to $13$ to $22$ all contain a number divisible by $13$.
\end{proof}

\begin{exercise}
    Find the least positive integer $n$ such that any set of $n$ pairwise relatively prime integers greater than 1 and less than 2005 contains at least one prime number.
\end{exercise}
\begin{proof}
    If $k$ is an integer between 1 and $2005$, then either $k$ is prime, or $k$ is divisible by a prime number less than $\sqrt{2005} < 47$. The prime numbers between 1 and 47 are described below
    %
    \[ 2,3,5,7,11,13,17,19,23,29,31,37,41,43 \]
    %
    We must have $n > 14$, for the numbers $2^2, 3^2, \dots, 43^2$ are between 1 and 2005, are definitely not prime, and are all relatively prime. We shall show $n = 15$. Suppose we have 15 integers $k_1, \dots, k_{15}$, none of which are prime, all of which are pairwise coprime, and all of which are divisible by a prime (since they are greater than one). This means that for every $i$, one of the primes above divides $k_i$. But since there are more integers than prime numbers above, by the pidgeonhole principle, there must be a prime that divides two different integers $k_i$ and $k_j$. But this implies the two integers are not coprime.
\end{proof}

\section{Putnum 2014}

\begin{exercise}
    Prove that every nonzero coefficient of the Taylor series of
    %
    \[ (1 - x + x^2) e^x \]
    %
    about $x = 0$ is a rational number whose numerator (in lowest terms) is either 1 or a prime number.
\end{exercise}
\begin{proof}
    Consider $f(x) = P e^x$, where $P$ is a polynomial. Then $f'(x) = (P + P')e^x$. Let $x_0, x_1, \dots$ be the coefficients of the Taylor expansion. Consider the recursive formula $y_0 = 1 - x + x^2$, and $y_{k + 1} = y_k + y_k'$. The first few terms are calculated below
    %
    \begin{align*}
        y_0 &= x^2 - x + 1\\
    %
        y_1 &= x^2 + x\\
    %
        y_2 &= x^2 + 3x + 1\\
    %
        y_3 &= x^2 + 5x + 4
    \end{align*}
    %
    We shall prove that
    %
    \[ y_k = x^2 + (2k - 1)x + (1 + k^2 - 2k) \]
    %
    This is true for $y_0$ by definition, and if it is true for $y_k$, then
    %
    \begin{align*}
        y_{k+1} &= y_k + y_k'\\
        &= [x^2 + (2k - 1)x + (1 + k^2 - 2k)] + [2x + (2k - 1)]\\
        &= x^2 + (2k + 1)x + k^2\\
        &= x^2 + (2(k+1) - 1)x + (1 + (k+1)^2 - 2(k+1))
    \end{align*}
    %
    This shows our formula is true by induction, and therefore
    %
    \[ x_k = \frac{y_k(0)}{k!} = \frac{k^2 - 2k + 1}{k!} = \frac{(k-1)^2}{k!} = \frac{k-1}{k \cdotp (k-2)!} \]
    %
    We shall show by strong induction that any fraction of the form
    %
    \[ \frac{m}{(m-1)!} \]
    %
    can be reduced to a fraction where the numerator is zero, one, or prime. The theorem is obviously true for $m = 1$. Suppose that $m$ is not a prime number. Then we may write $m = kn$, where $k,n \geq 2$. We must have $k,n < m$, for otherwise, if $k \geq m$, $kn \geq 2m > m$. If $k \neq n$, then both $k$ and $n$ appear in $(k - 2)!$, and we may factor them out, obtaining a fraction whose numerator is one. If $k = n$, then we may factor one of the numbers out, obtaining a fraction of the form
    %
    \[ \frac{n}{\frac{(m-1)!}{n}} = \frac{n}{(n-1)!} \frac{1}{\frac{(m-1)!}{(n-1)!}} \]
    %
    By the inductive hypothesis, we may reduce the fraction of the right hand side to one whose numerator is zero, one, or prime, as was required, and this completes the proof.
\end{proof}

\end{document}