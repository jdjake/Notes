\documentclass[12pt]{report}

\usepackage{amsmath}
\usepackage{amssymb}
\usepackage{amsthm}
\usepackage{amsopn}
\usepackage{kpfonts}
\usepackage{graphicx}
\usepackage{kbordermatrix}
\usepackage{tikz}
\usetikzlibrary{arrows, petri, topaths}%
\usepackage{tkz-berge}
\usepackage{multicol}

\usepackage{framed}
\usepackage{mathtools}
\usepackage{float}
\usepackage{subfig}
% \usepackage{cmbright}

\theoremstyle{plain}
\newtheorem{theorem}{Theorem}[chapter]
\newtheorem{lemma}[theorem]{Lemma}
\newtheorem{corollary}[theorem]{Corollary}
\newtheorem{prop}[theorem]{Proposition}
\newtheorem{exercise}{Exercise}[chapter]

\newtheorem*{example}{Example}
\newtheorem*{proof*}{Proof}

\theoremstyle{definition}
\newtheorem*{defi}{Definition}
\newenvironment{definition}
    {\begin{samepage}\begin{framed}\begin{defi}}
    {\end{defi}\end{framed}\end{samepage}}





\usepackage{hyperref} 
\hypersetup{
    colorlinks = true,
    linkcolor = black,
}

\makeatletter
\renewcommand*\env@matrix[1][*\c@MaxMatrixCols c]{%
  \hskip -\arraycolsep
  \let\@ifnextchar\new@ifnextchar
  \array{#1}}
\makeatother

\renewcommand*\contentsname{\hfill Table Of Contents \hfill}

\newcommand{\optionalsection}[1]{\section[* #1]{(Important) #1}}
\newcommand{\deriv}[3]{\left. \frac{\partial #1}{\partial #2} \right|_{#3}}

\title{Algebraic Geometry}
\author{Jacob Denson}

\begin{document}

\pagenumbering{gobble}
\maketitle
\tableofcontents
\pagenumbering{arabic}

\chapter{Affine Algebraic Sets}

In Euclidean geometry, we find that most of the interesting geometric objects can be described as sets of points which can be described as solutions to some multivariate polynomial. Almost all of the elegant mathematical formulas occur as relations between polynomials.
%
\begin{itemize}
    \item The unit circle in the plane can be described as the set of points satisfying the equation $X^2 + Y^2 = 1$.
    \item A parabola can be described as those points satisfying $Y = X^2$.
    \item A hyperbola the solution set to $X^2 - Y^2 = 1$.
\end{itemize}
%
These are the basic conic sections which occur in high school geometry. The field of algebraic geometry is the study of solution sets to polynomial equations. It has surprising applications to modern physics, combinatorics, and certain parts of analysis.

From a Euclidean perspective, geometry is done in the $n$ dimensional affine space $\mathbf{A}^n$ with respect to some field $K$. After a certain choice of coordinates, $\mathbf{A}^n$ can be identified with $K^n$, as is customary in analytic geometry. $\mathbf{A}^1$ is referred to coloquially as the affine line, and $\mathbf{A}^2$ as the affine plane. Given a polynomial $f \in K[X_1, \dots, X_n]$, we can consider the zero set
%
\[ V(f) = \{ p \in \mathbf{A}^n : f(p) = 0 \} \]
%
which is the {\bf hypersurface} defined by $f$. We can also consider
%
\[ V(f,g) = \{ p \in \mathbf{A}^n: f(p) = 0\ \text{and}\ g(p) = 0 \} = V(f) \cap V(g) \]
%
which is the intersection of two hyperplanes, More generally, given a set $S$ of polynomials, we can consider the set $V(S)$, which consists of the points forming the set of common zeroes of all polynomials in $S$. These zero sets are called {\bf affine varieties}, and they are the main object of study in algebraic geometry.

\begin{example}
    The unit circle is the variety $V(X^2 + Y^2 - 1)$.
\end{example}

\begin{example}
    The set $\{ (t,t^2,t^3): t \in K \}$ is the zero set of $V(Y^2 - X, Z - X^3)$, and this shows it is an affine variety in $\mathbf{A}^3$.
\end{example}

\begin{example}
    The set of points whose polar coordinates $(r,\theta)$ satisfy $r = \sin(\theta)$ form a variety. If these polar coordinates were induced by an affine coordinate system $(X,Y)$, then $r \sin(\theta) = Y$, and $r^2 = X^2 + Y^2$, so we may multiply the equation by $r$ to obtain the equation $X^2 + Y^2 = Y$. We find that the equation can be rearranged to $X^2 + (Y-1/2)^2 = 1/4$, so the solution set is just a circle of radius $1/2$ centered at $(0,1/2)$.
\end{example}

The class of affine varieties is interesting from the point of view of Euclidean geometry, because it is invariant under affine transformations, and many of the interesting shapes in Euclidean geometry can be identified with certain varieties through the tools of analytic geometry. Every affine transformation $T$ on $\mathbf{A}^n$ has an affine inverse $U = T^{-1}$, and $T$ maps $V(S)$ onto $V(U^*S)$, where $U^*: K[X_1, \dots, X_n] \to K[X_1, \dots, X_n]$ is the homomorphism $f \mapsto f \circ U$. It is easy to prove that $U^*$ preserves the degree of polynomials, which is useful in certain situations, for instance, to prove interesting results about intersections of varieties. First, define an {\bf affine plane curve} to be a hypersurface in $\mathbf{A}^2$.

\begin{theorem}
    An affine plane curve specified by a polynomial of degree $n$ intersects a line in at most $n$ places.
\end{theorem}
\begin{proof}
    Consider first the easy case where the line is just the $X$ axis. In this case, the zeroes of a polynomial $f(X,Y) = \sum a_{ij} X^iY^j$ which lie on the $X$ axis are in one to one correspondence with the set of solutions to the univariate polynomial $f(X,0) = \sum a_{i0} X^i$, which can have at most $\deg f(X,0) \leq \deg f = n$ separate points. Now in general, we can apply an affine transformation $T$ to any line to map it to the $X$ axis, and the points of $V(f)$ which lie on the line are in one to one correspondence with the points of $V(U^*f)$ which lie on the $X$ axis, where $U$ is the inverse of $T$. The theorem is then proved because $U^* f$ has the same degree as $f$.
\end{proof}

This theorem can be used to proved that certain planar curves are {\it not} affine varieties. We shall find that affine varieties are a very rigid class of objects, and we can prove many deep theorems about this class of objects.

\begin{example}
    The set of points $(x,y)$ in the real affine plane satisfying $y = \sin(x)$ cannot form a planar curve, because the curve intersects the $X$ axis infinitely often. If the points did form an algebraic variety $V(\mathfrak{a})$, where $\mathfrak{a}$ contains some nonzero polynomial $f$, then $V(f) \supset V(\mathfrak{a})$ would intersect the $X$ axis infinitely often, which is clearly impossible. This argument works for verifying that general varieties in the plane cannot intersect lines infinitely often, except in the trivial case where the variety is $\mathbf{A}^2$.
\end{example}

\begin{example}
    The complex sphere is the set of points $(z,w)$ in the complex affine plane satisfying $|z|^2 + |w|^2 = 1$, because the intersection of the complex sphere with the $z$ axis form a circle, which has infinitely many points. This justifies that the set cannot form a planar curve, and the same technique as in the last example shows the sphere cannot be an affine variety in general.
\end{example}

\begin{example}
    The set of points $\{ (\cos t, \sin t, t): t \in \mathbf{A}^3 \}$ over real affine space is not a variety, because it contains infinitely many points of the form $(1,0,\pi n)$, which lie on the same line.
\end{example}

There are some elementary observations we can make of this construction, which open the floodworks to the ring theory of $K[X_1, \dots, X_n]$.
%
\begin{itemize}
    \item If $\mathfrak{a}$ is the smallest ideal containing $S$, then $V(S) = V(\mathfrak{a})$, so every affine variety can be described as the common zeroes of some ideal. This follows because for any set $X \subset \mathbf{A}^n$, the set $I(X)$ of polynomials which vanish over $X$ forms an ideal, and it is clear that in the case where $X = V(S)$, the ideal contains all elements of $S$, hence all elements of $\mathfrak{a}$.

    \item If we have a family $\{ \mathfrak{a}_\alpha \}$ of ideals, then $V(\bigcup \mathfrak{a}_\alpha) = \bigcap V(\mathfrak{a}_\alpha)$, so the intersection of an arbitrary family of varieties forms a variety.

    \item If $\mathfrak{a} \subset \mathfrak{b}$, then $V(\mathfrak{b}) \subset V(\mathfrak{a})$.

    \item For any two polynomials $f$ and $g$, $V(fg) = V(f) \cup V(g)$. More generally, if $\mathfrak{a}$ and $\mathfrak{b}$ are ideals, then $V(\mathfrak{a}\mathfrak{b}) = V(\mathfrak{a}) \cup V(\mathfrak{b})$, so finite unions of varieties are varieties.

    \item $V(0) = \mathbf{A}^n$, $V(1) = \emptyset$, and for any $a \in K^n$, $V(X_1-a_1,\dots,X_n - a_n)$ is just the singleton set $\{ a \}$. It follows from the last point that finite point sets are varieties.

    \item If $\mathfrak{a} \subset K[X_1, \dots, X_n]$ and $\mathfrak{b} \subset K[Y_1, \dots, Y_m]$, then these ideals generate ideals $\mathfrak{a}'$ and $\mathfrak{b}'$ in $K[X_1, \dots, X_n, Y_1, \dots, Y_m]$. The intersection $V(\mathfrak{a}') \cap V(\mathfrak{b}')$ corresponds to the set of points $(x,y)$ with $x \in V(\mathfrak{a})$ and $y \in V(\mathfrak{b})$, and we call this the {\bf product variety} $V(\mathfrak{a}) \times V(\mathfrak{b})$.
\end{itemize}
%
Because of these properties, we begin to see that the analysis of $K[X_1, \dots, X_n]$ and its ideals is key to the study of algebraic geometry.

\begin{example}
    The varieties of $\mathbf{A}^1$ are exactly the finite point sets (other than the trivial variety $V(0) = \mathbf{A}^1$. First, note that since $K[X]$ is a principal ideal domain, we may assume we are considering the varieties of the form $V(f)$ for some particular polynomial $f(X) = \sum a_i X^i$. We know that $f(a) = 0$ if and only if $X - a$ is one of the prime factors of $f$. Since $f$ decomposes into finitely many prime factors, it follows that $V(f)$ can consist of at most $\text{deg}(f)$ points, a finite quantity. Thus the theory of one dimensional algebraic geometry is essentially trivial. This example shows that the countable union of affine varieties need not be a variety, because the countable union of finite sets need not be finite.
\end{example}

\begin{example}
    If $K$ is a finite field, then all subsets of $\mathbf{A}^n$ are varieties, because all subsets of $\mathbf{A}^n$ are finite subsets.
\end{example}

\begin{prop}
    If $f$ is a non constant polynomial over an algebraically complete field, $\mathbf{A}^n - V(f)$ contains infinitely many points for $n \geq 1$, and $V(f)$ contains infinitely many points for $n \geq 2$.
\end{prop}
\begin{proof}
    First, recall that every algebraically complete field $K$ must have infinitely many points, because if $K$ only contains $a_1, \dots, a_n$, we are unable to factor $(X - a_1) \dots (X - a_n) + 1$ into linear factors. It follows that $\mathbf{A}^1 - V(f)$ is infinite for any polynomial $f \in K[X]$, because $V(f)$ is finite. Given any polynomial $f \in K[X_1, \dots, X_n]$, there is a line in $\mathbf{A}^n$ upon which $f$ is not identically zero (for otherwise $f$ is equal to zero everywhere), and reducing our argument to the one dimensional case, we see that infinitely many points on this line cannot be zeroes of $f$. Arguing similarily, given any $f \in K[X_1, \dots, X_n]$, there is a plane upon which $f \neq 0$, and so we must show that any nonconstant $f(X,Y) = \sum a_{ij} X^i Y^j$ in the plane has infinitely many zeroes. For any line $[-a:1]$ through the origin, the polynomial takes the form $f(X,aX) = \sum a_{ij} a^j X^{i+j}$, and we know this polynomial has a zero unless it is constant, and if this occurs, then for any $1 \leq m < \infty$, $\sum a_{(m-k)k} a^k = 0$. Each of these are polynomials in $K[a]$, and at least one of these polynomials is nonzero, so we conclude there can only be finitely many values $a$ such that $[a:1]$ does not have an intersection with $V(f)$, and it follows that $V(f)$ is infinite because $K$ is infinite.
\end{proof}

The generation of an ideal $I(X)$ from a set $X$ is dual to the notion of generating a set $V(\mathfrak{a})$ from an ideal $\mathfrak{a}$. We make a few elementary observations about this operator.
%
\begin{itemize}
    \item It is clear that if $X \subset Y$, then $I(Y) \subset I(X)$.
    \item $I(\emptyset) = K[X_1, \dots, X_n]$, and $I(\mathbf{A}^n) = (0)$.
    \item $S \subset I(V(S))$ for any subset $S$ of polynomials, and $X \subset V(I(X))$.
    \item Combining the last two points, it follows that $V(I(V(S))) = V(S)$, because $V(S) \subset V(I(V(S)))$ follows from the second point of the last bullet, and $V(I(V(S))$ follows because $S \subset I(V(S))$, hence $V(S) \supset V(I(V(S))$. Similarily, we can argue that $I(V(I(X))) = I(X)$. Thus if $X$ is an algebraic set, then $V(I(X)) = X$, and if $\mathfrak{a}$ is an ideal equal to $I(X)$ for some set $X$, then $I(V(\mathfrak{a})) = \mathfrak{a}$.
    \item If $f^n \in I(X)$, then $f^n(p) = 0$ for all $p \in X$, which implies $f(p) = 0$ because $K$ is an integral domain, so that $f \in I(X)$. This means exactly that $I(X)$ is a {\it radical ideal}.
\end{itemize}

\begin{prop}
    For any two algebraic sets $V$ and $W$, $I(V) = I(W)$ if and only if $V = W$.
\end{prop}
\begin{proof}
    This follows because $V(I(V)) = V$, and $V(I(W)) = W$, so if $I(V) = I(W)$, then $V = V(I(V)) = V(I(W)) = W$.
\end{proof}

This is clearly not true if $V$ and $W$ are not algebraic sets. For instance, if $Y$ is the closure of some open set $X$, then $I(X) = I(Y)$, because polynomials are continuous so if they vanish on $X$, they certainly vanish on $Y$.

\begin{corollary}
    Show that if $V$ is an algebraic set in $\mathbf{A}^n$, and $p \not \in V$, then there is a polynomial $f$ which vanishes on $V$, but with $f(p) = 1$.
\end{corollary}
\begin{proof}
    Since $V \neq V \cup \{ p \}$, and $V$ and $V \cup \{ p \}$ are both algebraic sets, $I(V) \neq I(V \cup \{ p \})$, and since $I(V) \supset I(V \cup \{ p \})$, there must be a polynomial $f$ which vanishes on $V$, but with $f(p) \neq 0$. It follows by normalizing that we can assume $f(p) = 1$. Similarily, by taking an algebraic set $V$, and $n$ points $p_1, \dots, p_n \not \in V$, we may apply this theorem to find polynomials $f_1, \dots, f_n \in I(V)$ with $f_i(p_j) = \delta_{ij}$. By considering linear combinations of the $f_i$, for any $a_{ij} \in K$, we can find $f_1, \dots, f_n \in I(V)$ with $f_i(p_j) = a_{ij}$.
\end{proof}

Now you might wonder the extent to which the methods of algebraic geometry work. Given an infinite family of polynomials $S$, the algebraic set $V(S)$ might be much more difficult to understand than the algebraic set $V(S)$ obtained where $S$ is a finite set. However, the ideal theory of $K[X_1, \dots, X_n]$ (specifically, Hilbert's basis theorem) tells us that every ideal in $K[X_1, \dots, X_n]$ is finitely generated. In terms of algebraic geometry, this means that {\it every algebraic set is the intersection of finitely many hyperplanes}, because every algebraic set is specified by some ideal $\mathfrak{a}$, and we can write $\mathfrak{a} = (f_1,\dots,f_n)$, so that $V(\mathfrak{a}) = \bigcap V(f_i)$ is the intersection of hyperplanes.

\section{Reducibility}

An algebraic variety $V$ is said to be {\bf reducible} if it can be written as the union of two proper algebraic subsets (these subsets need not be reducible). Otherwise, we say $V$ is {\bf irreducible}. Ring theory allows us to characterize this criterion in terms of the ideal generating the ideal.

\begin{prop}
    A variety $V$ is irreducible if and only if $I(V)$ is prime.
\end{prop}
\begin{proof}
    Suppose that $I(V)$ is not prime, so $fg \in I(V)$, whereas $f \not \in I(V)$, $g \not \in I(V)$. It follows that $f$ cannot be a scalar multiple of $g$, because $I(V)$ is a radical ideal. The fact that $f \not \in I(V)$ and $g \not \in I(V)$ means that $f$ and $g$ do not vanish on $V$, so $V(f, I(V))$ and $V(g, I(V))$ are proper subsets of $V$. But $V(f, I(V)) \cup V(g, I(V)) = V(fg, I(V)) = V(I(V)) = V$, so $V$ is reducible. Conversely, if $V = W \cup U$, where $W$ and $U$ are proper algebraic subsets of $V$, then $I(V)$ is a proper subset of both $I(W)$ and $I(U)$, so we may select $f$ vanishing on $W$, but not on $V$, and $g$ vanishing on $U$, but not on all of $V$. This means that $fg$ vanishes on $W \cup U = V$. Thus we have found $f,g \not \in I(V)$, but with $fg \in I(V)$, so $I(V)$ cannot be prime.
\end{proof}

\begin{example}
    The parabola $V(Y - X^2)$ is an irreducible variety. First, we must justify that $I(V(Y - X^2)) = (Y - X^2)$. If $f(X,Y) = \sum a_{ij}X^iY^j$ is an element of $I(V(Y - X^2))$, then $f(X,X^2) = \sum a_{ij} X^{i + 2j} = 0$

    Then we prove that $(Y - X^2)$ is prime, and since $K[X_1, \dots, X_n]$ is a unique factorization, it suffices to show that $Y - X^2$ is an irreducible polynomial. If $Y - X^2$ is the product of two polynomials, write these two polynomials as $Yf(X) + g(X)$ and $h(X)$ (if there are more $Y$'s in the factorization, they clearly cannot multiply to $Y - X^2$), where $f$ is monic. But then $(Yf + g)(h) = Yfh + gh$, so $fh = 1$, implying that $h$ is a unit.
\end{example}

As in most of mathematics, irreducible varieties have a nice theory, and we can use this theory to understand the varieties obtainable from the union of irreducible varieties. The idea is simple. If a variety $V$ is not irreducible, then we can break it apart into two proper algebraic subsets $V_1 \cup W_1$. If $V_1$ is not irreducible, we can break it apart into two proper subsets $V_2 \cup W_2$. If this process is guaranteed to terminate at some point (so that $V_n$ is eventually irreducible), we can recursively break apart varieties into irreducible varieties. The ring theoretic property we need to employ here is the fact that $K[X_1, \dots, X_n]$ is a {\it Noetherian ring} -- every ascending chain of ideals is guaranteed to terminate.

\begin{prop}
    Every variety is the finite union of irreducible varieties.
\end{prop}
\begin{proof}
    If this theorem did not hold, we have justified that we can find an infinite sequence $V_1 \supsetneq V_2 \supsetneq V_3 \dots$ of descending algebraic subsets. This implies that $I(V_1) \subsetneq I(V_2) \subsetneq I(V_3)$, an infinite ascending chain of ideals. Because $K[X_1, \dots, X_n]$ is Noetherian, this situation cannot occur, so $V_n$ must eventually be an irreducible variety, and this implies that the process of breaking reducible varieties in a decomposition must eventually yield a set of irreducible varieties.
\end{proof}

The last proposition guarantees the existence of a decomposition of an arbitrary variety $V$ into a finite union $\bigcup V_i$ of irreducible varieties. If $V_j \subset V_k$ for $j \neq k$, then we may remove $V_j$ from the union, and we still obtain a decomposition of $V$. We may therefore assume that no element of the decomposition is a subset of any other. Once we assume this, we obtain a unique decomposiiton. Suppose we have $\bigcup V_i = \bigcup W_i$, for two families of irreducible varieties, where none of the $V_i$ is a subset of the $V_j$, and none of the $W_i$ is a subset of the $W_j$. Then for each $i,j$, $V_i = (W_j \cap V_i) \cup (\bigcup_{k \neq j} W_k \cap V_i)$, and these finite intersections form varieties, so either $W_j \cap V_i = \emptyset$, or $W_j \cap V_i = V_i$. If $W_j \cap V_i = \emptyset$ for all $j$, then $V_i = V_i \cap \bigcup V_i = V_i \cap \bigcup W_i = \bigcup V_i \cap W_i = \emptyset$, which we assumed was impossible. Thus $V_i \subset W_j$ for some $j$. Similarily, we may apply this technique to conclude that $W_j \subset V_k$ for some $k$, and by assumption, we must have $k = i$, so $W_j = V_i$. By matching up elements of the decomposition, we conclude that the decomposition is unique. The elements of this decomposition are known as the {\bf irreducible components} of $V$.

\end{document}