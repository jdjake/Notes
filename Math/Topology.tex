\documentclass{report}

\usepackage{kpfonts}
\usepackage{amsmath}
\usepackage{amssymb}

\usepackage{amsthm}
\usepackage{framed}

\theoremstyle{plain}
\newtheorem{theorem}{Theorem}[section]
\newtheorem{lemma}[theorem]{Lemma}
\newtheorem{corollary}[theorem]{Corollary}
\newtheorem{exercise}{Exercise}[section]

\theoremstyle{definition}
\newtheorem*{defi}{Definition}
\newenvironment{definition}
    {\begin{samepage}\begin{framed}\begin{defi}}
    {\end{defi}\end{framed}\end{samepage}}

\usepackage{hyperref} 
\hypersetup{
    colorlinks = true,
    linkcolor = black,
}

\renewcommand*\contentsname{\hfill Table Of Contents \hfill}

\title{Topology}
\author{Jacob Denson}

\begin{document}

\pagenumbering{gobble}

\maketitle

\tableofcontents

\pagenumbering{arabic}

\chapter{Analogies and Introductions}

Metaphor is at the heart of modern mathematics. Given a complex abstract problem, it is difficult to see beyong the logic of the situation. Via abstraction, we obtain analogies laden with intuition. Why try and see an infinite dimensional vector space, when we can see it as something similar to the three dimensional cartesian space we have spent our life in. In the context of number theory, linear algebra, and algebraic geometry, abstract algebra helps us with intuition. In analysis, Topology is king, generalizing that intuitive notion of space which encloses calculus, differential geometry, functional, real, and complex analysis under a single umbrella.

Seeing as how general topology arose from a diverse range of fields, it is unsurprising to see that a multitude of definitions arose to define what a topology was. The following definition, perhaps the most natural (or the most common) is due to the German mathematician Georg Cantor. Though it is not the only way to define what we mean by a topology, it is by far the most common.

\section{Space, and other delights}

I'm sure that, at one point in your childhood, you owned a colouring book. In this tome of black and white, you delighted in colouring between lines that define a figure. Shapes emphasized by a black silouhette are easily distinguished and filled by a nice blue, green, or pink fluorescent marker. Mathematically, we distinguish between shapes which contain their silouhette and shapes that do not. Those shapes that do contain their `boundary', as it is formally called, are called closed. The shapes that contain none of their boundary are open. Consider the interior of a circle (not including an outline). At any point on this shape, there is an infinitude of points from your pen tip to the edge of the circle. Sets with this sort of property are called open. The foundations of topology are obtained by abstracting the qualities that these `edgeless' shapes possess.

\begin{definition}
    A {\bf topological space} is a set $X$, together with a {\bf topological structure} $\Omega$, a designated collection of subsets, restricted by the following properties:
    %
    \begin{enumerate}
        \item Both $\emptyset$ and $X$ itself are in $\Omega$
        \item If $\{A_i\}_{i \in I}$ is an arbitrary family such that $A_i \in \Omega$ for each $i$, then their union $\bigcup_{i \in I} A_i$ is also in $\Omega$.
        \item If $\{A_i\}_{i = 1}^n$ is a finite collection of open sets, then their intersection $\bigcap_{i = 1}^n A_i$ is also open.
    \end{enumerate}
    %
    We apologize for the onslought of terminology which now follows: to learn a field, you must first speak its language.
    %
    \begin{itemize}
        \item Elements of a topological space are called points.
        \item We say a subset of a topological space is open if it is contained in the topological structure.
        \item A subset is closed is closed if its complement is open.
        \item A neigbourhood of a point $x \in X$ is an open set containing $x$\footnote{Analysts and French mathematicians extend the notion of a neighbourhood, defining it to be a set containing a neighbourhood in our sense. When reading other works of topology, make sure you remember which definition the author is using.}.
        \item A limit point of a subset $U$ of $X$ is a point $p$ such that every open set containing $U - \{p\}$ contains $p$. The set of limit points of a subset $U$ is denoted $U'$, and is called the derived set of $U$.
        \item If we have two topological structures $\Delta$ and $\Omega$, then we say $\Delta$ is finer than $\Omega$, or $\Omega$ is coarser, if $\Omega \subset \Delta$. Spaces are comparable in the same way that rocks, pebbles, and grains of sand are.
        \item A function $f:X \to Y$ between two topological spaces is continuous if $f^{-1}(U)$ is open for every open set $U$ in $Y$. A homeomorphism is a continuous bijective function whose inverse is continuous.
    \end{itemize}
    %
    All terminology will be expanded fully later on. This is just to provide a reference.
\end{definition}

Learn the concept of open-ness. It is the most crucial idea in topology; everything else we shall deal with is defined in terms of open sets and the qualities they possess amongst themselves. This is why all we need for a topology to form is a topological structure $\Omega$ of open sets, known universally as `the topology' of the space.

Topology also defines additional axioms which ensure your spaces are more reasonable. If your space does not have at least the very low level properties, then be sure that non-intuitive things will occur:

\begin{definition}
Additional Topological axioms which will come in handy later:
%
\begin{itemize}
    \item A topological space is Kolmogorov, or T0, if, for any points $x \neq y$, either $x$ has a neighbourhood not containing $y$, or $y$ has a neighbourhood not containing $x$. Almost every Topological space has this property.
    \item A topological space is Frechet, or T1 if, for any points $x \neq y$, we may find a neighbourhood of $x$ not containing $y$.
    \item A topological space is Hausdorff, or T2 if, for any points $x \neq y$, we may find disjoint open sets $U$ and $V$ such that $x \in U$, $y \in V$.
    \item A topological space is Tychonoff/Regular, or T3, if it is Frechet, and when $C$ is a closed set not containing a point $x$, we may find disjoint open sets $U$ and $V$, such that $C \subset U$, and $x \in V$.
    \item A topological space is normal, or T4, if it is Frechet and if two disjoint closed sets $C$ and $D$ can be covered by disjoint open neighbourhoods $U$ and $V$, such that $C \subset U$, $D \subset V$.
\end{itemize}
%
What is nice about our formulation is that the axiom $TA$ implies $TB$ if $A \geq B$. We leave this as an exercise to verify.
\end{definition}

We defined openness as above because these properties hold for the colouring book topology above. We define a shape to be open if we can never colour up to an edge. A shape (subset of the plane) is open if we may draw a circle around every point in the shape, and all points in the interior of the circle are contained in the shape itself. Intuition should tell you why the properties of open sets hold for this space -- we shall define this precisely later. In this topology, the finite assumption of open intersections is crucial. If we take the intersection of an infinite number of open sets, we can no longer colour in until we reach one boundary -- the boundary may have been stretched too thin. Consider the intersection of the `infinite venn diagram below'.

INFINITE CIRCLES INTERSECTING ONLY AT A POINT.

Given any set $X$ there are two spatial extremities we can use to form a topology. We either choose the minimal number of open sets, or the maximal number. The `discrete topology' on $X$ lets the topological structure $\Omega$ be equal to the power set $\mathcal{P}(X)$. In this case, every subset of the plane is open, and the axioms of trivially satisfied. The `lumpy topology' on $X$ has minimum structure, with topological structure $\Omega' = \{\emptyset, X\}$.

\begin{theorem}
    The discrete topology $\Omega$ makes $X$ a topological space, for any set $X$. The lumpy topology $\Omega'$ also makes $X$ a topological space.
\end{theorem}
\begin{proof}
    Both $\emptyset$ and $X$ are subsets of $X$, so they are both contained in $\Omega$. The other two axioms follow trivially, because if $A_i$ is a subset of $X$ for each $i$ in $I$, then set theory tells us that both $\bigcup_{i \in I} A_i$ and $\bigcap_{i \in I} A_i$ are also subsets of $X$. In this case, the finite assumption of $I$ is not needed to show intersections are open.

    Now lets verify the topological nature of $\Omega'$. By definition, $\emptyset$ and $X$ are elements of $\Omega'$. If $A_i$ is in $\Omega'$ for each $i$ in an indexing set $I$, then we either have $A_i = \emptyset$ or $A_i = X$. If all $A_i$ are equal to $\emptyset$, then
    %
    \[ \bigcup_{i \in I} A_i = \bigcup_{i \in I} \emptyset = \emptyset \in \Omega' \]
    %
    If there is any $A_i = \emptyset$, then
    %
    \[ \bigcap_{i \in I} A_i = \emptyset \in \Omega' \]
    %
    If there is some $A_i$ equal to $X$, then, since each $A_i \subset X$, we know
    %
    \[ \bigcup_{i \in I} A_i = X \in \Omega' \]
    %
    If all $A_i$ are equal to $X$, then
    %
    \[ \bigcap_{i \in I} A_i = X \in \Omega' \]
    %
    We have exhausted all cases for the indexing set.
\end{proof}

The discrete topology is named based on how closeness factors into topological spaces. `limit points', as we have defined them above, indicate points which reach infinitely close to a given set. This is as close to distance as we can get in topology -- we cannot tell if two points are miles or metres away, but only if they are infinitisimally close. The lumpy topology is named because topologically, every point is a limit point of every set (except in the case where the set is empty).

\begin{exercise}
    Let $X$ be the ray $[0,\infty)$, and let $\Omega$ consist of $\emptyset$, $X$, and all rays $(a,\infty)$ with $a \geq 0$. Prove that $\Omega$ is a topological structure.
\end{exercise}
\begin{proof}
    By definition, $\emptyset$ and $X$ are open in the topology. If $A = (a, \infty)$ and $B = (b, \infty)$ are two open sets in $\Omega$, then $A \cap B = (\max(a,b), \infty)$. By induction, arbitrary intersections are in the topology. If $\{(a_i, \infty]\}_{i \in I}$ is a non-empty family of open sets, then $\{a_i\}_{i \in I}$ is a non-empty set of real numbers bounded below (since $a_i \geq 0$). By the completeness property of $\mathbf{R}$, we therefore have an infininum $a = \inf_{i \in I} a_i$. We contend $(a, \infty) = \bigcup (a_i, \infty)$. If $x > a$ ($x \in (a,\infty)$), then $x > a_i$ for some $i$, since otherwise $x$ is a greater lower bound of $\{a_i\}$ than $a$. Therefore $x \in (a_i, \infty)$ for some $a_i$, so $(a,\infty) \subset \bigcup (a_i,\infty)$. Conversely, since $a \leq a_i$ for all $a_i$, $(a_i,\infty) \subset (a, \infty)$ and, taking the union, $\bigcup_{i \in I} (a_i, \infty) \subset (a, \infty)$. If a family of open sets contains $X$ or $\emptyset$, the outcome of the union is trivial. We have shown the intersection and union property hold in full.
\end{proof}

\begin{exercise}
    Let $X$ be a plane. Let $\Sigma$ consist of $\emptyset$, $X$, and all open disks with center at the origin. Verify this is a topological structure.
\end{exercise}
\begin{proof}
    We have $\emptyset$ and $X$ in $\Sigma$. Let $D(\varepsilon)$ denote the disk or radius $\varepsilon > 0$ centred at the origin. If $\{D(\varepsilon_i)\}_{i \in I}$ is a family of discs, then either
    %
    \begin{enumerate}
        \item $\{\varepsilon_i\}_{i \in I}$ is unbounded above. In this case we have $\bigcup D(\varepsilon_i) = X$, since any $x \in X$ is in some $D(\varepsilon_i)$.
        \item $\{\varepsilon_i\}_{i \in I}$ is bounded above. Consider $\varepsilon = \sup \varepsilon_i$. We will show $D(\varepsilon) = \bigcup_{i \in I} D(\varepsilon_i)$. Each $D(\varepsilon_i)$ is a subset of $D(\varepsilon)$, so $\bigcup_{i \in I} D(\varepsilon_i) \subset D(\varepsilon)$. Conversely, if $x \in D(\varepsilon_i)$, then $d(x,0) < \varepsilon$, so there is some $\varepsilon_i$ with $d(x,0) < \varepsilon_i$.
    \end{enumerate}
    %
    If $D(\varepsilon)$ and $D(\varepsilon')$ are two open discs, then their intersection is $D(\min(\varepsilon, \varepsilon'))$.
\end{proof}

The most important example of a topological space is $\mathbf{R}$, the space of real numbers. Many other examples will stem from this space. The topological structure consists of the empty set, and unions of intervals $(a,b)$, where $a < b$ can be infinite. We have $\mathbf{R} = (-\infty, \infty)$. Its a bit fiddly, but you should be able to show that the intersection of two intervals is the union of intervals, and thus that the intersection property holds in general. This topology is the standard topology on $\mathbf{R}$.

\begin{exercise}
    Show that every open set in $\mathbf{R}$ can be broken into the disjoint union of open intervals. Bonus points: Show that this union is countable!
\end{exercise}
\begin{proof}
    Let $U$ be an open set in $\mathbf{R}$, equal to $\bigcup_{i \in I} (a_i, b_i)$ for some indexing set $I$. Let $x \in U$. Consider $A_x = \{ a \in \mathbf{R} : (a,x)] \subset U \}$ and $B_x = \{ b \in \mathbf{R} : [x,b) \subset U$. Surely this set is not empty, since $x \in (a_i, b_i)$ for some $i$, and thus $a_i \in A_x$, $b_i \in B_x$. Consider $a_x = \inf A$, and $b_x = \sup B$, finite or infinite. We contend that $(a_x,b_x)$ is a neighbourhood of $x$ contained in $U$. Surely $x \in (a_x,b_x)$, for $a < x < b$. Furthermore, if $y \in (a_x,b_x)$, then $a_x < y < b_x$, and there is some $(a_i, b_i)$ such that $a_i < y < b_i$, so $y \in U$. Form the set of intervals $\{ (a_x, b_x): x \in U\}$. We claim that this is a pairwise disjoint set of intervals whose union is $U$. If $(a_x, b_x)$ and $(a_y, b_y)$ are non disjoint, consider an element $z$ in their intersection. Then $a_x, a_y \in A_z$, $b_x, b_y \in B_z$, so $(a_z, b_z)$ contains both intervals. Namely, $(a_z, b_z)$ contains $x$ and $y$, so $a_z \leq a_x, a_y$, and $b_z \leq b_x, b_y$. But also $a_z \geq a_x, a_y$, and $b_z \geq b_x, b_y$, since $a_z \in A_x, A_y$, and $b_z \in B_x, B_y$. By a simple manipulation of inequalities, we have shown that $a_x = a_z = a_y$, and $b_x = b_z = b_y$, so that intervals are pairwise disjoint, or equal.

    The fact that the union is countable stems from the fact that every interval contains a rational which we may uniquely identify with the interval in the case that the union is of disjoint intervals.
\end{proof}

Closed sets of $\mathbf{R}$ are not so simple to classify. Later on, we will see the Cantor set, a very strange closed set of $\mathbf{R}$.

\begin{exercise}
    Let $X$ be a set, and let $\Omega$ consist of all subsets of $X$ whose complement is finite, and $\emptyset$. Show that $\Omega$ is a topological structure, called the $T_1$ topology.
\end{exercise}
\begin{proof}
    If $X - A$ is finite, and $X - B$ is finite, then $X - (A \cap B) = (X - A) \cup (X - B)$ is a finite union of finite sets, hence finite. If $X - A$ is finite, and $B$ is any other subset, then $X - (A \cup B) \subset X - A$ is finite.
\end{proof}

This example should show us that our definition of topology is not trivial in the slightest.

\begin{exercise}
    Consider the set $\mathbf{Z}$, and let $\Omega$ be the set of all unions of arithmetical sequences of the form $a\mathbf{Z} + b = \{ ax + b: x \in \mathbf{Z}\}$, where $a \neq 0$. Show that $\Omega$ is a topological structure, and that there are infinitely many prime numbers.
\end{exercise}
\begin{proof}
    Since every integer is in some arithmetic sequence, we know that $\mathbf{Z}$ itself is open, as is the empty set. Here is an arbitrary intersection of two sequences
    %
    \[ [a\mathbf{Z} + b] \cap [m\mathbf{Z} + n] \]
    %
    If $x$ can be written $ay + b$ and $mz + n$ via two integers $y$ and $z$, then $x + am\mathbf{Z}$ is also contained in the intersection. so that the intersection of two sequences is the union of other arithmetical sequences. Hence the set of all union of arithmetical sequences specifies a topology.

    This topology is a very strange one. Consider the factorial sequence
    %
    \[ 1!,\ 2!,\ 3!, \dots = 1,\ 2,\ 6,\ 24,\ 120,\ 720, \dots \]
    %
    Any open set containing $0$ contains some arithmetical sequence $a\mathbf{Z}$, and the factorial sequence will eventually end up in this sequence since the sequence accumulates all integers as factors. Therefore this sequence is `infinitely close' to 0 in this toplogy, even though in the canonical topology on the integers the sequence is one of the one that grows faster than any other function. The main importance of this topological space is that it allows us to show there are infinitely many primes. Suppose there are only finitely many, which we may write as $\{ p_1, p_2, \dots, p_n \}$. Then
    %
    \[ A = \bigcap_{k = 1}^n p_k\mathbf{Z} \]
    %
    is the finite intersection of open sets, and is thus open. Since every non-zero non-unit integer can be written as the product of primes, it follows that the complement $A^c = \{ -1,0,1 \}$. In our topology, the complement of open sets is open (check this via the basis elements), so that $A^c$ is the union of open sets, which is clearly not true. Hence there must be infinitely many primes.
\end{proof}

\section{Closed topological sets}

A set in a topological space $X$ is closed if it is the complement of an open set. A closed interval $[a,b]$ is closed in $\mathbf{R}$, since $[a,b]^c = (-\infty, a) \cup (b,\infty)$. To specify the axioms of a topology, we could have taken as a primitive notion closedness instead of openness. Since
%
\[ \big( \bigcup_{i \in I} A_i \big)^c = \bigcap_{i \in I} A_i^c \]
%
\[ \big( \bigcap_{i \in I} A_i \big)^c = \bigcup_{i \in I} A_i^c \big) \]
%
we could have taken a topology as a collection of closed sets such that arbitrary intersections and finite unions of closed sets are closed.

\begin{exercise}
    Prove that $[0,1)$ is not closed nor open in $\mathbf{R}$, yet it is the union of closed sets and the intersection of open ones.
\end{exercise}
\begin{proof}
    Suppose $[0,1)$ was open, so it is equal to the union of intervals $\bigcup (a_i,b_i)$. We must have $0 \in (a_i, b_i)$ for some $i$, but then $a_i < 0$, and thus $a_i/2 < 0 \in [0,1)$, a contradiction. Conversely, suppose $[0,1)$ was closed, so $(-\infty, 0) \cup [1,\infty)$ was open. A similar argument to the one above shows that this is impossible.

    We have
    %
    \[ [0,1) = \bigcup_{n = 1}^\infty \left[0, 1 - \frac{1}{n} \right] = \bigcap_{n = 1}^\infty (-\frac{1}{n}, 1) \]
    %
    So we cannot take arbitrary unions of closed sets nor intersections of open sets.
\end{proof}

\begin{exercise}
    Prove that $\{0\} \cup \{ \frac{1}{n} : n \in \mathbf{Z} \}$ is closed in $\mathbf{R}$.
\end{exercise}
\begin{proof}
    \[ \left[\{0\} \cup \left\{ \frac{1}{n} : n \in \mathbf{Z} \right\}\right]^c = (-\infty, 0) \cup (1, \infty) \cup \left(\frac{1}{2},1\right) \cup \left(\frac{1}{3},\frac{1}{2}\right) \cup \dots \]
    %
    The complement is open, as it is just the union of open sets.
\end{proof}

Closed sets are seen as `containing the points' infinitely close to each other. Here we establish what exactly this means. Recall that $x$ is a limit point of a set $A$ if every open set containing $x$ contains some point in $A$.

\begin{exercise}
    If $C$ is a closed set if and only if set of limit points $C'$  is a subset of $C$.
\end{exercise}
\begin{proof}
    Let $C$ be a closed set. Then $C^c$ is open, so if $x$ is not in $C$, $x$ is not a limit point of $C$. We have shown what we needed to show. Now suppose $C'$ is contained in $C$, where $C$ is an arbitrary set. If $x$ is not an element of $C$, then it follows that there is an open set $U_x$ containing $x$ not containing any point in $C$. The union
    %
    \[ \bigcup_{x \notin C} U_x \]
    %
    is an open set containing no elements of $C$, and any point not in $C$. Thus the union is just $C^c$, and we have shown $C^c$ is open, hence $C$ is closed.
\end{proof}

\subsection{An excursion -- Cantor's set}

This excursion shows us that weird closed sets exist even in the most simple topologies -- here the real numbers $\mathbf{R}$.

\begin{exercise}
    Consider the Cantor set $K$ defined below:
    %
    \[ K = \left\{ x \in \mathbf{R} : x = \sum_{k = 1}^\infty \frac{a_k}{3^k} \ \text{with}\ a_k = 0\ \text{or}\ 2 \right\} \]
    %
    Show that $K$ can be inductively defined as $\bigcap_{i = 1}^\infty K_i$, where $K_0 = [0,1]$, and
    %
    \[ K_{i+1} = \frac{K_i}{3} \cup \left(\frac{2}{3} + \frac{K_i}{3}\right) \]
\end{exercise}
\begin{proof}
    Let us show first that $K \subset \bigcap_{i = 1}^\infty K_i$. Here is an arbitrary decimal expansion in the Cantor set
    %
    \[ x = \sum_{k = 1}^\infty \frac{a_k}{3^k} \]
    %
    where each $a_k = 0$ or $2$. Then $x \in K_0 = [0,1]$, since
    %
    \[ 0 = \sum_{k = 1}^\infty \frac{0}{3^k} \leq \sum_{k = 1}^\infty \frac{a_K}{3^k} \leq \sum_{k = 1}^\infty \frac{2}{3^k} = 2(\frac{1}{1 - 1/3} - 1) = 1 \]
    %
    For an induction, assume $K \subset K_n$. Suppose $a_1 = 0$. Then
    %
    \[ x = \frac{a_1}{3} + \frac{a_2}{3^2} \dots = \frac{1}{3} (a_2 + \frac{a_2}{3} + \dots) = y/3 \]
    %
    where $y \in K$, so $y \in K_n$ by hypothesis. Then $x \in K_n/3 \subset K_{n+1}$. If $a_1 = 2$,
    %
    \[ x = 2/3 + \frac{1}{3}(\frac{a_2}{3} + \frac{a_3}{3^2} + \dots) = 2/3 + y/3 \]
    %
    where $y \in K$. Therefore $y \in K_n$, and $x \in 2/3 + K_n/3 \subset K_{n+1}$. This exactly proves that $K \subset K_{n + 1}$. By induction, we have shown $K \subset \bigcap_{i = 1}^\infty K_i$.

    Now we must show, conversely, that $\bigcap K_i \subset K$. Let $x \in \bigcap K_i$. We will build a correct sequence of digits that converge to $x$. Let $x_0 = 0$, and
    %
    \[ x_n = \max \left\{ a_k \in \{0,2\} : \sum_{k = 0}^{n-1} x_k/3^k + a_k/3^n \leq x \right\} \]
    %
    Our first claim:
    %
    \[ \sum_{k = 0}^\infty x_k/3^k = x \]
    %
    We shall prove by induction that if $x \in K_n$, then $x - \sum_{k = 1}^\infty x_k/3^k < 1/3^n$, If $x$ is in all $n$, then $x - \sum_{k = 1}^\infty x_k/3^k$ must be equal to zero, since it is smaller than $1/3^n$ for all $n$, and greater than or equal to zero. This is trivially true for $K_0$, since any sum is between 0 and 1, and $x$ is also between 0 and 1. Assume this is true for all $y \in K_n$. Consider $x \in K_{n+1}$. Assume that there is $y = 3x$ in $K_n$. Then $x_1 = 0$ (since otherwise $x > 1/3$, and $y > 1$), and if we consider the expansion of $y$, we will see that $y_n = x_{n+1}$. This follows because
    %
    \[ \sum_{k = 1}^{n-1} y_k/3^{k+1} + a/3^n = \sum_{k = 1}^{n} x_k/3^k + a/3^{n+1} \leq x \]
    %
    holds if and only if the inequality
    %
    \[ 3\left(\sum_{k = 1}^{n-1} x_k/3^k + a_k/3^n\right) = \sum_{k = 1}^{n-1} y_k/3^k + a_k/3^n \leq 3x = y \]
    %
    holds, assuming for an induction that $y_k = x_{k+1}$ for $k \leq n-1$. If
    %
    \[ y - \sum_{k = 1}^\infty y_k/3^k < 1/3^n \]
    %
    then $y/3 - \sum_{k = 1}^\infty y_k/3^{k+1} = x - \sum_{k = 1}^n x_k/3^k < 1/3^{n+1}$. If $x = 2/3 + y$, where $y \in K_n$, the same technique establishes the inequality. Our claim is thus proved.
\end{proof}

\begin{exercise}
    Show that the Cantor set $K$ is closed in $\mathbf{R}$.
\end{exercise}
\begin{proof}
    We will show each $K_i$ is closed. Obviously $K_0 = [0,1]$ is closed. If $K_n$ is closed, then $K_{n+1} = K_n/3 \cup 2/3 + K_n/3$ is the union of two closed sets, since the maps $x \to x/3$ and $x \to x + 2/3$ are homeomorphisms of the real line (which we will define later), and map closed sets to closed sets, the claim is proved.
\end{proof}

\section{The Basis of a space}

Normally, a topology is not given via specifying every single open set in the topology. Since open sets are constructed from other open sets, we may specify some archetypal sets, and provided these sets are sufficient to describe a topology, define a topology in terms of them.

Let $X$ be a set, and $\{\Omega_i\}_{i \in I}$ a family of topological structures on $X$. Consider the intersection of all structures,
%
\[ \Delta = \bigcap_{i \in I} \Omega_i \]
%
Surely $\emptyset$ and $X$ itself are elements of $\Delta$, since they are an element of each $\Omega_i$. If $\{A_i\}_{i \in I} \in \Omega_i$ for each $i$, then surely $\bigcup_{i \in I} A_i \in \Omega_i$, so that if $\{A_i\}_{i \in I} \in \Delta$, we also have $\bigcup_{i \in I} A_i \in \Delta$. Similarily, if $\{A_i\}_{i \in I} \in \Delta$, then $\bigcap_{i = 1}^n A_i \in \Delta$, so that $\Delta$ is a topological space. This fact will allow us to generate structures from generating sets, as is done in many areas of mathematics.

If $X$ is a set, and $D$ is a family of subsets, then we may consider
%
\[ \Delta = \{ \Omega \in \mathcal{P}(X) : D \subset \Omega\ \text{and D is a topological space} \} \]
%
Taking $\Phi = \bigcap \Delta$, we obtain the coarsest topological space containing $D$, called the topology generated by $D$. If $\Omega$ is a topological space containing $D$, then $\Phi \subset \Delta$ also, by the construction above.

When $D$ has certain nice properties, the topological space generated by $D$ is much simpler to work with:

\begin{definition}
    Let $D$ be a subset of a topological space $X$. Suppose that
    %
    \begin{enumerate}
        \item $\emptyset, X \in D$.
        \item If $A$ and $B$ are in $D$, and $A \cap B$ is non-empty, containing a point $x$, then there is a set $C$ in $D$ containing $x$ with $C \subset A, B$.
    \end{enumerate}
    %
    In this case, the topological space $\Phi$ generated by $D$ has a following property. A set $U$ is open in $\Phi$ if and only if it is the union of sets in $D$. We call $D$ a {\bf basis} for the topology $\Phi$ if it satisfies properties (1) and (2), and a {\bf subbasis} if it is just a generating set.
\end{definition}

\begin{exercise}
    Can two distinct topological structures have the same base? That is, does the base of a topology uniquely define a topology.
\end{exercise}
\begin{proof}
    No. Let $X$ be a set, and $\Delta$, $\Omega$ two topological structures with the same base. Then $\Delta$ must be finer than $\Omega$, since $\Delta$ is the minimal structure containing the base. Similarily, $\Omega$ must be finer than $\Delta$, so we conclude the two structures are equal.
\end{proof}

\begin{exercise}
    Prove that there is no minimal topological basis for $\mathbf{R}$.
\end{exercise}
\begin{proof}
    Let $D$ be a basis for $\mathbf{R}$, and let $U \in D$ be a set in $D$ not equal to $\emptyset$ or $X$, which is therefore open in $\mathbf{R}$ and contains an interval $(a,b)$. Consider $D - U$. The first property of a basis is satisfied, and if $x$ is contained both in two subsets $A$ and $B$ in $D$, and if it were the case that $x \in U \subset A \cap B$ (without loss of generality, $x \in (a,b)$), then since $(a + \varepsilon,b - \varepsilon)$ is the union of sets in $D$, there must be some $U' \in D - U$ containing $x$, since $(a + \varepsilon, b - \varepsilon) \subsetneq U$. Thus $D - U$ is a basis, and generates a topological space. Obviously, $D$ is finer than $D -U$. Let $U = \bigcup (a_i, b_i)$. There exists $\delta$ such that, for each $0 < \varepsilon < \delta$, $(a_i + \varepsilon, b_i - \varepsilon)$ is open in $\mathbf{R}$ and is not equal to $U$, so $(a_i + \varepsilon, b_i - \varepsilon) = \bigcup U_i$, for some $U_i \in D - U$. But then
    %
    \[ U = \bigcup (a_i, b_i) = \bigcup_{i \in I} \bigcup_{\delta > \varepsilon > 0} \bigcup_{j \in J_i} U_{i,\varepsilon,j} \]
    %
    And $U$ is still open in the topology generated by $D - U$.
\end{proof}

A basis for a discrete space $X$ is just $X$ itself, since any open set in a discrete space is the union of some points in $X$. A basis for $\mathbf{R}$ is the set of all intervals $(a,b)$.

\begin{exercise}
    Show that two bases $D$ and $D'$ generate the same topological structure if every element of $D'$ is the union of $D$, and vice versa.
\end{exercise}

\section{Distances}

In topology, we escrew the concept of an exact distance -- we care only about the distance that lies infinitely small between different objects. Nonetheless, distances are more consistant with out intuitions about space, so it is useful to be able to construct topological space using an abstract definition of distance.

\begin{definition}
    A {\bf Metric} on a set $X$ is a function $\rho: X \times X \to \mathbf{R}$ such that
    %
    \begin{enumerate}
        \item (Nondegeneracy) For any $x,y$, $\rho(x,y) \geq 0$, and $\rho(x,y) = 0$ if and only if $x = y$.
        \item (Symmetry) For any $x,y$, $\rho(x,y) = \rho(y,x)$.
        \item (The Triangle Inequality) For any $x, y, z$, $\rho(x,z) \leq \rho(x,y) + \rho(y,z)$.
    \end{enumerate}
    %
    Given a metric, the open ball $B_\varepsilon(x) = \{ y \in X : \rho(x,y) < \varepsilon \} $. If we take the set of all open balls in a topological space (plus the empty set and all of space), we obtain a basis, and we may use this basis to generate a topological space called a {\bf Metric space}.

    We gain even more terminology via introducing a metric.

    \begin{enumerate}
        \item We can consider closed discs $D_r(x) = \{ y \in X : \rho(x,y) \leq r \}$
        \item Sphere in a metric space are $S_r(x) = \{ y \in X : \rho(x,y) = r \}$.
        \item The diameter of a subset $A$ is $\text{diam}(A) = \sup \{ \rho(x,y): x,y \in A \}$. A set is bounded if its diameter is finite.
    \end{enumerate}
\end{definition}

\begin{exercise}
    Show that the set of open balls in a metric space truly is a topological basis.
\end{exercise}
\begin{proof}
    We will verify the intersection property. Let $x \in B_r(y) \cap B_{r'}(z)$, so $\rho(x,y) < r = r - \varepsilon$, and $\rho(x,z) < r' = r' - \varepsilon'$. Without loss of generality, let $\varepsilon \leq \varepsilon'$. Consider the ball $B_\varepsilon(x)$. If $\rho(a,x) < \varepsilon$, then by the triangle inequality,
    %
    \[ \rho(a,y) < \rho(a,x) + \rho(x,y) < \varepsilon + r - \varepsilon = r \]
    \[ \rho(a,z) < \rho(a,x) + \rho(x,z) < \varepsilon + r - \varepsilon' \leq \varepsilon' + r - \varepsilon' = r \]
    %
    So $B_\varepsilon(x)$ is contained in the intersection of both balls, and contains $x$.
\end{proof}

Open balls do not always look like the nice ones in $\mathbf{R}^n$ under the usual metric. Even in the same space with a different metric, the metric
%
\[ \rho(x,y) = \sum_{k = 1}^n |y_i - x_i| \]
%
induces balls shaped like diamonds; the metric
%
\[ \rho(x,y) = \max_{i = 1,\dots,n} |y_i - x_i| \]
%
induces balls shaped like squares whose sides are oriented to the axes.

\begin{exercise}
    Verify that $\rho(x,y) = \delta_{x,y}$ (where $\delta_{x,y}$ is the kronecker delta function) defines a metric on any set, and that the topology the metric generates is the discrete topology.
\end{exercise}
\begin{proof}
    We certainly have nondegeneracy and symmetry. If $z \neq x,y$, then $\rho(x,y) \leq \rho(x,z) + \rho(z,y)$. If $x = y$, then surely the inequality holds by nondegeneracy. If $x \neq y$, then $z$ cannot equal both values, so $\rho(x,z)$ or $\rho(z,y)$ must be one. Since $\rho(x,y) \leq 1$, the inequality holds. Taking the ball $B_{0.1}(x)$ for any point $x$, we obtain that singletons are open, so the topology must be discrete.
\end{proof}

\begin{exercise}
    Show that on $\mathbf{R}$, $\rho(x,y) = |y - x|$ is a metric, and defines the canonical topology of $\mathbf{R}$. For bonus points, verify that on $\mathbf{R}^n$, $\rho(x,y) = \|y - x\|$, $\rho'(x,y) = \max_{i = 1,\dots,n} |y_i - x_i|$, and $\rho''(x,y) = \sum_{i = 1}^n |y_i - x_i|$ all define metrics on the space. These metrics all generate the same topology, yet this is a bit tricky to prove.
\end{exercise}

\begin{exercise}
    One of the most important metrics on $\mathbf{R}^n$ is
    %
    \[ \rho(x,y) = \|y - x\|_p = \left( \sum_{i = 1}^n (y_i - x_i)^p \right)^{1/p} \]
    %
    Use the fact that this is a metric to prove the {\it H\"{o}lder inequality}, that for any $x_i,y_i \geq 0$, and $1/p + 1/q = 1$, we have
    %
    \[ \sum_{i = 1}^n x_iy_i \leq \left( \sum_{i = 1}^n x_i^p \right)^{1/p} \left( \sum_{i = 1}^n y_i^q \right)^{1/q} \]
    %
    For $p = q = 2$, this is the Schwarz inequality.
\end{exercise}
\begin{proof}
    For $x_i,y_i > 0$, the function $f(x_1, \dots, x_n, y_1, \dots, y_n) = \sum_{i = 1}^n x_iy_i$ is convex.1

    Exponents are monotone, so we may prove the H\"{o}lder inequality instead by consider the equation
    %
    \[ \left(\sum_{i = 1}^n x_iy_i \right) \left( \sum_{i = 1}^n x_i^p \right)^{1/q} \left( \sum_{i = 1}^n y_i^p \right)^{1/p} \leq \left( \sum_{i = 1}^n x_i \right) \left( \sum_{i = 1}^n y_i \right) = \sum_{i = 1}^n \sum_{j = 1}^n x_iy_j \]
\end{proof}

\begin{exercise}
    Construct a metric space with two non disjoint balls $B_r(x) \supsetneq B_{r'}(y)$, such that $r < r'$. (a smaller ball contains a larger ball). Show this cannot occur if $r' > 2r$.
\end{exercise}
\begin{proof}
    Consider the space $X = \{ 0, 0.9, -0.9 \}$. Then the ball $X \cap (-1,1)$ has radius two centred at 0, and contains all points in the space. The ball $X \cap (-0.6.5,2.45)$ has radius $2.5$, centred at $0.9$, yet is a proper subset of the ball above.

    Now consider two balls $B_r(x) \supsetneq B_{r'}(y)$, and let $2r < r'$. If $x \in B_r(y)$, then $\rho(x,y) < r$, so for any $z$ with $\rho(x,z) < r$,
    %
    \[ \rho(y,z) \leq \rho(y,x) + \rho(x,z) < r + r = 2r < r' \]
    %
    Hence $B_r(x) \subset B_{r'}(y)$.
\end{proof}

Many useful algebraic structures have topological properties here. We shall now begin to consider them.

\begin{definition}
    Let $V$ be a real vector space. A norm on $V$ is a function $\| \cdot \|: V \to \mathbf{R}$, such that:
    %
    \begin{enumerate}
        \item For any scalar $\lambda \in \mathbf{R}$ and vector $v \in V$, $\| \lambda v \| = | \lambda | \| v \|$.
        \item For any vectors $v,w \in V$, $\| v + w \| \leq \| v \| + \| w \|$.
        \item $\| v \| = 0$ if and only if $v = 0$.
    \end{enumerate}
    %
    Every norm defines a metric $\rho$ defined by $\rho(x,y) = \| y - x \|$, and thus forms a topological space.
    %
    \begin{itemize}
        \item A set $A$ of $V$ is convex if, when two vectors $v$ and $w$ lie in $A$, we also have $\lambda v + (1 - \lambda) w$ lying in $A$ for any $\lambda \in [0,1]$.
    \end{itemize}
\end{definition}

\begin{exercise}
    Show that every metric space is normal
\end{exercise}
\begin{proof}
    If $x$ is a single point, and $y \neq x$, then the ball $B_{\rho(x,y)/2}(y)$ is an open set containing $y$ but not $x$. Hence $y$ is not a limit point, and the singleton $\{x\}$ is closed, so we have a Frechet space.

    Let $A$ and $B$ be two disjoint closed sets in a metric space $X$. Let $a$ be a point in $A$, and consider a neighbourhood $B_{\varepsilon_a}(a)$ not containing any points in $B$, whose closure contains no points in $B$. We can do this since $B$ is closed, so we can first select some ball $B_\varepsilon(a)$ containing no points in $b$ (since $a$ is not a limit point of $b$), and then take $B_{\varepsilon/2}(a)$ as $U_a$. The union of these neighbourhoods covers $A$, and its closure contains no points in $B$. To see this, let $U$ be the union, and consider some sequence $(x_i)_{i = 1}^\infty$ in $U$ which hypothetically converges to some point $b$ in $B$. Each $x_i$ an element of some $U_{a_i}$, where $a_i$ is a point in $A$. We must have the radius of $\varepsilon_{a_i}$ converging to 0, since otherwise $B_{\varepsilon_a}$
    Let $A$ and $B$ be two disjoint closed sets in a metric space $X$. Let $a$ be a point in $A$, and consider a neighbourhood $B_{\varepsilon_a}(a)$ not containing any points in $B$, whose closure contains no points in $B$. We can do this since $B$ is closed, so we can first select some ball $B_\varepsilon(a)$ containing no points in $b$ (since $a$ is not a limit point of $b$), and then take $B_{\varepsilon/2}(a)$ as $U_a$. The union of these neighbourhoods covers $A$, and its closure contains no points in $B$. To see this, let $U$ be the union, and consider some sequence $(x_i)_{i = 1}^\infty$ in $U$ which hypothetically converges to some point $b$ in $B$. Each $x_i$ an element of some $U_{a_i}$, where $a_i$ is a point in $A$. We must have the radius $\varepsilon_{a_i}$ converging to 0, since otherwise one of these results will obtain $b$. From this discussion, we may conclude that $a_i$ must also converge to $b$, since, for $a_i$ great enough, $d(a_i,b) < d(a_i,x_i) + d(x_i,b) < \varepsilon + \varepsilon \to 0$. This cannot occur, since $A$ is closed to begin with.

    Now we have an open set whose closure does not contain any points in $B$. We may repeat the process above to obtain an open set containing $B$, as is required for the proof.
\end{proof}

\vspace{10cm}



Surely this specification must involve the circles we were discussing as our fundamental example. Of course, these are not the only open sets, since the union of two balls is not necessarily an open ball, but we may use these as the fundamental sets from which all open sets are constructed.

The intuitive topology on $\mathbf{R}$ can be generalized to any linearly ordered set. In most cases, we may take the same open intervals. One case is more complicated; if the set contains a minimum or maximum element, then these elements are contained in no open intervals. We fix this by allowing in combination with open sets half closed rays $[-\infty, x)$ and $(x, \infty]$ into the definition of openess. We call the topology defined the order topology on the ordered set.

If $X$ is a topological space, and $Z$ is a subset, we can see intuitively how $Z$ may inherit the notion of space from $X$. We take open sets in $Z$ to be the intersection of open sets in $Z$ with $Z$, and we call this the subspace topology.

As examples, the topology $\mathbf{Z}$ inherits from the order topology on $\mathbf{R}$ the discrete topology (which is why we think of integers as being separated on the real line). Conversely, the set $\{ \frac{1}{n}: n \in \mathbf{N} \} \cup \{ 0 \}$ inherits a completely different topology (zero is no longer open on its own).

Most of the time, when specifying a set, it is difficult to specify precisely the set of all open sets that define a topology. Since, when $\{ C_i \}$ is a indexed set of topology, $\bigcap_{i \in I} C_i$ is also topology, we may, when given a subcollection of the power set of a set, generate a topology on that subset by taking the smallest topology which contains the subcollection. Special collections of subs
ets $\mathcal{C}$ with the following properties are of increased importance:
%
\begin{enumerate}
    \item Every element in the space is contained in one of the subsets
    \item If $x$ is contained in two sets $C_1$ and $C_2$ in $\mathcal{C}$, there is a third set $C_3$ in $\mathcal{C}$ such that $x \in C_3 \subset C_1 \cap C_2$.
\end{enumerate}
%
Then the topology generated has the property that a set $A$ is open if and only if, for any element $x$ in $A$, there is a set $C$ in $\mathcal{C}$ containing $x$ such that $C \subseteq A$. We call such a collection a basis for the topology. Specific examples include open intervals in $\mathbf{R}$. We say $\mathcal{C}$ covers the topology it generates.

Unlike in linear algebra, a basis for a topology is not unique up to bijection. We cannot even always find a minimal basis for all topologies in terms of containment (consider open balls in $\mathbf{R}^n$). We can only promise that there is a basis with minimum cardinality, which exists due to the well ordering property of cardinal numbers.

\begin{theorem}
    Let $X$ be a topological space, and $Y$ a subset with the subspace topology. Then a subset $A$ is closed in $Y$ if and only if $A = B \cap Y$ for a closed set $B$ in $X$.
\end{theorem}
\begin{proof}
    Suppose $A = B \cap Y$ as in the theorem's statemenet. Then $X - B$ is open in $X$, so $(X - B) \cap Y$ is open in $Y$, and this set is just $Y - B$. But this $B \cap Y$, which is the complement of $Y - B$, is closed in $Y$.

    Conversely, if $A$ is closed in $Y$, $Y - A$ is open in $Y$, hence $Y - A = V \cap Y$ for some open set $V$ in $X$. Since $V^c \cap Y = A$, we see that $A$ is the intersection of $Y$ with a closed set in $X$.
\end{proof}

\chapter{Constructions}

Here we get to the visually interesting part of topology, providing methods to mold and curve the topological structures of your choosing, gluing, stretching, and all other kinds of fun stuff. It will explain how we get from a plane to a torus, or from $\mathbf{R}^2$ to $S^2$. The construction we will make can be shown in a very general manner.

\begin{definition}
    Let $X$ be a topological space, and $\sim$ an equivalence relation on $X$. From this equivalence relation, we form the set $X/\sim$, and consider the projection mapping $\pi$ from $X$ to $X/\sim$. The quotient topology on $X/\sim$ is the coarsest topology that makes $\pi$ continuous, and makes $X/\sim$ the quotient space of $X$ and $\sim$. A set $A$ is open in $X/\sim$ if and only if $\pi^{-1}(A)$ is open in $X$.
\end{definition}

If we have a surjective map $f:X \to Y$, where $X$ is some topological space, and $Y$ is any set, then we may construct a topology on $Y$ analogous to the quotient topology. First we consider the fibers of $X$ relative to the mapping $f$, and identify the quotient topology on this set. We obtain a bijective mapping $\overline{f}$. The quotient topology on $Y$ is the topology which makes $\overline{f}$ a homeomorphism. Since, in the context of topology, homeomorphisms preserve all important properties, we may as well consider this definition no different from the definition in terms of equivalence relations.

\chapter{Algebraic Topology}

To verify that two topological spaces are homeomorphic, we need only find a single homeomorphism that connects the two spaces. On the contrary, to verify that two topological spaces are not homeomorphic, we need to somehow show that every function from one space to the other is not a homeomorphism, a computationally intractable problem. One trick we can use to separate topological spaces is to find fundamental topological properties which distinguish two topological spaces. Connectedness, Compactness, and Hausdorffiness are all preserved by homeomorphism, as does the topological properties of subspaces. Nonetheless, sometimes these properties are not enough to distinguish two spaces. This chapter shows a deep technique which is often useful for characterizing spaces.

Consider two functions $f$ and $g$ between topological spaces $X$ and $Y$. Though $f$ might not be equal to $g$, they may be in some sense topologically equal -- we may be able to deform one to the other in a continuous fashion. This is a homotopy.

\begin{definition}
    Let $f,g: X \to Y$ be two continuous functions. Define a topology on $Hom(X,Y)$ as a subspace of $Y^X$, which can be viewed as the product topology of $Y$ with itself $Y$ times. Then $f$ and $g$ are homotopic if there exists a path in $Hom(X,Y)$ which connects $f$ to $g$. Alternatively, these two functions are homotopic if there exists a continuous function $F:[0,1] \times X \to Y$ such that for all $x$, $F(0,x) = f(x)$, and $F(1,x) = g(x)$.
\end{definition}

The fact that homotopy is an equivalence relation will allow us to distill functions between spaces to their fundamental properties. We need to specialize our definition for it to be more of more use to us.

\begin{definition}
    Two paths in $X$ are path homotopic if they have the same start and end point, and are homotopic to each other.
\end{definition}

Let $f$ and $g$ be two paths in $X$, where the end point of $f$ is the start point of $g$. Then we may compose the two paths to form a new path $f * g$, defined by
%
\[ (f * g)(x) = f(2x): x \in [0,1/2]
                g(2x - 1): x \in [1/2,1] \]
%
By the pasting lemma, this function is a path which connects the start point of $f$ to the end point of $g$. Unfortunately, concatenation is not associative, we do not have that $f * (g * h) = (f * g) * h$. These paths are homotopic to each other, however, and moving to path homotopy classes makes the definition much simpler.

\begin{theorem}
    Let $f$ be path homotopic to $f'$, and $g$ path homotopic to $g'$. Then $f * g$ is homotopic to $f' * g'$.
\end{theorem}
\begin{proof}
    Let $F$ be the path homotopy from $f$ to $f'$, and $G$ the path homotopy from $g$ to $g'$. Define a homotopy $H$ between $f * g$ and $f' * g'$ by
    %
    \[ H(\cdot,y) = F(\cdot, y) * G(\cdot, y) \]
    %
    More specifically
    %
    \[ H(x,y) = \begin{cases}
        F(2x,y) & \text{if } x \in [0,1/2]\\
        G(2x - 1,y) & \text{if } x \in [1/2,1]\\
\end{cases} \]
    %
    The pasting lemma guarentees this function is a new homotopy.
\end{proof}

We now consider homotopy classes of paths, so when we talk about a path $f$, we are really talking about all paths homotopic to $f$.

\begin{theorem}
    $[f] * ([g] * [h]) = ([f] * [g]) * [h]$.
\end{theorem}

\end{document}