\documentclass[12pt]{report}

\usepackage{amsmath}
\usepackage{amssymb}
\usepackage{amsthm}
\usepackage{amsopn}
\usepackage{kpfonts}
\usepackage{graphicx}
\usepackage{kbordermatrix}
\usepackage{tikz}
\usetikzlibrary{arrows, petri, topaths}%
\usepackage{tkz-berge}
\usepackage{multicol}

\usepackage{framed}
\usepackage{mathtools}
\usepackage{float}
\usepackage{subfig}
% \usepackage{cmbright}

\theoremstyle{plain}
\newtheorem{theorem}{Theorem}[chapter]
\newtheorem{lemma}[theorem]{Lemma}
\newtheorem{corollary}[theorem]{Corollary}
\newtheorem{prop}[theorem]{Proposition}
\newtheorem{exercise}{Exercise}[chapter]

\newtheorem*{example}{Example}
\newtheorem*{proof*}{Proof}

\theoremstyle{definition}
\newtheorem*{defi}{Definition}
\newenvironment{definition}
    {\begin{samepage}\begin{framed}\begin{defi}}
    {\end{defi}\end{framed}\end{samepage}}





\usepackage{hyperref} 
\hypersetup{
    colorlinks = true,
    linkcolor = black,
}

\makeatletter
\renewcommand*\env@matrix[1][*\c@MaxMatrixCols c]{%
  \hskip -\arraycolsep
  \let\@ifnextchar\new@ifnextchar
  \array{#1}}
\makeatother

\renewcommand*\contentsname{\hfill Table Of Contents \hfill}

\newcommand{\optionalsection}[1]{\section[* #1]{(Important) #1}}
\newcommand{\deriv}[3]{\left. \frac{\partial #1}{\partial #2} \right|_{#3}}

\DeclareMathOperator{\Dom}{Dom}

\title{Topology}
\author{Jacob Denson}

\begin{document}

\pagenumbering{gobble}

\maketitle

\tableofcontents

\pagenumbering{arabic}

\chapter{Analogies}

Metaphor is at the heart of modern mathematics. Given a complex abstract problem, it is difficult to see beyong the logic. Via abstraction, we obtain analogies laden with intuition. Why try and see an infinite dimensional vector space, when we can see a three dimensional cartesian space which allows us to `see the problem'. In the context of number theory, linear algebra, and algebraic geometry, abstract algebra is the analogy constructor. In analysis, Topology is king, generalizing that intuitive notion of space which encloses calculus, differential geometry, functional, real, and complex analysis under a single umbrella.

Seeing as how general topology arose from a diverse range of fields, it is unsurprising to see that a multitude of definitions arose to define what a topology was. The following definition is due to the German mathematician Georg Cantor. Though it is not the only way to define what we mean by a topology, it is by far the most common.

\section{Space, and other delights}

Just as how numbers and rotations were the foundations of abstract algebra, the primary example of a topological space, from which all the terminology is obtained, is the euclidean plane. I'm sure that, at one point in your childhood, you owned a colouring book. In this tome of black and white, you delighted in colouring between lines that define a figure. Shapes emphasized by a black silouhette are easily distinguished and filled by a nice blue, green, or pink fluorescent marker. Mathematically, we distinguish between shapes which contain their silouhette and shapes that do not. Those shapes that do contain their `boundary', as it is formally called, are called closed. The shapes that contain none of their boundary are open. Consider the interior of a circle (not including an outline). At any point on this shape, there is an infinitude of points from your pen tip to the edge of the circle -- you'll never finish precisely coloring the shape in. Sets with this sort of property are called open. The foundations of topology are obtained by abstracting the qualities that these `edgeless' shapes possess.

\begin{definition}
    A {\bf topological space} is a set $X$, together with a {\bf topological structure} $\Omega$, a designated collection of subsets, restricted by the following properties:
    %
    \begin{enumerate}
        \item Both $\emptyset$ and $X$ itself are in $\Omega$
        \item If $\{A_i\}_{i \in I}$ is an arbitrary family such that $A_i \in \Omega$ for each $i$, then their union $\bigcup_{i \in I} A_i$ is also in $\Omega$.
        \item If $\{A_i\}_{i = 1}^n$ is a finite collection of open sets, then their intersection $\bigcap_{i = 1}^n A_i$ is also open.
    \end{enumerate}
\end{definition}

Together with the definition comes an onslaught of terminology: to learn a field, you must first speak its language. All terminology will be expanded fully later on. This is just to provide an easy reference.

\begin{itemize}
    \item Elements of a topological space are called {\bf points}.
    \item We say a subset of a topological space is {\bf open} if it is contained in the topological structure.
    \item A subset is {\bf closed} if its complement is open.
    \item A {\bf neighbourhood} of a point $x \in X$ is an open set containing $x$\footnote{Analysts and French mathematicians extend the notion of a neighbourhood, defining it to be a set containing a neighbourhood in our sense. When reading other works of topology, make sure you remember which definition the author is using.}.
    \item A {\bf limit point} of a subset $U$ of $X$ is a point $p$ such that every open set containing $p$ contains points in $U - \{p\}$. The set of limit points of a subset $U$ is denoted $U'$, and is called the derived set of $U$.
    \item If we have two topological structures $\Delta$ and $\Omega$, then we say $\Delta$ is {\bf finer} than $\Omega$, or $\Omega$ is coarser, if $\Omega \subset \Delta$. Spaces are comparable in the same way that rocks, pebbles, and grains of sand are.
    \item A function $f:X \to Y$ between two topological spaces is {\bf continuous} if $f^{-1}(U)$ is open for every open set $U$ in $Y$. A homeomorphism is a continuous bijective function whose inverse is continuous.
\end{itemize}

Learn the concept of open-ness. It is the most crucial idea in topology; everything else we shall deal with is defined in terms of open sets and the qualities they possess amongst themselves. This is why all we need for a topology to form is a topological structure $\Omega$ of open sets, known universally as `the topology' of the space.

Topology also defines additional axioms which ensure your spaces are more reasonable. If your space does not have at least the very low level properties, then be wary for non-intuitive spatial structure.

\begin{itemize}
    \item A topological space is {\bf Kolmogorov}, or {\bf T0}, if, for any points $x \neq y$, either $x$ has a neighbourhood not containing $y$, or $y$ has a neighbourhood not containing $x$. Almost every Topological space has this property.
    \item A topological space is {\bf Frechet}, or {\bf T1} if, for any points $x \neq y$, we may find a neighbourhood of $x$ not containing $y$.
    \item A topological space is {\bf Hausdorff}, or {\bf T2} if, for any points $x \neq y$, we may find disjoint open sets $U$ and $V$ such that $x \in U$, $y \in V$. This is the property which represents `unique convergence', which we shall discuss later.
    \item A topological space is {\bf Tychonoff/Regular}, or {\bf T3}, if it is Frechet, and when $C$ is a closed set not containing a point $x$, we may find disjoint open sets $U$ and $V$, such that $C \subset U$, and $x \in V$.
    \item A topological space is {\bf Normal}, or {\bf T4}, if it is Frechet and if two disjoint closed sets $C$ and $D$ can be covered by disjoint open neighbourhoods $U$ and $V$, such that $C \subset U$, $D \subset V$.
\end{itemize}

We defined openness as above because these properties hold for the colouring book topology above. We define a shape to be open if we can never colour up to an edge. A shape (subset of the plane) is open if we may draw a circle around every point in the shape, and all points in the interior of the circle are contained in the shape itself. Intuition should tell you why the properties of open sets hold for this space -- we shall define this precisely later. In this topology, the finite assumption of open intersections is crucial. If we take the intersection of an infinite number of open sets, we can no longer colour in until we reach one boundary -- the boundary may have been stretched too thin. Consider the intersection of the `infinite venn diagram below'.

INFINITE CIRCLES INTERSECTING ONLY AT A POINT.

Given any set $X$ there are two spatial extremities we can use to form a topology. We either choose the minimal number of open sets, or the maximal number. The `discrete topology' on $X$ lets the topological structure $\Omega$ be equal to the power set $\mathcal{P}(X)$. In this case, every subset of the plane is open, and the axioms of trivially satisfied. The `lumpy topology' on $X$ has minimum structure, with topological structure $\Omega' = \{\emptyset, X\}$. The fact that these are topologies is verified by trivial set theory.

The discrete topology is named based on how closeness factors into topological spaces. `limit points', as we have defined them above, indicate points which reach infinitely close to a given set. This is as close to distance as we can get in topology -- we cannot tell if two points are miles or metres away, but only if they are infinitisimally close. The lumpy topology is named because topologically, every point is a limit point of every set (except in the case where the set is empty).

\begin{exercise}
    Let $X$ be the ray $[0,\infty)$, and let $\Omega$ consist of $\emptyset$, $X$, and all rays $(a,\infty)$ with $a \geq 0$. Prove that $\Omega$ is a topological structure.
\end{exercise}
\begin{proof}
    By definition, $\emptyset$ and $X$ are open in the topology. If $A = (a, \infty)$ and $B = (b, \infty)$ are two open sets in $\Omega$, then $A \cap B = (\max(a,b), \infty)$. By induction, arbitrary intersections are in the topology. If $\{(a_i, \infty]\}_{i \in I}$ is a non-empty family of open sets, then $\{a_i\}_{i \in I}$ is a non-empty set of real numbers bounded below (since $a_i \geq 0$). By the completeness property of $\mathbf{R}$, we therefore have an infininum $a = \inf_{i \in I} a_i$. We contend $(a, \infty) = \bigcup (a_i, \infty)$. If $x > a$ ($x \in (a,\infty)$), then $x > a_i$ for some $i$, since otherwise $x$ is a greater lower bound of $\{a_i\}$ than $a$. Therefore $x \in (a_i, \infty)$ for some $a_i$, so $(a,\infty) \subset \bigcup (a_i,\infty)$. Conversely, since $a \leq a_i$ for all $a_i$, $(a_i,\infty) \subset (a, \infty)$ and, taking the union, $\bigcup_{i \in I} (a_i, \infty) \subset (a, \infty)$. If a family of open sets contains $X$ or $\emptyset$, the outcome of the union is trivial. We have shown the intersection and union property hold in full.
\end{proof}

\begin{exercise}
    Let $X$ be a plane. Let $\Sigma$ consist of $\emptyset$, $X$, and all open disks with center at the origin. Verify this is a topological structure.
\end{exercise}
\begin{proof}
    We have $\emptyset$ and $X$ in $\Sigma$. Let $D(\varepsilon)$ denote the disk or radius $\varepsilon > 0$ centred at the origin. If $\{D(\varepsilon_i)\}_{i \in I}$ is a family of discs, then either
    %
    \begin{enumerate}
        \item $\{\varepsilon_i\}_{i \in I}$ is unbounded above. In this case we have $\bigcup D(\varepsilon_i) = X$, since any $x \in X$ is in some $D(\varepsilon_i)$.
        \item $\{\varepsilon_i\}_{i \in I}$ is bounded above. Consider $\varepsilon = \sup \varepsilon_i$. We will show $D(\varepsilon) = \bigcup_{i \in I} D(\varepsilon_i)$. Each $D(\varepsilon_i)$ is a subset of $D(\varepsilon)$, so $\bigcup_{i \in I} D(\varepsilon_i) \subset D(\varepsilon)$. Conversely, if $x \in D(\varepsilon_i)$, then $d(x,0) < \varepsilon$, so there is some $\varepsilon_i$ with $d(x,0) < \varepsilon_i$.
    \end{enumerate}
    %
    If $D(\varepsilon)$ and $D(\varepsilon')$ are two open discs, then their intersection is $D(\min(\varepsilon, \varepsilon'))$.
\end{proof}

The most important example of a topological space is $\mathbf{R}$, the space of real numbers. Many other examples will stem from this space. The topological structure consists of the empty set, and unions of intervals $(a,b)$, where $a < b$ can be infinite. We have $\mathbf{R} = (-\infty, \infty)$. Its a bit fiddly, but you should be able to show that the intersection of two intervals is the union of intervals, and thus that the intersection property holds in general. This topology is the standard topology on $\mathbf{R}$.

\begin{exercise}
    Show that every open set in $\mathbf{R}$ can be broken into the disjoint union of open intervals. Bonus points: Show that this union is countable!
\end{exercise}
\begin{proof}
    Let $U$ be an open set in $\mathbf{R}$, equal to $\bigcup_{i \in I} (a_i, b_i)$ for some indexing set $I$. Let $x \in U$. Consider $A_x = \{ a \in \mathbf{R} : (a,x)] \subset U \}$ and $B_x = \{ b \in \mathbf{R} : [x,b) \subset U$. Surely this set is not empty, since $x \in (a_i, b_i)$ for some $i$, and thus $a_i \in A_x$, $b_i \in B_x$. Consider $a_x = \inf A$, and $b_x = \sup B$, finite or infinite. We contend that $(a_x,b_x)$ is a neighbourhood of $x$ contained in $U$. Surely $x \in (a_x,b_x)$, for $a < x < b$. Furthermore, if $y \in (a_x,b_x)$, then $a_x < y < b_x$, and there is some $(a_i, b_i)$ such that $a_i < y < b_i$, so $y \in U$. Form the set of intervals $\{ (a_x, b_x): x \in U\}$. We claim that this is a pairwise disjoint set of intervals whose union is $U$. If $(a_x, b_x)$ and $(a_y, b_y)$ are non disjoint, consider an element $z$ in their intersection. Then $a_x, a_y \in A_z$, $b_x, b_y \in B_z$, so $(a_z, b_z)$ contains both intervals. Namely, $(a_z, b_z)$ contains $x$ and $y$, so $a_z \leq a_x, a_y$, and $b_z \leq b_x, b_y$. But also $a_z \geq a_x, a_y$, and $b_z \geq b_x, b_y$, since $a_z \in A_x, A_y$, and $b_z \in B_x, B_y$. By a simple manipulation of inequalities, we have shown that $a_x = a_z = a_y$, and $b_x = b_z = b_y$, so that intervals are pairwise disjoint, or equal.

    The fact that the union is countable stems from the fact that every interval contains a rational which we may uniquely identify with the interval in the case that the union is of disjoint intervals.
\end{proof}

Closed sets of $\mathbf{R}$ are not so simple to classify. Later on, we will see the Cantor set, a very strange closed set of $\mathbf{R}$.

\begin{exercise}
    Let $X$ be a set, and let $\Omega$ consist of all subsets of $X$ whose complement is finite, and $\emptyset$. Show that $\Omega$ is a topological structure, called the $T_1$ topology.
\end{exercise}
\begin{proof}
    If $X - A$ is finite, and $X - B$ is finite, then $X - (A \cap B) = (X - A) \cup (X - B)$ is a finite union of finite sets, hence finite. If $X - A$ is finite, and $B$ is any other subset, then $X - (A \cup B) \subset X - A$ is finite.
\end{proof}

This example should show us that our definition of topology is not trivial in the slightest.

\begin{exercise}
    Consider the set $\mathbf{Z}$, and let $\Omega$ be the set of all unions of arithmetical sequences of the form $a\mathbf{Z} + b = \{ ax + b: x \in \mathbf{Z}\}$, where $a \neq 0$. Show that $\Omega$ is a topological structure, and that there are infinitely many prime numbers.
\end{exercise}
\begin{proof}
    Since every integer is in some arithmetic sequence, we know that $\mathbf{Z}$ itself is open, as is the empty set. Here is an arbitrary intersection of two sequences
    %
    \[ [a\mathbf{Z} + b] \cap [m\mathbf{Z} + n] \]
    %
    If $x$ can be written $ay + b$ and $mz + n$ via two integers $y$ and $z$, then $x + am\mathbf{Z}$ is also contained in the intersection. so that the intersection of two sequences is the union of other arithmetical sequences. Hence the set of all union of arithmetical sequences specifies a topology.

    This topology is a very strange one. Consider the factorial sequence
    %
    \[ 1!,\ 2!,\ 3!, \dots = 1,\ 2,\ 6,\ 24,\ 120,\ 720, \dots \]
    %
    Any open set containing $0$ contains some arithmetical sequence $a\mathbf{Z}$, and the factorial sequence will eventually end up in this sequence since the sequence accumulates all integers as factors. Therefore this sequence is `infinitely close' to 0 in this toplogy, even though in the canonical topology on the integers the sequence is one of the one that grows faster than any other function. The main importance of this topological space is that it allows us to show there are infinitely many primes. Suppose there are only finitely many, which we may write as $\{ p_1, p_2, \dots, p_n \}$. Then
    %
    \[ A = \bigcap_{k = 1}^n p_k\mathbf{Z} \]
    %
    is the finite intersection of open sets, and is thus open. Since every non-zero non-unit integer can be written as the product of primes, it follows that the complement $A^c = \{ -1,0,1 \}$. In our topology, the complement of open sets is open (check this via the basis elements), so that $A^c$ is the union of open sets, which is clearly not true. Hence there must be infinitely many primes.
\end{proof}

\section{Closed topological sets}

A set in a topological space $X$ is closed if it is the complement of an open set. A closed interval $[a,b]$ is closed in $\mathbf{R}$, since $[a,b]^c = (-\infty, a) \cup (b,\infty)$. To specify the axioms of a topology, we could have taken as a primitive notion closedness instead of openness. Since
%
\[ \big( \bigcup_{i \in I} A_i \big)^c = \bigcap_{i \in I} A_i^c \]
%
\[ \big( \bigcap_{i \in I} A_i \big)^c = \bigcup_{i \in I} A_i^c \big) \]
%
we could have taken a topology as a collection of closed sets such that arbitrary intersections and finite unions of closed sets are closed.

\begin{exercise}
    Prove that $[0,1)$ is not closed nor open in $\mathbf{R}$, yet it is the union of closed sets and the intersection of open ones.
\end{exercise}
\begin{proof}
    Suppose $[0,1)$ was open, so it is equal to the union of intervals $\bigcup (a_i,b_i)$. We must have $0 \in (a_i, b_i)$ for some $i$, but then $a_i < 0$, and thus $a_i/2 < 0 \in [0,1)$, a contradiction. Conversely, suppose $[0,1)$ was closed, so $(-\infty, 0) \cup [1,\infty)$ was open. A similar argument to the one above shows that this is impossible.

    We have
    %
    \[ [0,1) = \bigcup_{n = 1}^\infty \left[0, 1 - \frac{1}{n} \right] = \bigcap_{n = 1}^\infty (-\frac{1}{n}, 1) \]
    %
    So we cannot take arbitrary unions of closed sets nor intersections of open sets.
\end{proof}

\begin{exercise}
    Prove that $\{0\} \cup \{ \frac{1}{n} : n \in \mathbf{Z} \}$ is closed in $\mathbf{R}$.
\end{exercise}
\begin{proof}
    \[ \left[\{0\} \cup \left\{ \frac{1}{n} : n \in \mathbf{Z} \right\}\right]^c = (-\infty, 0) \cup (1, \infty) \cup \left(\frac{1}{2},1\right) \cup \left(\frac{1}{3},\frac{1}{2}\right) \cup \dots \]
    %
    The complement is open, as it is just the union of open sets.
\end{proof}

Closed sets are seen as `containing the points' infinitely close to each other. Here we establish what exactly this means. Recall that $x$ is a limit point of a set $A$ if every open set containing $x$ contains some point in $A$.

\begin{exercise}
    If $C$ is a closed set if and only if set of limit points $C'$  is a subset of $C$.
\end{exercise}
\begin{proof}
    Let $C$ be a closed set. Then $C^c$ is open, so if $x$ is not in $C$, $x$ is not a limit point of $C$. We have shown what we needed to show. Now suppose $C'$ is contained in $C$, where $C$ is an arbitrary set. If $x$ is not an element of $C$, then it follows that there is an open set $U_x$ containing $x$ not containing any point in $C$. The union
    %
    \[ \bigcup_{x \notin C} U_x \]
    %
    is an open set containing no elements of $C$, and any point not in $C$. Thus the union is just $C^c$, and we have shown $C^c$ is open, hence $C$ is closed.
\end{proof}

\subsection{An excursion -- Cantor's set}

This excursion shows us that weird closed sets exist even in the most simple topologies -- here the real numbers $\mathbf{R}$.

\begin{exercise}
    Consider the Cantor set $K$ defined below:
    %
    \[ K = \left\{ x \in \mathbf{R} : x = \sum_{k = 1}^\infty \frac{a_k}{3^k} \ \text{with}\ a_k = 0\ \text{or}\ 2 \right\} \]
    %
    Show that $K$ can be inductively defined as $\bigcap_{i = 1}^\infty K_i$, where $K_0 = [0,1]$, and
    %
    \[ K_{i+1} = \frac{K_i}{3} \cup \left(\frac{2}{3} + \frac{K_i}{3}\right) \]
\end{exercise}
\begin{proof}
    Let us show first that $K \subset \bigcap_{i = 1}^\infty K_i$. Here is an arbitrary decimal expansion in the Cantor set
    %
    \[ x = \sum_{k = 1}^\infty \frac{a_k}{3^k} \]
    %
    where each $a_k = 0$ or $2$. Then $x \in K_0 = [0,1]$, since
    %
    \[ 0 = \sum_{k = 1}^\infty \frac{0}{3^k} \leq \sum_{k = 1}^\infty \frac{a_K}{3^k} \leq \sum_{k = 1}^\infty \frac{2}{3^k} = 2(\frac{1}{1 - 1/3} - 1) = 1 \]
    %
    For an induction, assume $K \subset K_n$. Suppose $a_1 = 0$. Then
    %
    \[ x = \frac{a_1}{3} + \frac{a_2}{3^2} \dots = \frac{1}{3} (a_2 + \frac{a_2}{3} + \dots) = y/3 \]
    %
    where $y \in K$, so $y \in K_n$ by hypothesis. Then $x \in K_n/3 \subset K_{n+1}$. If $a_1 = 2$,
    %
    \[ x = 2/3 + \frac{1}{3}(\frac{a_2}{3} + \frac{a_3}{3^2} + \dots) = 2/3 + y/3 \]
    %
    where $y \in K$. Therefore $y \in K_n$, and $x \in 2/3 + K_n/3 \subset K_{n+1}$. This exactly proves that $K \subset K_{n + 1}$. By induction, we have shown $K \subset \bigcap_{i = 1}^\infty K_i$.

    Now we must show, conversely, that $\bigcap K_i \subset K$. Let $x \in \bigcap K_i$. We will build a correct sequence of digits that converge to $x$. Let $x_0 = 0$, and
    %
    \[ x_n = \max \left\{ a_k \in \{0,2\} : \sum_{k = 0}^{n-1} x_k/3^k + a_k/3^n \leq x \right\} \]
    %
    Our first claim:
    %
    \[ \sum_{k = 0}^\infty x_k/3^k = x \]
    %
    We shall prove by induction that if $x \in K_n$, then $x - \sum_{k = 1}^\infty x_k/3^k < 1/3^n$, If $x$ is in all $n$, then $x - \sum_{k = 1}^\infty x_k/3^k$ must be equal to zero, since it is smaller than $1/3^n$ for all $n$, and greater than or equal to zero. This is trivially true for $K_0$, since any sum is between 0 and 1, and $x$ is also between 0 and 1. Assume this is true for all $y \in K_n$. Consider $x \in K_{n+1}$. Assume that there is $y = 3x$ in $K_n$. Then $x_1 = 0$ (since otherwise $x > 1/3$, and $y > 1$), and if we consider the expansion of $y$, we will see that $y_n = x_{n+1}$. This follows because
    %
    \[ \sum_{k = 1}^{n-1} y_k/3^{k+1} + a/3^n = \sum_{k = 1}^{n} x_k/3^k + a/3^{n+1} \leq x \]
    %
    holds if and only if the inequality
    %
    \[ 3\left(\sum_{k = 1}^{n-1} x_k/3^k + a_k/3^n\right) = \sum_{k = 1}^{n-1} y_k/3^k + a_k/3^n \leq 3x = y \]
    %
    holds, assuming for an induction that $y_k = x_{k+1}$ for $k \leq n-1$. If
    %
    \[ y - \sum_{k = 1}^\infty y_k/3^k < 1/3^n \]
    %
    then $y/3 - \sum_{k = 1}^\infty y_k/3^{k+1} = x - \sum_{k = 1}^n x_k/3^k < 1/3^{n+1}$. If $x = 2/3 + y$, where $y \in K_n$, the same technique establishes the inequality. Our claim is thus proved.
\end{proof}

\begin{exercise}
    Show that the Cantor set $K$ is closed in $\mathbf{R}$.
\end{exercise}
\begin{proof}
    We will show each $K_i$ is closed. Obviously $K_0 = [0,1]$ is closed. If $K_n$ is closed, then $K_{n+1} = K_n/3 \cup 2/3 + K_n/3$ is the union of two closed sets, since the maps $x \to x/3$ and $x \to x + 2/3$ are homeomorphisms of the real line (which we will define later), and map closed sets to closed sets, the claim is proved.
\end{proof}

\section{The Basis of a space}

Normally, a topology is not given via specifying every single open set in the topology. Since open sets are constructed from other open sets, we may specify some archetypal sets, and provided these sets are sufficient to describe a topology, define a topology in terms of them.

Let $X$ be a set, and $\{\Omega_i\}_{i \in I}$ a family of topological structures on $X$. Consider the intersection of all structures,
%
\[ \Delta = \bigcap_{i \in I} \Omega_i \]
%
Surely $\emptyset$ and $X$ itself are elements of $\Delta$, since they are an element of each $\Omega_i$. If $\{A_i\}_{i \in I} \in \Omega_i$ for each $i$, then surely $\bigcup_{i \in I} A_i \in \Omega_i$, so that if $\{A_i\}_{i \in I} \in \Delta$, we also have $\bigcup_{i \in I} A_i \in \Delta$. Similarily, if $\{A_i\}_{i \in I} \in \Delta$, then $\bigcap_{i = 1}^n A_i \in \Delta$, so that $\Delta$ is a topological space. This fact will allow us to generate structures from generating sets, as is done in many areas of mathematics.

If $X$ is a set, and $D$ is a family of subsets, then we may consider
%
\[ \Delta = \{ \Omega \in \mathcal{P}(X) : D \subset \Omega\ \text{and D is a topological space} \} \]
%
Taking $\Phi = \bigcap \Delta$, we obtain the coarsest topological space containing $D$, called the topology generated by $D$. If $\Omega$ is a topological space containing $D$, then $\Phi \subset \Delta$ also, by the construction above.

When $D$ has certain nice properties, the topological space generated by $D$ is much simpler to work with:

\begin{definition}
    Let $D$ be a subset of a topological space $X$. Suppose that
    %
    \begin{enumerate}
        \item $\emptyset, X \in D$.
        \item If $A$ and $B$ are in $D$, and $A \cap B$ is non-empty, containing a point $x$, then there is a set $C$ in $D$ containing $x$ with $C \subset A, B$.
    \end{enumerate}
    %
    In this case, the topological space $\Phi$ generated by $D$ has a following property. A set $U$ is open in $\Phi$ if and only if it is the union of sets in $D$. We call $D$ a {\bf basis} for the topology $\Phi$ if it satisfies properties (1) and (2), and a {\bf subbasis} if it is just a generating set.
\end{definition}

\begin{exercise}
    Can two distinct topological structures have the same base? That is, does the base of a topology uniquely define a topology.
\end{exercise}
\begin{proof}
    No. Let $X$ be a set, and $\Delta$, $\Omega$ two topological structures with the same base. Then $\Delta$ must be finer than $\Omega$, since $\Delta$ is the minimal structure containing the base. Similarily, $\Omega$ must be finer than $\Delta$, so we conclude the two structures are equal.
\end{proof}

\begin{exercise}
    Prove that there is no minimal topological basis for $\mathbf{R}$.
\end{exercise}
\begin{proof}
    Let $D$ be a basis for $\mathbf{R}$, and let $U \in D$ be a set in $D$ not equal to $\emptyset$ or $X$, which is therefore open in $\mathbf{R}$ and contains an interval $(a,b)$. Consider $D - U$. The first property of a basis is satisfied, and if $x$ is contained both in two subsets $A$ and $B$ in $D$, and if it were the case that $x \in U \subset A \cap B$ (without loss of generality, $x \in (a,b)$), then since $(a + \varepsilon,b - \varepsilon)$ is the union of sets in $D$, there must be some $U' \in D - U$ containing $x$, since $(a + \varepsilon, b - \varepsilon) \subsetneq U$. Thus $D - U$ is a basis, and generates a topological space. Obviously, $D$ is finer than $D -U$. Let $U = \bigcup (a_i, b_i)$. There exists $\delta$ such that, for each $0 < \varepsilon < \delta$, $(a_i + \varepsilon, b_i - \varepsilon)$ is open in $\mathbf{R}$ and is not equal to $U$, so $(a_i + \varepsilon, b_i - \varepsilon) = \bigcup U_i$, for some $U_i \in D - U$. But then
    %
    \[ U = \bigcup (a_i, b_i) = \bigcup_{i \in I} \bigcup_{\delta > \varepsilon > 0} \bigcup_{j \in J_i} U_{i,\varepsilon,j} \]
    %
    And $U$ is still open in the topology generated by $D - U$.
\end{proof}

A basis for a discrete space $X$ is just $X$ itself, since any open set in a discrete space is the union of some points in $X$. A basis for $\mathbf{R}$ is the set of all intervals $(a,b)$.

\begin{exercise}
    Show that two bases $D$ and $D'$ generate the same topological structure if every element of $D'$ is the union of $D$, and vice versa.
\end{exercise}

\section{Topology and Convergence}

Analysis takes topology and uses it to study limit operations. For instance, when we initially began studying the topology of $\mathbf{R}$, we defined convergence (with $\varepsilon$'s and $\delta$'s), and then proceeded to define open and closed sets as a corollary. Most of the applications of topology to analysis deal with convergence, so it is natural to wonder whether it is possible to define all topologies by the convergent sequences that result.

\begin{definition}
    Let $\Omega$ and $\Psi$ be two topologies on a single set $X$. We will say $\Omega$ and $\Psi$ are {\bf sequentially equivalent} when the sequences that converge in $(X,\Omega)$ are exactly the same as those in $(X, \Psi)$, and to exactly the same points.
\end{definition}

Does it follow that, when two topologies $\Omega$ and $\Psi$ are sequentially equivalent, $\Omega = \Psi$? The next example shows this is, unfortunately, not the case.

\begin{example}
    Let $X$ consist of all countable ordinals, together with the first uncountable ordinal, denoted $\omega_1$. Let $\Omega$ be the order topology on $X$, and let $\Psi$ be the topology generated by $\Omega \cup \{ \{ \omega_1 \} \}$. Surely $\Psi \neq \Omega$, since $(\omega, \omega_1] \neq \{ \omega_1 \}$ for any choice of $\omega$, yet $\Omega$ is sequentially equivalent to $\Psi$. Take any sequence of ordinals $\{ x_i \}_{i \in \mathbf{N}}$ in $X$. Since the relative topologies on $[0,\omega_1)$ generated by $\Omega$ and $\Psi$ are the same, if $x < \omega_1$, then $x_i \to x$ in $(X, \Omega)$ if and only if the sequence converges to $x$ in $(X,\Psi)$. Suppose that $x_i \to \omega_1$ in $(X, \Psi)$. This means precisely that there is some $m \in \mathbf{N}$ such that, for $n > m$, $x_n = \omega_1$. Suppose that this is not true of some sequence $\{ x_i \}_{i \in \mathbf{N}}$, that is, $x_k \neq \omega_1$ for arbitrarily large integers $k$. Then we may select some subsequence $\{ y_i \}_{i \in \mathbf{N}}$, such that $y_k \neq \omega_1$ for any $k$. But then, as the countable union of countable ordinals, $\bigcup_{i \in \mathbf{N}} y_i$ is countable, so $\omega_1 > \bigcup_{i \in \mathbf{N}} y_i \geq y_k$, for any $k$, and thus $y_i \not \to \omega_1$ in $\Omega$, hence $x_i \not \to \omega_1$.
\end{example}

Topologists were exiled from sequential paradise as soon as arbitrary unions of open sets were allowed to be open, since this allowed uncountability to enter into topology. It is still possible to find salvation, nonetheless. One solution is to remain in a class of topologies which are determined by their convergent sequences.

\begin{definition}
    Let $(X,\Omega)$ be an arbitrary topological space. We say $U \subset X$ is {\bf sequentially open} if every sequence which converges to $x \in U$ is eventually in $U$. We say $C \subset X$ is {\bf sequentially closed} if it contains all sequential limits. A {\bf sequential space} is a space $(X,\Omega)$ where sequentially open sets are open, or equivalently, if all sequentially closed sets are closed.
\end{definition}

\begin{lemma}
    Every first-countable space is sequential.
\end{lemma}
\begin{proof}
    Let $(X,\Omega)$ be a first countable space. Suppose $C$ is a sequentially closed set. Fix some limit point $x$ of $C$.  Let $\{ V_i \}_{i \in \mathbf{N}}$ be a countable neighbourhood base of $x$. Let $W_1 = V_1$, and $W_{n+1} = W_n \cap V_{n+1}$. Then $\{ W_i \}_{i \in \mathbf{N}}$ is a countable neighbourhood base such that $i < j$ implies $W_i \supset W_j$. Since $x$ is a limit point of $C$, $W_i \cap C$ is non-empty, for any $i$. We may therefore define a choice function $s: \mathbf{N} \to C$ such that $s_i \in W_i \cap C$ for any integer $i$. By construction, $s_i \to x$, so $x \in C$. It follows that $C$ is closed.
\end{proof}

\begin{corollary}
    Every metric space is sequential.
\end{corollary}

\begin{exercise}
    Let $X$ be an arbitrary set, and $(X, \Omega)$, $(X, \Psi)$ two sequential spaces. $\Omega$ and $\Psi$ are sequentially equivalent if and only if $\Omega = \Psi$.
\end{exercise}

Our second method of salvation is to generalize what we mean by a sequence of points in a topology. If we allow our sequences to be `uncountable', in a manner which will now be made precise, equivalent convergence will imply two topologies are equal.

\begin{definition}
    A {\bf directed set} is a partial ordering $(X, \preceq)$ such that, for any two $x,y \in X$, there exists $z \in X$ such that $x,y \preceq z$. A {\bf net} is a function from a directed set to an arbitrary set. Fix some net $S:X \to Y$, where $(X, \preceq)$ is directed.
    %
    \begin{enumerate}
        \item $Y \subset X$ is {\bf cofinal} if, for any $x \in X$, there is $y \in Y$ with $x \preceq y$. A net $T:E \to D$ is a {\bf subnet} of $S$ if there exists a function $f:E \to D$ where $T = S \circ f$, and for any $m \in D$, there is $n \in E$ such that if $\alpha > n$, $S(\alpha) > m$.

        \item We say $S$ is {\bf eventually} in a subset $K \subset Y$ if there is $\beta \in X$ such that for $\alpha \geq \beta$, $S(\alpha) \in K$. $S$ is {\bf frequently} in a subset $K \subset Y$ if, for every $\alpha \in X$, there is some $\beta > \alpha$ with $S(\beta) \in K$.

        \item If $(Y, \Omega)$ is a topological space, and $y \in Y$, we will say that $S$ {\bf converges to} $y$, also symbolically written $S \to x$, if $S$ is eventually in every open neighbourhood of $x$.
    \end{enumerate}
\end{definition}

\begin{example}
    The construction of the Riemann integral corresponds to a net. Let $f$ be a real-valued function defined on an interval $(a,b)$. Consider the set of all possible finite partitions of $(a,b)$ - that is, finite increasing sequences $P = (P_1, \dots, P_n)$ where $P_1 = a$ and $P_n = b$. If $P$ and $Q$ are two partitions, we define $P \preceq Q$ to mean every point in $P$ is also in $Q$. Define two nets on this directed set:
    %
    \[ \mathbf{L}(P) = \sum_{P_i} (P_{i+1} - P_i) \inf f([P_i, P_{i+1}]) \hspace{0.5in} \mathbf{U}(P) = \sum_{P_i} (P_{i+1} - P_i) \sup f([P_i, P_{i+1}]) \]
    %
    Both nets are monotone, hence they converge to some extended real value:
    %
    \[ \mathbf{L} \to \mathbf{L} \int_a^b f \hspace{1.5in} \mathbf{U} \to \mathbf{U} \int_a^b f \]
    %
    We say $f$ is integrable if $\mathbf{L} \int_a^b f = \mathbf{U} \int_a^b f$, and define the shared value to be the integral of $f$, denoted $\int_a^b f$.
\end{example}

Nets behave very similarily to sequences, and most of the proofs of these properties carry through from sequences just by replacing the objects in the proof!

\begin{lemma}
The following properties generalize from sequences:
\begin{enumerate}
    \item If a net converges to a point $x$, then every subnet of the net converges to the same point.
    \item Any monotone real-valued subnet always converges.
\end{enumerate}
\end{lemma}

Nets are the natural counterpart of sequences in topological spaces.

\begin{lemma}
    Any limit point of a set $A$ is the limit of a net.
\end{lemma}
\begin{proof}
    If $a$ is a limit point, for every neighbourhood $U$ of $a$, $U \cap A$ is non-empty. Therefore, we may define a choice function $s$ on the set of neighbourhoods of $a$, such that $s(U) \in U \cap A$, for any neighbourhood $U$ of $a$. If we make the set of neighbourhoods a directed set by the partial order $\supset$, then $S$ is a net. The fact that $S \to a$ is almost too obvious: if $U$ is a neighbourhood surrounding $a$, then for $V \subset U$, $S(V) \in U$.
\end{proof}

\begin{corollary}
    A set is closed if and only if nets valued in the set converge only to elements in the set.
\end{corollary}

\begin{corollary}
    A set is open if and only if any net converging to a point in the set eventually ends up in the set.
\end{corollary}

As we have begun by defining a topology in terms of open sets, we now have the answer we are looking for. Since the above condition is sufficient to define open sets in terms of convergence, two topologies are equal if and only if they have the same convergent nets. We should therefore expect to obtain definitions of topological properties in terms of nets.

\begin{theorem}
    A topological space is Hausdorff if and only if each net converges to no more than one point.
\end{theorem}
\begin{proof}
    If $(X,\Omega)$ is Hausdorff, and $x,y \in X$, $x \neq y$, there are disjoint sets $U,V \in \Omega$ with $x \in U$, $y \in V$. If $S: D \to X$ is a net which converges to $x$, then $x_\alpha$ cannot converge to $y$, since for some $\beta \in D$, all $S(\alpha) \in U$, for $\alpha > \beta$, so $S(\alpha) \not \in V$.

    Conversely, suppose that each net converges to no more than one point in $(X, \Omega)$. Assume every neighbourhood of $x$ intersects every neighbourhood of $y$. Let $G$ be the set of open neighbourhoods of $x$, and $H$ the open neighbourhoods of $y$, and define an ordering on $G \times H$ by defining $(U,V) \preceq (U',V')$ if $U \supset U'$ and $V \supset V'$. For any $(U,V)$, $U \cap V$ is non-empty, and we may define a choice function $S$ such that $S(U,V) \in U \cap V$. Then $S$ converges to both $x$ and $y$, so $x = y$.
\end{proof}

\begin{theorem}
    A function $f:X \to Y$ between two topological spaces is continuous if and only if, for any net $S:D \to X$ which converges to a point $x$, $f \circ S$ converges to $f(x)$.
\end{theorem}
\begin{proof}
    Suppose $f$ is continuous. Let $S$ be a net which converges to a point $x$. Let $U$ be open in $Y$. Then $f^{-1}(U)$ is open in $X$, so there is some $\beta$ such that, for $\alpha > \beta$, $S(\alpha) \in U$. But then $f \circ S (\alpha) \in f(U)$.

    Consider the converse. Let $U$ be open in $Y$, and consider $V = f^{-1}(U)$. Let $S$ be a net tending to a point $v \in V$. Then $f \circ S$ tends to $f(v) \in U$. So there is some $\alpha$ such that if $\beta \geq \alpha$, $f \circ S (\beta) \in U$. But then $S(\beta) \in V$, so $V$ is open, and $f$ is continuous.
\end{proof}

\begin{theorem}
    A topological space is compact if and only if every net in that set contains a convergent subnet.
\end{theorem}
\begin{proof}
    Suppose that $(X,\Omega)$ is compact, and let $S: D \to X$ be a net. For each $\alpha$, let $B_\alpha$ denote all $S(\beta)$, for $\beta > \alpha$. The intersection of a finite number of $B_\alpha$ is non-empty. Therefore the infinite intersection is non-empty. That is, there is some $x$ such that for any $\alpha$, there is $\beta > \alpha$ with $S(\beta) = x$. But then the restriction of $S$ to the set of all $\beta$ with $S(\beta) = x$ is a subnet, and obviously converges to $x$.

    Conversely, let $\{ C_i \}_{i \in I}$ be a collection of closed sets in $X$, such that every finite intersection of sets is non-empty. Let $\mathcal{C} = \{ C_{k_1} \cap \dots \cap C_{k_n} : k_n \in I \}$ Then we may define a choice function $S$ on $\mathcal{C}$ (made into a net by the ordering $\supset$), such that $S(C) \in C$ for any $C$. Let $S'$ be a subnet that converges to a point $x \in X$. If $x \not \in \bigcap_{i \in I} C_i$, then there is some $C_i$ with $x \not \in C_i$, and $C_i^c$ is a neighbourhood of $x$, so for some $A \in \mathcal{C}$, if $B \subset A$, $S(B) \in C_i^c$. But then $S(A \cap C_i) \in C_i^c$, which is impossible.
\end{proof}

Now suppose we have a class $\mathcal{C}$ consisting of pairs $(S,x)$, where $S$ is a net from some directed set to a specified set $X$, and $x$ a point in $X$. Let $\Omega$ consist of all subsets $U$ of $X$ such that, if $x \in U$ and $(S,x) \in \mathcal{C}$, then $S$ is eventually in $U$. Then $\Omega$ is a topology on $X$, and if $(S,x) \in \mathcal{C}$, $S \to x$ in $\Omega$. This should be simple to verify, and left to the reader. It turns out that, under more assumptions, common to convergent nets on any topological space, we can make it so that the $(S,x) \in \mathcal{C}$ are the only nets which converge in our topology $\Omega$.

\begin{definition}
    A convergence class is a class $\mathcal{C}$ consisting of pairs $(S,x)$, where $S$ is a net from some directed set to a fixed set $X$, and $x \in X$, satisfying the following properties:
    %
    \begin{enumerate}
        \item If $S$ is a constant net, always valued at $x \in X$, then $(S,x) \in \mathcal{C}$.
        \item If $(S,x) \in \mathcal{C}$, and $T$ is a subnet of $S$, then $(S,x) \in \mathcal{C}$.
        \item If $(S,x) \not \in \mathcal{C}$, then there is a subnet $T$ of $S$ such that $(U,x) \not \in \mathcal{C}$ for any subnet $U$ of $T$.
        \item Suppose $(S,x) \in \mathcal{C}$, and for each $\alpha \in \Dom(S)$, we have a net $Z_\alpha$ with $(Z_\alpha, S(\alpha)) \in \mathcal{C}$. Then $(Z,x) \in \mathcal{C}$, where $Z$ is the net defined on $\Dom(S) \times \prod_{\alpha} \Dom(Z_\alpha)$, ordered by $(\alpha, v) \leq (\beta, w)$ if $\alpha \leq v$ and $\beta \leq w$, and such that $Z(\alpha, v) = Z_\alpha(v_\alpha)$.
    \end{enumerate}
\end{definition}

\begin{theorem}
    There is a one-to-one correspondence between convergence classes on a set, and topologies on the same set whose convergent nets are exactly those that are members of the convergence class.
\end{theorem}
\begin{proof}
    Consider a convergence class $\mathcal{C}$, with nets with values in a set $X$, and define, for $A \subset X$, $\overline{A}$ equal to the set of $x \in X$ such that $(S,x) \in \mathcal{C}$, and with $S$ only having values in $A$. This forms a closure operator, defining a topology on $X$.
    %
    \begin{enumerate}
        \item $(A \subset \overline{A})$: If $a \in A$, define a constant sequence $S = (a,a,\dots)$. Then, by property 1 of a convergence class, $(S,a) \in \mathcal{S}$, so $a \in \overline{A}$.
        \item $(A \subset B) \implies (\overline{A} \subset \overline{B})$: If $a \in \overline{A}$, there is a net $S$ with $(S,a) \in \mathcal{C}$, with $S$ only taking values in $A$. But then $S$ only takes values in $B$ as well, so $a \in \overline{B}$.
        \item $(\overline{\overline{A}} = \overline{A})$: Since $A \subset \overline{A}$, $\overline{A} \subset \overline{\overline{A}}$. Now let $a \in \overline{\overline{A}}$. There is some net $S$ with $(S,a) \in \mathcal{C}$ and such that $S$ only takes values in $\overline{A}$. This means that, for each $\alpha$ upon which $S$ is defined, there is a net $Z_\alpha$ with $(Z_\alpha, S(\alpha)) \in \mathcal{C}$, only taking values in $A$. This means, by property (4) of a convergence class, that there is $(Z,a) \in \mathcal{C}$, with $Z(\alpha, v) = Z_\alpha(v_\alpha) \in A$. Thus $a \in \overline{A}$.
    \end{enumerate}

    We therefore obtain a topology $\Omega$ on $X$, such that $C$ is closed if and only if $\overline{C} = C$. Let $(S,x) \in \mathcal{C}$. Suppose $S$ does not tend to $x$ in the topology $\Omega$, so there is some open set $U$ containing $x$ such that, for any $\alpha$, there is $\beta > \alpha$ with $S(\beta) \not \in U$. The set of such $\beta$ defines a cofinal set, and thus we gain a subnet $T$, only taking values in $U^c$, and such that $(T,x) \in \mathcal{C}$ (property (2)). It then follows that $U^c$ cannot be closed, since $T$ tends to $x \not \in U^c$.

    Conversely, suppose $S$ tends to a point $x$ in the topology $\Omega$, but $(S,x) \not \in \mathcal{C}$. Pick a subnet $T:D \to X$ for which $(U,x) \not \in \mathcal{C}$ for any subnet $U$. For each $\alpha \in D$, let $B_\alpha = \{ \beta \in D : \beta > \alpha \}$. Then $x \in \overline{T(B_\alpha)}$, for each $\alpha$, since $T$ converges according to $\Omega$ to $x$, so there is some net $Z_\alpha : E_\alpha \to B_\alpha$ such that $(T \circ Z_\alpha, x) \in \mathcal{C}$. Then, by (4), there is a net $Z$ such that $Z(\alpha, v) = Z_\alpha(v_\alpha)$ and $(T \circ Z,x) \in \mathcal{C}$. It turns out that $Z$ is a subnet of $S$, since given $\alpha \in D$, if $\beta > \alpha$, $Z(\beta, v) \in B_\beta$, so $Z(\beta, v) > \alpha$. We conclude our argument with this contradiction.
\end{proof}







\section{s}

Surely this specification must involve the circles we were discussing as our fundamental example. Of course, these are not the only open sets, since the union of two balls is not necessarily an open ball, but we may use these as the fundamental sets from which all open sets are constructed.

The intuitive topology on $\mathbf{R}$ can be generalized to any linearly ordered set. In most cases, we may take the same open intervals. One case is more complicated; if the set contains a minimum or maximum element, then these elements are contained in no open intervals. We fix this by allowing in combination with open sets half closed rays $[-\infty, x)$ and $(x, \infty]$ into the definition of openess. We call the topology defined the order topology on the ordered set.

If $X$ is a topological space, and $Z$ is a subset, we can see intuitively how $Z$ may inherit the notion of space from $X$. We take open sets in $Z$ to be the intersection of open sets in $Z$ with $Z$, and we call this the subspace topology.

As examples, the topology $\mathbf{Z}$ inherits from the order topology on $\mathbf{R}$ the discrete topology (which is why we think of integers as being separated on the real line). Conversely, the set $\{ \frac{1}{n}: n \in \mathbf{N} \} \cup \{ 0 \}$ inherits a completely different topology (zero is no longer open on its own).

Most of the time, when specifying a set, it is difficult to specify precisely the set of all open sets that define a topology. Since, when $\{ C_i \}$ is a indexed set of topology, $\bigcap_{i \in I} C_i$ is also topology, we may, when given a subcollection of the power set of a set, generate a topology on that subset by taking the smallest topology which contains the subcollection. Special collections of subs
ets $\mathcal{C}$ with the following properties are of increased importance:
%
\begin{enumerate}
    \item Every element in the space is contained in one of the subsets
    \item If $x$ is contained in two sets $C_1$ and $C_2$ in $\mathcal{C}$, there is a third set $C_3$ in $\mathcal{C}$ such that $x \in C_3 \subset C_1 \cap C_2$.
\end{enumerate}
%
Then the topology generated has the property that a set $A$ is open if and only if, for any element $x$ in $A$, there is a set $C$ in $\mathcal{C}$ containing $x$ such that $C \subseteq A$. We call such a collection a basis for the topology. Specific examples include open intervals in $\mathbf{R}$. We say $\mathcal{C}$ covers the topology it generates.

Unlike in linear algebra, a basis for a topology is not unique up to bijection. We cannot even always find a minimal basis for all topologies in terms of containment (consider open balls in $\mathbf{R}^n$). We can only promise that there is a basis with minimum cardinality, which exists due to the well ordering property of cardinal numbers.

\begin{theorem}
    Let $X$ be a topological space, and $Y$ a subset with the subspace topology. Then a subset $A$ is closed in $Y$ if and only if $A = B \cap Y$ for a closed set $B$ in $X$.
\end{theorem}
\begin{proof}
    Suppose $A = B \cap Y$ as in the theorem's statement. Then $X - B$ is open in $X$, so $(X - B) \cap Y$ is open in $Y$, and this set is just $Y - B$. But this $B \cap Y$, which is the complement of $Y - B$, is closed in $Y$.

    Conversely, if $A$ is closed in $Y$, $Y - A$ is open in $Y$, hence $Y - A = V \cap Y$ for some open set $V$ in $X$. Since $V^c \cap Y = A$, we see that $A$ is the intersection of $Y$ with a closed set in $X$.
\end{proof}




\chapter{Distances}

In topology, we escrew the concept of an exact distance -- we care only about the distance that lies infinitely small between different objects. Nonetheless, distances are more consistant with out intuitions about space, especially in the context of the euclidean plane and analysis, so it is useful to be able to construct topological space using an abstract definition of distance.

\begin{definition}
    A {\bf Metric} on a set $X$ is a function $d: X \times X \to \mathbf{R}$ such that for any $x,y,z \in X$
    %
    \begin{enumerate}
        \item (Nondegeneracy) $d(x,y) \geq 0$, and $d(x,y) = 0$ if and only if $x = y$.
        \item (Symmetry) $d(x,y) = d(y,x)$.
        \item (The Triangle Inequality) $d(x,z) \leq d(x,y) + d(y,z)$.
    \end{enumerate}
    %
    Given a subset $A$ of $X$, we define $d(x,A) = \inf\ \{ d(x,a) : a \in A \}$, and the {\bf diameter} of $A$ to be $\text{diam}(A) = \sup \{ d(x,y) : x,y \in A \}$. A set is {\bf bounded} if its diameter is finite. The {\bf open ball} of radius $r$ and center $x$ is $B_r(x) = \{ y \in X : d(x,y) < r \}$, and the {\bf closed ball} $D_r(x) = \{ y \in X : d(x,y) \leq r \}$. If we take the set of all open balls to be the basis of a topological space, we obtain a topological structure -- a {\bf metric space}.
\end{definition}

\begin{example}
    The canonical metric on $\mathbf{R}^n$ is the euclidean metric
    %
    \[ d(x,y) = \sum_{i = 1}^n (x_i - y_i)^2 \]
    %
    But we may consider many different metrics on $\mathbf{R}^n$.
    %
    \[ d(x,y) = \sum_{k = 1}^n |y_i - x_i| \]
    %
    The open balls of this metric are shaped like diamonds; the metric
    %
    \[ d(x,y) = \max_{i = 1,\dots,n} |y_i - x_i| \]
    %
    induces balls shaped like squares whose sides are oriented to the axes. The topologies induced by all three metrics are the same.
\end{example}

\begin{example}
    On any set $X$, take $d(x,y) = \delta_{x,y}$. The topology induced is discrete.
\end{example}

\begin{theorem}
    Every metric space is normal
\end{theorem}
\begin{proof}
    Let $A$ and $B$ be two closed, disjoint sets in a metric space. Consider $U = \{ x : d(x,A) < d(x,B) \}$, and $V = \{ x : d(x,B) < d(x,A) \}$. Then $U$ and $V$ are disjoint open sets (since $x \mapsto d(x,A)$ is continuous), with $U$ containing $A$ and $V$ containing $B$.
\end{proof}

\begin{theorem}
    A subset of a complete metric space is precompact if and only if it is totally bounded.
\end{theorem}
\begin{proof}
    Let $X$ be a totally bounded metric space. Let $\{ x_i \}$ be a sequence in $X$. Choose a finite covering of $X$ by balls of radius 1. Select a subsequence $x_{1,i}$ of $x_i$ which lies in some specific ball $B_1$. Cover $B_1$ by a finite covering of radius $1/2$, and take a subsequence $x_{2,i}$ of $x_{1,i}$. Proceed inductively, considering subsequences $x_{n,i}$. Then $x_{i,i}$ is a cauchy sequence in $X$, and we have shown $X$ is precompact.
\end{proof}

\begin{theorem}
    Let $X$ be a compact metric space, and let $\{ U_i \}$ be an open cover. Then the cover admits a {\bf Lebesgue number} $\delta$ such that if $\text{diam}(Y) < \delta$, then $Y \subset U_i$.
\end{theorem}
\begin{proof}
    Let $U_1, \dots, U_n$ be a finite subcover of $X$, and let $C_i = X - U_i$. Define
    %
    \[ f(x) = \frac{1}{n} \sum_{i = 1}^n d(x,C_i) \]
    %
    Then $f$ is continuous, and is always positive. Since $f$ is defined on a compact set, it attains a minimum value $\delta > 0$. Suppose $Y$ is a set with $\text{diam}(Y) < \delta$, containing points $y_1, \dots y_n$ in each $C_i$. But then
    %
    \[ f(y_1) = \frac{1}{n} \sum_{i = 1}^n d(x,C_i) = \frac{(n-1)}{n} \delta < \delta \]
    %
    A contradiction which proves the claim.
\end{proof}







\chapter{Constructions}

Here we get to the visually interesting part of topology, providing methods to mold and curve the topological structures of your choosing, gluing, stretching, and all other kinds of fun stuff. It will explain how we get from a plane to a torus, or from $\mathbf{R}^2$ to $S^2$. The construction we will make can be shown in a very general manner.

\begin{definition}
    Let $X$ be a topological space, and $\sim$ an equivalence relation on $X$. From this equivalence relation, we form the set $X/\sim$, and consider the projection mapping $\pi$ from $X$ to $X/\sim$. The quotient topology on $X/\sim$ is the coarsest topology that makes $\pi$ continuous, and makes $X/\sim$ the quotient space of $X$ and $\sim$. A set $A$ is open in $X/\sim$ if and only if $\pi^{-1}(A)$ is open in $X$.
\end{definition}

If we have a surjective map $f:X \to Y$, where $X$ is some topological space, and $Y$ is any set, then we may construct a topology on $Y$ analogous to the quotient topology. First we consider the fibers of $X$ relative to the mapping $f$, and identify the quotient topology on this set. We obtain a bijective mapping $\overline{f}$. The quotient topology on $Y$ is the topology which makes $\overline{f}$ a homeomorphism. Since, in the context of topology, homeomorphisms preserve all important properties, we may as well consider this definition no different from the definition in terms of equivalence relations.

\chapter{Algebraic Topology}

To verify that two topological spaces are homeomorphic, we need only find a single homeomorphism that connects the two spaces. On the contrary, to verify that two topological spaces are not homeomorphic, we need to somehow show that every function from one space to the other is not a homeomorphism, a computationally intractable problem. One trick we can use to separate topological spaces is to find fundamental topological properties which distinguish two topological spaces. Connectedness, Compactness, and Hausdorffiness are all preserved by homeomorphism, as does the topological properties of subspaces. Nonetheless, sometimes these properties are not enough to distinguish two spaces. This chapter shows a deep technique which is often useful for characterizing spaces.

Consider two functions $f$ and $g$ between topological spaces $X$ and $Y$. Though $f$ might not be equal to $g$, they may be in some sense topologically equal -- we may be able to deform one to the other in a continuous fashion. This is a homotopy.

\begin{definition}
    Let $f,g: X \to Y$ be two continuous functions. Define a topology on $Hom(X,Y)$ as a subspace of $Y^X$, which can be viewed as the product topology of $Y$ with itself $Y$ times. Then $f$ and $g$ are homotopic if there exists a path in $Hom(X,Y)$ which connects $f$ to $g$. Alternatively, these two functions are homotopic if there exists a continuous function $F:[0,1] \times X \to Y$ such that for all $x$, $F(0,x) = f(x)$, and $F(1,x) = g(x)$.
\end{definition}

The fact that homotopy is an equivalence relation will allow us to distill functions between spaces to their fundamental properties. We need to specialize our definition for it to be more of more use to us.

\begin{definition}
    Two paths in $X$ are path homotopic if they have the same start and end point, and are homotopic to each other.
\end{definition}

Let $f$ and $g$ be two paths in $X$, where the end point of $f$ is the start point of $g$. Then we may compose the two paths to form a new path $f * g$, defined by
%
\[ (f * g)(x) = f(2x): x \in [0,1/2]
                g(2x - 1): x \in [1/2,1] \]
%
By the pasting lemma, this function is a path which connects the start point of $f$ to the end point of $g$. Unfortunately, concatenation is not associative, we do not have that $f * (g * h) = (f * g) * h$. These paths are homotopic to each other, however, and moving to path homotopy classes makes the definition much simpler.

\begin{theorem}
    Let $f$ be path homotopic to $f'$, and $g$ path homotopic to $g'$. Then $f * g$ is homotopic to $f' * g'$.
\end{theorem}
\begin{proof}
    Let $F$ be the path homotopy from $f$ to $f'$, and $G$ the path homotopy from $g$ to $g'$. Define a homotopy $H$ between $f * g$ and $f' * g'$ by
    %
    \[ H(\cdot,y) = F(\cdot, y) * G(\cdot, y) \]
    %
    More specifically
    %
    \[ H(x,y) = \begin{cases}
        F(2x,y) & \text{if } x \in [0,1/2]\\
        G(2x - 1,y) & \text{if } x \in [1/2,1]\\
\end{cases} \]
    %
    The pasting lemma guarentees this function is a new homotopy.
\end{proof}

We now consider homotopy classes of paths, so when we talk about a path $f$, we are really talking about all paths homotopic to $f$.

\begin{theorem}
    $[f] * ([g] * [h]) = ([f] * [g]) * [h]$.
\end{theorem}

\chapter{Compactification}

In the following, we consider only locally connected, locally compact, connected Hausdorff spaces.

\begin{definition}
    An {\bf end} of a $X$ is a map $\varepsilon$ defined on compact subsets of $X$, such that $\varepsilon(C)$ is a connected component of $X - C$ for each compact $C$, and $C \subset D$ implies $\varepsilon(D) \subset \varepsilon(C)$. Denote the set of all ends on $X$ by $\mathcal{E}(X)$.
\end{definition}

\begin{definition}
    The end compactification of a space $X$ is the space $\mathbf{X} = X \cup \mathcal{E}(X)$, where a set is open if it is open in $X$, or if it is of the form $U_{\varepsilon(C)} := \varepsilon(C) \cup \{ \varepsilon' \in \mathcal{E}(X) : \varepsilon'(C) = \varepsilon(C) \}$, where $C$ is compact.
\end{definition}

\begin{lemma}
    The end compactification is Hausdorff.
\end{lemma}
\begin{proof}
    If $\varepsilon, \varepsilon \in \mathcal{E}(X)$ are two unequal ends, then there is some compact set $C$ for which $\varepsilon(C) \neq \varepsilon'(C)$. But then $U_{\varepsilon(C)}$ and $U_{\varepsilon'(C)}$ are disjoint. If $x, y \in X$, then they can surely be separated in the end compactification because $X$ is Hausdorff. If $x$ is a point in $X$, and $\varepsilon$ is an end, then because $X$ is locally compact, $x$ possesses a precompact neighbourhood $V$ of $x$, and then $U_{\varepsilon(\overline{V})}$ is disjoint from $V$.
\end{proof}

\begin{lemma}
    $\mathbf{R}$ has two ends, the `left' and `right' ends.
\end{lemma}
\begin{proof}
    $\mathbf{R} = \bigcup_{n = 1}^\infty [-n, n]$, and each $[-n, n]$ is compact. We contend every end $\varepsilon$ on $\mathbf{R}$ is defined by its action on $[-n, n]$. If $C$ is any compact set, then $C$ is contained in an interval of the form $[-n, n]$. Clearly, $\varepsilon(C)$ must be the unique connected extension of $\varepsilon([-n, n])$, since $\varepsilon(C) \supset \varepsilon([-n,n])$. In fact, $\varepsilon$ is defined solely by its action on $[-1,1]$, since $[-1,1] \subset [-2,2] \subset \dots$. Since the two choices $\varepsilon([-x,x]) = (-\infty, x)$ and $\varepsilon([-x,x]) = (x,\infty)$ constitute ends, the space has two ends.
\end{proof}

In general, if a space $X$ can be written as $C_1 \subset C_2 \subset \dots \to X$, where each $C_i$ is compact, then all ends are defined by their action on $C_1$. We shall call such a space {\bf hemicompact}. Not all choices of components of $X - C_1$ will work, however.

\begin{lemma}
    The end compactification of a hemicompact space is compact.
\end{lemma}
\begin{proof}
    s
\end{proof}

\end{document}