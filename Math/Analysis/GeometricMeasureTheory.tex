\documentclass[12pt]{report}

\usepackage{amsmath}
\usepackage{amssymb}
\usepackage{amsthm}
\usepackage{amsopn}
\usepackage{kpfonts}
\usepackage{graphicx}
\usepackage{kbordermatrix}
\usepackage{tikz}
\usetikzlibrary{arrows, petri, topaths}%
\usepackage{tkz-berge}
\usepackage{multicol}

\usepackage{framed}
\usepackage{mathtools}
\usepackage{float}
\usepackage{subfig}
% \usepackage{cmbright}

\theoremstyle{plain}
\newtheorem{theorem}{Theorem}[chapter]
\newtheorem{lemma}[theorem]{Lemma}
\newtheorem{corollary}[theorem]{Corollary}
\newtheorem{prop}[theorem]{Proposition}
\newtheorem{exercise}{Exercise}[chapter]

\newtheorem*{example}{Example}
\newtheorem*{proof*}{Proof}

\theoremstyle{definition}
\newtheorem*{defi}{Definition}
\newenvironment{definition}
    {\begin{samepage}\begin{framed}\begin{defi}}
    {\end{defi}\end{framed}\end{samepage}}





\usepackage{hyperref} 
\hypersetup{
    colorlinks = true,
    linkcolor = black,
}

\makeatletter
\renewcommand*\env@matrix[1][*\c@MaxMatrixCols c]{%
  \hskip -\arraycolsep
  \let\@ifnextchar\new@ifnextchar
  \array{#1}}
\makeatother

\renewcommand*\contentsname{\hfill Table Of Contents \hfill}

\newcommand{\optionalsection}[1]{\section[* #1]{(Important) #1}}
\newcommand{\deriv}[3]{\left. \frac{\partial #1}{\partial #2} \right|_{#3}}

\title{Geometric Measure Theory}
\author{Jacob Denson}

\begin{document}

\pagenumbering{gobble}

\maketitle

\tableofcontents

\pagenumbering{arabic}

\chapter{Fractal Dimensions}

The expression of geometric properties of subsets of $\mathbf{R}^d$ requires more than can be expressed using the Lebesgue measure. For instance, curves and surfaces all have measure zero in two and three dimensions respectively, and thus we cannot distinguish them by the Lebesgue measure from any of the other nasty Lebesgue measurable subsets of measure zero. Hausdorff showed that there is a notion of `dimension' of measure zero subsets of $\mathbf{R}^d$ which matches the dimension of corresponding curves and surfaces. Even more interestingly, Hausdorff's theory of dimension gives certain fractal subsets non-integer dimension. It is very useful when studying non-smooth shapes, like fractals.

%Here is the general idea. If $X = [0,1)$ is a unit interval, then $nX = [0,n)$ is the union of $n$ disjoint translates of $[0,1)$. If we instead consider the unit square $X = [0,1) \times [0,1)$, then $nX = [0,n) \times [0,n)$ is the union of $n^2$ disjoint translates of $[0,1)$. If $X$ is a unit cube, the $nX$ is the union of $n^3$ disjoint translates of $[0,1)$, and so on and so forth. Thus, it makes sense to define the dimension of $X$ to be the value $\alpha$ such that $nX$ is the union of $n^\alpha$ disjoint copies of $X$. Note that if $X$ is the Cantor set, then $3X$ is the union of two translates of $X$, so our previous intuition would be willing to say that the Cantor set has `dimension' $\log_3 2 = 0.6309\dots$.

\section{Minkowski Dimension}

The easiest fractal dimension to introduce is Minkowski dimension. If $E$ is a bounded set in $\mathbf{R}^n$, then we can consider the open set $E_\delta$, which is an `$\delta$ thickening' of $E$. We define the \emph{upper} and \emph{lower} Minkowski dimension as
%
\[ \upminkdim(E) = \limsup_{\delta \to 0} n - \frac{\log|E_\delta|}{\log \delta}\quad\text{and}\quad \lowminkdim(E) = \liminf_{\delta \to 0} n - \frac{\log|E_\delta|}{\log \delta}. \]
%
If $\upminkdim(E) = \lowminkdim(E)$, then we refer to this common quantity as the Minkowski dimension $\minkdim(E)$. One can interpret that if $\minkdim(E) = \alpha$, then $|E_\delta| = \delta^{n - \alpha + o(1)}$. This means that for sufficiently small $\delta$, if $\dim_M(E) = \alpha$, then $|E_\delta| = \delta^{n - \alpha + o(1)}$. These notions can also be extended to unbounded sets by considering the supremum over all bounded subsets, by setting for a set $E$,
%
\begin{align*}
	\upminkdim(E) &= \limsup_{r \to \infty} \upminkdim(E \cap B(0,r)),\\
	\lowminkdim(E) &= \liminf_{r \to \infty} \lowminkdim(E \cap B(0,r)),.
\end{align*}
%
The Minkowski dimension $\minkdim(E)$ is defined as the common value of these two functions, if they agree.

\begin{example}
	If $E = B^k \times \{ 0 \}^{n-k}$, where $B^k$ is the $k$ dimensional unit ball, then
	%
	\[ B^k \times \delta B^{n-k} \subset E_\delta \subset (1 + \delta)B^k \times \delta B^{n-k} \]
	%
	which shows that
	%
	\[ \delta^{n-k} \lesssim |E_\delta| \lesssim (1 + \delta)^k \delta^{n-k} \]
	%
	Thus $\minkdim(E) = k$. In particular, $\minkdim(r E) = k$ for all $r > 0$, and so taking $r \to \infty$ shows $\minkdim(\RR^k \times 0^{n-k}) = k$.
\end{example}

\begin{example}
	Let
	%
	\[ C = \left\{ \sum_{i = 1}^\infty a_i/4^i : a_i \in \{ 0, 3 \} \right\} \]
	%
	be a Cantor set. If $1/4^{N+1} \leq \delta \leq 1/4^N$, then
	%
	\[ \left\{ \sum_{i = 1}^\infty a_i/4^i : a_1, \dots, a_{N+1} \in \{ 0, 3 \} \right\} \subset C_\delta \subset \left\{ \sum_{i = 1}^\infty a_i/4^i : a_1, \dots, a_N \in \{ 0, 3 \} \right\}. \]
	%
	The latter set has volume $2^N/4^N = 1/2^N \leq (2\delta)^{1/2}$. The initial set has volume $2^{N+1}/4^{N+1} = 1/2^{N+1} \geq \delta^{1/2}$. Thus $\log |C_\delta| = \log(\delta) / 2 + O(1)$, and so $C$ has Minkowski dimension $1/2$.
\end{example}

\begin{example}
	We can modify the last example slightly, considering
	%
	\[ C = \left\{ \sum_{i = 1}^\infty a_i/4^i : a_i \in \{ 0, 3 \}\ \text{if there is $k$ s.t.}\ (2k)! \leq i \leq (2k+1)! \right\}. \]
	%
	Then $C$ has lower Minkowski dimension $1/2$ and upper Minkowski dimension 1. If one looks at the iterated construction of $C$, one sees that we only dissect $C$ at an incredibly sparse range of scales.
\end{example}

\begin{example}
	Let $S = \{ (x,\sin(1/x)) : 0 < x \leq 1 \}$. Then $S$ has Minkowski dimension $3/2$. Consider a fixed scale $\delta$. For any $x \in (0,1)$, Let $f(x) = \sin(1/x)$. Then $|f'(x)| \leq 1/x^2$, so for any fixed $x_0$, the length of the vertical segment $S_\delta \cap \{ x = x_0 \}$ is at most $2\delta / (x_0 - \delta)^2$. In particular, we may cover $S_\delta \cap [0,1] \times [-1,1]$ by an initial cube $[0, \delta^\alpha] \times [-1,1]$ for $\alpha < 1$, and then an integral over the bound obtained for the lengths of the vertical segments. Thus
	%
	\[ |S_\delta \cap [0,1] \times [-1,1]| \leq 2\delta^\alpha + \int_{\delta^\alpha}^1 \frac{2\delta}{(x_0 - \delta)^2} \lesssim \delta^\alpha + \delta^{1-\alpha}. \]
	%
	Choosing $\alpha = 1/2$ gives $|S_\delta| \lesssim \delta^{1/2}$. But $S_\delta$ certainly contains $[0,\delta^{1/2}] \times [-1,1]$, which gives $|S_\delta| \gtrsim \delta^{1/2}$. In particular, taking limits shows this estimate is enough to conclude $S$ has Minkowski dimension $3/2$.
\end{example}

Many fractals display self similarity properties. For instance, if $C$ is the classical Cantor set, then $3C$ is the union of two translates of $C$. The next lemma thus implies the Minkowski dimension of the Cantor set is $\log_3(2)$.

\begin{theorem}
	If $E$ is compact, $r > 1$, and there is $r$ such that $rE$ is the union of $k$ disjoint translates of $E$, then $\dim_M(E) = \log_r k$.
\end{theorem}
\begin{proof}
	For small $\delta$, $(rE)_\delta$ is the union of $k$ disjoint translates of $E_\delta$, so
	%
	\[ r^d |E_{\delta/r}| = |(rE)_\delta| = k |E_\delta|. \]
	%
	In particular, this means that $|E_{1/r^N}|$ is proportional to $(k/r^d)^N$. But this means that for any $1/r^{N+1} \leq \delta \leq 1/r^N$,
	%
	\[ |E_\delta| \sim (k/r^d)^N \sim (k/r^d)^{-\log_r \delta} = \delta^{d - \log_r k}. \]
	%
	Thus $\dim_M(E) = \log_r k$.
\end{proof}

There are several alternate definitions of Minkowski dimension. Given a bounded set $E$, and $\delta > 0$, we let
%
\begin{itemize}
	\item $N^\text{Ext}_\delta(E)$ denote the minimum number of $\delta$ balls required to cover $E$.
	\item $N^\text{Int}_\delta(E)$ denotes the minimum number of $\delta$ balls with centers in $E$ required to cover $E$.
	\item $N^\text{Pack}_\delta(E)$ is the largest number of disjoint open balls of radius $\delta$ with centers in $E$.
\end{itemize}
%
Given a cover of $E$ by $N$ balls of radius $\delta$, by doubling the radius of the balls, we can cover $E$ by $N$ balls of radius $2\delta$ with centers of $E$. Thus $N^\text{Int}_{2\delta}(E) \leq N^\text{Ext}_\delta(E) \leq N^{\text{Int}}_\delta(E)$. Conversely, if we have a maximal packing by $N$ radius $\delta$ balls, then we can cover $E$ by $N$ radius $2\delta$ balls. Thus $N^\text{Int}_{2\delta}(E) \leq N^\text{Pack}_\delta(E)$. On the other hand, $N^\text{Pack}_\delta(E) \leq |E_\delta|/|\delta B^n| \leq N_\delta^\text{Ext}(E)$, because a packing of balls inside $E$ provides a disjoint subset of balls in $E_\delta$, and if we cover $E$ by $\delta$ balls, then $E_\delta$ is covered by the radius $2\delta$ balls with the same centres. In particular, we have shown that as $\delta \to 0$, all the quantities $\log_{1/\delta} N^*_\delta(E)$ are comparable to one another. Since
%
\[ \delta^n |N_\delta^{\text{Pack}}(E)| \lesssim |E_\delta| \lesssim \delta^n |N_\delta^{\text{Ext}}(E)| \]
%
We find that
%
\[ \underline{\dim}_M(E) = \liminf_{\delta \to 0} \frac{\log N_\delta^*(E)}{\log(1/\delta)}\ \ \ \ \overline{\dim}_M(E) = \limsup_{\delta \to 0} \frac{\log N_\delta^*(E)}{\log(1/\delta)}. \]
%
These definitions are quite useful, because they can be defined for subsets of an arbitrary metric space.

\section{Hausdorff Dimension}

Hausdorff dimension is a more stable version of fractal dimension which is obtained by finding a canonical `$s$ dimensional measure' $H^s$ on $\mathbf{R}^n$ for each $s$, and then setting the dimension of $E$ to be the supremum of $s$ such that $H^s(E) < \infty$. A naive way to construct is to assign a mass $r^s$ to each radius $r$ ball in $\mathbf{R}^n$, and then define
%
\[ H^s_\infty(E) = \inf \left\{ \sum r_k^s : E \subset \bigcup B(x_k,r_k) \right\} \]
%
This is an outer measure, and so Caratheodory's extension theorem gives a $\sigma$ algebra of measurable sets. Unfortunately, not even intervals are measurable with respect to this $\sigma$ algebra, for non-integer values of $s$.

\begin{example}
	Let $s = 1/2$, and let $E = (a,b)$. On one hand, $H^s_\infty(E) \leq [(b-a)/2]^{1/2}$. On the other hand, if $(a,b)$ is covered by balls $B(x_k,r_k)$, then $\sum 2r_k \geq b - a$, so applying the concavity of $x \mapsto x^{1/2}$, we conclude
	%
	\[ \sum r_k^{1/2} \geq \left( \sum r_k \right)^{1/2} \geq \left( \frac{b - a}{2} \right)^{1/2} \]
	%
	Thus $H^s_\infty(E) = [(b-a)/2]^{1/2}$. But now we see that the additivity property begins to breakdown, since $H^{1/2,\infty}[0,1] = 2^{-1/2}$, whereas $H^{1/2,\infty}[0,1/2] = H^{1/2,\infty}[1/2,1] = 1/2$, and so $H^{1/2,\infty}[0,1] < H^{1/2,\infty}[0,1/2] + H^{1/2,\infty}[1/2,1]$.
\end{example}

The reason why intervals fail to be measurable is that $[0,1]$ is most efficiently coverable by a single large ball, rather than covering the set by the two intervals $[0,1/2]$ and $[1/2,1]$. We can fix this by limiting the Hausdorff measure to be the value of the most efficient cover by arbitrarily small balls.

For a subset $E$ of Euclidean space, we define
%
\[ H_\delta^s(E) = \inf \left\{ \sum_{n = 1}^\infty \text{diam}(B_n)^s : E \subset \bigcup_{n = 1}^\infty B_n, \text{diam}(B_n) \leq \delta \right\} \]
%
We then define $H^s(E) = \lim_{\delta \to 0} H_\delta^s(E)$. Then $H^s$ is an exterior measure, and $H^s(E \cup F) = H^s(E) + H^s(F)$ if $d(E,F) > 0$. Thus all Borel sets are measurable with respect to $H^s$, which is certainly more satisfactory than the last definition.

\begin{remark}
	Though not all sets are measurable with respect to $H^s_\infty$. Nonetheless, since the values $H^s_\delta(E)$ increase to the value $H^s(E)$, if $H^s(E) = 0$, then $H^s_\delta(E) = 0$ for all $\delta > 0$. Thus $H^s_\infty(E) = 0$. Conversely, if $H^s_\infty(E) = 0$, and $E$ is compact, then $H^s_\delta(E) = 0$ for all $\delta > 0$. To fix the compactness condition, we known $H^s_\infty(E \cap [-R,R]) = 0$ for all $R$, so $H^s(E \cap [-R,R]) = 0$, and then
	%
	\[ H^s(E) = \lim_{R \to \infty} H^s(E \cap [-R,R]) = 0. \]
	%
	Thus though the $\sigma$ algebra of measurable sets with respect to $H^s$ and $H^s_\infty$ may disagree, the null sets do agree.
\end{remark}

\begin{example}
	Let $s = 0$. Then $H_\delta^0(E) = N_\delta^{\text{Ext}}(E)$, which tends to $\infty$ as $\delta \to 0$ unless $E$ is finite, and then $H_\delta^0(E) \to \# E$. Thus $H^0$ is just the counting measure.
\end{example}

\begin{example}
	Let $s = n$. If $E$ has Lebesgue measure zero, then for any $\varepsilon > 0$, there exists countable many balls $B(x_k,r_k)$ covering $E$ with $\sum r_k^n < \varepsilon$. Then $r_k < \varepsilon^{1/n}$, so $H^n_{\varepsilon^{1/n}}(E) < \varepsilon$. Letting $\varepsilon \to 0$, we conclude $H^n(E) = 0$. Thus $H^n$ is absolutely continuous with respect to the Lebesgue measure. The measure $H^n$ is translation invariant, so $H^n$ is actually a constant multiple of the Lebesgue measure. We let the constant multiple be defined $1/\omega_n$. The value $\omega_n$ can be defined as the volume of a unit ball in $\mathbf{R}^n$, since $H^n(B) = 1$ if $B$ is a unit ball.
\end{example}

The same argument shows that if $V$ is an $m$ dimensional subspace of $\mathbf{R}^n$, then $H^m$, restricted to subsets of $V$, is a constant multiple of the $m$ dimensional Lebesgue measure on $V$. More generally, $H^m$ measures the $m$ dimensional surface area of smooth, $m$ dimensional submanifolds of $\mathbf{R}^n$.

\begin{theorem}
	Let $U$ be an open subset of $\mathbf{R}^d$, and let $\phi: U \to \mathbf{R}^n$ be a smooth immersion. Then for any compact set $E$,
	%
	\[ H^d(\phi(E)) \propto \frac{1}{\omega_d} \int_E J(x)\; dx \]
	%
	where $J(x)$ is the square root of the sums of squares of the $d \times d$ minors of $D\phi(x)$.
\end{theorem}
\begin{proof}
	We may cover $E$ by finitely many open sets $U_1, \dots, U_N$, together with coordinate charts $y_1, \dots, y_N$ such that $(y_k \circ \phi)(x) = (x,f_k(x))$ for some smooth $f_k$, and fix $J_k$ such that for any $x \in U_k$, $|J(x) - J_k| < \varepsilon$. TODO: PROVE REST OF THEOREM.
\end{proof}

\begin{lemma}
	If $t < s$ and $H^t(E) < \infty$, $H^s(E) = 0$, and if $H^s(E) = \infty$, $H^t(E) = \infty$.
\end{lemma}
\begin{proof}
	If, for any cover of $E$ by balls $B(x_k,r_k)$, $\sum r_k^t \leq A$, and $r_k \leq \delta$, then $\sum r_k^s \leq \sum r_k^{s-t} r_k^t \leq \delta^{s-t} A$. Thus $H^s_\delta(E) \leq \delta^{s-t} A $, and taking $\delta \to 0$, we conclude $H^s(E) = 0$. The latter point is just proved by taking contrapositives.
\end{proof}

Thus given any Borel set $E$, there is $s$ such that $H^{s_0}(E) = 0$ for $s_0 < s$, and $H^{s_1}(E) = \infty$ for $s_1 > s$. We refer to $s$ as the Hausdorff dimension of $E$, denoted $\dim_H(E)$.

\begin{example}
	Consider $S = \{ (x,\sin(1/x)) : 0 < x \leq 1 \}$. Then for each $\delta > 0$, the set $S \cap [\delta,1] \times \mathbf{R}$ is the image of a smooth curve, and therefore has Hausdorff dimension $1$. Thus for any $\varepsilon > 0$, $H^{1 + \varepsilon}(S \cap [\delta,1] \times \mathbf{R}) = 0$. But then taking limits as $\delta \to 0$, we conclude $H^{1+\varepsilon}(S) = 0$. Since $H^1(S) > 0$, this shows $S$ has Hausdorff dimension 1. Compare this to the Minkowski dimension $3/2$ result we obtained previously.
\end{example}

An easy way to compare the approaches to fractal dimension given by Minkowski and Hausdorff dimension is that Minkowski dimension measures the efficiency of covers of a set at a fixed scale, whereas Hausdorff dimension measures the efficiency of covers of a set at various, small scales.

\section{Energy Integrals and Frostman's Lemma}

By taking particular covers of a set, it is easy to upper bound the Hausdorff dimension of a set. On the other hand, finding a lower bound is a little more tricky. One method to finding to lower bound is constructing a measure on our set with a certain `measure' property. We say a finite Borel measure $\mu$ is a Frostman measure with dimension $\alpha$ if $\mu(B(x,r)) \lesssim r^\alpha$. For a set $E$, we let $M(E)$ denote all Borel measures supported on $E$.

\begin{theorem}[Frostman's Lemma]
	Let $0 \leq s \leq n$. For a compact set $E$, the Hausdorff dimension $H^\alpha(E) > 0$ if and only if there is an $\alpha$ dimensional Frostman measure supported on $E$. In particular
	%
	\[ \hausdim(E) = \sup \{ \alpha \geq 0: \text{there is an $\alpha$ dimensional measure}\ \mu \in M(E) \} \]
\end{theorem}
\begin{proof}
	Suppose $H^s(E) > 0$. Without loss of generality, assume $E \subset [0,1)^n$. We work dyadically. For each $k$, let $\mathcal{Q}_k$ denote the set of all cubes of the form $[a,a+1/2^k]$, with $a \in \mathbf{Z}/2^k$. A cube in $\bigcup \mathcal{Q}_k$ is known as a dyadic cube. We can define the $s$ dimensional dyadic Hausdorff exterior measure $H^s_{\Delta,\delta}$ as the exterior measure obtained by restricting to coverings by Dyadic cubes with sidelength bounded by $\delta$, and a cube in $\mathcal{Q}_k$ is assigned mass $1/2^{sk}$. The measure $H^s_\Delta$ is then obtained by taking limits. It is not difficult to show that there are universal constants such that $H^s_\Delta$ is comparable to $H^s$ for all $s$. We now construct a subadditive premeasure $\mu^+$ by defining $\mu^+(Q) = H^s_{\Delta,2^{-k}}(E \cap Q)$ for each dyadic $Q$. Then $\mu^+([0,1)^n) \geq H^s_\Delta(E) > 0$, and we can apply the Caratheodory extension theorem to extend the measure to all Borel sets (since all open sets are the countable union of dyadic cubes). Note that if $Q \in \mathcal{Q}_k$, then covering $E \cap Q$ by $Q$ gives $\mu^+(Q) \leq 1/2^{-sk}$. But we can find an additive measure $\mu$ on dyadic cubes such that $\mu([0,1)^n) = \mu^+([0,1)^n)$, and $\mu(Q) = \sum_{Q' \subset Q} \mu(Q')$ whenever $Q$ is dyadic, and $Q'$ ranges over dyadic cubes with half the sidelength of $Q$. This can be done by working downward `greedily'. And the Caratheodory extension theorem then gives that $\mu$ is the required Frostman lemma.

	Conversely, if an $s$ dimensional measure $\mu$ exists supported on $E$, then $\mu$ is absolutely continuous with respect to $H^s$, and therefore there is a locally integrable $f$ such that
	%
	\[ \mu(E) = \int f(x)\; dH^s(x) \]
\end{proof}

A fundamental concept in the lower bounding of dimensions is the $\alpha$ energy of a Borel measure $\mu$, which is
%
\[ I_\alpha(\mu) = \int \int |x-y|^{-\alpha}\; d\mu(x) d\mu(y) = \int k_s * \mu\; d\mu \]
%
where $k_s(x) = |x|^{-\alpha}$, for $x \in \mathbf{R}^d$. If $0 < \beta < \alpha$, and $\mu$ has compact support. Integrating Frostman's lemma gives that for a Borel $E$,
%
\[ \hausdim(E) = \sup \{ \alpha: \text{there is}\ \mu \in M(A)\ \text{such that}\ I_\alpha(\mu) < \infty \} \]
%
The $\alpha$ dimensional energy then
%
\[ I_\alpha(\mu) \propto_{n,\alpha} \int_{\mathbf{R}^d} |\widehat{\mu}(\xi)|^2 |\xi|^{\alpha-d}\; d\xi \]
%
Thus
%
\[ \hausdim(E) = \sup \left\{ \text{there is}\ \mu \in M(A)\ \text{such that}\ \int |\widehat{\mu}(x)|^2 |x|^{\alpha-d}\; dx < \infty \right\} \]

\section{Projection Theorems}

Recall that the Grassmanian manifold $G(n,m)$ is a space parameterizing the family of $m$ dimensional hyperplanes in $\mathbf{R}^n$. The orthogonal group $O(n)$ acts on $G(n,m)$, and we let $\gamma_{nm}$ denote the resultant Borel probablity measure. We then have Marstrand's projection theorem

\begin{theorem}[Marstrand]
	s
\end{theorem}






\chapter{Fourier Dimension}

The {\bf Fourier dimension} of a Borel set $E$ is
%
\[ \dim_{\mathbf{F}}(E) = \sup \{s : \text{there is}\ \mu \in M(E)\ \text{s.t.}\ |\widehat{\mu}(\xi)| \leq |\xi|^{-s/2}  \} \]
%
This implies the energy integrals of the right dimension to converge, implying $\dim_{\mathbf{F}}(E) \leq \hausdim(E)$. A set is {\bf Salem} if $\hausdim(E) = \dim_{\mathbf{F}}(E)$.

\section{Dimensions of Brownian Motion}

Consider a one dimensional Brownian motion $W$. Then almost surely, for each $0 < \alpha < 1/2$, $W$ is locally $\alpha$ H\"{o}lder continuous. For a fixed Borel set $E$, The bound
%
\[ \hausdim(W(E)) \leq \frac{1}{\alpha} \hausdim(E) \]
%
then holds for almost every path of the motion. Taking $\alpha \uparrow 1/2$, we find the $\hausdim(W(E)) \leq 2 \hausdim(E)$. In this lecture we focus on a converse.

\begin{theorem}[Mckean, 1955]
	Let $A \subset [0,\infty)$ be Borel. Then $\hausdim(W(E)) = 2\hausdim(E) \wedge 1$ almost surely.
\end{theorem}

More generally,

\begin{theorem}[Kaufman's Dimension Doubling Theorem]
	Let $B$ be a Brownian motion in $\mathbf{R}^d$, for $d \geq 2$, then almost surely, for every Borel set $E$,
	%
	\[\hausdim(B(E)) = 2\hausdim(E) \]
\end{theorem}

Note that the almost surely condition is now independent of $E$, so we can apply this theorem to random sets. If $Z = \{ t \geq 0: B_t = 0 \}$ is the random zero set of a path of Brownian motion, and $d \geq 2$, then almost surely we find $\hausdim(E) = 0$. For $d = 1$, the zero may not even be zero dimensional, so we know that Mckean's theorem cannot take out the almost surely over all subsets. We will follow Kahane's 1966 proof of Mckean's result. Consider the following lemma.

\begin{lemma}
	If $\mu$ is an $s$ dimensional measure supported on $[0,\infty)$, then almost surely, for all $|\xi| > 2$,
	%
	\[ |\widehat{\mu_W}(\xi)| \lesssim \frac{(\log |\xi|)^{1/2}}{|\xi|^{-s}} \]
	%
	where $\mu_W$ is the random pushforward measure of $\mu$ by the random path $W$, and the constant in the inequality is random.
\end{lemma}
\begin{proof}
	
\end{proof}

If $E$ was $s$ dimensional, we could find an $s$ dimensional probability measure on $E$. Then by Kahane's lemma, we compute, almost surely, that
%
\[ I_s(\mu_W) \lesssim O(1) + \int_{|\xi| > 2} |\widehat{\mu_W}(\xi)|^2 |\xi|^{s-1} \lesssim O(1) + \int_{-\infty}^\infty \frac{\log |x|}{|x|^{-1-s}} < \infty \]
%
so $\mu_W$ is $s$ dimensional, and so $\hausdim(W(E)) \geq s$. Note that we actually proved something even stronger. The inequality above implies that $\dim_{\mathbf{F}}(W(E)) \geq s$ almost surely, so $W(E)$ is a Salem set almost surely. In particular, Salem sets exist. An alternate proof is to calculate, using Fubini's theorem, we calculate that if $Z \sim N(0,1)$, then provided $r < 2s < 1$,
	%
	\[ \mathbf{E}[I_r(\mu_W)] = \int_{-\infty}^\infty \mathbf{E} \left( \frac{1}{|W_t - W_s|^r} \right)\; dr = \left( \int_{-\infty}^\infty \frac{dt\; ds}{|t-s|^{r/2}} \right) \mathbf{E} \left( \frac{1}{|Z|^r} \right) < \infty \]
	%
	This doesn't require the $K$ lemma at all.

\chapter{Properties of Generic Sets}

In modern analysis, methods often enable us to obtain properties of points lying in some space $X$ which hold on a `generic' set of points $x \in X$. For instance, if $X$ is a measure space, a set $E \subset X$ can be small if it is of measure zero, and then a property on $X$ holds `generically' if it holds except on a set of measure zero. However, here we want to discuss `generic' properties of sets in a metric space, and the family of subsets of a metric space does not have a natural measure theoretic structure. On the other hand, we will soon find that certain families of subsets do have a natural metric space structure, so we need to discuss `generic' properties of points on a topological space.

If $X$ is a topological space, we say $E \subset X$ is \emph{nowhere dense} if $\overline{E}$ has non-empty interior. However, we often rely on a more analytically stable small property. We say $E \subset X$ is of \emph{first category} if it is the countable union of nowhere dense sets. A set $E$ is of \emph{second category} if it is not of first category, and a property on $X$ holds \emph{quasi-always} if is true except on a set of first category. We recall that the \emph{Baire category theorem} says that, in a complete metric space, or a locally compact topological space, the complement of sets of first category are dense, which is one manifestation of the fact that holding `quasi-always' is an acceptable notion of being generic.

\begin{remark}
	The Baire category theorem also says that open subsets of a complete metric space are of second category in that metric space.
\end{remark}

Now let $X$ be a metric space. For $E \subset X$, and $\varepsilon > 0$, we let
%
\[ E_\varepsilon = \{ x \in E: \text{there is $y \in E$ such that $d(x,y) < \varepsilon$} \}. \]
%
Now let $\mathcal{E}$ be the family of all compact subsets of $X$. Given two sets $E, F \in \mathcal{E}$, we can define the \emph{Hausdorff distance}
%
\[ d(E,F) = \inf \{ \varepsilon > 0: E \subset F_\varepsilon\ \text{and}\ F \subset E_\varepsilon \}. \]
%
It is easy to see that since $E$ and $F$ are compact, this quantity is finite. Moreover, since $\overline{E} = \bigcap_{\varepsilon > 0} E_\varepsilon$, and $E$ and $F$ are closed, $d(E,F) = 0$ if and only if $E = F$. If $E \subset F_\varepsilon$, and $F \subset G_\delta$, then $E \subset G_{\varepsilon + \delta}$, which is a manifestation of the triangle inequality for the Hausdorff distance. Thus $\mathcal{E}$ has the natural structure of a metric space. Moreover, if $X$ is complete, $\mathcal{E}$ is complete.

\begin{theorem}
	If $X$ is complete, $\mathcal{E}$ is complete.
\end{theorem}
\begin{proof}
	Note that if $E = \lim_{i \to \infty} E_i$, and if $\{ x_i \}$ is a sequence of points with $x_i \in E_i$ for each $i$, and $\lim x_i = x$ for some $x \in X$, then we actually have $x \in E$. This is because we can find points $\{ y_i \} \in E$ with $d(x_i,y_i) \leq d(E,E_i)$, hence $d(y_i,x) \leq d(y_i,x_i) + d(x_i,x) \to 0$, so $y_i \to x$, and hence $x \in E$ since $E$ is closed. If $\{ E_i \}$ is a Cauchy sequence in $\mathcal{E}$, we let $E$ denote the family of limits of Cauchy sequences $\{ x_i \}$ with $x_i \in E_i$ for each $i$. If $\{ x_{i_j} \}$ is a Cauchy subsequence with $x_{i_j} \in E_{i_j}$ for each $j$, then we can extend it to a Cauchy sequence $\{ x_i \}$ with $x_i \in E_i$ for each $i$, so without loss of generality, we may assume by thinning to a subsequence that $d(E_i,E_{i+1}) < 1/2^{i+1}$.

	In this case, it is easy to see that $E = \bigcap_{i = 1}^\infty (E_i)_{1/2^i}$. It is certainly true that $\bigcap_{i = 1}^\infty (E_i)^{1/2^i} \subset E$. Conversely, if $\{ x_i \}$ is a Cauchy sequence with $x_i \in E_i$ for each $i$, and if $x = \lim_{i \to \infty} x_i$, then for each $i$, there is $\varepsilon_i > 0$, such that $d(E_i,E_{i+j}) + \varepsilon_i < 1/2^i$. If we fix $j > i$ large enough that $d(x_j,x) < \varepsilon_i$, then we can find $y_i \in E_i$ with $d(y_i,x_j) < 1/2^i - \varepsilon_i$, hence $d(y_i,x) < 1/2^i$. Since $i$ was arbitrary, $x \in \bigcap_{i = 1}^\infty (E_i)_{1/2^i}$. The family of sets $\{ (E_i)_{1/2^i} \}$ is nested. Moreover, if $x_i \in E_i$, then a sequence $\{ x_j \}$ extending $x_i$ with $x_j \in E_j$ for each $j$, and $d(x_j,x_{j+1}) < 1/2^{j+1}$, which is therefore Cauchy, and if $x = \lim x_i$ is in $E$, $d(x,x_i) < 1/2^i$. Thus $E_i \subset E_{1/2^i}$, and since $E$ is obviously a subset of $E_{1/2^i}$, $E_{1/2^i} \subset (E_i)_{1/2^i}$. Thus $d(E,E_i) \leq 1/2^i$. Our proof will be finished if we show $E$ is a compact set. Let us now show $E$ is a compact set. Let $A \subset E$ be an infinite subset of $E$. We define a nested family of infinite sets $\{ A_i \}$ inductively. by first defining $A_i = A_0$. Then, given $A_i$, there are finitely many balls $\{ B(x_1,1/2^{i+1}), \dots, B(x_N,1/2^{i+1})$ which cover $E_{i+1}$, and then $\{ B(x_1,1/2^i), \dots, B(x_N,1/2^i) \}$ is a cover of $(E_{i+1})_{1/2^{i+1}}$, and therefore a cover of $A_i$. Thus there is some $j$ such that $B(x_j,1/2^i) \cap A_i$ contains infinitely many points, and we set $A_{i+1} = B(x_j,1/2^i) \cap A_i$. Clearly this family of sets has a limit point, so $A$ has a limit point. This shows $E$ is compact.
\end{proof}

Thus it is natural to discuss a property holding for `quasi-all' compact subsets of a complete metric space. As a first example, we show that quasi-all compact subsets of $X$ have no isolated points.

\begin{lemma}
	Fix $\delta > 0$, and $0 < \varepsilon < \delta/3$. If $E,F \in \mathcal{E}$, $d(E,F) < \varepsilon$, and there is $x \in E$ such that $B(x,\delta) \cap E = \{ x \}$, then there is $y \in F$ such that $d(x,y) < \varepsilon$, and for any such $y$,
	%
	\[ B(y,\delta - 3\varepsilon) \cap F \subset B(x,\varepsilon) \subset B(y,2\varepsilon). \]
\end{lemma}
\begin{proof}
	There must certainly exists $y \in F$ with $d(x,y) < \varepsilon$. If $y' \in F$ satisfies $d(y,y') < \delta - 3\varepsilon$, then because $d(E,F) < \varepsilon$, there must be $x' \in E$ such that $d(x',y') < \varepsilon$. But the triangle inequality implies that $d(x,x') < 2\varepsilon < \delta$, so $x = x'$, and thus $d(y,y') < 2\varepsilon$.
\end{proof}

\begin{theorem}
	Let $X$ be a metric space with no isolated points. Then quasi-all compact subsets of $X$ have no isolated points.
\end{theorem}
\begin{proof}
	For each $\delta > 0$, let $A_\delta = \{ E \in \mathcal{E}: \text{there is $x \in E$ such that}\ B(x,\delta) \cap E = \{ x \} \}$. Our goal is to show $A_\delta$ is closed and nowhere dense. First, we show $A_\delta$ is closed. If $\{ E_i \}$ is a sequence in $A_\delta$ converging to some $E \in \mathcal{E}$. Pick $x_i \in E_i$ for each $i$ such that $B(x_i,\delta) \cap E_i = \{ x_i \}$. Without loss of generality, assume that $d(E_i, E) < \delta / 3$, and we can choose a decreasing sequence $\varepsilon_i < \delta / 3$ such that $d(E_i,E) < \varepsilon_i$. Then for each $i$, we can find $y_i \in E$ such that $d(x_i,y_i) < \varepsilon_i$ and $B(y_i,\delta - 3 \varepsilon_i) \cap F \subset B(y_i,2\varepsilon_i)$. Since $F$ is compact, we can assume $\{ y_i \}$ converges to some point $y$. But if $d(y_i,y) < \varepsilon'$ and $B(y_i,\delta - 3\varepsilon) \cap F \subset B(y_i,2\varepsilon)$, then $B(y,\delta-3\varepsilon - \varepsilon') \subset B(y,2\varepsilon + \varepsilon')$. Taking $i \to \infty$, and thus $\varepsilon, \varepsilon' \to 0$, we conclude that $B(y,\delta) = \{ y \}$. Thus $E \in A_\delta$. It is easy to show $A_\delta$ is nowhere dense. If $E \in A_\delta$, it can only have finitely many elements satsifying the property above by compactness, and we can always add a single point within a $\varepsilon$ neighbourhood of each such point to find $F \not \in A_\delta$ with $d(E,F) < \varepsilon$. Thus $\bigcup_{\delta > 0} A_\delta = \bigcup_{n = 1}^\infty A_{1/n}$ is a set of first category, and it's complement is the set of all compact sets with no isolated points.
\end{proof}

Let us now also discuss an application to harmonic analysis on $\mathbf{T} = \mathbf{R}/\mathbf{Z}$. Recall that distinct values $x_1, \dots, x_n \in \mathbf{T}$ are \emph{independent} if for any $m_1, \dots, m_n \in \ZZ$, $m_1 x_1 + \dots + m_n x_n = 0$ only if $m_1, \dots, m_n = 0$. It is a consequence of the classical theorem of Kronecker and Weyl that if $x_1, \dots, x_n \in \mathbf{T}$ are independant, then for any values $a_1, \dots, a_n \in \CC$ with $|a_i| = 1$ for all $i$, and for any $\varepsilon > 0$, there exists an integer $N$ such that $|\exp(2 \pi Nix_i) - a_i| < \varepsilon$ for all $i \in \{ 1, \dots, n \}$. We wish to try and understand this result for infinite families of sets. So we define $E \subset \mathbf{T}$ to be \emph{independent} if all finite subsets of $E$ are independent, and \emph{Kronecker} if for any $f \in C(\mathbf{T})$ with $|f(x)| = 1$ for all $x \in \mathbf{T}$, and for any $\varepsilon > 0$, there exists $N$ such that $|\exp(2 \pi Nix) - f(x)| < \varepsilon$ for all $x \in E$.

\begin{lemma}
	If $E$ is a Kronecker set, then $E$ is independant.
\end{lemma}
\begin{proof}
	If $m_1 x_1 + \dots + m_n x_n = 0$, for distinct $x_i$, then for any $N$,
	%
	\[ \exp(2\pi m_1 N x_1) \dots \exp(2\pi m_n N x_n) = 1. \]
	%
	If $|m_1| + \dots + |m_n| = M$, and if we choose $f \in C(\mathbf{T})$ such that $f(x_k) = \exp(i \pi / m_k n)$ for each $k$, and if $N$ is chosen such that $|f(x_k) - \exp(2 \pi N i x_i)| < 1/M$, then
	%
	\begin{align*}
		2 &= |1 - (-1)|\\
		&= \left| \prod_{k = 1}^n \exp(2 \pi N i x_k)^{m_k} - \prod_{k = 1}^n \exp(i \pi / m_k n)^{m_k} \right|\\
		&\leq \sum_{k = 1}^n |m_k| |\exp(2 \pi N i x_k) - \exp(i \pi / m_k n)| \leq 1,
	\end{align*}
	%
	which gives a contradiction.
\end{proof}

The converse, unfortunately, is not true. There are independant sets which are not Kronecker. For our purposes, we will now show that quasi-all compact sets are Kronecker. This is part of a general principle that says that a generic compact set is as `thin as possible', and Kronecker sets are very thin from the perspective of harmonic analysis. To prove the statement, we note that the family of all functions $f \in C(\mathbf{T})$ with $|f(x)| = 1$ for all $x \in \mathbf{T}$ is separable, and if $\{ f_i \}$ is a countable dense set, then $E$ is Kronecker if and only if each of the functions $f_i$ is arbitrarily approximable by a character.

\begin{theorem}
	Quasi-all compact subsets of $\mathbf{T}$ are Kronecker.
\end{theorem}
\begin{proof}
	For each function $f \in C(\mathbf{T})$ with $|f(x)| = 1$ for all $x \in \mathbf{T}$, let $A(f,\varepsilon)$ be the family of all sets $E$ such that there exists an integer $N$ such that $|f(x) - \exp(2 \pi i N x)| < \varepsilon$ for each $x \in E$. We claim $A(f,\varepsilon)$ is a dense, open subset of $\mathcal{E}$. If $E$ is fixed, since $E$ is compact and $f$ is continuous, there is $\delta > 0$ such that $|f(x) - \exp(2 \pi i N x)| < \varepsilon$ for each $x \in E_\delta$. But this implies that $F \in A(f,\varepsilon)$ for each $F$ with $d(E,F) < \delta$. Thus $A(f,\varepsilon)$ is open. If $\{ f_i \}$ is a dense family of continuous functions in the $L^\infty$ norm, then $\bigcap_{i = 1}^\infty \bigcap_{n = 1}^\infty A(f_i,1/n)$ is the family of all Kronecker sets. Hence to complete our proof it suffices to show $A(f,\varepsilon)$ is open and dense for any $f \in C(\mathbf{T})$ and $\varepsilon > 0$.

	To show density, let $E$ be an arbitrary set. Since $f$ is continuous, it is actually uniformly continuous, so we can find $\delta$ such that if $|x - y| < \delta$, then $|f(x) - f(y)| < \sqrt{2}$. This forces the values of $f$ to lie on a particular half of the circle over any $\delta$ interval. Thus if $N$ is a large integer such that $2/N < \delta$, then $\{ x \in \mathbf{T}: \exp(2 \pi i N x) = f(x) \}$ contains a point in any length $2/N$ interval of $\mathbf{T}$. In particular, this means that if
	%
	\[ F = \{ x \in E_{2/N} : \exp(2 \pi i N x) = f(x) \}, \]
	%
	then $d(F,E) < 2/N$ and $F \in A(f,\varepsilon)$ for all $\varepsilon > 0$. Since $N$ is arbitrary, $A(f,\varepsilon)$ is dense.
\end{proof}

This theorem also even holds under more algebraically restricted circumstances. We note that
%
\[ \{ (E_1,E_2) \in \mathcal{E}^2: E_1 + E_2 = \mathbf{T} \} \]
%
is a closed subset of $\mathcal{E}^2$, and is therefore complete. We shall show that quasi-all elements of this set are pairs of Kronecker sets.

\begin{theorem}
	Quasi-all $(E_1,E_2) \in \mathcal{E}^2$ with $E_1 + E_2 = \mathbf{T}$ are Kronecker sets.
\end{theorem}
\begin{proof}
	For each $f$, $\varepsilon > 0$, and $i \in \{ 0, 1 \}$, let $A(f,\varepsilon,i)$ be the family of all pairs of sets $(E_1, E_2)$ with $E_1 + E_2 = \mathbf{T}$ for which there exists an integer $N$ such that for each  $x \in E_i$, $|f(x) - \exp(2 \pi i N x)| < \varepsilon$. Then
	%
	\[ \bigcap_{i \in \{ 0, 1 \}} \bigcap_{j = 1}^\infty \bigcap_{k = 1}^\infty A(f_j,1/k,i) \]
	%
	is the family of all pairs $(E_1,E_2)$ where $E_1 + E_2 = \mathbf{T}$ and $E_1$ and $E_2$ are Kronecker. Thus it suffices to show $A(f,\varepsilon,i)$ is open and dense for each $f$, $i$, and $\varepsilon > 0$. It is easy to see that $A(f,\varepsilon,i)$ is open, as in the last proof. Without loss of generality, we can assume $i = 1$. So fix $\delta > 0$, and consider the sets $\widetilde{E_i} = (E_i)_{\delta/2}$. Then if $d(E_1,F) < \delta/4$, $F + \tilde{E_2} = \mathbf{T}$. Thus choosing $F$ as in the last theorem gives a pair $(F,\tilde{E_2})$ within a $\delta$ neighbourhood of $(E_1,E_2)$. Thus $A(f,\varepsilon,i)$ is open and dense, which proves the theorem.
\end{proof}

We can also use Baire category techniques to study the generic behaviour of Besicovitch sets, subsets of the plane containing a unit line segments in each direction. For each $a,b \in [-2,2]$, we let $l_{ab}$ be the line segment connecting $(a,0)$ to $(b,1)$. We let $\mathcal{F}$ consist of the family of all compact subsets $E$ of $[-2,2] \times [0,1]$ which are a union of line segments of the form $l_{ab}$, and such that for each $\nu \in [-1,1]$, there exists $a \in [-2,2]$ such that $l_{a(a+\nu)} \subset \mathcal{F}$. Such a set contains a unit line segment from any angle in a 45 degree arc, and so two copies of such a set form a Besicovitch set.

\begin{lemma}
	$\mathcal{F}$ is a closed subset of the family $\mathcal{E}$ of all compact subsets of $[-2,2] \times [0,1]$.
\end{lemma}
\begin{proof}
	Suppose $E = \lim_{i \to \infty} E_i$, with $E_i \in \mathcal{F}$ for all $i$. We begin by showing $E$ is a union of line segments. Suppose that $x \in E$. Then there are points $x_i \in E_i$ for each $i$ such that $x = \lim x_i$. Moreover, we can find pairs $a_i$ and $b_i$ such that $x_i \in l_{a_ib_i}$. Since $[-2,2]$ is compact, we can assume without loss of generality that $a_i$ and $b_i$ are convergent sequences to values $a$ and $b$. But this means that $l_{a_ib_i}$ converges in the Hausdorff metric to $l_{ab}$, and thus $x \in l_{ab}$. Moreover, $l_{ab} \subset E$. Thus $E$ is a union of lines. Similar techniques can be used to show that $E$ contains a line segment corresponding to each $\nu \in [-1,1]$.
 \end{proof}

\begin{theorem}
	$\mathcal{F}$ is a closed subset of the family $\mathcal{E}$ of all compact subsets of $[-2,2] \times [0,1]$, and quasi-all elements of $\mathcal{F}$ have Lebesgue measure zero.
\end{theorem}
\begin{proof}
	We show the stronger statement that quasi-all elements $E$ of $\mathcal{F}$, for all $y \in [0,1]$, $E \cap \RR \times \{ y \}$ has length zero (with respect to the Hausdorff measure), from which the general statement follows by Fubini's theorem. For each interval $I \subset [0,1]$ with length $\varepsilon$, we set $A(I)$ to be the set of all compact sets $E \in \mathcal{F}$ such that, for each $y \in I$, $E \cap [-2,2] \times \{ y \}$ has length less than $100 \varepsilon$. The set
	%
	\[ \bigcap_{n = 1}^\infty \bigcap_{k = 0}^{n-1} A([k/n,(k+1)/n]) \]
	%
	is the class of sets whose slices have length zero, so it suffices to show $A(I)$ is open and dense for each interval $I$. It is obvious that the family of such sets is open, since if $E \in A(I)$, there exists a sufficiently small $\delta$ such that $E_\delta \in A(I)$, and so if $d(E,F) < \delta$, $F \in A(I)$. Thus we must concentrate on showing $A(I)$ is dense in $\mathcal{F}$. So consider a set $E \in \mathcal{F}$, and fix a large integer $N > 4$, defining $\delta = 1/N$. If $E = \bigcup_\alpha l_{a_\alpha b_\alpha}$, and we define $x_\alpha = a_\alpha + 2\delta$ and $y_\alpha = b_\alpha + 2\delta$ if $a_\alpha \leq -1/2$, and $x_\alpha = a_\alpha - 2\delta$ and $y_\alpha = b_\alpha - 2\delta$ if $a_\alpha > -1/2$. Then $E' = \bigcup_\alpha l_{x_\alpha y_\alpha}$ is an element of $\mathcal{F}$, a subset of $[-2+\delta,2-\delta] \times \{ y \}$, and $d(E,E') \leq 2\delta$. Thus to prove density, it suffices to show that given $E \in \mathcal{F}$ which is also a subset of $[-2+2\delta,2-2\delta] \times \{ y \}$, we can find $F \in A(I)$ with $d(E,F) \lesssim \delta$. We begin by finding $\{ a_1, \dots, a_N \}$ such that $l_{a_k(a_k + k/N)}$ for each $k \in \{ -N, \dots, N \}$. Fix a point $u_k \in l_{a_k(a_k + k/N)} \cap ([-2,2] \times I)$, and consider the family of all lines through $u_k$ which lie in a $\delta$ thickening of $l_{a_k(a_k + k/N)}$. This set has length at most $\delta$ on each slice of $[-2,2] \times I$, and contains a line $l_{x(x + \nu)}$ for each $\nu \in [(k-1)/N,(k+1)/N]$. The union $G$ of such lines lies in $\mathcal{F}$, and is a subset of the $\delta$ thickening of $E$. If we consider a finite union of lines $H$ such that $d(H,E) < \delta$, then $G \cup H \in A(I)$ and $d(H,E) < \delta$.
\end{proof}




\begin{remark}
	If $X$ is a measure space, let $\mathcal{E}$ denote the family of all finite measure subsets of $X$. We can define the distance between two sets $E,F \in \mathcal{E}$ by setting
	%
	\[ d(E,F) = |E \bigtriangleup F| = |E - F| + |F - E|. \]
	%
	Then $d(E,F) = 0$ if and only if $E$ and $F$ differ by a set of measure zero. The map $E \mapsto \mathbf{I}_E$ embeds $\mathcal{E}$ in $L^1(X)$. If $\mathcal{I}_{E_i}$ is a sequence of sets and $\| \mathbf{I}_{E_i} - f \|_{L^1(X)} \to 0$, then a simple approximation argument can show $\{ x \in X: f(x) \not \in \{ 0, 1 \} \}$ is a set of measure zero, so there exists a finite measure set $F$ such that $\mathbf{I}_F$ equals $f$ almost everywhere. Thus $\mathcal{E}$ is a complete metric space. However, most properties of sets which can be tested generically in $\mathcal{E}$ also tend to extend to properties in $L^1(X)$, which makes this space less useful than the family of compact sets under the Hausdorff distance.
\end{remark}

\end{document}
