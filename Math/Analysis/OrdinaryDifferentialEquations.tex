\documentclass[12pt]{report}

\usepackage{amsmath}
\usepackage{amssymb}
\usepackage{amsthm}
\usepackage{amsopn}
\usepackage{kpfonts}
\usepackage{graphicx}
\usepackage{kbordermatrix}
\usepackage{tikz}
\usetikzlibrary{arrows, petri, topaths}%
\usepackage{tkz-berge}
\usepackage{multicol}

\usepackage{framed}
\usepackage{mathtools}
\usepackage{float}
\usepackage{subfig}
% \usepackage{cmbright}

\theoremstyle{plain}
\newtheorem{theorem}{Theorem}[chapter]
\newtheorem{lemma}[theorem]{Lemma}
\newtheorem{corollary}[theorem]{Corollary}
\newtheorem{prop}[theorem]{Proposition}
\newtheorem{exercise}{Exercise}[chapter]

\newtheorem*{example}{Example}
\newtheorem*{proof*}{Proof}

\theoremstyle{definition}
\newtheorem*{defi}{Definition}
\newenvironment{definition}
    {\begin{samepage}\begin{framed}\begin{defi}}
    {\end{defi}\end{framed}\end{samepage}}





\usepackage{hyperref} 
\hypersetup{
    colorlinks = true,
    linkcolor = black,
}

\makeatletter
\renewcommand*\env@matrix[1][*\c@MaxMatrixCols c]{%
  \hskip -\arraycolsep
  \let\@ifnextchar\new@ifnextchar
  \array{#1}}
\makeatother

\renewcommand*\contentsname{\hfill Table Of Contents \hfill}

\newcommand{\optionalsection}[1]{\section[* #1]{(Important) #1}}
\newcommand{\deriv}[3]{\left. \frac{\partial #1}{\partial #2} \right|_{#3}}

\title{Ordinary Differential Equations}
\author{Jacob Denson}

\begin{document}

\pagenumbering{gobble}

\maketitle

\tableofcontents

\pagenumbering{arabic}

\chapter{Introduction}

Here we consider \emph{ordinary differential equations}, i.e. we try and find solutions to equations involving an arbitrary one-variable function $f(x)$, it's derivatives, and functions of $x$, i.e. given some function $F$, we try and find a function $f$ such that
%
\[ F(x,f(x), \dots, f^{(n)}(x)) = 0. \]
%
It is important that $n$ here is a finite quantity. We say the equation has \emph{order} $n+1$. Furthermore, it is also important that the function $F$ cannot access the properties of the function $f$ globally, only pointwise.

\begin{example}
    An example of an ordinary differential equation is
    %
    \[ \frac{dy}{dx} + y = 0. \]
    %
    Here the function involved is $F(x,y,z) = y + z$, and the equation has order one.
\end{example}

\begin{example}
    The equation $(f \circ f)'(x) + f(x) = 0$ is \emph{not} a differential equation, since $(f \circ f)'(x)$ cannot be expressed as a function of $f(x)$ and $f'(x)$.
\end{example}

An interesting amount of geometry is going on here. For simplicity, consider the $n$'th order differential equation
%
\[ f^{(n)}(x) = G(x,f(x), \dots, f^{(n-1)}(x)) \]
%
Then a solution $f: I \to \mathbf{R}$ to an $n+1$ order differential equation as a \emph{curve} in the \emph{phase space} $\mathbf{R}^n$, i.e. a map $y: I \to \mathbf{R}^n$ satisfying the system of first order differential equations
%
\begin{align*}
    y_0' &= 1\\
    y_1' &= y_2\\
        & \vdots\\
    y_{n-1}' &= y_n\\
    y_n' &= F(y_0, \dots, y_{n-1})\\
\end{align*}
%
The connection is obvious by setting $y_0(x) = x$, $y_1(x) = f(x)$, $y_2(x) = f'(x)$, and so on. Thus, if we define a vector field
%
\[ X_{y} = (1,y_2,\dots,y_n,F(y_0, \dots, y_{n-1})), \]
%
then a solution to a differential equation can be viewed as a curve in phase space whose velocity vector agrees with $X$ at all points, i.e. $y' = X_y$. Such a curve is known as a integral curve.

\begin{example}
    Consider the vector field in $\mathbf{R}^2$ given by $X_{(x,y)} = (y,-x)$. Then the curve $c(t) = (\cos(t),\sin(t))$ is an integral curve for this vector field, which corresponds to the differential equation $f'' = -f$.
\end{example}

Another way of geometrically simplifying the situation is, rather than a vector field, to consider a collection of lines, i.e. a map $\Delta$ such that for each $x$, $\Delta_x \in \mathbf{P}(T\mathbf{R}^n_x)$ is a line in tangent space passing through $x$. The goal here is to find a one dimensional manifold $M$ such that $TM_x = \Delta_x$ for each $x \in M$. The manifold $M$ here is known as an \emph{integral curve}, and the function $\Delta$ is known as a (one dimensional) \emph{distribution}.

\begin{example}
    Consider the distribution $\Delta$ in $\mathbf{R}^2$ given by the two equations
    %
    \[ (xy^2 + 2xy - 1) dy + y^2 dx. \]
    %
    That is, at each point $p = (x,y)$, $\Delta_p$ is the line
    %
    \[ \{ (x + x_0,y + y_0): (xy^2 + 2xy - 1) y_0 + y^2 x_0 = 0 \}. \]
    %
    and we want to find a curve whose tangent at each point passes through this line. One such example is the curve $M$ given by the equation
    %
    \[ xy^2 - e^{-y} - 1 = 0. \]
    %
    By differentiating, we find that $TM_p$ is given by the set of solutions to the equation
    %
    \[ y^2 dx + (2xy + e^{-y}) dy + y^2 dx = 0. \]
    %
    This equation is not equal to the other equation at all points, but it is equal when $xy^2 - e^{-y} - 1$, so $M$ is an integral curve. In particular, a distribution in $\mathbf{R}^n$ can be given by $n-1$ linearly independant equations in $dx_1, \dots, dx_n$.
\end{example}

\begin{example}
    Let $X: \mathbf{R}^n \to \mathbf{R}^n$ be a vector field. Then we can define a distribution by letting $\Delta_x$ be the line lying along the vector $X_x$. If $M$ is an integral curve locally parameterized by some function $y: I \to \mathbf{R}^n$, then $y'(t)$ is a constant multiple of $X_{y(t)}$ for each time $t$. Thus there is a function $A: I \to \mathbf{R}$, never vanishing, such that $y'(t) = A(t) X_{y(t)}$. If we can write $t = f(s)$, such that $f'(s) = A(s(t))$, then $y \circ f$ is an integral curve to $X$. Thus integral curves to distributions, once appropriately parameterized are equivalent to vector fields.
\end{example}

To summarize, we have three problems:
%
\begin{itemize}
    \item Find solutions to differential equations.
    \item Find curves lying tangent to vector fields.
    \item Find integral curves to distributions
\end{itemize}
%
All of these problems are (roughly) equivalent, and the goal of differential equations is to find techniques for studying them.

In general, if a vector field $X: \mathbf{R}^n \to \mathbf{R}^n$ is \emph{locally Lipschitz}, then there is a unique integral curve passing through any point. And moreover, we can find a continuous function $\alpha: \mathbf{R}^n \times \mathbf{R}$ such that for each $x_0$, the function $y(t) = \alpha(x_0,t)$ is a curve with $y'(t) = X_{y(t)}$. Such a function is known as a \emph{flow}, and gives a complete description of the integral curves of the vector field. More generally, if $X$ is $C^k$, then $\alpha$ will also be $C^k$. Theoretically, $\alpha$ exists, but one still requires techniques to calculate $\alpha$ as an explicit function.   


\chapter{Basic Techniques}

\section{Reduction to Integration}

Calculus already teaches us some basic techniques for solving differential equations. The most basic differential equation $y' = f(x)$ can be solved as
%
\[ y(x) = A + \int_0^x f(t)\; dt,  \]
%
where $A$ is an arbitrary constant. If we know that $y(x_0) = y_0$, for some values $x_0$ and $y_0$, then we can also write
%
\[ y(x) = y_0 + \int_{x_0}^x f(t)\; dt. \]
%
This is often how a differential equation is solved; given certain \emph{initial conditions}, which is this case are $(x_0,y_0)$, i.e values of $y$ at a particular timepoint, one can uniquely solve the differential equation.

\section{Separable Equations}

Slightly more generally, integration enables us to solve a \emph{separable equation}
%
\[ f(x) dx + g(y) dy = 0. \]
%
We note that if $F'(x) = f(x)$ and $G'(y) = g(y)$, then $d(F + G) = f(x) dx + g(y) dy$, and so the family of curves defined by the equation $F(x) + G(y) = C$ give a family of integral curves to the distribution.

\begin{example}
Next, we consider a homogenous equation
%
\[ P(x,y)\; dx + Q(x,y)\; dy = 0 \]
%
where $P$ and $Q$ are each homogenous functions of order $n$, i.e. for each $x$ and $y$,
%
\[ P(tx,ty) = t^n P(x,y)\quad \text{and}\quad Q(tx,ty) = t^n Q(x,y). \]
%
To exploit this homogeneity, we switch variables by setting $y = ux$, provided we are working where $x \neq 0$. Then $dy = x du + u dx$, and
%
\begin{align*}
    P(x,y)\; dx + Q(x,y)\; dy &= x^n P(1,u)\; dx + x^n Q(1,u)(u\; dx + x\; du)\\
    &= x^n[(P(1,u) + u Q(1,u))\; dx + x Q(u,1)\; du].
\end{align*}
%
Thus the original equation is equivalent, when $x \neq 0$, to the equation
%
\[ \frac{dx}{x} = \frac{Q(1,u)}{P(1,u) + u Q(1,u)}\; du, \]
%
and this equation is separable.
\end{example}

\begin{example}
    Consider the differential equation
    %
    \[ (a_1x + b_1y + c_1)\; dx + (a_2 x + b_2 y + c_2)\; dy = 0. \]
    %
    where the coefficients $(a_1, b_1)$ and $(a_2, b_2)$ are not constant multiples of one another. Then the lines corresponding to these coefficients have a unique intersection $(x_0,y_0)$, i.e. such that $a_1x_0 + b_1y_0 + c_1 = a_2x_0 + b_2y_0 + c_2 = 0$. If we set $x = x_0 + h$, $y = y_0 + k$, then the equation becomes
    %
    \[ (a_1h + b_1k)\; dh + (a_2 h + b_2 k)\; dk = 0 \]
    %
    This is a homogenous differential equation, and therefore reduces to the last example. Alternatively, we can let $u = a_1x + b_1y + c_1$, and $v = a_2x + b_2y + c_2$, so that $du = a_1dx + b_1dy$ and $dv = a_2dx + b_2dy$. These two equations can then be solved for $dx$ and $dy$, and substitution gives another homogenous differential equation.
\end{example}

\begin{example}
    If we consider the `parallel case' of the previous problem, i.e.
    %
    \[ (a_1 x + b_1y + c_1)\; dx + A(a_1x + b_1y + c_2)\; dy \]
    %
    Then substituting $u = a_1x + b_1y$ gives a separable equation in $x$ and $u$, or in $y$ and $u$.
\end{example}

\section{Exact Differential Equation}

More general than the separable case is the case where we have
%
\[ \omega = f(x,y) dx + g(x,y) dy = 0 \]
%
where $d\omega = 0$. This occurs if
%
\[ \frac{\partial g}{\partial x} - \frac{\partial f}{\partial y} = 0. \]
%
In this case, there exists a function $F(x,y)$ such that
%
\[ \frac{\partial F}{\partial x} = f\quad\text{and}\quad \frac{\partial F}{\partial y} = g. \]
%
It then follows that $F(x,y) = 0$ gives a family of integral curves for the differential equation.

\section{Integrating Factors}

There is a trick which can be used to reduce non-exact differential equation to exact differential equations. Given a differential form $\omega$, there may exist a function $f$ such that $f \omega$ is exact, and therefore integral curves can be found.

\begin{example}
    The equation $(y^2 + y)dx - x dy = 0$ is not exact. Nonetheless, if we multiply by $1/y^2$, we obtain the equation $(1 + 1/y)dx - (x/y^2)dy = 0$, which is exact. In fact, the equation is $d(x/y + x) = 0$, so the integral curves are given by $x/y + x = C$.
\end{example}

Theoretically, if one can find



\chapter{Vector Fields on $\mathbf{RP}^n$}

On $\mathbf{R}^n$, local flows related to a smooth vector field $v$ may fail to extend to global flows because solutions approach $\infty$ in finite time. However, often we may be able to embed $\mathbf{R}^n$ in a compact manifold $K$, and extend $v$ to a smooth vector field on $K$. Since $K$ is compact, all local flows extend to global flows, and thus we can consider a global flow on $\mathbf{R}^n$ which `passes through $\infty$' in finite time.

For instance, recall that the space $\mathbf{RP}^n$ is the compact quotient space of $\mathbf{R}^{n+1} - \{ 0 \}$ by the group action of $\mathbf{R}^\times$ by scaling, so that $x$ is identified with $\lambda x$ for any $\lambda \neq 0$. The quotient structure gives it a natural topological structure, which can also be identified with the topology which makes the projection maps on each of the coordinate systems
%
\[ x_i: [x] \mapsto (x^1/x^i, \dots, \widehat{x^i/x^i}, \dots, x^{n+1}/x^i) \]
%
defined on $U_i = \{ [x] : x_i \neq 0 \}$, homeomorphisms. It is a smooth manifold if we consider the $x_i$ as diffeomorphisms.

\begin{example}
    The classic example of a vector field which cannot be extended to a global flow is $v(x) = x^2$ on $\mathbf{R}$, which has a flow
    %
    \[ \varphi_t(x) = \frac{x}{1 - xt} \]
    %
    Which has a singularity at $t = x^{-1}$. Note, however, that if we write this map in projective coordinates, then we find $\varphi_t[x:y] = [x:y - tx]$. In this formulation, it is easy to see that each map $\varphi_t$ can be extended uniquely to a smooth map from $\mathbf{RP}^1$ to $\mathbf{RP}^1$, and the group equation still holds.
    %
    \begin{align*}
        \varphi_{t+s}[x:y] &= [x:y - x(t+s)] = [x:(y - sx) - tx] = \varphi_t(\varphi_s[x:y])
    \end{align*}
    %
    An alternate way to see this is to let $y = 1/x$ denote the inverse coordinate system on projective space. We then calculate that for $y \neq 0, \infty$, that
    %
    \[ v(y) = x^2 \partial_x(y) = - x^2 y^2 = - \partial_y \]
    %
    and $v$ can be uniquely extended to a smooth vector field on $\mathbf{RP}^1$ by defining $v(\infty) = \partial_y$, and therefore generates a global flow on $\mathbf{RP}^1$ because $\mathbf{RP}^1$ is compact. This technique is not general, however. If we consider the vector field $v(x) = x^3 \partial_x$, then we find that $v(y) = -y^{-1} \partial_y$, which cannot be extended to a smooth vector field at $y = 0$. This is because solutions approach infinity `too fast' -- we find the flows take the form
    %
    \[ \varphi_t(y) = \sqrt{y^2 - 2t} \]
    %
    And these solutions approach $y = 0$ with infinite slope.
\end{example}

Sometimes the geometry of projective space provides an enlightening viewpoint on a particular differential equation.

\begin{example}
    Consider the differential equation $\ddot{u} + \alpha u = 0$, as $\alpha$ ranges over $\mathbf{R}$. This corresponds to the two dimensional first order system specified by the vector field $v(u,w) = (w, -\alpha u)$. This means that on the integral curves defined by this vector field,
    %
    \[ - \alpha u du = w dw \]
    %
    so the integral curves lie on the level curves to $w^2 + \alpha u^2$. For $\alpha > 0$, this value is always positive, and defines an ellipse. Since $v$ does not vanish on any ellipse of a positive radius, we see these ellipses must describe the integral curves. For $\alpha < 0$, the level curves of $w^2 + \alpha u^2$ describe hyperbolas not passing through the origin, so these hyperbolas are the integral curves. For $\alpha = 0$, the integral curves are easily seen to be the lines parallel to the $x$ axis. Switching to the coordinates $x = u/w$, $y = 1/w$, we find that for $y \neq 0$,
    %
    \begin{align*}
        v(x,y) &= (w \partial_u(x) - \alpha u \partial_w(x), w \partial_u(y) - \alpha u \partial_w(y))\\
        &= (1 + \alpha u^2/w^2, \alpha u/w^2) = (1 + \alpha x^2, \alpha xy)
    \end{align*}
    %
    This function can be extended to a smooth vector field on the whole of $\mathbf{RP}^2$ by defining $v(x,0) = (1 + \alpha x^2,0)$.
\end{example}

\chapter{Linear Differential Equations}

A linear differential equation is one of the form $Lf = 0$, where
%
\[ (Lf)(x) = a_n(x) (D^n f)(x) + \dots + a_0(x) f(x) = 0, \]
%
where the $a_0, \dots, a_n$ are functions, and $D$ is the differentiation operator, where we assume $a_n(x) \neq 0$ for all $x$. We note that if $Lf = 0$ and $Lg = 0$, then $L(af + bg) = 0$, for any constants $a$ and $b$. Thus the space of solutions to the differential equation spans a subspace of the space of all functions. For any $x_0$, the values of $D^{n-1}f(x_0), \dots, f(x_0)$ uniquely determine the function $f$ satisfying the differential equation. In particular, this means that the map $f \mapsto (f(x_0), \dots, D^{n-1}f(x_0))$ is an isomorphism between the solution set of functions and $\mathbf{R}^n$, so the solution set is $n$ dimensional.

There are some general tricks to determine if a given set of functions $\{ f_1, \dots, f_n \}$ satisfying an $n$th order linear differential equation are independant. One such trick is to consider the Wronskian determinant
%
\[ W(f_1, \dots, f_n)(x) = \det \begin{pmatrix} f_1(x) & \dots & f_n(x) \\ f_1'(x) & \dots & f_n'(x) \\ \vdots & \ddots & \vdots \\ f_1^{(n-1)}(x) & \dots & f_n^{(n-1)}(x) \end{pmatrix} \]
%
If there exists a point $x_0$ such that $W(f_1, \dots, f_n)(x_0) \neq 0$, then the vectors in the matrix span the set of all initial parameters, and therefore $\{ f_1, \dots, f_n \}$ is a basis. In particular, this also means $W(f_1, \dots, f_n)(x_0) \neq 0$ everywhere. Conversely, if the functions do not form a basis, then $W(f_1, \dots, f_n) = 0$.

Nonhomogenous equations can be viewed by very similar methods. Consider a differential equation of the form $Lf = g$, for some function $g$. If we can find a \emph{single} function $f_0$ such that $Lf_0 = g$, then any other solution is given by $f_0 + b$, where $Lb = 0$. Thus we need only find a single non-homogenous solution, and then solve the homogenous form of the equation.

\section{Homogenous Linear Differential Equations with Constant Coefficients}

A homogenous differential equation is one of the form
%
\[ a_n D^nf + \dots + a_0 f = 0, \]
%
where $D$ is the differential operator, and $a_0, \dots, a_n$ are constants. These are perhaps the simplest general class of differential equations to solve. The first idea is to see a correspondence between these operators and polynomials. We note that if $L$ and $S$ are two linear differential operators with constant coefficients, then these operators actually commute, i.e. $L \circ S = S \circ L$. Since these operators also form an algebra under composition, and have as a basis $\{ 1, D, D^2, \dots \}$, the space of linear differential operators with constant coefficients is actually isomorphic to $\mathbf{C}[D]$.

The advantage of this result is that we can apply the fundamental theorem of algebra. Thus we can write any homogenous differential equation as
%
\[ L = (D - \alpha_1)^{m_1} \dots (D - \alpha_n)^{m_n} \]
%
where $\alpha_1, \dots, \alpha_n \in \mathbf{C}$. Now we have to do some calculation. For each $k$, the solution space to $(D - \alpha_k)^{m_k} f = 0$ has dimension $m_k$, and is given by $\{ e^{\alpha_k x}, \dots, x^{m_k - 1} e^{\alpha_k x} \}$. Furthermore, the solution spaces to each $(D - \alpha_k)^{m_k}$ are disjoint to one another. Thus the solution space to $L$ breaks down into the direct sum of the solution spaces $(D - \alpha_k)^{m_k}$. And so we can solve each of the individual differential equations, and then work our way back up.

\begin{remark}
    If we are working over the real numbers, we cannot completely break polynomials into linear factors, instead having to deal with factors of the form $((D - \alpha)^2 + \beta^2)^m$. This solution space is spanned by the real vector space
    %
    \[ \{ e^{\alpha x} \cos(\beta x), e^{\alpha x} \sin(\beta x), \dots, x^{m-1} e^{\alpha x} \cos(\beta x), x^{m-1} e^{\alpha x} \sin(\beta x) \}. \]
    %
    and so we obtain sines and cosines in our decomposition as well as exponentials. Of course, the two representations of the solution space are connected by Euler's formula $e^{\beta ix} = \cos(\beta x) + \sin(\beta x)$.
\end{remark}


\section{The Method of Annihilation}

It is difficult to find explicit solutions to non-homogenous differential equations with constant coefficients. However, when the `constant' term of the differential equation is also given as a solution to a differential equation, the method of annihilation enables one to obtain an explicit solution.

Let $L_0$ be a linear differential operator with constant coefficients, and suppose we wish to solve the equation $L_0y = f$, for some function $f$, and suppose in addition that there exists a differential operator $L_1$ with constant coefficients such that $L_1f = 0$. Then $L_1(L_0 y) = L_1(f) = 0$, so $y$ is a solution to the homogenous equation $(L_1 \circ L_0)(y) = 0$, with constant coefficients. One can obtain the complete span of solutions to this equation, and then calculate coefficients to obtain a particular solution to $L_0y = 0$.

\begin{example}
    Consider the differential equation $L_0y = e^{5x}$, where $L_0 = D^2 + 2D + 1 = (D + 1)^2$. We note that if $f(x) = e^{5x}$, then $L_1 f = 0$ where $L_1 = D - 5$. Thus $L_0L_1y = 0$, and $L_0L_1 = (D - 5)(D+1)^2$. Thus there must be constants $A$, $B$, and $C$ such that
    %
    \[ y = A e^{5x} + B e^{-x} + C x e^{-x}. \]
    %
    Without loss of generality, we can set $B = C = 0$, since $L_0(e^{-x}) = L_0(xe^{-x}) = 0$. Since $L_0(e^{5x}) = 36 e^{5x}$, we can set $A = 1/36$. Thus a general solution to the differential equation is given by $y = e^{5x}/36 + Be^{-x} + Cxe^{-x}$.
\end{example}

\begin{thebibliography}{9}

%\bibitem{evans}
%Lawrence C. Evans
%\textit{Partial Differential Equations}

\end{thebibliography}

\end{document}