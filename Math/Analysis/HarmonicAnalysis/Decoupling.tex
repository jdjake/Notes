%% The following is a directive for TeXShop to indicate the main file
%%!TEX root = HarmonicAnalysis.tex

\part{Decoupling}

Decoupling Theory is an in depth study of how `interference patterns' can show up when combined waves with frequency supports in disjoint regions of space. The geometry of these regions effects how much constructive interference can happen. Of course decoupling theory is essential to studying many dispersive partial differential equations, but also has surprising applications in number theory as well, as well as other areas of harmonic analysis, such as restriction theory. The theory of decoupling was initiated by Wolff in the early 2000s, and Laba-Wolff, Laba-Pramanik, Pramanik-Seeger, and Garrigos and Seeger in the 2000s. But the theory was brought to the forefront of harmonic analysis by Bourgain, Demeter, and Guth in the mid 2010s.








\chapter{The General Framework}

In any norm space $X$, given $f = f_1 + \dots + f_N$, let's say, with $\| f_1 \|_X, \dots, \| f_N \|_X \leq 1$, then the triangle inequality implies that
%
\[ \| x \|_X \leq N. \]
%
In general, such a result is sharp, i.e. if all the $f_1,\dots, f_N \in X$ are equal to one another, in which case we see \emph{constructive interference}. On the other hand, one can often significantly improve this bound if one knows that the functions $f_1,\dots,f_N \in X$ do not interact with one another to a reasonable extent, i.e. if we assume some \emph{orthogonality}.

Classically, this situation comes about when $X$ is a Hilbert space. We can then write
%
\[ \| f \|_X^2 = \sum_i \| f_i \|_X^2 + 2 \sum_{i < j} \langle f_i, f_j \rangle. \]
%
If $f_1,\dots,f_N$ are pairwise orthogonal, i.e. $\langle f_i, f_j \rangle = 0$ for $i \neq j$, then we obtain that
%
\[ \| f \|_X = N^{1/2}. \]
%
Thus we see \emph{square root cancellation}. For general values $f_1,\dots, f_N \in X$, this calculation shows us that
%
\[ \| f \|_X = \left( \sum_i \| f_i \|_X^2 \right)^{1/2}. \]
%
More generally, if one assumes an \emph{almost orthogonality condition}. For instance, if we assume that for each $i$, there are at most $O(1)$ indices $j$ with $\langle f_i, f_j \rangle \neq 0$, and if we rearrange our vectors so that
%
\[ \| f_1 \|_X \geq \cdots \geq \| f_N \|_X, \]
%
then for any $i$ and $j$, we can use the trivial Cauchy-Schwartz bound
%
\[ |\langle f_i, f_j \rangle| \leq \| f_i \|_X \| f_j \|_X \leq \| f_i \|_X^2 \]
%
to conclude that
%
\begin{align*}
  \| f \|_X^2 &= \sum_i \| f_i \|_X^2 + 2 \sum_{i < j} \| f_i \|_X \| f_j \|_X\\
  &\leq \sum_i \| f_i \|_X^2 + O(1) \sum_i \| f_i \|_X^2\\
  &\lesssim \sum_i \| f_i \|_X^2,
\end{align*}
%
and thus
%
\[ \| f \|_X \lesssim \left( \sum_i \| f_i \|_X^2 \right)^{1/2}. \]
%
This situation, in which uncorrelated vectors have a \emph{square root cancellation inequality}, occurs often in harmonic analysis. For instance, in $X = L^2(\RR^d_x)$, harmonic analysts often decompose $\RR^d_\xi$ down into a union of finitely overlapping pieces $\{ \theta \}$, consider an associated partition of unity $\{ \chi_\theta \}$ adapted to this cover, and then performing a frequency decomposition, writing a general $f \in L^2(\RR^d_x)$ as
%
\[ f = \sum_\theta f_\theta, \]
%
where $\widehat{f_\theta} = \chi_\theta \widehat{f}$. The fact that these pieces are finitely overlapping implies they will almost orthogonal, so that
%
\[ \| f \|_X \lesssim \left( \sum_\theta \| f_\theta \|_X^2 \right)^{1/2}. \]
%
Decoupling theory is the study of when one can obtain square root cancellation bounds for more general norm spaces $X$, most importantly, when $X$ is an $L^p$ space for $p > 2$. Such estimates are called \emph{decoupling estimates}

One might expect that decoupling is connected to reverse square function estimates of the form
%
\[ \left\| f \right\|_X \lesssim \left\| \left( \sum_i |f_i|^2 \right)^{1/2} \right\|_X. \]
%
A classical example of such an estimate is Littlewood-Paley theory. In the region $p > 2$, reverse square function estimates are stronger than decoupling estimates. But the advantage of decoupling is that they same to be easier to \emph{iterate}. If we decompose $f = \sum f_i$, and then perform a further decomposition $f_i = \sum f_{ij}$, and we are able to obtain inequalities of the form
%
\[ \| f \|_X \lesssim \left( \sum_i \| f_i \|_X^2 \right)^{1/2} \]
%
and
%
\[ \| f_i \|_X \lesssim \left( \sum_j \| f_{ij} \|_X^2 \right)^{1/2}, \]
%
then by combining these bounds we automatically obtain a bound of the form
%
\[ \| f \|_X \lesssim \left( \sum_{i,j} \| f_{ij} \|_X^2 \right)^{1/2}. \]
%
Thus decoupling often works well with \emph{induction on scales}.


%For instance, for $1 < p < \infty$, if $f_1,\dots,f_N \in L^p(\RR^d)$ are arbitrary, and $\varepsilon_1,\dots,\varepsilon_N$ are independent $\{ \pm 1 \}$ valued Bernoulli random variables, then
%
%\[ \left( \EE \| \varepsilon_1 f_1 + \dots + \varepsilon_N f_N \|_{L^p(\RR^d)}^p \right)^{1/p} \sim_p \left\| \left( \sum |f_1|^2 + \dots + |f_N|^2 \right)^{1/2} \right\|_{L^p(\RR^d)}. \]
%
%But this means that for many choices of signs $\varepsilon_1,\dots,\varepsilon_N$, 






%
%\begin{itemize}
%  \item If $X = L^p(\RR^d)$ for $p > 1$, and $x_1,\dots,x_N$ are functions with disjoint supports, then
  %
%  \[ \| x_1 + \dots + x_N \|_X = N^{1/p}, \]
  %
%  which gives an improvement for $p > 1$.

%  \item If $X = L^p(\RR^d)$ for $1 < p < \infty$, and $\varepsilon_1,\dots,\varepsilon_N$ are $\{ \pm 1 \}$ valued uniform Bernoulli random variabes, then
  %
%  \[ \EE \| \varepsilon_1 x_1 + \dots + \varepsilon_N x_N \|_X \lesssim_p N^{1/2}. \]
  %
%  Thus unless significant constructive or deconstructive interference occurs, then we encounter significant square root cancellation.

%  \item If $X$ is a Hilbert space, and $x_1,\dots,x_N$ are pairwise orthogonal, then Bessel's inequality implies that
  %
%  \[ \| x_1 + \dots + x_N \|_X \leq N^{1/2}. \]
%\end{itemize}
%
%We will focus on bounds like the latter two examples with a bound of $N^{1/2}$, i.e. determining when we can get \emph{square root cancellation bounds}, which seems to be the case of most significance in many situations of harmonic analysis.

%We are interested in determining what causes `square root cancellation' in more general norm spaces than just Hilbert. The theory of \emph{almost orthogonality} studies this phenomena in Hilbert spaces, but we are interested in studying when this phenomenon in other norm spaces. Informally, we say $x_1, \dots, x_N$ satisfies a \emph{decoupling inequality} in a norm space $X$ if for all $\varepsilon > 0$, we have
%
%\[ \| x_1 + \dots + x_N \|_X \lesssim_\varepsilon N^\varepsilon \left( \| x_1 \|_X^2 + \dots + \| x_N \|_X^2 \right)^{1/2}. \]
%
%Thus decoupling theory is the study of when certain values can be correlated with one another, in general norm spaces. Of particular importance in harmonic analysis is the determination of the properties the Fourier transform of a family of functions must have to enable us to obtain decoupling phenomena.

\begin{remark}
  We are primarily interested in studying decoupling in $L^p(\Omega)$. However, in this case we should only expect decoupling to occur when $p \geq 2$. Functions in $L^p$ spaces are `most orthogonal' when their supports are disjoint. If a family of functions $\{ f_i \}$ have disjoint supports, then we actually have
  %
  \[ \| f_1 + \dots + f_N \|_{L^p(\Omega)} = \left( \| f_1 \|_{L^p(\Omega)}^p + \dots + \| f_N \|_{L^p(\Omega)}^p \right)^{1/p}. \]
  %
  For $p < 2$, we have a sharp inequality
  %
  \[ \left( \| f_1 \|_{L^p(\Omega)}^p + \dots + \| f_N \|_{L^p(\Omega)}^p \right)^{1/p} \leq N^{1/p - 1/2} \left( \| f_1 \|_{L^p(\Omega)}^2 + \dots + \| f_N \|_{L^2(\Omega)}^2 \right)^{1/2}, \]
  %
  so we do not have square root cancelation in this setting. On the other hand, when $p \geq 2$ we have
  %
  \[ \left( \| f_1 \|_{L^p(\Omega)}^p + \dots + \| f_N \|_{L^p(\Omega)}^p \right)^{1/p} \leq \left( \| f_1 \|_{L^p(\Omega)}^2 + \dots + \| f_N \|_{L^2(\Omega)}^2 \right)^{1/2} \]
  %
  and so we obtain a decoupling inequality for functions with disjoint support.
\end{remark}

The most basic case where we can obtain decoupling in $L^p(\Omega)$ for $p > 2$ is when $p$ is an even integer. As a basic example, say a family of functions $f_1, \dots, f_N \in L^4(\Omega)$ are \emph{biorthogonal} if $\{ f_i f_j : i < j \}$ forms an orthogonal family in $L^2(\Omega)$.

\begin{theorem}
  If $f_1, \dots, f_N$ are a biorthogonal family, then
  %
  \[ \| f_1 + \dots + f_N \|_{L^4(\Omega)} \lesssim \left( \| f_1 \|_{L^4(\Omega)}^2 + \dots + \| f_N \|_{L^4(\Omega)}^2 \right)^{1/2}. \]
\end{theorem}
\begin{proof}
  We write
  %
  \begin{align*}
    \left\| f_1 + \dots + f_N \right\|_{L^4(\Omega)}^2 &= \left\| (f_1 + \dots + f_N)^2 \right\|_{L^2(\Omega)}\\
    &= \left\| \sum_{1 \leq i,j \leq N} f_i f_j \right\|_{L^2(\Omega)}\\
    &\lesssim \sum_{i = 1}^N \| f_i^2 \|_{L^2(\Omega)} + \left\| \sum_{1 \leq i < j \leq N} f_i f_j \right\|_{L^2(\Omega)}
  \end{align*}
  %
  Applying Bessel's inequality, we conclude that
  %
  \begin{align*}
    \left\| \sum_{1 \leq i < j \leq N} f_i f_j \right\|_{L^2(\Omega)} &= \left( \sum_{1 \leq i < j \leq N} \| f_i f_j \|_{L^2(\Omega)}^2 \right)^{1/2}\\
    &= \left\| \sum_{i = 1}^N |f_i|^2 \right\|_{L^2(\Omega)} \lesssim \sum_{i = 1}^N \| f_i^2 \|_{L^2(\Omega)}.
  \end{align*}
  %
  Combining these calculations, noticing that $\| f_i^2 \|_{L^2(\Omega)} = \| f_i \|_{L^4(\Omega)}^2$, and taking in square roots completes the claim.
\end{proof}

\begin{remark}
  The calculation above shows a decoupling inequality still holds if the family $f_i f_j$ is almost biorthogonal, e.g. if each of the elements of the family $\{ f_i f_j \}$ are orthogonal to all but $O_\varepsilon(N^\varepsilon)$ of the other elements of the family.
\end{remark}

\begin{remark}
  Similarily, if $f_1, \dots, f_N \in L^6(\Omega)$ are chosen to be \emph{triorthogonal}, in the sense that the family of functions $\{ f_i f_j f_k \}$ are orthogonal to one another, one can obtain a decoupling inequality in the $L^6$ norm.
\end{remark}

We will be most interested in studying families of functions $f_1, \dots, f_N$ with disjoint Fourier supports in $L^p(\RR^d)$, where $p \geq 2$. It is then certainly true that
%
\[ \| f_1 + \dots + f_N \|_{L^2(\RR^d)} \leq \left( \| f_1 \|_{L^2(\RR^d)}^2 + \dots + \| f_N \|_{L^2(\RR^d)}^2 \right)^{1/2}. \]
%
However, for $p > 2$ having disjoint Fourier supports does not immediately imply decoupling, because constructive interference can still ocur in the $L^p$ norm in a way not detected in the $L^2$ norm, and we require addition features of the family in order to guarantee that this constructive interference does not occur.

\begin{theorem}
  If $f_1, \dots, f_N \in L^p(\RR^d)$, have disjoint Fourier support, and $2 \leq p \leq \infty$, then
  %
  \[ \| f_1 + \dots + f_N \|_{L^p(\RR^d)} \leq N^{1/2 - 1/p} \left( \| f_1 \|_{L^p(\RR^d)}^2 + \dots + \| f_N \|_{L^p(\RR^d)}^2 \right)^{1/2}. \]
\end{theorem}
\begin{proof}
  For $p = \infty$, we have the trivial inequality
  %
  \begin{align*}
    \| f_1 + \dots + f_N \|_{L^\infty(\RR^d)} &\leq \| f_1 \|_{L^\infty(\RR^d)} + \dots + \| f_N \|_{L^\infty(\RR^d)}\\
    &\leq N^{1/2} \left( \| f_1 \|_{L^\infty(\RR^d)}^2 + \dots + \| f_N \|_{L^\infty(\RR^d)}^2 \right)^{1/2}.
  \end{align*}
  %
  By a density argument (applying a cutoff in frequency space), we may assume without loss of generality that the functions $\{ f_i \}$ are Schwartz. But then by orthogonality of the Fourier transform implies the functions themselves are orthogonal, so we have
  %
  \[ \| f_1 + \dots + f_N \|_{L^2(\RR^d)} \leq \left( \| f_1 \|_{L^2(\RR^d)}^2 + \dots + \| f_N \|_{L^2(\RR^d)}^2 \right)^{1/2}. \]
  %
  We can then interpolate with the case $p = \infty$.
\end{proof}

In general, this result is optimal, as the next result shows one can have significant constructive interference for $p > 2$ if our functions are spaced about on a periodic family of frequencies.

\begin{example}
  Fix $\phi \in \mathcal{S}(\RR^d)$ with $\phi(0) = 1$, and with Fourier support in $[0,1]$. For each $k \in \{ 1, \dots, N \}$, set $f_k = e^{2 \pi i k x} \phi(x)$. Then $f_k$ has Fourier support in $[2k,2k+1]$, and $f_k(0) = 1$. The uncertainty principle thus implies that the functions $f_k$ are all roughly constant at a scale $O(1/N)$, which implies that in a ball of radius $O(1/N)$ around the origin, we have $f_1(x) + \dots + f_n(x) \approx 1$. Thus we conclude that
  %
  \[ \| f_1 + \dots + f_N \|_{L^p(\RR)} \gtrsim N^{1 - 1/p}. \]
  %
  On the other hand, we have
  %
  \[ \left( \| f_1 \|_{L^p(\RR)}^2 + \dots + \| f_N \|_{L^p(\RR)}^2 \right)^{1/2} \lesssim N^{1/2}, \]
  %
  Thus
  %
  \[ \| f_1 + \dots + f_N \|_{L^p(\RR)} \gtrsim N^{1/2 - 1/p} \left( \| f_1 \|_{L^p(\RR)}^2 + \dots + \| f_N \|_{L^p(\RR)}^2 \right)^{1/2}, \]
  %
  which shows our result is tight up to constants.
\end{example}

In the face of this result, we are interested in knowning, for a given family $\mathcal{S}$ of disjoint sets in $\RR^d$, what the smallest constant $\text{Dec}(S,p)$ is for which it is true that if $f_1, \dots, f_N$ have Fourier support on distinct regions $S_1, \dots, S_N \in \mathcal{S}$, we have
%
\[ \| f_1 + \dots + f_N \|_{L^p(\RR^d)} \leq \text{Dec}(\mathcal{S},p) \left( \| f_1 \|_{L^p(\RR^d)}^2 + \dots + \| f_N \|_{L^p(\RR^d)}^2 \right)^{1/2}. \]
%
Pure orthogonality gives $\text{Dec}(\mathcal{S},2) = 1$. The triangle inequality gives $\text{Dec}(\mathcal{S},\infty) \leq \#(S)^{1/2}$, and this is actually sharp: we have $\text{Dec}(\mathcal{S},\infty) $

If each element of $\mathcal{S}$ contains a ball of radius $\delta$, then by summing up modulated bump functions on these balls, we can find functions such that $\text{Dec}(\mathcal{S},\infty) \gtrsim \delta^d \#(S)^{1/2}$. Thus non-trivial decoupling in the framework we are considering is impossible in $L^\infty(\RR^d)$, i.e. $\text{Dec}(\mathcal{S},\infty) \sim_\delta \#(S)^{1/2}$. In between, we can interpolate to get $\text{Dec}(\mathcal{S},p) \leq \#(S)^{1/2 - 1/p}$.

Such a result depends significantly on the geometric structure of the regions in $\mathcal{S}$. The techniques we will use (e.g. induction on scales) imply the need for the `$\varepsilon$ loss' given by the $N^\varepsilon$ factor above. Below is a positive result for a particular family $\mathcal{S}$, easily proved using the biorthogonality arguments established above.

\begin{theorem}
  Suppose $\mathcal{S}$ is a family of sets in $\RR^d$ such that for any four sets $S_1,S_2,S_3,S_4 \in \mathcal{S}$ with $\{ S_1, S_2 \} \neq \{ S_3, S_4 \}$, the sets $S_1 + S_2$ are disjoint from $S_3 + S_4$. Then if distinct sets $S_1, \dots, S_N \in \mathcal{S}$ are selected from $\mathcal{S}$, and $f_1, \dots, f_N$ are a family of Schwartz functions in $\RR^d$ such that $f_i$ has Fourier support in $S_i$ for each $i$, we find`'
  %
  \[ \| f_1 + \dots + f_N \|_{L^4(\Omega)} \lesssim \left( \| f_1 \|_{L^4(\Omega)}^2 + \dots + \| f_N \|_{L^4(\Omega)}^2 \right)^{1/2}. \]
\end{theorem}

\begin{remark}
  We say a set of integers $A \subset \{ 1, \dots N \}$ is a \emph{Sidon set} if there does not exist a nontrivial solution to the equation $a_1 + a_2 = a_3 + a_4$. If $A$ is Sidon, then $\mathcal{S} = \{ [2k,2k+1]: k \in A \}$ satisfies the constraints of the result above, and so we obtain that if $\{ f_k: k \in A \}$ are a family of Schwartz functions such that $f_k$ has Fourier support in $[2k,2k+1]$, then
  %
  \[ \| \sum_{k \in A} f_k \|_{L^4(\RR)} \lesssim \left( \sum_{k \in A} \| f_k \|_{L_4(\RR)}^2 \right)^{1/2}. \]
  %
  On the other hand, a variant of the example above shows that for any Sidon set $A$, there is a family of functions $\{ f_k : k \in A \}$ with $f_k$ having Fourier support on $[2k,2k+1]$, and with
  %
  \[ \left\| \sum_{k \in A} f_k \right\|_{L^4(\RR)} \gtrsim \frac{\#(A)^{1/2}}{N^{1/4}} \left( \sum_{k \in A} \| f_k \|_{L^4(\RR)}^2 \right)^{1/2}. \]
  %
  Combining this inequality with the decoupling inequality, we obtain an interesting number theory result: any Sidon set $A$, must satisfy the bound $\#(A) \lesssim N^{1/2}$. More generally, we can extend this result to show that any set $A \subset \{ 0, \dots, N-1 \}$ having no nontrivial solutions to the equation
  %
  \[ x_1 + \dots + x_m = y_1 + \dots + y_m \]
  %
  should satisfy $\#(A) \lesssim N^{1/m}$.
\end{remark}

Another example family of sets $\mathcal{S}$ where Decoupling occurs occurs in Littlewood-Paley theory.

\begin{theorem}
  Let $\mathcal{S}$ be the collection of all boxes in $\RR^d$ of the form
  %
  \[ I_k = I_{\pm k_1} \times \dots \times I_{\pm k_d} \]
  %
  such that $I_{+ k_1} = [2^{k_1}, 2^{k_1 + 1}]$ and $I_{-k_1} = [-2^{k_1+1}, -2^{k_1}]$. Littlewood-Paley theory implies that if $S_1, \dots, S_N \in \mathcal{S}$ and $f_1, \dots, f_N$ are Schwartz functions with $f_i$ having Fourier support on $S_i$ for each $i$, then for each $2 \leq p < \infty$,
  %
  \[ \| f_1 + \dots + f_N \|_{L^p(\RR^d)} \sim_{p,d} \| f_i \|_{L^p(\RR^d) l^2_N} \lesssim \| f_i \|_{l^2_N L^p(\RR^d)}. \]
  %
  This is precisely a decoupling inequality. More generally, whenever we have a reverse square function estimate
  %
  \[ \| f_1 + \dots + f_N \|_{L^p(\RR^d)} \lesssim \| f_i \|_{L^p_x l^2_N}, \]
  %
  we get a decoupling inequality by changing norms.
\end{theorem}

We say a set $\omega \subset \RR^d$ is an \emph{almost rectangular box} with sidelengths $L_1,\dots,L_d$ if there is a rectangular box $R$ with sidelengths $L_1, \dots, L_d$ centered at the origin and $\theta \in \RR^d$ such that $\theta + C^{-1} R \subset \theta \subset \theta + C \cdot R$ for some universal constant $C$. As a special case, we have \emph{almost cubes} as well. For each $R \geq 1$, partition $[-1,1]^{d-1}$ into $\sim R^{(d-1)/2}$ almost rectangular boxes of sidelength $R^{-1/2}$. If we consider the paraboloid $\mathbf{P}^{n-1} = \{ \xi, |\xi|^2 \}$, then we obtain a partition of $N(\mathbf{P}^{n-1} \cap [0,1]^d, R^{-1})$ by a family $\Theta(1/R)$ of $R^{-1/2} \times R^{-1}$ almost rectangular boxes, by setting
%
\[ \Theta(1/R) = \{ (\omega \times \RR) \cap N(\mathbf{P}^{n-1} \cap [0,1]^d ) \} \]
%
It is conjectured that for $p = 2n/(n-1)$, and for Schwartz functions $f_1, \dots, f_N$ with Fourier support on distinct elements of $\Theta(1/R)$, we have
%
\[ \| f_1 + \dots + f_N \|_{L^p(\RR^d)} \lessapprox_R \| f_i \|_{L^p(\RR^d) l^2_N}, \]
%
This reverse square function estimates is strong. In particular, it implies the restriction conjecture. But it also implies the decoupling estimate
%
\[ \| f_1 + \dots + f_N \|_{L^p(\RR^d)} \lesssim_\varepsilon R^\varepsilon \left( \| f_1 \|_{L^p(\RR^d)} + \dots + \| f_N \|_{L^p(\RR^d)} \right)^{1/2} \].
%
However, unlike the restriction conjecture, this problem is closed: we have a proof of this decoupling estimate not only when $p = 2d/(d-1)$, but even when $2 \leq p \leq 2(d+1)/(d-1)$.

\begin{comment}
\begin{example}
  TODO: Move this. When $p < 2$, one does not usually expect to find decoupling inequalities in $L^p(\Omega)$. For instance, for any family of disjoint measurable sets $E_1, \dots, E_N \in \Omega$, each with non-negative measure, one can find $f_1, \dots, f_N \in L^p(\Omega)$, with $f_i$ supported on $E_i$ for each $i$ such that
  %
  \begin{align*}
    \| f_1 + \dots + f_N \|_{L^p(\Omega)} &= \left( \| f_1 \|_{L^p(\Omega)}^p + \dots + \| f_N \|_{L^p(\Omega)}^p \right)^{1/p}\\
    &\geq N^{1/p - 1/2} \left( \| f_1 \|_{L^p(\Omega)}^2 + \dots + \| f_N \|_{L^p(\Omega)}^2 \right)^{1/2}.
  \end{align*}
  %
  The idea of this is simple; we just choose a family of scalars $A_1, \dots, A_N$ such that
  %
  \[ (A_1^p + \dots + A_N^p)^{1/p} = N^{1/p - 1/2} (A_1^2 + \dots + A_N^2)^{1/2}. \]
  %
  Given functions $f_1, \dots, f_N$ such that $f_i$ is supported in $E_i$ for each $i$, we need only rescale each function such that $\| f_i \|_{L^p(\Omega)} = A_i$ for each $i$. Similarily, if $U_1, \dots, U_N$ are disjoint open sets in $\RR^d$, we can find Schwartz functions $f_1, \dots, f_N$, such that $f_i$ has Fourier support in $U_i$ for each $i$, such that
  %
  \[ \| f_1 + \dots + f_N \|_{L^p(\Omega)} \gtrsim N^{1/p - 1/2} \left( \| f_1 \|_{L^p(\Omega)}^2 + \dots + \| f_N \|_{L^p(\Omega)}^2 \right)^{1/2}, \]
  %
  where the implict constant is independant of $N$, and $U_1, \dots, U_N$. The idea here is to begin with Schwarz functions $f_1, \dots, f_N$ such that $f_i$ has Fourier suppport in $U_i$, and then replace these Schwarz functions with translations such that the masses of the $f_i$ are essentially disjoint from one another, which only modulates the Fourier transform and so does not affect the Fourier support of the functions. Rescaling then gives the result.
\end{example}
\end{comment}

\section{Localized Estimates}

Suppose $f_1, \dots, f_N$ are Schwartz functions in $\RR^d$ with disjoint Fourier supports, and $\Omega \subset \RR^d$. A natural question to ask is when one should expect
%
\[ \| f_1 + \dots + f_N \|_{L^2(\Omega)}^2 \lesssim \| f_1 \|_{L^2(\Omega)}^2 + \dots + \| f_N \|_{L^2(\Omega)}^2. \]
%
If we consider the bump function counterexample constructed from earlier, and let $\Omega = \{ x \in \RR: |x| \lesssim 1/N \}$, then $\| f_1 + \dots + f_N \|_{L^2(\Omega)} \gtrsim N$, whereas $\| f_k \|_{L^2(\Omega)}^2 \lesssim 1/N$ so $\| f_1 \|_{L^2(\Omega)}^2 + \dots + \| f_N \|_{L^2(\Omega)}^2 \lesssim 1$, which means such a result cannot be obtained. However, we shall find that such a result holds if $\Omega$ is large enough, depending on the supports of $f_1, \dots, f_N$, and if we allow weighted estimates.

Let us begin with the case in one dimension. Given an interval $I$ with centre $x_0$, and length $R$, we consider the weight function
%
\[ w_I(x) = \left( 1 + \frac{|x - x_0|}{R} \right)^{-M} \]
%
It is a useful heuristic that if $f$ has Fourier support in $I$, then $f$ is `locally constant' on intervals of length $1/|I|$.

In $\RR^d$, given a ball $B$ with centre $x_0$ and radius $R$, we consider the weight function
%
\[ w_B(x) = \left( 1 + \frac{|x - x_0|}{R} \right)^{-M}, \]
%
where $M$ is a large integer. Then
%
\[ \int w_B(x)\; dx \]

TODO FINISH THIS

\section{Vinogradov Systems}

One long standing number theoretic conjecture that has been particularly amenable to the use of decoupling techniques is \emph{Vinogradov's Conjecture}. Let $J_{s,k}(N)$ denote the number of solutions to the \emph{system} of equations
%
\[ x_1^i + \dots + x_s^i = y_1^i + \dots + y_s^i \]
%
for $1 \leq i \leq k$, where $x_1,\dots,x_s,y_1,\dots,y_s \in \{ 1, \dots, N \}$. Our primary interest is determining the asymptotic growth in this quantity as $N \to \infty$. Setting $y_i = x_i$ gives a family of $N^s$ solutions. Thus $J_{s,k}(N) \geq N^s$. On the other hand, dyadic pidgeonholing implies that there exists some family of integer tuples $I \subset [1,N] \times \dots \times [1,N^k]$ and some integer $M$ with $N^s = M \#(I)$ such that for each tuple $(r_1,\dots,r_k) \in I$, there are $\Omega(M)$ tuples of values $(x_1,\dots,x_s)$ such that for each $1 \leq i \leq k$,
%
\[ x_1^i + \dots + x_s^i = r_i. \]
%
For any $i$, $x_1^i + x_s^i$ takes on values between $1$ and $N^i$. Since $1 \leq \#(I) \leq N^{k(k+1)/2}$, we have $N^{s - k(k+1)/2} \lesssim M \lesssim N^s$, and so
%
\[ J_{s,k}(N) \gtrsim \#(I) M^2 \gtrsim N^{2s - k(k+1)/2}, \]
%
a bound that is much tighter if the number of variables is large, i.e. $s \geq k(k+1)/2$. Vinogradov's mean value conjecture is that
%
\[ J_{s,k}(N) \lesssim_{s,k,\varepsilon} N^\varepsilon ( N^{2s-k(k+1)/2} + N^s ), \]
%
for any $\varepsilon > 0$. One result of decoupling is a proof of this conjecture.

\begin{remark}
  Another approach to the Vinogradov mean value conjecture is to use more number theoretic techniques, for instance, efficient congruencing. There has even been some feedback from this alternate approach in the field of decoupling, e.g. using efficient congruencing to obtain decoupling inequalities (see Guo-Li-Yung-Zorin Kranich, 2021).
\end{remark}

Similar types of number theoretic problems had been studied using harmonic analysis techniques. Classically, Hardy and Littlewood studied the quantity $\text{HL}_{k,s}(N)$, which counts the number of solutions to the \emph{single equation}
%
\[ x_1^k + \dots + x_s^k = y_1^k + \dots + y_s^k \]
%
with $x_1,\dots,x_s,y_1,\dots,y_s \in \{ 1, \dots, N \}$. To study this quantity, Hardy and Littlewood introduced the function
%
\[ f_{k,N}(x) = \sum_{n = 1}^N e^{2 \pi i n^k x}. \]
%
Orthogonality then implies that
%
\[ \text{HL}_{k,s}(N) = \| h_{k,N} \|_{L^{2s}[0,1]}. \]
%
Bounding the nature of the function $h_{k,N}$ then involves Hardy and Littlewood's circle method. On the other hand, for Vinogradov's conjecture we are lead to study a higher dimensional function
%
\[ j_{k,N}(x) = \sum_{n = 1}^N e^{2 \pi i (n x_1 + n^2 x_2 + \dots + n^k x_k)}, \]
%
in which case
%
\[ J_{k,s}(N) = \| j_{k,N} \|_{L^{2s}[0,1]^k}. \]
%
Though in higher dimensions, Vinogradov initially expected his result to be \emph{easier} than Hardy and Littlewood's problem, because of the \emph{rescalable} nature of the problem, which we will get to later. Indeed, decoupling has lead to a solution of Vinogradov's conjecture for almost all parameters, but a complete solution to the conjectured bound
%
\[ \text{HL}_{k,s}(N) \lesssim_\varepsilon N^\varepsilon ( N^s + N^{2s - k} ) \]
%
has not yet been resolved.

To do our analysis, it will be helpful to rescale our quantities. Thus we let
%
\[ f_{k,N}(x) = \sum_{n = 1}^N e^{2 \pi i ( (n/N) x_1 + \dots + (n/N)^k x_k )}. \]
%
The function $f_{k,N}$ is then $N$-periodic, and so Vinogradov's conjecture will follow from showing that
%
\[ \| f_{k,N} \|_{L^{2s}[0,N^k]^k} = N^{k^2/2s} \| j_{k,N} \|_{L^{2s}[0,1]^k} \]
%
satisfies a bound
%
\[ \| f_{k,N} \|_{L^{2s}[0,N^k]^k} \lesssim_\varepsilon N^{k^2/2s + \varepsilon} (N^s + N^{2s - k}). \]
%
This is what we will address using the theory of decoupling.

From the perspective of decoupling, it will be more handy to study the quantities
%
\[ \tilde{j}_{k,N}(x) = \sum_{n = 1}^N e^{2 \pi i (n x_1 + \dots + n^k x_k)} \phi(x), \]
%
and
%
\[ \tilde{f}_{k,N}(x) = \sum_{n = 1}^N e^{2 \pi i ((n/N) x_1 + \dots + (n/N)^k x_k)} \phi(x_1 / N, \dots, x_k / N^k), \]
%
where $\phi \in \mathcal{S}(\RR^d)$ has Fourier support in $|\xi| \leq 1/10$ and $\phi(x) \geq 1$ in a neighborhood of the origin. Bounding the $L^{2s}$ norm of $\tilde{j}_{k,N}$ and $\tilde{f_{k,N}}$ is essentially equivalent to bounding $j_{k,N}$ since we can upper bound $j_{k,N}$ and $f_{k,N}$ pointwise by $O(1)$ sums of this form on the domain upon which we are taking the $L^{2s}$ norm. Now if we let
%
\[ f_i = e^{2 \pi i ((n/N) x_1 + \dots + (n/N)^k x_k)} \phi(x_1 / N, \dots, x_k / N^k), \]
%
then $\widehat{f_i}$ is supported on a rectangle with sidelengths $O(N) \times \dots \times O(N^k)$ centered at the point $(n/N, \dots, (n/N)^k)$ of the moment curve. The $l^2$ decoupling inequality for the moment curve thus implies that
%
\[ \| f \|_{L^{2s}(\RR^d)} \lesssim_\varepsilon N^{1/2 + \varepsilon}, \]
%
which proves Vinogradov's mean value result for $s \geq k^3 / (2k^2 - 1)$, i.e. for $s \gtrsim k$, which, up to constants, is the current state of the art for the mean-value theorem.










\section{Multilinear Kakeya}

If $f_1,\dots, f_N$ are functions, then H\"{o}lder's inequality implies that
%
\[ \| f_1 \dots f_N \|_{L^p(\RR^d)} \leq \| f_1 \|_{L^{Np}(\RR^d)} \cdots \| f_N \|_{L^{Np}(\RR^d)}. \]
%
This is tight if the $\{ f_i \}$ are large in the same place. In the case of decoupling into caps on a parabola, we end up with functions that are locally constant on transverse tubes. In this case, we can significantly improve upon H\"{o}lder's inequality. In two dimensions, if we have two functions $f_1$ and $f_2$ constant in orthogonal directions, then by a rotation, we can assume that $f_1$ is a function of the $x$ variable, and $f_2$ a function of the $y$ variable. We then compute using Fubini's theorem that
%
\[ \| f_1 f_2 \|_{L^p(\RR^2)} = \| f_1 \|_{L^p(\RR)} \| f_2 \|_{L^p(\RR)}. \]
%
This is a significant improvement upon an application of H\"{o}lder's inequality. The Loomis-Whitney inequality generalizes this to more general functions locally constant in orthogonal directions.

\begin{theorem}[Loomis-Whitney]
  If $\pi_i(x) = (x_1,\dots,x_{i-1},x_{i+1},\dots,x_d)$ are projections onto orthogonal hyperplanes, and $f_1,\dots,f_n: \RR^{n-1} \to [0,\infty]$ are functions, then
  %
  \[ \int_{\RR^d} \prod_{j = 1}^d f_j(\pi_j(x))^{\frac{1}{d-1}}\; dx \leq \prod_{j = 1}^d \| f_j \|_{L^1(\RR^{d-1})}^{\frac{1}{n-1}}. \]
\end{theorem}
\begin{proof}
  Prove $d = 2$ case as above, and then apply induction TODO.
\end{proof}

Changing coordinates gives an affine-invariant version of the ienquality.

\begin{theorem}
  If $\nu_1,\dots,\nu_d$ are unit normal vectors to hyperplanes $H_1,\dots,H_d$ through the origin with associated projection maps $\pi_j: \RR^d \to H_j$, then for any functions $f_j: H_j \to [0,\infty]$,
  %
  \[ \int_{\RR^d} \prod_{j = 1}^d f_j(\pi_j(x))^{\frac{1}{d-1}}\; dx \leq |\nu_1 \wedge \dots \wedge \nu_d|^{- \frac{1}{d-1}} \prod_{j = 1}^d \| f_j \|_{L^1(H_j)}^{\frac{1}{d-1}}. \]
\end{theorem}

The multilinear Kakeya inequality follows by replacing the $d$ unit vectors $\{ \nu_j \}$ by $d$ families of unit vectors $\mathcal{N}_1, \dots, \mathcal{N}_d$ such that any choice of $d$ from this familiy is $\alpha$ transverse, in the sense that
%
\[ |\nu_1 \wedge \dots \wedge \nu_j| \geq \alpha. \]
%
These unit vectors generate a family of radius $r$ tubes $\mathcal{T}_1, \dots, \mathcal{T}_j$. We then consider a ball $B$ of radius $R \geq r$.

\begin{theorem}[Multilinear Kakeya]
  We have
  %
  \[ \int_B \prod_{j = 1}^d \left( \sum_{T \in \mathcal{T}_j} a_T \mathbf{I}_T \right)^{\frac{1}{d-1}} \lesssim_\varepsilon \alpha^{-O(1)} (R/r)^{\varepsilon} r^d \prod_{j = 1}^d \left( \sum_{T \in \mathcal{T}_j} a_T \right)^{\frac{1}{d-1}} \]
\end{theorem}
\begin{proof}
  SEE ZORIN KRANICH NOTES: (Guth 10) shows $(R/r)^\varepsilon$ can be removed, and the power of $\alpha$ can be taken to be $1/(d-1)$.
\end{proof}




