\documentclass[12pt,a4paper]{article}

\usepackage{amsmath,amsthm,amsfonts,amssymb,amscd}
\usepackage{exsheets}
\usepackage{paralist}
\usepackage{fancyhdr}
\usepackage[top=2cm, bottom=4.5cm, left=2.5cm, right=2.5cm]{geometry}
\usepackage{enumitem}
\usepackage{multicol} %Allows to distribute \enumerates in multiple columns
\usepackage{multirow}

\theoremstyle{definition}
\newtheorem*{ssolution}{Solution} % This is implemented to not overlap with the exsheets package's solution environment.

\SetupExSheets{solution/print=true} %Set print=false to not print out solutions.
%\SetupExSheets{question/type=exam}
%\SetupExSheets[points]{name=point,name-plural=points}

\setlength{\parindent}{0.0in}
\setlength{\parskip}{0.05in}


\pagestyle{fancyplain}
\headheight 35pt
\lhead{\NetIDa}
\chead{\textbf{\Large Math 340 Worksheet 25}}
\lhead{Sections: 3.2 and 3.3}
\rhead{Date: 9.7.2022 \hspace{0.5in}}
\lfoot{}
\cfoot{}
\rfoot{\small\thepage}
\headsep 1.5em


\begin{document}
%% Written by: Jacob Denson
\PrintSolutionsTF{
    \textbf{The following are select solutions from the worksheet}.
}{
    \textbf{It may be useful for you to have this worksheet for future discussion sections.} \newline
    \textbf{It may be in your interest to solve the questions not in the order listed, but according to which questions you need practice with.}
    \textbf{Your TA may or may not give you specific advice or directions on which questions to try first.}
}


\begin{question} If $\det(A) = 3$ and $\det(B) = -1$, compute $\det(A^3 B^2 A (A^T)^{-2} B)$. \end{question}
\begin{solution}
	Using the multiplicative property of the determinant, we find that
	%
	\begin{align*}
		\det(A^3 B^2 A (A^T)^{-2} B ) &= \det(A)^3 \det(B)^2 \det(A) \det(A^T)^{-2} \det(B)\\
		&= \det(A)^4 \det(A^T)^{-2} \det(B)^3.
	\end{align*}
	%
	Using the fact that $\det(A^T) = \det(A)$, we find that
	%
	\[ \det(A)^4 \det(A^T)^{-2} \det(B)^3 = \det(A)^4 \det(A)^{-2} \det(B)^3 = \det(A)^2 \det(B)^3. \]
	%
	Now pluggin in the values above, we get that
	%
	\[ \det(A)^2 \det(B)^3 = 3^2 (-1)^3 = - 9. \]
\end{solution}




\begin{question}
	Compute the determinant of the $5 \times 5$ matrix
%
\[ A = \begin{pmatrix}
		3 & -1 & 1 & -3 & 2\\
		1 & -4 & -1 & 4 & 3\\
		-3 & 2 & 3 & 0 & -4\\
		-3 & 3 & -2 & -4 & 0\\
		-3 & 1 & -1 & 3 & 4
	\end{pmatrix} \]
	%
	You may use the following facts:
	%
	\[ \begin{vmatrix} -4 & -1 & 4 & 3\\
		2 & 3 & 0 & -4\\
		3 & -2 & -4 & 0\\
		1 & -1 & 3 & 4 \end{vmatrix} = -67 \quad \begin{vmatrix} -1 & 1 & -3 & 2 \\ 2 & 3 & 0 & -4\\
			3 & -2 & -4 & 0\\
		1 & -1 & 3 & 4 \end{vmatrix} = 354 \quad \begin{vmatrix}
		-1 & 1 & -3 & 2 \\
		-4 & -1 & 4 & 3\\
		3 & -2 & -4 & 0\\
		1 & -1 & 3 & 4 \end{vmatrix} = -294 \]
	\[ \begin{vmatrix}
		-1 & 1 & -3 & 2 \\
		-4 & -1 & 4 & 3\\
		2 & 3 & 0 & -4\\
		1 & -1 & 3 & 4 \end{vmatrix} = 300 \quad\text{and}\quad \begin{vmatrix}
		-1 & 1 & -3 & 2 \\
		-4 & -1 & 4 & 3\\
		2 & 3 & 0 & -4\\
		3 & -2 & -4 & 0 \end{vmatrix} = 5 \]
		%
\end{question}
\begin{solution}
	Using the cofactor expansion, we have
	%
	\[ \det(A) = (-67) - (354) + (-294) - (300) + (5) = -67 - 354 - 294 - 300 + 5 = -1010. \]
\end{solution}

\begin{question} Use determinants to compute the area of the triangle with points $(1,2)$, $(-1,3)$, and $(0,-1)$. How about the area of the five sided polygon with vertices $(1,0)$, $(1,3)$, $(-1,4)$, $(-2,0)$, and $(0,-1)$ (Hint: Break things into smaller triangles and apply the techniques of the last part of the question). \end{question}
\begin{solution}
	To find the first area, we use the fact that the area of the triangle is half the absolute value of the determinant of the matrix
	%
	\[ \begin{pmatrix} 1 & 2 & 1 \\ -1 & 3 & 1 \\ 0 & -1 & 1 \end{pmatrix}. \]
	%
	The determinant of this matrix is $7$, so the area of the triangle is $3/2$.

	To find the second area, we note that $(0,0)$ is an interior point of the polygon. The interior of the polygon thus decomposes into the five triangles, the first with vertices $(0,0)$ $(1,0)$, and $(1,3)$, the second with vertices $(0,0)$ $(1,3)$, $(-1,4)$, the third with vertices $(0,0)$, $(-1,4)$, $(-2,0)$, the fourth with vertices $(0,0)$, $(-2,0)$, $(0,-1)$, and the fifth with vertices $(0,0)$, $(0,-1)$, $(1,0)$. The advantage of using $(0,0)$ as a vertex is that the determinants simplify to $2 \times 2$ determinants. In general, given a triangle consisting of three vertices $(0,0)$, $(a,b)$, and $(c,d)$, the area of the triangle is half the absolute value of the determinant of the matrix
	%
	\[ \begin{pmatrix} 0 & 0 & 1 \\ a & b & 1 \\ c & d & 1 \end{pmatrix} \]
	%
	Using the cofactor expansion \emph{along columns} (which is equivalent to cofactor expansion along rows because -- think over why this connects -- the determinant of a matrix is the same as the determinant of it's transpose), we conclude that
	%
	\[ \begin{vmatrix} 0 & 0 & 1 \\ a & b & 1 \\ c & d & 1 \end{vmatrix} = 0 \cdot \begin{vmatrix} b & 1 \\ d & 1 \end{vmatrix} - 0 \cdot \begin{vmatrix} 0 & 1 \\ c & 1 \end{vmatrix} + 1 \cdot \begin{vmatrix} a & b \\ c & d \end{vmatrix} = ad - bc. \]
	%	
	Thus we find that the area is $(1/2)|ad - bc|$. The areas of the triangles above, in order, are $3/2$, $1/2$, $4$, $1$, and $1$, and summing up these areas gives that the area of the overall polygon is $8$.
\end{solution}

\begin{question}
Given  $n\times n$ matrices $A$, $B$ and $C$, decide if the following assertions are true or false, justify your answer. 
\begin{enumerate}
    \item $\det(A+B)=\det(A)+det(B)$.
    \item If $\det(AB)=0$, then $\det(A)=0$ or $\det(B)=0$.
    \item If $A$ is nonsingular and $A=A^{-1}$, then $\det(A)=1$.
    \item If $AB=AC$ and $\det(A)\neq 0$, then $B=C$.
\end{enumerate}
\end{question}

\begin{solution}
\begin{enumerate}
    \item False, take $n=2$, $A=\begin{bmatrix} 1 & 0 \\ 0 & 0\end{bmatrix}$ and $A=\begin{bmatrix} 0 & 0 \\ 0 & 1\end{bmatrix}$. we have $\det(A+B)=1$ but $\det(A)+\det(B)=0$.
    \item True, just notice that $\det(AB)=\det(A)\det(B)=0$. 
    \item False, Take $A=\begin{bmatrix} 1 & 0 \\ 0 & -1\end{bmatrix}$. We have $A=A^{-1}$ but $\det(A)=-1$.
    \item True, if $\det(A)\neq 0$ then $A$ is nonsingular. Multiplying at the left by $A^{-1}$ we get $B=A^{-1}AB=A^{-1}AC=C$.
\end{enumerate}
\end{solution}


\end{document}