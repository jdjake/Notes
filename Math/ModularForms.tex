\documentclass[12pt]{report}

\usepackage{amsmath}
\usepackage{amssymb}
\usepackage{amsthm}
\usepackage{amsopn}
\usepackage{kpfonts}
\usepackage{graphicx}
\usepackage{kbordermatrix}
\usepackage{tikz}
\usetikzlibrary{arrows, petri, topaths}%
\usepackage{tkz-berge}
\usepackage{multicol}

\usepackage{framed}
\usepackage{mathtools}
\usepackage{float}
\usepackage{subfig}
% \usepackage{cmbright}

\theoremstyle{plain}
\newtheorem{theorem}{Theorem}[chapter]
\newtheorem{lemma}[theorem]{Lemma}
\newtheorem{corollary}[theorem]{Corollary}
\newtheorem{prop}[theorem]{Proposition}
\newtheorem{exercise}{Exercise}[chapter]

\newtheorem*{example}{Example}
\newtheorem*{proof*}{Proof}

\theoremstyle{definition}
\newtheorem*{defi}{Definition}
\newenvironment{definition}
    {\begin{samepage}\begin{framed}\begin{defi}}
    {\end{defi}\end{framed}\end{samepage}}





\usepackage{hyperref} 
\hypersetup{
    colorlinks = true,
    linkcolor = black,
}

\makeatletter
\renewcommand*\env@matrix[1][*\c@MaxMatrixCols c]{%
  \hskip -\arraycolsep
  \let\@ifnextchar\new@ifnextchar
  \array{#1}}
\makeatother

\renewcommand*\contentsname{\hfill Table Of Contents \hfill}

\newcommand{\optionalsection}[1]{\section[* #1]{(Important) #1}}
\newcommand{\deriv}[3]{\left. \frac{\partial #1}{\partial #2} \right|_{#3}}

\DeclareMathOperator{\Dom}{Dom}

\title{Modular Forms}
\author{Jacob Denson}

\begin{document}

\pagenumbering{gobble}

\maketitle

\tableofcontents

\pagenumbering{arabic}

\chapter{Modular Forms}

\section{An Example From Number Theory}

Here is an interesting problem of number theory. The number 4 can written in 5 different ways as the sum of positive integers
%
\begin{align*}
    4\\
    1 + 3\\
    2 + 2\\
    1 + 1 + 2\\
    1 + 1 + 1 + 1
\end{align*}
%
The number 5 can be written in 7 different ways
%
\begin{align*}
    5\\
    4 + 1\\
    3 + 2\\
    3 + 1 + 1\\
    2 + 2 + 1\\
    2 + 1 + 1 + 1\\
    1 + 1 + 1 + 1 + 1
\end{align*}
%
We define the partition function $P(n)$ to be the number of ways a number $n$ can be written as the sum of positive integers. The challenge is to determine the properties of $P(n)$, since we are very unlikely to find an exact formula for the function. Thus $P(4) = 5$. The order of terms is irrelevant, so we normally assume the numbers are written in ascending or descending order. By bundling numbers which are repeated together, we can also define $P(n)$ as the number of ways we can write $n = \sum_{k = 1}^\infty k a_k$, where $a_k \geq 0$. For $4$, we have
%
\begin{align*}
    1 \cdot 4\\
    1 \cdot 1 + 1 \cdot 3\\
    2 \cdot 2\\
    2 \cdot 1 + 1 \cdot 2\\
    4 \cdot 1
\end{align*}
%
If we consider the infinite product
%
\[ \prod_{n = 1}^\infty \frac{1}{1 - q^n} = \prod_{n = 1}^\infty (1 + q^n + q^{2n} + \dots) = \sum_{n = 0}^\infty p_n q^n \]
%
which converges for $|q| < 1$, we see that $p_n = P(n)$. Thus $\eta$ `encodes' the partition function in someway. It turns out the Dedekind-Eta function
%
\[ \eta(q) = q^{\frac{1}{24}} \prod_{n = 1}^\infty (1 - q^n) \]
%
undoubtably linked to the partition function, is what is called a Modular form, an object with a beautiful general theory, which enables us to understand the properties of $\eta$, and thus of the partition function. These `modular forms' also occur when counting curves in algebraic geometry, and when understanding the dimensions of finite simple groups, so they are an incredibly useful object to study. Because they occur so often in mathematics, the number theorist Joseph Eichler has been told to have said that there are five fundamental arithmetical operations -- addition, subtraction, multiplication, division, and modular forms.

\section{The Modular Group}

The nicest holomorphic functions on $\mathbf{C}^\infty$ are the M\"{o}bius transformations, obtained as an action from $GL_2(\mathbf{C})$ by the map
%
\[ \begin{pmatrix} a & b \\ c & d \end{pmatrix}(z)  = \frac{az + b}{cz + d} \]
%
\[ \begin{pmatrix} a & b \\ c & d \end{pmatrix}(\infty) = \lim_{z \to \infty } \frac{az + b}{cz + d} = \frac{a}{c} \]
%
The kernel of the representation is
%
\[ K = \left\{ \begin{pmatrix} a & 0 \\ 0 & a \end{pmatrix} : a \in \mathbf{C} - \{ 0 \} \right\} \]
%
Indeed, if
%
\[ M = \begin{pmatrix} a & b \\ c & d \end{pmatrix} \in K \]
%
then $b = 0$, for $M(0) = b/d = 0$. Similarily, $M(\infty) = a/c = \infty$, so $c = 0$. But then the transformation is of the form $z \mapsto (a/d) z$, so $a = d$. We define the projective linear group
%
\[ PGL_2(\mathbf{C}) = GL_2(\mathbf{C})/K \]
%
which now acts faithfully on $\mathbf{C}^\infty$.

Modular forms arise from a study of the actions of certain M\"{o}bius transformations on the hyperbolic plane. We shall take the upper half of the complex plane, denoted $\mathbf{H}$, as our model, whose `straight lines' consist of circular arcs passing through the origin. The biholomorphisms of $\mathbf{H}$ are exactly the M\"{o}bius transformations obtained from matrices with determinant one, and can therefore be described as the image of $SL_2(\mathbf{R})$ in $PGL_2(\mathbf{C})$, denoted $PSL_2(\mathbf{R})$ and called the projective special linear group. The fact that the linear groups maps the upper half plane to itself follows because
%
\[ \Im \left( \frac{az + b}{cz + d} \right) = \frac{1}{2i} \frac{(az + b)(c\overline{z} + d) - (a\overline{z} + b)(cz + d)}{|cz + d|^2} = \frac{\Im(z)}{|cz+d|^2} \]
%
We shall, perhaps by a little notational abuse, denote an element of $PSL_2(\mathbf{R})$ as an ordinary matrix, even though elements of the group are really cosets. Note that the kernel of $SL_2(\mathbf{R})$ under the M\"{o}bius transformation representation is only $\{ \pm 1 \}$, ome of you may have had occasion to run into mathematicians and to wonder therefore how they got that way, and here, in partial explanation perhaps, is the story of the great Russian mathematician Nicolai Ivanovich Lobachevsky.*
so the quotient is discrete.

We define the modular group $\Gamma = PSL_2(\mathbf{Z})$ to be the discrete counterpart to the projective special linear group. Before we define Modular forms, let us understand the struture of $\Gamma$ by understanding one of its `fundamental region' (by this, we mean a subset $D$ of $\mathbf{H}$ which possesses a point in each orbit induces by $\Gamma$, except perhaps at the boundary -- this is the set that `fully represents' the group action). Perhaps the most `fundamental' fundamental region is the set
%
\[ D = \{ z \in \mathbf{H} : |z| \geq 1, |\Re(z)| \leq 1/2 \} \]
%
\begin{center}
\includegraphics[scale=0.5]{fundamentaldomain.png}
\end{center}
%
Let us verify that it is a fundamental region of $\Gamma$. First, consider two of the most important elements of $\Gamma$
%
\[ S = \begin{pmatrix} 0 & -1 \\ +1 & 0 \end{pmatrix}\ \ \ \ \ \ \ \ \ \ T = \begin{pmatrix} 1 & 1 \\ 0 & 1 \end{pmatrix} \]
%
which induce the maps $z \mapsto -1/z$ and $z \mapsto z + 1$ on $\mathbf{H}$.

\begin{theorem}
    Let $D$ be the fundamental region described above.
    %
    \begin{enumerate}
        \item For any $z \in \mathbf{H}$, there is $\gamma \in \Gamma$ for which $\gamma(z) \in D$.
        \item If $\gamma(z) = w$, and $z,w \in D$, then $z$ and $w$ both occur on the boundary of $D$ and are obtained from each other by reflection in the $y$ axis.
        \item For each $z$, the stabilizers $\Gamma_z$ are trivial except that
        %
        \begin{align*}
            \Gamma_i &= \{ 1, S \}\\
            \Gamma_{e^{\pi i/3}} &= \{ 1, ST, (ST)^2 \}\\
            \Gamma_{e^{2 \pi i/3}} &= \{ 1, TS, (TS)^2 \}
        \end{align*}
    \end{enumerate}
\end{theorem}
\begin{proof}
    Let $G = \langle S,T \rangle$. If $D$ really was a fundamental region, then we could identify the point in $D$ corresponding to each orbit $\Gamma x$ by taking the point in the strip $|\Re(z)| < 1$ such that $|\Im(\Gamma x)|$ is maximized. Since
    %
    \[ \Im(\gamma z) = \frac{\Im(z)}{|cz+d|^2} \]
    %
    Maximizing $\Im(\gamma z)$ happens by minimizing $|cz + d|$. For each $M$, there are only finitely many matrices with $|cz + d| < M$. We have
    %
    \[ |cz + d| \geq \Im(z) |c|\ \ \ \ \ \ \ \ \ \ |cz + d| \geq |c \Re(z) + d| \geq |d| - |c \Re(z)| \]
    %
    There are only finitely many integers $c$ for which $\Im(z) |c| < M$, and therefore only finitely many values of $d$ for which $|d| - |c \Re(z)| < M$. Let $\gamma$ maximize the value of $\Im(z)$. Without loss of generality, we may assume $|\Re(\gamma z)| < 1/2$, for otherwise we may apply $T$ to transport $\gamma z$ to this strip, without affecting the imaginary part. If $|\gamma z| < 1$, then
    %
    \[ \Im(S \gamma z) = \frac{\Im(\gamma z)}{|\gamma z|} > \Im(z) \]
    %
    contradicting the maximality of $\gamma$, so $\gamma z \in D$, and we have proved (i).

    Now suppose that for $z,w \in D$,
    %
    \[ \gamma z = \frac{az + b}{cz + d} = w \]
    %
    We may assume $\Im(w) \geq \Im(z)$. Since
    %
    \[ \Im(w) = \frac{\Im(z)}{|cz + d|^2} \]
    %
    $|cz + d| \leq 1$. Since $z \in D$, $\Im(z) \geq \sqrt{1/2}$, so $|c| \leq \sqrt{2} < 2$, and we can split the proof into two cases ($c = \pm 1$ or $c = 0$):

    \begin{itemize}
        \item First assume $c = 0$. Then
        %
        \[ \gamma = \begin{pmatrix} a & b \\ 0 & a^{-1} \end{pmatrix} \]
        %
        since $a$ and $a^{-1}$ are integers, we must have $a = \pm 1$, and we might as well let $a = 1$, since $\gamma = - \gamma$ in $PSL_2(\mathbf{R})$. Then
        %
        \[ w = z + b \]
        %
        Now $|b| = |\Re(z-w)| \leq 1$, so either $b = 0$, and $w = z$, $b = 1$, and $w = z + 1$, or $b = -1$, and $w = z - 1$.

        \item Assume $c = 1$ (if $c = -1$, replace $\gamma$ with $-\gamma$). We must have
        %
        \[ |z + d|^2 = \Im(z)^2 + (\Re(z) + d)^2 \leq 1 \]
        %
        One possibility is that $d = 0$, so that $z \in \partial D \cap S^1$. Because
        %
        \[ \gamma = \begin{pmatrix} a & -1 \\ 1 & 0 \end{pmatrix} \]
        %
        $w = a - 1/z$. Now $-1/z \in D$ is also on the unit circle, so either $a = 0$, and $w = -1/z$, which is a just a reflect in the $y$ axis, $a = 1$, $z = e^{i\pi/3}$, $w = z$, and $\gamma = ST$, or $a = -1$, $z = e^{2i\pi/3}$, $w = z$, and $\gamma = S^{-1}T = (TS)^2$.

        If $d = 1$, then
        %
        \[ \gamma = \begin{pmatrix} a & a-1 \\ 1 & 1 \end{pmatrix} \]
        %
        So
        %
        \[ w = \frac{az + (a-1)}{z + 1} = a - \frac{1}{z+1} \]
        %
        Now $|z+1| \geq 1$ if $z \in D$, so $|\frac{1}{z+1}| \leq 1$, and $a - \frac{1}{z+1}$ is in $D$ if and only if $a = 0$, $z = e^{2\pi i/3}$, $w = z$, and $\gamma = TS$. If $a = 1$, $- \frac{1}{z+1} = e^{2\pi i/3}$, so $z = e^{2 \pi i/3}$, and $w = e^{\pi i/3}$. If $a = -1$, $- \frac{1}{z+1} = e^{\pi i/3}$, which implies $z = e^{2 \pi i/3} - 1 \not \in D$.

        Finally, let $d = -1$. Then
        %
        \[ \gamma = \begin{pmatrix} a & -(1 + a) \\ 1 & -1 \end{pmatrix} \]
        %
        so
        %
        \[ w = \frac{az - (1+a)}{z-1} = a - \frac{1}{z-1} \]
        %
        As before, $|z-1| \geq 1$ if $z \in D$, so $|\frac{1}{z-1}| \leq 1$. Either $a = 0$, in which case $|z-1| = 1$, so $z = w = e^{\pi i/3}$, $\gamma = TS^{-1} = (ST)^2$, or $a = 1$, in which case $\frac{-1}{z-1} = e^{2\pi i/3}$, implying $z = e^{2\pi i/3}$, $w = e^{\pi i/3}$, or $a = -1$, in which case $\frac{-1}{z-1} = e^{\pi i/3}$, which would imply $z = e^{2\pi i/3} - 1 \not \in D$.
    \end{itemize}
    %
    We have addressed all cases, which shows that (ii) and (iii) hold.
\end{proof}

\begin{corollary}
    $PSL_2(\mathbf{R})$ is generated by $S$ and $T$.
\end{corollary}
\begin{proof}
    Given $\gamma \in PSL_2(\mathbf{R})$ pick $z$ in the interior of $D$, and let $w = \gamma z$. We have verified in the above proof that there is $\phi \in G$ for which $\phi w \in D$, and since the orbit is unique on the interior of $D$, $\phi w = z$. But then $\phi \gamma \in \Gamma_z = \{ I \}$, so $\phi = \gamma^{-1}$, and so $\gamma \in G$.
\end{proof}

\end{document}