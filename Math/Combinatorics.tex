\documentclass[12pt]{report}

\usepackage{amsmath}
\usepackage{amssymb}
\usepackage{amsthm}
\usepackage{amsopn}
\usepackage{kpfonts}
\usepackage{graphicx}
\usepackage{kbordermatrix}
\usepackage{tikz}
\usetikzlibrary{arrows, petri, topaths}%
\usepackage{tkz-berge}
\usepackage{multicol}

\usepackage{framed}
\usepackage{mathtools}
\usepackage{float}
\usepackage{subfig}
% \usepackage{cmbright}

\theoremstyle{plain}
\newtheorem{theorem}{Theorem}[chapter]
\newtheorem{lemma}[theorem]{Lemma}
\newtheorem{corollary}[theorem]{Corollary}
\newtheorem{prop}[theorem]{Proposition}
\newtheorem{exercise}{Exercise}[chapter]

\newtheorem*{example}{Example}
\newtheorem*{proof*}{Proof}

\theoremstyle{definition}
\newtheorem*{defi}{Definition}
\newenvironment{definition}
    {\begin{samepage}\begin{framed}\begin{defi}}
    {\end{defi}\end{framed}\end{samepage}}





\usepackage{hyperref} 
\hypersetup{
    colorlinks = true,
    linkcolor = black,
}

\makeatletter
\renewcommand*\env@matrix[1][*\c@MaxMatrixCols c]{%
  \hskip -\arraycolsep
  \let\@ifnextchar\new@ifnextchar
  \array{#1}}
\makeatother

\renewcommand*\contentsname{\hfill Table Of Contents \hfill}

\newcommand{\optionalsection}[1]{\section[* #1]{(Important) #1}}
\newcommand{\deriv}[3]{\left. \frac{\partial #1}{\partial #2} \right|_{#3}}

\DeclareMathOperator{\Dom}{Dom}

\title{Number Theory}
\author{Jacob Denson}

\begin{document}

\pagenumbering{gobble}

\maketitle

\tableofcontents

\pagenumbering{arabic}

\chapter{Generating Functions}

\begin{example}
    Suppose we are working in a country with only a one, a two, and a three penny coin. Given an integer $n$, let $r(n)$ denote the number of ways that a person can be paid $n$ pennies using these three coins. Since this is a question about the additivity of numbers, we can likely understand it using generating functions. Formally,
    %
    \[ r(n) = \# \{ (a,b,c): a + 2b + 3c = n \} \]
    %
    We note
    %
    \[ \left( \sum_{a = 0}^\infty z^a \right) \left( \sum_{b = 0}^\infty z^{2b} \right) \left( \sum_{c = 0}^\infty z^{3c} \right) = \sum_{a,b,c} z^{a + 2b + 3c} = \sum_{n = 0}^\infty r(n) z^n \]
    %
    Thus, for $|z| < 1$,
    %
    \[ \sum_{n = 0}^\infty r(n) z^n = \frac{1}{(1 - z)(1 - z^2)(1 - z^3)} \]
    %
    We can now perform a partial fraction decomposition, writing
    %
    \[ \frac{1}{(1-z)(1 - z^2)(1 - z^3)} = \frac{1}{(1-z)^3(1+z)(\omega - z)(\omega + z)} \]
    %
    where $\omega = e(1/3)$ is a primitive third root of unity. Some intense linear algebra shows this is equal to
    %
    \[ \frac{z + 2}{9(z^2 + z + 1)} + \frac{17 z^2 - 52 z + 47}{72 (1 - z)^3} + \frac{1}{8(1 + z)} \]
    %
    which can be further decomposed into
    %
    \begin{align*}
        - &\frac{\omega^2 + 3 \omega + 2}{9(1 - z/\omega)} + \frac{\omega^2 - \omega + 2}{9(1 - z/\omega^2)}\\
        &+ \frac{1}{6(1 - z)^3} + \frac{1}{4(1 - z)^2} + \frac{17}{72(1 - z)} + \frac{1}{8(1 + z)}
    \end{align*}
    %
    where $\omega = e(1/3)$. Taking power series and summing up, we find
    %
    \begin{align*}
        r(n) &= - \frac{\omega^2 + 3\omega + 2}{9 \omega^n} + \frac{\omega^2 - \omega + 2}{9 \omega^{2n}} + \frac{(n + 1)(n + 2)}{12} + \frac{n+1}{4} + \frac{17}{72} + \frac{(-1)^n}{8}\\
        &= \frac{6n^2 + 36n + 47 + 9(-1)^n}{72} + \begin{cases} 0 & n \equiv 0 (\text{mod}\ 3) \\ -2/9 & n \equiv 1\ \text{or}\ 2 (\text{mod}\ 3) \end{cases}\\
        &= \frac{(n + 3)^2}{12} + \frac{9 (-1)^n - 7}{72} + \begin{cases} 0 & n \equiv 0 (\text{mod}\ 3) \\ -16/72 & n \equiv 1\ \text{or}\ 2 (\text{mod}\ 3) \end{cases}    \end{align*}
        %
        We know $r(n)$ is an integer, and since
        %
        \[ \frac{9 + 7 + 16}{72} = \frac{32}{72} < \frac{1}{2} \]
        %
        So $r(n)$ is the closest integer to $(n + 3)^2/12$.
\end{example}

\chapter{Additive Combinatorics}

Given a subset $A$ of an abelian group, we say $A$ is {\bf sum free} if $A + A$ is disjoint from $A$.

\begin{theorem}
    If $A$ is an arbitrary finite subset of positive natural numbers, then $A$ contains a sum-free subset of size greater than $|A|/3$.
\end{theorem}
\begin{proof}
    The idea of this proof rests on two observations. If $B \subset [1,N]$, and $p > 2N$, then $B + p \mathbf{Z}$ is sumfree in $\mathbf{Z}_p$ if and only if $B$ is sumfree. Thus we can turn out problem into a problem modulo $p$. Next, we notice that if $f$ is an automorphism, then a subset $B$ of an abelian group is sumfree if and only if $f(B)$ is sumfree. The presense of many automorphisms of $\mathbf{Z}_p$ (one for each natural number between $1$ and $p-1$) enables us to exploit randomness to construct a sumfree subset in $A$. If $X \subset \mathbf{Z}_p$ is sumfree, and does {\it not} contain zero, we consider the sets $X,2X, \dots, (p-1)X$, which are all sumfree. For every $a \in X$, and nonzero $b \in \mathbf{Z}_p$, there is a unique $c \in \{ 1, \dots, p-1 \}$ such that $ca = b$. Thus every nonzero $b \in \mathbf{Z}_p$ occurs in $|X|/(p-1)$ of the sets $X,\dots, (p-1)X$. Thus means if we choose a nonzero $x \in \mathbf{Z}_p$ uniformly at random, then
    %
    \[ \mathbf{E} |(A + \mathbf{Z}_p) \cap xX| = \sum_{a \in A + \mathbf{Z}_p} \mathbf{P}(a \in xX) = \frac{|A| |X|}{p-1} \]
    %
    Since $xX$ is sumfree, so too is $(A + \mathbf{Z}_p) \cap xX$, and so lower bounding the expectation gives rise to a large sumfree sert. In $\mathbf{Z}_p$, a good candidate for a sumfree set should be an interval, since an arithmetic progression has a small sumset, and all arithmetic progressions are mapped to an interval by an automorphism. Thus, taking $X = \{ k, \dots, 2k - 1 \}$, where $4k - 2 < p + k$, we get a squarefree set. Thus taking $p$ congruent to two modulo 3, and setting $3k = p + 1$, we find a sumfree set of size
    %
    \[ \frac{k}{p-1} |A| = \frac{p + 1}{3(p-1)} |A| > |A|/3 \]
    %
    which completes the proof.
\end{proof}

A fundamental problem in additive combinatorics is the {\it inverse sumset} problem. If $A + B$ or $A - B$ is small, what can one say about $A$ and $B$? More specifically, if $A + A$ is small, what can one say about $A$? We have $|A| \leq |A + A| \leq [|A|^2 + |A|]/2$, and so we refer to the value $\sigma(A) = |A + A|/|A|$ as the {\bf doubling constant} of the set $A$. We have $1 \leq |A| \leq (|A| + 1)/2$.

\begin{example}
    Geometric progressions have the largest doubling constant possible. If
    %
    \[ A = \{ 1, a, a^2, \dots, a^{N-1} \} \]
    %
    then the sum of any two elements of $A$ is distinct, so $|A + A| = (N^2 + N)/2$, and so $\sigma(A) = (N+1)/2$.
\end{example}

A set $A$ with $\sigma(A)$ maximal among sets of size $N$ is known as a {\bf Sidon set}. This means that all pairwise sums of any two $a_0,a_1 \in A$ are distinct, modulo the trivial equalities $a_0 + a_1 = a_1 + a_0$. This is a `generic' behaviour: If $A$ is a subset of $N$ points chosen uniformly at random frmo $[0,1]$, then $A$ is Sidon with probability one. It is more interesting to characterize when $\sigma(A)$ is small.

\begin{example}
    In the other extreme, the main example of sets with small doubling constant is an arithmetic progression. If $A = b_0 + [0,N-1] a$, then $A + A = 2b_0 + [0,2N-2] a$, which consists of $2N-1$ points, so $\sigma(A) = 2 - 1/N$.
\end{example}

\begin{example}
    If $A \subset B$, and $|A| = \alpha |B|$, then $|A + A| \leq |B + B|$, so
    %
    \[ \sigma(A) \leq \frac{|B+B|}{K|B|} = \sigma(B)/\alpha \]
    %
    Thus if $\sigma(B)$ is small, and $A$ contains a large percentage of $B$, then $\sigma(A)$ is also small. In the other direction, if $|B| = \beta |A|$, then
    %
    \[ |B+B| \leq |A + A| + |A + (B - A)| + |(B-A) + (B-A)| \leq \sigma(A)|A| + (\beta - 1)|A|^2 + \beta^2 |A|^2 \]
    %
    so
    %
    \[ \sigma(B) \leq \sigma(A)/\beta + (\beta + 1 - 1/\beta)|B| \]
    %
    Thus if $\sigma(A)$ is small, and $B$ doesn't contain many more points than $A$, then $\sigma(B)$ is also small.
\end{example}

\begin{example}
    If we consider $N$ and $M$, and a resultant `rank 2' arithmetic progression $A = c + [0,N]a + [0,M]b$, then $\sigma(A) \leq 4$. These sets can look very different from the original arithmetic progressions we were considering.
\end{example}

If $A$ and $B$ are additive sets, and we form the graph $G$

\section{Graph Theoretic Techniques}

\begin{theorem}[Tur\'{a}n]
    Let $G$ be a graph of $n$ vertices. Then $G$ contains an independant set of size at least
    %
    \[ \sum_{v \in G} \frac{1}{\deg(v) + 1} \]
    %
    In particular, if the vertices have degree bounded by $d$, then there is an independant set of size $|G|(d+1)^{-1}$.
\end{theorem}
\begin{proof}
    Let $\pi: V \to \{ 1, \dots, n \}$ be a uniformly randomly chosen bijection. Let $S$ be the set of all vertices $v$ in $V$ such that for any neighbour $w$ of $v$, $\pi(v)$ is larger than $\pi(w)$. Then $S$ is an independant set, and it suffices to show $S$ is large in expectation. We find by the hockey stick identity that
    %
    \begin{align*}
        \mathbf{P}(v \in S) &= \frac{1}{n!} \sum_{m = 1}^n {m-1 \choose \deg(v)} \deg(v)! (n - 1 - \deg(v))!\\
        &= \frac{\deg(v)! (n - 1- \deg(v))!}{n!} {n \choose \deg(v)+1}\\
        &= \frac{1}{\deg(v)+1}
    \end{align*}
    %
    and so
    %
    \[ \mathbf{E}|S| = \sum_{v \in G} \mathbf{P}(v \in S) = \sum_{v \in G} \frac{1}{\deg(v) + 1} \]
    %
    and this gives the required set.
\end{proof}

Given $B \subset A$, we say $B$ is sumfree with respect to $A$ if no element of $A$ is the sum of two distinct elements of $B$. Given $A$, we let $\phi(A)$ denote the largest sumfree subset with respect to $A$. We let $\phi(n)$ be the smallest value of $\phi(A)$ among all sets $A \subset \mathbf{R}$ of size $n$.

\begin{theorem}[Choi]
    If $A$ is any set of $n$ real numbers, there is a set $B \subset A$ of cardinality $\log n - O(1)$ sumfree with respect to $A$. Thus $\phi(n) \geq \log n - O(1)$.
\end{theorem}
\begin{proof}
    Assume first that $A$ is a subset of positive reals. Order $A = \{ a_1 > a_2 > \dots > a_n > 0 \}$. Consider the graph $G$ with vertices $A$, and edges $(a_n, a_m)$ if $a_n + a_m \in A$. By Tur\'{a}n's theorem, since $\deg(a_i) \leq n - i$, we find an independant set $S$ with
    %
    \[ |S| \geq \sum_{i = 1}^n \frac{1}{n - i + 1} = \sum_{i = 1}^n \frac{1}{i} = \log n - O(1) \]
    %
    In general, any set $A$ of $n$ real numbers either contains $n/2 - O(1)$ positive real numbers or $n/2 - O(1)$ negative real numbers, and the theorem then follows in this case.
\end{proof}

The $n/(d+1)$ bound for graphs of bounded degree $d$ cannot be improved for general graphs $G$. However, it is surprising that one can improve the bound by a $\log d$ factor, provided that the resultant graph has no three cycles.

\begin{theorem}
    If $G$ has no three cycles with maximal degree $d$, then $G$ contains an independant set of size $\Omega(n \log d / d)$.
\end{theorem}
\begin{proof}
    Choose a set $I$ uniformly from the set of all independant sets in $G$. For each $v \in V$, define the random variable
    %
    \[ X_v = d |I \cap \{ v \}| + |N(v) \cap I| = \begin{cases} d & v \in I \\ |N(v) \cap I| & v \not \in I \end{cases} \]
    %
    Any vertex can be in the neighbourhood of at most $d$ other vertices, so
    %
    \[ \sum_v X_v = d|I| + \sum_{v \not \in I} |N(v) \cap I| \leq 2d |I| \]
    %
    Taking expectations gives that
    %
    \[ \mathbf{E} |I| \geq \frac{1}{2d} \sum_v \mathbf{E}(X_v) \]
    %
    Thus it suffices to show that $\mathbf{E}(X_v)$ is large for each $v$. TODO: FINISH LATER.
\end{proof}

The Balog-Szemer\'{e}di theorem says that if $E(A,B) \geq K_0 n^2$ and $|A +_G B| \leq K_1 n$, then one can find $A_0 \subset A$ and $B_0 \subset B$ such that $|A_0|, |B_0|$, and $|A_0 + B_0|$ are $\Theta_{K_0,K_1}(n)$. Gower's recently strengthened the theorem to showing the constants in the bound are polynomial in $1/K_0$ and $K_1$. We shall find that this result can be converted into a graph problem.

If $E(A,B) \gtrsim |A|^{3/2} |B|^{3/2}$, then there is $A_0 \subset A$ and $B_0 \subset B$ with $|A_0| \sim |A|$, $|B_0| \sim |B|$, and $|A_0 + B_0| \lesssim |A_0|^{1/2} |B_0|^{1/2}$. In particular, if $A$ and $B$ have $n$ elements, and $E(A,B) \gtrsim n^3$, then there is $A_0 \subset A$ and $B_0 \subset B$ with $|A_0|, |B_0| \sim n$, and $|A_0 + B_0| \lesssim n$. Can we generalize this theorem to more general operations than addition, i.e. linear transformations of the coordinates?

\begin{lemma}
    If $G$ is a bipartite graph with $|E| \geq |A||B|/K$ for some $K \geq 1$, then for any $0 < \varepsilon < 1$, there is $A_0 \subset A$ such that $|A_0| \geq |A|/K \sqrt{2}$, and such that $1 - \varepsilon$ of the pairs of vertices in $A_0$ are connected by $\varepsilon |B|/ 2K^2$ paths of length 2 in $G$.
\end{lemma}
\begin{proof}
    By decreasing $K$, we may assume that $|E| = |A||B|/K$. Now
    %
    \[ \frac{\mathbf{E}_b |N(b)|}{|A|} = \frac{\mathbf{E}_a |N(a)|}{|B|} = \frac{|E|}{|A||B|} = \frac{1}{K} \]
    %
    and
    %
    \[ \frac{\mathbf{E}_b |N(b)|^2}{|A|^2}  = \mathbf{E}_{a,a'} \frac{|N(a) \cap N(a')|}{|B|} \]
\end{proof}

Let $A_1, \dots, A_k$ be additive sets with cardinality $n$, and consider a $k$ uniform $k$-partite hypergraph $H$ on $A_1, \dots, A_k$. If $H$ has $\Omega(n^k)$ edges and $|\bigoplus^H A_i| = O(n)$, then we can find $A_i' \subset A_i$ with $|A_i'| = \Omega(n)$ and $|A_1' + \dots + A_k'| = \Omega(n)$. If we let $H$ be 

\end{document}