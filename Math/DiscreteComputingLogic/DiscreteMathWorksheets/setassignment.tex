\documentclass{exam}

\usepackage[margin=1in]{geometry}
\usepackage{kpfonts}
\usepackage{amsmath}
\usepackage{mathabx}
\usepackage{mathtools}
\usepackage{tikz,siunitx}
\usepackage{tikz-cd}
\usetikzlibrary{matrix,positioning,calc}
\usetikzlibrary{decorations.pathmorphing}


\DeclareMathOperator{\QQ}{\mathbf{Q}}
\DeclareMathOperator{\ZZ}{\mathbf{Z}}
\DeclareMathOperator{\RR}{\mathbf{R}}
\DeclareMathOperator{\HH}{\mathbf{H}}
\DeclareMathOperator{\CC}{\mathbf{C}}
\DeclareMathOperator{\AB}{\mathbf{A}}
\DeclareMathOperator{\PP}{\mathbf{P}}
\DeclareMathOperator{\MM}{\mathbf{M}}
\DeclareMathOperator{\VV}{\mathbf{V}}
\DeclareMathOperator{\TT}{\mathbf{T}}
\DeclareMathOperator{\LL}{\mathcal{L}}
\DeclareMathOperator{\EE}{\mathbf{E}}
\DeclareMathOperator{\NN}{\mathbf{N}}
\DeclareMathOperator{\DQ}{\mathcal{Q}}
\DeclareMathOperator{\IA}{\mathfrak{a}}
\DeclareMathOperator{\IB}{\mathfrak{b}}
\DeclareMathOperator{\IC}{\mathfrak{c}}
\DeclareMathOperator{\IP}{\mathfrak{p}}
\DeclareMathOperator{\IQ}{\mathfrak{q}}
\DeclareMathOperator{\IM}{\mathfrak{m}}
\DeclareMathOperator{\IN}{\mathfrak{n}}
\DeclareMathOperator{\IK}{\mathfrak{k}}
\DeclareMathOperator{\ord}{\text{ord}}
\DeclareMathOperator{\Ker}{\textsf{Ker}}
\DeclareMathOperator{\Coker}{\textsf{Coker}}
\DeclareMathOperator{\emphcoker}{\emph{coker}}
\DeclareMathOperator{\pp}{\partial}
\DeclareMathOperator{\tr}{\text{tr}}

\title{MATH 240 - INDUCTION}
\author{Jacob Denson}

\begin{document}

\pagestyle{headandfoot}
\firstpageheader{Math/CS 240}{Worksheet 6: INDUCTION}{}
\runningfooter{}{}{}
\firstpageheadrule

\begin{questions}

\maketitle

\question Consider the following sequence of numbers $\{ a_n : n \geq 0 \}$, where $a_0 = 1$, and successive elements of the sequence are defined by the recurrence relation $a_{n+1} = a_n + 5^n$. Prove using induction that for all $n \geq 0$,
%
\[ a_n = \frac{5^{n+1} - 1}{4}. \]
%
Show that if $a_0 = 0$, and $a_{n+1} = a_n + n \cdot 5^n$, then
%
\[ a_n = \frac{1 + n \cdot 5^{n+1} - (n+1) 5^n}{16} \]
\vspace{8em}

\question Let ${n \choose k}$ denote the total number of $k$ element subsets of a set with $n$ elements. For instance, $\{ 1, 2, 3, 4 \}$, a set with 4 elements, has 6 subsets of size 2, namely ${1,2}$, ${1,3}$, ${1,4}$, ${2,3}$, ${2,4}$, and ${2,5}$, so ${4 \choose 2} = 6$. Prove using induction that for $n \geq k$,
%
\[ {n \choose k} = \frac{n!}{k!(n-k)!}. \] 
%
Hint: Find a recurrence relation, and apply induction. What should the base case be?
\vspace{8em}

\question Consider the following inductive proof that all horses on earth are the same color. To do this, we proceed inductively, proving that any finite set of horses has the same color.
\begin{itemize}
	\item Base Case: If we consider a set consisting of a single horse, then certainly all horses in that set have the same color.
	\item Inductive Case: Consider a collection of n horses, with $n > 1$, and by the principle of induction, assume that any collection of k horses has the same color for any $k < n$. Let us label the present collection of horses 1 through n. By induction, all horses labelled 1 to $n-1$ have the same color. Thus horse 1 has the same color as any middle horse. Applying induction again, all horses labelled 2 to n have the same color. Thus horse n has the same color as any middle horse. But this means that horse 1 has the same color as horse n, because they both share the same color as the middle horses. Thus all horses in the present collection have the same color.
\end{itemize}
%
We have a proof of a base case and an inductive case, so we have shown that all finite sets of horses share the same color, and since there are only finitely many horses on earth, all such horses must have the same color. Experience tells us that it is \emph{not} true that all horses are the same color, if we are to believe in the principle of induction, we must conclude that argument is using the principle of induction correctly.  What is the problem with this argument?

\question What is the maximum number of pieces of cake one can make by making n cuts in a straight line? Hint: Start by reasoning things out for small n to gain intuition. Prove that if $L_n$ is the maximum number of cuts one can make with n lines, then $L_n = L_{n-1} + n$, then use induction to calculate a closed formula for $L_n$.
\vspace{10em}

\question The Tower of Hanoi is a mathematical puzzle involving three poles (a left pole, a centre pole, and a right pole), with n disks stacked in decreasing size on the left pole, with the largest disk on the bottom, and the smallest disk on the top. The objective of the puzzle is to move all n disks onto the right pole, moving one disk from the top of a rod at a time, in such a way that no disk is ever stacked on top of a disk that is smaller than itself. Prove that the minimum number of moves required to complete the puzzle is $2^n - 1$ (Hint: Use induction to come up with a solution that takes $2^n - 1$ steps, using the solution to the puzzle on $n-1$ disks as a step in the solution to the n disk puzzle. To show this solution is optimal, show that if $T_n$ is the minimum number of moves required to solve the puzzle, then $T_n \geq 2T_{n-1} + 1$, from which it follows by another induction that $T_n >= 2^n - 1$).

\begin{center}
\includegraphics[scale=0.3]{Tower_of_Hanoi.jpeg}
\end{center}

Bonus Problem: Suppose I can only move disks on the left rod to the centre rod, I can only move disks on the centre rod to disks on the right rod, and disks from the right rod only to disks on the left rod. What is the minimum number of moves under these additional conditions (you should be able to reduce the calculation to a more complicated linear recurrence relation).
\vspace{10em}

\end{questions}

\end{document}