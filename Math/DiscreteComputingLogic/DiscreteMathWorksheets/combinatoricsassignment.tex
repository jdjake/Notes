\documentclass{exam}

\usepackage[margin=1in]{geometry}
\usepackage{kpfonts}
\usepackage{amsmath}
\usepackage{mathabx}
\usepackage{mathtools}
\usepackage{tikz,siunitx}
\usepackage{tikz-cd}
\usetikzlibrary{matrix,positioning,calc}
\usetikzlibrary{decorations.pathmorphing}


\DeclareMathOperator{\QQ}{\mathbf{Q}}
\DeclareMathOperator{\ZZ}{\mathbf{Z}}
\DeclareMathOperator{\RR}{\mathbf{R}}
\DeclareMathOperator{\HH}{\mathbf{H}}
\DeclareMathOperator{\CC}{\mathbf{C}}
\DeclareMathOperator{\AB}{\mathbf{A}}
\DeclareMathOperator{\PP}{\mathbf{P}}
\DeclareMathOperator{\MM}{\mathbf{M}}
\DeclareMathOperator{\VV}{\mathbf{V}}
\DeclareMathOperator{\TT}{\mathbf{T}}
\DeclareMathOperator{\LL}{\mathcal{L}}
\DeclareMathOperator{\EE}{\mathbf{E}}
\DeclareMathOperator{\NN}{\mathbf{N}}
\DeclareMathOperator{\DQ}{\mathcal{Q}}
\DeclareMathOperator{\IA}{\mathfrak{a}}
\DeclareMathOperator{\IB}{\mathfrak{b}}
\DeclareMathOperator{\IC}{\mathfrak{c}}
\DeclareMathOperator{\IP}{\mathfrak{p}}
\DeclareMathOperator{\IQ}{\mathfrak{q}}
\DeclareMathOperator{\IM}{\mathfrak{m}}
\DeclareMathOperator{\IN}{\mathfrak{n}}
\DeclareMathOperator{\IK}{\mathfrak{k}}
\DeclareMathOperator{\ord}{\text{ord}}
\DeclareMathOperator{\Ker}{\textsf{Ker}}
\DeclareMathOperator{\Coker}{\textsf{Coker}}
\DeclareMathOperator{\emphcoker}{\emph{coker}}
\DeclareMathOperator{\pp}{\partial}
\DeclareMathOperator{\tr}{\text{tr}}

\title{MATH 240 - INDUCTION}
\author{Jacob Denson}

\begin{document}

\pagestyle{headandfoot}
\firstpageheader{Math/CS 240}{Worksheet: COMBINATORICS}{}
\runningfooter{}{}{}
\firstpageheadrule

\begin{questions}

\question Let
%
\[ {n \choose m} \]
%
denote the number of subsets of $\{ 1, \dots, n \}$ which have cardinality $m$. These are the \emph{binomial coefficients}. Thus ${3 \choose 2} = 3$, and ${3 \choose 0} = 1$.

\begin{parts}
	\part Argue that for any $n$ and $m$, with $n \geq 1$ and $m \leq n$,
	%
	\[ {n \choose m} = {n-1 \choose m} + {n-1 \choose m-1}. \]

	\part One can define $S = \{ (n,m) : n \geq m \}$ structurally by the following iterative method:
	%
	\begin{itemize}
		\item $(0,0)$ is in $S = \NN^2$.
		\item If $(n,m)$ is in $S$, then $(n+1,m)$ is in $S$.
		\item If $(n,m)$ is in $S$, and $n \leq m+1$, then $(n,m+1)$ is in $S$.
	\end{itemize}
	%
	Prove using structural induction that for all $(n,m) \in S$,
	%
	\[ {n \choose m} = \frac{n!}{m!(n-m)!}, \]
	%
	where we define $0! = 1$, and $(n+1)! = (n+1) \cdot n!$.

	\part Argue by induction on $n$ that for all $n \geq 0$, and any $x,y \in \RR$,
	%
	\[ (x + y)^n = \sum_{m = 0}^n {n \choose m} x^m y^{n-m}. \]

	\part Let
	%
	\[ {n \choose m, k} \]
	%
	denote the number of pairs of disjoint subsets $S_1, S_2 \subset \{ 1, \dots, n \}$ with $\#(S_1) = m$ and $\#(S_2) = k$. Can you find (and prove) analogous properties of the statements above for this function?
\end{parts}

\question How many numbers below 2022 are divisible by two, three, or five.

\question How many functions from $\{ 1, 2, 3, 4, 5 \}$ to $\{ A ,B, C, D \}$ are onto?

\question Recall that a graph $H$ is \emph{bipartite} if one can write it's vertex set as a disjoint union $V_1 \cup V_2$, where each edge in $H$ connects an element of $V_1$ to an element of $V_2$. Let $G = (V,E)$ be an arbitrary graph with $\#(E) = K$ edges, which is not necessarily bipartite. In this problem, we will prove $G$ has a Bipartite subgraph containing at least $K/2$ edges. One element of the proof will be a stronger version of the pidgeonhole principle:
%
\begin{enumerate}
	\item[ ] (Strong Pidgeonhole Principle) Given $N$ pidgeons placed in $M$ holes, there exists a hole containing at least $\lceil N / M \rceil$ pidgeons.
\end{enumerate}
\begin{parts}
	\part For any edge $e = (v_1,v_2) \in E$, let
	%
	\[ T_e = \{ W \subset V: \text{$v_1$ or $v_2$ is in $W$, but not both} \}. \]
	%
	Argue that $\#(T_e) = 2^{\#(V) - 1}$.

	\part Using the calculation above, show that there exists $W \subset V$ such that the number of edges between $W$ and $W^c$ is at least $K/2$ (Hint: The holes are the subsets of $V$, the pidgeons are the edges).

	\part Conclude that there exists a subgraph of $G$, which is bipartite, and contains at least $K/2$ edges.
\end{parts}

\end{questions}

\end{document}