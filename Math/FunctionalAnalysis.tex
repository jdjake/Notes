\documentclass[12pt]{report}

\usepackage{amsmath}
\usepackage{amssymb}
\usepackage{amsthm}
\usepackage{amsopn}
\usepackage{kpfonts}
\usepackage{graphicx}
\usepackage{kbordermatrix}
\usepackage{tikz}
\usetikzlibrary{arrows, petri, topaths}%
\usepackage{tkz-berge}
\usepackage{multicol}

\usepackage{framed}
\usepackage{mathtools}
\usepackage{float}
\usepackage{subfig}
% \usepackage{cmbright}

\theoremstyle{plain}
\newtheorem{theorem}{Theorem}[chapter]
\newtheorem{lemma}[theorem]{Lemma}
\newtheorem{corollary}[theorem]{Corollary}
\newtheorem{prop}[theorem]{Proposition}
\newtheorem{exercise}{Exercise}[chapter]

\newtheorem*{example}{Example}
\newtheorem*{proof*}{Proof}

\theoremstyle{definition}
\newtheorem*{defi}{Definition}
\newenvironment{definition}
    {\begin{samepage}\begin{framed}\begin{defi}}
    {\end{defi}\end{framed}\end{samepage}}





\usepackage{hyperref} 
\hypersetup{
    colorlinks = true,
    linkcolor = black,
}

\makeatletter
\renewcommand*\env@matrix[1][*\c@MaxMatrixCols c]{%
  \hskip -\arraycolsep
  \let\@ifnextchar\new@ifnextchar
  \array{#1}}
\makeatother

\renewcommand*\contentsname{\hfill Table Of Contents \hfill}

\newcommand{\optionalsection}[1]{\section[* #1]{(Important) #1}}
\newcommand{\deriv}[3]{\left. \frac{\partial #1}{\partial #2} \right|_{#3}}

\title{Functional Analysis}
\author{Jacob Denson}

\begin{document}

\pagenumbering{gobble}
\maketitle
\tableofcontents
\pagenumbering{arabic}

\chapter{Introduction}

Functional analysis is the interlace of algebra and analysis, in which we study algebraic structures endowed with topological properties. The utility of this approach is established by the rapid growth of applications over the past century, be it in quantum mechanics, statistics, or Computing science. Most of the time, a mathematician is not concerned with a single object, like a function, a random variable, or a measure, but instead considers large classes of these objects. One technique for handling these objects is to add algebraic structure which elaborates on the natural relation between these objects, and functional analysis provides the tools for handling this augmentation. Let's consider some examples of how this approach is used.

\begin{example}
    We rarely analyze a measurable function $f$ in isolation. Instead, we prove theorems about a class of measurable functions defined on the same measurable space. If $f$ and $g$ are measurable, then we may consider their addition $f + g$, their multiplication $fg$, and scaling $\lambda f$ (for $\lambda \in \mathbf{R}$), which are all measurable. Thus the space of measurable functions on a set has some predefined algebraic structure. Similarily, we may consider $C[0,1]$, the space of all continuous, real-valued functions defined on the unit interval in the same way.
\end{example}

The main reason why the subject is called functional analysis is because most of the time we shall be analyzing functions from one space to another, which naturally have an additive and multiplicative structure obtained from the space these functions are defined on.

\begin{definition}
    A {\bf topological vector space} is a vector space over a field (here assumed to be $\mathbf{R}$ or $\mathbf{C}$), endowed with  topology which makes the operations of addition and multiplication continuous.
\end{definition}

Some immediate corollaries of the definition include that

\begin{prop}
    The translation $U + v$ of any open set $U$ by a vector $v$ is open.
\end{prop}
\begin{prop}
    When $v_\alpha \to v$, $w_\alpha \to w$, and $\lambda_\alpha \to \lambda$, $\lambda_\alpha (v_\alpha + w_\alpha) = \lambda ( v + w)$.
\end{prop}
\begin{prop}
    A translated local base at the origin by a vector $v$ is a local base at $v$.
\end{prop}
%
Proposition (1.2) and (1.3) give different characterizations of topological vector spaces. We can define topological vector spaces either in terms of neighbourhood bases at the origin, or in terms of convergent nets obeying the rule denoted in (1.2). A neighbourhood containing the origin will be hereafter known as a {\bf 0-neighbourhood}.
%
\begin{lemma}
    Every 0-neighbourhood $U$ contains a 0-neighbourhood $W$ for which $W + W \subset U$.
\end{lemma}
\begin{proof}
    Since addition is continuous, and $U$ is a 0-neighbourhood, there are neighbourhoods $W$ and $V$ for which $W + V \subset U$. Our problem is solved by picking $W \cap V$ as our neighbourhood.
\end{proof}

\begin{definition}
    A set $B$ is {\bf balanced} if $\lambda B \subset B$ for $|\lambda| < 1$.
\end{definition}

\begin{lemma}
    Every 0-neighbourhood can be shrunk to a balanced neighbourhood.
\end{lemma}
\begin{proof}
    Let $U$ be a neighbourhood of zero. Then there is a scalar neighbourhood $\Lambda$ of zero, and a neighbourhood $V$ of any vector $V$, for which $\Lambda V \subset U$. We may choose $\Lambda$ so it is balanced, and then $\Lambda V$ is balanced.
\end{proof}

\begin{prop}
    If $K$ and $C$ are disjoint subsets of a T1 vector space $V$, where $K$ is compact and $C$ is closed, then 0 has a neighbourhood $V$ such that $K + V$ and $C + V$ are disjoint.
\end{prop}
\begin{proof}
    Fix some point $k \in K$. It suffices to show that there is a neighbourhood $U_k$ for which $k + U_k$ is disjoint from $C + U_k$. If this is true, (1.4) tells us we may pick a subset $W_k$ of $U_k$ for which $W_k + W_k \subset U_k$. Then we may choose a finite subcover $k_1 + W_{k_1}, \dots, k_n + W_{k_n}$ of $K$. If we let $V = \bigcap_{i = 1}^n W_{k_i}$, then we find $K + V$ is disjoint from $C + V$.

    To find this neighbourhood of $k$, we know that we may first pick a neighbourhood $U$ containing $C$, disjoint from $k$. Without loss of generality, by translation, we may assume $U$ is a neighbourhood of 0. Then we may choose $W$ for which $W + W \subset U$ and $-W = W$. It then follows that $k + W$ is disjoint from $W + C$, since if $k + w = w' + c$, $k + w - w' = c$, the left side is contained within $U$.
\end{proof}

\begin{corollary}
    A T1 vector space is Hausdorff.
\end{corollary}

\begin{corollary}
    Every set in a neighbourhood base contains the closure of another neighbourhood.
\end{corollary}
\begin{proof}
    Let 
\end{proof}

\begin{corollary}
    The closure of a set $A$ is the intersection of all $A + V$, where $V$ is a 0-neighbourhood.
\end{corollary}
\begin{proof}

\end{proof}

\begin{prop}
    A locally bounded space is first-countable.
\end{prop}
\begin{proof}
    
\end{proof}

\begin{definition}
Vector spaces should be familiar, but we specify some basic properties which may have been missed in a basic course.
\begin{enumerate}
    \item A set $C$ is {\bf convex} if $tC + (1 - t)C \subset C$ for any $t \in [0,1]$.
    \item A set $B$ is {\bf balanced} if $\lambda B \subset B$ for $| \lambda | \leq 1$.
    \item a set $B$ is {\bf (Von-Neumann) bounded} if, for any neighbourhood $E$ of 0, there is a scalar $\lambda > 0$ such that $B \subset \gamma E$ for every $\gamma > \lambda$.
\end{enumerate}
\end{definition}

\begin{definition}
    There are many additional properties one can ascribe to a topological vector space.
    %
    \begin{enumerate}
        \item A topological vector space is {\bf locally convex} if there is a neighbourhood base of convex sets.
        \item A space is {\bf locally bounded} if there a neighbourhood base of bounded sets.
        \item An {\bf $\mathbf{F}$-space} is a vector space endowed with a complete, invariant metric ($d(v + u, w + u) = d(v,w)$ for any vectors $v,w,u$).
        \item A {\bf Fr\'{e}chet} space is a locally convex $F$-space.
        \item A {\bf normed space} is a vector space endowed with a norm function $\| \cdotp \|: V \to \mathbf{R}+$, such that
        %
        \begin{itemize}
            \item $\| v \| = 0$ if and only if $v = 0$.
            \item $\| v + w \| \leq \| v \| + \| w \|$.
            \item $\| \lambda v \| = | \lambda | \| v \|$.
        \end{itemize}
        %
        We can consider a norm space as an $F$-space by defining a distance function $d(v,w) = \| v - w \|$.
        \item A {\bf Banach space} is a complete normed space.
        \item A space is {\bf normable} if its topology can be induced by a norm.
        \item A space is {\bf Heine-Borel} if every closed-bounded set is compact.
    \end{enumerate}
\end{definition}

\begin{theorem}
    Every locally bounded space is first countable.
\end{theorem}
\begin{proof}
    
\end{proof}

We will be studying a specific subclass of topological vector space.

\begin{definition}
    A norm space is a vector space $V$ endowed with a norm function $\| \cdotp \|: \mathbf{R} \to \mathbf{R}^+$, such that
    %
    \begin{enumerate}
        \item $\| v \| = 0$ if and only if $v = 0$.
        \item $\| \lambda v \| = | \lambda | \| v \|$.
        \item $\| v + w \| \leq \| v \| + \| w \|$.
    \end{enumerate}
    %
    We obtain a distance function on a norm space by defining $d(v,w) = \| v - w \|$.
\end{definition}

\section{Convexity}

A function $f:(a,b) \to \mathbf{R}$ is convex when the line segment between $(a,f(a))$ and $(b,f(b))$ lies above the graph of $f$. The line segment connecting these two points is described by
%
\[ \{ (\lambda a + (1 - \lambda) b, \lambda f(a) + (1 - \lambda) f(b)) : 0 \leq \lambda \leq 1 \} \]
%
and we require that $(\lambda a + (1 - \lambda)b, f(\lambda a + (1 - \lambda)b)$ lies below the corresponding point on the line defined above. This is satisfied exactly when we have a certain inequality:

\begin{definition}
    A function $f:U \to \mathbf{R}$ is {\bf convex} on $(a,b)$ if, for any $a \leq x < y \leq b$, and $0 \leq \lambda \leq 1$, we have
    %
    \begin{equation} \label{convex1} f(\lambda x + (1 - \lambda) y) \leq \lambda f(x) + (1 - \lambda) f(y) \end{equation}
    %
    By rewording the definition, convexity is also satisfied when, for $a \leq x < y < z \leq b$,
    %
    \begin{equation} \label{convex2} \frac{f(y) - f(x)}{y - x} \leq \frac{f(z) - f(y)}{z - y} \end{equation}
    %
    Geometrically, this equation says that the slope of the tangent line from $(x, f(x))$ to $(y, f(y))$ is smaller than the tangent line from $(y, f(y))$ to $(z, f(z))$.
\end{definition}

\begin{lemma}
    A $C^1$ convex function's derivative is non-decreasing on $(a,b)$.
\end{lemma}
\begin{proof}
    Suppose $f$ is a $C^1$ function, and $f'$ is non-decreasing, then consider any $a \leq x < y < z \leq b$. Then $f'(x) \leq f'(y) \leq f'(z)$. Applying the mean value theorem, we conclude there is $t \in (x,y)$, and $u \in (y,z)$, for which
    %
    \begin{equation} \label{convexderivative} f'(t) = \frac{f(y) - f(x)}{y - x}\ \ \ \ \ f'(u) = \frac{f(z) - f(y)}{z - y} \end{equation}
    %
    And since $t < u$, (\ref{convexderivative}) implies $f'(t) \leq f'(u)$, i.e. (\ref{convex2}) is satisfied.

    If $f$ is $C^1$ and convex, then surely $f'$ is non-decreasing. Fix $a \leq x < y \leq b$. In (\ref{convex2})), letting $y$ converge to $x$, we obtain
    %
    \begin{equation} \label{yconvergeleft} f'(x) = \lim_{y \to x^+} \frac{f(y) - f(x)}{y - x} \leq \lim_{y \to x} \frac{f(z) - f(y)}{z - y} = \frac{f(z) - f(x)}{z - x} \end{equation}
    %
    Conversely, letting $y \to z$, we obtain
    %
    \begin{equation} \label{yconvergeright} f'(y) = \lim_{y \to z^-} \frac{f(z) - f(y)}{z - y} \geq \lim_{y \to z} \frac{f(y) - f(x)}{y - x} = \frac{f(z) - f(x)}{z - x} \end{equation}
    %
    And in tandem, (\ref{yconvergeleft}), (\ref{yconvergeright}) and (\ref{convex2}) imply that $f'(x) \leq f'(y)$.
\end{proof}

\begin{example}
    $\exp: \mathbf{R} \to \mathbf{R}$ is convex on $(-\infty, \infty)$, since $\exp'' = \exp > 0$.
\end{example}

\begin{lemma}
    A function is continuous on the open segments where it is convex.
\end{lemma}

The most important inequality in analysis is the triangle inequality, undershadowed by the Schwarz inequality. Almost as important is Jensen's inequality. Despite its importance, the proof is fairly simple and intuitive. Consider the center of mass of an object.  

\begin{theorem}[Jensen's Inequality]
    Let $(\Omega, \mathbf{P})$ be a probability space. If $X \in L^1(\mathbf{P})$, where $a < X< b$, and if $f$ is a real function, convex on $(a,b)$, then
    %
    \begin{equation} \label{jensen} f(\mathbf{E}[X]) \leq \mathbf{E}[f \circ X] \end{equation}
\end{theorem}
\begin{proof}
    Since $a < X < b$, $a < \mathbf{E}[X] < b$. Let
    %
    \[ \beta = \sup_{a < x < \mathbf{E}[X]} \frac{g(\mathbf{E}[X]) - g(x)}{\mathbf{E}[X] - x} \]
    %
    For any $a < x < y$, $g(x) + \beta(\mathbf{E}[X] - x) \geq g(y)$. But also, by the right side of $(\ref{convex2})$, for any $z > y$, $g(z) - \beta(z - \mathbf{E}[X]) \geq g(\mathbf{E}[X])$. For any $\omega \in \Omega$, we may restate these equations as as
    %
    \[ g(f(\omega)) - \beta(f(\omega) - y) \geq g(y) \]
    %
    But then taking expectations, we obtain (\ref{jensen}).
\end{proof}

Jensen's inequality is incredibly useful. To see this, consider some examples.

\begin{example}
    We have seen the exponential function is convex. Hence for any $L_1(\mathbf{P})$, where $\mathbf{P}$ is a probability measure, we have
    %
    \begin{equation} \label{harmonic} exp\left(\int f d\mathbf{P} \right) \leq \int e^f d\mathbf{P} \end{equation}
    %
    Let $\mathbf{P}$ be the uniform measure over a finite set $\{ x_1, x_2, \dots, x_n \}$. Then (\ref{harmonic})) tells us that
    %
    \[ e^{f(x_1)/n}e^{f(x_2)/n} \dots e^{f(x_n)/n} \leq \frac{\sum e^{f(x_i)}}{n} \]
    %
    If we let $x_i$ be real numbers, and $f(x_i) = \log(x_i)$, we conclude
    %
    \[ (x_1x_2 \dots x_n)^{1/n} \leq \frac{\sum x_i}{n} \]
    %
    In other words, the geometric mean is always smaller than the arithmetic mean.
\end{example}

Jensen's inequality implies so many other inequalities in analysis.

\begin{definition}
    Let $1 \leq p \leq \infty$. We say $1 \leq q \leq \infty$ is the {\bf conjugate} of $p$ if $p^{-1} + q^{-1} = 1$, and write $q = p'$.
\end{definition}

\begin{theorem}[H\"{o}lder]
    If $p' = q$, $(\Omega,\mu)$ is a measure space, and $f,g$ are positive measurable functions on $\Omega$, then
    %
    \begin{equation} \label{holder} \| fg \|_1 \leq \| f \|_p \| g \|_q \end{equation}
\end{theorem}
\begin{proof}
    Let $A,B$ be the values on the right hand side of (\ref{holder})). If $A = 0$, then $f = 0$ almost everywhere, so that the theorem is trivial. Symmetry shows that the same is true if $B = 0$, so assume $A, B \neq 0$. Define $F = f/A$, and $G = g/B$. Then
    %
    \[ \left( \int F^p d\mu \right)^{1/p} = A^{-1} \left( \int f^p d\mu \right)^{1/p} = 1\ \ \ \ \ \left( \int G^q d\mu \right)^{1/q} = B^{-1} \left( \int g^q d\mu \right)^{1/q} = 1 \]
    %
    For any $x$, there is a number $s$ such that $e^{s/p} = F(x)$, and $t$ such that $e^{t/q} = G(x)$. By the convexity of the exponential function, $e^{s/p + t/q} \leq e^s p^{-1} + e^t q^{-1}$. Thus $FG \leq p^{-1} F^p + q^{-1} G^q$, and thus by integrating,
    %
    \[ \int FG\ d\mu \leq p^{-1} + q^{-1} = 1 \]
\end{proof}

\begin{corollary}[Minkowski]
    For
    \[ \| f + g \|_p \leq \| f \|_p + \| g \|_p \]
\end{corollary}



\chapter{Hilbert Spaces}

A Hilbert space is a Banach space most similar to Euclidean space. They have an incredibly structure, and occur in wide applications of functional analysis to mathematics, physics, and computing science.

\begin{definition}
    A {\bf Hilbert space} is a complete inner product space -- That is, a vector space equiped with an inner (hermitian) product such that the corresponding metric structure is complete.
\end{definition}

In this chapter, we shall let $H$ denote a general Hilbert space, and $\langle \cdot, \cdot \rangle$ the inner product space with which the space is equipped.

\begin{theorem}[Cauchy-Schwarz inequality]
    $\langle x, y \rangle \leq \| x \| \| y \|$.
\end{theorem}




\chapter{Operator Algebras}

In Linear algebra, one studies structure theorems for linear transformations from a finite dimensional vector space to itself. One shows that, in almost all cases, such a transformation can be `diagonalized': we are able to choose a basis of vectors which map into multiples of themselves under the linear transformation. We shall attempt to extend methods related to classification to infinite dimensional spaces.

\begin{definition}
    A {\bf compact operator} between two spaces $X$ and $Y$ maps bounded sets onto precompact sets. The set of all compact operators is denoted $K(X,Y)$.
\end{definition}

\begin{lemma}
    Every compact operator is bounded.
\end{lemma}
\begin{proof}
    The image of the unit ball is pre-compact, hence bounded.
\end{proof}

\begin{definition}
    A {\bf finite-rank operator}
\end{definition}




\chapter{Spectral Theory}

Recall that an algebra over a field $\mathbf{F}$ is an $\mathbf{F}$ vector-space equipped with an associative multiplicative structure which contains an identity. The set of all units (invertible elements) on such an algebra $A$ will be denoted $U(A)$. A consistant norm attached to an algebra is a nice situation to study characterization theorems, since all algebras can be identified with a set of linear maps over some vector space. We shall use capital letters, like $M$ and $N$, to denote abstract elements of a Banach algebra, and denote algebras by capital letters near the beginning of the alphabet, such as $A$ or $B$.

\begin{definition}
    A {\bf Banach Algebra} is a Banach space $A$ which is also an algebra, and satisfies $\| MN \| \leq \| M \| \| N \|$ for any $M,N \in A$, and $\| 1 \| = 1$.
\end{definition}

It is obvious that this makes multiplication (in addition to addition and scaling) a continuous operation on the space. It is interesting to note that this is really all you need to define a Banach algebra, provided it has an identity element.

\begin{theorem}
    Let $V$ be a Banach space upon which a continuous multiplication structure has been defined. Then there is an equivalent norm on $V$ which makes the space into a Banach algebra.
\end{theorem}
\begin{proof}
    Embed $V$ in $B(V)$ by defining $\Lambda_M(N) = MN$, for each $M \in V$ (since multiplication on the right is continuous, $\Lambda_M$ truly is in $B(V)$ rather than just being a linear map). This is an algebra isomorphism, which is also a Banach space isomorphism, provided  that the set of all $\Lambda_M$ is closed in $B(V)$, so we may apply the inverse function theorem. To see this, note two nice properties of the map. First,
    %
    \[ \| M \| = \| M \cdot 1 \| = \| \Lambda_{M}(1) \| \leq \| \Lambda_{M} \| \| 1 \|  \]
    %
    and second, $\Lambda_{M + N} = \Lambda_M + \Lambda_N$. Suppose $\Lambda_{M_i}$ is a Cauchy seqeunce. Then the $M_i$ are a Cauchy sequence, since
    %
    \[ \| M_i - M_j \| \leq \| \Lambda_{M_i - M_j} \| \| 1 \| = \| \Lambda_{M_i} - \Lambda_{M_j} \| \| 1 \| \]
    %
    and therefore they converge to an element $M$. By right continuity, $M_i N \to M N$ for all $N$, so by definition, $\Lambda_{M_i} \to \Lambda_M$. One may verify that the set $\{ \Lambda_m \}$ is a Banach algebra.
\end{proof}

\begin{example}
    Let $K$ be a compact space. Then the space $C(K)$ of $\mathbf{F}$-valued continuous functions is a Banach algebra over $\mathbf{F}$ under the norm $\| \cdot \|_\infty$ and with pointwise multiplication. If $K$ is finite, then $C(K) \cong \mathbf{F}^{|K|}$. More generally, we may consider the set of all essentially bounded functions on some measure space.
\end{example}

\begin{example}
    More generally, the space $C_b(X)$ of continuous, bounded $\mathbf{F}$-valued functions on any topological space $X$ is a Banach algebra. If $X$ is locally compact, then the space $C_0(X)$ of functions which vanish at infinity form a Banach algebra, except that the space does not always contain an identity. Normally, there exists a trick to add an identity to the space on any `Banach algebra without identity' -- in this case, we enlarge the space to consist of the space of functions which eventually become constant at infinity.
\end{example}

\begin{example}
    If $K$ is a compact neighbourhood in $\mathbf{C}$, then we define $A(K)$ to be the set of continuous functions defined on $K$ which are analytic in $K^\circ$. Since uniform convergence preserves holomorphicity, $A(K)$ is a closed subset of $C_0(K)$ and is therefore a Banach algebra. The algebra $A(\mathbf{D})$ is known as the disk algebra.
\end{example}

\begin{example}
    If $G$ is a group with measure $\mu$, $L_1(G, \mu)$ is a Banach algebra, where we define multiplication as convolution,
    %
    \[ (f * g)(x) = \int f(y) g(xy^{-1}) d\mu(y) \]
    %
    This algebra is commutative when $G$ is a commutative. It does not always possess a unit, unless we enlarge the space. We identify each $f$ with the measure $f \lambda$, where $(f \lambda) (E) = \int_E f d\lambda$. In other words, $f \lambda$ is the measure defined by the equation
    %
    \[ \frac{d(f \lambda)}{d \lambda} = f \]
    %
    Then the set of all measures of the form $f \lambda + \mu \delta$ where $\delta$ is the dirac function evaluated at the identity, and $\mu \in \mathbf{F}$ form a Banach algebra under the total variation norm, and multiplication is defined to be convolution of measures,
    %
    \[ (\nu * \eta)(E) = \int \chi_E (tu^{-1}) d \nu(t) d\eta(u) \]
\end{example}

All examples above have commutative algebra structures. One of the prime reasons to study Banach algebras is to study operators on a Banach space, which are almost always non-commutative. In fact, some folks call the study of operator algebras `non-commutative analysis'.

\begin{example}
    Let $E$ be a Banach space. The space $B(E)$ of all bounded linear operators from $E$ to itself is a Banach algebra with respect to the operator norm. Our theorem above essentially says every Banach algebra is a closed subalgebra of this kind of space. It is a unital algebra, since it possesses the identity operator. The subset $K(E)$ of compact linear operators is a closed (double-sided) ideal of $B(E)$, and so is also a Banach algebra. $K(E)$ is a unital algebra if and only if $E$ is finite dimensional.
\end{example}

The abstraction to Banach algebras is justified since we may talk about many classes of linear operators at once. Now to begin the Spectral theory. We shall restrict our attention almost everywhere to algebras over $\mathbf{C}$ because we may apply the deep and beautiful properties of holomorphic functions,

\begin{definition}
    The {\bf spectrum} and {\bf resolvent} of an element $M$ of a Banach algebra $A$ are defined respectively as
    %
    \[ \sigma_{A}(M) = \{ \lambda \in \mathbf{C} : \lambda - M \not \in U(A) \} \]
    %
    \[ \rho_{A}(M) = \{ \lambda \in \mathbf{C} : \lambda - M \in U(A) \} \]
    %
    The resolvent is the complement of the spectrum.
\end{definition}

\begin{example}
    Let $X$ be a space, and consider $f \in C_b(X)$. Then $\sigma_{C_b(X)}(f) = \overline{f(X)}$. If $\lambda \in \rho_{C_b(X)}(f)$, then
    %
    \[ (\lambda - f)^{-1}(x) = \frac{1}{\lambda - f(x)} \]
    %
    This continuous function is bounded if and only if $\lambda \not \in \overline{f(X)}$.
\end{example}

\begin{example}
    Consider a Banach space $E$. $T \in B(E)$ is invertible if and only if it is bijective, by the inverse mapping theorem. If $\dim(E) < \infty$, this is the set of injective operators. The spectrum is then exactly the set of eigenvalues of the operator. One can consider eigenvalues in the infinite operator, yet they are almost always a proper subset of the spectra.
\end{example}

\begin{definition}
    The {\bf point spectra} of an element $\Lambda \in B(V)$, where $V$ is an Banach algebra over $\mathbf{F}$, is
    %
    \[ \sigma_p(M) = \{ \lambda \in \mathbf{F} : \ker(\lambda I - M) \neq \{ 0 \} \} \]
    %
    If $\lambda \in \sigma_p(M)$, there is $v \in V$, $v \neq 0$, with $\Lambda v = \lambda v$. $v$ is known as an {\bf eigenvector}.
\end{definition}

\begin{lemma}[Neumann Series]
    If $M \in B_{A}$, then $1 - M \in U(A)$, and
    %
    \[ (1 - M)^{-1} = \sum_{k = 0}^\infty M^k \]
    %
    in the sense that the right hand side converges and satisfies the equality.
\end{lemma}
\begin{proof}
    The right side converges absolutely by the comparison test, since $\| M^k \| \leq \| M \|^k$. Since $A$ is Banach, the right side converges, and
    %
    \[ (1 - M) \sum_{k = 0}^\infty M^k = \sum_{k = 0}^\infty (1 - M)M^k = \sum_{k = 0}^\infty M^k - M^{k+1} = \lim_{n \to \infty} 1 - M^{n+1} \]
    %
    As $n \to \infty$, $M^{n+1} \to 0$, so the limit above tends to one. We may repeat this argument by multiplying on the right hand side, which shows the sum truly is the inverse.
\end{proof}

\begin{corollary}
    If $\| 1 - M \| < 1$, then $M \in U(A)$, and
    %
    \[ M^{-1} = \sum_{k = 0}^\infty (1 - M)^k \]
\end{corollary}
\begin{proof}
    Apply the theorem above, with $M$ replaced with $1 - M$.
\end{proof}

\begin{corollary}
    $U(A)$ is an open subset of $A$.
\end{corollary}
\begin{proof}
    If $M \in U(A)$, and if $\| M - N \| < 1/\| M^{-1} \|$, then 
    %
    \[ \| 1 - M^{-1}N \| \leq \| M^{-1} \| \| M - N \|  < 1 \]
    %
    so $M^{-1}N \in U(A)$, and thus $N \in U(A)$.
\end{proof}

\begin{corollary}
    $\sigma(M)$ is a closed subset of $\mathbf{F}$, and $\rho(M)$ is open.
\end{corollary}
\begin{proof}
    The map $f: \lambda \mapsto \lambda - M$ is a continuous operation, for
    %
    \[ \| (\lambda - M) - (\mu - M) \| = \| \lambda - \mu \| = | \lambda - \mu | \]
    %
    Since $U(A)$ is open, $f^{-1}(U(A)) = \rho(M)$ is open.
\end{proof}

\begin{theorem}
    $\sigma(M)$ is a compact subset of $A$.
\end{theorem}
\begin{proof}
    If $|\lambda| > \|M\|$, then $\| M/\lambda \| < 1$, so $(1 - M/\lambda) \in U(A)$, which means $\lambda - M$ is also invertible. Thus $\sigma(A)$ is closed and bounded, hence compact.
\end{proof}

We shall see that the spectra of a complex linear operator never has an empty spectra. This is why we mainly study $\mathbf{C}$-algebras, rather than $\mathbf{R}$-algebras -- there are even finite dimensional real operators with empty spectra (take a rotation matrix).

\begin{lemma}
    Inversion in an operator algebra $A$ is continuous.
\end{lemma}
\begin{proof}
    Let $(M_n)$ be a sequence in $U(A)$ converging to an invertible element $M$. Then, by continuity, $M_nM^{-1} \to 1$. If $\| 1 - M_n M^{-1} \| < 1/2$, then
    %
    \[ M_n^{-1} M = (M^{-1}M_n)^{-1} = \sum_{k = 0}^\infty (1 - M^{-1}M_n)^k \]
    %
    It follows that
    %
    \begin{align*}
        \| M_n^{-1} \| \leq \| M^{-1} \| \| M M_n^{-1} \| \leq \| M^{-1} \| &\sum_{k = 0}^\infty \| (1 - M_nM^{-1})^k \|\\
        &\leq \| M^{-1} \| \sum 2^{-k} = 2 \| M^{-1} \|
    \end{align*}
    %
    Finally, we obtain convergence of inverses,
    %
    \[ \| M_n^{-1} - M^{-1} \| = \| M_n^{-1} (M - M_n) M^{-1} \| \leq \overbrace{\| M_n^{-1} \|}^{\leq 2 \| M^{-1} \|} \overbrace{\| M - M_n \|}^{\to 0} \overbrace{\| M^{-1} \|}^{\text{constant}} \to 0 \]
    %
    so inversion is continuous.
\end{proof}

We can now prove the fundamental theorem of spectral theory, after a brief interlude with complex analysis.

\begin{definition}
    The resolvent of $M$, defined on $\rho(M)$, is the map
    %
    \[ R(z; M) = (A - zI)^{-1} \]
\end{definition}

\begin{lemma}
    $R$ is analytic, in that $\langle \phi, R(\cdot, M) \rangle$ is analytic for each $\phi \in A^*$.
\end{lemma}
\begin{proof}
    Let $f = \langle \phi, R(\cdot, M) \rangle$. Then
    %
    \begin{align*}
        \frac{f(\lambda + h) - f(\lambda)}{h} &= \frac{\langle \phi, (\lambda + h - a)^{-1} - (\lambda - a)^{-1} \rangle}{h}\\
        &= \frac{\langle \phi, (\lambda + h - a)^{-1} (\lambda - a)^{-1} [(\lambda - a) - (\lambda + h - a)] \rangle}{h}\\
        &= \langle \phi, -(\lambda + h - a)^{-1} (\lambda - a)^{-1} \rangle
    \end{align*}
    %
    As $h \to 0$, this tends to $\langle \phi, - (\lambda - a)^{-2} \rangle$, by continuity of inversion.
\end{proof}

In Banach theory, we call such a mapping {\bf weakly analytic}. A {\bf strongly analytic} function $f$ is then a mapping a subset of $\mathbf{C}$ to a Banach space such that the limit
%
\[ \lim_{h \to 0} \frac{f(z + h) - f(z)}{h} \]
%
exists at every point $z$ in the domain. By the chain rule, every strongly analytic function is weakly analytic. It is surprising that the converse also holds.

\begin{theorem}
    Every weakly analytic function $f$ into a Banach space $V$ is strongly analytic.
\end{theorem}
\begin{proof}
    Fix $\phi \in V^*$. Consider a particular contour winding counterclockwise around a point $w$ in the domain, which is at least a unit distance away from $w$ at any point on the contour. If $h,k \in \mathbf{C}$ are small enough that $w + h$ and $w + k$ are contained within the contour, then by the Cauchy integral theorem,
    %
    \begin{align*}
        \left\langle \phi, \frac{1}{h-k} \left[ \frac{f(w + h) - f(w)}{h} - \frac{f(w + k) - f(k)}{k} \right] \right\rangle\\
        = \frac{1}{2\pi i} \int_C \frac{\langle \phi, f(z) \rangle\ dz}{[z - (w + h)][z - (w + k)][z - w]}
    \end{align*}
    %
    Find $\delta$ such that if $\| h \| < \delta$, the distance between any point on $C$ and $w + h$ is greater than $1/2$. Then, if $M$ is the length of $C$, and $K$ is the supremum of $f$ on $C$, then
    %
    \[ \left| \frac{1}{2\pi i} \int_C \frac{\langle \phi, f(z) \rangle\ dz}{[z - (w + h)][z - (w + k)][z - w]}\right| \leq \frac{4MK}{2 \pi} \| \phi \| = \frac{2MK}{\pi} \| \phi \| \]
    %
    Applying the Banach Steinhaus theorem (on $X^*$, viewing elements of $X$ as elements of $X^{**}$), we conclude that for all $h,k$ sufficiently small, there exists $D$ such that
    %
    \[ \left| \frac{f(w + h) - f(w)}{h} - \frac{f(w + k) - f(k)}{k} \right| \leq D |h - k| \]
    %
    By the completeness of $V$, the quotients of $h$ and $k$ converge to a well defined quantity as $h - k$ converges to zero.
\end{proof}

There is a deep relationship between complex analysis and Banach algebras. We shall return to `holomorphic functional analysis' later.

\begin{theorem}
    Points of a complex Banach algebra has nonempty spectrum.
\end{theorem}
\begin{proof}
    Assume $\sigma(M)$ is empty. Then $\lambda - M$ is always invertible, for all $\lambda \in \mathbf{C}$. Fix an arbitrary $\phi \in A^*$. Then $f = \langle \phi, R(\cdot, M) \rangle$ is an entire function. What's more, $f$ is bounded, since, as $\lambda \to \infty$,
    %
    \[ | \langle \phi, (\lambda - M)^{-1} \rangle | \leq \| \phi \| \| (\lambda - M)^{-1} \| = \| \phi \| \left\| \sum_{k = 0}^\infty \frac{M^k}{\lambda^{k+1}} \right\| \to 0 \]
    %
    Which implies that $f$ is constant, and since it converges to zero at infinity, $f = 0$. In particular, this means that $\langle \phi, M^{-1} \rangle = 0$. But this contradicts the Hahn-Banach theorem, which says that for any non-zero element of a banach space there exists a bounded linear functional non-zero at the element. Hence $\sigma(M)$ must be non-empty.
\end{proof}

A cute little theorem results from this property of Banach algebras.

\begin{corollary}
    Every complex Banach division algebra is isometric to $\mathbf{C}$.
\end{corollary}
\begin{proof}
    Let $A$ be a complex division algebra, and fix $M \in A$. Pick some $\lambda \in \sigma(A)$. Then $\lambda - M \not \in U(A)$, hence $\lambda - M = 0$, i.e. $M = \lambda$. Thus $A = \mathbf{C} \cdot 1 \cong \mathbf{C}$.
\end{proof}

The real case is much more complicated. There are three real division algebras: $\mathbf{R}$, $\mathbf{C}$, and $\mathbf{Q}$ (the quaternions).

\begin{definition}
    The {\bf spectral radius} of an element $M \in A$ is defined to be
    %
    \[ r(M) = \sup \{ |\lambda| : \lambda \in \sigma(M) \} \]
\end{definition}

What is amazing is that we can define the spectral radius without any reference to the spectrum -- this is crazy, since if we enlarge our Banach algebra, more elements in the spectrum become invertible and are removed. Regardless, the supremum of the non-invertible elements will stay the same. To show this, we first prove a lemma

\begin{lemma}
    Let $M \in A$, $n \in \mathbf{N}$. Then, if $\lambda \in \sigma(M)$, $\lambda^n \in \sigma(M^n)$.
\end{lemma}
\begin{proof}
    Suppose $\lambda \in \sigma(M)$. Then
    %
    \[ \lambda^n - M^n = (\lambda - M) \left(\sum \lambda^{n-1-k} M^k \right) = \left(\sum \lambda^{n-1-k} M^k \right) (\lambda - M) \]
    %
    If $\lambda^n - M^n$ was invertible, then $\lambda - M$ would also be invertible.
\end{proof}

\begin{theorem}[Spectral Radius Theorem]
    \[ r(a) = \lim_{n \to \infty} \| a^n \|^{1/n} \]
\end{theorem}
\begin{proof}
    Let $\lambda \in \sigma(M)$. Then $\lambda^n \in \sigma(M^n)$. Thus $|\lambda|^n \leq \| M^n \|$, hence
    %
    \[ |\lambda| \leq \| M^n \|^{1/n} \]
    %
    Thus $r(M) \leq \| M^n \|^{1/n}$, so $r(M) \leq \liminf_{n \to \infty} \| M^n \|^{1/n}$.

    Set $R = 1/r(M)$ (which can be $\infty$, if $r(M) = 0$), and $r = 1/\|M\|$. Let $\lambda$ be a complex number with modulus less than $R$. Then $1/|\lambda| > r(M)$, so $1 - \lambda M \in U(A)$. If $\phi \in A^*$, define the function
    %
    \[ f: \lambda \mapsto \langle \phi, (1 - \lambda M)^{-1} \rangle \]
    %
    Then $f$ is holomorphic in the disk of radius $R$, as we have already verified. If $\lambda$ has radius less than $r$, then $\| \lambda M \| < 1$, $1 - \lambda M \in U(A)$, and
    %
    \[ \langle \phi, (1 - \lambda M)^{-1} \rangle = \sum_{k = 0}^\infty \langle \phi, M^k \rangle \lambda^k \]
    %
    %
    but power series expansions are unique, hence this expansion should work in the whole disk of radius $R$. But $\phi$ was arbitrary, so the sequence $\lambda^k M^k$ must be bounded. If $\lambda$ is fixed, then there is $C$ such that
    %
    \[ |\lambda^k| \|M^k\| \leq C \]
    %
    So
    %
    \[ \|M^k\|^{1/k} \leq \frac{C^{1/n}}{\lambda} \]
    %
    Hence $\limsup \|M^k\|^{1/k} \leq \frac{1}{\lambda}$. Letting $\lambda \to R$, we obtain that $\limsup \|M^k\|^{1/k} \leq r(a)$. We have shown $\liminf \|M^k\|^{1/k} = r(M) = \limsup \|M^k\|^{1/k}$, from which the theorem follows.
\end{proof}

\begin{corollary}
    If $A$ is an algebra, and $B$ a closed subalgebra both containing $M$, then $r_{A}(M) = r_B(a)$.
\end{corollary}

\begin{example}
    We can isometrically embed $A(\mathbf{D})$ in $C_0(\mathbf{T})$ via the map $f \mapsto f|_\mathbf{T}$. The fact that this is an isometry follows from the maximum modulus principle. Let $z: \mathbf{T} \to \mathbf{C}$ be the identity map. Then $\sigma_{A(\mathbf{D})}(z) = \mathbf{D} \supsetneqq \mathbf{T} = \sigma_{C_0(\mathbf{T})}(z|_\mathbf{T})$.
\end{example}

\begin{theorem}
    Let $A \subset B$ be Banach algebras, both containing an element $M$. Then $\sigma_B(M) \subset \sigma_B(M)$, and if $\lambda \in \sigma_A(M) \cap \rho_B(M)$, then the connected component of $\lambda$ in $\rho_B(M)$ is contained in $\sigma_A(M)$.
\end{theorem}
\begin{proof}
    Let $V$ be a component of $\rho_B(M)$. Obviously, $V \cap \rho_A(M)$ is open in $V$, since $\rho_A(M)$ is open. But $V \cap \rho_A(M)$ is also closed, for if $\lambda_n \in V \cap \rho_A(M)$ converges to $\lambda \in V$, then $\lambda_n - M \in U(A)$ converges to $\lambda - M \in U(B)$, then $\lambda - M \in U(A)$, since $A$ is a closed subalgebra, and $(\lambda_n - M)^{-1} \to (\lambda - M)^{-1}$. A non-empty open and closed subset of a connected space is either empty or the entire space. It follows that if $V \cap \sigma_A(M) \neq \emptyset$, then $V \subset \sigma_A(M)$.
\end{proof}

\begin{corollary}
    If $A \subset B$ are algebras, with $\sigma_B(M) \subset \mathbf{R}$, then $\sigma_A(M) = \sigma_A(M)$.
\end{corollary}
\begin{proof}
    If $\sigma_B(M)$ is a bounded subset of $\mathbf{R}$, so $\rho_B(M)$ is connected in $\mathbf{C}$. Hence $\sigma_A(M) = \sigma_B(M)$, since $\sigma_A(M) \neq \mathbf{C}$.
\end{proof}

\section{Gelfand Theory}

Gelfand realized that all commutative Banach algebras were really just spaces of continuous functions in disguise. This has incredibly important repurcussions which we will get to. The main part of Gelfand's theory is the connection between homomorphisms and maximal ideals -- an additive subgroup of the ring closed under multiplication by any element of the algebra, and under scalar multiplication by elements of a field. An arbitrary ideal of a algebra shall be denoted by gaudy letters near the beginning of the alphabet, like $\mathfrak{a}$ and $\mathfrak{b}$. We note that no proper ideal contains invertible elements in a ring, and that every ring possesses some maximal ideal (appealing to some method of transfinite induction).

\begin{lemma}
    Every maximal ideal of a Banach algebra is closed.
\end{lemma}
\begin{proof}
    Let $\mathfrak{a}$ be a maximal ideal of an algebra $A$. It is easy to show the closure of any ideal is an ideal. It follows that either $\overline{\mathfrak{a}} = \mathfrak{a}$ (so that $\mathfrak{a}$ is closed), or $\mathfrak{a}$ is dense in $A$. Suppose the second option holds. Let $a \in U(A)$ be chosen. Then there is $a_i \in \mathfrak{a}$ converging to $a$. But then the $a_i$ are eventually invertible, since $U(A)$ is open, from which we conclude $\mathfrak{a} = A$, a contradiction.
\end{proof}

Next, we consider the notion of homomorphism between two algebras $A$ and $B$, which is a map which is both linear and a ring homomorphism (we note that this means that no homomorphism can be zero). The first isomorphism theorem holds in such a space, and preserves completeness. If $f:A \to B$ is a continuous homomorphism from a Banach algebra $A$ to a normed algebra $B$, then $K$ is a closed ideal in $A$ contained within the kernel of $f$, so that $A/K$ is also a Banach space, then there is a continuous map $\overline{f}$ from $A/K$ to $B$ such that the standard diagram commutes. If $K = \ker(f)$, the map is injective, and therefore is an Banach algebra isomorphism if it is surjective.

The key to Gelfand theory is noticing that homomorphisms of commutative rings naturally reflect the structure of the Banach algebra in question.

\begin{definition}
    For a commutative algebra $A$, $\Delta_A$ is the set of all continuous algebra homomorphisms from $A$ to $\mathbf{C}$.
\end{definition}

\begin{lemma}
    Every maximal ideal of $A$ is the kernel of some complex homomorphism, and correspondingly, the kernel of every complex homomorphism is a maximal ideal.
\end{lemma}
\begin{proof}
    If $\mathfrak{a}$ is a maximal ideal, then $\mathfrak{a}$ is closed, so $A/\mathfrak{a}$ is a complex Banach algebra. But $A/\mathfrak{a}$ is also a division ring, so is really just $\mathbf{C}$ in disguise. The map $\phi: A \to A/\mathfrak{a} \cong \mathbf{C}$ is then a complex homomorphism we require. Conversely, let $\phi: A \to \mathbf{C}$ be a homomorphism. Then $\phi$ is surjective, for $\phi(\mathbf{C} \cdot 1) = \mathbf{C}$. Then if $K = \ker(f)$, $A/K \cong \mathbf{C}$, and thus $K$ is maximal.
\end{proof}

\begin{lemma}
    $M \in U(A)$ if and only if $\phi(M) \neq 0$ for all $\phi \in \Delta_A$.
\end{lemma}
\begin{proof}
    If $M \in U(A)$, $h(M) \neq 0$ for all $h \in \Delta_A$, for otherwise $\ker(h)$ would contain an invertible element. Conversely, if $M \not \in U(A)$, then $M$ is contained in a maximal ideal $\mathfrak{a}$, and the projection $\phi: A \to A/\mathfrak{a}$ is non-zero at $M$.
\end{proof}

\begin{corollary}
    $\lambda \in \sigma(M)$ if and only if $h(M) = \lambda$ for some $h \in \Delta$.
\end{corollary}
\begin{proof}
    Apply the lemma above, with $M$ replaced with $\lambda - M$.
\end{proof}

\begin{lemma}
    If $\|M\| \leq 1$,  $| \phi(M) | \leq 1$ for any complex homomorphism $\phi \in \Delta_A$.
\end{lemma}
\begin{proof}
    If $|\lambda| \geq 1$, $\lambda - M$ is invertible. This implies that $\phi(M) \neq \lambda$ for any $\phi \in \Delta_A$, for otherwise $\lambda - M$ would be in the kernel of $\phi$.
\end{proof}

The next lemma is part of classical harmonic analysis. It was proved in an extremely convoluted way by Norbert Weiner. With Gelfand's theory, the proof was shortened to a two-liner.

\begin{lemma}[Wiener's Lemma]
    Consider a function which admits a Fourier expansion
    %
    \[ f(z) = \sum_{n = -\infty}^\infty a_n z^n \]
    %
    If $\sum_{-\infty}^\infty |a_n| < \infty$, and if $f(z) \neq 0$ for all $z \in \mathbf{T}$, then $1/f$ also admits a Fourier expansion, and it coefficients converge absolutely.
\end{lemma}
\begin{proof}
    Let $A$ be the set of all functions with absolutely convergent Fourier expansions. If we define $\| f \| = \sum |a_i|$, then $A$ is a commutative Banach algebra, for if
    %
    \[ f(z) = \sum_{n = -\infty}^\infty a_n z^n\ \ \ \ \ g(z) = \sum_{n = -\infty}^\infty b_n z^n \]
    %
    Then
    %
    \[ (fg)(z) = \sum_{n = -\infty}^\infty \left ( \sum_{m = -\infty}^\infty a_m b_{n-m} \right) z^n \]
    %
    and
    %
    \begin{align*}
        \sum_{n = -\infty}^\infty \left| \sum_{m = -\infty}^\infty a_m b_{n-m} \right| &\leq \sum_{n,m = -\infty}^\infty |a_n b_m| = \left( \sum_{n = -\infty}^\infty |a_n| \right) \left( \sum_{m = -\infty}^\infty |b_m| \right)
    \end{align*}
    %
    The space is complete, since it is essentially just $l_1(\mathbf{Z})$.

    The map $f \mapsto f(w)$ is a complex homomorphism, for each $w \in \mathbf{T}$. We wish to show that these are the only such homomorphisms, so that $f$ is invertible if and only if it does not equal 0 at any point. If $z$ denotes the identity function, then it has norm 1, as does its inverse $1/z$. This implies that $|\phi(z)| \leq 1$ for any $\phi \in \Delta_A$, and that $|1/\phi(z)| = |\phi(1/z)| \leq 1$. Thus $\phi(z) = z(w) = w$, for some $w \in \mathbf{T}$. If $P = \sum_{k = -n}^m a_k z^k$ is a trigonometric polynomial function, it follows that $\phi(P) = P(w)$. But the trigonometric polynomials are dense in $A$, so we have proved what was needed to be shown.
\end{proof}

The next theorem applies ideal theory to complex analysis.

\begin{lemma}
    Let $f_1, \dots, f_n \in A(\mathbf{D})$, and suppose that for each $z \in \mathbf{D}$ at least one of the $f_i$ satisfy $f_i(z) \neq 0$. Then there are $g_1, \dots, g_n \in A(\mathbf{D})$ such that $\sum f_i g_i = 1$.
\end{lemma}
\begin{proof}
    In other words, $(f_1, \dots, f_n) = A(\mathbf{D})$. If $(f_1, \dots, f_n)$ is not $A(\mathbf{D})$, then $(f_1, \dots, f_n)$ is annihilated by some $\phi \in \Delta_{A(\mathbf{D})}$. We have $\phi(z) = w$, for some $w \in \mathbf{D}$. Then $\phi(P) = P(w)$, as in the last proof. Polynomials are dense in the set of holomorphic functions, so that $\phi(f) = f(w)$ for all $f \in A(\mathbf{D})$. But then $f(w) = 0$ for all $f \in (f_1, \dots, f_n)$, a contradiction which proves the theorem.
\end{proof}

\end{document}



There is a deep relationship between single variate complex analysis, and maps from $\mathbf{C}$ to complex Banach algebras. Complex analysis begins with an analysis of polynomials $P \in \mathbf{C}[X]$, mapping $M \in A$ to $P(M)$. We don't need anything topological here. However, we do need topology to consider power series. For instance, we define the exponential of $M \in A$ by
%
\[ e^M = \sum_{k = 0}^\infty \frac{M^k}{k!} \]
%
More generally, for any power series $f = \sum_{k = 0}^\infty c_k X^k$ with a radius of convergence $R$, we may define
%
\[ f(M) = \sum_{k = 0}^\infty c_k M^k \]
%
And this function is defined for any $M$ satisfying $\|M\| < R$. Furthermore, if $f$ is an analytic function, at any $w$ in the domain of $f$ we may write $f(z) = \sum c_k (z - w)^k$ in some radius $R$. We can then define $f(M) = \sum c_k (M - \lambda)^k$, for $\| M - \lambda \| < R$.