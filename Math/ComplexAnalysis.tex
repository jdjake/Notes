\documentclass[12pt]{report}

\usepackage{amsmath}
\usepackage{amssymb}
\usepackage{amsthm}
\usepackage{amsopn}
\usepackage{kpfonts}
\usepackage{graphicx}
\usepackage{kbordermatrix}
\usepackage{tikz}
\usetikzlibrary{arrows, petri, topaths}%
\usepackage{tkz-berge}
\usepackage{multicol}

\usepackage{framed}
\usepackage{mathtools}
\usepackage{float}
\usepackage{subfig}
% \usepackage{cmbright}

\theoremstyle{plain}
\newtheorem{theorem}{Theorem}[chapter]
\newtheorem{lemma}[theorem]{Lemma}
\newtheorem{corollary}[theorem]{Corollary}
\newtheorem{prop}[theorem]{Proposition}
\newtheorem{exercise}{Exercise}[chapter]

\newtheorem*{example}{Example}
\newtheorem*{proof*}{Proof}

\theoremstyle{definition}
\newtheorem*{defi}{Definition}
\newenvironment{definition}
    {\begin{samepage}\begin{framed}\begin{defi}}
    {\end{defi}\end{framed}\end{samepage}}





\usepackage{hyperref} 
\hypersetup{
    colorlinks = true,
    linkcolor = black,
}

\makeatletter
\renewcommand*\env@matrix[1][*\c@MaxMatrixCols c]{%
  \hskip -\arraycolsep
  \let\@ifnextchar\new@ifnextchar
  \array{#1}}
\makeatother

\renewcommand*\contentsname{\hfill Table Of Contents \hfill}

\newcommand{\optionalsection}[1]{\section[* #1]{(Important) #1}}
\newcommand{\deriv}[3]{\left. \frac{\partial #1}{\partial #2} \right|_{#3}}

\title{Complex Analysis}
\author{Jacob Denson}

\begin{document}

\pagenumbering{gobble}
\maketitle
\tableofcontents
\pagenumbering{arabic}

\chapter{Complexities}

In your mathematical career, you must have at least hear faint rumours of the exotic `complex numbers'. Like a creature from a far land, these numbers defy properties obvious to the real number line $\mathbf{R}$. This here atlas will uncover what gems lie beneath the seemingly abstract definition of complex numbers, and enable you to harness their power. We begin by categorizing what exactly these numbers are. Intuition should come later. Since there are many ways to define the complex number system, we take a categorical approach to defining what these numbers are.

\begin{definition}
    The complex number system $(\mathbf{C},+,\cdot)$ is the smallest field extension of the real line $(\mathbf{R}, +, \cdot)$ to contain a square root of negative one. This uniquely defines the field up to an isomorphism.
\end{definition}

There are many ways to construct the complex numbers. They can be constructed algebraically, to avoid certain problems when dealing with equations over the real field. Geometrically, they represent the Euclidean plane, so we can very concisely discuss the plane. Analytically, the operations of $\mathbf{C}$ allow us to study more specialized functions than those discussed in ordinary calculus.

\begin{example}
    Let $\mathbf{R}[X]$ be the set of all single variable polynomials over the real field. The polynomial $X^2 + 1$ is prime in $\mathbf{R}[X]$, and so $\mathbf{R}(X^2 + 1)$ is a maximal ideal in the polynomial ring. It follows that $\mathbf{C} := \mathbf{R}[X]/(X^2 + 1)$ is a field. Composing the natural embedding of $\mathbf{R}$ in $\mathbf{R}[X]$ with the projection onto the quotient space, we see that $\mathbf{R}$ naturally embeds in $\mathbf{C}$. Whats more, in $\mathbf{C}[X]$, the polynomial $X^2 + 1$ has a root, since $\overline{X}^2 + 1 = \overline{X^2 + 1} = 0$. We can write $X^2 + 1 = (X - \overline{X})(X + \overline{X})$. Suppose $\mathbf{F}$ is another field for which the polynomial $X^2 + 1$ has a root, and for which $\mathbf{R}$ embeds in $\mathbf{F}$. Then $\mathbf{R}[X]$ embeds in $\mathbf{F}[X]$ in the natural way by some morphism $f:\mathbf{R}[X] \to \mathbf{F}[X]$, defined by $f(\sum a_kX^k) = \sum f(a_k)X^k$. Fix some solution $y \in \mathbf{F}$ to $X^2 + 1$ in $\mathbf{F}$. If we compose $f$ with the morphism $g:\mathbf{F}[X] \to \mathbf{F}$ defined by $g(\sum a_k X^k) = \sum a_k y^k$, then $(X^2 + 1) \subset \ker(g \circ f)$. To prove this, just note that $(g \circ f)(X^2 + 1) = y^2 + 1 = 0$. Hence, by the first isomorphism theorem, $g \circ f$ induces a morphism $h:\mathbf{C} \hookrightarrow \mathbf{F}$, such that $h(\overline{k}) = g \circ f (k)$. Now suppose $h(\overline{k}) = 0$, for some $\overline{k} = \overline{\sum a_iX^i} \in \mathbf{C}$. Then $g \circ f (k) = \sum f(a_i)y^i = 0$.
\end{example}

\begin{thebibliography}{10}
    \bibitem{intro} Michael Spivak,
    \emph{A Concise Introduction to Differential Geometry: Vol. One}

    \bibitem{halm} Paul Halmos,
    \emph{Naive Set Theory}

    \bibitem{wiki} Wikipedia,
    \emph{Lie Groups}
\end{thebibliography}

\end{document}