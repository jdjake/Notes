\documentclass[12pt,a4paper]{article}

\usepackage{amsmath,amsthm,amsfonts,amssymb,amscd}
\usepackage{exsheets}
\usepackage{paralist}
\usepackage{fancyhdr}
\usepackage[top=2cm, bottom=4.5cm, left=2.5cm, right=2.5cm]{geometry}
\usepackage{enumitem}
\usepackage{multicol} %Allows to distribute \enumerates in multiple columns
\usepackage{multirow}

\theoremstyle{definition}
\newtheorem*{ssolution}{Solution} % This is implemented to not overlap with the exsheets package's solution environment.

\SetupExSheets{solution/print=true} %Set print=false to not print out solutions.
%\SetupExSheets{question/type=exam}
%\SetupExSheets[points]{name=point,name-plural=points}

\setlength{\parindent}{0.0in}
\setlength{\parskip}{0.05in}


\pagestyle{fancyplain}
\headheight 35pt
\lhead{\NetIDa}
\chead{\textbf{\Large Math 340 Worksheet 25}}
\lhead{Sections: 3.2 and 3.3}
\rhead{Date: 9.7.2022 \hspace{0.5in}}
\lfoot{}
\cfoot{}
\rfoot{\small\thepage}
\headsep 1.5em


\begin{document}
%% Written by: Jacob Denson
\PrintSolutionsTF{
    \textbf{The following are select solutions from the worksheet}.
}{
    \textbf{It may be useful for you to have this worksheet for future discussion sections.} \newline
    \textbf{It may be in your interest to solve the questions not in the order listed, but according to which questions you need practice with.}
    \textbf{Your TA may or may not give you specific advice or directions on which questions to try first.}
}


\begin{questions}

\begin{question}
True or False:
\begin{itemize}
	\item If $x \in V$ is an eigenvector for a linear operator $L: V \to V$, then so is $\lambda x$ for any scalar $\lambda$.

	\item If a $3 \times 3$ matrix $A$ has eigenvalues $-1$, $2$, and $5$, then $A$ is diagonalizable.

	\item Let $A$ and $B$ be $2 \times 2$ symmetric matrices. If $A$ and $B$ have the same trace and determinant, then $A$ and $B$ are similar.
\end{itemize}
\end{question}

\begin{question}
Find an \emph{orthogonal} matrix $P$ such that $P^T A P$ is diagonal:
%
\begin{itemize}
	\item
	%
	\[ A = \begin{pmatrix} 1 & 2 & 0 & 0 \\ 2 & 1 & 0 & 0 \\ 0 & 0 & 1 & 2 \\ 0 & 0 & 2 & 1 \end{pmatrix}. \]

	\item
	%
	\[ A = \begin{pmatrix} 0 & -1 & -1 \\ -1 & 0 & -1 \\ -1 & -1 & 0 \end{pmatrix}. \]

	\item
	%
	\[ A = \begin{pmatrix} -1 & 2 & 2 \\ 2 & -1 & 2 \\ 2 & 2 & -1 \end{pmatrix}. \]
\end{itemize}
\end{question}

\begin{question}
	If $A$ is an orthogonal matrix, prove that $1$ and $-1$ are the only real eigenvalues of $A$. Is there an orthogonal matrix which has neither $1$ nor $-1$ as an eigenvalue?
\end{question}

\begin{question}
	Let $V$ be the space of all differentiable functions on the real line. What are the eigenvalues and eigenvectors of the operator $L: V \to V$ given by \emph{differentiation}, i.e. $Lf = f'$, taking in a function, and outputting it's \emph{derivative}.
\end{question}

\begin{question}
	Construct an orthogonal basis for the subspace of $\mathbf{R}^3$ spanned by $[1, -1, 1]$, $[-2, 2, -2]$, $[2,-1,2]$, and $[0,0,0]$.
\end{question}

\begin{question}
	Consider the orthogonal basis $[1,0,2]$, $[-2,0,1]$, and $[0,1,0]$ for $\mathbf{R}^3$. Write $[2, -3, 1]$ as a linear combination of these three vectors (there's an easier way that solving a system of linear equations).
\end{question}

\begin{question}
	Consider the orthogonal basis
% 000 001 010 011 100 101 110 111
\[ [1,1,1,1,1,1,1,1], [1,-1,1,-1,1,-1,1,-1], [1,1,-1,-1,1,1,-1,-1] \]
\[ [1,1,1,1,-1,-1,-1,-1], [1,-1,-1,1,1,-1,-1,1], [1,-1,1,-1,-1,1,-1,1] \]
\[ [1,1,-1,-1,-1,-1,1,1], [1,-1,-1,1,-1,1,1,-1]. \]
%
This is the \emph{Haar basis} for $\mathbf{R}^8$, a very important basis in signals processing, compression, and more general theoretical computing science (the book by Ryan O'Donnell is a fun introduction for the interested reader). Write the vector $[5,3,2,7,2,6,5,3]$ as a linear combination of the vectors above.
\end{question}



\end{questions}	

\end{document}