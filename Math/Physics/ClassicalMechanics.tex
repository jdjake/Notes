\documentclass{article}

\usepackage{amsthm}
\newtheorem{exercise}{Exercise}
\newtheorem{theorem}{Theorem}

\title{Analytical Mechanics}
\author{Jacob Denson}

\begin{document}

\section{Space}



We begin by describing the structure of space, resulting from obvious but crucial observations. The first is that we live in a three-dimensional world, eternally attached to a one-dimensional time dimension. This validifies our choice to identify the universe as a four dimensional affine space $G$\footnote{Formally, an affine space is a set together with a free transitive vector space action from $\mathbf{R}^n$. In english, this means that, though for any two points $p$ and $q$ we may identify a unique vector $(q - p) \in \mathbf{R}^n$, such that $p + (q - p) = q$, we may not identify $p$ and $q$ with any specific vectors. There is no origin point to the universe (we no longer believe in a geocentric structure of the universe). We denote $v$ by $q - p$.}, together with a measurement of time: a non-trivial linear transformation $t:\mathbf{R}^4 \to \mathbf{R}$ which represents the displacement in time between points in the universe. The time between two events (points in affine space) $p$ and $q$ then becomes $t(q - p)$. We may divide our space into an equivalence class of simultaneous events (points for which $t(q - p) = 0$). For each such equivalence class (which forms a subspace of our affine space), we identify an inner product $\langle \cdot, \cdot \rangle$. This allows us to measure distances in space, defining, for two simultaneous events $p$ and $q$,
%
\[ d(p,q) = \| q - p \| = \sqrt{\langle q - p, q - p \rangle} \]
%
The resultant structure $(G,t,d)$ is called galilean space.

\begin{exercise} Verify that each affine subspace of simultaneous events in galilean space is three dimensional, and that all galilean transformations are homeomorphisms. \end{exercise}

Given two galilean spaces $(G,t,d)$ and $(H,t',d')$, we may define a structure preserving map $f:G \to H$ to be a bijection such that
%
\begin{enumerate}
    \item $f(q - p) = f(q) - f(p)$
    \item For any events $p$ and $q$ in $G$, $t(q - p) = t'(f(q) - f(p))$.
    \item For any simultaneous events $p$ and $q$, $d(p,q) = d'(f(p),q(p))$.
\end{enumerate}
%
so that it is an affine transformation that preserves time and relative distance in space. The resultant transformation is known as galilean. Given a galilean space, the set of structure preserving maps from the space to itself is known as the galilean group.

Every galilean space is isomorphic to the vector space $\mathbf{R} \times \mathbf{R}^3$, viewed as a four dimensional affine space, where $t((a,v) - (b,w)) = a - b$, and $\langle (a,v), (b,w) \rangle = \langle v, w \rangle$, so this will be our prime example to study the Galilean group. This isomorphism allows us to identify galilean space as a four dimensional manifold. There are four main examples of galilean transformations:
%
\begin{enumerate}
    \item Uniform motion: $f(t,v) = (a,v + tw)$, for $w \in \mathbf{R}^3$.
    \item Translation in time and space: $f(t,v) = (t + u, v + w)$, for $u \in \mathbf{R}$, $w \in \mathbf{R}^3$.
    \item Rotation about the origin: $f(t,v) = f(t,Mv)$, where $M \in O(3)$.
\end{enumerate}
%
In fact, any galilean group can be uniquely written as the composition of four of these transformations, so that the galilean group is $3 + 4 + 3 = 10$ dimensional.

\begin{exercise}
    Verify this!
\end{exercise}

Our experimental observation has allowed us to build a model of space $G$. Now we must begin to make measurements within it. We begin by considering the notion of a coordinate system. An arbitrary diffeomorphism $f:G \to \mathbf{R} \times \mathbf{R}^3$ is a coordinate system. It is galilean with respect to another transformation $g:G \to \mathbf{R} \times \mathbf{R}^3$ if $g \circ f^{-1}: \mathbf{R} \times \mathbf{R}^3 \to \mathbf{R} \times \mathbf{R}^3$ is galilean. Galilean transformations are integral to the theory of symmetry as exposed by Newton, and now we have space, we can reveal the laws of Newton is rigorous form. As should be expected, given the canonical order they are given, we shall exposit Newton's laws in the proper order they should be given: First the second, then the first, then the third (which will only come some time later).

\section{Newton's Laws}

Each particle in galilean space has ascribed to it a motion. This motion describes the particle so completely that we can, perhaps abusively, treat the particle as the motion itself. This motion is just a differentiable map $x:\mathbf{R} \to \mathbf{R}^n$. The graph of such a map, which appears in every galilean coordinate system as a motion is called a world line, and it is here where we discuss the most crucial tactic in analyzing the classical laws:

If we have $n$ motions $x_1, \dots, x_n$ into euclidean space $\mathbf{R}^3$, we may consider their accumulation as a motion in $\mathbf{R}^{3n}$; a map defined by
%
\[ x(t) = (x_1(t), x_2(t), \dots, x_n(t)) \]
%
In this manner, we have reduced the analysis of a multitude of particles to the analysis of one curve in space. The space created by $n$ adjoinments of $\mathbf{R}^3$ is known as the configuration space of the motion of $n$ particles.

Now we may discuss Newton's second law. You may have known it in the following concise form
%
\[ F = ma \]
%
where $F$ is `the force' of the system, $m$ is the mass, and $a$ the acceleration. The philosophical problem is what exactly `force' is, and what exactly `mass' is. We offer an alternative explanation. What we are saying is that, given any configuration space $\mathbf{R}^{3n}$ where particles move in space, there exists some function $F:\mathbf{R} \times \mathbf{R}^{3n} \times \mathbf{R}^{3n}$, such that, for any set of $n$ motions $x_1, \dots, x_n$ in configuration space, we have
%
\[ {\bf \ddot{x}}(t) = F(t, {\bf x}(t), {\bf \dot{x}}(t)) \]
%
Where ${\bf x} = (x_1, \dots, x_n)$, and ${\bf \dot{x}} = (\dot{x}_1, \dot{x_2}, \dots, \dot{x_n})$, and so on. We conclude that $F$ provides us with a second order differential equation to solve. This holds for any configuration space representing a system of coordinate systems on galilean space. From the theory of uniqueness in differential equations, we obtain Newton's principle of determinacy: If we know the absolute time, position, and velocity of all particles in space, we may determine the entire motion of the system, from past to present. Regardless of the physical system, according to classical mechanics there is some force describing it: Go find it! Normally, the force will be deduced from experimental evidence.

The first law, also known as Galileo's principle of relativity, tells us that there is a remarkable symmetry to the universe when it is seen from certain idealized viewpoints. It states that there exists a class of coordinate frames, called inertial frames, having the following property. If we have some world lines of some motion of particles $(x_1, \dots, x_n)$ in $\mathbf{R} \times \mathbf{R}^{3n}$, and we subject these world lines to some galilean transformation, then these world lines are also valid world lines in our system. From this statement, we can deduce that the universe has remarkable coordination with itself.

First, we know that translation through time is galilean
%
\[ f(t,v) = (t + u,v) \]
%
Hence if we take any set of world lines in space, and translate them only through time, then we obtain new world lines. In other words, the acceleration of an object at a point in space is irrelevant to the time in which it is situatied: The force in a system is independant of time.

\begin{exercise}
    Verify that Force depends only on relative distances $\|x_i - x_j\|$ and velocities $\|\dot{x_i} - \dot{x_j} \|$ rather than absolute distances and velocities in an inertial system by using the galilean transformations developed in the last chapter.
\end{exercise}

From the fact that we may shift by a constant velocity and position in space, we may determine that the force $F$

\begin{exercise}
    If the universe consists of a single particle, verify that in an inertial system it moves at a constant velocity in a single direction.
\end{exercise}

\begin{exercise}
    Verify that if the universe consists of two particles, and measuring in an inertial system both have no initial velocity, then the particles will remain in the line that initially connected them in all of their motion. Show that in general, two particles in an inertial reference frame always remain in some plane in space.
\end{exercise}

In an inertial system, a single particle should have trivial motion. Nonetheless, if we are analysing some motion in a system where we are unable to keep track of every motion (say for instance, the uncountable number of atoms in our universe), then our motion can still be quite complicated from interactions with parts of the system outside of our control. In the next few sections, we will analyse such a motion.

\section{The motion of a single particle in one-dimension}

Suppose our system has only one degree of freedom, and whose position in the system is described only by position in space.
%
\[ \ddot{x} = F(x) \]
%
Examples consist of oscillations, and falling from the sky without horizontal motion (or friction). Most equations we encounter in one dimension can be analytically solved, but there are still a few that escape the grasp of blind calculus. That is where geometrical methods begin to come in handy.

Let us consider an elementary example you should have encountered in high school physics. An approximate force on an object falling in the sky is described by a force $F(x) = -mg$. In this case, we can use elementary calculus to plug and chug right through. Firstly, by the second law,
%
\[ \ddot{x} = -mg \]
%
Integrating once,
%
\[ \dot{x} = -mgt + C_0 \]
%
And once again,
%
\[ x = -\frac{1}{2}mgt^2 + C_0t + C_1 \]
%
Given some initial conditions ($x(0) = h$, $\dot{x}(0) = v_0$), the equation of motion is described by
%
\[ x = -\frac{1}{2}mgt^2 + v_0t + h \]
%
Easy, right!

More accurately, the force should have some air resistance included, as
measured by experiments, so the force should be $F(x) = -mg $

... FILL THIS LATER

\begin{theorem}
    If $(x,0)$ is a point where $U$ is minimized, then it is Lyapunov stable.
\end{theorem}
\begin{proof}
    Without loss of generality, assume $x = U(x) = 0$. Since $U$ is minimized, we may assume that, for any $\varepsilon$, there exists $a$ such that on $[-a,a]$, $0 \leq U < \varepsilon$. Since closed intervals are compact, and $U$ is continuous, $U$ obtains a maximum at $x_0$ on $[-a,0]$ and a maximum at $x_1$ on $[0,a]$. Pick $b$ such that on $(-b,b)$,
    %
    \[ 0 \leq U \leq \frac{x_0}{2},\frac{x_1}{2} \]
    %
    We claim that if $c(0) \in [-b,b]$, and $c'(0) < \sqrt{U(x_0)}, \sqrt{U(x_1)}$, then $c(t) \in (-b,b)$ for all $t$. Surely
    %
    \[ E = U + T < U(x_0), U(x_1) \]
    %
    Suppose $c(t) < x_0$ for some $t$. Then, since $c$ is continuous, there is $t'$ such that $c(t') = x_0$. But then $U(t') = U(x_0) > E$, contradicting the conservation of energy. This contradiction occurs if $c(t) > x_0$ at any $t$. Thus we have bounded curves in a box around the minimized energy point. And the diagonal of this box is less than $2\varepsilon$.
\end{proof}








\end{document}