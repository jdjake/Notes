\documentclass[12pt]{report}

\usepackage{amsmath}
\usepackage{amssymb}
\usepackage{amsthm}
\usepackage{amsopn}
\usepackage{kpfonts}
\usepackage{graphicx}
\usepackage{kbordermatrix}
\usepackage{tikz}
\usetikzlibrary{arrows, petri, topaths}%
\usepackage{tkz-berge}
\usepackage{multicol}

\usepackage{framed}
\usepackage{mathtools}
\usepackage{float}
\usepackage{subfig}
% \usepackage{cmbright}

\theoremstyle{plain}
\newtheorem{theorem}{Theorem}[chapter]
\newtheorem{lemma}[theorem]{Lemma}
\newtheorem{corollary}[theorem]{Corollary}
\newtheorem{prop}[theorem]{Proposition}
\newtheorem{exercise}{Exercise}[chapter]

\newtheorem*{example}{Example}
\newtheorem*{proof*}{Proof}

\theoremstyle{definition}
\newtheorem*{defi}{Definition}
\newenvironment{definition}
    {\begin{samepage}\begin{framed}\begin{defi}}
    {\end{defi}\end{framed}\end{samepage}}





\usepackage{hyperref} 
\hypersetup{
    colorlinks = true,
    linkcolor = black,
}

\makeatletter
\renewcommand*\env@matrix[1][*\c@MaxMatrixCols c]{%
  \hskip -\arraycolsep
  \let\@ifnextchar\new@ifnextchar
  \array{#1}}
\makeatother

\renewcommand*\contentsname{\hfill Table Of Contents \hfill}

\newcommand{\optionalsection}[1]{\section[* #1]{(Important) #1}}
\newcommand{\deriv}[3]{\left. \frac{\partial #1}{\partial #2} \right|_{#3}}

\title{Quantum Mechanics}
\author{Jacob Denson}

\begin{document}

\chapter{The Setup}

In any physical theory, we must characterize mathematically the \emph{state} of a system (all information describing the situation of a physical system at a particular time), and the \emph{observables}, the functions of a state, which give ways in which the state of a system can be reduced to quantities that can be observed experimentally. For instance, in Hamiltonian classical mechanics, the state of a system is given by a point in a sympletic manifold $M$, and observables given by functions $f: M \to \RR$, which should be continuous if we are to correctly measure these observables up to a small degree of error. The observables are then `second order' as they are defined in terms of states, but we can also reverse the situation, describing the observables as the $C^*$ algebra $A = C(M)$. The states then become precisely a \emph{positive} linear functional $\phi: A \to \RR$ with $\phi(1) = 1$. It is natural in the later quantum mechanics to complexify the $C^*$ algebra $A$. Then the observables become the \emph{self-adjoint} elements of $A$, and the states the linear functionals $\phi: A \to \CC$ with $\phi(1) = 1$ and with $\phi(X) \geq 0$ if $X \geq 0$. The Riesz representation theorem allows us to identify an arbitrary positive linear functionals $\phi: A \to \RR$ such that $\phi(1) = 1$ with a Borel probability measure $\mu$ on $M$. We then think of an element $X \in A$ as a \emph{random variable} over the probability space $(M,\mu)$, because we then have
%
\[ \EE_\phi[X] = \int X\; d\mu = \phi(X). \]
%
Similarily,
%
\[ \sigma_\phi(X)^2 = \VV_\phi(X) = \phi(X^2) - \phi(X)^2. \]
%
The \emph{pure}, deterministic states $\phi$ can then be identified from general \emph{mixed states} as those states such that $\VV_\phi(X) = 0$ for all observables $X$.

%A practical principle often used in this formulation is the principle of \emph{super-position}. Any positive linear functional $\phi: A \to \RR$ can be mapped onto a state by normalizing, i.e. considering the positive linear functional $\tilde{\phi}(f) = \phi(f) / \phi(1)$. The inverse image of each state under this correspondence is then a ray of positive linear functionals. Given two states $\phi$ and $\psi$, we can consider therefore consider their \emph{superposition} state $a \phi + b \psi$, such that $(a \phi + b \psi)(f) = [a \phi(f) + b \psi(f)] / (a + b)$.

What caused this formulation to fail to explain quantum mechanical phenomena. The most fundamental experimental observation in the theory is the \emph{uncertainty principle}. It is an experimental observation that in any physical system, if $p: A \to \CC$ and $q: A \to \CC$ are the position and momentum observables, then for any state $\phi$,
%
\[ 2 \sigma_\phi(p) \sigma_\phi(q) \geq \hbar, \]
%
where $\hbar$ is \emph{Planck's constant}. But there are no two observables $\phi: A \to \RR$ with this property for all classical states, because $\sigma_\phi(p) = \sigma_\phi(q) = 0$ for any deterministic state. Thus it appears that the only physically possible states $\phi$ \emph{must be uncertain} in a suitable sense; this is the \emph{uncertainty principle}.

In the standard theory, this is remedied by replacing the observables of a system with elements of an abstract $C^*$ algebra $A$, and the states with normalized, positive linear functions $\phi: A \to \CC$. Each fixed state $\phi$ then induces an algebra homomorphism $\Phi$ from $A$ to the family of random variables in an appropriate probability space, such that $\EE(\Phi(X)) = \phi(X)$. Thus one can use the spectral calculus to obtain detailed information about the probability distribution of $\Phi(X)$, since for any continuous $f: \sigma(X) \to \CC$, we have $\EE(f(X)) = \phi(f(X))$. Note, in particular, that this means that the support of the random variable $X$ is on $\sigma(X)$.

The reason this formulation is useful is that we can theoretically derive the uncertainty principle, provided we are working in a \emph{non-commutative} $C^*$ algebra $A$. Indeed, if $X$ and $Y$ are any observables with $\phi(X) = \phi(Y) = 0$, we calculate that the matrix
%
\[ M = \begin{pmatrix} \phi(X^2) & (1/2) \phi(i [X,Y]) \\ (1/2) \phi(i[X,Y]) & \phi(Y^2) \end{pmatrix} \]
%
is positive-semidefinite, since for any $v = (\alpha,\beta)^T \in \RR$,
%
\[ v^T M v = \phi(X^2) \alpha^2 + \phi(i[X,Y]) \alpha \beta + \phi(Y^2) \beta^2 = \phi((\alpha X - i \beta Y)(\alpha X + i \beta Y)) \geq 0. \]
%
Thus $\det(M) = \phi(X^2) \phi(Y^2) - \phi(i[X,Y])^2 / 4$ is non-negative, which means that
%
\[ 2 \sigma_\phi(X) \sigma_\phi(Y) = 2 \phi(X^2)^{1/2} \phi(Y^2)^{1/2} \geq \phi(i[X,Y]). \]
%
Thus the uncertainty principle for position and momenta follows immediately if we model these quantities by observables $p$ and $q$ with $[p,q] = -i \hbar$.

\chapter{Quantum Information Theory}

The simplest unit of information in classical physics is a \emph{bit}, represented by an element of $\{ 0, 1 \}$. We can generalize 


 and the state of a collection of $n$ bits are represented by an element of $\{ 0, 1 \}^n$. From the quantum perspective, a \emph{quantum bit}, or \emph{qubit}, is represented by an element $\psi = \psi_0 \langle 0 | + \psi_1 \langle 1 |$ of a two dimension Hermitian product space with orthonormal basis $\{ \langle 0 |, \langle 1 | \}$.

(Gleason)





\newpage

\section{How can the Hamiltonian operator be Formed From the Classical Hamiltonian?}

Thanks to Viktor T. Toth for this explanation on Quora. The Hamiltonian of a system is given by
%
\[ H = p^2/2m + V(q), \]
%
where $p$ is the momentum, and $q$ the position. If we set
%
\[ \psi = e^{(i / \hbar) (p \cdot q - Ht)}, \]
%
then we notice that we can rewrite the Hamiltonian system above as
%
\[ 0 = (H - p^2/2m - V(q)) \psi = H \psi - \frac{p^2}{2m} \psi - V(q) \psi. \]
%
Note that
%
\[ H \psi = i \hbar \frac{\partial \psi}{\partial t} \quad\text{and}\quad p \psi = - i \hbar \nabla_q \psi, \]
% nabla_q . nabla_q psi = nabla_q . (i/\hbar) p \psi = -p^2/\hbar psi
and so
%
\[ i \hbar \frac{\partial \psi}{\partial t} + \frac{\hbar^2}{2m} \Delta_q \psi - V(q) \psi = 0, \]
%
or equivalently,
%
\[ i\hbar \frac{\partial \psi}{\partial t} = - \frac{\hbar^2}{2m} \Delta_q \psi + V(q) \psi, \]
%
which begins to look like the Schr\"{o}dinger equation. In \emph{classical physics}, we restrict ourselves to solutions of this equation of the form $e^{(i / \hbar)(p \cdot q - Ht)}$, from which we can extract out the particular positions and frequencies of objects. \emph{Quantum mechanics} begins when we now allow \emph{linear combinations} of solutions of this form, i.e. 'mixed states'.



\end{document}