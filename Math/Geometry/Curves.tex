\documentclass[12pt]{report}

\usepackage{amsmath}
\usepackage{amssymb}
\usepackage{amsthm}
\usepackage{amsopn}
\usepackage{kpfonts}
\usepackage{graphicx}
\usepackage{kbordermatrix}
\usepackage{tikz}
\usetikzlibrary{arrows, petri, topaths}%
\usepackage{tkz-berge}
\usepackage{multicol}

\usepackage{framed}
\usepackage{mathtools}
\usepackage{float}
\usepackage{subfig}
% \usepackage{cmbright}

\theoremstyle{plain}
\newtheorem{theorem}{Theorem}[chapter]
\newtheorem{lemma}[theorem]{Lemma}
\newtheorem{corollary}[theorem]{Corollary}
\newtheorem{prop}[theorem]{Proposition}
\newtheorem{exercise}{Exercise}[chapter]

\newtheorem*{example}{Example}
\newtheorem*{proof*}{Proof}

\theoremstyle{definition}
\newtheorem*{defi}{Definition}
\newenvironment{definition}
    {\begin{samepage}\begin{framed}\begin{defi}}
    {\end{defi}\end{framed}\end{samepage}}





\usepackage{hyperref} 
\hypersetup{
    colorlinks = true,
    linkcolor = black,
}

\makeatletter
\renewcommand*\env@matrix[1][*\c@MaxMatrixCols c]{%
  \hskip -\arraycolsep
  \let\@ifnextchar\new@ifnextchar
  \array{#1}}
\makeatother

\renewcommand*\contentsname{\hfill Table Of Contents \hfill}

\newcommand{\optionalsection}[1]{\section[* #1]{(Important) #1}}
\newcommand{\deriv}[3]{\left. \frac{\partial #1}{\partial #2} \right|_{#3}}

\title{Curves}
\author{Jacob Denson}

\begin{document}

\pagenumbering{gobble}
\maketitle
\tableofcontents
\pagenumbering{arabic}

\chapter{History of Curves}

The simplest geometric shapes are the circle and the line, and these were the first curves analyzed by the ancients. Even by the time of Euclid, it had been realized that a circle could be described as a locus of points at a fixed distance from a given point. In Descarte's analytic geometry, the theory of loci became even more important, because the loci defining these shapes became equations which could be analyzed algebraically to determine the nature of a geometric shape. For instance, in a particular coordinate system, a circle can be described as those points satisfying the polynomial relationship $X^2 + Y^2 = 1$. Similarily, a line can be described as the solution set to a relationship $aX + bY = c$, where $a$ and $b$ are not both zero. In terms of Euclid's geometry, these shapes are effectively all the tools that one can use to construct points -- a ruler corresponds to the ability to draw lines, and a compass the ability to draw circles, and we can determine their intersections. Algebraically, this means that in Euclidean geometry, we are only able to construct numbers satisfying linear and quadratic equations (linear equations are formed by the intersection of two lines, and quadratic equations by the intersection of circles, or by a circle and a line). These shapes seem a limited toolset to start, and because of this many problems proved unsolvable.
%
\begin{itemize}
    \item Squaring a Circle: Constructing a square with the same area as a circle. This essentially means constructing a line of length $\sqrt{\pi}$. This is impossible because $\sqrt{\pi}$ is a {\it trancendental number}, meaning it is not the root of any polynomial equation with coefficients in the rational numbers. In particular, this means that $\sqrt{\pi}$ cannot be constructed by taking intersections of lines and circles, or more generally, with any family of shapes whose intersections are specified as solutions to polynomial equations.
    \item Trisecting an arbitrary angle. Because we have the trigonometric identity $\sin 3x = 3 \sin x - 4 \sin^3 x$, projecting lines onto the $Y$ axis implies that trisecting an angle is equivalent to being able to find solutions to the equation $X = 3Y - 4Y^3$ for an arbitrary $X$. This is a cubic equation, and as such we cannot find the roots of this equation.
    \item Duplicating a Cube: Constructing a cube with volume twice that of a given cube. This essentially means constructing a line of length $\sqrt[3]{2}$. Since the smallest degree polynomial with coefficients in the rational numbers with $\sqrt[3]{2}$ as a root is $X^3 - 2$, we cannot construct $\sqrt[3]{2}$.
\end{itemize}
%
These were proved using the algebraic tools of Galois theory in the 19th century. However, in the 2000 years of trying to solve these problems, the Greeks didn't wait for algebra to solve their problems, instead introducing new tools into their geometry which enabled these problems to be solved.

In 350 BC, Menaechmus discovered the conic sections, which were generated as the intersections of cones with planes. Additionally, these sections could be classified by loci.
%
\begin{itemize}
    \item An ellipse is the locus of points for which the sum of distances between two other fixed points has a designated value. Analytically, this means that we can choose a coordinate system in which an ellipse is the set of solutions to
    %
    \[ Y^2 = pX - qX^2 \]
    \item A parabola is the locus of all points having equal distances between a given point and a line. Analytically this means that a parabola is specified as the solution set to
    %
    \[ Y^2 = pX \]
    \item A hyperbola is the locus of points for which the difference of distances between two other fixed points has a designated value.
    %
    \[ Y^2 = pX + qX^2 \]
\end{itemize}
%
Menaechmus immediately observed that if one was able to draw arbitrary conic sections in the plane, then one was able to duplicate a cube. Indeed, 

\end{document}